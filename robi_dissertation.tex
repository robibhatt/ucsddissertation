\documentclass[12pt]{ucsddissertation}
% mathptmx is a Times Roman look-alike (don't use the times package)
% It isn't clear if Times is required. The OGS manual lists several
% "standard fonts" but never says they need to be used.
\usepackage{mathptmx}
\usepackage[numbers]{natbib}
\usepackage[NoDate]{currvita}
\usepackage{array}
\usepackage{tabularx}
\usepackage{booktabs}
\usepackage{ragged2e}
\usepackage{microtype}
\usepackage{xcolor}
\usepackage[breaklinks=true,pdfborder={0 0 0}]{hyperref}
\usepackage{graphicx}
\usepackage{amsmath, amsfonts, amssymb}
\usepackage{dsfont}
\usepackage{nicefrac}
\usepackage{mathtools}
\usepackage[ruled, lined, linesnumbered, commentsnumbered, longend]{algorithm2e}
\usepackage{amsthm}
\usepackage{wrapfig}
\usepackage{geometry}
\usepackage{soul}
\usepackage{lipsum}
\usepackage{tikz}
\usetikzlibrary{bayesnet}
\usepackage{bbm}
\usepackage{subcaption}
\usepackage{yfonts}
\captionsetup[sub]{font=scriptsize,labelfont={bf}}
\usepackage[breakable, theorems, skins]{tcolorbox}
\usepackage[capitalize,noabbrev]{cleveref}
\AtBeginDocument{%
	\settowidth\cvlabelwidth{\cvlabelfont 0000--0000}%
}

\DeclareMathOperator*{\argmax}{arg\,max}
\DeclareMathOperator*{\argmin}{arg\,min}

% OGS recommends increasing the margins slightly.
\increasemargins{.1in}

% These are just for testing/examples, delete them
\usepackage{trace}
%\usepackage{showframe} % This package was just to see page margins
\usepackage[english]{babel}
\usepackage{blindtext}
\overfullrule5pt
% ---

%%For multi-file compilation 
%\usepackage{subfiles}

% Required information
\title{Theoretical Foundations of Trustworthy Machine Learning}
\author{Robi Bhattacharjee}
\degree{Computer Science}{Doctor of Philosophy}
% Each member of the committee should be listed as Professor Foo Bar.
% If Professor is not the correct title for one, then titles should be
% omitted entirely.
\chair{Professor Kamalika Chaudhuri}
% Your committee members (other than the chairs) must be in alphabetical order
\committee{Professor Mikhail Belkin}
\committee{Professor Sanjoy Dasgupta}
\committee{Professor Yoav Freund}

\degreeyear{2023}


% Start the document
\begin{document}
% Begin with frontmatter and so forth
\frontmatter
\maketitle
\makecopyright
\makesignature
% Optional
\begin{dedication}
\setsinglespacing
\raggedright % It would be better to use \RaggedRight from ragged2e
\parindent0pt\parskip\baselineskip
For Mom, Dad, and Sormeh.  
\end{dedication}

%% Optional
%\begin{epigraph}
%\vskip0pt plus.5fil
%\setsinglespacing
%%{\flushright
%%True ease in writing comes from art, not chance,\\
%%As those move easiest who have learn'd to dance.\\
%%'T is not enough to no harshness gives offence,---\\
%%The sound must seem an echo to the sense.
%%
%%\vskip\baselineskip
%%\textit{Alexander Pope}\par}
%\vfil
%\begin{center}
%Something pithy
%
%\vskip\baselineskip
%\textit{Someone smart}
%\end{center}
%\vfil
%%\noindent Writing, at its best, is a lonely life. Organizations for
%%writers palliate the writer's loneliness, but I doubt if they improve
%%his writing. He grows in public stature as he sheds his loneliness and
%%often his work deteriorates. For he does his work alone and if he is a
%%good enough writer he must face eternity, or the lack of it, each day.
%%
%%\vskip\baselineskip
%%\hskip0pt plus1fil\textit{Ernest Hemingway}\hskip0pt plus4fil\null
%
%%\vfil
%\end{epigraph}

% Next comes the table of contents, list of figures, list of tables,
% etc. If you have code listings, you can use \listoflistings (or
% \lstlistoflistings) to have it be produced here as well. Same with
% \listofalgorithms.
\tableofcontents
\listoffigures
\listoftables

%% Preface
%\begin{preface}
%Almost nothing is said in the manual about the preface. There is no
%indication about how it is to be typeset. Given that, one is forced to
%simply typeset it and hope it is accepted. It is, however, optional
%and may be omitted.
%\end{preface}

% Your fancy acks here. Keep in mind you need to ack each paper you
% use. See the examples here. In addition, each chapter ack needs to
% be repeated at the end of the relevant chapter.
\begin{acknowledgements}
First and foremost, I thank my advisor, Kamalika, for her guidance over my time in graduate school. Much of what I understand about conducting and communicating machine learning research, I owe directly to her mentorship. Beyond research, Kamalika has always been genuine and forthcoming to me with advice and guidance for shaping my academic career. 

I also thank my co-advisor, Sanjoy. His emphasis on simplicity and elegance has been a huge inspiration for my style of research. Additionally his willingness to listen to and give feedback for any of my half-baked ideas has been a consistent source of comfort and support. 

None of my research would be possible without all of my amazing collaborators; I'm very grateful to have worked with all of them. I especially want to thank Yoav Freund for  his amazing enthusiastic energy for conducting research, and Michal Moshkovitz for her sharp mathematical insight and careful attention to detail.

My time at UCSD would not have been anywhere near as enjoyable if it weren't for my lab-mates, Casey, Jacob, Yao-Yuan, Amrita, and Mary. I will miss our Wednesday lunches dearly. 

I also want to give special thanks to two people from before my Phd that have deeply shaped my life. My father shared with me his love for mathematics for as long as I can remember. In many ways, this thesis is a direct result of that. Later, through middle school and high school, I prepared for math competitions under Andrew Kahng's guidance. Those years were formative in teaching me how to be a skilled problem solver.

I would not have been able to accomplish anything without the loving support of my friends and family. I especially want to thank my brother Neil, my mother, and my friends Ahmed, Alex, and William. You guys have consistently been people I can rely on in any and all situations.

Finally, to my fiancee Sormeh, none of this would be possible or worth doing without you. 






%
%I cannot overemphasize the gratitude I have to my collaborators as well. First, to my collaborators at FAIR. I thank Chuan Guo for showing me how to navigate incredibly challenging and open-ended ML problems. His Socratic approach to mentoring helped me explore my own instincts and take ownership of the project without letting me slip down the wrong path. Florian Bordes taught me the ins and outs of contemporary vision modeling. Pascal Vincent's insights on how to design deep learning experiments were truly formative; I sincerely appreciate his effort and time. I owe an enormous debt of gratitude to Ashwin Machanavajjhala at Tumult for mentoring me on how how to reason about different privacy definitions and guarantees in applied settings. 
%
%Finally, to all of my friends. To the lab: I could not have landed in a better group. To be surrounded by such wonderful and bright companions is a win --- for them to be your colleagues as well is a gift. To all my dear pre-PhD friends (you know who you are, and you really don't have read this dissertation): you are the loves of my life! 

\pagebreak

\textbf{Chapter 1}, in full, is a reprint of the material as it appears in International Conference on Machine Learning, 2020. Robi Bhattacharjee, Kamalika Chaudhuri. \emph{When are Non-Parametric Methods Robust}. The dissertation author is the primary investigator and author of this paper. 

\textbf{Chapter 2}, in full, is a reprint of the material as it appears in Neural Information Processing Systems, 2021. Robi Bhattacharjee, Kamalika Chaudhuri. \emph{Consistent Non-Parametric Methods for Maximizing Robustness}. The dissertation author is the primary investigator and author of this paper. 

\textbf{Chapter 3}, in full, is a reprint of the material as it appears in International Conference on Machine Learning. 2022. Robi Bhattacharjee,  Somesh Jha, Kamalika Chaudhuri. \emph{Sample Complexity of Robust Linear Classification on Separated Data}. The dissertation author is the primary investigator and author of this paper.

\textbf{Chapter 4}, in full, is a reprint of the material as it appears in International Conference on Algorithmic Learning Theory, 2023. Robi Bhattacharjee, Max Hopkins, Akash Kumar, Hantao Yu, Kamalika Chaudhuri. \emph{Robust Empirical Risk Minimization with Tolerance}. The dissertation author is the primary investigator and author of this paper.

\textbf{Chapter 5}, in full, is a reprint of the material as it appears in International Conference on Machine Learning, 2023. Robi Bhattacharjee, Sanjoy Dasgupta, Kamalika Chaudhuri. \emph{Data-Copying in Generative Models: A Formal Framework}. The dissertation author is the primary investigator and author of this paper. 
\end{acknowledgements}

% Stupid vita goes next
\begin{vita}
\noindent
\begin{cv}{}
\begin{cvlist}{}
\item[2016] Bachelor of Science, Massachusetts Institute of Technology
\item[2022] Master of Science, University of California San Diego
\item[2023] Doctor of Philosophy, University of California San Diego
\end{cvlist}
\end{cv}

%% This puts in the PUBLICATIONS header. Note that it appears inside
%% the vita environment. It is optional.
%\publications
%\noindent``Distributions of Control Points in a System for Analysis of Stress
%Distribution'' IRE Transactions of the I.R.E\@. Professional Group on
%Automatic Control, vol. AC-7, pp 272--289, September 2005

%% This puts in the FIELDS OF STUDY. Also inside vita and also
%% optional.
%\fieldsofstudy
%\noindent Major Field: Engineering (Specialization or Focused Studies)
%\vskip\baselineskip
%Studies in Applied Mathematics\par
%Professors Alpha Beta and Gamma Delta
%\vskip\baselineskip
%Studies in Mechanices\par
%Professors Epsilon Zeta and Eta Theta
%\vskip\baselineskip
%Studies in Electromagnetism\par
%Professors Iota Kappa and Lambda Mu
\end{vita}

% Put your maximum 350 word abstract here.
\begin{dissertationabstract}
%The Abstract begins here. The abstract is limited to 350 words for a
%doctoral dissertation. It should consist of a short statement of the
%problem, a brief explanation of the methods and procedures employed in
%generating the data, and a condensed summary of the findings of the
%study. The abstract may continute onto a second page if necessary. The
%text of the abstract must be double spaced.
Machine learning models have become a ubiquitous part of society, and it has consequently become of paramount importance to understand how to design safe and reliable models. This dissertation attempts to take steps towards this direction by consider two specific problems in reliable machine learning: \textit{adversarial examples,} which are small test-time perturbations to the input designed to cause misclassification, and \textit{data-copying,} which occurs when a generative model simply memorizes its training data (giving poor generalization and dangerous security risks). 
\end{dissertationabstract}

% This is where the main body of your dissertation goes!
\mainmatter

% Optional Introduction
\begin{dissertationintroduction}
\section{Introduction} 
\label{sec:intro} 



Language models have now become ubiquitous in NLP \cite{devlin2019bert, liu2019roberta, alsentzer2019publicly}, pushing the state of the art in a variety of tasks \cite{strubell2018linguistically, liu2019multi, mrini-etal-2021-recursive}. While language models capture meaning and various linguistic properties of text \cite{jawahar2019does, yenicelik2020does}, an individual's written text can include highly sensitive information. Even if such details are not needed or used, sensitive information has been found to be vulnerable and detectable to attacks \cite{pan2020privacy, attack_word_embs, carlini_attack}. Reconstruction attacks \cite{xie2021reconstruction} have even successfully broken through private learning schemes that rely on encryption-type methods \cite{huang-etal-2020-texthide}.

As of now, there is no broad agreement on what constitutes good privacy for natural language \cite{kairouz2019advances}. \citet{huang-etal-2020-texthide} argue that different applications and models require different privacy definitions. Several emerging works propose to apply Metric Differential Privacy \cite{orig_metricdp} at the word level \cite{metricdp,  mdp_low_dim, TEM, another_metric_DP, fancy_metricdp, metricDP_gumbel} . They propose to add noise to word embeddings, such that they are indistinguishable from their nearest neighbours.

At the document level, however, the above definition has two areas for improvement. First, it may not offer the level of privacy desired. Having each word indistinguishable with similar words may not hide higher level concepts in the document, and may not be satisfactory for many users. Second, it may not be very interpretable or easy to communicate to end-users, since the privacy definition relies fundamentally on the choice of embedding model to determine which words are indistinguishable with a given word. This may not be clear and precise enough for end-users to grasp.
%
%The above definition is straightforward to implement and naturally takes advantage of the structure in precomputed word embeddings. At the document level, however, there are two areas for improvement. First, it may not offer the level of privacy desired. Having each word indistinguishable only with similar words may not be satisfactory for many users. Replacing words with similar words does not necessarily hide higher level concepts in the document. Second, it may not be very interpretable or easy to communicate to end-users. The privacy definition relies fundamentally on the choice of embedding model. Stating which words are indistinguishable with a given word requires querying the embedding model. This may not be a clear and precise enough for end users to grasp.

\begin{figure}
	\centering
	\includegraphics[width = 0.8\linewidth]{figures/first_pg.png}
	\label{fig:first page}
	\vspace{0cm}
	\caption{$x$ and $x'$ yield $z \in \R^d$ with similar probability.}
\end{figure}
 
\begin{figure*}
	\centering
	\vspace{-1cm}
	\includegraphics[width = \linewidth]{figures/block_diagram.png}
	\label{fig:block diagram}
	\vspace{-0.65cm}
	\caption[\technique\ generates a private embedding $z$ of document $x$ by selecting from a set $F$ of public, non-private document embeddings.]{\technique\ generates a private embedding \textcolor{green}{$z$} of document \textcolor{red}{$x$} by selecting from a set \textcolor{blue}{$F$} of public, non-private document embeddings. Sentences from \textcolor{red}{$x$} are encoded by $G'$. The privacy mechanism $\mname$, then privately samples from \textcolor{blue}{$F$}, with a preference for candidates with high Tukey Depth, `deep candidates'. $G'$ is trained beforehand to ensure that deep candidates are likely to exist and are relevant to \textcolor{red}{$x$}.}
\end{figure*}

In this work, we propose a new privacy definition for documents: sentence privacy. This guarantee is both strong and interpretable: any sentence in a document must be indistinguishable with \emph{any} other sentence. A document embedding is sentence-private if we can replace any single sentence in the document and have a similar probability of producing the same embedding. As such, the embedding only stores limited information unique to any given sentence. This definition is easy to communicate and strictly stronger than word-level definitions, as modifying a sentence can be changing one word.

Although this definition is strong, we are able to produce unsupervised, general embeddings of documents that are useful for downstream tasks like sentiment analysis and topic classification. To achieve this we propose a novel privacy mechanism, \technique, which privately samples a high-dimensional embedding from a preselected set of candidate embeddings derived from public, non-private data. \technique\  works by first pre-tuning a sentence encoder on public data such that semantically different document embeddings are far apart from each other. Then, we approximate each candidate's Tukey Depth within the private documents' sentence embeddings. Deeper candidates are the most likely to be sampled to represent the private document. We evaluate \technique\  on three illustrative datasets, and show that these unsupervised private embeddings are useful for both sentiment analysis and topic classification as compared to baselines. 

In summary, this work makes the following contributions to the language privacy literature:

\begin{squishlist}
	\item A new, strong, and interpretable privacy definition that offers complete indistinguishability to each sentence in a document. 
	\item A novel, unsupervised embedding technique, \technique, to generate sentence-private document embeddings. 
	\item An empirical assessment of \technique, demonstrating its advantage over baselines, delivering strong privacy and utility. 
\end{squishlist}  
\end{dissertationintroduction}

%\chapter{An ordinary page}
%The purpose of this page is to illustrate an ordinary page of text in
%a doctoral dissertation or master's thesis. All pages of the doctoral
%dissertation or master's thesis must be kept within the margins of
%1.5'' on the left, 1'' on the right, 1'' on the top and 1.25'' on the
%bottom. All text must be double spaced except as indicated below.
%
%It is recommended that to increase the margins as paper can shift in a
%printer and as some photocopiers tend to increase the image being
%copied.
%
%The first line of each paragraph must be indented at least one 0.5''
%tab, as done here.
%
%This text is intended to be a part of the dissertation, for a doctoral
%student, or the thesis if you are receiving a master's degree, and now
%a quote is included here:
%\begin{quote}
%All quotes of more than six lines, even though this one is not, are to
%be indented 0.5'' from the left and 0.5'' from the right. These longer
%quotes are to be single spaced. Don't forget to adjust for proper
%spacing after the last line of the quoted material.
%\end{quote}
%The rest of the paragraph would continue as so.

\graphicspath{{./chapters/chapter1}}
\chapter{When are Non-Parametric Methods Robust?} 

\newtheorem{thm}{Theorem}
\newtheorem{lem}[thm]{Lemma}
\newtheorem{cor}[thm]{Corollary}

\newtheorem{defn}[thm]{Definition}
\newtheorem{ex}{Example}

\def\D{{\mathcal D}}
\def\X{\mathcal X}
\def\R{\mathbb R}
\def\Y{\{\pm 1\}}
\def\w{\hat{w}}
\def\P{\mathbb{P}}
\def\I{\hat{I}}
\def\b{g^*}
\def\r{\rho}
\def\rcons{r-consistent}
\def\rconsy{r-consistency}
\def\ap{AdvPrun}
\def\ga{RobustNonPar}

\section{Introduction}

Recent work has shown that many classifiers tend to be highly non-robust and that small strategic modifications to regular test inputs can cause them to misclassify~\cite{Szegedy14, Goodfellow14, MeekLowd05}. Motivated by the use of machine learning in safety-critical applications, this phenomenon has recently received considerable interest; however, what exactly causes this phenomenon -- known in the literature as {\em{adversarial examples}} -- still remains a mystery.

Prior work has looked at three plausible reasons why adversarial examples might exist. The first, of course, is the possibility that in real data distributions, different classes are very close together in space -- which does not seem plausible in practice. Another possibility is that classification algorithms may require more data to be robust than to be merely accurate; some prior work~\cite{Madry18, WJC18, Srebro19} suggests that this might be true for certain classifiers or algorithms. Finally, others~\cite{Bubeck19, Vinod19, WJC18} have suggested that better training algorithms may give rise to more robust classifiers -- and that in some cases, finding robust classifiers may even be computationally challenging.

In this work, we consider this problem in the context of general non-parametric classifiers. Contrary to parametrics, non-parametric methods are a form of local classifiers, and include a large number of pattern recognition methods such as nearest neighbors, decision trees, random forests and kernel classifiers. There is a richly developed statistical theory of non-parametric methods~\cite{devroye96}, which focuses on accuracy, and provides very general conditions under which these methods converge to the Bayes optimal with growing number of samples. We, in contrast, analyze robustness properties of these methods, and ask instead when they converge to the classifier with the highest astuteness at a desired radius $r$. Recall that the astuteness of a classifier at radius $r$ is the fraction of points from the distribution on which it is accurate and has the same prediction up to a distance $r$~\cite{WJC18, Madry18}.

 We begin by looking at the very simple case when data from different classes is well-separated -- by at least a distance $2r$. Although achieving astuteness in this case may appear trivial, we show that even in this highly favorable case, not all non-parametric methods provide robust classifiers -- and this even holds for methods that converge to the Bayes optimal in the large sample limit.  

This raises the natural question -- when do non-parametric methods produce astute classifiers? We next provide conditions under which a non-parametric method converges to the most astute classifier in the large sample limit under well-separated data. Our conditions are analogous to the classical conditions for convergence to the Bayes optimal~\cite{devroye96, Stone77}, but a little stronger. We show that nearest neighbors and kernel classifiers whose kernel functions decay fast enough, satisfy these conditions, and hence converge to astute classifiers in the large sample limit. In constrast, histogram classifiers, which do converge to the Bayes optimal in the large sample limit, may not converge to the most astute classifier. This indicates that there may be some non-parametric methods, such as nearest neighbors and kernel classifiers, that are more naturally robust when trained on well-separated data, and some that are not.

What happens when different classes in the data are not as well-separated? For this case, \cite{YRWC19} proposes a method called Adversarial Pruning that preprocesses the training data by retaining the maximal set of points such that different classes are distance $\geq 2r$ apart, and then trains a non-parametric method on the pruned data. We next prove that if a non-parametric method has certain properties, then the classifier produced by Adversarial Pruning followed by the method does converges to the most astute classifier in the large sample limit. We show that again nearest neighbors and kernel classifiers whose kernel functions decay faster than inverse polynomials satisfy these properties. Our results thus complement and build upon the empirical results of~\cite{YRWC19} by providing a performance guarantee. 

What can we conclude about the cause for adversarial examples? Our results seem to indicate that at least for non-parametrics, it is mostly the training algorithms that are responsible. With a few exceptions, decades of prior work in machine learning and pattern recognition has largely focussed on designing training methods that provide increasingly accurate classifiers -- perhaps to the detriment of other aspects such as robustness. In this context, our results serve to (a) provide a set of guidelines that can be used for designing non-parametric methods that are robust and accurate on well-separated data and (b) demonstrate that when data is not well-separated, preprocessing through adversarial pruning~\cite{YRWC19} may be used to ensure convergence to optimally astute solutions in the large sample limit. 

\subsection{Related Work}

There is a large body of work on adversarial attacks~\cite{Carlini17, Liu17, Papernot17, Papernot16,Szegedy14} and defenses~\cite{Hein17,Katz17,Schmidt18,Wu16,Steinhardt18, Sinha18} in the parametric setting, specifically focusing on neural networks. On the other hand, adversarial examples for nonparametric classifiers have mostly been studied in a much more ad-hoc manner, and to our knowledge, there has been no theoretical investigation into general properties of algorithms that promote robustness in non-parametric classifiers.

For nearest neighbors, there has been some prior work on adversarial attacks~\cite{Amsaleg17, Sitawarin19, WJC18, YRWC19} as well as defenses. Wang et. al. \cite{WJC18} proposes a defense for 1-NN by pruning the input sample. However, their defense learns a classifier whose robustness regions converge towards those of the Bayes optimal classifier, which itself may potentially have poor robustness properties. Yang et. al. \cite{YRWC19} accounts for this problem by proposing the notion of the $r$-optimal classifier, and propose an algorithm called Adversarial Pruning which can be interpreted as a finite sample approximation to the $r$-optimal. However, they do not provide formal performance guarantees for Adversarial Pruning, which we do. 

For Kernel methods, Hein and Andriushchenko \cite{Hein17} study lower bounds on the norm of the adversarial manipulation that is required for changing a classifiers output. They specifically study bounds for Kernel Classifiers, and propose an empirically based regularization idea that improves robustness. In this work, we improve the robustness properties of kernel classification through adversarial pruning, and show formal guarantees regarding convergence towards the $r$-optimal classifier. 

For decision trees and random forests, attacks and defenses have been provided by \cite{Hein19, Kantchelian15, Hsiehicml19}. Again, most of the work here is empirical in nature, and convergence guarantees are not provided. 

Pruning has a long history of being applied for improving nearest neighbors \cite{Gates72, Gottlieb14, Hart68, KontorovichSW17, KontorovichW15, Hanneke19}, but this has been entirely done in the context of generalization, without accounting for robustness. In their work, Yang et. al. empirically show that adversarial pruning can improve robustness for nearest neighbor classifiers. However, they do not provide any formal guarantees for their algorithms. In this work, we prove formal guarantees for \textit{adversarial pruning} in the large sample limit, both for nearest neighbors as well as for more general \textit{weight functions.} 

There is a long history of literature for understanding the consistency of Kernel classifiers \cite{Steinwart05, Stone77}, but this has only been done for accuracy and generalization. In this work, we find different conditions are needed to ensure that a Kernel classifier converges in robustness in addition to accuracy.

\section{Preliminaries}

\subsection{Setting}
We consider binary classification where instances are drawn from a totally bounded metric space $\X$ that is equipped with distance metric denoted by $d$, and the label space is $\Y = \{ -1, +1 \}$. The classical goal of classification is to build a highly \textit{accurate} classifier, which we define as follows.

\begin{defn}
(Accuracy) Let $\D$ be a distribution over $\X \times \Y$, and let $f \in \Y^\X$ be a classifier. Then the \textbf{accuracy} of $f$ over $\D$, denoted $A(f, \D)$, is the fraction of examples $(x,y) \sim \D$ for which $f(x) = y$. Thus $$A(f, \D) = P_{(x,y) \sim \D}[f(x) = y].$$
\end{defn}

In this work, we consider \textit{robustness} in addition to accuracy. Let $B(x,r)$ denoted the closed ball of radius $r$ centered at $x$. 

\begin{defn}
(Robustness) A classifier $f \in \Y^\X$ is said to be \textbf{robust} at $x$ with radius $r$ if $f(x) = f(x')$ for all $x' \in B(x,r)$.
\end{defn}

Our goal is to find non-parametric algorithms that output classifiers that are robust, in addition to being accurate. To account for both criteria, we combine them into a notion of \textit{astuteness}~\cite{WJC18, Madry18}. 

\begin{defn}
(Astuteness) A classifier $f \in \Y^\X$ is said to be \textbf{astute} at $(x,y)$ with radius $r$ if $f$ is robust at $x$ with radius $r$ and $f(x) = y$. The \textbf{astuteness} of $f$ over $\D$, denoted $A_r(f, \D)$, is the fraction of examples $(x,y) \sim \D$ for which $f$ is astute at $(x,y)$ with radius $r$. Thus $$A_r(f, \D) = P_{(x, y) \sim \D}[f(x') = y, \forall x' \in B(x,r)].$$
\end{defn}

It is worth noting that $A_0(f, \D) = A(f, \D)$, since astuteness with radius $0$ is simply the accuracy. For this reason, we will use $A_0(f, \D)$ to denote accuracy from this point forwards.

\subsection{Notions of Consistency}

Traditionally, a classification algorithm is said to be consistent if as the sample size grows to infinity, the accuracy of the classifier it learns converges towards the best possible accuracy on the underlying data distribution. We next introduce and formalize an alternative form of consistency, called $r$-consistency, that applies to robust classifiers.

We begin with a formal definition of the Bayes Optimal Classifier -- the most accurate classifier on a distribution -- and consistency. 

\begin{defn}
(Bayes Optimal Classifier) The \textbf{Bayes Optimal Classifier} on a distribution $\D$, denoted by $\b$, is defined as follows. Let $\eta(x) = p_\D(+1|x)$. Then
 \[ \b(x) = \begin{cases} 
      +1 & \eta(x) \geq 0.5 \\
      -1 & \eta(x) < 0.5 \\
   \end{cases}
\]
It can be shown that $\b$ achieves the highest accuracy over $\D$ over all classifiers.
\end{defn}

\begin{defn}
(Consistency) Let $M$ be a classification algorithm  over $\X \times \Y$. $M$ is said to be \textbf{consistent} if for any $\D$ over $\X \times \Y$, and any $\epsilon, \delta$ over $(0,1)$, there exists $N$ such that for $n \geq N$, with probability $1-\delta$ over $S \sim \D^n$, we have: $$A(M(S), \D) \geq A(\b, \D) - \epsilon,$$ where $\b$ is the Bayes optimal classifier for $\D$. 
\end{defn}

How can we incorporate robustness in addition to accuracy in this notion? A plausible way, as used in~\cite{WJC18}, is that the classifier should converge towards being astute where the Bayes Optimal classifier is astute. However, the Bayes Optimal classifier is not necessarily the most astute classifier and may even have poor astuteness. To see this, consider the following example. 

\paragraph{Example 1}
Consider $\D$ over $\X = [0,1]$ such that $\D_\X$ is the uniform distribution and $$p(y=1|x) = \frac{1}{2} + \sin \frac{4 \pi x}{r}.$$ For any point $x$, there exists $x_1, x_2 \in ([x-r, x+r] \cap [0,1])$ such that $p(y=1|x_1) > \frac{1}{2}$ and $p(y=1|x_2) < \frac{1}{2}$. $A_r(\b, r) = 0$. However, the classifier that always predicts $f(x) = +1$ does better. It is robust everywhere, and since $P_{(x,y) \sim \D}[y = +1] = \frac{1}{2}$, it follows that $A_r(f, \D) = \frac{1}{2}$. \\ \\

This motivates the notion of the $r$-optimal classifier, introduced by~\cite{YRWC19}, which is the classifier with maximum astuteness. 

\begin{defn}
($r$-optimal classifier) The \textbf{$r$-optimal classifier} of a distribution $G$ denoted by $\b_r$ is the classifier with maximum astuteness. Thus $$\b_r = \argmax_{f \in \Y^\X} A_r(f, \D).$$ We let $A_r^*(\D)$ denote $A_r(\b_r, \D)$. 
\end{defn}

Observe that $\b_r$ is not necessarily unique. To account for this, we use $A_r^*(\D)$ in our definition for \rconsy. 

\begin{defn} \label{defn_archons}
(\rcons) Let $M$ be a classification algorithm over $\X \times \Y$. $M$ is said to be \textbf{\rcons} if for any $\D$,  any $\epsilon, \delta \in (0,1)$, and $0 < \gamma < r$, there exists $N$ such that for $n \geq N$, with probability $1-\delta$ over $S \sim \D^n$, $$A_{r-\gamma}(M(S), \D) \geq A_r^*(\D) - \epsilon.$$ if the above conditions hold for a specific distribution $\D$, we say that $M$ is \rcons\emph{ }with respect to $\D$. 
\end{defn}

Observe that in addition to the usual $\epsilon$ and $\delta$, there is an extra parameter $\gamma$ which measures the gap in the robustness radius. We may need this parameter as when classes are exactly $2r$ apart, we may not be able to find the exact robust boundary with only finite samples. 

Our analysis will be centered around understanding what kinds of algorithms $M$ provide highly astute classifiers for a given radius $r$. We begin by first considering the special case of \textit{$r$-separated} distributions. 

\begin{defn}
($r$-separated distributions) A distribution $\D$ is said to be \textbf{$r$-separated} if there exist subsets $T^+, T^- \subset \X$ such that 
\begin{enumerate}
	\item $\P_{(x,y) \sim \D}[x \in T^y] = 1$. 
	\item $\forall x_1 \in T^+, \forall x_2 \in T^-$, $d(x_1, x_2) > 2r$.
\end{enumerate}
\end{defn}

Observe that if $\D$ is $r$-separated, $A_r(\b_r, \D) = 1$.

\subsection{Non-parametric Classifiers}\label{classifiers}

Many non-parametric algorithms classify points by averaging labels over a local neighborhood from their training data. A very general form of this idea is encapsulated in \textit{weight functions} -- which is the general form we will use.

\begin{defn} \label{def:weight}
\cite{devroye96} A \textbf{weight function} $W$ is a non-parametric classifier with the following properties.
\begin{enumerate}
	\item Given input $S = \{(x_1, y_1), (x_2, y_2,), \dots, (x_n, y_n)\} \sim \D^n$, $W$ constructs functions $w_1^S, \dots, w_n^S: \X \to [0, 1]$ such that for all $x \in \X$, $\sum_1^n w_i^S(x) = 1$. The functions $w_i^S$ are allowed to depend on $x_1, x_2, \dots x_n$ but must be independent of $y_1, y_2, \dots, y_n$. 
	\item $W$ has output $W_S$ defined as \[ W_S(x) = \begin{cases} 
      +1 & \sum_1^n w_i^S(x)y_i > 0 \\
      -1 & \sum_1^n w_i^S(x)y_i \leq 0 \\
   \end{cases}
\]
As a result, $w_i^S(x)$ can be thought of as the weight that $(x_i, y_i)$ has in classifying $x$.
\end{enumerate}
\end{defn}

Weight functions encompass a fairly extensive set of common non-parametric classifiers, which is the motivation for considering them. We now define several common non-parametric algorithms that can be construed as weight functions. 

\begin{defn}
A \textbf{histogram classifier}, $H$, is a non-parametric classification algorithm over $\R^d \times \Y$ that works as follows. For a distribution $\D$ over $\R \times \Y$, $H$ takes $S = \{(x_i, y_i): 1 \leq i \leq n\} \sim \D^n$ as input. Let $k_i$ be a sequence with $\lim_{i \to \infty} k_i = \infty$ and $\lim_{i \to \infty} \frac{k_i}{i} = 0$. $H$ constructs a set of hypercubes $C = \{c_1, c_2, \dots, c_m\}$ as follows:
\begin{enumerate}
	\item Initially $C = \{c\}$, where $S \subset c$.
	\item For $c \in C$, if $c$ contains more than $k_n$ points of $S$, then partition $c$ into $2^d$ equally sized hypercubes, and insert them into $C$.
	\item Repeat step $2$ until all cubes in $C$ have at most $k_n$ points. 
\end{enumerate}
For $x \in \R$ let $c(x)$ denote the unique cell in $C$ containing $x$. If $c(x)$ doesn't exist, then $H_S(x) = -1$ by default. Otherwise, \[ H_S(x) = \begin{cases} 
      +1 & \sum_{x_i \in c(x)} y_i > 0 \\
      -1 & \sum_{x_i \in c(x)}y_i \leq 0 \\
   \end{cases}.
\]
\end{defn}

Histogram classifiers are weight functions in which all $x_i$ contained within the same cell as $x$ are given the same weight $w_i^S(x)$ in predicting $x$, while all other $x_i$ are given weight $0$. 

\begin{defn}
A \textbf{kernel classifier} is a weight function $W$ over $\X \times \Y$ constructed from function $K: \R^+ \cup \{0\} \to \R^+$ and some sequence $\{h_n\} \subset \R^+$ in the following manner. Given $S = \{(x_i, y_i)\} \sim \D^n$, we have $$w_i^S(x) = \frac{K(\frac{d(x, x_i)}{h_n})}{\sum_{j = 1}^n K(\frac{d(x, x_j)}{h_n})}.$$ Then, as above, $W$ has output \[ W_S(x) = \begin{cases} 
      +1 & \sum_1^n w_i^S(x)y_i > 0 \\
      -1 & \sum_1^n w_i^S(x)y_i \leq 0 \\
   \end{cases}
\]
\end{defn}

Finally, we note that $k_n$-nearest neighbors is also a weight function; $w_i^S(x) = \frac{1}{k_n}$ if $x_i$ is one of the $k_n$ closest neighbors of $x$ and $0$ otherwise.

\section{Warm Up: $r$-separated distributions}

We begin by considering the case when the data distribution is $r$-separated; the more general case is considered in Section~\ref{sec:general}. While classifying $r$-separated distributions robustly may appear almost trivial, learning an arbitrary classifier does not necessarily produce an astute result. To see this, consider the following example of a histogram classifier -- which is known to be consistent. 


We let $H$ denote the histogram classifier over $\R$.

\paragraph{Example 2}
Consider the data distribution $\D = \D^+ \cup \D^-$ where $D^+$ is the uniform distribution over $[0, \frac{1}{4})$ and $D^-$ is the uniform distribution over $(\frac{1}{2}, 1]$, $p(+1|x) = 1$ for $x \in \D^+$, and $p(-1|x) = 1$ for $x \in \D^-$. 

We make the following observations (refer to Figure \ref{fig:histogram}).
\begin{enumerate}
	\item $\D$ is $0.1$-separated, since the supports of $\D^+$ and $\D^-$ have distance $0.25 > 0.2$. 
	\item If $n$ is sufficiently large, $H$ will construct the cell $[0.25, 0.5)$, which will not be split because it will never contain any points. 
	\item $H_S(x) = -1$ for $x \in [0.25, 0.5)$.
	\item $H_S$ is not astute at $(x,1)$ for $x \in (0.15, 0.25)$. Thus $A_{0.1}(H_S, \D) = 0.8$.
\end{enumerate}

\begin{figure}
\centering
\definecolor{dtsfsf}{rgb}{0.8274509803921568,0.1843137254901961,0.1843137254901961}
\definecolor{sexdts}{rgb}{0.1803921568627451,0.49019607843137253,0.19607843137254902}
\begin{tikzpicture}[line cap=round,line join=round,>=triangle 45,x=1cm,y=1cm]
\clip(-4.68,0) rectangle (2.46,3);
\draw [line width=2pt] (-3.8971542001619626,1.7601524817792173)-- (-3.90319974718047,1.0800284421971598);
\draw [line width=2pt] (-2.7721542001619626,1.7501524817792171)-- (-2.7781997471804694,1.0700284421971598);
\draw [line width=2pt] (-1.6471542001619623,1.7401524817792173)-- (-1.6531997471804694,1.0600284421971597);
\draw [line width=2pt] (0.6028457998380374,1.720152481779217)-- (0.5968002528195302,1.0400284421971597);
\draw [line width=2pt] (-2.775,1.43)-- (-1.65,1.42);
\draw [line width=2pt,color=sexdts] (-1.65,1.42)-- (0.6,1.4);
\draw [line width=2pt,color=sexdts] (-3.9,1.44)-- (-3.2002330680045032,1.4337798494933733);
\draw [line width=2pt,color=dtsfsf] (-3.2002330680045032,1.4337798494933733)-- (-2.775,1.43);
\draw (-3.54,2.48) node[anchor=north west] {$\mathcal{D}^+$};
\draw (-0.76,2.46) node[anchor=north west] {$\mathcal{D}^-$};
\draw (-4,1.1) node[anchor=north west] {0};
\draw (-3.14,1.08) node[anchor=north west] {0.25};
\draw (-1.86,1.1) node[anchor=north west] {0.5};
\draw (0.5,1.08) node[anchor=north west] {1};
\end{tikzpicture}
\caption{$H_S$ is astute in the green region, but not robust in the red region.} \label{fig:histogram}
\end{figure}

Example 2 shows that histogram classifiers do not always learn astute classifiers even when run on $r$-separated distributions. This motivates the question: which non-parametric classifiers do?

We answer this question in the following theorem, which gives sufficient conditions for a weight function (definition \ref{def:weight}) to be $r$-consistent over an $r$-separated distribution.

\begin{thm}\label{thm_stone_cons}
Let $\D$ be a distribution over $\X \times \Y$, and let $W$ be a weight function. Let $X$ be a random variable with distribution $\D_\X$, and $S = \{(x_1, y_1), (x_2, y_2), \dots, (x_n, y_n)\} \sim \D^n$. Suppose that for any $0 < a < b,$ $$\lim_{n \to \infty} \mathbb{E}_{X, S} \big [ \sup_{x' \in B(X, a)} \sum_1^n w_i^S(x')I_{||x_i - x'|| > b} \big] = 0.$$  Then if $\D$ is $r$-separated, $W$ is \rcons\emph{ } with respect to $\D$.  
\end{thm}

First, we compare Theorem \ref{thm_stone_cons} to Stone's theorem \cite{Stone77}, which gives sufficient conditions for a weight function to be consistent (i.e. converge in accuracy towards the Bayes optimal). For convenience, we include a statement of Stone's theorem. 
\begin{thm}\label{thm_stone}
\cite{Stone77} Let $W$ be weight function over $\X \times \Y$. Suppose the following conditions hold for any distribution $\D$ over $\X \times \Y$.  Let $X$ be a random variable with distribution $\D_\X$, and $S = \{(x_1, y_1), (x_2, y_2), \dots, (x_n, y_n)\} \sim \D^n$. All expectations are taken over $X$ and $S$. 
\begin{enumerate}
	\item There is a constant $c$ such that, for every nonnegative measurable function $f$ satisfying $\mathbb{E} [f(X)] < \infty$, $$\mathbb{E} [\sum_1^n w_i^S(X)f(x_i)] \leq c \mathbb{E} [f(x)].$$
	\item For all $a > 0$, $$\lim_{n \to \infty} \mathbb{E}[\sum_1^n w_i^S(x)I_{||x_i - X|| > a}] = 0,$$ where $I_{||x_i - X|| > a}$ is an indicator variable. 
	\item $$\lim_{n \to \infty} \mathbb{E}[\max_{1 \leq i \leq n} w_i^S(X)] = 0.$$
\end{enumerate}
Then $W$ is consistent. 
\end{thm}
There are two main differences between Theorem \ref{thm_stone_cons} and Stone's theorem.
 \begin{enumerate}
 	\item Conditions 1. and 3. of Stone's theorem are no longer necessary. This is because $r$-separated distributions are well-separated and thus have simpler conditions for consistency. In fact, a slight modification of the arguments of~\cite{Stone77} shows that for $r$-separated distributions, condition 2. alone is sufficient for consistency.
 	\item Condition 2. is strengthened. Instead of requiring the weight of $x_i$'s outside of a given radius to go to $0$ for $X \sim \D$, we require the same to \textit{uniformly} hold over a ball centered at $X$. 
\end{enumerate}  
 


Theorem \ref{thm_stone_cons} provides a general condition that allows us to verify the $r$-consistency of non-parametric methods. We now show below that two common non-parametric algorithms -- $k_n$-nearest neighbors and kernel classifiers with rapidly decaying kernel functions -- satisfy the conditions of Theorem~\ref{thm_stone_cons}.
 
\begin{cor}\label{nn_sep_thm}
Let $\D$ be any $r$-separated distribution. Let $k_n$ be any sequence such that $\lim_{n \to \infty} \frac{k_n}{n} = 0$, and let $M$ be the $k_n$-nearest neighbors classifier on a sample $S \sim \D^n$. Then $M$ is \rcons\emph{ }with respect to $\D$. 
\end{cor}

\paragraph{\textbf{Remarks:}}
\begin{enumerate}
	\item Because the data distribution is $r$-separated, $k_n = 1$ will be $r$-consistent. Also observe that for $r$-separated distributions, $k_n = 1$ will converge towards the Bayes Optimal classifier.
	\item In general, $M$ converges towards the Bayes Optimal classifier provided that $k_n \to \infty$ in addition to $k_n /n \to 0$. This condition is not necessary for \rconsy -- because the distribution is $r$-separated. 
\end{enumerate}



We next show that kernel classifiers are also $r$-consistent on $r$-separated data distributions, provided the kernel function decreases rapidly enough. 

\begin{cor}\label{thm_kernel}
Let $W$ be a kernel classifier over $\X \times \Y$ constructed from $K$ and $h_n$. Suppose the following properties hold for $K$ and $h_n$.
\begin{enumerate}
	\item For any $c > 1$, $\lim_{x \to \infty} \frac{K(cx)}{K(x)} = 0.$
	\item $\lim_{n \to \infty} h_n = 0.$
\end{enumerate}
If $\D$ is an $r$-separated distribution over $\X \times \Y$, then $W$ is \rcons\emph{ }with respect to $\D$. 
\end{cor}

Observe that Condition 1. is satisfied for any $K(x)$ that decreases more rapidly than an inverse polynomial -- and is hence satisfied by most popular kernels like the Gaussian kernel. Is the condition on $K$ in Corollary~\ref{thm_kernel} necessary? The following example illustrates that a kernel classifier with any arbitrary $K$ is not necessarily $r$-consistent. This indicates that some sort of condition needs to be imposed on $K$ to ensure $r$-consistency; finding a tight necessary condition however is left for future work. 

 \paragraph{Example 3} Let $\X = [-1, 1]$ and let $\D$ be a distribution with $p_\D(-1, -1) = 0.1$ and $p_\D(1, 1) = 0.9$. Clearly, $\D$ is $0.3$-separated. Let $K(x) = e^{-\min(|x|, 0.2)^2}$. Let $h_n$ be any sequence with $\lim_{n \to \infty} h_n = 0$ and $\lim_{n \to \infty} nh_n = \infty$. Let $W$ be the weight classifier with input $S = \{(x_1, y_1), (x_2, y_2), \dots, (x_n, y_n)\}$ such that $$w_i^S(x) = \frac{K(\frac{|x- x_i|}{h_n})}{\sum_{j=1}^n K(\frac{|x-x_j|}{h_n})}.$$ $W$ can be shown to satisfy all the conditions of Theorem \ref{thm_stone} (the proof is analogous to the case for a Gaussian Classifier), and is therefore consistent. However, $W$ does not learn a robust classifier on $\D$ for $r = 0.3$. 

Consider $x = -0.7$. For any $\{(x_1, y_1), (x_2, y_2), \dots, (x_n, y_n)\} \sim \D^n$, all $x_i$ will either be $-1$ or $1$. Therefore, since $K(|x - (-1)|) = K(|x - 1|)$, it follows that $w_i^S(x) = \frac{1}{n}$ for all $1 \leq i \leq n$. Since $x_i = 1$ with probability $0.9$, it follows that with high probability $x$ will be classified as $1$ which means that $f$, the output of $W$, is not robust at $x = -1$. Thus $f$ has astuteness at most $0.9$ which means that $W$ is \textit{not} \rcons\ for $r=0.3$. 

\section{General Distributions}\label{sec:general}


We next consider more general data distributions, where data from different classes may be close together in space, and may even overlap. Observe that unlike the $r$-separated case, here there may be no classifier with astuteness one. Thus, a natural question is: what does the optimally astute classifier look like, and how can we build non-parametric classifiers to this limit?

\subsection{The $r$-Optimal Classifier and Adversarial Pruning}

\cite{YRWC19} propose a large-sample limit -- called the $r$-optimal -- and show that it is analogous to the Bayes Optimal classifier for robustness. More specifically, given a data distribution $D$, to find the $r$-optimal classifier, we solve the following optimization problem.  

\begin{equation}\label{optim_prob}
\begin{split}
\max_{S_{+1}, S_{-1}} &\int_{x \in S_{+1}} p(y=+1|x)d\mu_{\D}(x) + \\
&\int_{x \in S_{-1}} p(y=-1|x)d\mu_{\D}(x) \\
&\text{ subject to } d(S_{+1}, S_{-1}) > 2r 
\end{split}
\end{equation}

Then, the $r$-optimal classifier is defined as follows. 

\begin{defn}
\cite{YRWC19} Fix $r, \D$. Let $S_{+1}^*$ and $S_{-1}^*$ be any optimizers of (\ref{optim_prob}). Then the $r$-optimal classifier, $\b_r$ is any classifier such that $\b_r(x) = j$ whenever $d(S_j^*, x) \leq r$. 
\end{defn}

\cite{YRWC19} show that the $r$-optimal classifier achieves the optimal astuteness -- out of all classifiers on the data distribution $\D$; hence, it is a robustness analogue to the Bayes Optimal Classifier. Therefore, for general distributions, the goal in robust classification is to find non-parametric algorithms that output classifiers that converge towards $\b_r$. 

To find robust classifiers, \cite{YRWC19} propose Adversarial Pruning -- a defense method that preprocesses the training data by making it better separated. More specifically, Adversarial Pruning takes as input a training dataset $S$ and a radius $r$, and finds the largest subset of the training set where differently labeled points are at least distance $2r$ apart. 


\begin{defn}
A set $S_r \subset \X \times \Y$ is said to be \textbf{$r$-separated} if for all $(x_1, y_1), (x_2, y_2) \in S_r$, if $y_1 \neq y_2$, then $d(x_1, x_2) > 2r$. To \textbf{adversarially prune} a set $S$ is to return its largest $r$-separated subset. We let $\ap(S, r)$ denote the result of adversarially pruning $S$.  
\end{defn}

Once an $r$-separated subset $S_r$ of the training set is found, a standard non-parametric method is trained on $S_r$.  While~\cite{YRWC19} show good empirical performance of such algorithms, no formal guarantees are provided. We next formally characterize when adversarial pruning followed by a non-parametric method results in a classifier that is provably $r$-consistent.

Specifically, we consider analyzing the general algorithm provided in Algorithm \ref{alg:gen}.

%\begin{algorithm}[tb]
%   \caption{\ga}
%   \label{alg:gen}
%\begin{algorithmic}
%   \STATE {\bfseries Input:} $S \sim \D^n$, weight function $W$, 
%   robustness radius $r$
%   \STATE $S_r \leftarrow \ap(S, r)$
%   \STATE{\bfseries Output:} $W_{S_r}$
%\end{algorithmic}
%\end{algorithm}

\begin{algorithm}[H]
    \SetAlgoLined
    {\bfseries Input:} $S \sim \D^n$, weight function $W$, robustness radius $r$\;
    
    $S_r \leftarrow \ap(S, r)$\;
    
    {\bfseries Output:} $W_{S_r}$\;
    

\caption{\ga}\label{alg:gen}
\end{algorithm}

\subsection{Convergence Guarantees}

We begin with some notation. For any weight function $W$ and radius $r > 0$, we let $\ga(W,r)$ represent the weight function that outputs weights for $S \sim \D^n$ according to $\ga(S, W, r)$. In particular, this can be used to convert any weight function algorithm into a new weight function which takes robustness into account. A natural question is, for which weight functions $W$ is $\ga(W,r)$ \rcons? Our next theorem provides sufficient conditions for this.

\begin{thm}\label{thm_weight_general}
Let $W$ be a weight function over $\X \times \Y$, and let $\D$ be a distribution over $\X \times \Y$. Fix $r >0$. Let $S_r = \ap(S, r)$.  For convenience, relabel $x_i, y_i$ so that $S_r = \{(x_1, y_1), (x_2, y_2), \dots, (x_m, y_m)\}$. Suppose that for any $0 < a < b,$ 
\begin{equation*}\label{condition}
\lim_{n \to \infty} \mathbb{E}_{S \sim \D^n}\big [ \frac{1}{m} \sum_{i = 1}^m \sup_{x \in B(x_i, a)} \sum_{j = 1}^m w_j^{S_r}(x)I_{||x_j - x|| > b} \big] = 0. 
\end{equation*}
Then $\ga(W,r)$ is \rcons\emph{ }with respect to $\D$. 
\end{thm}

\paragraph{\textbf{Remark}:}
There are two important differences between the conditions in Theorem \ref{thm_weight_general} and Theorem~\ref{thm_stone_cons}.
\begin{enumerate}
	\item We replace $S$ with $S_r$.
	\item The expectation over $X \sim \D_\X$ is replaced with an average over $\{x_1, x_2, \dots, x_m\}$. The intuition here is that we are replacing $\D$ with a uniform distribution over $S_r$. While $\D$ may not be $r$-separated, the uniform distribution over $S_r$ is, and represents the region of points where our classifier is astute. 
\end{enumerate}

A natural question is what satisfies the conditions in Theorem~\ref{thm_weight_general}. We next show that $k_n$-nearest neighbors and kernel classifiers with rapidly decaying kernel functions continue to satisfy the conditions in Theorem \ref{thm_weight_general}; this means that these classifiers, when combined with Adversarial Pruning, will converge to $r$-optimal classifiers in the large sample limit.

\begin{cor}\label{thm_NN_gen}
Let $k_n$ be a sequence with $\lim_{n \to \infty} \frac{k_n}{n} = 0$, and let $M$ denote the $k_n$-nearest neighbor algorithm. Then for any $r > 0$, $\ga(M, r)$ is \rcons.
\end{cor}
\paragraph{\textbf{Remark}:} Corollary \ref{thm_NN_gen} gives a formal guarantee in the large sample limit for the modified nearest-neighbor algorithm proposed by \cite{YRWC19}.

\begin{cor}\label{thm_kern_gen}
Let $W$ be a kernel classifier over $\X \times \Y$ constructed from $K$ and $h_n$. Suppose the following properties hold for $K$ and $h_n$.
\begin{enumerate}
	\item For any $c > 1$, $\lim_{x \to \infty} \frac{K(cx)}{K(x)} = 0.$
	\item $\lim_{n \to \infty} h_n = 0.$
\end{enumerate}
Then for any $r > 0$, $\ga(W, r)$ is \rcons.
\end{cor}

Observe again that Condition 1. is satisfied by any $K$ that decreases more rapidly than an inverse polynomial kernel; it is thus satisfied by most popular kernels, such as the Gaussian kernel. 

\section{Validation}
%\begin{figure}[ht]
%\vskip 0.2in
%\begin{center}
%\subfloat[][Noiseless Histogram]{\includegraphics[width=.29\textwidth]{hist_noiseless_final}}\quad
%   \subfloat[][Noisy Histogram]{\includegraphics[width=.29\textwidth]{hist_noisy_final}} \quad
%   \subfloat[][Histogram trained on 500 samples]{\includegraphics[width=.29\textwidth]{visual500}}\\
%   \subfloat[][Noiseless 1-NN]{\includegraphics[width=.29\textwidth]{nn_noiseless_final}} \quad
%   \subfloat[][Noisy 1-NN]{\includegraphics[width=.29\textwidth]{nn_noisy_final}}\quad
%   \subfloat[][Histogram trained on 3000 samples]{\includegraphics[width=.29\textwidth]{visual3000}}
%\end{center}
%\caption{Empirical accuracy/astuteness of different classifiers as a function of training sample size. Accuracy is shown in green, astuteness in purple. Left : Noiseless Setting. Right: Noisy Setting. Top Row: Histogram Classifier, Bottom Row: 1-Nearest Neighbor}
%\label{fig:1}
%\vskip -0.2in
%\end{figure}

\begin{figure}
\begin{subfigure}{0.31\textwidth}
\includegraphics[width=\linewidth]{hist_noiseless_final}
%\caption{First subfigure} \label{fig:a}
\end{subfigure}\hspace*{\fill}
\begin{subfigure}{0.31\textwidth}
\includegraphics[width=\linewidth]{hist_noisy_final}
%\caption{Second subfigure} \label{fig:b}
\end{subfigure}\hspace*{\fill}
\begin{subfigure}{0.31\textwidth}
\includegraphics[width=\linewidth]{visual500}
%\caption{Fifth subfigure} \label{fig:e}
\end{subfigure}

\medskip
\begin{subfigure}{0.31\textwidth}
\includegraphics[width=\linewidth]{nn_noiseless_final}
%\caption{Third subfigure} \label{fig:c}
\end{subfigure}\hspace*{\fill}
\begin{subfigure}{0.31\textwidth}
\includegraphics[width=\linewidth]{nn_noisy_final}
%\caption{Fourth subfigure} \label{fig:d}
\end{subfigure}\hspace*{\fill}
\begin{subfigure}{0.31\textwidth}
 \includegraphics[width=\linewidth]{visual3000}
%\caption{Sixth subfigure} \label{fig:f}

\end{subfigure}

\caption{Empirical accuracy/astuteness of different classifiers as a function of training sample size. Accuracy is shown in green, astuteness in purple. Left : Noiseless Setting. Right: Noisy Setting. Top Row: Histogram Classifier, Bottom Row: 1-Nearest Neighbor} \label{fig:1}
\end{figure}

Our theoretical results are, by nature, large sample; we next validate how well they apply to the finite sample case by trying them out on a simple example. In particular, we ask the following question:

\begin{quote}
How does the robustness of non-parametric classifiers change with increasing sample size?
\end{quote}

This question is considered in the context of two simple non-parametric classifiers -- one nearest neighbor (which is guaranteed to be $r$-consistent) and histograms (which is not). To be able to measure performance with increasing data size, we look at a simple synthetic dataset -- the Half Moons. 

\subsection{Experimental Setup}

\paragraph{Classifiers and Dataset.} We consider two different classification algorithms -- one nearest neighbor (NN) and a Histogram Classifier (HC).  We use the Halfmoon dataset with two settings of the gaussian noise parameter $\sigma$, $\sigma = 0$ (Noiseless) and $\sigma =0.08$ (Noisy). For the Noiseless setting, observe that the data is already $0.1$-separated; for the Noisy setting, we use Adversarial Pruning (Algorithm~\ref{alg:gen}) with parameter $r = 0.1$ for both classification methods.

\paragraph{Performance Measure.} We evaluate robustness with respect to the $\ell_{\infty}$ metric, that is commonly used in the adversarial examples literature. Specifically, for each classifier, we calculate the {\em{empirical astuteness}}, which is the fraction of test examples on which it is astute.

Observe that computing the empirical astuteness of a classifier around an input $x$ amounts to finding the adversarial example that is {\em{closest to}} $x$ according to the $\ell_{\infty}$ norm. For the $1$-nearest neighbor, we do this using the optimal attack algorithm proposed by Yang et. al.~\cite{YRWC19}. For the histogram classifier, we use the optimal attack framework proposed by~\cite{YRWC19}, and show that the structure of the classifier can be exploited to solve the convex program efficiently. Details are in Appendix C.

We use an attack radius of $r = 0.1$ for the Noiseless setting, and $r = 0.09$ for the Noisy setting. For all classification algorithms, we plot the empirical astuteness as a function of the training set size. As a baseline, we also plot their standard accuracy on the test set. 

\subsection{Results}

The results are presented in Figure~\ref{fig:1}; the left two panels are for the Noiseless setting while the two center ones are for the Noisy setting.  

The results show that as predicted by our theory, for the Noiseless setting, the empirical astuteness of nearest neighbors converges to $1$ as the training set grows. For Histogram Classifiers, the astuteness converges to $0.5$ -- indicating that the classifier may grow less and less astute with higher sample size even for well-separated data. This is plausibly because the cell size induced by the histogram grows smaller with growing training data; thus, the classifier that outputs the default label $-1$ in empty cells is incorrect on adversarial examples that are close to a point with $+1$ label, but belongs to a different, empty cell. The rightmost panels in Figure~\ref{fig:1} provide a visual illustration of this process. 

For the Noisy setting, the empirical astuteness of adversarial pruning followed by nearest neighbors converges to $0.8$. For histograms with adversarial pruning, the astuteness converges to $0.7$, which is higher than the noiseless case but still clearly sub-optimal.

\subsection{Discussion}

Our results show that even though our theory is asymptotic, our predictions continue to be relevant in finite sample regimes. In particular, on well-separated data, nearest neighbors that we theoretically predict to be intrinsically robust is robust; histogram classifiers, which do not satisfy the conditions in Theorem~\ref{thm_stone_cons} are not. Our predictions continue to hold for data that is not well-separated. Nearest neighbors coupled with Adversarial
Pruning continues to be robust with growing sample size, while histograms continue to be non-robust. Thus our theory is confirmed by practice.

\section{Conclusion}

In conclusion, we rigorously analyze when non-parametric methods provide classifiers that are robust in the large sample limit. We provide a general condition that characterizes when non-parametric methods are robust on well-separated data, and show that Adversarial Pruning of~\cite{YRWC19} works on data that is not well-separated. 

Our results serve to provide a set of guidelines that can be used for designing non-parametric methods that are robust and accurate on well-separated data; additionally, we demonstrate that when data is not well-separated, preprocessing by adversarial pruning~\cite{YRWC19} does lead to optimally astute solutions in the large sample limit. 




%\newcommand{\crop}[1]{\mathrm{crop}({#1})}
\newcommand{\object}[1]{\mathrm{object}({#1})}
\newcommand{\ba}{A_i}
\newcommand{\bb}{B_i}
\newcommand{\calA}{\mathcal{A}}
\newcommand{\calB}{\mathcal{B}}
\newcommand{\calX}{\mathcal{X}}
\newcommand{\masked}[1]{\mathrm{masked}({#1})}
\newcommand{\bx}{\mathbf{x}}
\newcommand{\SSL}{\textsc{SSL}}
\newcommand{\SSLbb}{\SSL^\mathrm{back}}
\newcommand{\SSLpj}{\SSL^\mathrm{proj}}
\newcommand{\CLF}{\textsc{CLF}}
\newcommand{\CLFbb}{\CLF^\mathrm{back}}
\newcommand{\CLFpj}{\CLF^\mathrm{proj}}
\newcommand{\SUP}{\textsc{SUP}}
\newcommand{\KNN}{\textsc{KNN}}
\newcommand{\KNNset}{\textsc{KNN}^\mathrm{set}}
\newcommand{\KNNprob}{\textsc{KNN}^\mathrm{prob}}
\newcommand{\KNNcl}{\textsc{KNN}^\mathrm{cl}}
\newcommand{\KNNconf}{\textsc{KNN}^\mathrm{conf}}
\newcommand{\RCDM}{\textsc{RCDM}}
\newcommand{\cl}{\mathrm{cl}}
\newcommand{\clpred}{\tilde{\mathrm{cl}}}
\newcommand{\Abox}{\overline{\calA}}
\newcommand{\Bbox}{\overline{\calB}}
\newcommand{\dejavu}{\emph{déjà vu }}
\newcommand{\Dejavu}{\emph{Déjà vu }}

\newcommand{\citations}{{\color{green}[CITE]}}

\definecolor{part_blue}{rgb}{0.2824, 0.4706, .8157}
\definecolor{part_red}{rgb}{0.8392, 0.3725, 0.3725}
\definecolor{part_orange}{rgb}{0.9333, 0.5216, 0.2902}

\DeclareRobustCommand{\mybox}[2][gray!20]{%
\begin{tcolorbox}[   %% Adjust the following parameters at will.
        % breakable,
        left=0pt,
        right=0pt,
        top=0pt,
        bottom=0pt,
        colback=#1,
        colframe=#1,
        width=\dimexpr\columnwidth\relax, 
        % width=\textwidth, 
        enlarge left by=0mm,
        boxsep=5pt,
        arc=0pt,outer arc=0pt,
        ]
        #2
\end{tcolorbox}
}
%\section{Introduction}
\label{sec:intro}
Self-supervised learning (SSL)~\citep{chen2020simclr, chen2020simsiam, zbontar2021barlow, vicreg, caron2020swav, MAE} aims to learn general representations of content-rich data without explicit labels by solving a \textit{pretext task}. In many recent works, such pretext tasks rely on joint-embedding architectures whereby randomized image augmentations are applied to create multiple views of a training sample, and the model is trained to produce similar representations for those views. When using cropping as random image augmentation, the model learns to associate objects or parts (including the background scenery) that co-occur in an image.
However, doing so also arguably exposes the training data to higher privacy risk as objects in training images can be explicitly memorized by the SSL model. For example, if the training data contains the photos of individuals, the SSL model may learn to associate the face of a person with their activity or physical location in the photo. This may allow an adversary to extract such information from the trained model for targeted individuals.

\begin{figure}[t]
    \centering
    \includegraphics[width=1.0\columnwidth]{figures/new_black_swan.pdf}
    \caption{\textbf{Left:} Reconstruction of an SSL training image from a crop containing only the background. The SSL model memorizes the association of this \emph{specific} patch of water (pink square) to this \emph{specific} foreground object (a black swan) in its embedding, which we decode to visualize the full training image. \textbf{Right:} The reconstruction technique fails on a public test image that the SSL model has not seen before.}
    \label{fig:black_swan}
\end{figure}

In this work, we aim to evaluate to what extent SSL models memorize the association of specific objects in training images or the association of objects and their specific backgrounds, and whether this memorization signal can be used to reconstruct the model's training samples. Our results demonstrate that SSL models memorize such associations beyond simple correlation. For instance, in Figure \ref{fig:black_swan} (\textbf{left}), we use the SSL representation of a \emph{training image crop containing only water} and this enables us to reconstruct the object in the foreground with remarkable specificity---in this case a black swan.
By contrast, in Figure \ref{fig:black_swan} (\textbf{right}), when using the \emph{crop from the background of a test set image} that the SSL model \emph{has not seen before}, its representation only contains enough information to infer, through correlation, that the foreground object was likely some kind of waterbird --- but not the specific one in the image.

Figure \ref{fig:black_swan} shows that SSL models suffer from the unintended memorization of images in their training data---a phenomenon we refer to as \emph{déjà vu memorization}
%\footnote{The French loanword \emph{déjà vu} means already-seen, which reflects the type of unintended memorization of objects that the SSL model saw during training.}.
\footnote{The French loanword \emph{déjà vu} means `already-seen', just as an image is seen and memorized in training.}
Beyond visualizing \emph{déjà vu} memorization through data reconstruction, we also design a series of experiments to quantify the degree of memorization for different SSL algorithms, model architectures, training set size, \emph{etc.} We observe that \emph{déjà vu} memorization is exacerbated by the atypically large number of training epochs often recommended in SSL training, as well as certain hyperparameters in the SSL training objective. Perhaps surprisingly, we show that \emph{déjà vu} memorization occurs even when the training set is large---as large as half of ImageNet~\citep{imagenet}---and can continually worsen even when standard techniques for evaluating learned representation quality (such as linear probing) do not suggest increased overfitting. Our work serves as the first systematic study of unintended memorization in SSL models and motivates future work on understanding and preventing this behavior. Specifically, we: 
\begin{itemize}
    \vspace{-0.5em}
    \item Elucidate how SSL representations memorize aspects of individual training images, what we call \emph{déjà vu} memorization;
    \item Design a novel training data reconstruction pipeline for non-generative vision models. This is in contrast to many prominent reconstruction algorithms like \citep{carlini2021extracting, google_diffusion}, which rely on the model itself to generate its own memorized samples and is not possible for SSL models or classifiers;
    \item Propose metrics to quantify the degree of \dejavu memorization committed by an SSL model. This allows us to observe how \dejavu changes with training epochs, dataset size, training criteria, model architecture and more. 
\end{itemize}

%\section{Preliminaries and Related Work}
\label{sec:related}

\textbf{Self-supervised learning} (SSL) is a machine learning paradigm that leverages unlabeled data to learn representations. Many SSL algorithms rely on \emph{joint-embedding} architectures (\emph{e.g.}, SimCLR~\citep{chen2020simclr}, Barlow Twins~\citep{zbontar2021barlow}, VICReg~\citep{vicreg} and Dino~\citep{Dino}), which are trained to associate different augmented views of a given image. For example, in SimCLR, given a set of images $\calA = \{A_1,\ldots,A_n\}$ and a randomized augmentation function $\mathrm{aug}$, the model is trained to maximize the cosine similarity of draws of $\SSL(\mathrm{aug}(A_i))$ with each other and minimize their similarity with $\SSL(\mathrm{aug}(A_j))$ for $i \neq j$. The augmentation function $\mathrm{aug}$ typically consists of operations such as cropping, horizontal flipping, and color transformations to create different views that preserve an image's semantic properties. 

\paragraph{SSL representations.} Once an SSL model is trained, its learned representation can be transferred to different downstream tasks. This is often done by extracting the representation of an image from the \emph{backbone model}\footnote{SSL methods often use a trick called \emph{guillotine regularization}~\citep{Guillotine}, which decomposes the model into two parts: a \emph{backbone model} and a \emph{projector} consisting of a few fully-connected layers. Such trick is needed to handle the misalignment between the pretext SSL task and the downstream task.} and either training a linear probe on top of this representation or finetuning the backbone model with a task-specific head~\citep{Guillotine}.
%Compared to representations learned by supervised learning, SSL representations are often more robust and transferable~\citep{hendrycks2019using, ericsson2021self}, leading to state-of-the-art result on many downstream tasks. To understand the effectiveness of SSL algorithms, several prior works investigated what kind of information the SSL model has learned~\citep{jing2021understanding, ericsson2021self, kalibhat2022towards, RCDM}. In particular, \citet{RCDM} trained a conditional generative model on SSL representations and showed that they encode richer visual details about the input image compared to supervised learning. 
%However, from a privacy perspective, this may be a cause for concern as the model also has more potential to overfit and memorize precise details about the training data compared to supervised learning. We show concretely that this privacy risk can indeed be realized by defining and measuring \emph{déjà vu} memorization.
It has been shown that SSL representations encode richer visual details about input images than supervised models do \cite{RCDM}. However, from a privacy perspective, this may be a cause for concern as the model also has more potential to overfit and memorize precise details about the training data compared to supervised learning. We show concretely that this privacy risk can indeed be realized by defining and measuring \emph{déjà vu} memorization.
\vspace{-0.5em} 
% \paragraph{Privacy risks in ML.} Overfitting in ML occurs when a model memorizes information specific to its training data rather than general population-level information. When the model is trained on privacy-sensitive data, overfitting is especially harmful as an adversary can infer private information about the training data when given access to the model~\citep{yeom2018privacy, feldman2020does}. The simplest and most well-studied form of privacy risk in ML is susceptibility to \emph{membership inference attacks}~\citep{shokri2017membership, salem2018ml, sablayrolles2019white}, where the adversary infers whether an individual is part of the training set or not. More sophisticated privacy attacks include \emph{attribute inference}~\citep{fredrikson2014privacy, mehnaz2022your, jayaraman2022attribute}, where specific attributes about an individual are inferred given others, and \emph{data reconstruction}~\citep{carlini2021extracting, balle2022reconstructing, guo2022bounding}, where entire training samples are recovered from the trained model. Our study of \emph{déjà vu} memorization is similar to both attribute inference and data reconstruction, leveraging SSL representations of the training image background to infer and reconstruct the foreground object.
% \vspace{-0.5em} 
% \paragraph{Training data extraction in NLP.} Our study of \dejavu memorization in SSL models is inspired by similar work in the natural language processing (NLP) domain. \citet{carlini2019secret} first showed that language models exhibit unintended memorization, where given a context string present in its training data, the model can generate the remaining text at test time. This unintended memorization has been further exploited in \citet{carlini2021extracting} to extract training data from GPT-2~\citep{radford2019language} and, more recently, extended to extract memorized images from Stable Diffusion \citep{google_diffusion}. The way by which these works exploit unintended memorization is similar to ours: given partial information about a training sample, the model is prompted to reveal the rest of the sample. In our case, however, since the SSL model is not generative, extraction is significantly harder and requires careful design.

\paragraph{Privacy risks in ML.} When a model is overfit on privacy-sensitive data, it memorizes specific information about its training examples, allowing an adversary with access to the model to learn private information~\citep{yeom2018privacy, feldman2020does}. Privacy attacks in ML range from the simplest and best-studied \emph{membership inference attacks}~\citep{shokri2017membership, salem2018ml, sablayrolles2019white} to \emph{attribute inference}~\citep{fredrikson2014privacy, mehnaz2022your, jayaraman2022attribute} and \emph{data reconstruction}~\citep{carlini2021extracting, balle2022reconstructing, guo2022bounding} attacks. In the former, the adversary only infers whether an individual participated in the training set. Our study of \emph{déjà vu} memorization is most similar to the latter: we leverage SSL representations of the training image background to infer and reconstruct the foreground object. Our approach reflects similar work in the NLP domain \citep{carlini2019secret, carlini2021extracting}: when prompted with a context string present in the training data, a large language model is shown to generate the remainder of string at test time, revealing sensitive text like home addresses. This method was recently extended to extract memorized images from Stable Diffusion \citep{google_diffusion}.  We exploit memorization in a similar manner: given partial information about a training sample, the model is prompted to reveal the rest of the sample. In our case, however, since the SSL model is not generative, extraction is significantly harder and requires careful design.

%\section{Defining \emph{Déjà Vu} Memorization}
\label{sec:definition}

\paragraph{What is \dejavu memorization?} At a high level, the objective of SSL is to learn general representations of objects that occur in nature. This is often accomplished by associating different parts of an image with one another in the learned embedding. Returning to our example in Figure \ref{fig:black_swan}, given an image whose background contains a patch of water, the model may learn that the foreground object is a water animal such as duck, pelican, otter, \emph{etc.}, by observing different images that contain water from the training set. We refer to this type of learning as \emph{correlation}: the association of objects that tend to co-occur in images from the training data distribution.

A natural question to ask is \emph{``Can the reconstruction of the black swan in Figure \ref{fig:black_swan} be reasoned as correlation?''} The intuitive answer may be no, since the reconstructed image is qualitatively very similar to the original image. However, this reasoning implicitly assumes that for a random image from the training data distribution containing a patch of water, the foreground object is unlikely to be a black swan. Mathematically, if we denote by $\mathcal{P}$ the training data distribution and $A$ the image, then
\begin{equation*}
\label{eq:p_corr}
p_\text{corr} := \mathbb{P}_{A \sim \mathcal{P}}(\mathrm{object}(A) = \texttt{black swan} ~|~ \mathrm{crop}(A) = \texttt{water})
\end{equation*}
is the probability of inferring that the foreground object is a black swan through \emph{correlation}. This probability may be naturally high due to biases in the distribution $\mathcal{P}$, \emph{e.g.}, if $\mathcal{P}$ contains no other water animal except for black swans. In fact, such correlations are often exploited to learn a model for image inpainting with great success~\citep{yu2018generative, ulyanov2018deep}.

Despite this, we argue that reconstruction of the black swan in Figure \ref{fig:black_swan} is \emph{not} due to correlation, but rather due to \emph{unintended memorization}: the association of objects unique to a single training image. As we will show in the following sections, the example in Figure \ref{fig:black_swan} is not a rare success case and can be replicated across many training samples. More importantly, failure to reconstruct the foreground object in Figure \ref{fig:black_swan} (\textbf{right}) on test images hints at inferring through correlation is unlikely to succeed---a fact that we verify quantitatively in Section \ref{sec:label inference accuracy}. Motivated by this discussion, we give a verbal definition of \dejavu memorization below, and design a testing methodology to quantify \dejavu memorization in Section \ref{sec:notation and setup}.
\mybox{\textbf{Definition:} A model exhibits \emph{déjà vu memorization} when it retains information so specific to an individual training image, that it enables recovery of aspects particular to that image given a part that does not contain them.
The recovered aspect must be beyond what can be inferred using only correlations in the data distribution.} 

% \textbf{Definition:} A model exhibits \emph{déjà vu memorization} when it retains information so specific to an individual training image, that it enables recovery of aspects particular to that image given a part that does not contain them.
% The recovered aspect must be beyond what can be inferred using only correlations in the data distribution.


 We intentionally kept the above definition broad enough to encompass different types of information that can be inferred about the training image, including but not restricted to object category, shape, color and position. For example, if one can infer that the foreground object is red given the background patch with accuracy significantly beyond correlation, we consider this an instance of \dejavu memorization as well. We mainly focus on object category to quantify \dejavu memorization in Section \ref{sec:quant} since the ground truth label can be easily obtained. We consider other types of information more qualitatively in the visual reconstruction experiments in Section \ref{sec:visualizing}.

\paragraph{Privacy implications of \dejavu memorization.} \Dejavu memorization can be a cause for concern when the training data contains privacy-sensitive information. As a motivating example, consider an SSL model trained on photos of individuals. If the model exhibits \dejavu memorization then, given the face of an individual, it may be possible to infer where the individual was or even visually reconstruct their location in the training image. Such information leakage raises privacy concerns, especially if there was no prior agreement that the trained model may reveal such information to third parties. This hypothetical scenario serves as a motivation that \dejavu memorization should be carefully examined to avoid unintended disclosure of private information in practical applications.

% \begin{figure*}[h]
%     \centering
%     \includegraphics[width = 0.85\textwidth]{figures/SSL_attack_cartoon.png}
%     \caption{We measure memorization by comparing the `target model' trained on the target image ($\SSL_A$ trained on $A_i$ in above example) with the `reference model' not trained on it ($\SSL_B$, above). \textbf{[Top Strip]} A cropping of the image disjoint from the labeled foreground object is embedded using the target model. This embedding is then labeled by a K-Nearest Neighbor (KNN) adversary built on a public set of labeled images, $X$, which it has also embedded using the target model. \textbf{[Bottom Strip]} To account for correlation, the same procedure is followed with the reference model. If the label is only extracted using the target model, it is counted as memorization. If it is extracted using either model, it is counted as correlation. We find that the KNN adversary's predictions using the target model (trained on attacked examples) are significantly more accurate than they are using the reference model, indicating routine memorization of training examples.}
%     \label{fig:ssl attack cartoon}
% \end{figure*}

\begin{figure}[t]
%%%
%SPIDER
%%%
     % \centering
     % \begin{subfigure}[b]{0.25\textwidth}
     %     \centering
     %     \includegraphics[width=\textwidth]{figures/data_split.png}
     %     % \caption{SimCLR correlated \textit{yellow garden spider} examples}
     %     \label{fig:data split}
     % \end{subfigure}
     % \hfill
     % \begin{subfigure}[b]{0.7\textwidth}
     %     \centering
     %     \includegraphics[width=\textwidth]{figures/pipeline_cartoon.png}
     %     \begin{minipage}{5cm}
     %        \vfill
     %    \end{minipage}
     %     % \caption{SimCLR memorized \textit{yellow garden spider} examples}
     %     \label{fig:pipeline cartoon}
     % \end{subfigure}
     \includegraphics[width=\textwidth]{figures/split_and_pipeline_cartoon.png}
\caption[Overview of testing methodology.]{
Overview of testing methodology. \textbf{Left:} Data is split into \emph{target set} $\calA$, \emph{reference set} $\calB$ and \emph{public set} $\calX$ that are pairwise disjoint. $\calA$ and $\calB$ are used to train two SSL models $\SSL_A$ and $\SSL_B$ in the same manner. $\calX$ is used for KNN decoding or for training an RCDM to reconstruct the input at test time. \textbf{Right:} Given a training image $A_i \in \calA$, we use $\SSL_A$ to embed $\crop{A_i}$ containing only the background, as well as the entire set $\calX$ and find the $k$-nearest neighbors of $\crop{A_i}$ in $\calX$ in the embedding space. These KNN samples can be used directly to infer the foreground object (\emph{i.e.}, class label) in $A_i$ using a KNN classifier, or their embeddings can be averaged as input to the trained RCDM to visually reconstruct the image $A_i$. For instance, the RCDM reconstruction results in Figure \ref{fig:black_swan} (left) when given $\SSL_A(\crop{A_i})$ and results in Figure \ref{fig:black_swan} (right) when given $\SSL_A(\crop{B_i})$ for an image $B_i \in \calB$.
%\textbf{Left:} illustration of the three datasets used in our tests. Two private data sets, $A$ and $B$, of equal size are used to train two SSL models, $\SSL_A$ and $\SSL_B$, respectively. A disjoint public set, $X$, is made available to the memorization test to help decode model embeddings. Memorization is only tested on examples $A_i \in A$ that are unique to set $A$. \textbf{Right:} illustration of inference pipeline used in tests. A periphery cropping that excludes the foreground object is taken from private image $A_i$. The KNN then finds the $k$ public set nearest neighbors of the periphery crop in the embedding space of $\SSL_A$. 
%The $\SSL_A$ representation of these $k$ neighbors and of the crop are used by the conditional generative model, RCDM, to reconstruct the foreground object. The labels of these $k$ neighbors are used to recover the foreground object label. (Not pictured) We repeat this process using reference model $\SSL_B$, not trained on image $A_i$, to determine whether the foreground object is still recoverable by learned correlations, e.g. if black swans were the only objects appearing near water in the data distribution. In this instance, the crop's public set neighbors in $\SSL_B$'s representation space include a variety of water animals like ducks, pelicans, and otters. Meanwhile, with $\SSL_A$, the neighbors are nearly all black swans in the same position as the swan of $A_i$.
}
\label{fig:split_and_pipeline_cartoon}
\end{figure}

\textbf{Distinguishing memorization from correlation.} When measuring \dejavu memorization, it is crucial to differentiate what the model associates through \emph{memorization} and what it associates through \emph{correlation}. Our testing methodology is based on the following intuitive definition.
\mybox{\textbf{Definition:} If an SSL model associates two parts in a training image, we say that it is due to \emph{correlation} if other SSL models trained on a similar dataset from $\mathcal{P}$ without this image would likely make the same association. Otherwise, we say that it is due to \emph{memorization}.}

Notably, such intuition forms the basis for differential privacy (DP; \cite{dwork2006calibrating, dwork2013algorithmic})---the most widely accepted notion of privacy in ML.

\subsection{Testing Methodology for Measuring \emph{Déjà Vu} Memorization}
\label{sec:notation and setup}

In this section, we use the above intuition to measure the extent of \dejavu memorization in SSL. Figure \ref{fig:split_and_pipeline_cartoon} gives an overview of our testing methodology.
\vspace{-0.75em}
\paragraph{Dataset splitting.} We focus on testing \dejavu memorization for SSL models trained on the ImageNet-1K dataset~\citep{imagenet}. Our test first splits the ImageNet training set into three independent and disjoint subsets $\calA$, $\calB$ and $\calX$. The dataset $\calA$ is called the \emph{target set} and $\calB$ is called the \emph{reference set}. The two datasets are used to train two separate SSL models, $\SSL_A$ and $\SSL_B$, called the \emph{target model} and the \emph{reference model}. Finally, the dataset set $\calX$ is used as an auxiliary public dataset to extract information from $\SSL_A$ and $\SSL_B$.
%\footnote{See Appendix \ref{sec:appx splits} for details on how the dataset splits are generated.}.
Our dataset splitting serves the purpose of distinguishing memorization from correlation in the following manner. Given a sample $A_i \in \calA$, if our test returns the same result on $\SSL_A$ and $\SSL_B$ then it is likely due to correlation because $A_i$ is not a training sample for $\SSL_B$. Otherwise, because $\calA$ and $\calB$ are drawn from the same underlying distribution, our test must have inferred some information unique to $A_i$ due to memorization. Thus, by comparing the difference in the test results for $\SSL_A$ and $\SSL_B$, we can measure the degree of \dejavu memorization\footnote{See Appendix \ref{sec:appx splits} for details on how the dataset splits are generated.}.
\vspace{-0.75em}
\paragraph{Extracting foreground and background crops.} Our testing methodology aims at measuring what can be inferred about the foreground object in an ImageNet sample given a background crop. This is made possible because ImageNet provides bounding box annotations for a subset of its training images---around 150K out of 1.3M samples. We split these annotated images equally between $\calA$ and $\calB$. Given an annotated image $A_i$, we treat everything inside the bounding box as the foreground object associated with the image label, denoted $\object{A_i}$. We take the largest possible crop that does not intersect with any bounding box as the background crop (or \emph{periphery crop}), denoted $\crop{A_i}$\footnote{We also present another heuristic in \cref{sec:appx corner crop} which takes a corner crop as the background crop, allowing our test to be run without bounding box annotations.}
%Since the labeled object tends to be at the image's center, the corner crop usually excludes it. }
%Because most images in ImageNet are object centric, an image's corner would not include the foreground object.}.
\vspace{-0.75em}
\paragraph{KNN-based test design.} Joint-embedding SSL approaches encourage the embeddings of random crops of a training image $A_i \in \calA$ to be similar. Intuitively, if the model exhibits \dejavu memorization, it is reasonable to expect that the embedding of $\crop{A_i}$ is similar to that of $\object{A_i}$ since both crops are from the same training image. In other words, $\SSL_A(\crop{A_i})$ encodes information about $\object{A_i}$ that cannot be inferred through correlation. However, decoding such information is challenging as these approaches do not learn a decoder associated with the encoder $\SSL_A$.

Here, we leverage the public set $\calX$ to decode the information contained in $\crop{A_i}$ about $\object{A_i}$. More specifically, we map images in $\calX$ to their embeddings using $\SSL_A$ and extract the $k$-nearest-neighbor (KNN) subset of $\SSL_A(\crop{A_i})$ in $\calX$. We can then decode the information contained in $\crop{A_i}$ in one of two ways:
\begin{itemize}
\item \emph{Label inference:} Since $\calX$ is a subset of ImageNet, each embedding in the KNN subset is associated with a class label. If $\crop{A_i}$ encodes information about the foreground object, its embedding will be close to samples in $\calX$ that have the same class label (\emph{i.e.}, foreground object category). We can then use a KNN classifier to infer the foreground object in $A_i$ given $\crop{A_i}$.
\item \emph{Visual reconstruction:} Following \citet{RCDM}, we train an RCDM---a conditional generative model---on $\calX$ to decode $\SSL_A$ embeddings into images. The RCDM reconstruction can recover qualitative aspects of an image remarkably well, such as recovering object color or spatial orientation using its SSL embedding. Given the KNN subset, we average their SSL embeddings and use the trained RCDM model to visually reconstruct $A_i$.
\end{itemize}
In Section \ref{sec:quant}, we focus on quantitatively measuring \dejavu memorization with label inference, and then use the RCDM reconstruction to visualize \dejavu memorization in Section \ref{sec:visualizing}.
%\section{Quantifying \emph{Déjà Vu} Memorization}
\label{sec:quant}

We apply our testing methodology to quantify a specific form of \dejavu memorization: inferring the foreground object (class label) given a crop of the background.

% \paragraph{Extracting model embeddings.} We test \dejavu memorization on two popular SSL algorithms, SimCLR~\citep{chen2020simclr} and VICReg~\citep{vicreg}.
% %\footnote{We present additional SSL models in \cref{sec:appx simclr results}} 
% As described in Section \ref{sec:related}, these algorithms produce two embeddings given an input image: a \emph{backbone} embedding and a \emph{projector} embedding that is derived by applying a small fully-connected network on top of the backbone embedding. Unless otherwise noted, all SSL embeddings refer to the projector embedding.
% To understand whether \dejavu memorization is particular to SSL, we also evaluate embeddings produced by a supervised model $\CLF_A$ trained on $\calA$. We apply the same set of image augmentations as those used in SSL and train $\CLF_A$ using the cross-entropy loss to predict ground truth labels. 
\vspace{-0.75em}
\paragraph{Extracting model embeddings.} We test \dejavu memorization on a variety of popular SSL algorithms, with a focus on VICReg~\citep{vicreg}. These algorithms produce two embeddings given an input image: a \emph{backbone} embedding and a \emph{projector} embedding that is derived by applying a small fully-connected network on top of the backbone embedding. Unless otherwise noted, all SSL embeddings refer to the projector embedding. 
To understand whether \dejavu memorization is particular to SSL, we also evaluate embeddings produced by a supervised model $\CLF_A$ trained on $\calA$. We apply the same set of image augmentations as those used in SSL and train $\CLF_A$ using the cross-entropy loss to predict ground truth labels. 
\vspace{-0.75em}
\paragraph{Identifying the most memorized samples.} Prior works have shown that certain training samples can be identified as more prone to memorization than others~\citep{feldman2020does, watson2021importance, ye2021enhanced}. Similarly, we provide a heuristic to identify the most memorized samples in our label inference test using confidence of the KNN prediction.
Given a periphery crop, $\crop{A_i}$, let $\KNN_A \big( \crop{A_i} \big) \subseteq \calX$ denote its $k$-nearest neighbors in the embedding space of $\SSL_A$. From this KNN subset we can obtain: \textbf{(1)} $\KNNprob_A \big( \crop{A_i} \big)$, the vector of class probabilities (normalized counts) induced by the KNN subset, and \textbf{(2)} $\KNNconf_A \big( \crop{A_i} \big)$, the negative entropy of the probability vector $\KNNprob_A \big( \crop{A_i} \big)$, as confidence of the KNN prediction. When entropy is low, the neighbors agree on the class of $A_i$ and hence confidence is high. 
% \begin{itemize}[noitemsep, leftmargin=*, topsep=0pt]
%     \item $\KNN_A \big( \crop{A_i} \big)$: The most prevalent class in the KNN subset as prediction for the class label $\cl(A_i)$. 
%     \item $\KNNprob_A \big( \crop{A_i} \big)$: The vector of class probabilities (normalized counts) induced by the KNN subset.
%     \item $\KNNconf_A \big( \crop{A_i} \big)$: Negative entropy of the probability vector $\KNNprob_A \big( \crop{A_i} \big)$ as confidence of the KNN prediction. When entropy is low, the neighbors agree on the class of $A_i$ and hence confidence is high. 
% \end{itemize}
We can sort the confidence score $\KNNconf_A \big( \crop{A_i} \big)$ across samples $A_i$ in decreasing order to identify the most confidently predicted samples, which likely correspond to the most memorized samples when $A_i \in \calA$.

\subsection{Population-level Memorization}
\label{sec:label inference accuracy}

%ORIGINAL FIGURE SETUP IN ARXIV: 
% \input{dejavu_training_epochs.tex}
% \input{dejavu_training_set_size.tex}
%PUT ORIGINAL FIGURES SIDE BY SIDE: 
% \input{dejavu_training_epochs_set_size.tex}
%PUT IN NEW FIGURES: 

\begin{wrapfigure}{r}{0.4\textwidth} 
    \centering
    \includegraphics[width=0.4\textwidth]{figures/dejavu_main.pdf}
    \caption{Accuracy of label inference using the target model (trained on $\calA$) vs. the reference model (trained on $\calB$) on the top $\%$ most confident examples $A_i \in \calA$ using only $\crop{A_i}$. For VICReg, there is a large accuracy gap between the two models, indicating a significant degree of \dejavu memorization.}
    \label{fig:dejavu main}
    \vspace{-2ex}
\end{wrapfigure}

Our first measure of \dejavu memorization is population-level label inference accuracy: \emph{What is the average label inference accuracy over a subset of SSL training images given their periphery crops?} 
To understand how much of this accuracy is due to $\SSL_A$'s \dejavu memorization, we compare with a correlation baseline using the reference model: $\KNN_B$'s label inference accuracy on images $A_i \in \calA$. 
In principle, this inference accuracy should be significantly above chance level ($1/1000$ for ImageNet) because the periphery crop may be highly indicative of the foreground object through correlation, \emph{e.g.}, if the periphery crop is a basketball player then the foreground object is likely a basketball.

Figure \ref{fig:dejavu main} compares the accuracy of $\KNN_A$ to that of $\KNN_B$ when inferring the labels of images in $A_i \in \calA$\footnote{The sets $\calA$ and $\calB$ are exchangeable, and in practice we repeat this test on images from $\calB$ using $\SSL_B$ as the target model and $\SSL_A$ as the reference model, and average the two sets of results.} using $\crop{A_i}$.
Results are shown for VICReg and the supervised model; trends for other models are shown in Appendix \ref{sec:appx simclr results}. For both VICReg and supervised models, inferring the class of $\crop{A_i}$ using $\KNN_B$ (dashed line) through correlation achieves a reasonable accuracy that is significantly above chance level. However, for VICReg, the inference accuracy using $\KNN_A$ (solid red line) is significantly higher, and the accuracy gap between $\KNN_A$ and $\KNN_B$ indicates the degree of \dejavu memorization. We highlight two observations: 
\begin{itemize}
    \item The accuracy gap of VICReg is significantly larger than that of the supervised model. This is especially notable when accounting for the fact that the supervised model is trained to associate randomly augmented crops of images with their ground truth labels. In contrast, VICReg has no label access during training but the embedding of a periphery crop can still encode the image label. 
    \item For VICReg, inference accuracy on the $1\%$ most confident examples is nearly $95\%$, which shows that our simple confidence heuristic can effectively identify the most memorized samples. This result suggests that an adversary can use this heuristic to identify vulnerable training samples to launch a more focused privacy attack.
\end{itemize}
\vspace{-.75em}
\paragraph{The \dejavu score. }
The curves of Figure \ref{fig:dejavu main} show memorization across confidence values for a single training scenario.  To study how memorization changes with different hyperparamters, we extract a single value from these curves: the \dejavu \emph{score} at confidence level $p$. In Figure \ref{fig:dejavu main}, this is the gap between the solid red (or gray) and dashed red (or gray) where confidence ($x$-axis) equal $p\%$. In other words, given the periphery crops of set $\calA$, $\KNN_A$ and $\KNN_B$ separately select and label their top $p\%$ most confident examples, and we report the difference in their accuracy. The \dejavu score captures both the degree of memorization by the accuracy gap and the \emph{ability to identify memorized examples} by the confidence level. If the score is 10\% for $p=33\%$, $\KNN_A$ has 10\% higher accuracy on its most confident third of $\calA$ than $\KNN_B$ does on its most confident third. In the following, we set $p = 20\%$, approximately the largest gap for VICReg (red lines) in Figure \ref{fig:dejavu main}. 
% Specifically, the \dejavu \emph{score} on the top $p\%$ most confident examples is,  
% \begin{equation}
%     \mathrm{DejaVu}(p) = \mathrm{Acc}_{\SSL_A}\big( \calA_{\SSL_A, p}  \big) - \mathrm{Acc}_{\SSL_B}\big( \calA_{\SSL_B, p}  \big) \ ,
%     \label{eqn:dejavu score}
% \end{equation}
% where $\calA_{\SSL_A, p}$
% Here we introduce a DejaVu memorization metric that quantify how much a target model is able to retrieve more class information from a crop than the reference model. We define it as:
% where $p$ is a function that take the $p$ purcent most confident samples.
%Figure \ref{fig:dejavu v. training epochs} shows how \dejavu memorization changes with the number of epochs used to train the embedding model (VICReg and supervised, respectively). The training set size is fixed to 300K samples, and label inference accuracy is computed on the top $20\%$ highest confidence examples. The number of epochs has a very strong influence on the degree of memorization for VICReg as the accuracy gap widens when number of epochs increases. We note that 1000 training epochs is used in several SSL works \citep{vicreg, simclr}. Remarkably, this trend in memorization is \emph{not} reflected in the standard metric for evaluating SSL representations: linear probe accuracy. The gray line in Figure \ref{fig:dejavu v. training epochs} shows the train-test accuracy gap of a linear classifier trained on top of the VICReg embeddings. Although there is a sizeable train-test gap, it does not grow significantly beyond 500 epochs. In contrast, \dejavu memorization (blue line) continues to worsen after 500 epochs. Thus, our test can be used as an alternative to linear probe accuracy to evaluate the memorization of SSL models.
% \vspace{-.75em}

% \paragraph{Comparison with the generalization gap} A network that perform very well on a training set while performing poorly on a test set (assuming the training set and test set sampled uniformly from the same distribution) is probably memorizing the training examples without being able to generalize on the test data. One could expect that measuring the difference in accuracy between the training and test set could give us insights on the degree of \dejavu memorization. However, we show in Figure  \ref{fig:dejavu v. training epochs} and \ref{fig:dejavu v. n} that this is not the case. In fact \dejavu memorization can significantly increase while the train-test gap decrease. In our experiments, we did not find a correlation between \dejavu and generalization.
\vspace{-0.75em}
\paragraph{Comparison with the linear probe train-test gap.} A standard method for measuring SSL performance is to train a linear classifier---what we call a `linear probe'---on its embeddings and compute its performance on a held out test set. From a learning theory standpoint, one might expect the linear probe's train-test accuracy gap to be indicative of memorization: the more a model overfits, the larger is the difference between train set and test set accuracy. However, as seen in Figure \ref{fig:dejavu epochs train set size}, the linear probe gap (dark blue) fails to reveal memorization captured by the \dejavu score (red) \footnote{See section \ref{sec:mitigation} for further discussion of the \dejavu score trends of Figure \ref{fig:dejavu epochs train set size}.}.

% \paragraph{Effect of training epochs.} 
% Figure \ref{fig:dejavu v. training epochs} shows how \dejavu memorization changes with training epochs for VICReg. The training set size is fixed to 300K samples. We observe that the number of epochs has a very strong influence on the degree of memorization for VICReg. From 250 to 1000 epochs, the \dejavu score (red curve) grows threefold: from under 10\% to over 30\%. Remarkably, this trend in memorization is \emph{not} reflected in the standard metric for evaluating SSL representations: linear probe accuracy. The dark blue curve shows the train-test linear probe accuracy gap. Although there is a sizeable train-test gap, it only changes by a few percent beyond 250 epochs. %Thus, our test can be used as an alternative to linear probe accuracy to evaluate the memorization of SSL models.
% \vspace{-.75em}
\begin{figure}[ht]
\label{fig:dejavu epochs and dataset}
\begin{minipage}[t]{0.49\textwidth}
\centering
     \begin{subfigure}[b]{0.48\textwidth}
         \centering
         \includegraphics[width=\textwidth]{figures/deja_vu_vs_epochs.png}
         \vspace{-1.5em}
         \caption{\dejavu vs. epochs}
         \label{fig:dejavu v. training epochs}
     \end{subfigure}
     \begin{subfigure}[b]{0.48\textwidth}
         \centering
         \includegraphics[width=\textwidth]{figures/deja_vu_vs_n.png}
         \vspace{-1.5em}
         \caption{\dejavu vs. train set size}
         \label{fig:dejavu v. n}
     \end{subfigure}~
     \vspace{-0.5em}
    \caption{
    Effect of training epochs and train set size with VICReg on \dejavu score (red) in comparison with linear probe accuracy train-test gap (dark blue). 
    \textbf{Left:} \dejavu score increases with training epochs, indicating growing memorization while the linear probe baseline decreases significantly.  
    \textbf{Right:} \dejavu score stays roughly constant with training set size suggesting that memorization may be problematic even for large datasets. %By comparison, the baseline \emph{declines} by half, spuriously suggesting less memorization. 
    %Both trends are not captured according to the linear probe train-test gap---a common method to evaluate generalization of SSL representations.}
    }
    \label{fig:dejavu epochs train set size}
\end{minipage}
\hfill
\begin{minipage}[t]{0.49\textwidth}
\centering
     \begin{subfigure}[b]{0.48\textwidth}
         \centering
         \includegraphics[width=\textwidth]{figures/vicreg_samples_epochs.pdf}
         \vspace{-1.5em}
         \caption{\dejavu vs. epochs}
         \label{fig:per sample v. training epochs}
     \end{subfigure}
     \begin{subfigure}[b]{0.48\textwidth}
         \centering
         \includegraphics[width=\textwidth]{figures/vicreg_samples_datasets.pdf}
         \vspace{-1.5em}
         \caption{\dejavu vs. train set size}
         \label{fig:per sample v. n}
     \end{subfigure}~
     \vspace{-0.5em}
    \caption{
    \definecolor{part_blue}{rgb}{0.2824, 0.4706, .8157}
	\definecolor{part_red}{rgb}{0.8392, 0.3725, 0.3725}
	\definecolor{part_orange}{rgb}{0.9333, 0.5216, 0.2902}
    Partition of samples $A_i \in \calA$ into the four categories: {\color{gray}unassociated} (not shown), {\color{part_orange}memorized}, {\color{part_red}misrepresented} and {\color{part_blue}correlated} for VICReg. The {\color{part_orange}memorized} samples---those whose labels are predicted by $\KNN_A$ but not by $\KNN_B$---occupy a significantly larger share of the training set than the {\color{part_red}misrepresented} samples---those predicted by $\KNN_B$ but not $\KNN_A$ by chance. %At 1000 epochs, $\approx 15\%$ of the training set is {\color{part_orange}memorized}. The trends across training epochs and training set sizes are consistent with those observed in Figure \ref{fig:dejavu epochs train set size}
    }
    \label{fig:partition attack main}
    \end{minipage}
\vspace{-1em} 
\end{figure}

\iffalse

\begin{minipage}[t]{0.49\textwidth}
\centering
     \begin{subfigure}[b]{0.48\textwidth}
         \centering
         \includegraphics[width=0.95\textwidth]{figures/deja_vu_vs_parameters.png}
         \vspace{-0.4em}
         \caption{\dejavu vs. capacity}
         \label{fig:dejavu v. capacity}
     \end{subfigure}
     \hfill
     \begin{subfigure}[b]{0.48\textwidth}
          \tiny
          \centering
          \setlength{\tabcolsep}{3pt}
          \begin{tabular}{|c|c|c|}
            \hline
            Criteria & DV & Acc P/B \\
            \hline
            Supervised & 8.9 & 55.3/61.1\\
            \hline
            Byol\citep{grill2020byol} & 8.0& 54.3/59.4\\
            \hline
            SimCLR\citep{chen2020simclr} & 10.0 & 44.2/54.1\\
            \hline
            Dino\citep{Dino} & 14.5 & 26.3/55.7 \\
            \hline
            Barlow T.\citep{zbontar2021barlow} & 30.5 & 33.7/54.4\\
            \hline
            VICReg\citep{vicreg} & \textbf{33.2} & 40.3/55.2\\
            \hline
          \end{tabular}
          \vspace{1.3em}
          % \caption{\dejavu (DV) vs. SSL Criterion}
          \caption{\dejavu (DV) vs. Criterion}
          \label{tab:dejavu vs. criterion}
    \end{subfigure}
    \vspace{-0.5em}
    \caption{
    Comparison of \dejavu score for different architectures and training criteria. \textbf{Left:} \dejavu score with VICReg for resnet (purple) and vision transformer (green) architectures versus number of model parameters. As expected, memorization grows with larger model capacity. This trend is more pronounced for convolutional (resnet) than transformer (ViT) architectures. \textbf{Right:} Comparison of \dejavu score and ImageNet validation accuracy (P: using projector embeddings, B: using backbone embeddings) for various SSL criteria. \textbf{Nearly all SSL models have more memorization than the supervised baseline.} 
    % Effect of training epochs and train set size on \dejavu score.
    % \textbf{Left:} \dejavu score increases with higher number of training epochs, indicating worsening memorization.
    % \textbf{Right:} \dejavu score stays roughly constant with training set size. Both trends are not captured according to the linear probe train-test gap---a common method to evaluate generalization of SSL representations.
    }
\end{minipage}
\vspace{-2em} 
\end{figure}

\begin{figure}[ht]
\begin{minipage}[t]{0.49\textwidth}
\centering
     \begin{subfigure}[b]{0.49\textwidth}
         \centering
         \includegraphics[width=\textwidth]{figures/epochs_lb_attk_epochs_acc_top1_legend.pdf}
         \caption{\dejavu vs. epochs}
         \label{fig:dejavu v. training epochs}
     \end{subfigure}
     \begin{subfigure}[b]{0.49\textwidth}
         \centering
         \includegraphics[width=\textwidth]{figures/epochs_lb_attk_datasets_acc_top1_legend.pdf}
         \caption{\dejavu vs. train set size}
         \label{fig:dejavu v. n}
     \end{subfigure}~
     \begin{subfigure}[b]{0.32\textwidth}
         \centering
         \includegraphics[width=0.8\textwidth]{figures/dejavu_vs_parameters.pdf}
         \caption{\dejavu vs. capacity}
         \label{fig:dejavu v. n}
     \end{subfigure}
    \caption{
    Effect of training epochs and train set size on \dejavu score.
    \textbf{Left:} \dejavu score increases with higher number of training epochs, indicating worsening memorization.
    \textbf{Right:} \dejavu score stays roughly constant with training set size. Both trends are not captured according to the linear probe train-test gap---a common method to evaluate generalization of SSL representations.}
    \end{minipage}
\vspace{-1em} 
\end{figure}

\begin{table}[ht]
  \footnotesize
  \centering
  \begin{tabular}{|c|c|}
    \hline
    Supervised & 8.9\\
    \hline
    SimCLR\citep{chen2020simclr} & 10.0\\
    \hline
    Byol\citep{grill2020byol} & 8.0\\
    \hline
    Dino\citep{Dino} & 14.5\\
    \hline
    Barlow T.\citep{zbontar2021barlow} & 30.5\\
    \hline
    VICReg\citep{vicreg} & \textbf{33.2}\\
    \hline
  \end{tabular}
  \caption{DejaVu Score 20\% Conf for various SSL methods.}
  \label{tab:two-row-table}
\end{table}
\vspace{-1em} 
\fi

\iffalse
\begin{figure}[ht]
\begin{minipage}[t]{.49\textwidth}
\centering
     \begin{subfigure}[b]{0.49\textwidth}
         \centering
         \includegraphics[width=\textwidth]{figures/epochs_lb_attk_epochs_acc_top1_legend.pdf}
         \caption{\dejavu vs. epochs}
         \label{fig:dejavu v. training epochs}
     \end{subfigure}
     \hfill
     \begin{subfigure}[b]{0.49\textwidth}
         \centering
         \includegraphics[width=\textwidth]{figures/epochs_lb_attk_datasets_acc_top1_legend.pdf}
         \caption{\dejavu vs. train set size}
         \label{fig:dejavu v. n}
     \end{subfigure}
\caption{
Effect of training epochs and train set size on \dejavu score.
\textbf{Left:} \dejavu score increases with higher number of training epochs, indicating worsening memorization.
\textbf{Right:} \dejavu score stays roughly constant with training set size. Both trends are not captured according to the linear probe train-test gap---a common method to evaluate generalization of SSL representations.}
\label{fig:dejavu epochs and dataset}
\end{minipage}
\hfill
\begin{minipage}[t]{.49\textwidth}
     \centering
     \begin{subfigure}[b]{0.49\textwidth}
         \centering
         \includegraphics[width=\textwidth]{figures/criteria_epochs.pdf}
         \caption{criteria comparison}
         \label{fig:dejavu v. criteria}
     \end{subfigure}
     \hfill
     \begin{subfigure}[b]{0.49\textwidth}
         \centering
         \includegraphics[width=\textwidth]{figures/architecture_epochs.pdf}
         \caption{architecture comparison}
         \label{fig:dejavu v. arch}
     \end{subfigure}
\caption{
Effect of SSL training criteria and model architectures on \dejavu score.
%the accuracy gap between target model (trained on $\calA$) and reference model (trained on $\calB$) making predictions on their 20\% most confident examples.
\textbf{Left:} \dejavu score for various training criteria.
%Barlow and VICReg have the heaviest degree of memorization, while SimCLR and BYOL have the least. 
%Note that we show detailed reconstructions of SimCLR's training data in Section \ref{sec:visualizing} despite its relatively low degree of \dejavu. 
%Regardless, Although SimCLR and BYOL have the least, we  visualize detailed reconstructions with SimCLR in section \ref{sec:mem v corr} 
All SSL models have significantly more \dejavu than the supervised baseline. \textbf{Right:} \dejavu score versus epochs for various training architectures. As expected, lower capacity architectures (Resnet18, Resnet34) reduce \dejavu but not completely. 
}
\label{fig:dejavu criteria and architecture}
\end{minipage}
\vspace{-1em} 
\end{figure}
\fi
% %\begin{figure}[ht]
%%%
%VICREG
%%%
     \centering
     \begin{subfigure}[b]{0.49\textwidth}
         \centering
         \includegraphics[width=\textwidth]{figures/sample_level_training_epochs.pdf}
         \caption{Categories of training samples vs. number of epochs}
         \label{fig:sample level epochs}
     \end{subfigure}
     \hfill
     \begin{subfigure}[b]{0.49\textwidth}
         \centering
         \includegraphics[width=\textwidth]{figures/sample_level_training_set_size.pdf}
         \caption{Categories of training samples vs. training set size}
         \label{fig:sample level training size}
     \end{subfigure}
\caption{
\definecolor{part_blue}{rgb}{0.2824, 0.4706, .8157}
\definecolor{part_red}{rgb}{0.8392, 0.3725, 0.3725}
\definecolor{part_orange}{rgb}{0.9333, 0.5216, 0.2902}
Partition of samples $A_i \in \calA$ into the four categories: {\color{gray}unassociated} (not shown), {\color{part_orange}memorized}, {\color{part_red}misrepresented} and {\color{part_blue}correlated}. The {\color{part_orange}memorized} samples---ones whose labels are predicted by $\KNN_A$ but not by $\KNN_B$---occupy a significantly larger share for VICReg compared to the supervised model, indicating that sample-level \dejavu memorization is more prevalent in VICReg. %The trends across number of training epochs and training set sizes are consistent with those observed in Figures \ref{fig:dejavu epochs and dataset} and \ref{fig:dejavu criteria and architecture}.
}
\label{fig:partition attack main appendix}
\end{figure}
% \paragraph{Effect of training set size.} 
% Figure \ref{fig:dejavu v. n} shows how \dejavu memorization responds to the model's training set size. The number of training epochs is fixed to 1000. Interestingly, training set size appears to have almost \emph{no} influence on the \dejavu score (red line), indicating that memorization is equally prevalent with a 100K dataset and a 500K dataset (which suggests that \dejavu memorization may be detectable for larger datasets). Meanwhile, the linear probe train-test accuracy gap \emph{declines} by half as the dataset size grows, failing to represent the memorization quantified by our test. 
% The trend is completely different according to linear probe accuracy (dark blue line), the train-test gap shrinks substantially when increasing the training set size from 100K to 500K. This highlights that the train-test gap is not able to capture \dejavu memorization. %Our evidence suggests that \dejavu memorization may be detectable even for large-scale training datasets. 
%\vspace{-.75em}

\vspace{-.75em} 
\subsection{Sample-level Memorization}
\label{sec:dissection}

% Section \ref{sec:label inference accuracy} shows the \emph{average} level of \dejavu memorization on a subset of the training set $\calA$. However, this average tell us only what the attacker success rate might be without explicitly describing how much of the datatset is \dejavu memorized.
The \dejavu score shows, \emph{on average}, how much better an adversary can select and classify images when using the target model trained on them. 
This average score does not tell us how many individual images have their label successfully recovered by $\KNN_A$ but not by $\KNN_B$. In other words, how many images are exposed by virtue of \emph{being in training set} $\calA$: a risk notion foundational to differential privacy. 
% However, from the perspective of an individual image $A_i \in \calA$, it is informative to know whether it was correctly classified 
To better quantify what fraction of the dataset is at risk, we perform a sample-level analysis by fixing a sample $A_i \in \calA$ and observing the label inference result of $\KNN_A$ vs. $\KNN_B$.
To this end, we partition samples $A_i \in \calA$ based on the result of label inference into four distinct categories: {\color{gray}\textbf{Unassociated}} - label inferred with neither KNN; {\color{part_orange}\textbf{Memorized}} - label inferred only with $\KNN_A$; {\color{part_red}\textbf{Misrepresented}} - label inferred only with $\KNN_B$; {\color{part_blue}\textbf{Correlated}} - label inferred with both KNNs. 
% \begin{multicols}{2}
% \begin{itemize}
%     \vspace{-.75em}
%     \setlength\itemsep{0.15em}
%     \item {\color{gray}Unassociated}: label inferred with neither KNN   
%     \item {\color{part_orange}Memorized}: label only inferred by $\KNN_A$
%     \item {\color{part_red}Misrepresented}: label only inferred with $\KNN_B$
%     \item {\color{part_blue}Correlated}: label inferred with both KNNs
%     \vspace{-.75em}
% \end{itemize}
% \end{multicols}
Intuitively, {\color{gray}unassociated} samples are ones where the embedding of $\crop{A_i}$ does not encode information about the label. {\color{part_blue}Correlated} samples are ones where the label can be inferred from $\crop{A_i}$ using correlation, \emph{e.g.}, inferring the foreground object is basketball given a crop showing a basketball player. Ideally, the {\color{part_red}misrepresented} set should be empty but contains a small portion of examples due to chance.
\emph{Déjà vu} memorization occurs for {\color{part_orange}memorized} samples where the embedding of $\SSL_B$ does not encode the label but the embedding of $\SSL_A$ does. To measure the pervasiveness of \dejavu memorization, we compare the size of the {\color{part_orange}memorized} and {\color{part_red}misrepresented} sets.
Figure \ref{fig:partition attack main} shows how the four categories of examples change with number of training epochs and training set size. The {\color{gray}unassociated} set is not shown since the total share adds up to one. The {\color{part_red}misrepresented} set remains under $5\%$ and roughly unchanged across all settings, consistent with our explanation that it is due to chance. In comparison, VICReg's {\color{part_orange}memorized} set surpasses $15\%$ at 1000 epochs. Considering that up to 5\% of these memorized examples could also be due to chance, we conclude that \textbf{at least 10\% of VICReg's training set is \dejavu memorized.} 
%is many times larger than its {\color{part_red}misrepresented} set, indicating substantial sample-level \dejavu memorization. 
%In fact, \textbf{it is 15\% of the training set that is \dejavu memorized with VICReg.}
%The trends across different number of training epochs and training set sizes match those observed in Section \ref{sec:label inference accuracy}. % On the other hand, the supervised model's {\color{part_orange}memorized} set is only marginally larger than its {\color{part_red}misrepresented} set.

% The trends across different number of training epochs and training set sizes match those observed in Section \ref{sec:label inference accuracy}: Increasing the number of epochs increases \dejavu memorization (Figure \ref{fig:per sample v. training epochs}), while increasing the training set size does not appear to reduce \dejavu memorization (Figure \ref{fig:per sample v. n}). 
%\section{Visualizing \emph{Déjà Vu} Memorization}
\label{sec:visualizing}
Beyond enabling label inference using a periphery crop, we show that \dejavu memorization allows the SSL model to encode other forms of information about a training image. Namely, we train an RCDM \citep{RCDM} on the public dataset $\calX$ and use it to visually reconstruct training images given their periphery crop.
We aim to answer the following two questions: \textbf{(1)} Can we visualize the distinction between correlation and \dejavu memorization? \textbf{(2)} What foreground object details can be extracted from the SSL model beyond class label? 
% \begin{enumerate}[noitemsep, leftmargin=*, topsep=0pt]
%     \item Can we visualize the distinction between correlation and \dejavu memorization? 
%     \item What foreground object details can be extracted from the SSL model beyond class label? 
% \end{enumerate}
\vspace{-0.5em}
\paragraph{Reconstruction pipeline.}
RCDM is a conditional generative model that is trained on the \emph{backbone embedding} of images $X_i \in \calX$ to generate an image that resembles $X_i$. All training images are first face-blurred for privacy purposes. \citet{RCDM} showed that the backbone embedding of SSL models contains more low-level information about the image, making them better suited for conditioning the RCDM.
At test time, following the pipeline in Figure \ref{fig:split_and_pipeline_cartoon}, we first use the projector embedding to find the KNN subset for the periphery crop, $\crop{A_i}$, and then average their backbone embeddings as input to the RCDM model. Ideally, when the public set contains enough representative images, the average representation of the KNN subset encodes objects present in $A_i$, and the RCDM model decodes this representation to visualize these objects.
% \begin{figure}[ht]
%%%
%VICREG
%%%
     \centering
     \begin{subfigure}[b]{0.49\textwidth}
         \centering
         \includegraphics[width=\textwidth]{figures/sample_level_training_epochs.pdf}
         \caption{Categories of training samples vs. number of epochs}
         \label{fig:sample level epochs}
     \end{subfigure}
     \hfill
     \begin{subfigure}[b]{0.49\textwidth}
         \centering
         \includegraphics[width=\textwidth]{figures/sample_level_training_set_size.pdf}
         \caption{Categories of training samples vs. training set size}
         \label{fig:sample level training size}
     \end{subfigure}
\caption{
\definecolor{part_blue}{rgb}{0.2824, 0.4706, .8157}
\definecolor{part_red}{rgb}{0.8392, 0.3725, 0.3725}
\definecolor{part_orange}{rgb}{0.9333, 0.5216, 0.2902}
Partition of samples $A_i \in \calA$ into the four categories: {\color{gray}unassociated} (not shown), {\color{part_orange}memorized}, {\color{part_red}misrepresented} and {\color{part_blue}correlated}. The {\color{part_orange}memorized} samples---ones whose labels are predicted by $\KNN_A$ but not by $\KNN_B$---occupy a significantly larger share for VICReg compared to the supervised model, indicating that sample-level \dejavu memorization is more prevalent in VICReg. %The trends across number of training epochs and training set sizes are consistent with those observed in Figures \ref{fig:dejavu epochs and dataset} and \ref{fig:dejavu criteria and architecture}.
}
\label{fig:partition attack main appendix}
\end{figure}
%\begin{figure*}[t!]
%%%
%DAM
%%%
     \centering
     \begin{subfigure}[b]{0.49\textwidth}
         \centering
         \includegraphics[width=\textwidth]{figures/dam_corr.png}
         \caption{A {\color{part_blue}correlated} dam example}
         \label{fig:dam correlated}
     \end{subfigure}
     \hfill
     \begin{subfigure}[b]{0.49\textwidth}
         \centering
         \includegraphics[width=\textwidth]{figures/dam_mem.png}
         \caption{A {\color{part_orange}memorized} dam example}
         \label{fig:dam memorized}
     \end{subfigure}
\caption{
{\color{part_blue}Correlated} and {\color{part_orange}Memorized} examples from the \emph{dam} class. Both $\SSL_A$ and $\SSL_B$ are SimCLR models.
\textbf{Left:} The periphery crop (pink square) contains a concrete structure that is often present in images of dams. Consequently, the trained RCDM can reconstruct the foreground object using representations from both $\SSL_A$ and $\SSL_B$ through this correlation.
\textbf{Right:} The periphery crop only contains a patch of water. The embedding produced by $\SSL_B$ only contains enough information to infer that the foreground object is related to water, as reflected by its KNN set and RCDM reconstruction. In contrast, the embedding produced by $\SSL_A$ memorizes the association of this patch of water with dam and the RCDM can visualize the embedding to produce images of dams.
}
\vspace{-1ex}
\label{fig:mem v corr dam}
\end{figure*}


\begin{figure*}[t!]
%%%
%DAM
%%%
     \centering
     \begin{subfigure}[b]{0.49\textwidth}
         \centering
         \includegraphics[width=\textwidth]{figures/dam_corr.png}
         \caption{A {\color{part_blue}correlated} dam example}
         \label{fig:dam correlated}
     \end{subfigure}
     \hfill
     \begin{subfigure}[b]{0.49\textwidth}
         \centering
         \includegraphics[width=\textwidth]{figures/dam_mem.png}
         \caption{A {\color{part_orange}memorized} dam example}
         \label{fig:dam memorized}
     \end{subfigure}
\caption[Correlated and Memorized examples from the \emph{dam} class.]{
Correlated and Memorized examples from the \emph{dam} class. Both $\SSL_A$ and $\SSL_B$ are SimCLR models.
\textbf{Left:} The periphery crop (pink square) contains a concrete structure that is often present in images of dams. Consequently, the trained RCDM can reconstruct the foreground object using representations from both $\SSL_A$ and $\SSL_B$ through this correlation.
\textbf{Right:} The periphery crop only contains a patch of water. The embedding produced by $\SSL_B$ only contains enough information to infer that the foreground object is related to water, as reflected by its KNN set and RCDM reconstruction. In contrast, the embedding produced by $\SSL_A$ memorizes the association of this patch of water with dam and the RCDM can visualize the embedding to produce images of dams.
}
\label{fig:mem v corr dam}
\end{figure*}


\begin{figure}[t!]
%%%
%BADGER
%%%
     \centering
     \begin{subfigure}[b]{0.49\textwidth}
         \centering
         \includegraphics[width=\textwidth]{figures/euro_badgers.png}
         \caption{{\color{part_orange}Memorized} European badgers}
         \label{fig:euro badgers}
     \end{subfigure}
     \hfill
     \begin{subfigure}[b]{0.49\textwidth}
         \centering
         \includegraphics[width=\textwidth]{figures/amer_badgers.png}
         \caption{{\color{part_orange}Memorized} American badgers}
         \label{fig:amer badgers}
     \end{subfigure}
\caption[Visualization of \dejavu memorization beyond class label.]{
Visualization of \dejavu memorization beyond class label. Both $\SSL_A$ and $\SSL_B$ are VICReg models. 
The four images shown belong to the memorized set of $\SSL_A$ from the \emph{badger} class. RCDM reconstruction using embeddings from $\SSL_A$ can reveal not only the correct class label, but also the specific badger species: \emph{European} (left) and \emph{American} (right). Such information does not appear to be memorized by the reference model $\SSL_B$.
} 
\label{fig:in class badger}
\end{figure}


% \subsection{Visualizing Correlation vs. Memorization}
\label{sec:mem v corr}
\vspace{-0.5em} 
\paragraph{Visualizing Correlation vs. Memorization.}
Figure \ref{fig:mem v corr dam} shows examples of dams from the {\color{part_blue}correlated} set (left) and the {\color{part_orange}memorized} set (right) as defined in Section \ref{sec:dissection}, along with the associated KNN set and RCDM reconstruction. Both $\SSL_A$ and $\SSL_B$ are SimCLR models. In Figure \ref{fig:dam correlated}, the periphery crop is represented by the pink square, which contains concrete structure attached to the dam's main structure. As a result, both $\SSL_A$ and $\SSL_B$ produce embeddings of $\crop{A_i}$ whose KNN set in $\calX$ consist of dams, \emph{i.e.}, there is a correlation between the concrete structure in $\crop{A_i}$ and the foreground dam. The RCDM reconstructions also consist of dams or structures that closely resemble dams. 
In Figure \ref{fig:dam memorized}, the periphery crop only contains a patch of water, which does not strongly correlate with dams in the ImageNet distribution. Evidently, the reference model $\SSL_B$ embeds $\crop{A_i}$ close to that of other objects commonly found in water, such as sea turtle and submarine. In contrast, the KNN set according to $\SSL_A$ all contain dams despite the vast number of alternative possibilities within the ImageNet classes, and the RCDM reconstruction outputs dams as well which highlight memorization in $\SSL_A$ between this specific patch of water and the dam. %\footnote{See Appendix \ref{sec:appx visualization} to see the same trend in the \emph{yellow garden spider} class.}


% \subsection{Visualizing Memorization Beyond Class Label}
% \label{sec:in class variation}
\vspace{-0.5em} 
\paragraph{Visualizing Memorization Beyond Class Label.}
We now use our reconstruction algorithm to show that \dejavu memorization can be exploited to reveal detailed information beyond class label. Figure \ref{fig:in class badger} shows four examples of badgers from the {\color{part_orange}memorized} set. In all four images, the periphery crop (pink square) does not contain any indication that the foreground object is a badger. Despite this, the KNN set and the RCDM reconstruction using $\SSL_A$ consistently produce images of badgers, while the same does not hold for $\SSL_B$.
More interestingly, reconstructions using $\SSL_A$ in Figure \ref{fig:euro badgers} all contain \emph{European} badgers, while reconstructions in Figure \ref{fig:amer badgers} all contain \emph{American} badgers, accurately reflecting the species of badger present in the respective training images. Since ImageNet-1K does \emph{not} differentiate between these two species of badgers, our reconstructions show that SSL models can memorize information that is highly specific to a training sample beyond its class label\footnote{See Appendix \ref{sec:appx visualization} for additional visualization experiments.}.%\footnote{See Appendix \ref{sec:appx visualization} for the same trend in the \emph{aircraft carrier} class.}.





%\vspace{-.5em} 
\section{Mitigation of \dejavu memorization}
\label{sec:mitigation}
% We do not have an understanding on why \dejavu occur so strongly in some SSL pretraining, however we present additional experiments that shed light on which parameters have the biggest impact on \dejavu memorization.
\begin{figure}[ht]
\label{fig:mitigations}
\begin{minipage}[t]{0.5\textwidth}
\centering
     \begin{subfigure}[b]{0.47\textwidth}
         \centering
         \includegraphics[width=\textwidth]{figures/dejavu_vicreg_param.png}
         \vspace{-1.5em}
         \caption{Loss hyper-parameter}
         \label{fig:dejavu v. invariance}
     \end{subfigure}
     \begin{subfigure}[b]{0.49\textwidth}
         \centering
         \includegraphics[width=\textwidth]{figures/deja_vu_vs_layer.png}
         \vspace{-1.5em}
         \caption{Guillotine regularization}
         \label{fig:dejavu v. guillotine}
     \end{subfigure}~
     \vspace{-0.5em}
    \caption[Effect of two kinds of hyper-parameters on VICReg memorization. ]{
    Effect of two kinds of hyper-parameters on VICReg memorization. \textbf{Left:} \dejavu score (red) versus the \emph{invariance} loss parameter, $\lambda$, used in the VICReg criterion (100k dataset). Larger $\lambda$ significantly reduces \dejavu, with minimal effect on linear probe validation performance (green). $\lambda = 25$ (near maximum \dejavu) is recommended in the original paper \textbf{Right:} \dejavu score versus projector layer---guillotine regularization \cite{Guillotine}---from projector to backbone. Removing the projector can significantly reduce \dejavu. Appendix \ref{sec:guillotine} shows that the backbone still can memorize, however; we demonstrate reconstructions using the SimCLR backbone.
    }
\end{minipage}
\hfill
\begin{minipage}[t]{0.48\textwidth}
\centering
     \begin{subfigure}[b]{0.46\textwidth}
         \centering
         \includegraphics[width=\textwidth]{figures/deja_vu_vs_parameters.png}
         \vspace{-1.3em}
         \caption{\dejavu vs. capacity}
         \label{fig:dejavu v. capacity}
     \end{subfigure}
     \hfill
     \begin{subfigure}[b]{0.52\textwidth}
          \tiny
          \centering
          \setlength{\tabcolsep}{3pt}
          \begin{tabular}{|c|c|c|}
            \hline
            Criteria & DV & Acc P/B \\
            \hline
            Supervised & 8.9 & 55.3/61.1\\
            \hline
            Byol\citep{grill2020byol} & 8.0& 54.3/59.4\\
            \hline
            SimCLR\citep{chen2020simclr} & 10.0 & 44.2/54.1\\
            \hline
            Dino\citep{Dino} & 14.5 & 26.3/55.7 \\
            \hline
            Barlow T.\citep{zbontar2021barlow} & 30.5 & 33.7/54.4\\
            \hline
            VICReg\citep{vicreg} & \textbf{33.2} & 40.3/55.2\\
            \hline
          \end{tabular}
          \vspace{1.3em}
          % \caption{\dejavu (DV) vs. SSL Criterion}
          \caption{\dejavu (DV) vs. Criterion}
          \label{tab:dejavu vs. criterion}
    \end{subfigure}
    \vspace{-1.4em}
    \caption[Effect of model architecture and criterion on \dejavu memorization.]{
    %Comparison of \dejavu score for different architectures and training criteria. 
    Effect of model architecture and criterion on \dejavu memorization. 
    \textbf{Left:} \dejavu score with VICReg for resnet (purple) and vision transformer (green) architectures versus number of model parameters. As expected, memorization grows with larger model capacity. This trend is more pronounced for convolutional (resnet) than transformer (ViT) architectures. \textbf{Right:} Comparison of \dejavu score 20\% conf. and ImageNet linear probe validation accuracy (P: using projector embeddings, B: using backbone embeddings) for various SSL criteria. %\textbf{Nearly all SSL models have more memorization than the supervised baseline.} 
    % Effect of training epochs and train set size on \dejavu score.
    % \textbf{Left:} \dejavu score increases with higher number of training epochs, indicating worsening memorization.
    % \textbf{Right:} \dejavu score stays roughly constant with training set size. Both trends are not captured according to the linear probe train-test gap---a common method to evaluate generalization of SSL representations.
    }
    \end{minipage}
\end{figure}
We cannot yet make claims on why \dejavu occurs so strongly for some SSL training settings and not for others. To gain some intuition for future work, we present additional observations that shed light on which parameters have the most salient impact on \dejavu memorization.
\vspace{-.75em}
\paragraph{Déjà vu memorization worsens by increasing number of training epochs.} 
Figure \ref{fig:dejavu v. training epochs} shows how \dejavu memorization changes with number of training epochs for VICReg. The training set size is fixed to 300K samples. From 250 to 1000 epochs, the \dejavu score (red curve) grows \emph{threefold}: from under 10\% to over 30\%. Remarkably, this trend in memorization is \emph{not} reflected by the linear probe gap (dark blue), which only changes by a few percent beyond 250 epochs. 

%\vspace{-.75em}
\paragraph{Training set size has minimal effect on \dejavu memorization.} Figure \ref{fig:dejavu v. n} shows how \dejavu memorization responds to the model's training set size. The number of training epochs is fixed to 1000. Interestingly, training set size appears to have almost \emph{no} influence on the \dejavu score (red line), indicating that memorization is equally prevalent with a 100K dataset and a 500K dataset. This result suggests that \dejavu memorization may be detectable even for large datasets. Meanwhile, the standard linear probe train-test accuracy gap \emph{declines} by more than half as the dataset size grows, failing to represent the memorization quantified by our test. 
% The trend is completely different according to linear probe accuracy (dark blue line), the train-test gap shrinks substantially when increasing the training set size from 100K to 500K. This highlights that the train-test gap is not able to capture \dejavu memorization. Our evidence suggests that \dejavu memorization may be detectable even for large-scale training datasets. 
\vspace{-0.5em}
\paragraph{Training loss hyper-parameter has a strong effect.} 
%We show in Figure \ref{fig:dejavu v. training epochs} that the number of training epochs is an important factor that can increase significantly \dejavu memorization. In contrast, the dataset size does not impact much \dejavu as shown in Figure \ref{fig:dejavu epochs train set size}. 
Loss hyper-parameters, like VICReg's invariance coefficient (Figure \ref{fig:dejavu v. invariance}) or SimCLR's temperature parameter (Appendix Figure \ref{fig:simclr temperature}) significantly impact \dejavu with minimal impact on the linear probe validation accuracy.

\vspace{-0.5em}
\paragraph{Some SSL criteria promote stronger \dejavu memorization.} Table \ref{tab:dejavu vs. criterion} demonstrates that the degree of memorization varies widely for different training criteria. VICReg and Barlow Twins have the highest \dejavu scores while SimCLR and Byol have the lowest.
%\footnote{We show detailed reconstructions of SimCLR's training data in Section \ref{sec:visualizing} despite its relatively low degree of \dejavu.}.
With the exception of Byol, all SSL models have more \dejavu memorization than the supervised model. Interestingly, different criteria can lead to similar linear probe validation accuracy and very different degrees of \dejavu as seen with SimCLR and Barlow Twins. Note that low degrees of \dejavu can still risk training image reconstruction, as exemplified by the SimCLR reconstructions in Figures \ref{fig:mem v corr dam} and \ref{fig:mem v corr spider}. 
%\vspace{-1em}
\vspace{-0.5em}
\paragraph{Larger models have increased \dejavu memorization.} Figure \ref{fig:dejavu v. capacity} validates the common intuition that lower capacity architectures (Resnet18/34) result in less memorization than their high capacity counterparts (Resnet50/101). 
% \begin{wrapfigure}{r}{0.25\textwidth} 
%     \centering
%     \includegraphics[width=0.25\textwidth]{figures/attk_layer_acc_top1_legend.pdf}
%     \caption{\dejavu memorization versus layer from backbone (0) to projector output (3).}
%     \label{fig:dejavu vs layer}
%     \vspace{-8ex}
% \end{wrapfigure}
We see the same trend for vision transformers as well. %This comes with a tradeoff, since reduced model capacity can result in a nontrivial degradation of representation quality\cite{vicreg, simclr}.  
\vspace{-0.5em}
\paragraph{Guillotine regularization can help reduce \dejavu memorization.} Previous experiments were done using the projector embedding. In Figure \ref{fig:dejavu v. guillotine}, we present how Guillotine regularization\citep{Guillotine} (removing final layers in a trained SSL model) impacts \dejavu with VICReg\footnote{Further experiments are available in Appendix \ref{sec:guillotine}.}. Using the backbone embedding instead of the projector embedding seems to be the most straightforward way to mitigate \dejavu memorization. However, as demonstrated in Appendix \ref{sec:appx backbone results}, backbone representation with low \dejavu score can still be leveraged to reconstruct some of the training images.

\section{Conclusion}
\label{sec:conclusion}

We defined and analyzed \dejavu memorization, a notion of unintended memorization of partial information in image data. As shown in Sections \ref{sec:quant} and \ref{sec:visualizing}, SSL models can largely exhibit \dejavu memorization on their training data, and this memorization signal can be extracted to infer or visualize image-specific information.
Since SSL models are becoming increasingly widespread as foundation models for image data, negative consequences of \dejavu memorization can have profound downstream impact and thus deserves further attention. 
Future work should focus on understanding how \dejavu emerges in the training of SSL models and why methods like Byol are much more robust to \dejavu than VICReg and Barlow Twins. In addition, trying to characterize which data points are the most at risk of \dejavu could be crucial to get a better understanding on this phenomenon. 

\graphicspath{{./chapters/chapter1/}}
%\newtheorem{thm}{Theorem}
%\newtheorem{lem}[thm]{Lemma}
\def\supp{supp}
%\newtheorem{cor}[thm]{Corollary}
%\newtheorem{defn}[thm]{Definition}
\def\D{{\mathcal D}}
\def\U{{\mathcal U}}
\def\V{{\mathcal V}}
\def\X{\mathcal X}
\def\R{\mathbb R}
\def\Y{\{\pm 1\}}
\def\d{\rho}
\def\E{\mathbb{E}}
\def\N{\mathbb{N}}
\def\g{g}
\def\A{\mathcal{A}}
\def\nat{g_{neighbor}}
\def\bad{\D_{1/2}^{-}}
\def\natural{neighborhood preserving}
\def\Natural{Neighborhood preserving}
\def\ncons{neighborhood}
\def\Ncons{Neighborhood}

\def\calD{\mathcal{D}}
\def\calU{\mathcal{U}}
\def\calV{\mathcal{V}}

\chapter{Consistent Non-Parametric Methods for Maximizing Robustness} 

\section{Introduction}
Adversarially robust classification, that has been of much recent interest, is typically formulated as follows. We are given data drawn from an underlying distribution $D$, a metric $d$, as well as a pre-specified robustness radius $r$. We say that a classifier $c$ is $r$-robust at an input $x$ if it predicts the same label on a ball of radius $r$ around $x$. Our goal in robust classification is to find a classifier $c$ that maximizes astuteness, which is defined as accuracy on those examples where $c$ is also $r$-robust. 

While this formulation has inspired a great deal of recent work, both theoretical and empirical \cite{Carlini17, Liu17, Papernot17, Papernot16,Szegedy14, Hein17,Schmidt18,Wu16,Steinhardt18, Sinha18, Yang20}, a major limitation is that enforcing a pre-specified robustness radius $r$ may lead to sub-optimal accuracy {\em{and}} robustness. To see this, consider what would be an ideally robust classifier the example in Figure~\ref{fig:intro}. For simplicity, suppose that we know the data distribution. In this case, a classifier that has an uniformly large robustness radius $r$ will misclassify some points from the blue cluster on the left, leading to lower accuracy. This is illustrated in panel (a), in which large robustness radius leads to intersecting robustness regions. On the other hand, in panel (b), the blue cluster on the right is highly separated from the red cluster, and could be accurately classified with a high margin. But this will not happen if the robustness radius is set small enough to avoid the problems posed in panel (a). Thus, enforcing a fixed robustness radius that applies to the entire dataset may lead to lower accuracy and lower robustness.

In this work, we propose an alternative formulation of robust classification that ensures that in the large sample limit, there is no robustness-accuracy trade off, and that regions of space with higher separation are classified more robustly. An extra advantage is that our formulation is achievable by existing methods. In particular, we show that two very common non-parametric algorithms -- nearest neighbors and kernel classifiers -- achieve these properties in the large sample limit.


\begin{figure}
\begin{subfigure}{0.45\textwidth}
\includegraphics[width=\linewidth]{big_radius_robustness.png}
\caption{Large robustness radii} 
\end{subfigure}\hspace*{\fill}
\begin{subfigure}{0.45\textwidth}
\includegraphics[width=\linewidth]{small_radius_robustness.png}
\caption{Small robustness radii} 
\end{subfigure}

\caption{A data distribution demonstrating the difficulties with fixed radius balls for robustness regions. The red represents negatively labeled points, and the blue positive. If the robustness radius is set too large (panel (a)), then the regions of A and B intersect leading to a loss of accuracy. If the radius is set too small (panel (b)), this leads to a loss of robustness at point C where in principle it should be possible to defend against a larger amount of adversarial attacks.}\label{fig:intro}
\end{figure}

%\begin{figure}[ht]
%	
%	\subfloat[Large robustness radii]{\includegraphics[width=.45\textwidth]{big_radius_robustness.png}}\hfill
%	\subfloat[Small robustness radii]{\includegraphics[width=.45\textwidth]{small_radius_robustness.png}}
%	\caption{A data distribution demonstrating the difficulties with fixed radius balls for robustness regions. The red represents negatively labeled points, and the blue positive. If the robustness radius is set too large (panel (a)), then the regions of A and B intersect leading to a loss of accuracy. If the radius is set too small (panel (b)), this leads to a loss of robustness at point C where in principle it should be possible to defend against a larger amount of adversarial attacks.}
%	
%	\label{fig:intro}
%\end{figure}

Our formulation is built on the notion of a new large-sample limit. In the standard statistical learning framework, the large-sample ideal is the Bayes optimal classifier that maximizes accuracy on the data distribution, and is undefined outside. Since this is not always robust with radius $r$, prior work introduces the notion of an $r$-optimal classifier~\cite{YRWC19} that maximizes accuracy on points where it is also $r$-robust. However, this classifier also suffers from the same challenges as the example in Figure~\ref{fig:intro}. 

We depart from both by introducing a new limit that we call the \natural\emph{ }Bayes optimal classifier, described as follows. Given an input $x$ that lies in the support of the data distribution $D$, it predicts the same label as the Bayes optimal. On an $x$ outside the support, it outputs the prediction of the Bayes Optimal on the nearest neighbor of $x$ {\em{within}} the support of $D$. The first property ensures that there is no loss of accuracy -- since it always agrees with the Bayes Optimal within the data distribution. The second ensures higher robustness in regions that are better separated. Our goal is now to design classifiers that converge to the \natural\emph{ }Bayes optimal in the large sample limit; this ensures that with enough data, the classifier will have accuracy approaching that of the Bayes optimal, as well as higher robustness where possible without sacrificing accuracy. 

We next investigate how to design classifiers with this convergence property. Our starting point is classical statistical theory~\cite{Stone77} that shows that a class of methods known as weight functions will converge to a Bayes optimal in the large sample limit provided certain conditions hold; these include $k$-nearest neighbors under certain conditions on $k$ and $n$, certain kinds of decision trees as well as kernel classifiers. Through an analysis of weight functions, we next establish precise conditions under which they converge to the \natural\emph{ }Bayes optimal in the large sample limit. As expected, these are stronger than standard convergence to the Bayes optimal. In the large sample limit, we show that $k_n$-nearest neighbors converge to the \natural\emph{ }Bayes optimal provided $k_n = \omega(\log n)$, and kernel classifiers converge to the \natural\emph{ }Bayes optimal provided certain technical conditions (such as the bandwidth shrinking sufficiently slowly). By contrast, certain types of histograms do not converge to the \natural\emph{ }Bayes optimal, even if they do converge to the Bayes optimal.  We round these off with a lower bound that shows that for nearest neighbor, the condition that $k_n = \omega(\log n)$ is tight. In particular, for $k_n = O(\log n)$, there exist distributions for which $k_n$-nearest neighbors provably fails to converge towards the \natural\emph{ }Bayes optimal (despite converging towards the standard Bayes optimal). 


%For $r$-robustness, the story is a little different, and in the general case -- that is, when the classes are close together or overlapping in space -- none of these classifiers may converge to the $r$-optimal. Instead\cite{Bhattacharjee20} shows that the Adversarial Pruning procedure of~\cite{YRWC19} that removes a subset of training examples to make data well-separated, followed by $k$-nearest neighbors or kernel classifiers, does converge to the $r$-optimal in the large sample limit.


In summary, the contributions of the paper are as follows. First, we propose a new large sample limit the \natural\emph{ }Bayes optimal and a new formulation for robust classification. We then establish conditions under which weight functions, a class of non-parametric methods, converge to the \natural\emph{ }Bayes optimal in the large sample limit. Using these conditions, we show that $k_n$-nearest neighbors satisfy these conditions when $k_n = \omega(\log n)$, and kernel classifiers satisfy these conditions provided the kernel function $K$ has faster than polynomial decay, and the bandwidth parameter $h_n$ decreases sufficiently slowly. 

To complement these results, we also include negative examples of non-parametric classifiers that do not converge. We provide an example where histograms do not converge to the \natural\emph{ }Bayes optimal with increasing $n$. We also show a lower bound for nearest neighbors, indicating that $k_n = \omega(\log n)$ is both necessary and sufficient for convergence towards the \natural\emph{ }Bayes optimal. 

Our results indicate that the \natural\emph{ }Bayes optimal formulation shows promise and has some interesting theoretical properties. We leave open the question of coming up with other alternative formulations that can better balance both robustness and accuracy for all kinds of data distributions, as well as are achievable algorithmically. We believe that addressing this would greatly help address the challenges in adversarial robustness.

\section{Preliminaries}


We consider binary classification over $\R^d \times \Y$, and let $\d$ denote any distance metric on $\R^d$. We let $\mu$ denote the measure over $\R^d$ corresponding to the probability distribution over which instances $x \in \R^d$ are drawn. Each instance $x$ is then labeled as $+1$ with probability $\eta(x)$  and $-1$ with probability $1 - \eta(x)$. Together, $\mu$ and $\eta$ comprise our data distribution $\D = (\mu, \eta)$ over $\R^d \times \Y$. 


For comparison to the robust case, for a classifier $f: \R^d \to \{\pm 1\}$ and a distribution $\D$ over $\R^d \times \{\pm 1\}$, it will be instructive to consider its  \textbf{accuracy},  denoted $A(f, \D)$, which is defined as the fraction of examples from $\D$ that $f$ labels correctly. Accuracy is maximized by the \textbf{Bayes Optimal classifier}: which we denote by $\g$. It can be shown that for any $x \in \supp(\mu)$, $\g(x) = 1$ if $\eta(x) \geq \frac{1}{2}$, and $\g(x) = -1$ otherwise. 

Our goal is to build classifiers $\R^d \to \Y$ that are both accurate and robust to small perturbations. For any example $x$, perturbations to it are constrained to taking place in the \textbf{robustness region} of $x$, denoted $U_x$. We will let $\U = \{U_x: x \in \R^d\}$ denote the collections of all robustness regions. 

We say that a classifier $f: \R^d \to \{\pm 1\}$ is \textbf{robust} at $x$ if for all $x' \in U_x$, $f(x') = f(x)$. Combining robustness and accuracy, we say that classifier is \textbf{astute} at a point $x$ if it is both accurate and robust. Formally, we have the following definition. 

\begin{defn}
A classifier $f: \R^d \to \Y$ is said to be \textbf{astute} at $(x,y)$ with respect to robustness collection $\U$ if $f(x) = y$ and $f$ is robust at $x$ with respect to $\U$. If $\D$ is a data distribution over $\R^d \times \Y$, the \textbf{astuteness} of $f$ over $\D$ with respect to $\U$, denoted $A_\U(f, \D)$, is the fraction of examples $(x,y) \sim \D$ for which $f$ is astute at $(x,y)$ with respect to $\U$. Thus $$A_\U(f, \D) = P_{(x, y) \sim \D}[f(x') = y, \forall x' \in \U_x].$$
\end{defn}




\paragraph{Non-parametric Classifiers} We now briefly review several kinds of non-parametric classifiers that we will consider throughout this paper. We begin with \textit{weight functions}, which are a general class of non-parametric algorithms that encompass many classic algorithms, including nearest neighbors and kernel classifiers.

\textbf{Weight functions} are built from training sets, $S = \{(x_1, y_1), (x_2, y_2,), \dots, (x_n, y_n)\}$ by assigning a function $w_i^S: \R^d \to [0, 1]$ that essentially scores how relevant the training point $(x_i, y_i)$ is to the example being classified. The functions $w_i^S$ are allowed to depend on $x_1, \dots, x_n$ but must be independent of the labels $y_1, \dots, y_n$. Given these functions, a point $x$ is classified by just checking whether $\sum y_iw_i^S(x) \geq 0$ or not. If it is nonnegative, we output $+1$ and otherwise $-1$. A complete description of weight functions is included in the appendix. 

Next, we enumerate several common Non-parametric classifiers that can be construed as weight functions. Details can be found in the appendix.

\textbf{Histogram classifiers} partition the domain $\R^d$ into cells recursively by splitting cells that contain a sufficiently large number of points $x_i$. This corresponds to a weight function in which $w_i^S(x) = \frac{1}{k_x}$ if $x_i$ is in the same cell as $x$, where $k_x$ denotes the number of points in the cell containing $x$.

$k_n$-\textbf{nearest neighbors} corresponds to a weight function in which $w_i^S(x) = \frac{1}{k_n}$ if $x_i$ is one of the $k_n$ nearest neighbors of $x$, and $w_i^S(x) = 0$ otherwise.

\textbf{Kernel-Similarity classifiers} are weight functions built from a kernel function $K:\R_{\geq 0} \to \R_{\geq 0}$ and a window size $(h_n)_1^\infty$ such that $w_i^S(x) \propto  K(\d(x, x_i)/h_n)$ (we normalize by dividing by $\sum_1^n K((\d(x, x_i)/h_n))$).
\section{The \Natural\emph{ }Bayes optimal classifier}

Robust classification is typically studied by setting the robustness regions,  $\mathcal{U} = \{U_x\}_{x \in \R^d}$, to be balls of radius $r$ centered at $x$, $U_x = \{x': \d(x, x') \leq r\}$. The quantity $r$ is the robustness radius, and is typically set by the practitioner (before any training has occurred). 

This method has a limitation with regards to trade-offs between accuracy and robustness. To increase the margin or robustness, we must have a large robustness radius (thus allowing us to defend from larger adversarial attacks). However, with large robustness radii, this can come at a cost of accuracy, as it is not possible to robustly give different labels to points with intersecting robustness regions. 

For an illustration, consider Figure \ref{fig:intro}. Here we consider a data distribution $D = (\mu, \eta)$ in which the blue regions denote all points with $\eta(x) > 0.5$ (and thus should be labeled $+$), and the red regions denote all points with $\eta(x) < 0.5$ (and thus should be labeled $-$). Observe that it is not possible to be simultaneously accurate and robust at points $A, B$ while enforcing a large robustness radius, as demonstrated by the intersecting balls. While this can be resolved by using a smaller radius, this results in losing out on potential robustness at point $C$. In principal, we should be able to afford a large margin of robustness about $C$ due to its relatively far distance from the red regions. 

Motivated by this issue, we seek to find a formalism for robustness that allows us to simultaneously avoid paying for any accuracy-robustness trade-offs and \textit{adaptively} size robustness regions (thus allowing us to defend against a larger range of adversarial attacks at points that are located in more homogenous zones of the distribution support). To approach this, we will first provide an ideal limit object: a classifier that has the same accuracy as the Bayes optimal (thus meeting our first criteria) that has good robustness properties. We call this the the \natural\emph{ }Bayes optimal classifier, defined as follows.

\begin{defn}\label{defn:includes_mu_plus_and_minus}
Let $\D = (\mu, \eta)$ be a distribution over $\R^d \times \{\pm 1\}$. Then the \textbf{\natural\emph{ }Bayes optimal classifier of $\D$}, denoted $\nat$, is the classifier defined as follows. Let $\mu^+ = \{x: \eta(x) \geq \frac{1}{2}\}$ and $\mu^- = \{x: \eta(x) < \frac{1}{2}\}$. Then for any $x \in \R^d$, $\nat(x) = +1$ if $\d(x, \mu^+) \leq \d(x, \mu^-)$, and $\nat(x) = -1$ otherwise. 
\end{defn}

This classifier can be thought of as the most robust classifier that matches the accuracy of the Bayes optimal. We call it \textit{\natural} because it extends the Bayes optimal classifier into a local neighborhood about every point in the support. For an illustration, refer to Figure \ref{fig:decision_boundary2}, which plots the decision boundary of the \natural\emph{ }Bayes optimal for an example distribution. 

\begin{figure}
    \centering
        \includegraphics[scale=0.30] {decision_boundary.png}
    \caption{The decision boundary of the \natural\emph{ }Bayes optimal classifier is shown in green,  and the \natural\emph{ }robust region of $x$ is shown in pink. The former consists of points equidistant from $\mu^+, \mu^-$, and the latter consists of points equidistant from $x$, $\mu^+$.}
    \label{fig:decision_boundary2}
\end{figure}

Next, we turn our attention towards measuring its robustness, which must be done with respect to some set of robustness regions $\mathcal{U} = \{U_x\}$. While these regions $U_x$ can be nearly arbitrary, we seek regions $U_x$ such that $A_\U(g_{max}, \D) = A(g_{bayes}, \D)$ (our astuteness equals the maximum possible accuracy) and $U_x$ are ``as large as possible" (representing large robustness). To this end, we propose the following regions.

\begin{defn}\label{def:nat_region}
Let $\D = (\mu, \eta)$ be a data distribution over $\R^d \times \{\pm 1\}$. Let $\mu^+ = \{x: \eta(x) > \frac{1}{2}\}$, $\mu^- = \{x: \eta(x) < \frac{1}{2}\}$, and $\mu^{1/2} = \{x: \eta(x) = \frac{1}{2}\}$. For $x \in \mu^+$, we define the \textbf{\natural\emph{ }robustness region}, denoted $V_x$, as $$V_x = \{x': \rho(x, x') < \rho(\mu^- \cup \mu^{\frac{1}{2}}, x')\}.$$ It consists of all points that are closer to $x$ than they are to $\mu^- \cup \mu^{1/2}$ (points oppositely labeled from $x$). We can use a similar definition for $x \in \mu^{-}$. Finally, if $x \in \mu^{1/2}$, we simply set $V_x = \{x\}$. 
\end{defn}

These robustness regions take advantage of the structure of the \natural\emph{ }Bayes optimal. They can essentially be thought of as regions that maximally extend from any point $x$ in the support of $\D$ to the decision boundary of the \natural\emph{ }Bayes optimal. We include an illustration of the regions $V_x$ for an example distribution in Figure \ref{fig:decision_boundary2}. 

As a technical note, for $x \in supp(\D)$ with $\eta(x) = 0.5$, we give them a trivial robustness region. The rational for doing this is that $\eta(x) = 0.5$ is an edge case that is arbitrary to classify, and consequently enforcing a robustness region at that point is arbitrary and difficult to enforce. 

We now formalize the robustness and accuracy guarantees of the max-margin Bayes optimal classifier with the following two results.

\begin{thm}\label{thm:accuracy_margin}
(Accuracy) Let $\D$ be a data distribution. Let $\V$ denote the collection of \natural\emph{ }robustness regions, and let $g$ denote the Bayes optimal classifier. Then the \natural\emph{ }Bayes optimal classifier, $\nat$, satisfies $A_\V(\nat, \D) = A(g, \D)$, where $A(g, \D)$ denotes the accuracy of the Bayes optimal. Thus, $\nat$ maximizes accuracy.
\end{thm}

\begin{thm}\label{thm:robust_margin}
(Robustness) Let $\D$ be a data distribution, let $f$ be a classifier, and let $\U$ be a set of robustness regions. Suppose that $A_\U(f, \D) = A(g, \D)$, where $g$ denotes the Bayes optimal classifier. Then there exists $x \in \supp(\D)$ such that $V_x \not \subset U_x$, where $V_x$ denotes the \natural\emph{ }robustness region about $x$. In particular, we cannot have $V_x$ be a strict subset of $U_x$ for all $x$. 
\end{thm}

Theorem \ref{thm:accuracy_margin} shows that the \natural\emph{ }Bayes classifier achieves maximal accuracy, while Theorem \ref{thm:robust_margin} shows that achieving a strictly higher robustness (while maintaining accuracy) is not possible; while it is possible to make accurate classifiers which have higher robustness than $\nat$ in some regions of space, it is not possible for this to hold across all regions. Thus, the \natural\emph{ }Bayes optimal classifier can be thought of as a local maximum to the constrained optimization problem of maximizing robustness subject to having maximum (equal to the Bayes optimal) accuracy.

\subsection{\Ncons\emph{ }Consistency}

Having defined the \natural\emph{ }Bayes optimal classifier, we now turn our attention towards building classifiers that converge towards it. Before doing this, we must precisely define what it means to converge. Intuitively, this consists of building classifiers whose robustness regions ``approach" the robustness regions of the \natural\emph{ }Bayes optimal classifier. This motivates the definition of \textit{partial \natural\emph{ }robustness regions}.
\begin{defn}\label{def:partial_nat_region}
Let $0 < \kappa < 1$ be a real number, and let $\D = (\mu, \eta)$ be a data distribution over $\R^d \times \{\pm 1\}$. Let $\mu^+ = \{x: \eta(x) > \frac{1}{2}\}$, $\mu^- = \{x: \eta(x) < \frac{1}{2}\}$, and $\mu^{1/2} = \{x: \eta(x) = \frac{1}{2}\}$. For $x \in \mu^+$, we define the \textbf{\natural\emph{ }robustness region}, denoted $V_x$, as $$V_x = \{x': \rho(x, x') < \kappa\rho(\mu^- \cup \mu^{\frac{1}{2}}, x')\}.$$ It consists of all points that are closer to $x$ than they are to $\mu^- \cup \mu^{1/2}$ (points oppositely labeled from $x$) by a factor of $\kappa$. We can use a similar definition for $x \in \mu^{-}$. Finally, if $\eta(x) = \frac{1}{2}$, we simply set $V_x^\kappa = \{x\}$.
\end{defn}

Observe that $V_x^{\kappa} \subset V_x$ for all $0 < \kappa < 1$, and thus being robust with respect to $V_x^{\kappa}$ is a milder condition than $V_x$. Using this notion, we can now define margin consistency.

\begin{defn}\label{definition:neighborhood_consistent}
A learning algorithm $A$ is said to be \textbf{\ncons\emph{ }consistent} if the following holds for any data distribution $\D = (\mu, \eta)$ where $\eta$ is continuous on its support. For any $0 < \epsilon, \delta, \kappa < 1$, there exists $N$ such that for all $n \geq N$, with probability at least $1- \delta$ over $S \sim \D^n$, $$A_{\V^\kappa}(A_S, D) \geq A(g, \D) - \epsilon,$$  where $g$ denotes the Bayes optimal classifier and $A_S$ denotes the classifier learned by algorithm $A$ from dataset $S$. 
\end{defn}

This condition essentially says that the astuteness of the classifier learned by the algorithm converges towards the accuracy of the Bayes optimal classifier. Furthermore, we stipulate that this holds as long as the astuteness is measured with respect to some $\V^\kappa$. Observe that as $\kappa \to 1$, these regions converge towards the \natural\emph{ }robustness regions, thus giving us a classifier with robustness effectively equal to that of the \natural\emph{ }Bayes optimal classifier. 

\section{\Ncons\emph{ }Consistent Non-Parametric Classifiers}

Having defined \ncons\emph{ }consistency, we turn to the following question: which non-parametric algorithms are \ncons\emph{ }consistent? Our starting point will be the standard literature for the convergence of non-parametric classifiers with regard to accuracy. We begin by considering the standard conditions for $k_n$-nearest neighbors to converge (in accuracy) towards the Bayes optimal.

$k_n$-nearest neighbors is \textit{consistent} if and only if the following two conditions are met:  $\lim_{n \to \infty} k_n = \infty$, and $\lim_{n \to \infty} \frac{k_n}{n} = 0$. The first condition guarantees that each point is classified by using an increasing number of nearest neighbors (thus making the probability of a misclassification small), and the second condition guarantees that each point is classified using only points very close to it. We will refer to the first condition as \textit{precision}, and the second condition as \textit{locality.}  A natural question is whether the same principles suffice for \ncons\emph{ }consistency as well. We began by showing that without any additional constraints, the answer is no.

\begin{thm}\label{thm:lower_bound}
Let $\D = (\mu, \eta)$ be the data distribution where $\mu$ denotes the uniform distribution over $[0,1]$ and $\eta$ is defined as: $\eta(x) = x$. Over this space, let $\d$ be the euclidean distance metric. Suppose $k_n = O(\log n)$ for $1 \leq n < \infty$. Then $k_n$-nearest neighbors is not \ncons\emph{ }consistent with respect to $\D$. 
\end{thm}

The issue in the example above is that for smaller $k_n$, $k_n$-nearest neighbors lacks sufficient precision. For \ncons\emph{ }consistnecy, points must be labeled using even more training points than are needed accuracy. This is because the classifier must be uniformly correct across the entirety of $V_x^\kappa$. Thus, to build \ncons\emph{ }consistent classifiers, we must bolster the precision from the standard amount used for standard consistency. To do this, we begin by introducing \textit{splitting numbers}, a useful tool for bolstering the precision of weight functions.

\subsection{Splitting Numbers}

We will now generalize beyond nearest neighbors to consider weight functions. Doing so will allow us to simultaneously analyze nearest neighbors and kernel classifiers. To do so, we must first rigorously substantiate our intuitions about increasing precision into concrete requirements. This will require several technical definitions.

\begin{defn}\label{defn:prob_radius}
Let $\mu$ be a probability measure over $\R^d$. For any $x \in \R^d$, the \textbf{probability radius} $r_p(x)$ is the smallest radius for which $B(x, r_p(x))$ has probability mass at least $p$. More precisely, $r_p(x) = \inf\{r: \mu(B(x,r)) \geq p\}.$ 
\end{defn}

\begin{defn}\label{defn:splitting_number}
Let $W$ be a weight function and let $S = \{x_1, x_2, \dots, x_n\}$ be any finite subset of $\R^d$. For any $x \in \R^d$, $\alpha \geq 0$, and $0 \leq \beta \leq 1$, let $W_{x, \alpha, \beta}  = \{i: \d(x, x_i) \leq \alpha, w_i^S(x) \geq \beta\}.$ Then the \textbf{splitting number} of $W$ with respect to $S$, denoted as $T(W, S)$ is the number of distinct subsets generated by $W_{x, \alpha \beta}$ as $x$ ranges over $\R^d$, $\alpha$ ranges over $[0, \infty)$, and $\beta$ ranges over $[0,1]$. Thus $T(W, S) = |\{W_{x, \alpha, \beta}: x \in \R^d, 0 \leq \alpha, 0 \leq \beta \leq 1\}|.$
\end{defn}

Splitting numbers allow us to ensure high amounts of precision over a weight function. To prove \ncons\emph{ }consistency, it is necessary for a classifier to be correct at \textit{all} points in a given region. Consequently, techniques that consider a single point will be insufficient. The splitting number provides a mechanism for studying entire regions simultaneously. For more details on splitting numbers, we include several examples in the appendix.

\subsection{Sufficient Conditions for Neighborhood Consistency}

We now state our main result.
\begin{thm}\label{thm:main}
Let $W$ be a weight function, $\D$ a distribution over $\R^d \times \{\pm 1\}$, $\U$ a neighborhood preserving collection, and $(t_n)_1^{\infty}$ be a sequence of positive integers such that the following four conditions hold. 

1. $W$ is consistent (with resp. to accuracy) with resp. to $\D$.

2. For any $0 < p < 1$, $\lim_{n \to \infty} E_{S \sim \D^n} [\sup_{x \in \R^d} \sum_1^n w_i^S(x)1_{\d(x, x_i) > r_p(x)}] = 0.$

3. $\lim_{n \to \infty} E_{S \sim D^n}[t_n \sup_{x \in \R^d} w_i^S(x)] = 0$.

4. $\lim_{n \to \infty} E_{S \sim D^n}\frac{\log T(W,S)}{t_n} = 0$.

Then $W$ is \ncons\emph{ }consistent with respect to $\D$.
\end{thm}

\textbf{Remarks:} Condition 1 is necessary because \ncons\emph{ }consistency implies standard consistency -- or, convergence in accuracy to the Bayes Optimal. Standard consistency has been well studied for non-parametric classifiers, and there are a variety of results that can be used to ensure it -- for example, Stone's Theorem (included in the appendix). 

Conditions 2. and 3. are stronger version of conditions 2. and 3. of Stone's theorem. In particular, both include a supremum taken over all $x \in \R^d$ as opposed to simply considering a random point $x \sim \D$. This is necessary for ensuring correct labels on entire regions of points simultaneously. We also note that the dependence on $r_p(x)$ (as opposed to some fixed $r$) is a key property used for adaptive robustness. This allows the algorithm to adjust to potential differing distance scales over different regions in $\R^d$. This idea is reminiscent of the analysis given in \cite{Dasgupta14}, which also considers probability radii.

Condition 4. is an entirely new condition which allows us to simultaneously consider all $T(W,S)$ subsets of $S$. This is needed for analyzing weighted sums with arbitrary weights.

Next, we apply Theorem \ref{thm:main} to get specific examples of margin consistent non-parametric algorithms.

\subsection{Nearest Neighbors and Kernel Classifiers}

We now provide sufficient conditions for $k_n$-nearest neighbors to be  \ncons\emph{ }consistent.

\begin{cor}\label{cor:nn}
Suppose $(k_n)_1^{\infty}$ satisfies (1) $\lim_{n \to \infty} \frac{k_n}{n} = 0$, and (2) $\lim_{n \to \infty} \frac{\log n}{k_n} = 0$. Then $k_n$-nearest neighbors is \ncons\emph{ }consistent.
\end{cor}

As a result of Theorem \ref{thm:lower_bound}, corollary \ref{cor:nn} is tight for nearest neighbors. Thus $k_n$ nearest neighbors is \ncons\emph{ }consistent if and only if $k_n = \omega(\log n)$. 

Next, we give sufficient conditions for a kernel-similarity classifier.
\begin{cor}\label{cor:kern}
Let $W$ be a kernel classifier over $\R^d \times \Y$ constructed from $K: \R^+ \to \R^+$ and $h_n$. Suppose the following properties hold.

1. $K$ is decreasing, and satisfies $\int_{\R^d}K(||x||)dx < \infty.$

2. $\lim_{n \to \infty} h_n = 0$ and $\lim_{n \to \infty} nh_n^d = \infty$.

3. For any $c > 1$, $\lim_{x \to \infty} \frac{K(cx)}{K(x)} = 0$.

4. For any $x \geq 0$, $\lim_{n \to \infty} \frac{n}{\log n}K(\frac{x}{h_n}) = \infty$.

Then $W$ is \ncons\emph{ }consistent.
\end{cor}

Observe that conditions 1. 2. and 3. are satisfied by many common Kernel functions such as the Gaussian or Exponential kernel ($K(x) = \exp(-x^2)$/ $K(x) = \exp(-x)$). Condition 4. can be similarly satisfied by just increasing $h_n$ to be sufficiently large. Overall, this theorem states that Kernel classification is \ncons\emph{ }consistent as long as the bandwidth shrinks slowly enough.

\begin{figure}
    \centering
        \includegraphics[scale=0.25] {histogram_pic.png}
    \caption{we have a histogram classifier being applied to the blue and red regions. The classifier will be unable to construct good labels in the cells labeled $A, B, C$, and consequently will not be robust with respect to $V_x^{\kappa}$ for sufficiently large $\kappa$.}
    \label{fig:histogram1}
\end{figure}

\subsection{Histogram Classifiers}

Having discussed \ncons\emph{ }consistent nearest-neighbors and kernel classifier, we now turn our attention towards another popular weight function, histogram classifiers. Recall that histogram classifiers operate by partitioning their input space into increasingly small cells, and then classifying each cell by using a majority vote from the training examples within that cell (a detailed description can be found in the appendix). We seek to answer the following question: is increasing precision sufficient for making histogram classifiers \ncons\emph{ }consistent? Unfortunately, the answer this turns out not to be no. The main issue is that histogram classifiers have no mechanism for performing classification outside the support of the data distribution. 

For an example of this, refer to Figure \ref{fig:histogram1}. Here we see a distribution being classified by a histogram classifier. Observe that the cell labeled $A$ contains points that are strictly closer to $\mu^+$ than $\mu^-$, and consequently, for sufficiently large $\kappa$, $V_x^{\kappa}$ will intersect $A$ for some point $x \in \mu^+$. A similar argument holds for the cells labeled $B$ and $C.$. However, since $A, B, C$ are all in cells that will never contain any data, they will never be labeled in a meaningful way. Because of this, histogram classifiers are not \ncons\emph{ }consistent.

\section{Validation}

%\begin{figure}[ht]
%	\centering
%	\subfloat[exponential kernel]{\includegraphics[width=.4\textwidth]{exponential.png}}
%	\subfloat[polynomial kernel]{\includegraphics[width=.4\textwidth]{polynomial.png}}
%	\caption{Plots of astuteness against the training sample size. In both panels, accuracy is plotted in red, and the varying levels of robustness regions $(\kappa = 0.1, 0.3, 0.5)$ are givne in blue, green and purple. In panel (a), observe that as sample size increases, every measure of astuteness converges towards $0.8$ which is as predicted by Corollary \ref{cor:kern}. In panel (b), although the accuracy appears to converge, none of the robustness measure. In fact, they get progressively worse the larger $\kappa$ gets. 
%	}
%	\label{fig:validation}
%\end{figure}

\begin{figure}
\begin{subfigure}{0.45\textwidth}
\includegraphics[width=\linewidth]{exponential.png}
\caption{exponential kernel} 
\end{subfigure}\hspace*{\fill}
\begin{subfigure}{0.45\textwidth}
\includegraphics[width=\linewidth]{polynomial.png}
\caption{polynomial kernel} 
\end{subfigure}

\caption{Plots of astuteness against the training sample size. In both panels, accuracy is plotted in red, and the varying levels of robustness regions $(\kappa = 0.1, 0.3, 0.5)$ are givne in blue, green and purple. In panel (a), observe that as sample size increases, every measure of astuteness converges towards $0.8$ which is as predicted by Corollary \ref{cor:kern}. In panel (b), although the accuracy appears to converge, none of the robustness measure. In fact, they get progressively worse the larger $\kappa$ gets.}
	\label{fig:validation}
\end{figure}

To complement our theoretical large sample results for non-parametric classifiers, we now include several experiments to understand their behavior for finite samples. We seek to understand how quickly non-parametic classifiers converge towards the \natural\emph{ }Bayes optimal. 

We focus our attention on kernel classifiers and use two different kernel similarity functions: the first, an exponential kernel, and the second, a polynomial kernel.  These classifiers were chosen so that the former meets the conditions of Corollary \ref{cor:kern}, and the latter does not. Full details on these classifiers can be found in the appendix.

To be able to measure performance with increasing data size, we look at a simple synthetic dataset over overlayed circles (see Figure \ref{fig:distribution} for an illustration) with support designed so that the data is intrinsically multiscaled. In particular, this calls for different levels of robustness in different regions. For simplicity, we use a global label noise parameter of $0.2$, meaning that any sample drawn from this distribution is labeled differently than its support with probability $0.2$. Further details about our dataset are given in section \ref{sec:experiment_details}. 

\textbf{Performance Measure.} For a given classifier, we evaluate its astuteness at a test point $x$ with respect to the robustness region $V_x^{\kappa}$ (Definition \ref{def:partial_nat_region}). While these regions are not computable in practice due to their dependency on the  support of the data distribution, we are able to approximate them for this synthetic example due to our explicit knowledge of the data distribution. Details for doing this can be found in the appendix. To compute the empirical astuteness of a kernel classifier $W_K$ about test point $x$, we perform a grid search over all points in $V_x^{\kappa}$ to ensure that all points in the robustness region are labeled correctly.  

For each classifier, we measure the empirical astuteness by using three trials of $20$ test points and taking the average. While this is a relatively small amount of test data, it suffices as our purpose is to just verify that the algorithm roughly converges towards the optimal possible astuteness. Recall that for any \ncons\emph{ }consistent algorithm, as $n \to \infty$, $A_{\mathcal{V}^\kappa}$ should converge towards $A^*$, the accuracy of the Bayes optimal classifier, for \textit{any} $0 < \kappa < 1$. Thus, to verify this holds, we use $\kappa = 0.1, 0.3, 0.5$. For each of these values, we plot the empirical astuteness as the training sample size $n$ gets larger and larger. As a baseline, we also plot their standard accuracy on the test set. 

\textbf{Results and Discussion:} The results are presented in Figure~\ref{fig:validation}; the left panel is for the exponential kernel, while the right one is for the polynomial kernel. As predicted by our theory, we see that in all cases, the exponential kernel converges towards the maximum astuteness regardless of the value of $\kappa$: the only difference is that the rate of convergence is slower for larger values of $\kappa$. This is, of course, expected because larger values of $\kappa$ entail larger robustness regions. 

By contrast, the polynomial kernel performs progressively worse for larger values of $\kappa$. This kernel was selected specifically to violate the conditions of Corollary \ref{cor:kern}, and in particular fails criteria 3. However, note that the polynomial kernel nevertheless performs will with respect to accuracy thus giving another example demonstrating the added difficulty of \ncons\emph{ }consistency.

Our results bridge the gap between our asymptotic theoretical results and finite sample regimes. In particular, we see that kernel classifiers that meet the conditions of Corollary \ref{cor:kern} are able to converge in astuteness towards the \natural\emph{ }Bayes optimal classifier, while classifiers that do not meet these conditions fail.

\section{Related Work}\label{sec:rel_work}

There is a wealth of literature on robust classification, most of which impose the same robustness radius $r$ on the entire data.  \cite{Carlini17, Liu17, Papernot17, Papernot16,Szegedy14, Hein17,Katz17,Schmidt18,Wu16,Steinhardt18, Sinha18}, among others, focus primarily on neural networks, and robustness regions that are $\ell_1, \ell_2, $ or $\ell_\infty$ norm balls of a given radius $r$. 

\cite{Cheng20} and~\cite{Ding20} show how to train neural networks with different robustness radii at different points by trading off robustness and accuracy; their work differ from ours in that they focus on neural networks, their robustness regions are still norm balls, and that their work is largely empirical.

Our framework is also related to large margin classification -- in the sense that the robustness regions $\U$ induce a {\em{margin constraint}} on the decision boundary. The most popular large margin classifier is the Support Vector Machine\cite{cortes95, Bennett00, Freund99} -- a large margin linear classifier that minimizes the worst-case margin over the training data. Similar ideas have also been used to design classifiers that are more flexible than linear; for example, \cite{Luxburg03} shows how to build large margin Lipschitz classifiers by rounding globally Lipschitz functions. Finally, there has also been purely empirical work on achieving large margins for more complex classifiers -- such as~\cite{Samy18} for deep neural networks that minimizes the worst case margin, and~\cite{Weinberger05} for metric learning to find large margin nearest neighbors. Our work differs from these in that our goal is to ensure a high enough local margin at each $x$, (by considering the \natural\emph{ }regions $V_x$) as opposed to optimizing a global margin. 

%However, in all of these cases, the goal is substantially different from building astute classifiers. Broadly speaking, this area is centered around minimizing \textit{global} margin, whereas our work can be thought of us optimizing local margin, since the regions $\U_x$ vary considerably in size based on $x$. 


%The relationship here is that building a robust classifier for a neighborhood preserving collection $\U$, bears some resemblance to building a large margin classifier.. In particular, the conditions on $\U$ indicates that the decision boundary for the data distribution, $\D$, should ideally be fairly far away from the support. 





%Nearly all of these works consider robustness regions $\U_x$ that are balls of radius $r$, typically using the $\ell_1, \ell_2, $ and $\ell_\infty$ norms. 

Finally, our analysis builds on prior work on robust classification for non-parametric methods in the standard framework. \cite{Amsaleg17, Sitawarin19, WJC18, YRWC19} provide adversarial attacks on non-parametric methods. Wang et. al. \cite{WJC18} develops a defense for $1$-NN that removes a subset of the training set to ensure higher robustness. Yang et. al~\cite{YRWC19} proposes the $r$-optimal classifier -- which is the maximally astute classifier in the standard robustness framework -- and proposes a defense called Adversarial Pruning. 


%that relies on \textbf{adversarial pruning}, which is process in which points are removed from the training set so that any two elements with opposite labels have distance at least $2r$, where $r$ is the robustness radius.  They show that their algorithm converges towards the same robustness as the Bayes optimal classifier. This was subsequently improved upon in Yang et. al. \cite{YRWC19} which proposes the $r$-optimal classifier, which is the maximally robust classifier. 

Theoretically, \cite{Bhattacharjee20} provide conditions under which weight functions converge towards the $r$-optimal classifier in the large sample limit. They show that for $r$-separated distributions, where points from different classes are at least distance $2r$ or more apart, nearest neighbors and kernel classifiers satisfy these conditions. In the more general case, they use Adversarial Pruning as a preprocessing step to ensure that the training data is $r$-separated, and show that this preprocessing step followed by nearest neighbors or kernel classifiers leads to solutions that are robust and accurate in the large sample limit. Our result fundamentally differs from theirs in that we analyze a different algorithm, and our proof techniques are quite different. In particular, the fundamental differences between the $r$-optimal classifier and the \natural\emph{ }Bayes optimal classifier call for different algorithms and different analysis techniques.

In concurrent work, \cite{Chowdhury21} proposes a similar limit to the neighborhood preserving Bayes optimal which they refer to as the margin canonical Bayes. However, their work then focuses on a data augmentation technique that leads to convergence whereas we focus on proving the neighborhood consistency of classical non-parametric classifiers.



%\newcommand{\crop}[1]{\mathrm{crop}({#1})}
\newcommand{\object}[1]{\mathrm{object}({#1})}
\newcommand{\ba}{A_i}
\newcommand{\bb}{B_i}
\newcommand{\calA}{\mathcal{A}}
\newcommand{\calB}{\mathcal{B}}
\newcommand{\calX}{\mathcal{X}}
\newcommand{\masked}[1]{\mathrm{masked}({#1})}
\newcommand{\bx}{\mathbf{x}}
\newcommand{\SSL}{\textsc{SSL}}
\newcommand{\SSLbb}{\SSL^\mathrm{back}}
\newcommand{\SSLpj}{\SSL^\mathrm{proj}}
\newcommand{\CLF}{\textsc{CLF}}
\newcommand{\CLFbb}{\CLF^\mathrm{back}}
\newcommand{\CLFpj}{\CLF^\mathrm{proj}}
\newcommand{\SUP}{\textsc{SUP}}
\newcommand{\KNN}{\textsc{KNN}}
\newcommand{\KNNset}{\textsc{KNN}^\mathrm{set}}
\newcommand{\KNNprob}{\textsc{KNN}^\mathrm{prob}}
\newcommand{\KNNcl}{\textsc{KNN}^\mathrm{cl}}
\newcommand{\KNNconf}{\textsc{KNN}^\mathrm{conf}}
\newcommand{\RCDM}{\textsc{RCDM}}
\newcommand{\cl}{\mathrm{cl}}
\newcommand{\clpred}{\tilde{\mathrm{cl}}}
\newcommand{\Abox}{\overline{\calA}}
\newcommand{\Bbox}{\overline{\calB}}
\newcommand{\dejavu}{\emph{déjà vu }}
\newcommand{\Dejavu}{\emph{Déjà vu }}

\newcommand{\citations}{{\color{green}[CITE]}}

\definecolor{part_blue}{rgb}{0.2824, 0.4706, .8157}
\definecolor{part_red}{rgb}{0.8392, 0.3725, 0.3725}
\definecolor{part_orange}{rgb}{0.9333, 0.5216, 0.2902}

\DeclareRobustCommand{\mybox}[2][gray!20]{%
\begin{tcolorbox}[   %% Adjust the following parameters at will.
        % breakable,
        left=0pt,
        right=0pt,
        top=0pt,
        bottom=0pt,
        colback=#1,
        colframe=#1,
        width=\dimexpr\columnwidth\relax, 
        % width=\textwidth, 
        enlarge left by=0mm,
        boxsep=5pt,
        arc=0pt,outer arc=0pt,
        ]
        #2
\end{tcolorbox}
}
%\section{Introduction}
\label{sec:intro}
Self-supervised learning (SSL)~\citep{chen2020simclr, chen2020simsiam, zbontar2021barlow, vicreg, caron2020swav, MAE} aims to learn general representations of content-rich data without explicit labels by solving a \textit{pretext task}. In many recent works, such pretext tasks rely on joint-embedding architectures whereby randomized image augmentations are applied to create multiple views of a training sample, and the model is trained to produce similar representations for those views. When using cropping as random image augmentation, the model learns to associate objects or parts (including the background scenery) that co-occur in an image.
However, doing so also arguably exposes the training data to higher privacy risk as objects in training images can be explicitly memorized by the SSL model. For example, if the training data contains the photos of individuals, the SSL model may learn to associate the face of a person with their activity or physical location in the photo. This may allow an adversary to extract such information from the trained model for targeted individuals.

\begin{figure}[t]
    \centering
    \includegraphics[width=1.0\columnwidth]{figures/new_black_swan.pdf}
    \caption{\textbf{Left:} Reconstruction of an SSL training image from a crop containing only the background. The SSL model memorizes the association of this \emph{specific} patch of water (pink square) to this \emph{specific} foreground object (a black swan) in its embedding, which we decode to visualize the full training image. \textbf{Right:} The reconstruction technique fails on a public test image that the SSL model has not seen before.}
    \label{fig:black_swan}
\end{figure}

In this work, we aim to evaluate to what extent SSL models memorize the association of specific objects in training images or the association of objects and their specific backgrounds, and whether this memorization signal can be used to reconstruct the model's training samples. Our results demonstrate that SSL models memorize such associations beyond simple correlation. For instance, in Figure \ref{fig:black_swan} (\textbf{left}), we use the SSL representation of a \emph{training image crop containing only water} and this enables us to reconstruct the object in the foreground with remarkable specificity---in this case a black swan.
By contrast, in Figure \ref{fig:black_swan} (\textbf{right}), when using the \emph{crop from the background of a test set image} that the SSL model \emph{has not seen before}, its representation only contains enough information to infer, through correlation, that the foreground object was likely some kind of waterbird --- but not the specific one in the image.

Figure \ref{fig:black_swan} shows that SSL models suffer from the unintended memorization of images in their training data---a phenomenon we refer to as \emph{déjà vu memorization}
%\footnote{The French loanword \emph{déjà vu} means already-seen, which reflects the type of unintended memorization of objects that the SSL model saw during training.}.
\footnote{The French loanword \emph{déjà vu} means `already-seen', just as an image is seen and memorized in training.}
Beyond visualizing \emph{déjà vu} memorization through data reconstruction, we also design a series of experiments to quantify the degree of memorization for different SSL algorithms, model architectures, training set size, \emph{etc.} We observe that \emph{déjà vu} memorization is exacerbated by the atypically large number of training epochs often recommended in SSL training, as well as certain hyperparameters in the SSL training objective. Perhaps surprisingly, we show that \emph{déjà vu} memorization occurs even when the training set is large---as large as half of ImageNet~\citep{imagenet}---and can continually worsen even when standard techniques for evaluating learned representation quality (such as linear probing) do not suggest increased overfitting. Our work serves as the first systematic study of unintended memorization in SSL models and motivates future work on understanding and preventing this behavior. Specifically, we: 
\begin{itemize}
    \vspace{-0.5em}
    \item Elucidate how SSL representations memorize aspects of individual training images, what we call \emph{déjà vu} memorization;
    \item Design a novel training data reconstruction pipeline for non-generative vision models. This is in contrast to many prominent reconstruction algorithms like \citep{carlini2021extracting, google_diffusion}, which rely on the model itself to generate its own memorized samples and is not possible for SSL models or classifiers;
    \item Propose metrics to quantify the degree of \dejavu memorization committed by an SSL model. This allows us to observe how \dejavu changes with training epochs, dataset size, training criteria, model architecture and more. 
\end{itemize}

%\section{Preliminaries and Related Work}
\label{sec:related}

\textbf{Self-supervised learning} (SSL) is a machine learning paradigm that leverages unlabeled data to learn representations. Many SSL algorithms rely on \emph{joint-embedding} architectures (\emph{e.g.}, SimCLR~\citep{chen2020simclr}, Barlow Twins~\citep{zbontar2021barlow}, VICReg~\citep{vicreg} and Dino~\citep{Dino}), which are trained to associate different augmented views of a given image. For example, in SimCLR, given a set of images $\calA = \{A_1,\ldots,A_n\}$ and a randomized augmentation function $\mathrm{aug}$, the model is trained to maximize the cosine similarity of draws of $\SSL(\mathrm{aug}(A_i))$ with each other and minimize their similarity with $\SSL(\mathrm{aug}(A_j))$ for $i \neq j$. The augmentation function $\mathrm{aug}$ typically consists of operations such as cropping, horizontal flipping, and color transformations to create different views that preserve an image's semantic properties. 

\paragraph{SSL representations.} Once an SSL model is trained, its learned representation can be transferred to different downstream tasks. This is often done by extracting the representation of an image from the \emph{backbone model}\footnote{SSL methods often use a trick called \emph{guillotine regularization}~\citep{Guillotine}, which decomposes the model into two parts: a \emph{backbone model} and a \emph{projector} consisting of a few fully-connected layers. Such trick is needed to handle the misalignment between the pretext SSL task and the downstream task.} and either training a linear probe on top of this representation or finetuning the backbone model with a task-specific head~\citep{Guillotine}.
%Compared to representations learned by supervised learning, SSL representations are often more robust and transferable~\citep{hendrycks2019using, ericsson2021self}, leading to state-of-the-art result on many downstream tasks. To understand the effectiveness of SSL algorithms, several prior works investigated what kind of information the SSL model has learned~\citep{jing2021understanding, ericsson2021self, kalibhat2022towards, RCDM}. In particular, \citet{RCDM} trained a conditional generative model on SSL representations and showed that they encode richer visual details about the input image compared to supervised learning. 
%However, from a privacy perspective, this may be a cause for concern as the model also has more potential to overfit and memorize precise details about the training data compared to supervised learning. We show concretely that this privacy risk can indeed be realized by defining and measuring \emph{déjà vu} memorization.
It has been shown that SSL representations encode richer visual details about input images than supervised models do \cite{RCDM}. However, from a privacy perspective, this may be a cause for concern as the model also has more potential to overfit and memorize precise details about the training data compared to supervised learning. We show concretely that this privacy risk can indeed be realized by defining and measuring \emph{déjà vu} memorization.
\vspace{-0.5em} 
% \paragraph{Privacy risks in ML.} Overfitting in ML occurs when a model memorizes information specific to its training data rather than general population-level information. When the model is trained on privacy-sensitive data, overfitting is especially harmful as an adversary can infer private information about the training data when given access to the model~\citep{yeom2018privacy, feldman2020does}. The simplest and most well-studied form of privacy risk in ML is susceptibility to \emph{membership inference attacks}~\citep{shokri2017membership, salem2018ml, sablayrolles2019white}, where the adversary infers whether an individual is part of the training set or not. More sophisticated privacy attacks include \emph{attribute inference}~\citep{fredrikson2014privacy, mehnaz2022your, jayaraman2022attribute}, where specific attributes about an individual are inferred given others, and \emph{data reconstruction}~\citep{carlini2021extracting, balle2022reconstructing, guo2022bounding}, where entire training samples are recovered from the trained model. Our study of \emph{déjà vu} memorization is similar to both attribute inference and data reconstruction, leveraging SSL representations of the training image background to infer and reconstruct the foreground object.
% \vspace{-0.5em} 
% \paragraph{Training data extraction in NLP.} Our study of \dejavu memorization in SSL models is inspired by similar work in the natural language processing (NLP) domain. \citet{carlini2019secret} first showed that language models exhibit unintended memorization, where given a context string present in its training data, the model can generate the remaining text at test time. This unintended memorization has been further exploited in \citet{carlini2021extracting} to extract training data from GPT-2~\citep{radford2019language} and, more recently, extended to extract memorized images from Stable Diffusion \citep{google_diffusion}. The way by which these works exploit unintended memorization is similar to ours: given partial information about a training sample, the model is prompted to reveal the rest of the sample. In our case, however, since the SSL model is not generative, extraction is significantly harder and requires careful design.

\paragraph{Privacy risks in ML.} When a model is overfit on privacy-sensitive data, it memorizes specific information about its training examples, allowing an adversary with access to the model to learn private information~\citep{yeom2018privacy, feldman2020does}. Privacy attacks in ML range from the simplest and best-studied \emph{membership inference attacks}~\citep{shokri2017membership, salem2018ml, sablayrolles2019white} to \emph{attribute inference}~\citep{fredrikson2014privacy, mehnaz2022your, jayaraman2022attribute} and \emph{data reconstruction}~\citep{carlini2021extracting, balle2022reconstructing, guo2022bounding} attacks. In the former, the adversary only infers whether an individual participated in the training set. Our study of \emph{déjà vu} memorization is most similar to the latter: we leverage SSL representations of the training image background to infer and reconstruct the foreground object. Our approach reflects similar work in the NLP domain \citep{carlini2019secret, carlini2021extracting}: when prompted with a context string present in the training data, a large language model is shown to generate the remainder of string at test time, revealing sensitive text like home addresses. This method was recently extended to extract memorized images from Stable Diffusion \citep{google_diffusion}.  We exploit memorization in a similar manner: given partial information about a training sample, the model is prompted to reveal the rest of the sample. In our case, however, since the SSL model is not generative, extraction is significantly harder and requires careful design.

%\section{Defining \emph{Déjà Vu} Memorization}
\label{sec:definition}

\paragraph{What is \dejavu memorization?} At a high level, the objective of SSL is to learn general representations of objects that occur in nature. This is often accomplished by associating different parts of an image with one another in the learned embedding. Returning to our example in Figure \ref{fig:black_swan}, given an image whose background contains a patch of water, the model may learn that the foreground object is a water animal such as duck, pelican, otter, \emph{etc.}, by observing different images that contain water from the training set. We refer to this type of learning as \emph{correlation}: the association of objects that tend to co-occur in images from the training data distribution.

A natural question to ask is \emph{``Can the reconstruction of the black swan in Figure \ref{fig:black_swan} be reasoned as correlation?''} The intuitive answer may be no, since the reconstructed image is qualitatively very similar to the original image. However, this reasoning implicitly assumes that for a random image from the training data distribution containing a patch of water, the foreground object is unlikely to be a black swan. Mathematically, if we denote by $\mathcal{P}$ the training data distribution and $A$ the image, then
\begin{equation*}
\label{eq:p_corr}
p_\text{corr} := \mathbb{P}_{A \sim \mathcal{P}}(\mathrm{object}(A) = \texttt{black swan} ~|~ \mathrm{crop}(A) = \texttt{water})
\end{equation*}
is the probability of inferring that the foreground object is a black swan through \emph{correlation}. This probability may be naturally high due to biases in the distribution $\mathcal{P}$, \emph{e.g.}, if $\mathcal{P}$ contains no other water animal except for black swans. In fact, such correlations are often exploited to learn a model for image inpainting with great success~\citep{yu2018generative, ulyanov2018deep}.

Despite this, we argue that reconstruction of the black swan in Figure \ref{fig:black_swan} is \emph{not} due to correlation, but rather due to \emph{unintended memorization}: the association of objects unique to a single training image. As we will show in the following sections, the example in Figure \ref{fig:black_swan} is not a rare success case and can be replicated across many training samples. More importantly, failure to reconstruct the foreground object in Figure \ref{fig:black_swan} (\textbf{right}) on test images hints at inferring through correlation is unlikely to succeed---a fact that we verify quantitatively in Section \ref{sec:label inference accuracy}. Motivated by this discussion, we give a verbal definition of \dejavu memorization below, and design a testing methodology to quantify \dejavu memorization in Section \ref{sec:notation and setup}.
\mybox{\textbf{Definition:} A model exhibits \emph{déjà vu memorization} when it retains information so specific to an individual training image, that it enables recovery of aspects particular to that image given a part that does not contain them.
The recovered aspect must be beyond what can be inferred using only correlations in the data distribution.} 

% \textbf{Definition:} A model exhibits \emph{déjà vu memorization} when it retains information so specific to an individual training image, that it enables recovery of aspects particular to that image given a part that does not contain them.
% The recovered aspect must be beyond what can be inferred using only correlations in the data distribution.


 We intentionally kept the above definition broad enough to encompass different types of information that can be inferred about the training image, including but not restricted to object category, shape, color and position. For example, if one can infer that the foreground object is red given the background patch with accuracy significantly beyond correlation, we consider this an instance of \dejavu memorization as well. We mainly focus on object category to quantify \dejavu memorization in Section \ref{sec:quant} since the ground truth label can be easily obtained. We consider other types of information more qualitatively in the visual reconstruction experiments in Section \ref{sec:visualizing}.

\paragraph{Privacy implications of \dejavu memorization.} \Dejavu memorization can be a cause for concern when the training data contains privacy-sensitive information. As a motivating example, consider an SSL model trained on photos of individuals. If the model exhibits \dejavu memorization then, given the face of an individual, it may be possible to infer where the individual was or even visually reconstruct their location in the training image. Such information leakage raises privacy concerns, especially if there was no prior agreement that the trained model may reveal such information to third parties. This hypothetical scenario serves as a motivation that \dejavu memorization should be carefully examined to avoid unintended disclosure of private information in practical applications.

% \begin{figure*}[h]
%     \centering
%     \includegraphics[width = 0.85\textwidth]{figures/SSL_attack_cartoon.png}
%     \caption{We measure memorization by comparing the `target model' trained on the target image ($\SSL_A$ trained on $A_i$ in above example) with the `reference model' not trained on it ($\SSL_B$, above). \textbf{[Top Strip]} A cropping of the image disjoint from the labeled foreground object is embedded using the target model. This embedding is then labeled by a K-Nearest Neighbor (KNN) adversary built on a public set of labeled images, $X$, which it has also embedded using the target model. \textbf{[Bottom Strip]} To account for correlation, the same procedure is followed with the reference model. If the label is only extracted using the target model, it is counted as memorization. If it is extracted using either model, it is counted as correlation. We find that the KNN adversary's predictions using the target model (trained on attacked examples) are significantly more accurate than they are using the reference model, indicating routine memorization of training examples.}
%     \label{fig:ssl attack cartoon}
% \end{figure*}

\begin{figure}[t]
%%%
%SPIDER
%%%
     % \centering
     % \begin{subfigure}[b]{0.25\textwidth}
     %     \centering
     %     \includegraphics[width=\textwidth]{figures/data_split.png}
     %     % \caption{SimCLR correlated \textit{yellow garden spider} examples}
     %     \label{fig:data split}
     % \end{subfigure}
     % \hfill
     % \begin{subfigure}[b]{0.7\textwidth}
     %     \centering
     %     \includegraphics[width=\textwidth]{figures/pipeline_cartoon.png}
     %     \begin{minipage}{5cm}
     %        \vfill
     %    \end{minipage}
     %     % \caption{SimCLR memorized \textit{yellow garden spider} examples}
     %     \label{fig:pipeline cartoon}
     % \end{subfigure}
     \includegraphics[width=\textwidth]{figures/split_and_pipeline_cartoon.png}
\caption[Overview of testing methodology.]{
Overview of testing methodology. \textbf{Left:} Data is split into \emph{target set} $\calA$, \emph{reference set} $\calB$ and \emph{public set} $\calX$ that are pairwise disjoint. $\calA$ and $\calB$ are used to train two SSL models $\SSL_A$ and $\SSL_B$ in the same manner. $\calX$ is used for KNN decoding or for training an RCDM to reconstruct the input at test time. \textbf{Right:} Given a training image $A_i \in \calA$, we use $\SSL_A$ to embed $\crop{A_i}$ containing only the background, as well as the entire set $\calX$ and find the $k$-nearest neighbors of $\crop{A_i}$ in $\calX$ in the embedding space. These KNN samples can be used directly to infer the foreground object (\emph{i.e.}, class label) in $A_i$ using a KNN classifier, or their embeddings can be averaged as input to the trained RCDM to visually reconstruct the image $A_i$. For instance, the RCDM reconstruction results in Figure \ref{fig:black_swan} (left) when given $\SSL_A(\crop{A_i})$ and results in Figure \ref{fig:black_swan} (right) when given $\SSL_A(\crop{B_i})$ for an image $B_i \in \calB$.
%\textbf{Left:} illustration of the three datasets used in our tests. Two private data sets, $A$ and $B$, of equal size are used to train two SSL models, $\SSL_A$ and $\SSL_B$, respectively. A disjoint public set, $X$, is made available to the memorization test to help decode model embeddings. Memorization is only tested on examples $A_i \in A$ that are unique to set $A$. \textbf{Right:} illustration of inference pipeline used in tests. A periphery cropping that excludes the foreground object is taken from private image $A_i$. The KNN then finds the $k$ public set nearest neighbors of the periphery crop in the embedding space of $\SSL_A$. 
%The $\SSL_A$ representation of these $k$ neighbors and of the crop are used by the conditional generative model, RCDM, to reconstruct the foreground object. The labels of these $k$ neighbors are used to recover the foreground object label. (Not pictured) We repeat this process using reference model $\SSL_B$, not trained on image $A_i$, to determine whether the foreground object is still recoverable by learned correlations, e.g. if black swans were the only objects appearing near water in the data distribution. In this instance, the crop's public set neighbors in $\SSL_B$'s representation space include a variety of water animals like ducks, pelicans, and otters. Meanwhile, with $\SSL_A$, the neighbors are nearly all black swans in the same position as the swan of $A_i$.
}
\label{fig:split_and_pipeline_cartoon}
\end{figure}

\textbf{Distinguishing memorization from correlation.} When measuring \dejavu memorization, it is crucial to differentiate what the model associates through \emph{memorization} and what it associates through \emph{correlation}. Our testing methodology is based on the following intuitive definition.
\mybox{\textbf{Definition:} If an SSL model associates two parts in a training image, we say that it is due to \emph{correlation} if other SSL models trained on a similar dataset from $\mathcal{P}$ without this image would likely make the same association. Otherwise, we say that it is due to \emph{memorization}.}

Notably, such intuition forms the basis for differential privacy (DP; \cite{dwork2006calibrating, dwork2013algorithmic})---the most widely accepted notion of privacy in ML.

\subsection{Testing Methodology for Measuring \emph{Déjà Vu} Memorization}
\label{sec:notation and setup}

In this section, we use the above intuition to measure the extent of \dejavu memorization in SSL. Figure \ref{fig:split_and_pipeline_cartoon} gives an overview of our testing methodology.
\vspace{-0.75em}
\paragraph{Dataset splitting.} We focus on testing \dejavu memorization for SSL models trained on the ImageNet-1K dataset~\citep{imagenet}. Our test first splits the ImageNet training set into three independent and disjoint subsets $\calA$, $\calB$ and $\calX$. The dataset $\calA$ is called the \emph{target set} and $\calB$ is called the \emph{reference set}. The two datasets are used to train two separate SSL models, $\SSL_A$ and $\SSL_B$, called the \emph{target model} and the \emph{reference model}. Finally, the dataset set $\calX$ is used as an auxiliary public dataset to extract information from $\SSL_A$ and $\SSL_B$.
%\footnote{See Appendix \ref{sec:appx splits} for details on how the dataset splits are generated.}.
Our dataset splitting serves the purpose of distinguishing memorization from correlation in the following manner. Given a sample $A_i \in \calA$, if our test returns the same result on $\SSL_A$ and $\SSL_B$ then it is likely due to correlation because $A_i$ is not a training sample for $\SSL_B$. Otherwise, because $\calA$ and $\calB$ are drawn from the same underlying distribution, our test must have inferred some information unique to $A_i$ due to memorization. Thus, by comparing the difference in the test results for $\SSL_A$ and $\SSL_B$, we can measure the degree of \dejavu memorization\footnote{See Appendix \ref{sec:appx splits} for details on how the dataset splits are generated.}.
\vspace{-0.75em}
\paragraph{Extracting foreground and background crops.} Our testing methodology aims at measuring what can be inferred about the foreground object in an ImageNet sample given a background crop. This is made possible because ImageNet provides bounding box annotations for a subset of its training images---around 150K out of 1.3M samples. We split these annotated images equally between $\calA$ and $\calB$. Given an annotated image $A_i$, we treat everything inside the bounding box as the foreground object associated with the image label, denoted $\object{A_i}$. We take the largest possible crop that does not intersect with any bounding box as the background crop (or \emph{periphery crop}), denoted $\crop{A_i}$\footnote{We also present another heuristic in \cref{sec:appx corner crop} which takes a corner crop as the background crop, allowing our test to be run without bounding box annotations.}
%Since the labeled object tends to be at the image's center, the corner crop usually excludes it. }
%Because most images in ImageNet are object centric, an image's corner would not include the foreground object.}.
\vspace{-0.75em}
\paragraph{KNN-based test design.} Joint-embedding SSL approaches encourage the embeddings of random crops of a training image $A_i \in \calA$ to be similar. Intuitively, if the model exhibits \dejavu memorization, it is reasonable to expect that the embedding of $\crop{A_i}$ is similar to that of $\object{A_i}$ since both crops are from the same training image. In other words, $\SSL_A(\crop{A_i})$ encodes information about $\object{A_i}$ that cannot be inferred through correlation. However, decoding such information is challenging as these approaches do not learn a decoder associated with the encoder $\SSL_A$.

Here, we leverage the public set $\calX$ to decode the information contained in $\crop{A_i}$ about $\object{A_i}$. More specifically, we map images in $\calX$ to their embeddings using $\SSL_A$ and extract the $k$-nearest-neighbor (KNN) subset of $\SSL_A(\crop{A_i})$ in $\calX$. We can then decode the information contained in $\crop{A_i}$ in one of two ways:
\begin{itemize}
\item \emph{Label inference:} Since $\calX$ is a subset of ImageNet, each embedding in the KNN subset is associated with a class label. If $\crop{A_i}$ encodes information about the foreground object, its embedding will be close to samples in $\calX$ that have the same class label (\emph{i.e.}, foreground object category). We can then use a KNN classifier to infer the foreground object in $A_i$ given $\crop{A_i}$.
\item \emph{Visual reconstruction:} Following \citet{RCDM}, we train an RCDM---a conditional generative model---on $\calX$ to decode $\SSL_A$ embeddings into images. The RCDM reconstruction can recover qualitative aspects of an image remarkably well, such as recovering object color or spatial orientation using its SSL embedding. Given the KNN subset, we average their SSL embeddings and use the trained RCDM model to visually reconstruct $A_i$.
\end{itemize}
In Section \ref{sec:quant}, we focus on quantitatively measuring \dejavu memorization with label inference, and then use the RCDM reconstruction to visualize \dejavu memorization in Section \ref{sec:visualizing}.
%\section{Quantifying \emph{Déjà Vu} Memorization}
\label{sec:quant}

We apply our testing methodology to quantify a specific form of \dejavu memorization: inferring the foreground object (class label) given a crop of the background.

% \paragraph{Extracting model embeddings.} We test \dejavu memorization on two popular SSL algorithms, SimCLR~\citep{chen2020simclr} and VICReg~\citep{vicreg}.
% %\footnote{We present additional SSL models in \cref{sec:appx simclr results}} 
% As described in Section \ref{sec:related}, these algorithms produce two embeddings given an input image: a \emph{backbone} embedding and a \emph{projector} embedding that is derived by applying a small fully-connected network on top of the backbone embedding. Unless otherwise noted, all SSL embeddings refer to the projector embedding.
% To understand whether \dejavu memorization is particular to SSL, we also evaluate embeddings produced by a supervised model $\CLF_A$ trained on $\calA$. We apply the same set of image augmentations as those used in SSL and train $\CLF_A$ using the cross-entropy loss to predict ground truth labels. 
\vspace{-0.75em}
\paragraph{Extracting model embeddings.} We test \dejavu memorization on a variety of popular SSL algorithms, with a focus on VICReg~\citep{vicreg}. These algorithms produce two embeddings given an input image: a \emph{backbone} embedding and a \emph{projector} embedding that is derived by applying a small fully-connected network on top of the backbone embedding. Unless otherwise noted, all SSL embeddings refer to the projector embedding. 
To understand whether \dejavu memorization is particular to SSL, we also evaluate embeddings produced by a supervised model $\CLF_A$ trained on $\calA$. We apply the same set of image augmentations as those used in SSL and train $\CLF_A$ using the cross-entropy loss to predict ground truth labels. 
\vspace{-0.75em}
\paragraph{Identifying the most memorized samples.} Prior works have shown that certain training samples can be identified as more prone to memorization than others~\citep{feldman2020does, watson2021importance, ye2021enhanced}. Similarly, we provide a heuristic to identify the most memorized samples in our label inference test using confidence of the KNN prediction.
Given a periphery crop, $\crop{A_i}$, let $\KNN_A \big( \crop{A_i} \big) \subseteq \calX$ denote its $k$-nearest neighbors in the embedding space of $\SSL_A$. From this KNN subset we can obtain: \textbf{(1)} $\KNNprob_A \big( \crop{A_i} \big)$, the vector of class probabilities (normalized counts) induced by the KNN subset, and \textbf{(2)} $\KNNconf_A \big( \crop{A_i} \big)$, the negative entropy of the probability vector $\KNNprob_A \big( \crop{A_i} \big)$, as confidence of the KNN prediction. When entropy is low, the neighbors agree on the class of $A_i$ and hence confidence is high. 
% \begin{itemize}[noitemsep, leftmargin=*, topsep=0pt]
%     \item $\KNN_A \big( \crop{A_i} \big)$: The most prevalent class in the KNN subset as prediction for the class label $\cl(A_i)$. 
%     \item $\KNNprob_A \big( \crop{A_i} \big)$: The vector of class probabilities (normalized counts) induced by the KNN subset.
%     \item $\KNNconf_A \big( \crop{A_i} \big)$: Negative entropy of the probability vector $\KNNprob_A \big( \crop{A_i} \big)$ as confidence of the KNN prediction. When entropy is low, the neighbors agree on the class of $A_i$ and hence confidence is high. 
% \end{itemize}
We can sort the confidence score $\KNNconf_A \big( \crop{A_i} \big)$ across samples $A_i$ in decreasing order to identify the most confidently predicted samples, which likely correspond to the most memorized samples when $A_i \in \calA$.

\subsection{Population-level Memorization}
\label{sec:label inference accuracy}

%ORIGINAL FIGURE SETUP IN ARXIV: 
% \input{dejavu_training_epochs.tex}
% \input{dejavu_training_set_size.tex}
%PUT ORIGINAL FIGURES SIDE BY SIDE: 
% \input{dejavu_training_epochs_set_size.tex}
%PUT IN NEW FIGURES: 

\begin{wrapfigure}{r}{0.4\textwidth} 
    \centering
    \includegraphics[width=0.4\textwidth]{figures/dejavu_main.pdf}
    \caption{Accuracy of label inference using the target model (trained on $\calA$) vs. the reference model (trained on $\calB$) on the top $\%$ most confident examples $A_i \in \calA$ using only $\crop{A_i}$. For VICReg, there is a large accuracy gap between the two models, indicating a significant degree of \dejavu memorization.}
    \label{fig:dejavu main}
    \vspace{-2ex}
\end{wrapfigure}

Our first measure of \dejavu memorization is population-level label inference accuracy: \emph{What is the average label inference accuracy over a subset of SSL training images given their periphery crops?} 
To understand how much of this accuracy is due to $\SSL_A$'s \dejavu memorization, we compare with a correlation baseline using the reference model: $\KNN_B$'s label inference accuracy on images $A_i \in \calA$. 
In principle, this inference accuracy should be significantly above chance level ($1/1000$ for ImageNet) because the periphery crop may be highly indicative of the foreground object through correlation, \emph{e.g.}, if the periphery crop is a basketball player then the foreground object is likely a basketball.

Figure \ref{fig:dejavu main} compares the accuracy of $\KNN_A$ to that of $\KNN_B$ when inferring the labels of images in $A_i \in \calA$\footnote{The sets $\calA$ and $\calB$ are exchangeable, and in practice we repeat this test on images from $\calB$ using $\SSL_B$ as the target model and $\SSL_A$ as the reference model, and average the two sets of results.} using $\crop{A_i}$.
Results are shown for VICReg and the supervised model; trends for other models are shown in Appendix \ref{sec:appx simclr results}. For both VICReg and supervised models, inferring the class of $\crop{A_i}$ using $\KNN_B$ (dashed line) through correlation achieves a reasonable accuracy that is significantly above chance level. However, for VICReg, the inference accuracy using $\KNN_A$ (solid red line) is significantly higher, and the accuracy gap between $\KNN_A$ and $\KNN_B$ indicates the degree of \dejavu memorization. We highlight two observations: 
\begin{itemize}
    \item The accuracy gap of VICReg is significantly larger than that of the supervised model. This is especially notable when accounting for the fact that the supervised model is trained to associate randomly augmented crops of images with their ground truth labels. In contrast, VICReg has no label access during training but the embedding of a periphery crop can still encode the image label. 
    \item For VICReg, inference accuracy on the $1\%$ most confident examples is nearly $95\%$, which shows that our simple confidence heuristic can effectively identify the most memorized samples. This result suggests that an adversary can use this heuristic to identify vulnerable training samples to launch a more focused privacy attack.
\end{itemize}
\vspace{-.75em}
\paragraph{The \dejavu score. }
The curves of Figure \ref{fig:dejavu main} show memorization across confidence values for a single training scenario.  To study how memorization changes with different hyperparamters, we extract a single value from these curves: the \dejavu \emph{score} at confidence level $p$. In Figure \ref{fig:dejavu main}, this is the gap between the solid red (or gray) and dashed red (or gray) where confidence ($x$-axis) equal $p\%$. In other words, given the periphery crops of set $\calA$, $\KNN_A$ and $\KNN_B$ separately select and label their top $p\%$ most confident examples, and we report the difference in their accuracy. The \dejavu score captures both the degree of memorization by the accuracy gap and the \emph{ability to identify memorized examples} by the confidence level. If the score is 10\% for $p=33\%$, $\KNN_A$ has 10\% higher accuracy on its most confident third of $\calA$ than $\KNN_B$ does on its most confident third. In the following, we set $p = 20\%$, approximately the largest gap for VICReg (red lines) in Figure \ref{fig:dejavu main}. 
% Specifically, the \dejavu \emph{score} on the top $p\%$ most confident examples is,  
% \begin{equation}
%     \mathrm{DejaVu}(p) = \mathrm{Acc}_{\SSL_A}\big( \calA_{\SSL_A, p}  \big) - \mathrm{Acc}_{\SSL_B}\big( \calA_{\SSL_B, p}  \big) \ ,
%     \label{eqn:dejavu score}
% \end{equation}
% where $\calA_{\SSL_A, p}$
% Here we introduce a DejaVu memorization metric that quantify how much a target model is able to retrieve more class information from a crop than the reference model. We define it as:
% where $p$ is a function that take the $p$ purcent most confident samples.
%Figure \ref{fig:dejavu v. training epochs} shows how \dejavu memorization changes with the number of epochs used to train the embedding model (VICReg and supervised, respectively). The training set size is fixed to 300K samples, and label inference accuracy is computed on the top $20\%$ highest confidence examples. The number of epochs has a very strong influence on the degree of memorization for VICReg as the accuracy gap widens when number of epochs increases. We note that 1000 training epochs is used in several SSL works \citep{vicreg, simclr}. Remarkably, this trend in memorization is \emph{not} reflected in the standard metric for evaluating SSL representations: linear probe accuracy. The gray line in Figure \ref{fig:dejavu v. training epochs} shows the train-test accuracy gap of a linear classifier trained on top of the VICReg embeddings. Although there is a sizeable train-test gap, it does not grow significantly beyond 500 epochs. In contrast, \dejavu memorization (blue line) continues to worsen after 500 epochs. Thus, our test can be used as an alternative to linear probe accuracy to evaluate the memorization of SSL models.
% \vspace{-.75em}

% \paragraph{Comparison with the generalization gap} A network that perform very well on a training set while performing poorly on a test set (assuming the training set and test set sampled uniformly from the same distribution) is probably memorizing the training examples without being able to generalize on the test data. One could expect that measuring the difference in accuracy between the training and test set could give us insights on the degree of \dejavu memorization. However, we show in Figure  \ref{fig:dejavu v. training epochs} and \ref{fig:dejavu v. n} that this is not the case. In fact \dejavu memorization can significantly increase while the train-test gap decrease. In our experiments, we did not find a correlation between \dejavu and generalization.
\vspace{-0.75em}
\paragraph{Comparison with the linear probe train-test gap.} A standard method for measuring SSL performance is to train a linear classifier---what we call a `linear probe'---on its embeddings and compute its performance on a held out test set. From a learning theory standpoint, one might expect the linear probe's train-test accuracy gap to be indicative of memorization: the more a model overfits, the larger is the difference between train set and test set accuracy. However, as seen in Figure \ref{fig:dejavu epochs train set size}, the linear probe gap (dark blue) fails to reveal memorization captured by the \dejavu score (red) \footnote{See section \ref{sec:mitigation} for further discussion of the \dejavu score trends of Figure \ref{fig:dejavu epochs train set size}.}.

% \paragraph{Effect of training epochs.} 
% Figure \ref{fig:dejavu v. training epochs} shows how \dejavu memorization changes with training epochs for VICReg. The training set size is fixed to 300K samples. We observe that the number of epochs has a very strong influence on the degree of memorization for VICReg. From 250 to 1000 epochs, the \dejavu score (red curve) grows threefold: from under 10\% to over 30\%. Remarkably, this trend in memorization is \emph{not} reflected in the standard metric for evaluating SSL representations: linear probe accuracy. The dark blue curve shows the train-test linear probe accuracy gap. Although there is a sizeable train-test gap, it only changes by a few percent beyond 250 epochs. %Thus, our test can be used as an alternative to linear probe accuracy to evaluate the memorization of SSL models.
% \vspace{-.75em}
\begin{figure}[ht]
\label{fig:dejavu epochs and dataset}
\begin{minipage}[t]{0.49\textwidth}
\centering
     \begin{subfigure}[b]{0.48\textwidth}
         \centering
         \includegraphics[width=\textwidth]{figures/deja_vu_vs_epochs.png}
         \vspace{-1.5em}
         \caption{\dejavu vs. epochs}
         \label{fig:dejavu v. training epochs}
     \end{subfigure}
     \begin{subfigure}[b]{0.48\textwidth}
         \centering
         \includegraphics[width=\textwidth]{figures/deja_vu_vs_n.png}
         \vspace{-1.5em}
         \caption{\dejavu vs. train set size}
         \label{fig:dejavu v. n}
     \end{subfigure}~
     \vspace{-0.5em}
    \caption{
    Effect of training epochs and train set size with VICReg on \dejavu score (red) in comparison with linear probe accuracy train-test gap (dark blue). 
    \textbf{Left:} \dejavu score increases with training epochs, indicating growing memorization while the linear probe baseline decreases significantly.  
    \textbf{Right:} \dejavu score stays roughly constant with training set size suggesting that memorization may be problematic even for large datasets. %By comparison, the baseline \emph{declines} by half, spuriously suggesting less memorization. 
    %Both trends are not captured according to the linear probe train-test gap---a common method to evaluate generalization of SSL representations.}
    }
    \label{fig:dejavu epochs train set size}
\end{minipage}
\hfill
\begin{minipage}[t]{0.49\textwidth}
\centering
     \begin{subfigure}[b]{0.48\textwidth}
         \centering
         \includegraphics[width=\textwidth]{figures/vicreg_samples_epochs.pdf}
         \vspace{-1.5em}
         \caption{\dejavu vs. epochs}
         \label{fig:per sample v. training epochs}
     \end{subfigure}
     \begin{subfigure}[b]{0.48\textwidth}
         \centering
         \includegraphics[width=\textwidth]{figures/vicreg_samples_datasets.pdf}
         \vspace{-1.5em}
         \caption{\dejavu vs. train set size}
         \label{fig:per sample v. n}
     \end{subfigure}~
     \vspace{-0.5em}
    \caption{
    \definecolor{part_blue}{rgb}{0.2824, 0.4706, .8157}
	\definecolor{part_red}{rgb}{0.8392, 0.3725, 0.3725}
	\definecolor{part_orange}{rgb}{0.9333, 0.5216, 0.2902}
    Partition of samples $A_i \in \calA$ into the four categories: {\color{gray}unassociated} (not shown), {\color{part_orange}memorized}, {\color{part_red}misrepresented} and {\color{part_blue}correlated} for VICReg. The {\color{part_orange}memorized} samples---those whose labels are predicted by $\KNN_A$ but not by $\KNN_B$---occupy a significantly larger share of the training set than the {\color{part_red}misrepresented} samples---those predicted by $\KNN_B$ but not $\KNN_A$ by chance. %At 1000 epochs, $\approx 15\%$ of the training set is {\color{part_orange}memorized}. The trends across training epochs and training set sizes are consistent with those observed in Figure \ref{fig:dejavu epochs train set size}
    }
    \label{fig:partition attack main}
    \end{minipage}
\vspace{-1em} 
\end{figure}

\iffalse

\begin{minipage}[t]{0.49\textwidth}
\centering
     \begin{subfigure}[b]{0.48\textwidth}
         \centering
         \includegraphics[width=0.95\textwidth]{figures/deja_vu_vs_parameters.png}
         \vspace{-0.4em}
         \caption{\dejavu vs. capacity}
         \label{fig:dejavu v. capacity}
     \end{subfigure}
     \hfill
     \begin{subfigure}[b]{0.48\textwidth}
          \tiny
          \centering
          \setlength{\tabcolsep}{3pt}
          \begin{tabular}{|c|c|c|}
            \hline
            Criteria & DV & Acc P/B \\
            \hline
            Supervised & 8.9 & 55.3/61.1\\
            \hline
            Byol\citep{grill2020byol} & 8.0& 54.3/59.4\\
            \hline
            SimCLR\citep{chen2020simclr} & 10.0 & 44.2/54.1\\
            \hline
            Dino\citep{Dino} & 14.5 & 26.3/55.7 \\
            \hline
            Barlow T.\citep{zbontar2021barlow} & 30.5 & 33.7/54.4\\
            \hline
            VICReg\citep{vicreg} & \textbf{33.2} & 40.3/55.2\\
            \hline
          \end{tabular}
          \vspace{1.3em}
          % \caption{\dejavu (DV) vs. SSL Criterion}
          \caption{\dejavu (DV) vs. Criterion}
          \label{tab:dejavu vs. criterion}
    \end{subfigure}
    \vspace{-0.5em}
    \caption{
    Comparison of \dejavu score for different architectures and training criteria. \textbf{Left:} \dejavu score with VICReg for resnet (purple) and vision transformer (green) architectures versus number of model parameters. As expected, memorization grows with larger model capacity. This trend is more pronounced for convolutional (resnet) than transformer (ViT) architectures. \textbf{Right:} Comparison of \dejavu score and ImageNet validation accuracy (P: using projector embeddings, B: using backbone embeddings) for various SSL criteria. \textbf{Nearly all SSL models have more memorization than the supervised baseline.} 
    % Effect of training epochs and train set size on \dejavu score.
    % \textbf{Left:} \dejavu score increases with higher number of training epochs, indicating worsening memorization.
    % \textbf{Right:} \dejavu score stays roughly constant with training set size. Both trends are not captured according to the linear probe train-test gap---a common method to evaluate generalization of SSL representations.
    }
\end{minipage}
\vspace{-2em} 
\end{figure}

\begin{figure}[ht]
\begin{minipage}[t]{0.49\textwidth}
\centering
     \begin{subfigure}[b]{0.49\textwidth}
         \centering
         \includegraphics[width=\textwidth]{figures/epochs_lb_attk_epochs_acc_top1_legend.pdf}
         \caption{\dejavu vs. epochs}
         \label{fig:dejavu v. training epochs}
     \end{subfigure}
     \begin{subfigure}[b]{0.49\textwidth}
         \centering
         \includegraphics[width=\textwidth]{figures/epochs_lb_attk_datasets_acc_top1_legend.pdf}
         \caption{\dejavu vs. train set size}
         \label{fig:dejavu v. n}
     \end{subfigure}~
     \begin{subfigure}[b]{0.32\textwidth}
         \centering
         \includegraphics[width=0.8\textwidth]{figures/dejavu_vs_parameters.pdf}
         \caption{\dejavu vs. capacity}
         \label{fig:dejavu v. n}
     \end{subfigure}
    \caption{
    Effect of training epochs and train set size on \dejavu score.
    \textbf{Left:} \dejavu score increases with higher number of training epochs, indicating worsening memorization.
    \textbf{Right:} \dejavu score stays roughly constant with training set size. Both trends are not captured according to the linear probe train-test gap---a common method to evaluate generalization of SSL representations.}
    \end{minipage}
\vspace{-1em} 
\end{figure}

\begin{table}[ht]
  \footnotesize
  \centering
  \begin{tabular}{|c|c|}
    \hline
    Supervised & 8.9\\
    \hline
    SimCLR\citep{chen2020simclr} & 10.0\\
    \hline
    Byol\citep{grill2020byol} & 8.0\\
    \hline
    Dino\citep{Dino} & 14.5\\
    \hline
    Barlow T.\citep{zbontar2021barlow} & 30.5\\
    \hline
    VICReg\citep{vicreg} & \textbf{33.2}\\
    \hline
  \end{tabular}
  \caption{DejaVu Score 20\% Conf for various SSL methods.}
  \label{tab:two-row-table}
\end{table}
\vspace{-1em} 
\fi

\iffalse
\begin{figure}[ht]
\begin{minipage}[t]{.49\textwidth}
\centering
     \begin{subfigure}[b]{0.49\textwidth}
         \centering
         \includegraphics[width=\textwidth]{figures/epochs_lb_attk_epochs_acc_top1_legend.pdf}
         \caption{\dejavu vs. epochs}
         \label{fig:dejavu v. training epochs}
     \end{subfigure}
     \hfill
     \begin{subfigure}[b]{0.49\textwidth}
         \centering
         \includegraphics[width=\textwidth]{figures/epochs_lb_attk_datasets_acc_top1_legend.pdf}
         \caption{\dejavu vs. train set size}
         \label{fig:dejavu v. n}
     \end{subfigure}
\caption{
Effect of training epochs and train set size on \dejavu score.
\textbf{Left:} \dejavu score increases with higher number of training epochs, indicating worsening memorization.
\textbf{Right:} \dejavu score stays roughly constant with training set size. Both trends are not captured according to the linear probe train-test gap---a common method to evaluate generalization of SSL representations.}
\label{fig:dejavu epochs and dataset}
\end{minipage}
\hfill
\begin{minipage}[t]{.49\textwidth}
     \centering
     \begin{subfigure}[b]{0.49\textwidth}
         \centering
         \includegraphics[width=\textwidth]{figures/criteria_epochs.pdf}
         \caption{criteria comparison}
         \label{fig:dejavu v. criteria}
     \end{subfigure}
     \hfill
     \begin{subfigure}[b]{0.49\textwidth}
         \centering
         \includegraphics[width=\textwidth]{figures/architecture_epochs.pdf}
         \caption{architecture comparison}
         \label{fig:dejavu v. arch}
     \end{subfigure}
\caption{
Effect of SSL training criteria and model architectures on \dejavu score.
%the accuracy gap between target model (trained on $\calA$) and reference model (trained on $\calB$) making predictions on their 20\% most confident examples.
\textbf{Left:} \dejavu score for various training criteria.
%Barlow and VICReg have the heaviest degree of memorization, while SimCLR and BYOL have the least. 
%Note that we show detailed reconstructions of SimCLR's training data in Section \ref{sec:visualizing} despite its relatively low degree of \dejavu. 
%Regardless, Although SimCLR and BYOL have the least, we  visualize detailed reconstructions with SimCLR in section \ref{sec:mem v corr} 
All SSL models have significantly more \dejavu than the supervised baseline. \textbf{Right:} \dejavu score versus epochs for various training architectures. As expected, lower capacity architectures (Resnet18, Resnet34) reduce \dejavu but not completely. 
}
\label{fig:dejavu criteria and architecture}
\end{minipage}
\vspace{-1em} 
\end{figure}
\fi
% %\begin{figure}[ht]
%%%
%VICREG
%%%
     \centering
     \begin{subfigure}[b]{0.49\textwidth}
         \centering
         \includegraphics[width=\textwidth]{figures/sample_level_training_epochs.pdf}
         \caption{Categories of training samples vs. number of epochs}
         \label{fig:sample level epochs}
     \end{subfigure}
     \hfill
     \begin{subfigure}[b]{0.49\textwidth}
         \centering
         \includegraphics[width=\textwidth]{figures/sample_level_training_set_size.pdf}
         \caption{Categories of training samples vs. training set size}
         \label{fig:sample level training size}
     \end{subfigure}
\caption{
\definecolor{part_blue}{rgb}{0.2824, 0.4706, .8157}
\definecolor{part_red}{rgb}{0.8392, 0.3725, 0.3725}
\definecolor{part_orange}{rgb}{0.9333, 0.5216, 0.2902}
Partition of samples $A_i \in \calA$ into the four categories: {\color{gray}unassociated} (not shown), {\color{part_orange}memorized}, {\color{part_red}misrepresented} and {\color{part_blue}correlated}. The {\color{part_orange}memorized} samples---ones whose labels are predicted by $\KNN_A$ but not by $\KNN_B$---occupy a significantly larger share for VICReg compared to the supervised model, indicating that sample-level \dejavu memorization is more prevalent in VICReg. %The trends across number of training epochs and training set sizes are consistent with those observed in Figures \ref{fig:dejavu epochs and dataset} and \ref{fig:dejavu criteria and architecture}.
}
\label{fig:partition attack main appendix}
\end{figure}
% \paragraph{Effect of training set size.} 
% Figure \ref{fig:dejavu v. n} shows how \dejavu memorization responds to the model's training set size. The number of training epochs is fixed to 1000. Interestingly, training set size appears to have almost \emph{no} influence on the \dejavu score (red line), indicating that memorization is equally prevalent with a 100K dataset and a 500K dataset (which suggests that \dejavu memorization may be detectable for larger datasets). Meanwhile, the linear probe train-test accuracy gap \emph{declines} by half as the dataset size grows, failing to represent the memorization quantified by our test. 
% The trend is completely different according to linear probe accuracy (dark blue line), the train-test gap shrinks substantially when increasing the training set size from 100K to 500K. This highlights that the train-test gap is not able to capture \dejavu memorization. %Our evidence suggests that \dejavu memorization may be detectable even for large-scale training datasets. 
%\vspace{-.75em}

\vspace{-.75em} 
\subsection{Sample-level Memorization}
\label{sec:dissection}

% Section \ref{sec:label inference accuracy} shows the \emph{average} level of \dejavu memorization on a subset of the training set $\calA$. However, this average tell us only what the attacker success rate might be without explicitly describing how much of the datatset is \dejavu memorized.
The \dejavu score shows, \emph{on average}, how much better an adversary can select and classify images when using the target model trained on them. 
This average score does not tell us how many individual images have their label successfully recovered by $\KNN_A$ but not by $\KNN_B$. In other words, how many images are exposed by virtue of \emph{being in training set} $\calA$: a risk notion foundational to differential privacy. 
% However, from the perspective of an individual image $A_i \in \calA$, it is informative to know whether it was correctly classified 
To better quantify what fraction of the dataset is at risk, we perform a sample-level analysis by fixing a sample $A_i \in \calA$ and observing the label inference result of $\KNN_A$ vs. $\KNN_B$.
To this end, we partition samples $A_i \in \calA$ based on the result of label inference into four distinct categories: {\color{gray}\textbf{Unassociated}} - label inferred with neither KNN; {\color{part_orange}\textbf{Memorized}} - label inferred only with $\KNN_A$; {\color{part_red}\textbf{Misrepresented}} - label inferred only with $\KNN_B$; {\color{part_blue}\textbf{Correlated}} - label inferred with both KNNs. 
% \begin{multicols}{2}
% \begin{itemize}
%     \vspace{-.75em}
%     \setlength\itemsep{0.15em}
%     \item {\color{gray}Unassociated}: label inferred with neither KNN   
%     \item {\color{part_orange}Memorized}: label only inferred by $\KNN_A$
%     \item {\color{part_red}Misrepresented}: label only inferred with $\KNN_B$
%     \item {\color{part_blue}Correlated}: label inferred with both KNNs
%     \vspace{-.75em}
% \end{itemize}
% \end{multicols}
Intuitively, {\color{gray}unassociated} samples are ones where the embedding of $\crop{A_i}$ does not encode information about the label. {\color{part_blue}Correlated} samples are ones where the label can be inferred from $\crop{A_i}$ using correlation, \emph{e.g.}, inferring the foreground object is basketball given a crop showing a basketball player. Ideally, the {\color{part_red}misrepresented} set should be empty but contains a small portion of examples due to chance.
\emph{Déjà vu} memorization occurs for {\color{part_orange}memorized} samples where the embedding of $\SSL_B$ does not encode the label but the embedding of $\SSL_A$ does. To measure the pervasiveness of \dejavu memorization, we compare the size of the {\color{part_orange}memorized} and {\color{part_red}misrepresented} sets.
Figure \ref{fig:partition attack main} shows how the four categories of examples change with number of training epochs and training set size. The {\color{gray}unassociated} set is not shown since the total share adds up to one. The {\color{part_red}misrepresented} set remains under $5\%$ and roughly unchanged across all settings, consistent with our explanation that it is due to chance. In comparison, VICReg's {\color{part_orange}memorized} set surpasses $15\%$ at 1000 epochs. Considering that up to 5\% of these memorized examples could also be due to chance, we conclude that \textbf{at least 10\% of VICReg's training set is \dejavu memorized.} 
%is many times larger than its {\color{part_red}misrepresented} set, indicating substantial sample-level \dejavu memorization. 
%In fact, \textbf{it is 15\% of the training set that is \dejavu memorized with VICReg.}
%The trends across different number of training epochs and training set sizes match those observed in Section \ref{sec:label inference accuracy}. % On the other hand, the supervised model's {\color{part_orange}memorized} set is only marginally larger than its {\color{part_red}misrepresented} set.

% The trends across different number of training epochs and training set sizes match those observed in Section \ref{sec:label inference accuracy}: Increasing the number of epochs increases \dejavu memorization (Figure \ref{fig:per sample v. training epochs}), while increasing the training set size does not appear to reduce \dejavu memorization (Figure \ref{fig:per sample v. n}). 
%\section{Visualizing \emph{Déjà Vu} Memorization}
\label{sec:visualizing}
Beyond enabling label inference using a periphery crop, we show that \dejavu memorization allows the SSL model to encode other forms of information about a training image. Namely, we train an RCDM \citep{RCDM} on the public dataset $\calX$ and use it to visually reconstruct training images given their periphery crop.
We aim to answer the following two questions: \textbf{(1)} Can we visualize the distinction between correlation and \dejavu memorization? \textbf{(2)} What foreground object details can be extracted from the SSL model beyond class label? 
% \begin{enumerate}[noitemsep, leftmargin=*, topsep=0pt]
%     \item Can we visualize the distinction between correlation and \dejavu memorization? 
%     \item What foreground object details can be extracted from the SSL model beyond class label? 
% \end{enumerate}
\vspace{-0.5em}
\paragraph{Reconstruction pipeline.}
RCDM is a conditional generative model that is trained on the \emph{backbone embedding} of images $X_i \in \calX$ to generate an image that resembles $X_i$. All training images are first face-blurred for privacy purposes. \citet{RCDM} showed that the backbone embedding of SSL models contains more low-level information about the image, making them better suited for conditioning the RCDM.
At test time, following the pipeline in Figure \ref{fig:split_and_pipeline_cartoon}, we first use the projector embedding to find the KNN subset for the periphery crop, $\crop{A_i}$, and then average their backbone embeddings as input to the RCDM model. Ideally, when the public set contains enough representative images, the average representation of the KNN subset encodes objects present in $A_i$, and the RCDM model decodes this representation to visualize these objects.
% \begin{figure}[ht]
%%%
%VICREG
%%%
     \centering
     \begin{subfigure}[b]{0.49\textwidth}
         \centering
         \includegraphics[width=\textwidth]{figures/sample_level_training_epochs.pdf}
         \caption{Categories of training samples vs. number of epochs}
         \label{fig:sample level epochs}
     \end{subfigure}
     \hfill
     \begin{subfigure}[b]{0.49\textwidth}
         \centering
         \includegraphics[width=\textwidth]{figures/sample_level_training_set_size.pdf}
         \caption{Categories of training samples vs. training set size}
         \label{fig:sample level training size}
     \end{subfigure}
\caption{
\definecolor{part_blue}{rgb}{0.2824, 0.4706, .8157}
\definecolor{part_red}{rgb}{0.8392, 0.3725, 0.3725}
\definecolor{part_orange}{rgb}{0.9333, 0.5216, 0.2902}
Partition of samples $A_i \in \calA$ into the four categories: {\color{gray}unassociated} (not shown), {\color{part_orange}memorized}, {\color{part_red}misrepresented} and {\color{part_blue}correlated}. The {\color{part_orange}memorized} samples---ones whose labels are predicted by $\KNN_A$ but not by $\KNN_B$---occupy a significantly larger share for VICReg compared to the supervised model, indicating that sample-level \dejavu memorization is more prevalent in VICReg. %The trends across number of training epochs and training set sizes are consistent with those observed in Figures \ref{fig:dejavu epochs and dataset} and \ref{fig:dejavu criteria and architecture}.
}
\label{fig:partition attack main appendix}
\end{figure}
%\begin{figure*}[t!]
%%%
%DAM
%%%
     \centering
     \begin{subfigure}[b]{0.49\textwidth}
         \centering
         \includegraphics[width=\textwidth]{figures/dam_corr.png}
         \caption{A {\color{part_blue}correlated} dam example}
         \label{fig:dam correlated}
     \end{subfigure}
     \hfill
     \begin{subfigure}[b]{0.49\textwidth}
         \centering
         \includegraphics[width=\textwidth]{figures/dam_mem.png}
         \caption{A {\color{part_orange}memorized} dam example}
         \label{fig:dam memorized}
     \end{subfigure}
\caption{
{\color{part_blue}Correlated} and {\color{part_orange}Memorized} examples from the \emph{dam} class. Both $\SSL_A$ and $\SSL_B$ are SimCLR models.
\textbf{Left:} The periphery crop (pink square) contains a concrete structure that is often present in images of dams. Consequently, the trained RCDM can reconstruct the foreground object using representations from both $\SSL_A$ and $\SSL_B$ through this correlation.
\textbf{Right:} The periphery crop only contains a patch of water. The embedding produced by $\SSL_B$ only contains enough information to infer that the foreground object is related to water, as reflected by its KNN set and RCDM reconstruction. In contrast, the embedding produced by $\SSL_A$ memorizes the association of this patch of water with dam and the RCDM can visualize the embedding to produce images of dams.
}
\vspace{-1ex}
\label{fig:mem v corr dam}
\end{figure*}


\begin{figure*}[t!]
%%%
%DAM
%%%
     \centering
     \begin{subfigure}[b]{0.49\textwidth}
         \centering
         \includegraphics[width=\textwidth]{figures/dam_corr.png}
         \caption{A {\color{part_blue}correlated} dam example}
         \label{fig:dam correlated}
     \end{subfigure}
     \hfill
     \begin{subfigure}[b]{0.49\textwidth}
         \centering
         \includegraphics[width=\textwidth]{figures/dam_mem.png}
         \caption{A {\color{part_orange}memorized} dam example}
         \label{fig:dam memorized}
     \end{subfigure}
\caption[Correlated and Memorized examples from the \emph{dam} class.]{
Correlated and Memorized examples from the \emph{dam} class. Both $\SSL_A$ and $\SSL_B$ are SimCLR models.
\textbf{Left:} The periphery crop (pink square) contains a concrete structure that is often present in images of dams. Consequently, the trained RCDM can reconstruct the foreground object using representations from both $\SSL_A$ and $\SSL_B$ through this correlation.
\textbf{Right:} The periphery crop only contains a patch of water. The embedding produced by $\SSL_B$ only contains enough information to infer that the foreground object is related to water, as reflected by its KNN set and RCDM reconstruction. In contrast, the embedding produced by $\SSL_A$ memorizes the association of this patch of water with dam and the RCDM can visualize the embedding to produce images of dams.
}
\label{fig:mem v corr dam}
\end{figure*}


\begin{figure}[t!]
%%%
%BADGER
%%%
     \centering
     \begin{subfigure}[b]{0.49\textwidth}
         \centering
         \includegraphics[width=\textwidth]{figures/euro_badgers.png}
         \caption{{\color{part_orange}Memorized} European badgers}
         \label{fig:euro badgers}
     \end{subfigure}
     \hfill
     \begin{subfigure}[b]{0.49\textwidth}
         \centering
         \includegraphics[width=\textwidth]{figures/amer_badgers.png}
         \caption{{\color{part_orange}Memorized} American badgers}
         \label{fig:amer badgers}
     \end{subfigure}
\caption[Visualization of \dejavu memorization beyond class label.]{
Visualization of \dejavu memorization beyond class label. Both $\SSL_A$ and $\SSL_B$ are VICReg models. 
The four images shown belong to the memorized set of $\SSL_A$ from the \emph{badger} class. RCDM reconstruction using embeddings from $\SSL_A$ can reveal not only the correct class label, but also the specific badger species: \emph{European} (left) and \emph{American} (right). Such information does not appear to be memorized by the reference model $\SSL_B$.
} 
\label{fig:in class badger}
\end{figure}


% \subsection{Visualizing Correlation vs. Memorization}
\label{sec:mem v corr}
\vspace{-0.5em} 
\paragraph{Visualizing Correlation vs. Memorization.}
Figure \ref{fig:mem v corr dam} shows examples of dams from the {\color{part_blue}correlated} set (left) and the {\color{part_orange}memorized} set (right) as defined in Section \ref{sec:dissection}, along with the associated KNN set and RCDM reconstruction. Both $\SSL_A$ and $\SSL_B$ are SimCLR models. In Figure \ref{fig:dam correlated}, the periphery crop is represented by the pink square, which contains concrete structure attached to the dam's main structure. As a result, both $\SSL_A$ and $\SSL_B$ produce embeddings of $\crop{A_i}$ whose KNN set in $\calX$ consist of dams, \emph{i.e.}, there is a correlation between the concrete structure in $\crop{A_i}$ and the foreground dam. The RCDM reconstructions also consist of dams or structures that closely resemble dams. 
In Figure \ref{fig:dam memorized}, the periphery crop only contains a patch of water, which does not strongly correlate with dams in the ImageNet distribution. Evidently, the reference model $\SSL_B$ embeds $\crop{A_i}$ close to that of other objects commonly found in water, such as sea turtle and submarine. In contrast, the KNN set according to $\SSL_A$ all contain dams despite the vast number of alternative possibilities within the ImageNet classes, and the RCDM reconstruction outputs dams as well which highlight memorization in $\SSL_A$ between this specific patch of water and the dam. %\footnote{See Appendix \ref{sec:appx visualization} to see the same trend in the \emph{yellow garden spider} class.}


% \subsection{Visualizing Memorization Beyond Class Label}
% \label{sec:in class variation}
\vspace{-0.5em} 
\paragraph{Visualizing Memorization Beyond Class Label.}
We now use our reconstruction algorithm to show that \dejavu memorization can be exploited to reveal detailed information beyond class label. Figure \ref{fig:in class badger} shows four examples of badgers from the {\color{part_orange}memorized} set. In all four images, the periphery crop (pink square) does not contain any indication that the foreground object is a badger. Despite this, the KNN set and the RCDM reconstruction using $\SSL_A$ consistently produce images of badgers, while the same does not hold for $\SSL_B$.
More interestingly, reconstructions using $\SSL_A$ in Figure \ref{fig:euro badgers} all contain \emph{European} badgers, while reconstructions in Figure \ref{fig:amer badgers} all contain \emph{American} badgers, accurately reflecting the species of badger present in the respective training images. Since ImageNet-1K does \emph{not} differentiate between these two species of badgers, our reconstructions show that SSL models can memorize information that is highly specific to a training sample beyond its class label\footnote{See Appendix \ref{sec:appx visualization} for additional visualization experiments.}.%\footnote{See Appendix \ref{sec:appx visualization} for the same trend in the \emph{aircraft carrier} class.}.





%\vspace{-.5em} 
\section{Mitigation of \dejavu memorization}
\label{sec:mitigation}
% We do not have an understanding on why \dejavu occur so strongly in some SSL pretraining, however we present additional experiments that shed light on which parameters have the biggest impact on \dejavu memorization.
\begin{figure}[ht]
\label{fig:mitigations}
\begin{minipage}[t]{0.5\textwidth}
\centering
     \begin{subfigure}[b]{0.47\textwidth}
         \centering
         \includegraphics[width=\textwidth]{figures/dejavu_vicreg_param.png}
         \vspace{-1.5em}
         \caption{Loss hyper-parameter}
         \label{fig:dejavu v. invariance}
     \end{subfigure}
     \begin{subfigure}[b]{0.49\textwidth}
         \centering
         \includegraphics[width=\textwidth]{figures/deja_vu_vs_layer.png}
         \vspace{-1.5em}
         \caption{Guillotine regularization}
         \label{fig:dejavu v. guillotine}
     \end{subfigure}~
     \vspace{-0.5em}
    \caption[Effect of two kinds of hyper-parameters on VICReg memorization. ]{
    Effect of two kinds of hyper-parameters on VICReg memorization. \textbf{Left:} \dejavu score (red) versus the \emph{invariance} loss parameter, $\lambda$, used in the VICReg criterion (100k dataset). Larger $\lambda$ significantly reduces \dejavu, with minimal effect on linear probe validation performance (green). $\lambda = 25$ (near maximum \dejavu) is recommended in the original paper \textbf{Right:} \dejavu score versus projector layer---guillotine regularization \cite{Guillotine}---from projector to backbone. Removing the projector can significantly reduce \dejavu. Appendix \ref{sec:guillotine} shows that the backbone still can memorize, however; we demonstrate reconstructions using the SimCLR backbone.
    }
\end{minipage}
\hfill
\begin{minipage}[t]{0.48\textwidth}
\centering
     \begin{subfigure}[b]{0.46\textwidth}
         \centering
         \includegraphics[width=\textwidth]{figures/deja_vu_vs_parameters.png}
         \vspace{-1.3em}
         \caption{\dejavu vs. capacity}
         \label{fig:dejavu v. capacity}
     \end{subfigure}
     \hfill
     \begin{subfigure}[b]{0.52\textwidth}
          \tiny
          \centering
          \setlength{\tabcolsep}{3pt}
          \begin{tabular}{|c|c|c|}
            \hline
            Criteria & DV & Acc P/B \\
            \hline
            Supervised & 8.9 & 55.3/61.1\\
            \hline
            Byol\citep{grill2020byol} & 8.0& 54.3/59.4\\
            \hline
            SimCLR\citep{chen2020simclr} & 10.0 & 44.2/54.1\\
            \hline
            Dino\citep{Dino} & 14.5 & 26.3/55.7 \\
            \hline
            Barlow T.\citep{zbontar2021barlow} & 30.5 & 33.7/54.4\\
            \hline
            VICReg\citep{vicreg} & \textbf{33.2} & 40.3/55.2\\
            \hline
          \end{tabular}
          \vspace{1.3em}
          % \caption{\dejavu (DV) vs. SSL Criterion}
          \caption{\dejavu (DV) vs. Criterion}
          \label{tab:dejavu vs. criterion}
    \end{subfigure}
    \vspace{-1.4em}
    \caption[Effect of model architecture and criterion on \dejavu memorization.]{
    %Comparison of \dejavu score for different architectures and training criteria. 
    Effect of model architecture and criterion on \dejavu memorization. 
    \textbf{Left:} \dejavu score with VICReg for resnet (purple) and vision transformer (green) architectures versus number of model parameters. As expected, memorization grows with larger model capacity. This trend is more pronounced for convolutional (resnet) than transformer (ViT) architectures. \textbf{Right:} Comparison of \dejavu score 20\% conf. and ImageNet linear probe validation accuracy (P: using projector embeddings, B: using backbone embeddings) for various SSL criteria. %\textbf{Nearly all SSL models have more memorization than the supervised baseline.} 
    % Effect of training epochs and train set size on \dejavu score.
    % \textbf{Left:} \dejavu score increases with higher number of training epochs, indicating worsening memorization.
    % \textbf{Right:} \dejavu score stays roughly constant with training set size. Both trends are not captured according to the linear probe train-test gap---a common method to evaluate generalization of SSL representations.
    }
    \end{minipage}
\end{figure}
We cannot yet make claims on why \dejavu occurs so strongly for some SSL training settings and not for others. To gain some intuition for future work, we present additional observations that shed light on which parameters have the most salient impact on \dejavu memorization.
\vspace{-.75em}
\paragraph{Déjà vu memorization worsens by increasing number of training epochs.} 
Figure \ref{fig:dejavu v. training epochs} shows how \dejavu memorization changes with number of training epochs for VICReg. The training set size is fixed to 300K samples. From 250 to 1000 epochs, the \dejavu score (red curve) grows \emph{threefold}: from under 10\% to over 30\%. Remarkably, this trend in memorization is \emph{not} reflected by the linear probe gap (dark blue), which only changes by a few percent beyond 250 epochs. 

%\vspace{-.75em}
\paragraph{Training set size has minimal effect on \dejavu memorization.} Figure \ref{fig:dejavu v. n} shows how \dejavu memorization responds to the model's training set size. The number of training epochs is fixed to 1000. Interestingly, training set size appears to have almost \emph{no} influence on the \dejavu score (red line), indicating that memorization is equally prevalent with a 100K dataset and a 500K dataset. This result suggests that \dejavu memorization may be detectable even for large datasets. Meanwhile, the standard linear probe train-test accuracy gap \emph{declines} by more than half as the dataset size grows, failing to represent the memorization quantified by our test. 
% The trend is completely different according to linear probe accuracy (dark blue line), the train-test gap shrinks substantially when increasing the training set size from 100K to 500K. This highlights that the train-test gap is not able to capture \dejavu memorization. Our evidence suggests that \dejavu memorization may be detectable even for large-scale training datasets. 
\vspace{-0.5em}
\paragraph{Training loss hyper-parameter has a strong effect.} 
%We show in Figure \ref{fig:dejavu v. training epochs} that the number of training epochs is an important factor that can increase significantly \dejavu memorization. In contrast, the dataset size does not impact much \dejavu as shown in Figure \ref{fig:dejavu epochs train set size}. 
Loss hyper-parameters, like VICReg's invariance coefficient (Figure \ref{fig:dejavu v. invariance}) or SimCLR's temperature parameter (Appendix Figure \ref{fig:simclr temperature}) significantly impact \dejavu with minimal impact on the linear probe validation accuracy.

\vspace{-0.5em}
\paragraph{Some SSL criteria promote stronger \dejavu memorization.} Table \ref{tab:dejavu vs. criterion} demonstrates that the degree of memorization varies widely for different training criteria. VICReg and Barlow Twins have the highest \dejavu scores while SimCLR and Byol have the lowest.
%\footnote{We show detailed reconstructions of SimCLR's training data in Section \ref{sec:visualizing} despite its relatively low degree of \dejavu.}.
With the exception of Byol, all SSL models have more \dejavu memorization than the supervised model. Interestingly, different criteria can lead to similar linear probe validation accuracy and very different degrees of \dejavu as seen with SimCLR and Barlow Twins. Note that low degrees of \dejavu can still risk training image reconstruction, as exemplified by the SimCLR reconstructions in Figures \ref{fig:mem v corr dam} and \ref{fig:mem v corr spider}. 
%\vspace{-1em}
\vspace{-0.5em}
\paragraph{Larger models have increased \dejavu memorization.} Figure \ref{fig:dejavu v. capacity} validates the common intuition that lower capacity architectures (Resnet18/34) result in less memorization than their high capacity counterparts (Resnet50/101). 
% \begin{wrapfigure}{r}{0.25\textwidth} 
%     \centering
%     \includegraphics[width=0.25\textwidth]{figures/attk_layer_acc_top1_legend.pdf}
%     \caption{\dejavu memorization versus layer from backbone (0) to projector output (3).}
%     \label{fig:dejavu vs layer}
%     \vspace{-8ex}
% \end{wrapfigure}
We see the same trend for vision transformers as well. %This comes with a tradeoff, since reduced model capacity can result in a nontrivial degradation of representation quality\cite{vicreg, simclr}.  
\vspace{-0.5em}
\paragraph{Guillotine regularization can help reduce \dejavu memorization.} Previous experiments were done using the projector embedding. In Figure \ref{fig:dejavu v. guillotine}, we present how Guillotine regularization\citep{Guillotine} (removing final layers in a trained SSL model) impacts \dejavu with VICReg\footnote{Further experiments are available in Appendix \ref{sec:guillotine}.}. Using the backbone embedding instead of the projector embedding seems to be the most straightforward way to mitigate \dejavu memorization. However, as demonstrated in Appendix \ref{sec:appx backbone results}, backbone representation with low \dejavu score can still be leveraged to reconstruct some of the training images.

\section{Conclusion}
\label{sec:conclusion}

We defined and analyzed \dejavu memorization, a notion of unintended memorization of partial information in image data. As shown in Sections \ref{sec:quant} and \ref{sec:visualizing}, SSL models can largely exhibit \dejavu memorization on their training data, and this memorization signal can be extracted to infer or visualize image-specific information.
Since SSL models are becoming increasingly widespread as foundation models for image data, negative consequences of \dejavu memorization can have profound downstream impact and thus deserves further attention. 
Future work should focus on understanding how \dejavu emerges in the training of SSL models and why methods like Byol are much more robust to \dejavu than VICReg and Barlow Twins. In addition, trying to characterize which data points are the most at risk of \dejavu could be crucial to get a better understanding on this phenomenon. 

\graphicspath{{./chapters/chapter2/}}
%\newtheorem{thm}{Theorem}
%\newtheorem{lem}[thm]{Lemma}
%\DeclareMathOperator*{\argmax}{arg\,max}
%\DeclareMathOperator*{\argmin}{arg\,min}

\def\D{{\mathcal D}}
\def\Pphi{\overline{\Phi}}
\def\F{{\mathcal F}}
\def\N{{\mathcal N}}
%\def\R{{\mathbb R}}
\def\E{{\mathbb E}}
\def\A{\Pi}
\def\B{\Sigma}
\def\diam{\text{diam}}
\def\c{\mathcal L}
\def\l{\ell}
\def\seq{seq}
\def\R{\mathbb{R}}
\def\C{\mathcal C}
\def\p{p}
\def\s{size}
\def\L{\mathcal L}
\def\o{opt}
\def\H{\mathcal H}
\def\calH{\mathcal H}
\def\of{approxCluster}
\def\on{onlineCluster}
\def\R{\mathbb R}
\def\Y{\{\pm 1\}}
\def\U{\mathbb U}
\def\dd{\Delta}
\def\simp{{U\Delta}}
\def\g{g}
\def\rr{R}
\def\f{f}


\chapter{Sample Complexity of Robust Linear Classification on Separated Data} 

\section{Introduction}

Motivated by the use of machine learning in safety-critical settings, adversarially robust classification has been of much recent interest. Formally, the problem is as follows. A learner is given training data drawn from an underlying distribution $D$, a hypothesis class $\calH$, a robustness metric $d$, and a radius $r$. The learner's goal is to find a classifier $h \in \calH$ which has the lowest robust loss at radius $r$. The robust loss of a classifier is the expected fraction of examples where either $f(x) \neq y$ or where there exists an $x'$ at distance $d(x, x') \leq r$ such that $f(x) \neq f(x')$.  Robust classification thus aims to find a classifier that maximizes accuracy on examples that are distance $r$ or more from the decision boundary, where distances are measured according to the metric $d$.


In this work, we ask: how many samples are needed to learn a classifier with low robust loss when $\calH$ is the class of linear classifiers, and $d$ is an $\ell_p$-metric? Prior work has provided both upper~\cite{bartlett19, ravikumar20} as well as lower bounds~\cite{Schmidt18, ravikumar20} on the sample complexity of the problem. However, almost all look at settings where the data distribution itself is not separated --  data from different classes overlap or are close together in space. In this case, the classifier that minimizes robust loss is quite different from the one that minimizes error, which often leads to strong sample complexity gaps. Many real tasks where robust solutions are desired however tend to involve well-separated data~\cite{Yang20}, and hence it is instructive to look at what happens in these cases.

With this motivation, we consider in this work robust classification of data that is linearly $r$-separable. Specifically, there exists a linear classifier which has zero robust loss at robustness radius $r$. This case is thus the analog of the realizable case for robust classification, and we consider both upper and lower bounds in this setting.

For lower bounds, prior work \cite{Cullina18} shows that both standard and robust linear classification have VC-dimension $O(d)$, and consequently have similar bounds on the expected loss in the worst case. However, these results do not apply to this setting since we are specifically considering well-separated data, which greatly restricts the set of possible worst-case distributions.  For our lower bound, we provide a family of distributions that are linearly $r$-separable and where the maximum margin classifier, given $n$ independent samples, has error $O(1/n)$. In contrast, any algorithm for finding the minimum robust loss classifier has robust loss at least $\Omega(d/n)$, where $d$ is the data dimension. These bounds hold for all $\ell_p$-norms provided $p > 1$, including $p=2$ and $p=\infty$. Unlike prior work, our bounds do not rely on the difference in loss between the solutions with optimal robust loss and error, and hence cannot be obtained by prior techniques. Instead, we introduce a new geometric construction that exploits the fact that learning a classifier with low robust loss when data is linearly $r$-separated requires seeing a certain number of samples close to the margin.

For upper bounds, prior work \cite{bartlett19} provides a bound on the Rademacher complexity of adversarially robust learning, and show that it can be worse than the standard Rademacher complexity by a factor of $d^{1/q}$ for $\ell_p$-norm robustness where $1/p + 1/q = 1$. Thus, an interesting question is whether dimension-independent bounds, such as those for the accuracy under large margin classification, can be obtained for robust classification as well. Perhaps surprisingly, we show that when data is really well-separated, the answer is yes. Specifically, if the data distribution is linearly $r + \gamma$-separable, then there exists an algorithm that will find a classifier with robust loss $O(\Delta^2/\gamma^2 n)$ at radius $r$ where $\Delta$ is the diameter of the instance space. Observe that much like the usual sample complexity results on SVM and perceptron, this upper bound is independent of the data dimension and depends only on the excess margin (over $r$). This establishes that when data is really well-separated, finding robust linear classifiers does not require a very large number of samples. 

While the main focus of this work is on linear classifiers, we also show how to generalize our upper bounds to Kernel Classification, where we find a similar dynamic with the loss being governed by the excess margin in the embedded kernel space. However, we defer a thorough investigation of robust kernel classification as an avenue for future work.

Our results imply that while adversarially robust classification may be more challenging than simply accurate classification when the classes overlap, the story is different when data is well-separated. Specifically, when data is linearly (exactly) $r$-separable, finding an $r$-separated solution to robust loss $\epsilon$ may require $\Omega(d/\epsilon)$ samples for some distribution families where finding an accurate solution is easier. Thus in this case, there is a gap between the sample complexities of robust and simply accurate solutions, and this is true regardless of the $\ell_p$ norm in which robustness is measured. In contrast, if data is even more separated -- linearly $r + \gamma$-separable --  then we can obtain a dimension-independent upper bound on the sample complexity, much like the sample complexity of SVMs and perceptron. Thus, how separable the data is matters for adversarially robust classification, and future works in the area should consider separability while discussing the sample complexity.

\subsection{Related Work}

There is a large body of work \cite{Carlini17, Liu17, Papernot17, Papernot16, Szegedy14, Hein17, Katz17, Wu16,Steinhardt18, Sinha18} empirically studying adversarial examples primarily in the context of neural networks. Several works \cite{Schmidt18, Raghunathan20, Tsipras19} have empirically investigated trade-offs between robust and standard classification.

On the theoretical side, this phenomenon has been studied in both the parametric and non-parametric settings. On the parametric side, several works \cite{loh18, attias19, Srebro19, bartlett19, pathak20} have focused on finding distribution agnostic bounds of the sample complexity for robust classification. In \cite{Srebro19}, Srebro et. al. showed through an example that the VC dimension of robust learning may be much larger than standard or accurate learning indicating that the sample complexity bounds may be higher. However, their example did not apply to linear classifiers. 

\cite{Kane20} considers learning linear classifiers robustly, but is primarily focused on computational complexity as opposed to sample complexity.

In \cite{bartlett19}, Bartlett et. al. investigated the Rademacher complexity of robustly learning linear classifiers as well as neural networks. They showed that in both cases, the robust Rademacher complexity can be bounded in terms of the dimension of the input space -- thus indicating a possible gap between standard and robust learning. However, as with the works considering VC dimension, this work is fundamentally focused on upper bounds  -- they do not show true lower bounds on data requirements.

Because of it's simplicity and elegance, the case where the data distribution is a mixture of Gaussians has been particularly well-studied. The first such work was \cite{Schmidt18}, in which Schmidt et. al. showed an $\Omega(\sqrt{d})$ gap between the standard and robust sample complexity for a mixture of two Gaussians using the $\ell_\infty$ norm. This was subsequently expanded upon in \cite{Bhagoji19}, \cite{robey20} and  \cite{ravikumar20}. \cite{Bhagoji19} introduces a notion of ``optimal transport," which they subsequently apply to the Gaussian case, deriving a closed form expression for the optimally robust linear classifier. Their results apply to any $\ell_p$ norm. \cite{robey20} applies expands upon \cite{Schmidt18} by consider mixtures of three Gaussians in both the $\ell_2$ and $\ell_\infty$ norms. Finally, \cite{ravikumar20} fully generalizes the results of \cite{Schmidt18} providing tight upper and lower bounds on the standard and robust sample complexities of a mixture of two Gaussians, in any norm (including $\ell_p$ for $p \in [1, \infty]$). \cite{Schmidt18} and \cite{ravikumar20} bear the most relevance with our work, and we consequently carefully compare our results in section \ref{sec:comparison}.

Another approach for lower and upper bounds on sample complexities for linear classifiers can be found in \cite{Cullina18}, which examines the robust VC dimension of learning linear classifiers. They show that the VC dimension is $d+1$, just as it is in the standard case. This implies that the bounds in the robust case match the bounds in the standard case and in particular shows a lower bound of $\Omega(d/n)$ on the expected loss of learning a robust linear classifier from $n$ samples.

While this result appears to match our lower bound, there is a crucial distinction between the bounds. Our bound implies that there exists some distribution with a large $\ell_2$ margin for which the expected robust loss must be $\Omega(d/n)$. On the other hand, standard results about learning linear classifiers on large margin data implies that the expected standard loss will be $O(1/n)$ (when running the max-margin algorithm). For this reason, our paper provides a case in the well-separated setting in which learning linear classifiers is provably more difficult (in terms of sample complexity) in the robust setting than in the standard setting. By contrast, \cite{Cullina18} does not show this. Their paper only implies (through standard VC constructions) the existence of \textit{some} distribution that is difficult to learn, and the standard PAC bounds cannot ensure that such a distribution also has a large $\ell_2$ margin.

In the non-parametric setting, there are several works which contrast standard learning with robust learning. \cite{WJC18} considers the nearest neighbors algorithm, and shows how to adapt it for converging towards a robust classifier. In \cite{YRWC19}, Yang et. al. propose the $r$\textit{-optimal classifier}, which is the robust analog of the Bayes optimal classifier. Through several examples they show that it is often a fundamentally different classifier - which can lead to different convergence behavior in the standard and robust settings. \cite{Bhattacharjee20} unified these approaches by specifying conditions under which non-parametric algorithms can be adapted to converge towards the $r$-optimal classifier, thus introducing $r$-consistency, the robust analog of consistency.

\section{Preliminaries}
We consider binary classification over $\R^d \times \Y$. Our metric of choice is the $\ell_p$ norm, where $p > 1$ (including $p = \infty$) is arbitrary. For $x \in \R^d$, we will use $||x||_p$ to denote the $\ell_p$ norm of $x$, and consequently will use $||x - y ||_p$ to denote the $\ell_p$ distance between $x$ and $y$. We will also let $\ell_q$ denote the dual norm to $\ell_p$ - that is, $\frac{1}{q} + \frac{1}{p}= 1$.

 We use $B_p(x,r)$ to denote the closed $\ell_p$ ball with center $x$ and radius $r$. For any $S \subset \R^d$, we let $diam_p(S)$ denote its diameter: that is, $diam_p(S) = \sup_{x, y \in S} ||x - y||_p.$

\subsection{Standard and Robust Loss}

In classical statistical learning, the goal is to learn an accurate classifier, which is defined as follows:

\begin{defn}
Let $\D$ be a distribution over $\R^d \times \Y$, and let $f \in \Y^{\R^d}$ be a classifier. Then the \textbf{standard loss} of $f$ over $\D$, denoted $\L(f, \D)$, is the fraction of examples $(x,y) \sim \D$ for which $f$ is not accurate. Thus $$\L(f, \D) = P_{(x,y) \sim \D}[f(x) \neq y].$$
\end{defn}

Next, we define robustness, and the corresponding robust loss.

\begin{defn}
A classifier $f \in \Y^{\R^d}$ is said to be \textbf{robust} at $x$ with radius $r$ if $f(x) = f(x')$ for all $x' \in B_p(x,r)$. 
\end{defn}


\begin{defn}
The \textbf{robust loss} of $f$ over $\D$, denoted $\L_r(f, \D)$, is the fraction of examples $(x,y) \sim \D$ for which $f$ is either inaccurate at $(x,y)$, or $f$ is not robust at $(x,y)$ with radius $r$. Observe that this occurs if and only if there is some $x' \in B_p(x,r)$ such that $f(x') \neq y$. Thus $$\L_r(f, \D) = P_{(x, y) \sim \D}[\exists x' \in B_p(x,r)\text{ s.t. }f(x') \neq y].$$ 
\end{defn}

\subsection{Expected Loss and  Sample Complexity}

The most common way to characterize the performance of a learning algorithm is through an $(\epsilon, \delta)$ guarantee, which computes $\epsilon_n, \delta_n$ such that an algorithm trained over $n$ samples has loss at most $\epsilon_n$ with probability at least $1 - \delta_n$. 

In this work, we use the simpler notion of \textit{expected loss}, which is defined as follows: 

\begin{defn}
Let $A$ be a learning algorithm and let $\D$ be a distribution over $\R^d \times \{\pm 1\}$. For any $S \sim \D^n$, we let $A_S$ denote the classifier learned by $A$ from training data $S$. Then the \textbf{expected standard loss} of $A$ with respect to $\D$, denoted $EL^n(A, \D)$ where $n$ is the number of training samples, is defined as $$E\L^n(A, \D) = \E_{S \sim \D^n} \L(A_S, \D).$$ Similarly, we define the \textbf{expected robust loss} of $A$ with respect to $\D$ as $$E\L_r^n(A , \D) = \E_{S \sim \D^n}\L_r(A_S, \D).$$ 
\end{defn}

Our main motivation for using this criteria is simplicity. Our primary goal is to compare and contrast the performances of algorithms in the standard and robust cases, and this contrast clearest when the performances are summarized as a single number (namely the expected loss) rather than an $(\epsilon, \delta)$ pair. 

Next, we address the notion of sample complexity. As above, sample complexity is typically defined as the minimum number of samples needed to guarantee $(\epsilon, \delta)$ performance. In this work, we will instead define it solely with respect to $\epsilon$, the expected loss. 

\begin{defn}
Let $\D$ be a distribution over $\R^d \times \{\pm 1\}$ and $A$ be a learning algorithm. Then the \textbf{standard sample complexity} of $A$ with respect to $\D$, denoted $m^\epsilon(A, \D)$, is the minimum number of training samples needed such that $A$ has  expected standard loss at most $\epsilon$. Formally, $$m^\epsilon(A, \D) = \min(\{n: E\L^n(A, D) \leq \epsilon\}).$$ Similarly, we can define the \textbf{robust sample complexity} as $$m_r^\epsilon(A, \D) = \min(\{n: E\L^n(A, D) \leq \epsilon\}).$$
\end{defn}

\subsection{Linear classifiers}

In this work, we consider linear classifiers, formally defined as follows:
\begin{defn}
Let $w \in \R^d$ be a vector. Then the \textbf{linear classifier} with parameters $w \in \R^d$ and $b \in \R$ over $\R^d \times {\pm 1}$, denoted $f_{w, b}$, is defined as , $$f_{w,b}(x) = \begin{cases} +1 & \langle w, x \rangle  \geq b \\ -1 & \langle w, x \rangle < b  \end{cases}.$$ 
\end{defn}

Learning linear classifiers is well understood in the standard classification setting. We now consider the linearly \textit{separable} case, in which some linear classifier has perfect accuracy. We will later define linear $r$-separability as the robust analog of separability.

\begin{defn}
A distribution $\D$ over $\R^d \times Y$ is \textbf{linearly separable} if its support can be partitioned into sets $S^+$ and $S^-$ such that:

1. $S^+$ and $S^-$ correspond to the positively and negatively labeled subsets of $\R^d$. In particular, $P_{(x,y) \sim \D}[x \in S^y] = 1.$

2. There exists a linear classifier, $f_{w, b}$, that has perfect accuracy. That is, $\L(f_{w, b}, \D) = 0$. 
\end{defn}

The standard sample complexity for linearly separable distributions can be characterized through their margin, which is defined as follows.

\begin{defn}\label{defn:margin}
Let $\D$ be a linearly separable distribution over $\R^d \times \{\pm 1\}$. Let $S^+$ and $S^-$ be as above. Then $\D$ has \textbf{margin} $\gamma$ if $\gamma$ is the largest real number such that there exists a linear classifier $f_{w,b}$ with the following properties:

1. $f_{w,b}$ has perfect accuracy. That is, $\L(f_{w,b}, \D) = 0$.

2. Let $H_{w,b} = \{x: \langle x, w \rangle = b\}$ denote the decision boundary of $f_{w,b}$. Then for all $x \in (S^+ \cup S^-)$, $x$ has $\ell_2$ distance at least $\gamma$ from $H_{w,b}$. That is, $$\inf_{x \in S^+ \cup S^-, z \in H_{w,b}} ||x - z||_2 \geq \gamma.$$ We let $\gamma(\D)$ denote the margin of $\D$.
\end{defn}

Observe that although we use a general norm, $\ell_p$, to measure robustness, the margin is always measured in $\ell_2$. This is because the $\ell_2$ norm plays a fundamental role in  bounding the number of samples needed to learn a linear classifier. 

The basic idea is that when the $\ell_2$ margin is large relative to the $\ell_2$ diameter of the distribution, the max margin algorithm requires fewer samples needed to learn a linear classifier. In particular, the ratio between the $\ell_2$ margin and the $\ell_2$ diameter fully characterizes the standard sample complexity of the max margin algorithm. To further simplify our notation, we define this ratio as the aspect ratio.

\begin{defn}\label{defn:aspect_ratio}
Let $\D$ be a linearly separable distribution over $\R^d \times \{\pm 1\}$. Then the \textbf{aspect ratio} of $\D$, $\rho(\D)$ is defined as, $$\rho(\D) = \frac{diam_2(S^+ \cup S^-)}{\gamma(\D)},$$ where $\diam_2(S^+ \cup S^-)$ denotes its diameter in the $\ell_2$ norm.
\end{defn}

We now have the following well-known result, which characterizes the expected standard loss with the aspect ratio.
\begin{thm}\label{thm:standard}
\emph{(Chapter 10 in \cite{vapnik1998})} Let $M$ denote the hard margin SVM algorithm. If $\D$ is a distribution with aspect ratio $\rho = \rho(\D)$, then for any $n > 0$ we have $\E_{S \sim \D^n}\L(M_S, \D) \leq O(\frac{\rho^2}{n}),$ where $M_S$ denotes the classifier learned by $M$ from training data $S$. 
\end{thm}

We can also express this result in terms of standard sample complexity.
\begin{cor}\label{cor:standard}
Let $M$ denote the hard margin SVM algorithm. If $\D$ is a distribution with aspect ratio $\rho = \rho(\D)$, then for any $\epsilon > 0$ we have $m^\epsilon(M_S, \D) \leq O(\frac{\rho^2}{\epsilon}),$ where $M_S$ denotes the classifier learned by $M$ from training data $S$. 
\end{cor}

Theorem \ref{thm:standard} and Corollary \ref{cor:standard} will serve as a benchmark for comparison with the robust sample complexity.  
\subsection{Linear $r$-separability}

Finally, we introduce linear $r$-separability, which is the key characteristic of distributions considered in this paper. This can be thought of as the robust analog of linear separability.
\begin{defn}\label{defn:r_separability}
For any $r > 0$, a distribution $\D$ over $\R^d \times \{\pm 1\}$ is \textbf{linearly} $r$-\textbf{separable} if there exists a linear classifier $f_{w, b}$ such that $\L_r(f_{w, b}, \D) = 0$.
\end{defn}
This definition is the fundamental property considered in this paper. Our goal is to understand the sample complexity required for learning robust linear classifiers on linearly $r$-separable distributions, and compare it with the standard sample complexity given in Theorem \ref{thm:standard}.

\section{Lower Bounds}\label{sec:lower_bounds}

In this section, we consider $r$-separated distributions whose aspect ratio is constant. By Theorem \ref{thm:standard}, the standard sample complexity for learning them is independent of $d$. We will show that in contrast, the robust sample complexity has a linear dependence on $d$, and consequently establish a substantial gap between the standard and robust cases.

We begin by defining the family of such distributions.
\begin{defn}
For any $\rho, r$, the set $\F_{r, \rho}$ is defined as the set of all distributions $\D$ over $\R^d \times \{\pm 1\}$ such that $\D$ is $r$-separated and has aspect ratio at most $\rho$.
\end{defn} 

We now state our main result.
\begin{thm}\label{thm:lower}
Let $r > 0$ and $\rho > 20$. Then the following hold.
\begin{enumerate}
	\item For every learning algorithm $A$, and any $n > 0$, there exists $\D \in \F_{r, \rho}$ such that the expected robust loss when $A$ is trained on a sample of size $n$ from $\D$ is at least $\Omega(\frac{d}{n})$. Formally, there exists a constant $c > 0$ such that $\E_{S \sim \D^n}[\L_r(A_S, \D)] \geq \frac{cd}{n}.$
	\item  In contrast, by Theorem \ref{thm:standard}, for \textit{any} $\D \in \F_{r, D}$, the max margin algorithm has expected standard loss $O(\frac{\rho^2}{n})$, when trained on a sample of size $n$ from $\D$. Formally, there exists a constant $c' > 0$ such that $\E_{S \sim \D^n}[\L(A_S, \D)] \leq \frac{c'\rho^2}{n}.$
\end{enumerate}
\end{thm}
The condition $\rho > 20$ is required to rule out degenerate cases. This is because for small values of $\rho$, the $\ell_2$ diameter of $\D$ is not much larger than the $\ell_2$ margin of $\D$. This forces $\D$ to be mostly clustered around a line which leads to more complicated behavior.

Observe that when $\rho$ is a constant independent of $d$, the expected  standard loss is $O(\frac{1}{n})$ while the expected robust loss is $\Omega(\frac{d}{n})$. Thus, the ratio between the expected robust loss and the expected standard loss is $\Omega(d)$, leading to a dimensional dependent gap between the robust and standard cases. 

We also note that these bounds hold regardless of which $\ell_p$ ($p \in (1, \infty])$ norm is being used. This is because our construction of $\D \in \F_{r, \rho}$ for which the lower bound holds is given in terms of the norm $p$. More generally, the family $\F_{r, \rho}$ is implicitly defined with respect to $p$.

Furthermore, our lower bound differs from the lower bound of $\Omega(\frac{d}{n})$ shown in prior work \cite{Cullina18} because it specifically holds for $\F_{r, \rho}$, a linearly $r$-separated family of distributions with constant aspect ratio. Thus, while \cite{Cullina18} has shown the existence of distributions satisfying the first condition of Theorem \ref{thm:lower}, our result is the first to exhibit a distribution satisfying both conditions.


Finally, we note that Theorem \ref{thm:lower} can also be expressed in terms of sample complexities. We include this in the following corollary.

\begin{cor}
Let $r > 0$ and $\rho > 20$. Then the following hold.

1. For every learning algorithm $A$, and any $\epsilon > 0$, there exists $\D \in \F_{r, \rho}$ such that the robust sample complexity of $A$ with respect to $\D$ is at least $\Omega(\frac{d}{\epsilon})$. Formally, there exists a constant $c > 0$ such that $m_r^\epsilon(A, \D) \geq \frac{cd}{\epsilon}.$

2. In contrast, by Theorem \ref{thm:standard}, for \textit{any} $\D \in \F_{r, D}$, the max margin algorithm has standard sample complexity $O(\frac{\rho^2}{\epsilon})$. Formally, there exists a constant $c' > 0$ such that $m^\epsilon(A, \D) \leq \frac{c'\rho^2}{\epsilon}.$

\end{cor}

\subsection{Comparison with \cite{ravikumar20} and~\cite{Schmidt18}}\label{sec:comparison}

The first work to provide a robust sample complexity lower bound that applied to linear classifiers is~\cite{Schmidt18}; they showed a gap of $\Omega(\sqrt{d})$ between the robust and accuracy loss for a specific mixture of two Gaussians. This was later generalized to mixtures of any two Gaussians by~\cite{ravikumar20}, who also established more general lower bounds for any $\ell_p$ norm. Since \cite{ravikumar20} is a strict generalization of \cite{Schmidt18}, we next explain how our lower bounds differ from~\cite{ravikumar20}, and why their techniques do not lead to our results. We begin by summarizing their results.

\paragraph{Summary of \cite{ravikumar20}} \cite{ravikumar20} considers data distributions $\D$ that are parametrized by $\mu \in \R^d$ and $\Sigma \in \R^{d \times d}$, $\Sigma \succcurlyeq 0$. $\D_{\mu, \Sigma}$ is the mixture of two Gaussians, $\N(\mu, \Sigma)$ and $\N(-\mu, \Sigma)$, with equal mass, where instances drawn from $\N(\mu, \Sigma)$ are labeled as $+$, and instances drawn from $\N(-\mu, \Sigma)$ are labeled as $-$. They consider robustness measured in any normed metric in $\R^d$, including the $\ell_p$ norm for $p \in (1, \infty]$. Although their bounds apply to any classifier, this effectively deals with linear classifiers since it can be shown that the optimally robust and accurate classifiers are both linear.

For any distribution $\D_{\mu, \Sigma}$, let $L_{rob}$ denote the optimal robust loss of any classifier on $\D_{\mu,\Sigma}$, and let $L_{std}$ denote the optimal standard loss. Then the bounds shown in \cite{ravikumar20} can restated as follows (a detailed derivation from \cite{ravikumar20} appears in Appendix \ref{sec:appendix_comparison}). 

\begin{thm}\label{thm:ravikumar}
\cite{ravikumar20}
\begin{enumerate}
	\item For any learning algorithm $A$ and any $n > 0$, there exists some mixture of Gaussians, $\D_{\mu, \Sigma}$ such that the expected \textit{excess} robust loss is at least $\Omega(L_{rob}\frac{d}{n}),$ when $A$ is trained on a sample of size $n$ from $\D$. 
	\item For any distribution $\D_{\mu, \Sigma}$, it is possible to learn a classifier with expected \textit{excess} standard loss at most $O(L_{std}\frac{d}{n})$.
	\item By (1.) and (2.), the ratio between the expected excess loss and expected excess standard loss can be expressed as $ratio \geq \Omega(\frac{L_{rob}}{L_{std}}).$
\end{enumerate}
\end{thm}

Observe that their bounds are given through \textit{excess} losses, which is the amount by which the loss exceeds to the optimal loss. This is necessary because in their setting, the optimal classifiers do not have $0$ loss. 

\paragraph{Comparison with our bounds} Recall that in our work, we are concerned with the \textit{linearly $r$-separated case}, which occurs precisely when the optimal robust and standard losses both equal $0$. However, from Theorem \ref{thm:ravikumar}, we see that although \cite{ravikumar20} proves a gap between standard and robust sample complexity, this gap is predicated on distributions for which the optimal robust loss, $L_{rob}$ and optimal standard loss, $L_{std}$ differ. Furthermore, in the case where they obtain a gap of $\Omega(d)$, we see that this requires $\frac{L_{rob}}{L_{std}} = \Omega(d)$ which is a substantial difference. By contrast, our results characterize a gap exclusively in the case that this does not occur. 

Finally, in the limiting case where the Gaussians they consider are sufficiently far apart, their data will begin to appear linearly $r$-separated, meaning both $L_{rob}$ and $L_{std}$ are close to $0$. However, even in this case, it can be shown that the ratio $\frac{L_{rob}}{L_{std}}$ diverges towards infinity, meaning that their lower bound characterizes a very different dynamic from ours. Precise details on this comparison can be found in appendix \ref{sec:appendix_comparison}.

\subsection{Intuition behind Theorem \ref{thm:lower}}

The proof idea for Theorem \ref{thm:lower} can be summarized with a simple example (Figure \ref{fig:small_margin_robust}). In this example, we seek to learn a linear classifier for a linearly $r$-separated distribution in $\R^2$. The key idea is to contrast the necessary conditions for learning a robust classifier, and the necessary conditions for learning an accurate classifier. 

Observe that the distribution is \textit{precisely} linearly $r$-separated, that is, it is not possible to achieve robustness for radii larger than $r$. Because of this, there is a unique linear classifier $f_{rob}$ that has perfect robustness. In order to learn this classifier, we must see examples from $S^+ \cup S^-$ that are close to the ``boundary" of $S^+ \cup S^-$. In our figure, this consists of points that are close to the dotted blue and red lines. Moreover, it can be shown that the number of such examples we must see is related to $d$, the dimension.

By contrast, any classifier that separates $S^+$ from $S^-$ has perfect accuracy (take for example $f_{std}$ shown in the figure). It is possible to exploit this by using margin based algorithms for learning linear classifiers. In particular, we no longer need to see points that are extremely close to the boundary of $S^+ \cup S^-$.

\begin{figure}[h]
\centering
\vspace{.3in}
\includegraphics[scale=0.55]{small_margin_robust}
\vspace{.3in}
\caption{An example of a linearly $r$-separated distribution, with positively and negatively labeled examples in $S^+$ and $S^-$ respectively. The optimally robust classifier, $f_{rob}$ is shown in purple, while the (not necessarily unique) optimally accurate classifier, $f_{std}$, is shown in green.}
\label{fig:small_margin_robust}
\end{figure}

\paragraph{General Hypothesis Classes:} We now briefly consider how to extend our methods to other hypothesis classes. For any hypothesis class $\H$ and distribution $\D$ let $$\H_{\D, \alpha} = \{h: h \in \H,\L(h, \D) \leq \alpha\}$$ and let $$\H_{\D, \alpha}^r = \{h: h \in \H,\L_r(h, \D) \leq \alpha\}.$$ $\H_{\D, \alpha}$ can be thought of as the set of accurate classifiers while $\H_{\D, \alpha}^r$ can be thought of as the set of astute classifiers. By their definitions, it is clear that $\H_{\D, \alpha}^r \subseteq \H_{\D, \alpha}$. However, in the case when $\H$ is the set of linear classifiers, we see that for small $\alpha$, $\H_{\D, \alpha}^r$ is a much ``smaller" set than $\H_{\D, \alpha}$. By exploiting the geometric structure inherent to $\H$, we can much more efficiently search for some $h \in \H_{\D, \alpha}$ than we can in $\H_{\D, \alpha}^r$. This dynamic is the crux of our lower bound: as we essentially show that there are far more critical points (i.e. points near the decision boundary) that we must see for learning $\H_{\D, \alpha}^r$ that aren't required for $\H_{\D, \alpha}$. 

Thus, for our methods to extend to an arbitrary hypothesis class, we would require a similar dynamic. We need two properties to hold: (1) $\H_{\D, \alpha}^r$ must be a very strict subset of $\H_{\D, \alpha}$ for sufficiently small alpha. (2) We must have some kind of exploitable geometric structure about $\H$ which allows us to exploit this gap. For the case of linear classifiers, this was the $\ell_2$ measured aspect ratio, $\gamma(\D)$. 

\paragraph{Kernel Classifiers: } A natural choice of a more general hypothesis class would be Kernel Classifiers, which are linear classifiers that operate in an embedded space, $H$. The main difficulty in expanding our lower bound to this more general setting comes from the behavior near the margin: the effects of the robustness radius in the embedded space are considerably less behaved than they are in the standard linear case. Nevertheless, we leave this as an important avenue for future work.

%\begin{algorithm}[H]
%   \caption{Adversarial-Perceptron}
%   \label{alg:upper_bound}
%\begin{algorithmic}[1]
%    \STATE \textbf{Input}:  $S = \{(x_1, y_1), \dots, (x_n, y_n)\} \sim \D^n,$
%    \STATE $w \leftarrow 0$ 
%    \FOR{$i = 1 \dots n$}
%    	\STATE $z = \argmin_{||z - x_i||_p \leq r}  y_i\langle w, z \rangle$ \COMMENT{\rbb{finds adv. ex.}}
%        \IF{$\langle w, y_iz \rangle \leq 0$ \COMMENT{\rbb{checks label}}}
%            \STATE $w \leftarrow w + y_iz$ \COMMENT{\rbb{perceptron update}}
%        \ENDIF           
%    \ENDFOR
%    \STATE return $f_{w, 0}$
%\end{algorithmic}
%\end{algorithm}

\begin{algorithm}[H]
    \SetAlgoLined
    {\bfseries Input:} $S = \{(x_1, y_1), \dots, (x_n, y_n)\} \sim \D^n,$\;
    
    $w \leftarrow 0$ \;
    
    \For{$i = 1 \dots n$}{
    	$z = \argmin_{||z - x_i||_p \leq r}  y_i\langle w, z \rangle$ {\color{red} finds adversarial example}\;
    	
    	\If{$\langle w, y_iz \rangle \leq 0$ {\color{red} checks label}}{
    	
	     	 $w \leftarrow w + y_iz$ {\color{red} perceptron update}\;  	
	     	 
    	}
    }
    
    {\bfseries Return:} $f_{w, 0}$\;
    

\caption{Adversarial-Perceptron}\label{alg:upper_bound}
\end{algorithm}

\section{Upper Bounds}\label{sec:upper_bound}

In the previous section, we showed that for any algorithm, there is some distribution $\D \in \F_{r, \rho}$ that is difficult (i.e. requires high sample complexity) to learn robustly. A natural follow-up question is: what about distributions for which the margin, $\gamma$ is very large compared to $r$. 

Observe that in Figure \ref{fig:small_margin_robust} the robustness radius $r$ is very close to the margin. In particular, we can find adversarial examples from $S^+$ and $S^-$ that are very close to the decision boundary $f_{rob}$. By contrast, if $\gamma >> r$, then this no longer holds which suggests that better robust sample complexities might be possible.

In this section, we will describe a subset of $\F_{r, \rho}$ that can be learned with expected loss $O(\frac{1}{n})$, thus matching the standard sample complexity up to a constant factor. To do so, we will introduce a novel concept: the \textit{robust margin}. The basic intuition is that distributions for which the margin greatly exceeds the robustness radius are precisely distributions with a large robust margin. We use the following notation.

Observe that if $\D$ is a linearly $r$-separated distribution, then $\D$ must also be linearly separable. As earlier, let $S^+, S^- \subset \R^d$ denote the positively and negatively labeled examples from $\D$. We now define \begin{equation}\label{eqn:s_plus_s_minus} S_r^+ = \cup_{s \in S^+} B_p(s, r)\text{ and }S_r^{-} = \cup_{s \in S^-} B_p(s,r).\end{equation} It follows that the decision boundary of any linear classifier with perfect robustness over $\D$ must separate $S_r^+$ and $S_r^-$. We now define the robust margin as a measurement of this separation.

\begin{defn}\label{def:robust_margin}
Let $\D$ be a linearly $r$-separable distribution over $\R^d \times \{\pm 1\}$. Let $S_r^+$ and $S_r^-$ be as above. Then $\D$ has \textbf{robust margin} $\gamma_r$ if $\gamma_r$ is the largest real number such that there exists a linear classifier $f_{w,b}$ with the following properties: 

1. $f_{w,b}$ has perfect astuteness. That is, $\L_r(f_{w,b}, \D) = 0$. 

2. Let $H_{w,b} = \{x: \langle x, w \rangle = b\}$ denote the decision boundary of $f_{w,b}$. Then for all $x \in (S_r^+ \cup S_r^-)$, $x$ has $\ell_2$ distance at least $\gamma$ from $H_{w,b}$. That is, $$\inf_{x \in S_r^+ \cup S_r^-} \inf_{z \in H_{w,b}} ||x - z||_2 \geq \gamma.$$ We let $\gamma_r(\D)$ denote the margin of $\D$, and say that such a distribution is $r, \gamma_r$-separated. 
\end{defn}

It is crucial to note that although adversarial perturbations are measured in $\ell_p$, the robust margin is measured in $\ell_2$. This is because while the metric $\ell_p$ plays a role in constructing $B(x,r)$, it can be completely disregarded once the sets $S_r^+$ and $S_r^-$ are considered, as any hyperplane separating $S_r^+$ and $S_r^-$ will have perfect robustness.  

We now define the robust aspect ratio, which is the robust analog of standard aspect ratio.

\begin{defn}
Let $\D$ be a distribution over $\R^d \times \{\pm 1\}$. Then the \textbf{robust aspect ratio} of $\D$, $\rho_r(\D)$ is defined as $$\rho_r(\D) = \frac{diam_2(S_r^+ \cup S_r^-)}{\gamma_r(\D)},$$ where as before, $diam_2(S_r^+ \cup S_r^-)$ denotes its diameter in the $\ell_2$ norm.
\end{defn}

We will now show that just as the aspect ratio, $\rho(\D)$, characterized the sample complexity for standard classification, the robust aspect ratio, $\rho_r(\D)$ will characterize the sample complexity for robust learning. To do so, we present a perceptron-inspired algorithm (Algorithm \ref{alg:upper_bound}) for learning a robust classifier on $r$-separated data with robust aspect ratio $\rho_r$. 

The basic idea behind Algorithm \ref{alg:upper_bound} is to combine the standard perceptron algorithm with adversarial training. In particular, we iterate through the training set and do the following on each point (refer to Algorithm \ref{alg:upper_bound} for precise details). 

1. Find an adversarial example $(z, y_i)$ by attacking our classifier, $f_{w, 0}$, at $(x_i, y_i)$ (line 4). This is a straightforward convex optimization problem for linear classifiers.

2. If $f_{w,0}(z) \neq y_i$, we update our weight vector with $(z, y_i)$ by using the standard perceptron update (lines 5-6).

We have the following upper bound on the expected robust loss of our algorithm.

\begin{thm}\label{thm:upper_bound}
Let $\D$ be a distribution with robust aspect ratio $\rho_r(\D)$. Then for any $n > 0$, we have $$\E_{S \sim \D^n} [\L_r(A_S, \D)] \leq O(\frac{\rho_r(\D)^2}{n}),$$ where $A_S$ denotes the classifier learned by Algorithm \ref{alg:upper_bound} from training data $S$. 
\end{thm}

Observe that this expected loss is still larger than the expected standard loss in  Theorem \ref{thm:standard} as $\rho_r(\D) > \rho(\D)$ for any $\D$. We also note that this result is not contradictory with our lower bound; there exist distributions $\D \in \F_{r, \rho}$ such that $\gamma_r(\D) = 0$, and these are precisely the distributions for which our lower bounds hold.

\subsection{Generalization to Kernel Classifiers}

Algorithm \ref{alg:upper_bound} can be thought of as the robust analog to the perceptron algorithm. We now generalize this algorithm to obtain a robust variant of the \textit{kernel perceptron algorithm}. We first briefly review kernel classifiers. A detailed explanation of our generalized algorithm along with requisite background material can be found in Appendix \ref{sec:kernel_appendix}

\begin{defn}\label{defn:kernel}
Let $K: \R^d \times \R^d \to \R$ be a kernel similarity function, $T = \{(x_1, y_1), \dots, (x_m, y_m)\} \subset \R^d \times \{\pm 1\}$ be a set of labeled points, and $\alpha \in \R^m$ be a vector of $m$ real numbers. Then the \textbf{kernel classifier} with similarity function $K$, parameters $T, \alpha$, and denoted by $f_{T, K}^\alpha$ is defined as $$f_{T, \alpha}^K(x) = \begin{cases} +1 &  \sum_1^m \alpha_iy_iK(x_i, x) \geq 0\\ -1 &  \sum_1^m \alpha_iy_iK(x_i, x) < 0  \end{cases}.$$
\end{defn}

Conceptually, kernel classifiers are linear classifiers operating in embedded space. With each kernel similarity function $K$, there is a map $\phi: \R^d \to H$ (where $H$ is some Hilbert space) such that $K(x, x') = \langle \phi(x), \phi(x') \rangle$. Thus we can think of kernel classifiers as having a linear decision boundary in $H$. 

We now present an analog of Algorithm \ref{alg:upper_bound} that we call the Adversarial Kernel-Perceptron. The essence of this algorithm has not changed. For each $(x_t, y_t)$ in our training set, we do the following.

1. Find an adversarial example $(z, y_i)$ by attacking our classifier, $f_{T, \alpha}^K$, at $(x_i, y_i)$ (line 4). 

2. If $f_{T, \alpha}^K(z) \neq y_i$, we update our weight vector with $(z, y_i)$ by appending $(z, y_i)$ to $T$ lines (5-6). This corresponds to a kernel-perceptron update that uses $(z, y_i)$ instead of $(x_i, y_i)$.

\begin{algorithm}[H]

\textbf{Input}:  $S = \{(x_1, y_1), \dots, (x_n, y_n)\} \sim \D^n,$ Similarity function, $K$

$T \leftarrow \emptyset$, $\alpha \leftarrow 0$

\For{$i = 1 \dots n$}{
    	$z = \argmin_{||z - x||_p \leq r}  y_if_{T, \alpha}^K(z)$ {\color{red}finds adv. ex.}
    	
        \If{$f_{T, \alpha}^k(z) \leq 0$ {\color{red} checks label}}{
        
            $T = T \cup \{(z, y_i)\}$ {\color{red} kern. percep. update}
            
            $\alpha = (1, \dots, 1)_{|T|}$
            
         }
}
        
Return $f_{T, \alpha}^K$
   \caption{Adversarial-Kernel-Perceptron}
   \label{alg:upper_bound_kernel}
\end{algorithm}

One challenging aspect of this algorithm is minimizing $f_{T, \alpha}^k(z)$. For linear classifiers, this has a closed form solution that utilizes the dual norm. For arbitrary Kernel classifiers, this is a somewhat more challenging problem. However, we note that this can be solved using standard optimization techniques, and in some cases (when $K$ is particularly simple), it can be solved with basic gradient descent.

Finally, we show that this Algorithm has similar performance to the linear case. Instead of using the robust aspect ratio, $\rho_r(\D)$, to bound the performance, we will require the \textbf{robust $K$-aspect ratio}, which is the kernel analog of this quantity. It can be thought of as the robust aspect ratio in the embedded space $H$. Details about this quantity (along with the proof of the theorem) can be found in Appendix \ref{sec:kernel_appendix}.

\begin{thm}\label{thm:upper_bound_kernel}
Let $\D$ be a distribution with robust $K$-aspect ratio $\rho_r^K(\D)$. Then for any $n > 0$, we have $$\E_{S \sim \D^n} [\L_r(A_S, \D)] \leq O(\frac{\rho_r^K(\D)^2}{n}),$$ where $A_S$ denotes the classifier learned by Algorithm \ref{alg:upper_bound_kernel} from training data $S$. 
\end{thm}

This result indicates that for small values of $\rho_r^k(\D)$, we can achieve a very good robust sample complexity for kernel classifiers. However, as the size of the perturbations approach this margin, this quantity goes to infinity. This phenomenon mirrors the linearly separable case, and suggests that a similar overall dynamic holds for kernel classification. We leave finding a full generalization (including our lower bound) for a direction in future work.






%\newcommand{\crop}[1]{\mathrm{crop}({#1})}
\newcommand{\object}[1]{\mathrm{object}({#1})}
\newcommand{\ba}{A_i}
\newcommand{\bb}{B_i}
\newcommand{\calA}{\mathcal{A}}
\newcommand{\calB}{\mathcal{B}}
\newcommand{\calX}{\mathcal{X}}
\newcommand{\masked}[1]{\mathrm{masked}({#1})}
\newcommand{\bx}{\mathbf{x}}
\newcommand{\SSL}{\textsc{SSL}}
\newcommand{\SSLbb}{\SSL^\mathrm{back}}
\newcommand{\SSLpj}{\SSL^\mathrm{proj}}
\newcommand{\CLF}{\textsc{CLF}}
\newcommand{\CLFbb}{\CLF^\mathrm{back}}
\newcommand{\CLFpj}{\CLF^\mathrm{proj}}
\newcommand{\SUP}{\textsc{SUP}}
\newcommand{\KNN}{\textsc{KNN}}
\newcommand{\KNNset}{\textsc{KNN}^\mathrm{set}}
\newcommand{\KNNprob}{\textsc{KNN}^\mathrm{prob}}
\newcommand{\KNNcl}{\textsc{KNN}^\mathrm{cl}}
\newcommand{\KNNconf}{\textsc{KNN}^\mathrm{conf}}
\newcommand{\RCDM}{\textsc{RCDM}}
\newcommand{\cl}{\mathrm{cl}}
\newcommand{\clpred}{\tilde{\mathrm{cl}}}
\newcommand{\Abox}{\overline{\calA}}
\newcommand{\Bbox}{\overline{\calB}}
\newcommand{\dejavu}{\emph{déjà vu }}
\newcommand{\Dejavu}{\emph{Déjà vu }}

\newcommand{\citations}{{\color{green}[CITE]}}

\definecolor{part_blue}{rgb}{0.2824, 0.4706, .8157}
\definecolor{part_red}{rgb}{0.8392, 0.3725, 0.3725}
\definecolor{part_orange}{rgb}{0.9333, 0.5216, 0.2902}

\DeclareRobustCommand{\mybox}[2][gray!20]{%
\begin{tcolorbox}[   %% Adjust the following parameters at will.
        % breakable,
        left=0pt,
        right=0pt,
        top=0pt,
        bottom=0pt,
        colback=#1,
        colframe=#1,
        width=\dimexpr\columnwidth\relax, 
        % width=\textwidth, 
        enlarge left by=0mm,
        boxsep=5pt,
        arc=0pt,outer arc=0pt,
        ]
        #2
\end{tcolorbox}
}
%\section{Introduction}
\label{sec:intro}
Self-supervised learning (SSL)~\citep{chen2020simclr, chen2020simsiam, zbontar2021barlow, vicreg, caron2020swav, MAE} aims to learn general representations of content-rich data without explicit labels by solving a \textit{pretext task}. In many recent works, such pretext tasks rely on joint-embedding architectures whereby randomized image augmentations are applied to create multiple views of a training sample, and the model is trained to produce similar representations for those views. When using cropping as random image augmentation, the model learns to associate objects or parts (including the background scenery) that co-occur in an image.
However, doing so also arguably exposes the training data to higher privacy risk as objects in training images can be explicitly memorized by the SSL model. For example, if the training data contains the photos of individuals, the SSL model may learn to associate the face of a person with their activity or physical location in the photo. This may allow an adversary to extract such information from the trained model for targeted individuals.

\begin{figure}[t]
    \centering
    \includegraphics[width=1.0\columnwidth]{figures/new_black_swan.pdf}
    \caption{\textbf{Left:} Reconstruction of an SSL training image from a crop containing only the background. The SSL model memorizes the association of this \emph{specific} patch of water (pink square) to this \emph{specific} foreground object (a black swan) in its embedding, which we decode to visualize the full training image. \textbf{Right:} The reconstruction technique fails on a public test image that the SSL model has not seen before.}
    \label{fig:black_swan}
\end{figure}

In this work, we aim to evaluate to what extent SSL models memorize the association of specific objects in training images or the association of objects and their specific backgrounds, and whether this memorization signal can be used to reconstruct the model's training samples. Our results demonstrate that SSL models memorize such associations beyond simple correlation. For instance, in Figure \ref{fig:black_swan} (\textbf{left}), we use the SSL representation of a \emph{training image crop containing only water} and this enables us to reconstruct the object in the foreground with remarkable specificity---in this case a black swan.
By contrast, in Figure \ref{fig:black_swan} (\textbf{right}), when using the \emph{crop from the background of a test set image} that the SSL model \emph{has not seen before}, its representation only contains enough information to infer, through correlation, that the foreground object was likely some kind of waterbird --- but not the specific one in the image.

Figure \ref{fig:black_swan} shows that SSL models suffer from the unintended memorization of images in their training data---a phenomenon we refer to as \emph{déjà vu memorization}
%\footnote{The French loanword \emph{déjà vu} means already-seen, which reflects the type of unintended memorization of objects that the SSL model saw during training.}.
\footnote{The French loanword \emph{déjà vu} means `already-seen', just as an image is seen and memorized in training.}
Beyond visualizing \emph{déjà vu} memorization through data reconstruction, we also design a series of experiments to quantify the degree of memorization for different SSL algorithms, model architectures, training set size, \emph{etc.} We observe that \emph{déjà vu} memorization is exacerbated by the atypically large number of training epochs often recommended in SSL training, as well as certain hyperparameters in the SSL training objective. Perhaps surprisingly, we show that \emph{déjà vu} memorization occurs even when the training set is large---as large as half of ImageNet~\citep{imagenet}---and can continually worsen even when standard techniques for evaluating learned representation quality (such as linear probing) do not suggest increased overfitting. Our work serves as the first systematic study of unintended memorization in SSL models and motivates future work on understanding and preventing this behavior. Specifically, we: 
\begin{itemize}
    \vspace{-0.5em}
    \item Elucidate how SSL representations memorize aspects of individual training images, what we call \emph{déjà vu} memorization;
    \item Design a novel training data reconstruction pipeline for non-generative vision models. This is in contrast to many prominent reconstruction algorithms like \citep{carlini2021extracting, google_diffusion}, which rely on the model itself to generate its own memorized samples and is not possible for SSL models or classifiers;
    \item Propose metrics to quantify the degree of \dejavu memorization committed by an SSL model. This allows us to observe how \dejavu changes with training epochs, dataset size, training criteria, model architecture and more. 
\end{itemize}

%\section{Preliminaries and Related Work}
\label{sec:related}

\textbf{Self-supervised learning} (SSL) is a machine learning paradigm that leverages unlabeled data to learn representations. Many SSL algorithms rely on \emph{joint-embedding} architectures (\emph{e.g.}, SimCLR~\citep{chen2020simclr}, Barlow Twins~\citep{zbontar2021barlow}, VICReg~\citep{vicreg} and Dino~\citep{Dino}), which are trained to associate different augmented views of a given image. For example, in SimCLR, given a set of images $\calA = \{A_1,\ldots,A_n\}$ and a randomized augmentation function $\mathrm{aug}$, the model is trained to maximize the cosine similarity of draws of $\SSL(\mathrm{aug}(A_i))$ with each other and minimize their similarity with $\SSL(\mathrm{aug}(A_j))$ for $i \neq j$. The augmentation function $\mathrm{aug}$ typically consists of operations such as cropping, horizontal flipping, and color transformations to create different views that preserve an image's semantic properties. 

\paragraph{SSL representations.} Once an SSL model is trained, its learned representation can be transferred to different downstream tasks. This is often done by extracting the representation of an image from the \emph{backbone model}\footnote{SSL methods often use a trick called \emph{guillotine regularization}~\citep{Guillotine}, which decomposes the model into two parts: a \emph{backbone model} and a \emph{projector} consisting of a few fully-connected layers. Such trick is needed to handle the misalignment between the pretext SSL task and the downstream task.} and either training a linear probe on top of this representation or finetuning the backbone model with a task-specific head~\citep{Guillotine}.
%Compared to representations learned by supervised learning, SSL representations are often more robust and transferable~\citep{hendrycks2019using, ericsson2021self}, leading to state-of-the-art result on many downstream tasks. To understand the effectiveness of SSL algorithms, several prior works investigated what kind of information the SSL model has learned~\citep{jing2021understanding, ericsson2021self, kalibhat2022towards, RCDM}. In particular, \citet{RCDM} trained a conditional generative model on SSL representations and showed that they encode richer visual details about the input image compared to supervised learning. 
%However, from a privacy perspective, this may be a cause for concern as the model also has more potential to overfit and memorize precise details about the training data compared to supervised learning. We show concretely that this privacy risk can indeed be realized by defining and measuring \emph{déjà vu} memorization.
It has been shown that SSL representations encode richer visual details about input images than supervised models do \cite{RCDM}. However, from a privacy perspective, this may be a cause for concern as the model also has more potential to overfit and memorize precise details about the training data compared to supervised learning. We show concretely that this privacy risk can indeed be realized by defining and measuring \emph{déjà vu} memorization.
\vspace{-0.5em} 
% \paragraph{Privacy risks in ML.} Overfitting in ML occurs when a model memorizes information specific to its training data rather than general population-level information. When the model is trained on privacy-sensitive data, overfitting is especially harmful as an adversary can infer private information about the training data when given access to the model~\citep{yeom2018privacy, feldman2020does}. The simplest and most well-studied form of privacy risk in ML is susceptibility to \emph{membership inference attacks}~\citep{shokri2017membership, salem2018ml, sablayrolles2019white}, where the adversary infers whether an individual is part of the training set or not. More sophisticated privacy attacks include \emph{attribute inference}~\citep{fredrikson2014privacy, mehnaz2022your, jayaraman2022attribute}, where specific attributes about an individual are inferred given others, and \emph{data reconstruction}~\citep{carlini2021extracting, balle2022reconstructing, guo2022bounding}, where entire training samples are recovered from the trained model. Our study of \emph{déjà vu} memorization is similar to both attribute inference and data reconstruction, leveraging SSL representations of the training image background to infer and reconstruct the foreground object.
% \vspace{-0.5em} 
% \paragraph{Training data extraction in NLP.} Our study of \dejavu memorization in SSL models is inspired by similar work in the natural language processing (NLP) domain. \citet{carlini2019secret} first showed that language models exhibit unintended memorization, where given a context string present in its training data, the model can generate the remaining text at test time. This unintended memorization has been further exploited in \citet{carlini2021extracting} to extract training data from GPT-2~\citep{radford2019language} and, more recently, extended to extract memorized images from Stable Diffusion \citep{google_diffusion}. The way by which these works exploit unintended memorization is similar to ours: given partial information about a training sample, the model is prompted to reveal the rest of the sample. In our case, however, since the SSL model is not generative, extraction is significantly harder and requires careful design.

\paragraph{Privacy risks in ML.} When a model is overfit on privacy-sensitive data, it memorizes specific information about its training examples, allowing an adversary with access to the model to learn private information~\citep{yeom2018privacy, feldman2020does}. Privacy attacks in ML range from the simplest and best-studied \emph{membership inference attacks}~\citep{shokri2017membership, salem2018ml, sablayrolles2019white} to \emph{attribute inference}~\citep{fredrikson2014privacy, mehnaz2022your, jayaraman2022attribute} and \emph{data reconstruction}~\citep{carlini2021extracting, balle2022reconstructing, guo2022bounding} attacks. In the former, the adversary only infers whether an individual participated in the training set. Our study of \emph{déjà vu} memorization is most similar to the latter: we leverage SSL representations of the training image background to infer and reconstruct the foreground object. Our approach reflects similar work in the NLP domain \citep{carlini2019secret, carlini2021extracting}: when prompted with a context string present in the training data, a large language model is shown to generate the remainder of string at test time, revealing sensitive text like home addresses. This method was recently extended to extract memorized images from Stable Diffusion \citep{google_diffusion}.  We exploit memorization in a similar manner: given partial information about a training sample, the model is prompted to reveal the rest of the sample. In our case, however, since the SSL model is not generative, extraction is significantly harder and requires careful design.

%\section{Defining \emph{Déjà Vu} Memorization}
\label{sec:definition}

\paragraph{What is \dejavu memorization?} At a high level, the objective of SSL is to learn general representations of objects that occur in nature. This is often accomplished by associating different parts of an image with one another in the learned embedding. Returning to our example in Figure \ref{fig:black_swan}, given an image whose background contains a patch of water, the model may learn that the foreground object is a water animal such as duck, pelican, otter, \emph{etc.}, by observing different images that contain water from the training set. We refer to this type of learning as \emph{correlation}: the association of objects that tend to co-occur in images from the training data distribution.

A natural question to ask is \emph{``Can the reconstruction of the black swan in Figure \ref{fig:black_swan} be reasoned as correlation?''} The intuitive answer may be no, since the reconstructed image is qualitatively very similar to the original image. However, this reasoning implicitly assumes that for a random image from the training data distribution containing a patch of water, the foreground object is unlikely to be a black swan. Mathematically, if we denote by $\mathcal{P}$ the training data distribution and $A$ the image, then
\begin{equation*}
\label{eq:p_corr}
p_\text{corr} := \mathbb{P}_{A \sim \mathcal{P}}(\mathrm{object}(A) = \texttt{black swan} ~|~ \mathrm{crop}(A) = \texttt{water})
\end{equation*}
is the probability of inferring that the foreground object is a black swan through \emph{correlation}. This probability may be naturally high due to biases in the distribution $\mathcal{P}$, \emph{e.g.}, if $\mathcal{P}$ contains no other water animal except for black swans. In fact, such correlations are often exploited to learn a model for image inpainting with great success~\citep{yu2018generative, ulyanov2018deep}.

Despite this, we argue that reconstruction of the black swan in Figure \ref{fig:black_swan} is \emph{not} due to correlation, but rather due to \emph{unintended memorization}: the association of objects unique to a single training image. As we will show in the following sections, the example in Figure \ref{fig:black_swan} is not a rare success case and can be replicated across many training samples. More importantly, failure to reconstruct the foreground object in Figure \ref{fig:black_swan} (\textbf{right}) on test images hints at inferring through correlation is unlikely to succeed---a fact that we verify quantitatively in Section \ref{sec:label inference accuracy}. Motivated by this discussion, we give a verbal definition of \dejavu memorization below, and design a testing methodology to quantify \dejavu memorization in Section \ref{sec:notation and setup}.
\mybox{\textbf{Definition:} A model exhibits \emph{déjà vu memorization} when it retains information so specific to an individual training image, that it enables recovery of aspects particular to that image given a part that does not contain them.
The recovered aspect must be beyond what can be inferred using only correlations in the data distribution.} 

% \textbf{Definition:} A model exhibits \emph{déjà vu memorization} when it retains information so specific to an individual training image, that it enables recovery of aspects particular to that image given a part that does not contain them.
% The recovered aspect must be beyond what can be inferred using only correlations in the data distribution.


 We intentionally kept the above definition broad enough to encompass different types of information that can be inferred about the training image, including but not restricted to object category, shape, color and position. For example, if one can infer that the foreground object is red given the background patch with accuracy significantly beyond correlation, we consider this an instance of \dejavu memorization as well. We mainly focus on object category to quantify \dejavu memorization in Section \ref{sec:quant} since the ground truth label can be easily obtained. We consider other types of information more qualitatively in the visual reconstruction experiments in Section \ref{sec:visualizing}.

\paragraph{Privacy implications of \dejavu memorization.} \Dejavu memorization can be a cause for concern when the training data contains privacy-sensitive information. As a motivating example, consider an SSL model trained on photos of individuals. If the model exhibits \dejavu memorization then, given the face of an individual, it may be possible to infer where the individual was or even visually reconstruct their location in the training image. Such information leakage raises privacy concerns, especially if there was no prior agreement that the trained model may reveal such information to third parties. This hypothetical scenario serves as a motivation that \dejavu memorization should be carefully examined to avoid unintended disclosure of private information in practical applications.

% \begin{figure*}[h]
%     \centering
%     \includegraphics[width = 0.85\textwidth]{figures/SSL_attack_cartoon.png}
%     \caption{We measure memorization by comparing the `target model' trained on the target image ($\SSL_A$ trained on $A_i$ in above example) with the `reference model' not trained on it ($\SSL_B$, above). \textbf{[Top Strip]} A cropping of the image disjoint from the labeled foreground object is embedded using the target model. This embedding is then labeled by a K-Nearest Neighbor (KNN) adversary built on a public set of labeled images, $X$, which it has also embedded using the target model. \textbf{[Bottom Strip]} To account for correlation, the same procedure is followed with the reference model. If the label is only extracted using the target model, it is counted as memorization. If it is extracted using either model, it is counted as correlation. We find that the KNN adversary's predictions using the target model (trained on attacked examples) are significantly more accurate than they are using the reference model, indicating routine memorization of training examples.}
%     \label{fig:ssl attack cartoon}
% \end{figure*}

\begin{figure}[t]
%%%
%SPIDER
%%%
     % \centering
     % \begin{subfigure}[b]{0.25\textwidth}
     %     \centering
     %     \includegraphics[width=\textwidth]{figures/data_split.png}
     %     % \caption{SimCLR correlated \textit{yellow garden spider} examples}
     %     \label{fig:data split}
     % \end{subfigure}
     % \hfill
     % \begin{subfigure}[b]{0.7\textwidth}
     %     \centering
     %     \includegraphics[width=\textwidth]{figures/pipeline_cartoon.png}
     %     \begin{minipage}{5cm}
     %        \vfill
     %    \end{minipage}
     %     % \caption{SimCLR memorized \textit{yellow garden spider} examples}
     %     \label{fig:pipeline cartoon}
     % \end{subfigure}
     \includegraphics[width=\textwidth]{figures/split_and_pipeline_cartoon.png}
\caption[Overview of testing methodology.]{
Overview of testing methodology. \textbf{Left:} Data is split into \emph{target set} $\calA$, \emph{reference set} $\calB$ and \emph{public set} $\calX$ that are pairwise disjoint. $\calA$ and $\calB$ are used to train two SSL models $\SSL_A$ and $\SSL_B$ in the same manner. $\calX$ is used for KNN decoding or for training an RCDM to reconstruct the input at test time. \textbf{Right:} Given a training image $A_i \in \calA$, we use $\SSL_A$ to embed $\crop{A_i}$ containing only the background, as well as the entire set $\calX$ and find the $k$-nearest neighbors of $\crop{A_i}$ in $\calX$ in the embedding space. These KNN samples can be used directly to infer the foreground object (\emph{i.e.}, class label) in $A_i$ using a KNN classifier, or their embeddings can be averaged as input to the trained RCDM to visually reconstruct the image $A_i$. For instance, the RCDM reconstruction results in Figure \ref{fig:black_swan} (left) when given $\SSL_A(\crop{A_i})$ and results in Figure \ref{fig:black_swan} (right) when given $\SSL_A(\crop{B_i})$ for an image $B_i \in \calB$.
%\textbf{Left:} illustration of the three datasets used in our tests. Two private data sets, $A$ and $B$, of equal size are used to train two SSL models, $\SSL_A$ and $\SSL_B$, respectively. A disjoint public set, $X$, is made available to the memorization test to help decode model embeddings. Memorization is only tested on examples $A_i \in A$ that are unique to set $A$. \textbf{Right:} illustration of inference pipeline used in tests. A periphery cropping that excludes the foreground object is taken from private image $A_i$. The KNN then finds the $k$ public set nearest neighbors of the periphery crop in the embedding space of $\SSL_A$. 
%The $\SSL_A$ representation of these $k$ neighbors and of the crop are used by the conditional generative model, RCDM, to reconstruct the foreground object. The labels of these $k$ neighbors are used to recover the foreground object label. (Not pictured) We repeat this process using reference model $\SSL_B$, not trained on image $A_i$, to determine whether the foreground object is still recoverable by learned correlations, e.g. if black swans were the only objects appearing near water in the data distribution. In this instance, the crop's public set neighbors in $\SSL_B$'s representation space include a variety of water animals like ducks, pelicans, and otters. Meanwhile, with $\SSL_A$, the neighbors are nearly all black swans in the same position as the swan of $A_i$.
}
\label{fig:split_and_pipeline_cartoon}
\end{figure}

\textbf{Distinguishing memorization from correlation.} When measuring \dejavu memorization, it is crucial to differentiate what the model associates through \emph{memorization} and what it associates through \emph{correlation}. Our testing methodology is based on the following intuitive definition.
\mybox{\textbf{Definition:} If an SSL model associates two parts in a training image, we say that it is due to \emph{correlation} if other SSL models trained on a similar dataset from $\mathcal{P}$ without this image would likely make the same association. Otherwise, we say that it is due to \emph{memorization}.}

Notably, such intuition forms the basis for differential privacy (DP; \cite{dwork2006calibrating, dwork2013algorithmic})---the most widely accepted notion of privacy in ML.

\subsection{Testing Methodology for Measuring \emph{Déjà Vu} Memorization}
\label{sec:notation and setup}

In this section, we use the above intuition to measure the extent of \dejavu memorization in SSL. Figure \ref{fig:split_and_pipeline_cartoon} gives an overview of our testing methodology.
\vspace{-0.75em}
\paragraph{Dataset splitting.} We focus on testing \dejavu memorization for SSL models trained on the ImageNet-1K dataset~\citep{imagenet}. Our test first splits the ImageNet training set into three independent and disjoint subsets $\calA$, $\calB$ and $\calX$. The dataset $\calA$ is called the \emph{target set} and $\calB$ is called the \emph{reference set}. The two datasets are used to train two separate SSL models, $\SSL_A$ and $\SSL_B$, called the \emph{target model} and the \emph{reference model}. Finally, the dataset set $\calX$ is used as an auxiliary public dataset to extract information from $\SSL_A$ and $\SSL_B$.
%\footnote{See Appendix \ref{sec:appx splits} for details on how the dataset splits are generated.}.
Our dataset splitting serves the purpose of distinguishing memorization from correlation in the following manner. Given a sample $A_i \in \calA$, if our test returns the same result on $\SSL_A$ and $\SSL_B$ then it is likely due to correlation because $A_i$ is not a training sample for $\SSL_B$. Otherwise, because $\calA$ and $\calB$ are drawn from the same underlying distribution, our test must have inferred some information unique to $A_i$ due to memorization. Thus, by comparing the difference in the test results for $\SSL_A$ and $\SSL_B$, we can measure the degree of \dejavu memorization\footnote{See Appendix \ref{sec:appx splits} for details on how the dataset splits are generated.}.
\vspace{-0.75em}
\paragraph{Extracting foreground and background crops.} Our testing methodology aims at measuring what can be inferred about the foreground object in an ImageNet sample given a background crop. This is made possible because ImageNet provides bounding box annotations for a subset of its training images---around 150K out of 1.3M samples. We split these annotated images equally between $\calA$ and $\calB$. Given an annotated image $A_i$, we treat everything inside the bounding box as the foreground object associated with the image label, denoted $\object{A_i}$. We take the largest possible crop that does not intersect with any bounding box as the background crop (or \emph{periphery crop}), denoted $\crop{A_i}$\footnote{We also present another heuristic in \cref{sec:appx corner crop} which takes a corner crop as the background crop, allowing our test to be run without bounding box annotations.}
%Since the labeled object tends to be at the image's center, the corner crop usually excludes it. }
%Because most images in ImageNet are object centric, an image's corner would not include the foreground object.}.
\vspace{-0.75em}
\paragraph{KNN-based test design.} Joint-embedding SSL approaches encourage the embeddings of random crops of a training image $A_i \in \calA$ to be similar. Intuitively, if the model exhibits \dejavu memorization, it is reasonable to expect that the embedding of $\crop{A_i}$ is similar to that of $\object{A_i}$ since both crops are from the same training image. In other words, $\SSL_A(\crop{A_i})$ encodes information about $\object{A_i}$ that cannot be inferred through correlation. However, decoding such information is challenging as these approaches do not learn a decoder associated with the encoder $\SSL_A$.

Here, we leverage the public set $\calX$ to decode the information contained in $\crop{A_i}$ about $\object{A_i}$. More specifically, we map images in $\calX$ to their embeddings using $\SSL_A$ and extract the $k$-nearest-neighbor (KNN) subset of $\SSL_A(\crop{A_i})$ in $\calX$. We can then decode the information contained in $\crop{A_i}$ in one of two ways:
\begin{itemize}
\item \emph{Label inference:} Since $\calX$ is a subset of ImageNet, each embedding in the KNN subset is associated with a class label. If $\crop{A_i}$ encodes information about the foreground object, its embedding will be close to samples in $\calX$ that have the same class label (\emph{i.e.}, foreground object category). We can then use a KNN classifier to infer the foreground object in $A_i$ given $\crop{A_i}$.
\item \emph{Visual reconstruction:} Following \citet{RCDM}, we train an RCDM---a conditional generative model---on $\calX$ to decode $\SSL_A$ embeddings into images. The RCDM reconstruction can recover qualitative aspects of an image remarkably well, such as recovering object color or spatial orientation using its SSL embedding. Given the KNN subset, we average their SSL embeddings and use the trained RCDM model to visually reconstruct $A_i$.
\end{itemize}
In Section \ref{sec:quant}, we focus on quantitatively measuring \dejavu memorization with label inference, and then use the RCDM reconstruction to visualize \dejavu memorization in Section \ref{sec:visualizing}.
%\section{Quantifying \emph{Déjà Vu} Memorization}
\label{sec:quant}

We apply our testing methodology to quantify a specific form of \dejavu memorization: inferring the foreground object (class label) given a crop of the background.

% \paragraph{Extracting model embeddings.} We test \dejavu memorization on two popular SSL algorithms, SimCLR~\citep{chen2020simclr} and VICReg~\citep{vicreg}.
% %\footnote{We present additional SSL models in \cref{sec:appx simclr results}} 
% As described in Section \ref{sec:related}, these algorithms produce two embeddings given an input image: a \emph{backbone} embedding and a \emph{projector} embedding that is derived by applying a small fully-connected network on top of the backbone embedding. Unless otherwise noted, all SSL embeddings refer to the projector embedding.
% To understand whether \dejavu memorization is particular to SSL, we also evaluate embeddings produced by a supervised model $\CLF_A$ trained on $\calA$. We apply the same set of image augmentations as those used in SSL and train $\CLF_A$ using the cross-entropy loss to predict ground truth labels. 
\vspace{-0.75em}
\paragraph{Extracting model embeddings.} We test \dejavu memorization on a variety of popular SSL algorithms, with a focus on VICReg~\citep{vicreg}. These algorithms produce two embeddings given an input image: a \emph{backbone} embedding and a \emph{projector} embedding that is derived by applying a small fully-connected network on top of the backbone embedding. Unless otherwise noted, all SSL embeddings refer to the projector embedding. 
To understand whether \dejavu memorization is particular to SSL, we also evaluate embeddings produced by a supervised model $\CLF_A$ trained on $\calA$. We apply the same set of image augmentations as those used in SSL and train $\CLF_A$ using the cross-entropy loss to predict ground truth labels. 
\vspace{-0.75em}
\paragraph{Identifying the most memorized samples.} Prior works have shown that certain training samples can be identified as more prone to memorization than others~\citep{feldman2020does, watson2021importance, ye2021enhanced}. Similarly, we provide a heuristic to identify the most memorized samples in our label inference test using confidence of the KNN prediction.
Given a periphery crop, $\crop{A_i}$, let $\KNN_A \big( \crop{A_i} \big) \subseteq \calX$ denote its $k$-nearest neighbors in the embedding space of $\SSL_A$. From this KNN subset we can obtain: \textbf{(1)} $\KNNprob_A \big( \crop{A_i} \big)$, the vector of class probabilities (normalized counts) induced by the KNN subset, and \textbf{(2)} $\KNNconf_A \big( \crop{A_i} \big)$, the negative entropy of the probability vector $\KNNprob_A \big( \crop{A_i} \big)$, as confidence of the KNN prediction. When entropy is low, the neighbors agree on the class of $A_i$ and hence confidence is high. 
% \begin{itemize}[noitemsep, leftmargin=*, topsep=0pt]
%     \item $\KNN_A \big( \crop{A_i} \big)$: The most prevalent class in the KNN subset as prediction for the class label $\cl(A_i)$. 
%     \item $\KNNprob_A \big( \crop{A_i} \big)$: The vector of class probabilities (normalized counts) induced by the KNN subset.
%     \item $\KNNconf_A \big( \crop{A_i} \big)$: Negative entropy of the probability vector $\KNNprob_A \big( \crop{A_i} \big)$ as confidence of the KNN prediction. When entropy is low, the neighbors agree on the class of $A_i$ and hence confidence is high. 
% \end{itemize}
We can sort the confidence score $\KNNconf_A \big( \crop{A_i} \big)$ across samples $A_i$ in decreasing order to identify the most confidently predicted samples, which likely correspond to the most memorized samples when $A_i \in \calA$.

\subsection{Population-level Memorization}
\label{sec:label inference accuracy}

%ORIGINAL FIGURE SETUP IN ARXIV: 
% \input{dejavu_training_epochs.tex}
% \input{dejavu_training_set_size.tex}
%PUT ORIGINAL FIGURES SIDE BY SIDE: 
% \input{dejavu_training_epochs_set_size.tex}
%PUT IN NEW FIGURES: 

\begin{wrapfigure}{r}{0.4\textwidth} 
    \centering
    \includegraphics[width=0.4\textwidth]{figures/dejavu_main.pdf}
    \caption{Accuracy of label inference using the target model (trained on $\calA$) vs. the reference model (trained on $\calB$) on the top $\%$ most confident examples $A_i \in \calA$ using only $\crop{A_i}$. For VICReg, there is a large accuracy gap between the two models, indicating a significant degree of \dejavu memorization.}
    \label{fig:dejavu main}
    \vspace{-2ex}
\end{wrapfigure}

Our first measure of \dejavu memorization is population-level label inference accuracy: \emph{What is the average label inference accuracy over a subset of SSL training images given their periphery crops?} 
To understand how much of this accuracy is due to $\SSL_A$'s \dejavu memorization, we compare with a correlation baseline using the reference model: $\KNN_B$'s label inference accuracy on images $A_i \in \calA$. 
In principle, this inference accuracy should be significantly above chance level ($1/1000$ for ImageNet) because the periphery crop may be highly indicative of the foreground object through correlation, \emph{e.g.}, if the periphery crop is a basketball player then the foreground object is likely a basketball.

Figure \ref{fig:dejavu main} compares the accuracy of $\KNN_A$ to that of $\KNN_B$ when inferring the labels of images in $A_i \in \calA$\footnote{The sets $\calA$ and $\calB$ are exchangeable, and in practice we repeat this test on images from $\calB$ using $\SSL_B$ as the target model and $\SSL_A$ as the reference model, and average the two sets of results.} using $\crop{A_i}$.
Results are shown for VICReg and the supervised model; trends for other models are shown in Appendix \ref{sec:appx simclr results}. For both VICReg and supervised models, inferring the class of $\crop{A_i}$ using $\KNN_B$ (dashed line) through correlation achieves a reasonable accuracy that is significantly above chance level. However, for VICReg, the inference accuracy using $\KNN_A$ (solid red line) is significantly higher, and the accuracy gap between $\KNN_A$ and $\KNN_B$ indicates the degree of \dejavu memorization. We highlight two observations: 
\begin{itemize}
    \item The accuracy gap of VICReg is significantly larger than that of the supervised model. This is especially notable when accounting for the fact that the supervised model is trained to associate randomly augmented crops of images with their ground truth labels. In contrast, VICReg has no label access during training but the embedding of a periphery crop can still encode the image label. 
    \item For VICReg, inference accuracy on the $1\%$ most confident examples is nearly $95\%$, which shows that our simple confidence heuristic can effectively identify the most memorized samples. This result suggests that an adversary can use this heuristic to identify vulnerable training samples to launch a more focused privacy attack.
\end{itemize}
\vspace{-.75em}
\paragraph{The \dejavu score. }
The curves of Figure \ref{fig:dejavu main} show memorization across confidence values for a single training scenario.  To study how memorization changes with different hyperparamters, we extract a single value from these curves: the \dejavu \emph{score} at confidence level $p$. In Figure \ref{fig:dejavu main}, this is the gap between the solid red (or gray) and dashed red (or gray) where confidence ($x$-axis) equal $p\%$. In other words, given the periphery crops of set $\calA$, $\KNN_A$ and $\KNN_B$ separately select and label their top $p\%$ most confident examples, and we report the difference in their accuracy. The \dejavu score captures both the degree of memorization by the accuracy gap and the \emph{ability to identify memorized examples} by the confidence level. If the score is 10\% for $p=33\%$, $\KNN_A$ has 10\% higher accuracy on its most confident third of $\calA$ than $\KNN_B$ does on its most confident third. In the following, we set $p = 20\%$, approximately the largest gap for VICReg (red lines) in Figure \ref{fig:dejavu main}. 
% Specifically, the \dejavu \emph{score} on the top $p\%$ most confident examples is,  
% \begin{equation}
%     \mathrm{DejaVu}(p) = \mathrm{Acc}_{\SSL_A}\big( \calA_{\SSL_A, p}  \big) - \mathrm{Acc}_{\SSL_B}\big( \calA_{\SSL_B, p}  \big) \ ,
%     \label{eqn:dejavu score}
% \end{equation}
% where $\calA_{\SSL_A, p}$
% Here we introduce a DejaVu memorization metric that quantify how much a target model is able to retrieve more class information from a crop than the reference model. We define it as:
% where $p$ is a function that take the $p$ purcent most confident samples.
%Figure \ref{fig:dejavu v. training epochs} shows how \dejavu memorization changes with the number of epochs used to train the embedding model (VICReg and supervised, respectively). The training set size is fixed to 300K samples, and label inference accuracy is computed on the top $20\%$ highest confidence examples. The number of epochs has a very strong influence on the degree of memorization for VICReg as the accuracy gap widens when number of epochs increases. We note that 1000 training epochs is used in several SSL works \citep{vicreg, simclr}. Remarkably, this trend in memorization is \emph{not} reflected in the standard metric for evaluating SSL representations: linear probe accuracy. The gray line in Figure \ref{fig:dejavu v. training epochs} shows the train-test accuracy gap of a linear classifier trained on top of the VICReg embeddings. Although there is a sizeable train-test gap, it does not grow significantly beyond 500 epochs. In contrast, \dejavu memorization (blue line) continues to worsen after 500 epochs. Thus, our test can be used as an alternative to linear probe accuracy to evaluate the memorization of SSL models.
% \vspace{-.75em}

% \paragraph{Comparison with the generalization gap} A network that perform very well on a training set while performing poorly on a test set (assuming the training set and test set sampled uniformly from the same distribution) is probably memorizing the training examples without being able to generalize on the test data. One could expect that measuring the difference in accuracy between the training and test set could give us insights on the degree of \dejavu memorization. However, we show in Figure  \ref{fig:dejavu v. training epochs} and \ref{fig:dejavu v. n} that this is not the case. In fact \dejavu memorization can significantly increase while the train-test gap decrease. In our experiments, we did not find a correlation between \dejavu and generalization.
\vspace{-0.75em}
\paragraph{Comparison with the linear probe train-test gap.} A standard method for measuring SSL performance is to train a linear classifier---what we call a `linear probe'---on its embeddings and compute its performance on a held out test set. From a learning theory standpoint, one might expect the linear probe's train-test accuracy gap to be indicative of memorization: the more a model overfits, the larger is the difference between train set and test set accuracy. However, as seen in Figure \ref{fig:dejavu epochs train set size}, the linear probe gap (dark blue) fails to reveal memorization captured by the \dejavu score (red) \footnote{See section \ref{sec:mitigation} for further discussion of the \dejavu score trends of Figure \ref{fig:dejavu epochs train set size}.}.

% \paragraph{Effect of training epochs.} 
% Figure \ref{fig:dejavu v. training epochs} shows how \dejavu memorization changes with training epochs for VICReg. The training set size is fixed to 300K samples. We observe that the number of epochs has a very strong influence on the degree of memorization for VICReg. From 250 to 1000 epochs, the \dejavu score (red curve) grows threefold: from under 10\% to over 30\%. Remarkably, this trend in memorization is \emph{not} reflected in the standard metric for evaluating SSL representations: linear probe accuracy. The dark blue curve shows the train-test linear probe accuracy gap. Although there is a sizeable train-test gap, it only changes by a few percent beyond 250 epochs. %Thus, our test can be used as an alternative to linear probe accuracy to evaluate the memorization of SSL models.
% \vspace{-.75em}
\begin{figure}[ht]
\label{fig:dejavu epochs and dataset}
\begin{minipage}[t]{0.49\textwidth}
\centering
     \begin{subfigure}[b]{0.48\textwidth}
         \centering
         \includegraphics[width=\textwidth]{figures/deja_vu_vs_epochs.png}
         \vspace{-1.5em}
         \caption{\dejavu vs. epochs}
         \label{fig:dejavu v. training epochs}
     \end{subfigure}
     \begin{subfigure}[b]{0.48\textwidth}
         \centering
         \includegraphics[width=\textwidth]{figures/deja_vu_vs_n.png}
         \vspace{-1.5em}
         \caption{\dejavu vs. train set size}
         \label{fig:dejavu v. n}
     \end{subfigure}~
     \vspace{-0.5em}
    \caption{
    Effect of training epochs and train set size with VICReg on \dejavu score (red) in comparison with linear probe accuracy train-test gap (dark blue). 
    \textbf{Left:} \dejavu score increases with training epochs, indicating growing memorization while the linear probe baseline decreases significantly.  
    \textbf{Right:} \dejavu score stays roughly constant with training set size suggesting that memorization may be problematic even for large datasets. %By comparison, the baseline \emph{declines} by half, spuriously suggesting less memorization. 
    %Both trends are not captured according to the linear probe train-test gap---a common method to evaluate generalization of SSL representations.}
    }
    \label{fig:dejavu epochs train set size}
\end{minipage}
\hfill
\begin{minipage}[t]{0.49\textwidth}
\centering
     \begin{subfigure}[b]{0.48\textwidth}
         \centering
         \includegraphics[width=\textwidth]{figures/vicreg_samples_epochs.pdf}
         \vspace{-1.5em}
         \caption{\dejavu vs. epochs}
         \label{fig:per sample v. training epochs}
     \end{subfigure}
     \begin{subfigure}[b]{0.48\textwidth}
         \centering
         \includegraphics[width=\textwidth]{figures/vicreg_samples_datasets.pdf}
         \vspace{-1.5em}
         \caption{\dejavu vs. train set size}
         \label{fig:per sample v. n}
     \end{subfigure}~
     \vspace{-0.5em}
    \caption{
    \definecolor{part_blue}{rgb}{0.2824, 0.4706, .8157}
	\definecolor{part_red}{rgb}{0.8392, 0.3725, 0.3725}
	\definecolor{part_orange}{rgb}{0.9333, 0.5216, 0.2902}
    Partition of samples $A_i \in \calA$ into the four categories: {\color{gray}unassociated} (not shown), {\color{part_orange}memorized}, {\color{part_red}misrepresented} and {\color{part_blue}correlated} for VICReg. The {\color{part_orange}memorized} samples---those whose labels are predicted by $\KNN_A$ but not by $\KNN_B$---occupy a significantly larger share of the training set than the {\color{part_red}misrepresented} samples---those predicted by $\KNN_B$ but not $\KNN_A$ by chance. %At 1000 epochs, $\approx 15\%$ of the training set is {\color{part_orange}memorized}. The trends across training epochs and training set sizes are consistent with those observed in Figure \ref{fig:dejavu epochs train set size}
    }
    \label{fig:partition attack main}
    \end{minipage}
\vspace{-1em} 
\end{figure}

\iffalse

\begin{minipage}[t]{0.49\textwidth}
\centering
     \begin{subfigure}[b]{0.48\textwidth}
         \centering
         \includegraphics[width=0.95\textwidth]{figures/deja_vu_vs_parameters.png}
         \vspace{-0.4em}
         \caption{\dejavu vs. capacity}
         \label{fig:dejavu v. capacity}
     \end{subfigure}
     \hfill
     \begin{subfigure}[b]{0.48\textwidth}
          \tiny
          \centering
          \setlength{\tabcolsep}{3pt}
          \begin{tabular}{|c|c|c|}
            \hline
            Criteria & DV & Acc P/B \\
            \hline
            Supervised & 8.9 & 55.3/61.1\\
            \hline
            Byol\citep{grill2020byol} & 8.0& 54.3/59.4\\
            \hline
            SimCLR\citep{chen2020simclr} & 10.0 & 44.2/54.1\\
            \hline
            Dino\citep{Dino} & 14.5 & 26.3/55.7 \\
            \hline
            Barlow T.\citep{zbontar2021barlow} & 30.5 & 33.7/54.4\\
            \hline
            VICReg\citep{vicreg} & \textbf{33.2} & 40.3/55.2\\
            \hline
          \end{tabular}
          \vspace{1.3em}
          % \caption{\dejavu (DV) vs. SSL Criterion}
          \caption{\dejavu (DV) vs. Criterion}
          \label{tab:dejavu vs. criterion}
    \end{subfigure}
    \vspace{-0.5em}
    \caption{
    Comparison of \dejavu score for different architectures and training criteria. \textbf{Left:} \dejavu score with VICReg for resnet (purple) and vision transformer (green) architectures versus number of model parameters. As expected, memorization grows with larger model capacity. This trend is more pronounced for convolutional (resnet) than transformer (ViT) architectures. \textbf{Right:} Comparison of \dejavu score and ImageNet validation accuracy (P: using projector embeddings, B: using backbone embeddings) for various SSL criteria. \textbf{Nearly all SSL models have more memorization than the supervised baseline.} 
    % Effect of training epochs and train set size on \dejavu score.
    % \textbf{Left:} \dejavu score increases with higher number of training epochs, indicating worsening memorization.
    % \textbf{Right:} \dejavu score stays roughly constant with training set size. Both trends are not captured according to the linear probe train-test gap---a common method to evaluate generalization of SSL representations.
    }
\end{minipage}
\vspace{-2em} 
\end{figure}

\begin{figure}[ht]
\begin{minipage}[t]{0.49\textwidth}
\centering
     \begin{subfigure}[b]{0.49\textwidth}
         \centering
         \includegraphics[width=\textwidth]{figures/epochs_lb_attk_epochs_acc_top1_legend.pdf}
         \caption{\dejavu vs. epochs}
         \label{fig:dejavu v. training epochs}
     \end{subfigure}
     \begin{subfigure}[b]{0.49\textwidth}
         \centering
         \includegraphics[width=\textwidth]{figures/epochs_lb_attk_datasets_acc_top1_legend.pdf}
         \caption{\dejavu vs. train set size}
         \label{fig:dejavu v. n}
     \end{subfigure}~
     \begin{subfigure}[b]{0.32\textwidth}
         \centering
         \includegraphics[width=0.8\textwidth]{figures/dejavu_vs_parameters.pdf}
         \caption{\dejavu vs. capacity}
         \label{fig:dejavu v. n}
     \end{subfigure}
    \caption{
    Effect of training epochs and train set size on \dejavu score.
    \textbf{Left:} \dejavu score increases with higher number of training epochs, indicating worsening memorization.
    \textbf{Right:} \dejavu score stays roughly constant with training set size. Both trends are not captured according to the linear probe train-test gap---a common method to evaluate generalization of SSL representations.}
    \end{minipage}
\vspace{-1em} 
\end{figure}

\begin{table}[ht]
  \footnotesize
  \centering
  \begin{tabular}{|c|c|}
    \hline
    Supervised & 8.9\\
    \hline
    SimCLR\citep{chen2020simclr} & 10.0\\
    \hline
    Byol\citep{grill2020byol} & 8.0\\
    \hline
    Dino\citep{Dino} & 14.5\\
    \hline
    Barlow T.\citep{zbontar2021barlow} & 30.5\\
    \hline
    VICReg\citep{vicreg} & \textbf{33.2}\\
    \hline
  \end{tabular}
  \caption{DejaVu Score 20\% Conf for various SSL methods.}
  \label{tab:two-row-table}
\end{table}
\vspace{-1em} 
\fi

\iffalse
\begin{figure}[ht]
\begin{minipage}[t]{.49\textwidth}
\centering
     \begin{subfigure}[b]{0.49\textwidth}
         \centering
         \includegraphics[width=\textwidth]{figures/epochs_lb_attk_epochs_acc_top1_legend.pdf}
         \caption{\dejavu vs. epochs}
         \label{fig:dejavu v. training epochs}
     \end{subfigure}
     \hfill
     \begin{subfigure}[b]{0.49\textwidth}
         \centering
         \includegraphics[width=\textwidth]{figures/epochs_lb_attk_datasets_acc_top1_legend.pdf}
         \caption{\dejavu vs. train set size}
         \label{fig:dejavu v. n}
     \end{subfigure}
\caption{
Effect of training epochs and train set size on \dejavu score.
\textbf{Left:} \dejavu score increases with higher number of training epochs, indicating worsening memorization.
\textbf{Right:} \dejavu score stays roughly constant with training set size. Both trends are not captured according to the linear probe train-test gap---a common method to evaluate generalization of SSL representations.}
\label{fig:dejavu epochs and dataset}
\end{minipage}
\hfill
\begin{minipage}[t]{.49\textwidth}
     \centering
     \begin{subfigure}[b]{0.49\textwidth}
         \centering
         \includegraphics[width=\textwidth]{figures/criteria_epochs.pdf}
         \caption{criteria comparison}
         \label{fig:dejavu v. criteria}
     \end{subfigure}
     \hfill
     \begin{subfigure}[b]{0.49\textwidth}
         \centering
         \includegraphics[width=\textwidth]{figures/architecture_epochs.pdf}
         \caption{architecture comparison}
         \label{fig:dejavu v. arch}
     \end{subfigure}
\caption{
Effect of SSL training criteria and model architectures on \dejavu score.
%the accuracy gap between target model (trained on $\calA$) and reference model (trained on $\calB$) making predictions on their 20\% most confident examples.
\textbf{Left:} \dejavu score for various training criteria.
%Barlow and VICReg have the heaviest degree of memorization, while SimCLR and BYOL have the least. 
%Note that we show detailed reconstructions of SimCLR's training data in Section \ref{sec:visualizing} despite its relatively low degree of \dejavu. 
%Regardless, Although SimCLR and BYOL have the least, we  visualize detailed reconstructions with SimCLR in section \ref{sec:mem v corr} 
All SSL models have significantly more \dejavu than the supervised baseline. \textbf{Right:} \dejavu score versus epochs for various training architectures. As expected, lower capacity architectures (Resnet18, Resnet34) reduce \dejavu but not completely. 
}
\label{fig:dejavu criteria and architecture}
\end{minipage}
\vspace{-1em} 
\end{figure}
\fi
% %\begin{figure}[ht]
%%%
%VICREG
%%%
     \centering
     \begin{subfigure}[b]{0.49\textwidth}
         \centering
         \includegraphics[width=\textwidth]{figures/sample_level_training_epochs.pdf}
         \caption{Categories of training samples vs. number of epochs}
         \label{fig:sample level epochs}
     \end{subfigure}
     \hfill
     \begin{subfigure}[b]{0.49\textwidth}
         \centering
         \includegraphics[width=\textwidth]{figures/sample_level_training_set_size.pdf}
         \caption{Categories of training samples vs. training set size}
         \label{fig:sample level training size}
     \end{subfigure}
\caption{
\definecolor{part_blue}{rgb}{0.2824, 0.4706, .8157}
\definecolor{part_red}{rgb}{0.8392, 0.3725, 0.3725}
\definecolor{part_orange}{rgb}{0.9333, 0.5216, 0.2902}
Partition of samples $A_i \in \calA$ into the four categories: {\color{gray}unassociated} (not shown), {\color{part_orange}memorized}, {\color{part_red}misrepresented} and {\color{part_blue}correlated}. The {\color{part_orange}memorized} samples---ones whose labels are predicted by $\KNN_A$ but not by $\KNN_B$---occupy a significantly larger share for VICReg compared to the supervised model, indicating that sample-level \dejavu memorization is more prevalent in VICReg. %The trends across number of training epochs and training set sizes are consistent with those observed in Figures \ref{fig:dejavu epochs and dataset} and \ref{fig:dejavu criteria and architecture}.
}
\label{fig:partition attack main appendix}
\end{figure}
% \paragraph{Effect of training set size.} 
% Figure \ref{fig:dejavu v. n} shows how \dejavu memorization responds to the model's training set size. The number of training epochs is fixed to 1000. Interestingly, training set size appears to have almost \emph{no} influence on the \dejavu score (red line), indicating that memorization is equally prevalent with a 100K dataset and a 500K dataset (which suggests that \dejavu memorization may be detectable for larger datasets). Meanwhile, the linear probe train-test accuracy gap \emph{declines} by half as the dataset size grows, failing to represent the memorization quantified by our test. 
% The trend is completely different according to linear probe accuracy (dark blue line), the train-test gap shrinks substantially when increasing the training set size from 100K to 500K. This highlights that the train-test gap is not able to capture \dejavu memorization. %Our evidence suggests that \dejavu memorization may be detectable even for large-scale training datasets. 
%\vspace{-.75em}

\vspace{-.75em} 
\subsection{Sample-level Memorization}
\label{sec:dissection}

% Section \ref{sec:label inference accuracy} shows the \emph{average} level of \dejavu memorization on a subset of the training set $\calA$. However, this average tell us only what the attacker success rate might be without explicitly describing how much of the datatset is \dejavu memorized.
The \dejavu score shows, \emph{on average}, how much better an adversary can select and classify images when using the target model trained on them. 
This average score does not tell us how many individual images have their label successfully recovered by $\KNN_A$ but not by $\KNN_B$. In other words, how many images are exposed by virtue of \emph{being in training set} $\calA$: a risk notion foundational to differential privacy. 
% However, from the perspective of an individual image $A_i \in \calA$, it is informative to know whether it was correctly classified 
To better quantify what fraction of the dataset is at risk, we perform a sample-level analysis by fixing a sample $A_i \in \calA$ and observing the label inference result of $\KNN_A$ vs. $\KNN_B$.
To this end, we partition samples $A_i \in \calA$ based on the result of label inference into four distinct categories: {\color{gray}\textbf{Unassociated}} - label inferred with neither KNN; {\color{part_orange}\textbf{Memorized}} - label inferred only with $\KNN_A$; {\color{part_red}\textbf{Misrepresented}} - label inferred only with $\KNN_B$; {\color{part_blue}\textbf{Correlated}} - label inferred with both KNNs. 
% \begin{multicols}{2}
% \begin{itemize}
%     \vspace{-.75em}
%     \setlength\itemsep{0.15em}
%     \item {\color{gray}Unassociated}: label inferred with neither KNN   
%     \item {\color{part_orange}Memorized}: label only inferred by $\KNN_A$
%     \item {\color{part_red}Misrepresented}: label only inferred with $\KNN_B$
%     \item {\color{part_blue}Correlated}: label inferred with both KNNs
%     \vspace{-.75em}
% \end{itemize}
% \end{multicols}
Intuitively, {\color{gray}unassociated} samples are ones where the embedding of $\crop{A_i}$ does not encode information about the label. {\color{part_blue}Correlated} samples are ones where the label can be inferred from $\crop{A_i}$ using correlation, \emph{e.g.}, inferring the foreground object is basketball given a crop showing a basketball player. Ideally, the {\color{part_red}misrepresented} set should be empty but contains a small portion of examples due to chance.
\emph{Déjà vu} memorization occurs for {\color{part_orange}memorized} samples where the embedding of $\SSL_B$ does not encode the label but the embedding of $\SSL_A$ does. To measure the pervasiveness of \dejavu memorization, we compare the size of the {\color{part_orange}memorized} and {\color{part_red}misrepresented} sets.
Figure \ref{fig:partition attack main} shows how the four categories of examples change with number of training epochs and training set size. The {\color{gray}unassociated} set is not shown since the total share adds up to one. The {\color{part_red}misrepresented} set remains under $5\%$ and roughly unchanged across all settings, consistent with our explanation that it is due to chance. In comparison, VICReg's {\color{part_orange}memorized} set surpasses $15\%$ at 1000 epochs. Considering that up to 5\% of these memorized examples could also be due to chance, we conclude that \textbf{at least 10\% of VICReg's training set is \dejavu memorized.} 
%is many times larger than its {\color{part_red}misrepresented} set, indicating substantial sample-level \dejavu memorization. 
%In fact, \textbf{it is 15\% of the training set that is \dejavu memorized with VICReg.}
%The trends across different number of training epochs and training set sizes match those observed in Section \ref{sec:label inference accuracy}. % On the other hand, the supervised model's {\color{part_orange}memorized} set is only marginally larger than its {\color{part_red}misrepresented} set.

% The trends across different number of training epochs and training set sizes match those observed in Section \ref{sec:label inference accuracy}: Increasing the number of epochs increases \dejavu memorization (Figure \ref{fig:per sample v. training epochs}), while increasing the training set size does not appear to reduce \dejavu memorization (Figure \ref{fig:per sample v. n}). 
%\section{Visualizing \emph{Déjà Vu} Memorization}
\label{sec:visualizing}
Beyond enabling label inference using a periphery crop, we show that \dejavu memorization allows the SSL model to encode other forms of information about a training image. Namely, we train an RCDM \citep{RCDM} on the public dataset $\calX$ and use it to visually reconstruct training images given their periphery crop.
We aim to answer the following two questions: \textbf{(1)} Can we visualize the distinction between correlation and \dejavu memorization? \textbf{(2)} What foreground object details can be extracted from the SSL model beyond class label? 
% \begin{enumerate}[noitemsep, leftmargin=*, topsep=0pt]
%     \item Can we visualize the distinction between correlation and \dejavu memorization? 
%     \item What foreground object details can be extracted from the SSL model beyond class label? 
% \end{enumerate}
\vspace{-0.5em}
\paragraph{Reconstruction pipeline.}
RCDM is a conditional generative model that is trained on the \emph{backbone embedding} of images $X_i \in \calX$ to generate an image that resembles $X_i$. All training images are first face-blurred for privacy purposes. \citet{RCDM} showed that the backbone embedding of SSL models contains more low-level information about the image, making them better suited for conditioning the RCDM.
At test time, following the pipeline in Figure \ref{fig:split_and_pipeline_cartoon}, we first use the projector embedding to find the KNN subset for the periphery crop, $\crop{A_i}$, and then average their backbone embeddings as input to the RCDM model. Ideally, when the public set contains enough representative images, the average representation of the KNN subset encodes objects present in $A_i$, and the RCDM model decodes this representation to visualize these objects.
% \begin{figure}[ht]
%%%
%VICREG
%%%
     \centering
     \begin{subfigure}[b]{0.49\textwidth}
         \centering
         \includegraphics[width=\textwidth]{figures/sample_level_training_epochs.pdf}
         \caption{Categories of training samples vs. number of epochs}
         \label{fig:sample level epochs}
     \end{subfigure}
     \hfill
     \begin{subfigure}[b]{0.49\textwidth}
         \centering
         \includegraphics[width=\textwidth]{figures/sample_level_training_set_size.pdf}
         \caption{Categories of training samples vs. training set size}
         \label{fig:sample level training size}
     \end{subfigure}
\caption{
\definecolor{part_blue}{rgb}{0.2824, 0.4706, .8157}
\definecolor{part_red}{rgb}{0.8392, 0.3725, 0.3725}
\definecolor{part_orange}{rgb}{0.9333, 0.5216, 0.2902}
Partition of samples $A_i \in \calA$ into the four categories: {\color{gray}unassociated} (not shown), {\color{part_orange}memorized}, {\color{part_red}misrepresented} and {\color{part_blue}correlated}. The {\color{part_orange}memorized} samples---ones whose labels are predicted by $\KNN_A$ but not by $\KNN_B$---occupy a significantly larger share for VICReg compared to the supervised model, indicating that sample-level \dejavu memorization is more prevalent in VICReg. %The trends across number of training epochs and training set sizes are consistent with those observed in Figures \ref{fig:dejavu epochs and dataset} and \ref{fig:dejavu criteria and architecture}.
}
\label{fig:partition attack main appendix}
\end{figure}
%\begin{figure*}[t!]
%%%
%DAM
%%%
     \centering
     \begin{subfigure}[b]{0.49\textwidth}
         \centering
         \includegraphics[width=\textwidth]{figures/dam_corr.png}
         \caption{A {\color{part_blue}correlated} dam example}
         \label{fig:dam correlated}
     \end{subfigure}
     \hfill
     \begin{subfigure}[b]{0.49\textwidth}
         \centering
         \includegraphics[width=\textwidth]{figures/dam_mem.png}
         \caption{A {\color{part_orange}memorized} dam example}
         \label{fig:dam memorized}
     \end{subfigure}
\caption{
{\color{part_blue}Correlated} and {\color{part_orange}Memorized} examples from the \emph{dam} class. Both $\SSL_A$ and $\SSL_B$ are SimCLR models.
\textbf{Left:} The periphery crop (pink square) contains a concrete structure that is often present in images of dams. Consequently, the trained RCDM can reconstruct the foreground object using representations from both $\SSL_A$ and $\SSL_B$ through this correlation.
\textbf{Right:} The periphery crop only contains a patch of water. The embedding produced by $\SSL_B$ only contains enough information to infer that the foreground object is related to water, as reflected by its KNN set and RCDM reconstruction. In contrast, the embedding produced by $\SSL_A$ memorizes the association of this patch of water with dam and the RCDM can visualize the embedding to produce images of dams.
}
\vspace{-1ex}
\label{fig:mem v corr dam}
\end{figure*}


\begin{figure*}[t!]
%%%
%DAM
%%%
     \centering
     \begin{subfigure}[b]{0.49\textwidth}
         \centering
         \includegraphics[width=\textwidth]{figures/dam_corr.png}
         \caption{A {\color{part_blue}correlated} dam example}
         \label{fig:dam correlated}
     \end{subfigure}
     \hfill
     \begin{subfigure}[b]{0.49\textwidth}
         \centering
         \includegraphics[width=\textwidth]{figures/dam_mem.png}
         \caption{A {\color{part_orange}memorized} dam example}
         \label{fig:dam memorized}
     \end{subfigure}
\caption[Correlated and Memorized examples from the \emph{dam} class.]{
Correlated and Memorized examples from the \emph{dam} class. Both $\SSL_A$ and $\SSL_B$ are SimCLR models.
\textbf{Left:} The periphery crop (pink square) contains a concrete structure that is often present in images of dams. Consequently, the trained RCDM can reconstruct the foreground object using representations from both $\SSL_A$ and $\SSL_B$ through this correlation.
\textbf{Right:} The periphery crop only contains a patch of water. The embedding produced by $\SSL_B$ only contains enough information to infer that the foreground object is related to water, as reflected by its KNN set and RCDM reconstruction. In contrast, the embedding produced by $\SSL_A$ memorizes the association of this patch of water with dam and the RCDM can visualize the embedding to produce images of dams.
}
\label{fig:mem v corr dam}
\end{figure*}


\begin{figure}[t!]
%%%
%BADGER
%%%
     \centering
     \begin{subfigure}[b]{0.49\textwidth}
         \centering
         \includegraphics[width=\textwidth]{figures/euro_badgers.png}
         \caption{{\color{part_orange}Memorized} European badgers}
         \label{fig:euro badgers}
     \end{subfigure}
     \hfill
     \begin{subfigure}[b]{0.49\textwidth}
         \centering
         \includegraphics[width=\textwidth]{figures/amer_badgers.png}
         \caption{{\color{part_orange}Memorized} American badgers}
         \label{fig:amer badgers}
     \end{subfigure}
\caption[Visualization of \dejavu memorization beyond class label.]{
Visualization of \dejavu memorization beyond class label. Both $\SSL_A$ and $\SSL_B$ are VICReg models. 
The four images shown belong to the memorized set of $\SSL_A$ from the \emph{badger} class. RCDM reconstruction using embeddings from $\SSL_A$ can reveal not only the correct class label, but also the specific badger species: \emph{European} (left) and \emph{American} (right). Such information does not appear to be memorized by the reference model $\SSL_B$.
} 
\label{fig:in class badger}
\end{figure}


% \subsection{Visualizing Correlation vs. Memorization}
\label{sec:mem v corr}
\vspace{-0.5em} 
\paragraph{Visualizing Correlation vs. Memorization.}
Figure \ref{fig:mem v corr dam} shows examples of dams from the {\color{part_blue}correlated} set (left) and the {\color{part_orange}memorized} set (right) as defined in Section \ref{sec:dissection}, along with the associated KNN set and RCDM reconstruction. Both $\SSL_A$ and $\SSL_B$ are SimCLR models. In Figure \ref{fig:dam correlated}, the periphery crop is represented by the pink square, which contains concrete structure attached to the dam's main structure. As a result, both $\SSL_A$ and $\SSL_B$ produce embeddings of $\crop{A_i}$ whose KNN set in $\calX$ consist of dams, \emph{i.e.}, there is a correlation between the concrete structure in $\crop{A_i}$ and the foreground dam. The RCDM reconstructions also consist of dams or structures that closely resemble dams. 
In Figure \ref{fig:dam memorized}, the periphery crop only contains a patch of water, which does not strongly correlate with dams in the ImageNet distribution. Evidently, the reference model $\SSL_B$ embeds $\crop{A_i}$ close to that of other objects commonly found in water, such as sea turtle and submarine. In contrast, the KNN set according to $\SSL_A$ all contain dams despite the vast number of alternative possibilities within the ImageNet classes, and the RCDM reconstruction outputs dams as well which highlight memorization in $\SSL_A$ between this specific patch of water and the dam. %\footnote{See Appendix \ref{sec:appx visualization} to see the same trend in the \emph{yellow garden spider} class.}


% \subsection{Visualizing Memorization Beyond Class Label}
% \label{sec:in class variation}
\vspace{-0.5em} 
\paragraph{Visualizing Memorization Beyond Class Label.}
We now use our reconstruction algorithm to show that \dejavu memorization can be exploited to reveal detailed information beyond class label. Figure \ref{fig:in class badger} shows four examples of badgers from the {\color{part_orange}memorized} set. In all four images, the periphery crop (pink square) does not contain any indication that the foreground object is a badger. Despite this, the KNN set and the RCDM reconstruction using $\SSL_A$ consistently produce images of badgers, while the same does not hold for $\SSL_B$.
More interestingly, reconstructions using $\SSL_A$ in Figure \ref{fig:euro badgers} all contain \emph{European} badgers, while reconstructions in Figure \ref{fig:amer badgers} all contain \emph{American} badgers, accurately reflecting the species of badger present in the respective training images. Since ImageNet-1K does \emph{not} differentiate between these two species of badgers, our reconstructions show that SSL models can memorize information that is highly specific to a training sample beyond its class label\footnote{See Appendix \ref{sec:appx visualization} for additional visualization experiments.}.%\footnote{See Appendix \ref{sec:appx visualization} for the same trend in the \emph{aircraft carrier} class.}.





%\vspace{-.5em} 
\section{Mitigation of \dejavu memorization}
\label{sec:mitigation}
% We do not have an understanding on why \dejavu occur so strongly in some SSL pretraining, however we present additional experiments that shed light on which parameters have the biggest impact on \dejavu memorization.
\begin{figure}[ht]
\label{fig:mitigations}
\begin{minipage}[t]{0.5\textwidth}
\centering
     \begin{subfigure}[b]{0.47\textwidth}
         \centering
         \includegraphics[width=\textwidth]{figures/dejavu_vicreg_param.png}
         \vspace{-1.5em}
         \caption{Loss hyper-parameter}
         \label{fig:dejavu v. invariance}
     \end{subfigure}
     \begin{subfigure}[b]{0.49\textwidth}
         \centering
         \includegraphics[width=\textwidth]{figures/deja_vu_vs_layer.png}
         \vspace{-1.5em}
         \caption{Guillotine regularization}
         \label{fig:dejavu v. guillotine}
     \end{subfigure}~
     \vspace{-0.5em}
    \caption[Effect of two kinds of hyper-parameters on VICReg memorization. ]{
    Effect of two kinds of hyper-parameters on VICReg memorization. \textbf{Left:} \dejavu score (red) versus the \emph{invariance} loss parameter, $\lambda$, used in the VICReg criterion (100k dataset). Larger $\lambda$ significantly reduces \dejavu, with minimal effect on linear probe validation performance (green). $\lambda = 25$ (near maximum \dejavu) is recommended in the original paper \textbf{Right:} \dejavu score versus projector layer---guillotine regularization \cite{Guillotine}---from projector to backbone. Removing the projector can significantly reduce \dejavu. Appendix \ref{sec:guillotine} shows that the backbone still can memorize, however; we demonstrate reconstructions using the SimCLR backbone.
    }
\end{minipage}
\hfill
\begin{minipage}[t]{0.48\textwidth}
\centering
     \begin{subfigure}[b]{0.46\textwidth}
         \centering
         \includegraphics[width=\textwidth]{figures/deja_vu_vs_parameters.png}
         \vspace{-1.3em}
         \caption{\dejavu vs. capacity}
         \label{fig:dejavu v. capacity}
     \end{subfigure}
     \hfill
     \begin{subfigure}[b]{0.52\textwidth}
          \tiny
          \centering
          \setlength{\tabcolsep}{3pt}
          \begin{tabular}{|c|c|c|}
            \hline
            Criteria & DV & Acc P/B \\
            \hline
            Supervised & 8.9 & 55.3/61.1\\
            \hline
            Byol\citep{grill2020byol} & 8.0& 54.3/59.4\\
            \hline
            SimCLR\citep{chen2020simclr} & 10.0 & 44.2/54.1\\
            \hline
            Dino\citep{Dino} & 14.5 & 26.3/55.7 \\
            \hline
            Barlow T.\citep{zbontar2021barlow} & 30.5 & 33.7/54.4\\
            \hline
            VICReg\citep{vicreg} & \textbf{33.2} & 40.3/55.2\\
            \hline
          \end{tabular}
          \vspace{1.3em}
          % \caption{\dejavu (DV) vs. SSL Criterion}
          \caption{\dejavu (DV) vs. Criterion}
          \label{tab:dejavu vs. criterion}
    \end{subfigure}
    \vspace{-1.4em}
    \caption[Effect of model architecture and criterion on \dejavu memorization.]{
    %Comparison of \dejavu score for different architectures and training criteria. 
    Effect of model architecture and criterion on \dejavu memorization. 
    \textbf{Left:} \dejavu score with VICReg for resnet (purple) and vision transformer (green) architectures versus number of model parameters. As expected, memorization grows with larger model capacity. This trend is more pronounced for convolutional (resnet) than transformer (ViT) architectures. \textbf{Right:} Comparison of \dejavu score 20\% conf. and ImageNet linear probe validation accuracy (P: using projector embeddings, B: using backbone embeddings) for various SSL criteria. %\textbf{Nearly all SSL models have more memorization than the supervised baseline.} 
    % Effect of training epochs and train set size on \dejavu score.
    % \textbf{Left:} \dejavu score increases with higher number of training epochs, indicating worsening memorization.
    % \textbf{Right:} \dejavu score stays roughly constant with training set size. Both trends are not captured according to the linear probe train-test gap---a common method to evaluate generalization of SSL representations.
    }
    \end{minipage}
\end{figure}
We cannot yet make claims on why \dejavu occurs so strongly for some SSL training settings and not for others. To gain some intuition for future work, we present additional observations that shed light on which parameters have the most salient impact on \dejavu memorization.
\vspace{-.75em}
\paragraph{Déjà vu memorization worsens by increasing number of training epochs.} 
Figure \ref{fig:dejavu v. training epochs} shows how \dejavu memorization changes with number of training epochs for VICReg. The training set size is fixed to 300K samples. From 250 to 1000 epochs, the \dejavu score (red curve) grows \emph{threefold}: from under 10\% to over 30\%. Remarkably, this trend in memorization is \emph{not} reflected by the linear probe gap (dark blue), which only changes by a few percent beyond 250 epochs. 

%\vspace{-.75em}
\paragraph{Training set size has minimal effect on \dejavu memorization.} Figure \ref{fig:dejavu v. n} shows how \dejavu memorization responds to the model's training set size. The number of training epochs is fixed to 1000. Interestingly, training set size appears to have almost \emph{no} influence on the \dejavu score (red line), indicating that memorization is equally prevalent with a 100K dataset and a 500K dataset. This result suggests that \dejavu memorization may be detectable even for large datasets. Meanwhile, the standard linear probe train-test accuracy gap \emph{declines} by more than half as the dataset size grows, failing to represent the memorization quantified by our test. 
% The trend is completely different according to linear probe accuracy (dark blue line), the train-test gap shrinks substantially when increasing the training set size from 100K to 500K. This highlights that the train-test gap is not able to capture \dejavu memorization. Our evidence suggests that \dejavu memorization may be detectable even for large-scale training datasets. 
\vspace{-0.5em}
\paragraph{Training loss hyper-parameter has a strong effect.} 
%We show in Figure \ref{fig:dejavu v. training epochs} that the number of training epochs is an important factor that can increase significantly \dejavu memorization. In contrast, the dataset size does not impact much \dejavu as shown in Figure \ref{fig:dejavu epochs train set size}. 
Loss hyper-parameters, like VICReg's invariance coefficient (Figure \ref{fig:dejavu v. invariance}) or SimCLR's temperature parameter (Appendix Figure \ref{fig:simclr temperature}) significantly impact \dejavu with minimal impact on the linear probe validation accuracy.

\vspace{-0.5em}
\paragraph{Some SSL criteria promote stronger \dejavu memorization.} Table \ref{tab:dejavu vs. criterion} demonstrates that the degree of memorization varies widely for different training criteria. VICReg and Barlow Twins have the highest \dejavu scores while SimCLR and Byol have the lowest.
%\footnote{We show detailed reconstructions of SimCLR's training data in Section \ref{sec:visualizing} despite its relatively low degree of \dejavu.}.
With the exception of Byol, all SSL models have more \dejavu memorization than the supervised model. Interestingly, different criteria can lead to similar linear probe validation accuracy and very different degrees of \dejavu as seen with SimCLR and Barlow Twins. Note that low degrees of \dejavu can still risk training image reconstruction, as exemplified by the SimCLR reconstructions in Figures \ref{fig:mem v corr dam} and \ref{fig:mem v corr spider}. 
%\vspace{-1em}
\vspace{-0.5em}
\paragraph{Larger models have increased \dejavu memorization.} Figure \ref{fig:dejavu v. capacity} validates the common intuition that lower capacity architectures (Resnet18/34) result in less memorization than their high capacity counterparts (Resnet50/101). 
% \begin{wrapfigure}{r}{0.25\textwidth} 
%     \centering
%     \includegraphics[width=0.25\textwidth]{figures/attk_layer_acc_top1_legend.pdf}
%     \caption{\dejavu memorization versus layer from backbone (0) to projector output (3).}
%     \label{fig:dejavu vs layer}
%     \vspace{-8ex}
% \end{wrapfigure}
We see the same trend for vision transformers as well. %This comes with a tradeoff, since reduced model capacity can result in a nontrivial degradation of representation quality\cite{vicreg, simclr}.  
\vspace{-0.5em}
\paragraph{Guillotine regularization can help reduce \dejavu memorization.} Previous experiments were done using the projector embedding. In Figure \ref{fig:dejavu v. guillotine}, we present how Guillotine regularization\citep{Guillotine} (removing final layers in a trained SSL model) impacts \dejavu with VICReg\footnote{Further experiments are available in Appendix \ref{sec:guillotine}.}. Using the backbone embedding instead of the projector embedding seems to be the most straightforward way to mitigate \dejavu memorization. However, as demonstrated in Appendix \ref{sec:appx backbone results}, backbone representation with low \dejavu score can still be leveraged to reconstruct some of the training images.

\section{Conclusion}
\label{sec:conclusion}

We defined and analyzed \dejavu memorization, a notion of unintended memorization of partial information in image data. As shown in Sections \ref{sec:quant} and \ref{sec:visualizing}, SSL models can largely exhibit \dejavu memorization on their training data, and this memorization signal can be extracted to infer or visualize image-specific information.
Since SSL models are becoming increasingly widespread as foundation models for image data, negative consequences of \dejavu memorization can have profound downstream impact and thus deserves further attention. 
Future work should focus on understanding how \dejavu emerges in the training of SSL models and why methods like Byol are much more robust to \dejavu than VICReg and Barlow Twins. In addition, trying to characterize which data points are the most at risk of \dejavu could be crucial to get a better understanding on this phenomenon. 

\graphicspath{{./chapters/chapter3/}}
\chapter{Sentence-level Privacy for Document Embeddings}

\newcommand{\calS}{\mathcal{S}}
%\newcommand{\calX}{\mathcal{X}}
\newcommand{\calM}{\mathcal{M}}
\newcommand{\calD}{\mathcal{D}}
%\newcommand{\calA}{\mathcal{A}}
\newcommand{\calO}{\mathcal{O}}
\newcommand{\calU}{\mathcal{U}}
%\newcommand{\R}{\mathbb{R}}
%\newcommand{\cm}[1]{\textcolor{blue}{#1}}
\newcommand{\SDP}{SentDP}
\newcommand{\needcite}{\textcolor{red}{[cite]}}
\newcommand{\mname}{\calM_{\text{TD}}}
\newcommand{\tmed}{\text{T}_{\text{MED}}}
\newcommand{\argmax}[1]{\underset{#1}{\text{arg max }}}
\newcommand{\refsec}{\textcolor{red}{[ref section]}}
\newcommand{\tdappx}{\widehat{\text{TD}}}
\newcommand{\technique}{\textsc{d}eep\textsc{c}andidate}
\newcommand{\MLP}[1]{$\textbf{MLP}^{#1}$}

\newcommand\mycommfont[1]{\small\ttfamily\textcolor{blue}{#1}}
\SetCommentSty{mycommfont}

\newenvironment{squishlist}
{ \begin{enumerate}
    \setlength{\itemsep}{0pt}
    \setlength{\parskip}{0pt}
    \setlength{\parsep}{0pt}     }
{ \end{enumerate}    }

%datasets
\newcommand{\goodreads}{\emph{Good Reads}}
\newcommand{\imdb}{\emph{IMDB}}
\newcommand{\tnews}{\emph{20 News Groups}}


\newtheorem{theorem}{Theorem}[section]
\newtheorem{corollary}{Corollary}[theorem]
%\newtheorem{lemma}[theorem]{Lemma}
\newtheorem{remark}[theorem]{Remark}


%\theoremstyle{definition}
%\newtheorem{definition}{Definition}[section]
\section{Introduction} 
\label{sec:intro} 



Language models have now become ubiquitous in NLP \cite{devlin2019bert, liu2019roberta, alsentzer2019publicly}, pushing the state of the art in a variety of tasks \cite{strubell2018linguistically, liu2019multi, mrini-etal-2021-recursive}. While language models capture meaning and various linguistic properties of text \cite{jawahar2019does, yenicelik2020does}, an individual's written text can include highly sensitive information. Even if such details are not needed or used, sensitive information has been found to be vulnerable and detectable to attacks \cite{pan2020privacy, attack_word_embs, carlini_attack}. Reconstruction attacks \cite{xie2021reconstruction} have even successfully broken through private learning schemes that rely on encryption-type methods \cite{huang-etal-2020-texthide}.

As of now, there is no broad agreement on what constitutes good privacy for natural language \cite{kairouz2019advances}. \citet{huang-etal-2020-texthide} argue that different applications and models require different privacy definitions. Several emerging works propose to apply Metric Differential Privacy \cite{orig_metricdp} at the word level \cite{metricdp,  mdp_low_dim, TEM, another_metric_DP, fancy_metricdp, metricDP_gumbel} . They propose to add noise to word embeddings, such that they are indistinguishable from their nearest neighbours.

At the document level, however, the above definition has two areas for improvement. First, it may not offer the level of privacy desired. Having each word indistinguishable with similar words may not hide higher level concepts in the document, and may not be satisfactory for many users. Second, it may not be very interpretable or easy to communicate to end-users, since the privacy definition relies fundamentally on the choice of embedding model to determine which words are indistinguishable with a given word. This may not be clear and precise enough for end-users to grasp.
%
%The above definition is straightforward to implement and naturally takes advantage of the structure in precomputed word embeddings. At the document level, however, there are two areas for improvement. First, it may not offer the level of privacy desired. Having each word indistinguishable only with similar words may not be satisfactory for many users. Replacing words with similar words does not necessarily hide higher level concepts in the document. Second, it may not be very interpretable or easy to communicate to end-users. The privacy definition relies fundamentally on the choice of embedding model. Stating which words are indistinguishable with a given word requires querying the embedding model. This may not be a clear and precise enough for end users to grasp.

\begin{figure}
	\centering
	\includegraphics[width = 0.8\linewidth]{figures/first_pg.png}
	\label{fig:first page}
	\vspace{0cm}
	\caption{$x$ and $x'$ yield $z \in \R^d$ with similar probability.}
\end{figure}
 
\begin{figure*}
	\centering
	\vspace{-1cm}
	\includegraphics[width = \linewidth]{figures/block_diagram.png}
	\label{fig:block diagram}
	\vspace{-0.65cm}
	\caption[\technique\ generates a private embedding $z$ of document $x$ by selecting from a set $F$ of public, non-private document embeddings.]{\technique\ generates a private embedding \textcolor{green}{$z$} of document \textcolor{red}{$x$} by selecting from a set \textcolor{blue}{$F$} of public, non-private document embeddings. Sentences from \textcolor{red}{$x$} are encoded by $G'$. The privacy mechanism $\mname$, then privately samples from \textcolor{blue}{$F$}, with a preference for candidates with high Tukey Depth, `deep candidates'. $G'$ is trained beforehand to ensure that deep candidates are likely to exist and are relevant to \textcolor{red}{$x$}.}
\end{figure*}

In this work, we propose a new privacy definition for documents: sentence privacy. This guarantee is both strong and interpretable: any sentence in a document must be indistinguishable with \emph{any} other sentence. A document embedding is sentence-private if we can replace any single sentence in the document and have a similar probability of producing the same embedding. As such, the embedding only stores limited information unique to any given sentence. This definition is easy to communicate and strictly stronger than word-level definitions, as modifying a sentence can be changing one word.

Although this definition is strong, we are able to produce unsupervised, general embeddings of documents that are useful for downstream tasks like sentiment analysis and topic classification. To achieve this we propose a novel privacy mechanism, \technique, which privately samples a high-dimensional embedding from a preselected set of candidate embeddings derived from public, non-private data. \technique\  works by first pre-tuning a sentence encoder on public data such that semantically different document embeddings are far apart from each other. Then, we approximate each candidate's Tukey Depth within the private documents' sentence embeddings. Deeper candidates are the most likely to be sampled to represent the private document. We evaluate \technique\  on three illustrative datasets, and show that these unsupervised private embeddings are useful for both sentiment analysis and topic classification as compared to baselines. 

In summary, this work makes the following contributions to the language privacy literature:

\begin{squishlist}
	\item A new, strong, and interpretable privacy definition that offers complete indistinguishability to each sentence in a document. 
	\item A novel, unsupervised embedding technique, \technique, to generate sentence-private document embeddings. 
	\item An empirical assessment of \technique, demonstrating its advantage over baselines, delivering strong privacy and utility. 
\end{squishlist}
\section{Background and Related Work}

\paragraph{Setting.}We denote a `document' as a sequence of sentences. Let $s \in \calS$ be any finite-length sentence. Then, the space of all documents is $\calX = \calS^*$ and document $x \in \calX$ is written as $x = (s_1, s_2, \dots, s_k)$ for any non-negative integer $k$ of sentences. In this work, we focus on cohesive documents of sentences written together like reviews or emails, but our methods and guarantees apply to any sequence of sentences, such as a collection of messages written by an individual over some period of time.

Our task is to produce an embedding $z \in \R^d$ of any document $x \in \calX$ such that any single sentence $s_i \in x$ is indistinguishable with every other sentence $s_i' \in \calS \backslash s_i$. That is, if one were to replace any single sentence in the document $s_i \in x$ with \emph{any other} sentence $s_i' \in \calS \backslash s_i$, the probability of producing a given embedding $z$ is similar. To achieve this, we propose a randomized embedding function (the embedding \emph{mechanism}) $\calM : \calX \rightarrow \R^d$, that generates a private embedding $z = \calM(x)$ that is useful for downstream tasks. 

\subsection{Differential Privacy}
The above privacy notion is inspired by Differential Privacy (DP) \cite{DP}. It guarantees that --- whether an individual participates (dataset $D$) or not (dataset $D'$) --- the probability of any output only chances by a constant factor. 

\begin{definition}[Differential Privacy]
	Given any pair of datasets $D, D' \in \calD$ that differ only in the information of a single individual, we say that the mechanism $\calA : \calD \rightarrow \calO$, satisfies $\epsilon$-DP if 
	\begin{align*}
		\Pr[\calA(D) \in O] \leq e^\epsilon \Pr[\calA(D') \in O]
	\end{align*}
	for any event $O \subseteq \calO$. 
\end{definition}
Note that we take probability over the randomness of the mechanism $\calA$ only, not the data distribution. DP has several nice properties that make it easy to work with including closure under post-processing, an additive privacy budget (composition), and closure under group privacy guarantees (guarantees to a \emph{subset} of multiple participants). See \citealt{DPbook} for more details. 

%A useful property of DP is closure under post-processing: if an $\epsilon$-DP mechanism produces output $o$, then anything derived from $o$, $u = f(o)$, also satisfies $\epsilon$-DP. 
%
%\begin{corollary}[Post-processing]
%\label{cor:post processing}
%	If $\calA : \calD \rightarrow \calO$ satisfies $\epsilon$-DP, then so does $f \circ \calA$ for any (possibly randomized) function $f : \calO \rightarrow \calU$. 
%\end{corollary}
%
%Additionally, if we release $r$ queries from a given dataset, each of which satisfies $\epsilon$-DP, the \emph{composition} of these satisfies at least $r\epsilon$-DP. 
%
%\begin{corollary}[Composition]
%	If mechanisms $\calA_1, \calA_2, \dots, \calA_r$ each satisfy $\epsilon$-DP, then the composition of them $\calA(D) = \big(\calA_1(D), \calA_2(D), \dots, \calA_r(D)\big)$ satisfies $\epsilon'$-DP for some $\epsilon' \leq r\epsilon$. 
%\end{corollary}
%
%Finally, DP offers closure under \emph{group privacy}. If two datasets differ in $r \geq 1$ elements, they are still indistinguishable with $r \epsilon$-DP. 
%
%\begin{corollary}[Group privacy]
%\label{cor:group privacy} 
%	If $\calA : \calD \rightarrow \calO$ satisfies $\epsilon$-DP, then 
%	\begin{align*}
%		\Pr[\calA(D) \in O] \leq e^{r\epsilon} \Pr[\calA(D') \in O]
%	\end{align*}
%	for any pair of datasets $D, D' \in \calD$ that differ in $r$ elements. 
%\end{corollary} 
%
The \emph{exponential mechanism} \cite{exp_mech} allows us to make a DP selection from an arbitrary output space $\calO$ based on private dataset $D$. A \emph{utility function} over input/output pairs, $u : \calD \times \calO \rightarrow \R$ determines which outputs are the best selection given dataset $D$. The log probability of choosing output $o \in \calO$ when the input is dataset $D \in \calD$ is then proportional to its utility $u(D,o)$. The \emph{sensitivity} of $u(\cdot, \cdot)$ is the worst-case change in utility over pairs of neighboring datasets $(D,D')$ that change in one entry, $\Delta u = \max_{D, D', o} | u(D,o) - u(D', o)|$. 
\begin{definition}
\label{def: exp mech} 
	The \emph{exponential mechanism} $\calA_{Exp}: \calD \rightarrow \calO$ is a randomized algorithm with output distribution
	\vspace{-0.3cm}
	\begin{align*}
	\Pr[\calA_{Exp}(D) = o] \propto \exp\big( \frac{\epsilon u(D, o)}{2 \Delta u} \big) \quad .
	\end{align*}
\end{definition}

\subsection{Related Work}
\paragraph{Natural Language Privacy.} Previous work has demonstrated that NLP models and embeddings are vulnerable to reconstruction attacks \cite{carlini_attack, attack_word_embs, pan2020privacy}. In response there have been various efforts to design privacy-preserving techniques and definitions across NLP tasks. A line of work focuses on how to make NLP model training satisfy DP \cite{DP_training, DP_training_II}. This is distinct from our work in that it satisfies central DP -- where data is first aggregated non-privately and then privacy preserving algorithms (i.e. training) are run on that data. We model this work of the \emph{local} version of DP \cite{ldp}, wherein each individual's data is made private before centralizing. Our definition guarantees privacy to a single document as opposed to a single individual. 

A line of work more comparable to our approach makes documents locally private by generating a randomized version of a document that satisfies some formal privacy definition. As with the private embedding of our work, this generates locally private \emph{representation} of a given document $x$. The overwhelming majority of these methods satisfy an instance of Metric-DP \cite{orig_metricdp} at the word level \cite{metricdp,  mdp_low_dim, TEM, another_metric_DP, fancy_metricdp, metricDP_gumbel}. As discussed in the introduction, this guarantees that a document $x$ is indistinguishable with any other document $x'$ produced by swapping a single word in $x$ with a similar word. Two words are `similar' if they are close in the word embeddings space (e.g. GloVe). This guarantee is strictly weaker than our proposed definition, \SDP, which offers indistinguishability to any two documents that differ in an entire sentence. 

\paragraph{Privacy-preserving embeddings.} There is a large body of work on non-NLP privacy-preserving embeddings, as these embeddings have been shown to be vulnerable to attacks \cite{attack_on_embeddings}. \citet{clifton} attempt to generate locally private embeddings by bounding the embedding space, and we compare with this method in our experiments. \citet{kamath_high_dim} propose a method for privately publishing the average of embeddings, but their algorithm is not suited to operate on the small number of samples (sentences) a given document offers. Finally, \citet{private_halfspaces} propose a method for privately learning halfspaces in $\R^d$, which is relevant to private Tukey Medians, but their method would restrict input examples (sentence embeddings) to a finite discrete set in $\R^d$, a restriction we cannot tolerate. 


\section{Sentence-level Privacy}
We now introduce our simple, strong privacy definition, along with concepts we use to satisfy it. 
\subsection{Definition}
In this work, we adopt the \emph{local} notion of DP \cite{ldp}, wherein each individual's data is guaranteed privacy locally before being reported and centralized. Our mechanism $\calM$ receives a single document from a single individual, $x \in \calX$. We require that $\calM$ provides indistinguishability between documents $x, x'$ differing \emph{in one sentence}. 
%We call documents $x,x'$ \emph{neighboring documents}. 

\begin{definition}[Sentence Privacy, \SDP]
Given any pair of documents $x, x' \in \calX$ that differ only in one sentence, we say that a mechanism\\ $\calM : \calX \rightarrow \calO$ satisfies $\epsilon$-\SDP~if 
	\begin{align*}
		\Pr[\calM(x) \in O] \leq e^\epsilon \Pr[\calM(x') \in O]
	\end{align*}
	for any event $O \subseteq \calO$. 
\end{definition}

We focus on producing an embedding of the given document $x$, thus the output space is $\calO = \R^d$. For instance, consider the neighboring documents $x = (s_1, s_2, \dots, s_k)$ and $x' = (s_1, s_2', \dots, s_k)$ that differ in the second sentence, i.e. $s_2, s_2'$ can be \emph{any} pair of sentences in $\calS^2$. 
%If our embedding mechanism $\calM$ satisfies $\epsilon$-SentDP, we are guaranteed that 
%\begin{align*}
%	\log \bigg| \frac{\Pr[\calM(x) = z]}{\Pr[\calM(x') = z]} \bigg| \leq \epsilon 
%\end{align*}
%for any released embedding $z \in \R^d$. Thus for low values of $\epsilon$, no one sentence in the document can be reliably reconstructed from the embedding $z$ alone. 
This is a strong notion of privacy in comparison to existing definitions across NLP tasks. However, we show that we can guarantee SentDP while still providing embeddings that are useful for downstream tasks like sentiment analysis and classification. In theory, a SentDP private embedding $z$ should be able to encode any information from the document that is not unique to a small subset of sentences. For instance, $z$ can reliably encode the sentiment of $x$ as long as \emph{multiple} sentences reflect the sentiment. By the group privacy property of DP, which \SDP~maintains, two documents differing in $a$ sentences are $a\epsilon$ indistinguishable. So, if more sentences reflect the sentiment, the more $\calM$ can encode this into $z$ without compromising on privacy. 

\subsection{Sentence Mean Embeddings} 

Our approach is to produce a private version of the average of general-purpose sentence embeddings. By the post-processing property of DP, this embedding can be used repeatedly in any fashion desired without degrading the privacy guarantee. Our method makes use of existing pre-trained sentence encoding models. We denote this general sentence encoder as $G : \calS \rightarrow \R^d$. We show in our experiments that the mean of sentence embeddings,  
\begin{align}
	\overline{g}(x) = \frac{1}{k} \sum_{s_i \in x} G(s_i) \ , 
	\label{eqn: doc emb}
\end{align}
maintains significant information unique to the document and is useful for downstream tasks like classification and sentiment analysis.

We call $\overline{g}(x)$ the \emph{document embedding} since it summarizes the information in document $x$. While there exist other definitions of document embeddings \cite{yang2016hierarchical, thongtan2019sentiment, bianchi2020pre}, we decide to use averaging as it is a simple and established embedding technique \cite{bojanowski2017enriching, gupta2019better, li2020sentence}.
\subsection{Tukey Depth}
Depth is a concept in robust statistics used to describe how central a point is to a distribution. We borrow the definition proposed by \citet{tukeydepth}:

\begin{definition}
\label{def: tukey} 
	Given a distribution $P$ over $\R^d$, the Tukey Depth of a point $y \in \R^d$ is 
\begin{align*}
	\text{TD}_P(y) 
	&= \inf_{w \in \R^d} P\{y' : w \cdot (y' - y) \geq 0\} \quad. 
\end{align*} 
\end{definition}
In other words, take the hyperplane orthogonal to vector $w$, $h_w$, that passes through point $y$. Let $P_1^w$ be the probability under $P$ that a point lands on one side of $h_w$ and let $P_2^w$ be the probability that a point lands on the other side, so $P_1^w + P_2^w = 1$. $y$ is considered deep if $\min(P_1^w, P_2^w)$ is close to a half for \emph{all} vectors $w$ (and thus all $h$ passing through $y$). The \emph{Tukey Median} of distribution $P$, $\tmed(P)$, is the set of all points with maximal Tukey Depth, 
\begin{align}
	\tmed(P) = \argmax{y \in \R^d} \text{TD}_P(y) \quad .
	\label{eqn:tukey median}
\end{align}
We only access the distribution $P$ through a finite sample of i.i.d. points, $Y = \{y_1, y_2, \dots, y_n\}$. The Tukey Depth w.r.t. $Y$ is given by 
\begin{align*}
	\text{TD}_Y(y) = \inf_{w \in \R^d} |\{y' \in Y : w \cdot (y' - y) \geq 0\}| \ , 
%	\label{eqn:tukey median discrete}
\end{align*}
and the median, $\tmed(Y)$, maximizes the depth and is at most half the size of our sample $\big \lfloor \frac{n}{2} \big  \rfloor$. 

Generally, finding a point in $\tmed(Y)$ is hard; SOTA algorithms have an exponential dependency in dimension \cite{optimal_tukey}, which is a non-starter when working with high-dimensional embeddings. However, there are efficient approximations which we will take advantage of.  










\section{\technique}
\label{sec:deepcandidate}
While useful and general, the document embedding $\overline{g}(x)$ does not satisfy \SDP. We now turn to describing our privacy-preserving technique, \technique, which generates general, $\epsilon$-\SDP~document embeddings that preserve relevant information in $\overline{g}(x)$, and are useful for downstream tasks. To understand the nontrivial nature of this problem, we first analyze why the simplest, straightfoward approaches are insufficient. 

%Outline
%First reiterate our goal (preserve relevant information in bar g) 
%Describe how this is difficult with simply adding noise 
%new subsection -- tukey approach using exponentiala mechanism -- use depth as utility  
%with infinite points to choose from this would work fine, but we can't do that 
%new subsection -- cluster preserving embeddings 
%talk about how we take advantage of the structure in this document's domain. In other words, we inject bias into the mechanism based off of publicly accessible non-private information. Use of public information is also used in clifton's approach  
%train new embedding to preserve the cluster -- contains more information than the cluster alone. Add'ly transmitting the cluster through e.g. randomized response is not possible 1) with utility (too many clusters) and 2) with sentDP 

%Then for algorithm section -- we talk about appx to depth 

%since DP defers to randomized algorithms we can't just publish the exact mean embedding 

\paragraph{Motivation.}
  Preserving privacy for high dimensional objects is known to be challenging \cite{kamath_high_dim, mdp_low_dim, DP_compression} . For instance, adding Laplace noise directly to $\overline{g}(x)$, as done to satisfy some privacy definitions \cite{metricdp, orig_metricdp}, does not guarantee \SDP~for any $\epsilon$. Recall that the embedding space is all of $\R^d$. A change in one sentence can lead to an unbounded change in $\overline{g}(x)$, since we do not put any restrictions on the general encoder $G$. Thus, no matter how much noise we add to $\overline{g}(x)$ we cannot satisfy \SDP. 

A straightforward workaround might be to simply truncate embeddings such that they all lie in a limited set such as a sphere or hypercube as done in prior work \cite{clifton, abadi}. In doing so, we bound how far apart embeddings can be for any two sentences, $\|G(s_i) - G(s_i')\|_1$, thus allowing us to satisfy \SDP~by adding finite variance noise. However, such schemes offer poor utility due to the high dimensional nature of useful document embeddings (we confirm this in our experiments). We must add noise with standard deviation proportional to the dimension of the embedding, thus requiring an untenable degree of noise for complex encoders like BERT which embed into $\R^{768}$. 

Our method has three pillars: \textbf{(1)} sampling from a candidate set of public, non-private document embeddings to represent the private document, \textbf{(2)} using the Tukey median to approximate the document embedding, and \textbf{(3)} pre-training the sentence encoder, $G$, to produce relevant candidates with high Tukey depth for private document $x$. 

\subsection{Taking advantage of public data: sampling from candidates}
Instead of having our mechanism select a private embedding $z$ from the entire space of $\R^d$, we focus the mechanism to select from a set of $m$ candidate  embeddings, $F$, generated by $m$ public, non-private documents. We assume the document $x$ is drawn from some distribution $\mu$ over documents $\calX$. For example, if we know $x$ is a restaurant review, $\mu$ may be the distribution over all restaurant reviews. $F$ is then a collection of document embeddings over $m$ publicly accessible documents $x_i \sim \mu$, 
\begin{align*}
	F = \{f_i = \overline{g}(x_i) : x_1, \dots, x_m \overset{\text{iid}}{\sim} \mu\} \ , 
\end{align*}
and denote the corresponding distribution over $f_i$ as $\overline{g}(\mu)$. By selecting candidate documents that are similar in nature to the private document $x$, we inject an advantageous inductive bias into our mechanism, which is critical to satisfy strong privacy while preserving information relevant to $x$. 

%Such inductive bias is critical to satisfy the strong privacy offered by \SDP~while preserving meaningful information unique to $x$. 

\subsection{Approximating the document embedding:\\ \quad \quad \  The Tukey Median}
\label{sec:tukey}
We now propose a novel mechanism $\mname$, which approximates $\overline{g}(x)$ by sampling a candidate embedding from $F$. $\mname$ works by concentrating probability on candidates with high Tukey Depth w.r.t. the set of sentence embeddings $S_x = \{G(s_i) : s_i \in x\}$. We model sentences $s_i$ from document $x$ as i.i.d. draws from distribution $\nu_x$. Then, $S_x$ is $k$ draws from $g(\nu_x)$, the distribution of sentences from $\nu_x$ passing through $G$. Deep points are a good approximation of the mean under light assumptions. If $g(\nu_x)$ belongs to the set of halfspace-symmetric distributions (including all elliptic distributions e.g. Gaussians), we know that its mean lies in the Tukey Median \cite{tukey_props}. 

Formally, $\mname$ is an instance of the exponential mechanism (Definition \ref{def: exp mech}), and is defined by its utility function. We set the utility of a candidate document embedding $f_i \in F$ to be an approximation of its depth w.r.t. sentence embeddings $S_x$, 
\begin{align}
	u(x, f_i) = \tdappx_{S_x}(f_i) \quad. 
	\label{eqn:utility}
\end{align}
The approximation $\tdappx_{S_x}$, which we detail in the Appendix, is necessary for computational efficiency. If the utility of $f_i$ is high, we call it a `deep candidate' for sentence embeddings $S_x$.

The more candidates sampled (higher $m$), the higher the probability that at least one has high depth. Without privacy, we could report the deepest candidate, $z = \argmax{f_i \in F} \tdappx_{S_x}(f_i)$. However, when preserving privacy with $\mname$, increasing $m$ has diminishing returns. To see this, fix a set of sentence embeddings $S_x$ for document $x$ and the i.i.d. distribution over candidate embeddings $f_i \sim \overline{g}(\mu)$. This induces a multinomial distribution over depth,  
\vspace{-0.6cm}
\begin{align*}
	u_j(x) = \Pr[u(x, f_i) = j], \ \ \sum_{j = 0}^{\lfloor \frac{k}{2} \rfloor} u_j(x) = 1 \ ,
\end{align*}
\vspace{-0.5cm}

\noindent where randomness is taken over draws of $f_i$. 

For candidate set $F$ and sentence embeddings $S_x$, the probability of $\mname$'s selected candidate, $z$, having (approximated) depth $j^*$ is given by 
\begin{align}
	\Pr[u(x, z) = j^*] = \frac{a_{j^*}(x)e^{\epsilon j^* / 2}}{\sum_{j=0}^{\lfloor \frac{k}{2} \rfloor} a_j(x) e^{\epsilon j / 2}}
	\label{eqn:prob deep}
\end{align}
where $a_j(x)$ is the fraction of candidates in $F$ with depth $j$ w.r.t. the sentence embeddings of document $x$, $S_x$. For $m$ sufficiently large, $a_j(x)$ concentrates around $u_j(x)$, so further increasing $m$ 
% Thus, while increasing $m$ may increase the number of deep candidates w.r.t. $S_x$, 
does not increase the probability of $\mname$ \emph{sampling} a deep candidate. 

\begin{table}[h!]
  \begin{center}
  \vspace{-0.25cm}
    \caption{Conditions for deep candidates}
    \label{tab:mech example}
    \begin{tabular}{l|c|r} % <-- Alignments: 1st column left, 2nd middle and 3rd right, with vertical lines in between
      $\epsilon$ & $b$ & $j^*$ \\
      \hline
      3 & 55 & 5\\
      6 & 25 & 3\\
      10 & 5 & 2\\
      23 & 1 & 1\\
    \end{tabular}
  \end{center}
  \vspace{-0.5cm}
\end{table}

For numerical intuition, suppose $m = 5000$ (as in our experiments), $\geq b$ candidates have depth $\geq j^*$, and all other candidates have depth 0, $\mname$ will sample one of these deep candidates w.p. $\geq 0.95$ under the settings in Table \ref{tab:mech example}. 

\begin{figure}
	\centering
	\includegraphics[width = \columnwidth]{figures/training_diagram.png} 
	\vspace{-0.65cm}
	\caption[$G'$ is trained to encourage similar documents to embed close together and different documents to embed far apart. ]{$G'$ is trained to encourage similar documents to embed close together and different documents to embed far apart. We first compute embeddings of all (public, non-private) training set documents $T$ with pretrained encoder $G$, $T_G = \{t_i = \overline{g}(x_i) : x_i \in T\}$ (blue dots). We run $k$-means to define $n_c$ clusters, and label each training document embedding $t_i \in T_G$ with its cluster $c$. We then train $H$ to recode sentences to $S_x'$ such that their mean $\overline{g}'(x)$ can be used by a linear model $L$ to predict cluster $c$. Our training objective is the cross-entropy loss of the linear model $L$ in predicting $c$.}
%	\caption{$G'$ is trained to recode document embeddings such that clusters can be predicted by linear model $L$. This encourages similar documents to embed close together and different documents to embed far apart.}
	\label{fig:training diagram}
	\vspace{-0.4cm}
\end{figure}

For low $\epsilon < 10$ (high privacy), about 1\% of candidates need to have high depth $(\geq 3)$ in order to be reliably sampled. Note that this is only possible for documents with $\geq 6$ sentences. For higher $\epsilon \geq 10$, $\mname$ will reliably sample low depth candidates even if there are only a few. 
 
From these remarks we draw two insights on how \technique\ can achieve high utility.\\
\textbf{(1)} \emph{More sentences} A higher $k$ enables greater depth, and thus a higher probability of sampling deep candidates with privacy. We explore this effect in our experiments. \\
\textbf{(2)} \emph{ Tuned encoder} By tuning the sentence encoder $G$ for a given domain, we can modify the distribution over document embeddings $\overline{g}(\mu)$ and sentence embeddings $g(\nu_x)$ to encourage deep candidates (high probability $u_j$ for deep $j$) that are relevant to document $x$.

\begin{figure*}[ht]
\centering 
   \begin{subfigure}[b]{0.30\linewidth}
       \centering
       \includegraphics[width=0.95\linewidth]{./figures/20news_sweep_eps.png}
      \vspace{-0.15cm}
       \caption{\textit{20 News}: Sweep $\epsilon$}
       \label{fig:eps:tnews}
    \end{subfigure}%%
    \begin{subfigure}[b]{0.30\linewidth}
       \centering
       \includegraphics[width=0.95\linewidth]{./figures/gr_sweep_eps.png}
      \vspace{-0.15cm}
       \caption{\textit{GoodReads}: Sweep $\epsilon$}
       \label{fig:eps:gr}
    \end{subfigure}%%
    \begin{subfigure}[b]{0.30\linewidth}
       \centering
       \includegraphics[width=0.95\linewidth]{./figures/imdb_sweep_eps.png}
      \vspace{-0.15cm}
       \caption{\textit{IMDB}: Sweep $\epsilon$}
       \label{fig:eps:imdb}
    \end{subfigure}%%
    \hfill
    \begin{subfigure}[b]{0.30\linewidth}
       \centering
       \includegraphics[width=0.95\linewidth]{./figures/20news_sweep_stcs.png}
      \vspace{-0.15cm}
       \caption{\textit{20 News}: Sweep $k$}
       \label{fig:k:tnews}
    \end{subfigure}%%
    \begin{subfigure}[b]{0.30\linewidth}
       \centering
       \includegraphics[width=0.95\linewidth]{./figures/gr_sweep_stcs.png}
      \vspace{-0.15cm}
       \caption{\textit{GoodReads}: Sweep $k$}
       \label{fig:k:gr}
    \end{subfigure}%%
    \begin{subfigure}[b]{0.30\linewidth}
       \centering
       \includegraphics[width=0.95\linewidth]{./figures/imdb_sweep_stcs.png}
      \vspace{-0.15cm}
       \caption{\textit{IMDB}: Sweep $k$}
       \label{fig:k:imdb}
    \end{subfigure}%%
    \caption{Comparison of our mechanism with two baselines: truncation \cite{clifton} and word-level Metric DP \cite{metricdp} for both sentiment analysis (\emph{IMDB}) and topic classification (\emph{GoodReads}, \emph{20News}) on private, unsupervised embeddings. All plots show test-set macro $F_1$ scores. The top row shows performance vs. privacy parameter $\epsilon$ (lower is better privacy). The bottom row shows performance vs. number of sentences $k$ with $\epsilon = 10$. \technique\ outperforms both baselines across datasets and tasks. Note that at a given $\epsilon$, word-level Metric-DP is a significantly weaker privacy guarantee.}
\end{figure*}


\subsection{Taking advantage of structure: cluster-preserving embeddings}

So far, we have identified that deep candidates from $F$ can approximate $\overline{g}(x)$. To produce a good approximation, we need to ensure that 1) there reliably exist deep candidates for any given set of sentence embeddings $S_x$, and 2) that these deep candidates are good representatives of document $x$. The general sentence encoder $G$ used may not satisfy this `out of the box'. If the distribution on document embeddings $\overline{g}(\mu)$ is very scattered around the instance space $\R^{768}$, it can be exceedingly unlikely to have a deep candidate $f_i$ among sentence embeddings $S_x$. On the other hand, if distribution $\overline{g}(\mu)$ is tightly concentrated in one region (e.g. `before training' in Figure \ref{fig:training diagram}), then we may reliably have many deep candidates, but several will be poor representatives of the document embedding $\overline{g}(x)$. 

To prevent this, we propose an unsupervised, efficient, and intuitive modification to the (pretrained) sentence encoder $G$. We freeze the weights of $G$ and add additional perceptron layers mapping into the same embeddings space $H:\R^d \rightarrow \R^d$, producing the extended encoder $G' = H \circ G$. Broadly, we train $H$ to place similar document embeddings close together, and different embeddings far part.  To do so, we leverage the assumption that a given domain's distribution over document embeddings $\overline{g}(\mu)$ can be parameterized by $n_c$ clusters, visualized as the black circles in Figure \ref{fig:training diagram}. $H$'s aim is to recode sentence embeddings such that document embedding clusters are preserved, but spaced apart from each other. By preserving clusters, we are more likely to have deep candidates (increased probability $u_j$ for high depth $j$). By spacing clusters apart, these deep candidates are more likely to come from the same or a nearby cluster as document $x$, and thus be good representatives. Note that $H$ is domain-specific: we train separate $H$ encoders for each dataset. 

%Consider a document $x$ with sentence embeddings $S_x$ and document embedding $\overline{g}(x)$ (mean of $S_x$) belonging to cluster $c \in [n_c]$. $H$'s task is to recode sentence embeddings $S_x$ to $S_x'$ such that the new document embedding $\overline{g}'(x)$ is close to those from cluster $c$ and far from those of other clusters $b \in [n_c] \backslash c$.   


%\paragraph{Training $G'$:} To achieve this end, we first compute embeddings of all (public, non-private) training set documents $T$ with pretrained encoder $G$, $T_G = \{t_i = \overline{g}(x_i) : x_i \in T\}$ (blue dots in Figure \ref{fig:training diagram}). We run $k$-means to define $n_c$ clusters, and label each training document embedding $t_i \in T_G$ with its cluster $c$. We then train $H$ to recode sentences to $S_x'$ such that their mean $\overline{g}'(x)$ can be used by a linear model $L$ to predict cluster $c$. Our training objective is the cross-entropy loss of the linear model $L$ in predicting the cluster. Note that this is subtly different from simply taking the activations of a final layer since we average the $k$ sentence embeddings into a single document embedding before passing it to $L$. 
%
%Intuitively, $L$ chooses $n_c$ directions in $\R^d$, and $G'$ is encouraged to generate sentence embeddings such that the resulting average document embedding $\overline{g}'(x)$ lies along one of those directions. So, if cluster 1 includes emails discussing global affairs and cluster 5 includes emails discussing celebrity gossip, $G'$ should easily distinguish the two and place their corresponding sentence embeddings along very different directions of $L$.  
%By embedding similar documents to close together and different documents far apart, $G'$ helps $\mname$'s utility by increasing the likelihood of deep candidates that good representatives of private document $x$. 

\subsection{Sampling Algorithm}
The final component of \technique\ is computing the approximate depth of a candidate for use as utility in the exponential mechanism as in Eq. \eqref{eqn:utility}. We use a version of the approximation algorithm proposed in \cite{median_hyp}. Intuitively, our algorithm computes the one-dimensional depth of each $f_i$ among $x$'s sentence embeddings $S_x$ on each of $p$ random projections. The approximate depth of $f_i$ is then its lowest depth across the $p$ projections. We are guaranteed that $\tdappx_{S_x}(f_i) \geq \text{TD}_{S_x}(f_i)$. Due to space constraints, we leave the detailed description of the algorithm for the Appendix.
\begin{theorem}
\label{thm:mainthm}
	$\mname$ satisfies $\epsilon$-Sentence Privacy
\end{theorem}
Proof follows from the fact that $\tdappx_{S_x}(f_i)$ has bounded sensitivity (changing one sentence can only change depth of $f_i$ by one). We expand on this, too, in the Appendix. 








\section{Experiments}
\label{sec:experiments}
\subsection{Datasets}
\label{sec: datasets} 
We produce private, general embeddings of documents from three English-language datasets: 

\textbf{\goodreads} \cite{goodreads} 60k book reviews from four categories: fantasy, history, romance, and childrens literature.  Train-48k | Val-8k | Test-4k 

\textbf{\tnews} \cite{20newsgroup} 11239 correspondences from 20 different affinity groups. Due to similarity between several groups (e.g. \texttt{comp.os.ms-windows.misc} and \texttt{comp.sys.ibm.pc.hardware}), the dataset is partitioned into nine categories. Train-6743k | Val-2247k | Test-2249k

\textbf{\imdb} \cite{imdb} 29k movie reviews from the IMDB database, each labeled as a positive or negative review. Train-23k | Val-2k | Test-4k 

To evaluate utility of these unsupervised, private embeddings, we check if they are predictive of document properties. For the \goodreads\ and \tnews\ datasets, we evaluate how useful the embeddings are for topic classification. For \imdb\ we evaluate how useful the embeddings are for sentiment analysis (positive or negative review). Our metric for performance is test-set macro $F_1$ score. 

\subsection{Training Details \& Setup}
For the general encoder, $G:\calS \rightarrow \R^{768}$, we use SBERT \cite{sbert}, a version of BERT fine-tuned for sentence encoding. Sentence embeddings are generated by mean-pooling output tokens. In all tasks, we freeze the weights of SBERT. The cluster-preserving recoder, $H$, as well as every classifier is implemented as an instance of a 4-layer MLP taking $768$-dimension inputs and only differing on output dimension. We denote an instance of this MLP with output dimension $o$ as \MLP{o}. We run 5 trials of each experiment with randomness taken over the privacy mechanisms, and plot the mean along with a $\pm$ 1 standard deviation envelope. 

\paragraph{\technique:} The candidate set $F$ consists of 5k document embeddings from the training set, each containing at least 8 sentences. To train $G'$, we find $n_c = 50$ clusters with $k$-means. We train a classifier $C_{\text{dc}} = $ \MLP{r} on document embeddings $g'(x)$ to predict class, where $r$ is the number of classes (topics or sentiments). 

\subsection{Baselines}
We compare the performance of \technique\ with 4 baselines: \textbf{Non-private}, \textbf{Truncation}, \textbf{Word-level Metric-DP}, and \textbf{Random Guesser}. 

\textbf{Non-private:} This demonstrates the usefulness of non-private sentence-mean document embeddings $\overline{g}(x)$. We generate $\overline{g}(x)$ for every document using SBERT, and then train a classifier $C_{\text{nonpriv}} = $ \MLP{r} to predict $x$'s label from $\overline{g}(x)$. 

\textbf{Truncation:} We adopt the method from \citealt{clifton} to truncate (clip) sentence embeddings within a box in $\R^{768}$, thereby bounding sensitivity as described at the beginning of Section \ref{sec:deepcandidate}. Laplace noise is then added to each dimension. Documents with more sentences have proportionally less noise added due to the averaging operation reducing sensitivity. 

%\paragraph{Truncation:} The truncation baseline \needcite\ requires first constraining the embedding instance space. We do so by computing the 75\% median interval on each of the 768 dimensions of training document embeddings $T_G$. Sentence embeddings are truncated at each dimension to lie in this box. In order to account for this distribution shift, a new classifier $C_{\text{trunc}} = $ \MLP{r} is trained on truncated mean embeddings to predict class. The number of epochs is determined with the validation set. At test time, a document's sentence embeddings $S_x$ are truncated and averaged. We then add Laplace noise to each dimension with scale factor $\frac{768 w}{k \epsilon}$, where $w$ is the width of the box on that dimension (\emph{sensitivity} in DP terms). Note that the standard deviation of noise added is inversely proportional to the number of sentences in the document, due to the averaging operation reducing sensitivity. 

\textbf{Word Metric-DP (MDP):} The method from \citealt{metricdp} satisfies $\epsilon$-word-level metric DP by randomizing words. We implement MDP to produce a randomized document $x'$, compute $\overline{g}(x')$ with SBERT, and predict class using $C_{\text{nonpriv}}$. 

\textbf{Random Guess:} To set a bottom-line, we show the theoretical performance of a random guesser only knowing the distribution of labels. 

\subsection{Results \& Discussion} 
\textbf{How does performance change with privacy parameter $\epsilon$?}\\ 
This is addressed in Figures \ref{fig:eps:tnews} to \ref{fig:eps:imdb}. Here, we observe how the test set macro $F_1$ score changes with privacy parameter $\epsilon$ (a lower $\epsilon$ offers stronger privacy). Generally speaking, for local differential privacy, $\epsilon < 10$ is taken to be a strong privacy regime, $10 \leq \epsilon < 20$ is moderate privacy, and $\epsilon \geq 25$ is weak privacy. The \textbf{truncation} baseline mechanism does increase accuracy with increasing $\epsilon$, but never performs much better than the random guesser. This is to be expected with high dimension embeddings, since the standard deviation of noise added increases linearly with dimension. 

The word-level \textbf{MDP} mechanism performs significantly better than \textbf{truncation}, achieving relatively good performance for $\epsilon \geq 30$. There are two significant caveats, however. First, is the privacy definition: as discussed in the Introduction, for the same $\epsilon$, word-level MDP is strictly weaker than \SDP. 
%The MDP definition only guarantees that changing a single word with a similar word is $\epsilon$-indistinguishable to an adversary trying to recover the original document. Sentence-DP guarantees that adding, removing, or modifying \emph{any number of words in any way} in a given sentence is $\epsilon$-indistinguishable -- a significantly stronger guarantee. 
The second caveat is the level of $\epsilon$ at which privacy is achieved. Despite a weaker privacy definition, the MDP mechanism does not achieve competitive performance until the weak-privacy regime of $\epsilon$. We suspect this is due to two reasons. First, is the fact that the MDP mechanism does not take advantage of contextual information in each sentence as our technique does; randomizing each word independently does not use higher level linguistic information. Second, is the fact that the MDP mechanism does not use domain-specific knowledge as our mechanism does with use of relevant candidates and domain specific sentence encodings. 

In comparison, \technique\ offers strong utility across tasks and datasets for relatively low values of $\epsilon$, even into the strong privacy regime. Beyond $\epsilon = 25$, the performance of \technique\ tends to max out, approximately 10-15\% below the non-private approach. This is due to the fact that \technique\ offers a noisy version of an \emph{approximation} of the document embedding $\overline{g}(x)$ -- it cannot perform any better than deterministically selecting the deepest candidate, and even this candidate may be a poor representative of $x$. We consider this room for improvement, since there are potentially many other ways to tune $G'$ and select the candidate pool $F$ such that deep candidates are nearly always good representatives of a given document $x$. 

\noindent\textbf{How does performance change with the number of sentences $k$?}\\
This is addressed in Figures \ref{fig:k:tnews} to \ref{fig:k:imdb}. We limit the test set to those documents with $k$ in the listed range on the x-axis. We set $\epsilon = 10$, the limit of the strong privacy regime. Neither baseline offers performance above that of the random guesser at this value of $\epsilon$.  \technique\ produces precisely the performance we expect to see: documents with more sentences result in sampling higher quality candidates, confirming the insights of Section \ref{sec:tukey}. Across datasets and tasks, documents with more than 10-15 sentences tend to have high quality embeddings. 

\section{Conclusions and Future Work}
\vspace{-0.5em}
We introduce a strong and interpretable local privacy guarantee for documents, \SDP, along with \technique, a technique that combines principles from NLP and robust statistics to generate general $\epsilon$-\SDP\ embeddings. Our experiments confirm that such methods can outperform existing approaches even with with more relaxed privacy guarantees. Previous methods have argued that it is ``virtually impossible'' to satisfy pure local DP \cite{metricdp, mdp_low_dim} at the word level while capturing linguistic semantics. Our work appears to refute this notion at least at the document level. 

To follow up, we plan to explore other approaches (apart from $k$-means) of capturing the structure of the embedding distribution $\overline{g}(\mu)$ to encourage better candidate selection. We also plan to experiment with decoding private embeddings back to documents by using novel candidates produced by a generative model trained on $F$. 

% \clearpage

\section*{Acknowledgements} 
KC and CM would like to thank ONR under N00014-20-1-2334. KM gratefully acknowledges funding from an Amazon Research Award and Adobe Unrestricted Research Gifts. We would would also like to thank our reviewers for their insightful feedback.
%Questions are 
%\begin{enumerate}
%	\item How does performance compare w/ baselines as we change privacy parameter $\epsilon$.  
%	\item How does performance compare w/ baselines as we change number of sentences $k$ in the private document. 
%\end{enumerate}
%
%\cm{First subsection is datasets, maybe a table. Tran/val/test splits and explain what the dataset is (sentiment / classification). } \\
%\cm{Second subsection: training details \& setup. Talk about SBERT and MLP layer shit and dimension and train with this kind of loss for this many epochs with this particular optimizer. Don't mention SBERT earlier just here. Number of parameters in MLP.} \\
%\cm{In second subsection: setup: e.g. ``for all experiments we get the frozen embeddings'' we select the best MLP classifier based on da da da. We use Macro F1 for metric. } \\
%\cm{Results \& Discussion: show questions here. talk about how the figures answer the questions. } \\


\graphicspath{{./chapters/chapter4/}}
%\newtheorem{thm}{Theorem}
%\newtheorem{lem}[thm]{Lemma}
%\DeclareMathOperator*{\argmax}{arg\,max}
%\DeclareMathOperator*{\argmin}{arg\,min}
\newtheorem{theorem}[thm]{Theorem}
\newtheorem{proposition}[thm]{Proposition}
\newtheorem{lemma}[thm]{Lemma}
\newtheorem{corollary}[thm]{Corollary}
%\theoremstyle{definition}
\newtheorem{definition}[thm]{Definition}
\newtheorem{assumption}[thm]{Assumption}
%\theoremstyle{remark}
\newtheorem{remark}[thm]{Remark}

\def\c{cr}
\def\hc{\hat{cr}}
\def\d{\Lambda}
\def\ind{\mathbbm{1}}
\def\oc{Online\_Cluster}
\def\mem{\mathcal M}
\def\E{\mathbb{E}}
\def\R{\mathbb{R}}
\def\O{\mathcal{O}}
\def\calP{\mathcal{P}}
\def\dc{Data\_Copy\_Detect}
\def\a{\kappa}
\def\hq{\widehat{q(B(x, r))}}
\def\alpaca{\epsilon}

\chapter{Data-Copying in Generative Models: A Formal Framework} 

\section{Introduction}

Deep generative models have shown impressive performance. However, given how large, diverse, and uncurated their training sets are, a big question is whether, how often, and how closely they are memorizing their training data. This question has been of considerable interest in generative modeling~\citep{lopez2016revisiting,XHYGSWK18} as well as supervised learning~\citep{BBFST21, Feldman20}. However, a clean and formal definition of memorization that captures the numerous complex aspects of the problem, particularly in the context of continuous data such as images, has largely been elusive.

For generative models,~\cite{MCD2020} proposed a formal definition of memorization called ``data-copying'', and showed that it was orthogonal to various prior notions of overfitting such as mode collapse~\citep{TT20}, mode dropping~\citep{YFWYC20}, and precision-recall~\citep{SBLBG18}. Specifically, their definition looks at three datasets -- a training set, a set of generated example, and an independent test set. Data-copying happens when the training points are considerably closer on average to the generated data points than to an independently drawn test sample. Otherwise, if the training points are further on average to the generated points than test, then there is underfitting. They proposed a three sample test to detect this kind of data-copying, and empirically showed that their test had good performance.

\begin{figure}[ht]
\centering
	\includegraphics[width=.45\textwidth]{page_2_figure_yo.png}
	\caption{In this figure, the blue points are sampled from the halfmoons dataset (with Gaussian noise). The red points are sampled from a generated distribution that is a mixture of (40 \%) blatant data copier (that outputs a random subset of the training set), and (60 \%) a noisy underfit version of halfmoons. Although the generated distribution is clearly doing some form of copying at points $x_1$ and $x_2$, detecting this is challenging because of the canceling effect of the underfit points.}
	
	\label{fig:page_2_figure}
\end{figure}

However, despite its practical success, this method may not capture even blatant cases of memorization. To see this, consider the example illustrated in Figure \ref{fig:page_2_figure}, in which a generated model for the halfmoons dataset outputs one of its training points with probability $0.4$, and otherwise outputs a random point from an underfit distribution. When the test of~\cite{MCD2020} is applied to this distribution, it is unable to detect any form of data copying; the generated samples drawn from the underfit distribution are sufficient to cancel out the effect of the memorized examples. Nevertheless, this generative model is clearly an egregious memorizer as shown in points $x_1$ and $x_2$ of Figure \ref{fig:page_2_figure}.

This example suggests a notion of \textit{point-wise} data copying, where a model $q$ can be thought of as copying a given training point $x$. Such a notion would be able to detect $q$'s behavior nearby $x_1$ and $x_2$ regardless of the confounding samples that appear at a global level. This stands in contrast to the more global distance based approach taken in Meehan et. al. which is unable to detect such instances. Motivated by this, we propose an alternative point-by-point approach to defining data-copying.

We say that a generative model $q$  data-copies an individual training point, $x$, if it has an unusually high concentration in a small area centered at $x$. Intuitively, this implies $q$ is highly likely to output examples that are very similar to $x$. In the example above, this definition would flag $q$ as copying $x_1$ and $x_2$. 

To parlay this definition into a global measure of data-copying, we define the overall \textit{data-copying rate} as the total fraction of examples from $q$ that are copied from some training example. In the example above, this rate is $40\%$, as this is the fraction of examples that are blatant copies of the training data.

\begin{figure}[ht]
    \begin{subfigure}{0.31\textwidth}\includegraphics[width=\linewidth]{default.png}
    \end{subfigure}\hspace*{\fill}
	\begin{subfigure}{0.31\textwidth}\includegraphics[width=\linewidth]{add_regions.png}
	\end{subfigure}\hspace*{\fill}
	\begin{subfigure}{0.31\textwidth}\includegraphics[width=\linewidth]{data_copy.png}
	\end{subfigure}
	\caption{In the three panels above, the blue points are a training sample from $p$, and the red points are generated examples from $q$. In the middle panel, we highlight in green regions that are defined to be \textit{data-copying regions}, as $q$ overrepresents them with comparison to $p$. In the third panel, we then color all points from $q$ that are considered to be copied green.}
	
	\label{fig:triptic}
\end{figure}

Next, we consider how to detect data-copying according to this definition. To this end, we provide an algorithm, \dc{}, that outputs an estimate for the overall data-copying rate. We then show that under a natural smoothness assumption on the data distribution, which we call \textit{regularity}, \dc{} is able to guarantee an accurate estimate of the total data-copying rate. We then give an upper bound on the amount of data needed for doing so. 

We complement our algorithm with a lower bound on the minimum amount of a data needed for data-copying detection. Our lower bound also implies that some sort of smoothness condition (such as regularity) is necessary for guaranteed data-copying detection; otherwise, the required amount of data can be driven arbitrarily high.

\subsection{Related Work}

Recently, understanding failure modes for generative models has been an important growing body of work e.g. \citep{SGZCRC16, RW18, SBLBG18}. However, much of this work has been focused on other forms of overfitting, such as mode dropping or mode collapse.

A more related notion of overfitting is \textit{memorization} \citep{lopez2016revisiting,XHYGSWK18, C18}, in which a model outputs exact copies of its training data. This has been studied in both supervised \citep{BBFST21, Feldman20} and unsupervised \citep{BGWC21, CHCW21} contexts. Memorization has also been considered in language generation models \cite{Carlini22}. 

The first work to explicitly consider the more general notion of \textit{data-copying} is \citep{MCD2020}, which gives a three sample test for data-copy detection. We include an empirical comparison between our methods in Section \ref{sec:experiments}, where we demonstrate that ours is able to capture certain forms of data-copying that theirs is not. 

Finally, we note that this work focuses on detecting natural forms of memorization or data-copying, that likely arises out of poor generalization, and is not concerned with detecting \textit{adversarial} memorization or prompting, such as in \cite{Carlini19}, that are designed to obtain sensitive information about the training set. This is reflected in our definition and detection algorithm which look at the specific generative model, and not the algorithm that trains it.  Perhaps the best approach to prevent adversarial  memorization is training the model with differential privacy~\cite{Dwork06}, which ensures that the model does not change much when one training sample changes. However such solutions come at an utility cost. 

\section{A Formal Definition of Data-Copying}

We begin with the following question: what does it mean for a generated distribution $q$ to copy a single training example $x$? Intuitively, this means that $q$ is guilty of overfitting $x$ in some way, and consequently produces examples that are very similar to it. 

However, determining what constitutes a `very similar'  generated example must be done contextually. Otherwise the original data distribution, $p$, may itself be considered a copier, as it will output points nearby $x$ with some frequency depending on its density at $x$. Thus, we posit that $q$ data copies training point $x$ if it has a significantly higher concentration nearby $x$ than $p$ does. We express this in the following definition. 

\begin{definition}\label{defn:data_copy}
Let $p$ be a data distribution, $S \sim p^n$ a training sample, and $q$ be a generated distribution trained on $S$. Let $x \in S$ be a training  point, and let $\lambda > 1$ and $0 < \gamma < 1$ be constants. A generated example $x' \sim q$ is said to be a \textbf{$(\lambda, \gamma)$-copy} of $x$ if there exists a ball $B$ centered at $x$ (i.e. $\{x': ||x' - x|| \leq r\}$) such that following hold:
\begin{itemize}
	\item $x' \in B$.
	\item $q(B) \geq \lambda p(B)$
	\item $p(B) \leq \gamma$
\end{itemize}
\end{definition}

Here $q(B)$ and $p(B)$ denote the probability mass assigned to $B$ by $p$ and $q$ respectively.

The parameters $\lambda$ and $\gamma$ are user chosen parameters that characterize data-copying. $\lambda$ represents the rate at which $q$ must overrepresent points close to $x$, with higher values of $\lambda$ corresponding to more egregious examples of data-copying. $\gamma$ represents the maximum size (by probability mass) of a region that is considered to be data-copying -- the ball $B$ represents all points that are ``copies" of $x$. Together, $\lambda$ and $\gamma$ serve as practitioner controlled knobs that characterize data-copying about $x$.

Our definition is illustrated in Figure \ref{fig:triptic} -- the training data is shown in blue, and generated samples are shown in red. For each training point, we highlight a region (in green) about that point in which the red density is much higher than the blue density, thus constituting data-copying. The intuition for this is that the red points within any ball can be thought of as ``copies" of the blue point centered in the ball.

Having defined data-copying with respect to a single training example, we can naturally extend this notion to the entire training dataset. We say that $x' \sim q$ is copied from training set $S$ if $x'$ is a $(\lambda,\gamma)$-copy of some training example $x \in S$. We then define the \textit{data-copy rate} of $q$ as the fraction of examples it generates that are copied from $S$. Formally, we have the following: 

\begin{definition}
Let $p, S, q, \lambda,$ and $\gamma$ be as defined in Definition \ref{defn:data_copy}. Then the \textbf{data-copy rate}, $\c\left(q, \lambda, \gamma\right)$ of $q$ (with respect to $p, S$) is the fraction of examples from $q$ that are $(\lambda, \gamma)$-copied. That is, $$\c\left(q, \lambda, \gamma\right) = \Pr_{x' \sim q}[q\text{ }(\lambda,\gamma)\text{-copies }x'].$$ In cases where $\lambda, \gamma$ are fixed, we use $\c_q = \c(q, \lambda, \gamma)$ to denote the data-copy rate.
\end{definition}

Despite its seeming global nature, $\c_q$ is simply an aggregation of the point by point data-copying done by $q$ over its entire training set. As we will later see, estimating $\c_q$ is often reduced to determining which subset of the training data $q$ copies. 

\subsection{Examples of data-copying}

We now give several examples illustrating our definitions. In all cases, we let $p$ be a data distribution, $S$, a training sample from $p$, and $q$, a generated distribution that is trained over $S$. 

\paragraph{The uniform distribution over $S$:} In this example, $q$ is an egregious data copier that memorizes its training set and randomly outputs a training point. This can be considered as the canonical \textit{worst} data copier. This is reflected in the value of $\c_q$ -- if $p$ is a continuous distribution with finite probability density, then for any $x \in S$, there exists a ball $B$ centered at $x$ for which $q(B) >> p(B)$. It follows that $q$ $(\lambda,\gamma)$- copies $x$ for all $x \in S$ which implies that $\c_q = 1$.

\paragraph{The perfect generative model: $q = p$:} In this case, $q(B) = p(B)$ for all balls, $B$, which implies that $q$ does not perform any data-copying (Definition \ref{defn:data_copy}). It follows that $\c_q = 0$, matching the intuition that $q$ does not data-copy at all.

\paragraph{Kernel Density Estimators:} Finally, we consider a more general situation, where $q$ is trained by a \textit{kernel density estimator} (KDE) over $S \sim p^n$. Recall that a kernel density estimator outputs a generated distribution, $q$, with pdf defined by $$q(x) = \frac{1}{n\sigma_n}\sum_{x_i \in S} K\left(\frac{x - x_i}{\sigma_n}\right).$$ Here, $K$ is a kernel similarity function, and $\sigma_n$ is the bandwidth parameter. It is known that for $\sigma_n = O(n^{-1/5})$, $q$ converges towards $p$ for sufficiently well behaved probability distributions. 

Despite this guarantee, KDEs intuitively appear to perform some form of data-copying -- after all they implicitly include each training point in memory as it forms a portion of their outputted pdf. However, recall that our main focus is in understanding \textit{overfitting} due to data-copying. That is, we view data-copying as a function of the outputted pdf, $q$, and not of the training algorithm used. 

To this end, for KDEs the question of data-copying reduces to the question of whether $q$ overrepresents areas around its training points. As one would expect, this occurs \textit{before} we reach the large sample limit. This is expressed in the following theorem.

\begin{theorem}\label{thm:KDE}
Let $1 < \lambda$ and $\gamma > 0$. Let $\sigma_n$ be a sequence of bandwidths and $K$ be any regular kernel function. For any $n > 0$ there exists a probability distribution $\pi$ with full support over $\R^d$ such that with probability at least $\frac{1}{3}$ over $S \sim \pi^n$, a KDE trained with bandwidth $\sigma_n$ and kernel function $K$ has data-copy rate $\c_q \geq \frac{1}{10}$.
\end{theorem}

This theorem completes the picture for KDEs with regards to data-copying -- when $n$ is too low, it is possible for the KDE to have a significant amount of data-copying, but as $n$ continues to grow, this is eventually smoothed out.

\paragraph{The Halfmoons dataset}

Returning to the example given in Figure \ref{fig:page_2_figure}, observe that our definition exactly captures the notion of data-copying that occurs at points $x_1$ and $x_2$. For even strict choices of $\lambda$ and $\gamma$, Definition \ref{defn:data_copy} indicates that the red distribution copies both $x_1$ and $x_2$. Furthermore, the data-copy rate, $\c_q$, is $40\%$ by construction, as this is the proportion of points that are outputted nearby $x_1$ and $x_2$.

\subsection{Limitations of our definition}\label{sec:limitations}

Definition \ref{defn:data_copy} implicitly assumes that the goal of the generator is to output a distribution $q$ that approaches $p$ in a mathematical sense; a perfect generator would output $q$ so that $q(M) = p(M)$ for all measurable sets. In particular, instances where $q$ outputs examples that are far away from the training data are considered completely irrelevant in our definition.

This restriction prevents our definition from capturing instances in which $q$ memorizes its training data and then applies some sort of transformation to it. For example, consider an image generator that applies a color filter to its training data. This would not be considered a data-copier as its output would be quite far from the training data in pixel space. Nevertheless, such a generated distribution can be very reasonably considered as an egregious data copier, and a cursory investigation between its training data and its outputs would reveal as much. 

The key difference in this example is that the generative algorithm is no longer trying to closely approximate $p$ with $q$ -- it is rather trying to do so in some kind of transformed space. Capturing such interactions is beyond the scope of our paper, and we firmly restrict ourselves to the case where a generator is evaluated based on how close $q$ is to $p$ with respect to their measures over the input space. 

\section{Detecting data-copying}

Having defined $\c_q$, we now turn our attention towards \textit{estimating it.} To formalize this problem, we will require a few definitions. We begin by defining a generative algorithm.

\begin{definition}
A \textbf{generative algorithm}, $A$, is a potentially randomized algorithm that outputs a distribution $q$ over $\R^d$ given an input of training points, $S \subset \R^d$. We denote this relationship by $q \sim A(S)$.
\end{definition}

This paradigm captures most typical generative algorithms including both non-parametric methods such as KDEs and parametric methods such as variational autoencoders.

As an important distinction, in this work we define data-copying as a property of the generated distribution, $q$, rather than the generative algorithm, $A$. This is reflected in our definition which is given solely with respect to $q, S,$ and $p$. For the purposes of this paper, $A$ can be considered an arbitrary process that takes $S$ and outputs a distribution $q$. We include it in our definitions to emphasize that while $S$ is an i.i.d sample from $p$, it is \textit{not} independent from $q$. 

Next, we define a \textit{data-copying detector} as an algorithm that estimates $\c_q$ based on access to the training sample, $S$, along with the ability to draw any number of samples from $q$. The latter assumption is quite typical as sampling from $q$ is a purely computational operation. We do not assume any access to $p$ beyond the training sample $S$. Formally, we have the following definition.

\begin{definition}\label{def:data_copy_detector}
A \textbf{data-copying detector} is an algorithm $D$ that takes as input a training sample, $S \sim p^n$, and access to a sampling oracle for $q \sim A(S)$ (where $A$ is an arbitrary generative algorithm). $D$ then outputs an estimate, $D(S, q) = \hc_q$, for the data-copy rate of $q$. 
\end{definition}

Naturally, we assume $D$ has access to $\lambda, \gamma >0$ (as these are practitioner chosen values), and by convention don't include $\lambda, \gamma$ as formal inputs into $D$. 

The goal of a data-copying detector is to provide accurate estimates for $\c_q$. However, the precise definition of $\c_q$ poses an issue: data-copy rates for varying values of $\lambda$ and $ \gamma$ can vastly differ. This is because $\lambda, \gamma$ act as thresholds with everything above the threshold being counted, and everything below it being discarded. Since $\lambda, \gamma$ cannot be perfectly accounted for, we will require some tolerance in dealing with them. This motivates the following.

\begin{definition}\label{defn:approx_data_copy_rate}
Let $0 < \alpaca$ be a tolerance parameter. Then the \textbf{approximate data-copy rates}, $\c_q^{-\alpaca}$ and $\c_q^\alpaca$, are defined as the values of $\c_q$ when the parameters $(\lambda, \gamma)$ are shifted by a factor of $(1+\alpaca)$ to respectively decrease and increase the copy rate. That is, $$\c_q^{-\alpaca} = \c\left(q, \lambda (1+\alpaca), \gamma (1+\alpaca)^{-1}\right),$$ $$\c_q^{\alpaca} = \c\left(q, \lambda (1+\alpaca)^{-1}, \gamma (1+\alpaca)\right).$$
\end{definition}

The shifts in $\lambda$ and $\gamma$ are chosen as above because increasing $\lambda$ and decreasing $\gamma$ both reduce $\c_q$ seeing as both result in more restrictive conditions for what qualifies as data-copying. Conversely, decreasing $\lambda$ and increasing $\gamma$ has the opposite effect. It follows that $$\c_q^{-\alpaca} \leq \c_q \leq \c_q^{\alpaca},$$ meaning that $\c_q^{-\alpaca}$ and $\c_q^{\alpaca}$ are lower and upper bounds on $\c_q$. 

In the context of data-copying detection, the goal is now to estimate $\c_q$ in comparison to $\c_q^{\pm \alpaca}$. We formalize this by defining \textit{sample complexity} of a data-copying detector as the amount of data needed for accurate estimation of $\c_q$. 

\begin{definition}\label{def:sample_complexity}
Let $D$ be a data-copying detector and $p$ be a data distribution. Let $\epsilon, \delta > 0$ be standard tolerance parameters. Then $D$ has \textbf{sample complexity}, $m_p(\epsilon, \delta)$, with respect to $p$ if for all $n \geq m_p(\epsilon, \delta)$, $\lambda >1$, $0 < \gamma < 1$, and generative algorithms $A$, with probability at least $1 - \delta$ over $S \sim p^n$ and $q \sim A(S)$, $$\c_q^{-\alpaca} - \epsilon \leq D(S, q) \leq \c_q^{\alpaca} + \epsilon.$$
\end{definition}

Here the parameter $\epsilon$ takes on a somewhat expanded as it is both used to additively bound our estimation of $\c_q$ and to multiplicatively bound $\lambda$ and $\gamma$.

Observe that there is no mention of the number of calls that $D$ makes to its sampling oracle for $q$. This is because samples from $q$ are viewed as \textit{purely computational}, as they don't require any natural data source. In most cases, $q$ is simply some type of generative model (such as a VAE or a GAN), and thus sampling from $q$ is a matter of running the corresponding neural network.

\section{Regular Distributions}\label{sec:regular_dist}

Our definition of data-copying (Definition \ref{defn:data_copy}) motivates a straightforward point by point method for data-copying detection, in which for every training point, $x_i$, we compute the largest ball $B_i$ centered at $x_i$ for which $q(B_i) \geq \lambda p(B_i)$ and $p(B_i) \leq \gamma$. Assuming we compute these balls accurately, we can then query samples from $q$ to estimate the total rate at which $q$ outputs within those balls, giving us our estimate of $\c_q$.

The key ingredient necessary for this idea to work is to be able to reliably estimate the masses, $q(B)$ and $p(B)$ for any ball in $\R^d$. The standard approach to doing this is through \textit{uniform convergence}, in which large samples of points are drawn from $p$ and $q$ (in $p$'s case we use $S$), and then the mass of a ball is estimated by counting the proportion of sampled points within it. For balls with a sufficient number of points (typically $O( d\log n)$), standard uniform convergence arguments show that these estimates are reliable.

However, this method has a major pitfall for our purpose -- in most cases the balls $B_i$ will be very small because data-copying intrinsically deals with points that are very close to a given training point. While one might hope that we can simply ignore all balls below a certain threshold, this does not work either, as the sheer number of balls being considered means that their union could be highly non-trivial. 

To circumvent this issue, we will introduce an interpolation technique that estimates the probability mass of a small ball by scaling down the mass of a sufficiently large ball with the same center. While obtaining a general guarantee is impossible -- there exist pathological  distributions that drastically change their behavior at small scales -- it turns out there is a relatively natural condition under which such interpolation will work. We refer to this condition as \textit{regularity,} which is defined as follows.

\begin{definition}\label{def:regular}
Let $k> 0$ be an integer. A probability distribution $p$ is \textbf{$k$-regular} the following holds. For all $\alpaca > 0$, there exists a constant $0 < p_\alpaca \leq 1$ such that for all $x$ in the support of $p$, if $0 < s < r$ satisfies that $p(B(x, r)) \leq p_\alpaca$, then $$\left(1+\frac{\alpaca}{3}\right)^{-1}\frac{r^k}{s^{k}} \leq \frac{p(B(x, r))}{p(B(x, s))} \leq \left(1+\frac{\alpaca}{3}\right)\frac{r^k}{s^{k}}.$$ Finally, a distribution is \textbf{regular} if it is $k$-regular for some integer $k > 0$. 
\end{definition}

Here we let $B(x, r) = \{x': ||x - x'|| \leq r\}$ denote the closed $\ell_2$ ball centered at $x$ with radius $r$. 

The main intuition for a $k$-regular distribution is that at a sufficiently small scale, its probability mass scales with distance according to a power law, determined by $k$. The parameter $k$ dictates how the probability density behaves with respect to the distance scale. In most common examples, $k$ will equal the \textit{intrinsic dimension}  of $p$.

As a technical note, we use an error factor of $\frac{\alpaca}{3}$ instead of $\alpaca$ for technical details that enable cleaner statements and proofs in our results (presented later). 

\subsection{Distributions with Manifold Support}

We now give an important class of $k$-regular distributions.

\begin{proposition}\label{prop:manifold_works}
Let $p$ be a probability distribution with support precisely equal to a compact $k$ dimensional sub-manifold (with or without boundary) of $\R^d$, $M$. Additionally, suppose that $p$ has a continuous density function over $M$. Then it follows that $p$ is $k$-regular.
\end{proposition}

Proposition \ref{prop:manifold_works} implies that most data distributions that adhere to some sort of manifold-hypothesis will also exhibit regularity, with the regularity constant, $k$, being the intrinsic dimension of the manifold.

\subsection{Estimation over regular distributions}

We now turn our attention towards designing estimation algorithms over regular distributions, with our main goal being to estimate the probability mass of arbitrarily small balls. We begin by first addressing a slight technical detail -- although the data distribution $p$ may be regular, this does not necessarily mean that the regularity constant, $k$, is known. Knowledge of $k$ is crucial because it determines how to properly interpolate probability masses from large radius balls to smaller ones. 

Luckily, estimating $k$ turns out to be an extremely well studied task, as for most probability distributions, $k$ is a measure of the \textit{intrinsic dimension}. Because there is a wide body of literature in this topic, we will assume from this point that $k$ has been correctly estimated from $S$ using any known algorithm for doing so (for example \cite{BJPR22}). Nevertheless, for completeness, we provide an algorithm with provable guarantees for estimating $k$ (along with a corresponding bound on the amount of needed data) in Appendix \ref{sec:estimating_alpha}.

We now return to the problem of $p(B(x, r))$ for a small value of $r$, and present an algorithm, $Est(x, r, S)$ (Algorithm \ref{alg:estimate}), that estimates $p(B(x, r))$ from an i.i.d sample $S \sim p^n$.

\begin{algorithm}
   \caption{$Est(x, r, S)$}
   \label{alg:estimate}

   \DontPrintSemicolon
   
	$n \leftarrow |S|$\;
	
   $b \leftarrow O\left(\frac{d \ln \frac{n}{\delta}}{\epsilon^2} \right)$\;
   
   $r_* = \min \{s > 0, |S \cap B(x, s)| = b\}$.\;
   
   \uIf{$r_* > r$}{
   Return $\frac{br^k}{nr_*^k}$\;
   }
   \uElse {
	Return $\frac{|T \cap B(x, r)|}{n}$\;
	}

\end{algorithm}

$Est$ uses two ideas: first, it leverages standard uniform convergence results to estimate the probability mass of all balls that contain a sufficient number of training examples. This is what leads to the specific value of $b$ that is chosen. Second, it estimates the mass of smaller balls by interpolating from its estimates from larger balls. The $k$-regularity assumption is crucial for this second step as it is the basis on which such interpolation is done. 

$Est$ has the following performance guarantee, which follows from standard uniform convergence bounds and the definition of $k$-regularity. 
\begin{proposition}\label{prop:est_works}
Let $p$ be a regular distribution, and let $\alpaca >0$ be arbitrary. Then if $n = O\left(\frac{d\ln\frac{d}{\delta \alpaca p_\alpaca}}{\alpaca^2 p_\alpaca}\right)$ with probability at least $1 - \delta$ over $S \sim p^n$, for all $x \in \R^d$ and $r > 0$, $$\left(1+\frac{\alpaca}{2}\right)^{-1}\leq \frac{Est(x, r, S)}{p(B(x, r))} \leq \left(1+\frac{\alpaca}{2}\right).$$
\end{proposition}

\section{A Data-copy detecting algorithm}

\begin{algorithm}    

\caption{$DataCopyDetect(S, q, m)$}
\label{alg:main}   

   \DontPrintSemicolon
   
   $m \leftarrow O\left(\frac{dn^2\ln \frac{nd}{\delta\epsilon}}{\epsilon^4}\right)$\;
   
   Sample $T \sim q^m$\;
   
   $\{x_1, x_2, \dots, x_n\} \leftarrow S$\;
   
   $\{z_1, z_2, \dots, z_m\} \leftarrow T$\;

	\For{$i = 1, \dots, n$}{
	
	Let $p_i(r)$ denote $Est(x_i, r, S)$\;
	
	Let $q_i(r)$ denote $\frac{|B(x_i, r) \cap T|}{m}$\;
	
	$radii \leftarrow \{||z - x_i||: z \in T\} \cup \{0\}$\;
	
	$radii \leftarrow \{r: p_i(r) \leq \gamma, r \in radii\}$\;

	$r_i^* \leftarrow \max \{r: q_i(r) \geq \lambda p_i(r), r \in radii\}$\;
		
	}
	Sample $U \sim q^{20/\epsilon^2}$\;
	$V \leftarrow U \cap \left(\bigcup_{i=1}^n B(x_i, r_i^*)\right)$\;
	Return $\frac{|V|}{|U|}$.\;
	
	

\end{algorithm}


We now now leverage our subroutine, $Est$, to construct a data-copying detector, $Data\_Copy\_Detect$ (Algorithm \ref{alg:main}), that has bounded sample complexity when $p$ is a regular distribution. Like all data-copying detectors (Definition \ref{def:data_copy_detector}), $Data\_Copy\_Detect$ takes as input the training sample $S$, along with the ability to sample from a generated distribution $q$ that is trained from $S$. It then performs the following steps:
\begin{enumerate}
	\item (line 1) Draw an i.i.d sample of $m = O\left(\frac{dn^2\ln \frac{nd}{\delta\epsilon}}{\epsilon^4}\right)$ points from $q$. 
	\item (lines 6 - 10) For each training point, $x_i$, determine the largest radius $r_i$ for which 
	\begin{equation*}
	\begin{split}
	&\frac{|B(x_i, r_i) \cap T|}{m} \geq \lambda Est(x_i, r_i ,S), \\ 
	&Est(x_i, r_i , S) \leq \gamma.
	\end{split}
	\end{equation*}
	\item (lines 12 - 13) Draw a fresh sample of points from $U \sim q^{O(1/\epsilon^2)}$, and use it to estimate the probability mass under $q$ of $\cup_{i=1}^n B(x_i, r_i)$.
\end{enumerate}

In the first step, we draw a \textit{large} sample from $q$. While this is considerably larger than the amount of training data we have, we note that samples from $q$ are considered free, and thus do not affect the sample complexity. The reason we need this many samples is simple -- unlike $p$, $q$ is not necessarily regular, and consequently we need enough points to properly estimate $q$ around every training point in $S$.

The core technical details of $\dc{}$ are contained within step 2, in which data-copying regions surrounding each training point, $x_i$, are found. We use $Est(x, r, S)$ and $\frac{|B(x, r) \cap T|}{m}$ as proxies for $p$ and $q$ in Definition \ref{defn:data_copy}, and then search for the maximal radius $r_i$ over which the desired criteria of data-copying are met for these proxies.  

The only difficulty in doing this is that this could potentially require checking an infinite number of radii, $r_i$. Fortunately, this turns out not to be needed because of the following observation -- we only need to check radii at which a new point from $T$ is included in the estimation $q_i(r)$. This is because these our estimation for $q_i(r)$ does not change between them meaning that our estimate of the ratio between $q$ and $p$ is maximal nearby these points. 

Once we have computed $r_i$, all that is left is to estimate the data-copy rate by sampling $q$ once more to find the total mass of data-copying region, $\cup_{i=1}^n B(x_i, r_i)$. 

\subsection{Performance of Algorithm \ref{alg:main}}

We now show that given enough data, $\dc{}$ provides a close approximation of $\c_q$. 

\begin{theorem}\label{thm:upper_bound5}
$\dc{}$ is a data-copying detector (Definition \ref{def:data_copy_detector}) with sample complexity at most $$m_p(\epsilon, \delta) = O\left(\frac{d\ln\frac{d}{\delta\alpaca p_\alpaca}}{\alpaca^2 p_\alpaca}\right),$$ for all regular distributions, $p$. 
\end{theorem}

Theorem \ref{alg:main} shows that our algorithm's sample complexity has standard relationships with the tolerance parameters, $\epsilon$ and $\delta$, along with the input space dimension $d$. However, it includes an additional factor of $\frac{1}{p_\epsilon}$, which is a distribution specific factor measuring the regularity of the probability distribution. Thus, our bound cannot be used to give a bound on the amount of data needed without having a bound on $p_\epsilon$. 

We consequently view our upper bound as more akin to a convergence result, as it implies that our algorithm is guaranteed to converge as the amount of data goes towards infinity.

\subsection{Applying Algorithm \ref{alg:main} to Halfmoons}\label{sec:experiments}

We now return to the example presented in Figure \ref{fig:halfmoons} and empirically investigate the following question: is our algorithm able to outperform the one given in \cite{MCD2020} over this example? 

To investigate this, we test both algorithms over a series of distributions by varying the parameter $\rho$, which is the proportion of points that are ``copied." Figure \ref{fig:halfmoons} demonstrates a case in which $\rho = 0.4$. Additionally, we include a parameter, $c$, for \cite{MCD2020}'s algorithm which represents the number of clusters the data is partitioned into (with $c$-means clustering) prior to running their test. Intuitively, a larger number of clusters means a better chance of detecting more localized data-copying.

The results are summarized in the following table where we indicate whether the algorithm determined a statistically significant amount of data-copying over the given generated distribution and corresponding training dataset. Full experimental details can be found in Sections \ref{sec:app_experiments} and \ref{sec:experiments_details} of the appendix.

\begin{table}[h]
\caption{Statistical Significance of data-copying Rates over Halfmoons} \label{results_main}
\begin{center}
\begin{tabular}{ |c||c|c|c|c|c| } 
 \hline
 \textbf{Algo} & $\mathbf{q = p}$ & $\mathbf{\rho = 0.1}$ & $\mathbf{0.2}$ & $\mathbf{0.3}$ & $\mathbf{0.4}$ \\ 
 \hline
 \hline
 \textbf{Ours} & \color{blue}no & \color{red}yes & \color{red}yes & \color{red}yes & \color{red}yes \\ 
 \hline
 $\mathbf{c=1}$ & \color{blue}no & \color{blue}no & \color{blue}no & \color{blue}no & \color{blue}no \\ 
 \hline
 $\mathbf{c=5}$ & \color{blue}no & \color{blue}no & \color{blue}no & \color{blue}no & \color{red}yes \\ 
 \hline
 $\mathbf{c=10}$ & \color{blue}no & \color{blue}no & \color{blue}no & \color{blue}no & \color{red}yes \\ 
 \hline
 $\mathbf{c=20}$ & \color{blue}no & \color{blue}no& \color{blue}no & \color{red}yes & \color{red}yes\\ 
 \hline
\end{tabular}
\end{center}
\end{table}

As the table indicates, our algorithm is able to detect statistically significant data-copying rates in all cases it exists. By contrast, \cite{MCD2020}'s test is only capable of doing so when there is a large data-copy rate and when the number of clusters, $c$, is quite large.

\section{Is smoothness necessary for data copying detection?}\label{sec:lower_bound}

Algorithm \ref{alg:main}'s performance guarantee requires that the input distribution, $p$, be regular (Definition \ref{def:regular}). This condition is essential for the algorithm to successfully estimate the probability mass of arbitrarily small balls. Additionally, the parameter, $p_\epsilon$, plays a key role as it serves as a measure of how ``smooth" $p$ is with larger values implying a higher degree of smoothness. 

This motivates a natural question -- can data copying detection be done over unsmooth data distributions? Unfortunately, the answer turns out to be no. In the following result, we show that if the parameter, $p_\epsilon$ is allowed to be arbitrarily small, then this implies that for any data-copy detector, there exists $p$ for which the sample complexity is arbitrarily large.

\begin{theorem}\label{thm:lower_bound5}
Let $B$ be a data-copying detector. Let $\epsilon = \delta = \frac{1}{3}$. Then, for all integers $\a > 0$, there exists a probability distribution $p$ such that $\frac{1}{9\a} \leq p_\alpaca \leq \frac{1}{\a}$, and $m_p(\epsilon, \delta) \geq \a$, implying that $$m_p(\epsilon, \delta) \geq \Omega\left(\frac{1}{p_\epsilon}\right).$$
\end{theorem}

Although Theorem \ref{thm:lower_bound5} is restricted to regular distributions, it nevertheless demonstrates that a bound on smoothness is essential for data copying detection. In particular, non-regular distributions (with no bound on smoothness) can be thought of as a degenerate case in which $p_\epsilon = 0$. 

Additionally, Theorem \ref{thm:lower_bound5} provides a lower bound that complements the Algorithm \ref{alg:main}'s performance guarantee (Theorem \ref{thm:upper_bound5}). Both bounds have the same dependence on $p_\alpaca$ implying that our algorithm is optimal at least in regards to $p_\alpaca$. However, our upper bound is significantly larger in its dependence on $d$, the ambient dimension, and $\alpaca$, the tolerance parameter itself. 

While closing this gap remains an interesting direction for future work, we note that the existence of a gap isn't too surprising for our algorithm, $\dc{}$. This is because $\dc{}$ essentially relies on manually finding the entire region in which data-copying occurs, and doing this requires precise estimates of $p$ at all points in the training sample.  

Conversely, detecting data-copying only requires an \textit{overall} estimate for the data-copying rate, and doesn't necessarily require finding all of the corresponding regions. It is plausible that more sophisticated techniques might able to estimate the data-copy rate \textit{without} directly finding these regions.

\section{Conclusion}

In conclusion, we provide a new modified definition of ``data-copying'' or generating memorized training samples for generative models that addresses some of the failure modes of previous definitions~\cite{MCD2020}. We provide an algorithm for detecting data-copying according to our definition, establish performance guarantees, and show that at least some smoothness conditions are needed on the data distribution for successful detection. 

With regards to future work, one important direction is in addressing the limitations discussed in section \ref{sec:limitations}. Our definition and algorithm are centered around the assumption that the goal of a generative model is to output $q$ that is close to $p$ in a mathematical sense. As a result, we are unable to handle cases where the generator tries to generate \textit{transformed} examples that lie outside the support of the training distribution. For example, a generator restricted to outputting black and white images (when trained on color images) would remain completely undetected by our algorithm regardless of the degree with which it copies its training data. To this end, we are very interested in finding generalizations of our framework that are able to capture such broader forms of data-copying. 








%\newcommand{\crop}[1]{\mathrm{crop}({#1})}
\newcommand{\object}[1]{\mathrm{object}({#1})}
\newcommand{\ba}{A_i}
\newcommand{\bb}{B_i}
\newcommand{\calA}{\mathcal{A}}
\newcommand{\calB}{\mathcal{B}}
\newcommand{\calX}{\mathcal{X}}
\newcommand{\masked}[1]{\mathrm{masked}({#1})}
\newcommand{\bx}{\mathbf{x}}
\newcommand{\SSL}{\textsc{SSL}}
\newcommand{\SSLbb}{\SSL^\mathrm{back}}
\newcommand{\SSLpj}{\SSL^\mathrm{proj}}
\newcommand{\CLF}{\textsc{CLF}}
\newcommand{\CLFbb}{\CLF^\mathrm{back}}
\newcommand{\CLFpj}{\CLF^\mathrm{proj}}
\newcommand{\SUP}{\textsc{SUP}}
\newcommand{\KNN}{\textsc{KNN}}
\newcommand{\KNNset}{\textsc{KNN}^\mathrm{set}}
\newcommand{\KNNprob}{\textsc{KNN}^\mathrm{prob}}
\newcommand{\KNNcl}{\textsc{KNN}^\mathrm{cl}}
\newcommand{\KNNconf}{\textsc{KNN}^\mathrm{conf}}
\newcommand{\RCDM}{\textsc{RCDM}}
\newcommand{\cl}{\mathrm{cl}}
\newcommand{\clpred}{\tilde{\mathrm{cl}}}
\newcommand{\Abox}{\overline{\calA}}
\newcommand{\Bbox}{\overline{\calB}}
\newcommand{\dejavu}{\emph{déjà vu }}
\newcommand{\Dejavu}{\emph{Déjà vu }}

\newcommand{\citations}{{\color{green}[CITE]}}

\definecolor{part_blue}{rgb}{0.2824, 0.4706, .8157}
\definecolor{part_red}{rgb}{0.8392, 0.3725, 0.3725}
\definecolor{part_orange}{rgb}{0.9333, 0.5216, 0.2902}

\DeclareRobustCommand{\mybox}[2][gray!20]{%
\begin{tcolorbox}[   %% Adjust the following parameters at will.
        % breakable,
        left=0pt,
        right=0pt,
        top=0pt,
        bottom=0pt,
        colback=#1,
        colframe=#1,
        width=\dimexpr\columnwidth\relax, 
        % width=\textwidth, 
        enlarge left by=0mm,
        boxsep=5pt,
        arc=0pt,outer arc=0pt,
        ]
        #2
\end{tcolorbox}
}
%\section{Introduction}
\label{sec:intro}
Self-supervised learning (SSL)~\citep{chen2020simclr, chen2020simsiam, zbontar2021barlow, vicreg, caron2020swav, MAE} aims to learn general representations of content-rich data without explicit labels by solving a \textit{pretext task}. In many recent works, such pretext tasks rely on joint-embedding architectures whereby randomized image augmentations are applied to create multiple views of a training sample, and the model is trained to produce similar representations for those views. When using cropping as random image augmentation, the model learns to associate objects or parts (including the background scenery) that co-occur in an image.
However, doing so also arguably exposes the training data to higher privacy risk as objects in training images can be explicitly memorized by the SSL model. For example, if the training data contains the photos of individuals, the SSL model may learn to associate the face of a person with their activity or physical location in the photo. This may allow an adversary to extract such information from the trained model for targeted individuals.

\begin{figure}[t]
    \centering
    \includegraphics[width=1.0\columnwidth]{figures/new_black_swan.pdf}
    \caption{\textbf{Left:} Reconstruction of an SSL training image from a crop containing only the background. The SSL model memorizes the association of this \emph{specific} patch of water (pink square) to this \emph{specific} foreground object (a black swan) in its embedding, which we decode to visualize the full training image. \textbf{Right:} The reconstruction technique fails on a public test image that the SSL model has not seen before.}
    \label{fig:black_swan}
\end{figure}

In this work, we aim to evaluate to what extent SSL models memorize the association of specific objects in training images or the association of objects and their specific backgrounds, and whether this memorization signal can be used to reconstruct the model's training samples. Our results demonstrate that SSL models memorize such associations beyond simple correlation. For instance, in Figure \ref{fig:black_swan} (\textbf{left}), we use the SSL representation of a \emph{training image crop containing only water} and this enables us to reconstruct the object in the foreground with remarkable specificity---in this case a black swan.
By contrast, in Figure \ref{fig:black_swan} (\textbf{right}), when using the \emph{crop from the background of a test set image} that the SSL model \emph{has not seen before}, its representation only contains enough information to infer, through correlation, that the foreground object was likely some kind of waterbird --- but not the specific one in the image.

Figure \ref{fig:black_swan} shows that SSL models suffer from the unintended memorization of images in their training data---a phenomenon we refer to as \emph{déjà vu memorization}
%\footnote{The French loanword \emph{déjà vu} means already-seen, which reflects the type of unintended memorization of objects that the SSL model saw during training.}.
\footnote{The French loanword \emph{déjà vu} means `already-seen', just as an image is seen and memorized in training.}
Beyond visualizing \emph{déjà vu} memorization through data reconstruction, we also design a series of experiments to quantify the degree of memorization for different SSL algorithms, model architectures, training set size, \emph{etc.} We observe that \emph{déjà vu} memorization is exacerbated by the atypically large number of training epochs often recommended in SSL training, as well as certain hyperparameters in the SSL training objective. Perhaps surprisingly, we show that \emph{déjà vu} memorization occurs even when the training set is large---as large as half of ImageNet~\citep{imagenet}---and can continually worsen even when standard techniques for evaluating learned representation quality (such as linear probing) do not suggest increased overfitting. Our work serves as the first systematic study of unintended memorization in SSL models and motivates future work on understanding and preventing this behavior. Specifically, we: 
\begin{itemize}
    \vspace{-0.5em}
    \item Elucidate how SSL representations memorize aspects of individual training images, what we call \emph{déjà vu} memorization;
    \item Design a novel training data reconstruction pipeline for non-generative vision models. This is in contrast to many prominent reconstruction algorithms like \citep{carlini2021extracting, google_diffusion}, which rely on the model itself to generate its own memorized samples and is not possible for SSL models or classifiers;
    \item Propose metrics to quantify the degree of \dejavu memorization committed by an SSL model. This allows us to observe how \dejavu changes with training epochs, dataset size, training criteria, model architecture and more. 
\end{itemize}

%\section{Preliminaries and Related Work}
\label{sec:related}

\textbf{Self-supervised learning} (SSL) is a machine learning paradigm that leverages unlabeled data to learn representations. Many SSL algorithms rely on \emph{joint-embedding} architectures (\emph{e.g.}, SimCLR~\citep{chen2020simclr}, Barlow Twins~\citep{zbontar2021barlow}, VICReg~\citep{vicreg} and Dino~\citep{Dino}), which are trained to associate different augmented views of a given image. For example, in SimCLR, given a set of images $\calA = \{A_1,\ldots,A_n\}$ and a randomized augmentation function $\mathrm{aug}$, the model is trained to maximize the cosine similarity of draws of $\SSL(\mathrm{aug}(A_i))$ with each other and minimize their similarity with $\SSL(\mathrm{aug}(A_j))$ for $i \neq j$. The augmentation function $\mathrm{aug}$ typically consists of operations such as cropping, horizontal flipping, and color transformations to create different views that preserve an image's semantic properties. 

\paragraph{SSL representations.} Once an SSL model is trained, its learned representation can be transferred to different downstream tasks. This is often done by extracting the representation of an image from the \emph{backbone model}\footnote{SSL methods often use a trick called \emph{guillotine regularization}~\citep{Guillotine}, which decomposes the model into two parts: a \emph{backbone model} and a \emph{projector} consisting of a few fully-connected layers. Such trick is needed to handle the misalignment between the pretext SSL task and the downstream task.} and either training a linear probe on top of this representation or finetuning the backbone model with a task-specific head~\citep{Guillotine}.
%Compared to representations learned by supervised learning, SSL representations are often more robust and transferable~\citep{hendrycks2019using, ericsson2021self}, leading to state-of-the-art result on many downstream tasks. To understand the effectiveness of SSL algorithms, several prior works investigated what kind of information the SSL model has learned~\citep{jing2021understanding, ericsson2021self, kalibhat2022towards, RCDM}. In particular, \citet{RCDM} trained a conditional generative model on SSL representations and showed that they encode richer visual details about the input image compared to supervised learning. 
%However, from a privacy perspective, this may be a cause for concern as the model also has more potential to overfit and memorize precise details about the training data compared to supervised learning. We show concretely that this privacy risk can indeed be realized by defining and measuring \emph{déjà vu} memorization.
It has been shown that SSL representations encode richer visual details about input images than supervised models do \cite{RCDM}. However, from a privacy perspective, this may be a cause for concern as the model also has more potential to overfit and memorize precise details about the training data compared to supervised learning. We show concretely that this privacy risk can indeed be realized by defining and measuring \emph{déjà vu} memorization.
\vspace{-0.5em} 
% \paragraph{Privacy risks in ML.} Overfitting in ML occurs when a model memorizes information specific to its training data rather than general population-level information. When the model is trained on privacy-sensitive data, overfitting is especially harmful as an adversary can infer private information about the training data when given access to the model~\citep{yeom2018privacy, feldman2020does}. The simplest and most well-studied form of privacy risk in ML is susceptibility to \emph{membership inference attacks}~\citep{shokri2017membership, salem2018ml, sablayrolles2019white}, where the adversary infers whether an individual is part of the training set or not. More sophisticated privacy attacks include \emph{attribute inference}~\citep{fredrikson2014privacy, mehnaz2022your, jayaraman2022attribute}, where specific attributes about an individual are inferred given others, and \emph{data reconstruction}~\citep{carlini2021extracting, balle2022reconstructing, guo2022bounding}, where entire training samples are recovered from the trained model. Our study of \emph{déjà vu} memorization is similar to both attribute inference and data reconstruction, leveraging SSL representations of the training image background to infer and reconstruct the foreground object.
% \vspace{-0.5em} 
% \paragraph{Training data extraction in NLP.} Our study of \dejavu memorization in SSL models is inspired by similar work in the natural language processing (NLP) domain. \citet{carlini2019secret} first showed that language models exhibit unintended memorization, where given a context string present in its training data, the model can generate the remaining text at test time. This unintended memorization has been further exploited in \citet{carlini2021extracting} to extract training data from GPT-2~\citep{radford2019language} and, more recently, extended to extract memorized images from Stable Diffusion \citep{google_diffusion}. The way by which these works exploit unintended memorization is similar to ours: given partial information about a training sample, the model is prompted to reveal the rest of the sample. In our case, however, since the SSL model is not generative, extraction is significantly harder and requires careful design.

\paragraph{Privacy risks in ML.} When a model is overfit on privacy-sensitive data, it memorizes specific information about its training examples, allowing an adversary with access to the model to learn private information~\citep{yeom2018privacy, feldman2020does}. Privacy attacks in ML range from the simplest and best-studied \emph{membership inference attacks}~\citep{shokri2017membership, salem2018ml, sablayrolles2019white} to \emph{attribute inference}~\citep{fredrikson2014privacy, mehnaz2022your, jayaraman2022attribute} and \emph{data reconstruction}~\citep{carlini2021extracting, balle2022reconstructing, guo2022bounding} attacks. In the former, the adversary only infers whether an individual participated in the training set. Our study of \emph{déjà vu} memorization is most similar to the latter: we leverage SSL representations of the training image background to infer and reconstruct the foreground object. Our approach reflects similar work in the NLP domain \citep{carlini2019secret, carlini2021extracting}: when prompted with a context string present in the training data, a large language model is shown to generate the remainder of string at test time, revealing sensitive text like home addresses. This method was recently extended to extract memorized images from Stable Diffusion \citep{google_diffusion}.  We exploit memorization in a similar manner: given partial information about a training sample, the model is prompted to reveal the rest of the sample. In our case, however, since the SSL model is not generative, extraction is significantly harder and requires careful design.

%\section{Defining \emph{Déjà Vu} Memorization}
\label{sec:definition}

\paragraph{What is \dejavu memorization?} At a high level, the objective of SSL is to learn general representations of objects that occur in nature. This is often accomplished by associating different parts of an image with one another in the learned embedding. Returning to our example in Figure \ref{fig:black_swan}, given an image whose background contains a patch of water, the model may learn that the foreground object is a water animal such as duck, pelican, otter, \emph{etc.}, by observing different images that contain water from the training set. We refer to this type of learning as \emph{correlation}: the association of objects that tend to co-occur in images from the training data distribution.

A natural question to ask is \emph{``Can the reconstruction of the black swan in Figure \ref{fig:black_swan} be reasoned as correlation?''} The intuitive answer may be no, since the reconstructed image is qualitatively very similar to the original image. However, this reasoning implicitly assumes that for a random image from the training data distribution containing a patch of water, the foreground object is unlikely to be a black swan. Mathematically, if we denote by $\mathcal{P}$ the training data distribution and $A$ the image, then
\begin{equation*}
\label{eq:p_corr}
p_\text{corr} := \mathbb{P}_{A \sim \mathcal{P}}(\mathrm{object}(A) = \texttt{black swan} ~|~ \mathrm{crop}(A) = \texttt{water})
\end{equation*}
is the probability of inferring that the foreground object is a black swan through \emph{correlation}. This probability may be naturally high due to biases in the distribution $\mathcal{P}$, \emph{e.g.}, if $\mathcal{P}$ contains no other water animal except for black swans. In fact, such correlations are often exploited to learn a model for image inpainting with great success~\citep{yu2018generative, ulyanov2018deep}.

Despite this, we argue that reconstruction of the black swan in Figure \ref{fig:black_swan} is \emph{not} due to correlation, but rather due to \emph{unintended memorization}: the association of objects unique to a single training image. As we will show in the following sections, the example in Figure \ref{fig:black_swan} is not a rare success case and can be replicated across many training samples. More importantly, failure to reconstruct the foreground object in Figure \ref{fig:black_swan} (\textbf{right}) on test images hints at inferring through correlation is unlikely to succeed---a fact that we verify quantitatively in Section \ref{sec:label inference accuracy}. Motivated by this discussion, we give a verbal definition of \dejavu memorization below, and design a testing methodology to quantify \dejavu memorization in Section \ref{sec:notation and setup}.
\mybox{\textbf{Definition:} A model exhibits \emph{déjà vu memorization} when it retains information so specific to an individual training image, that it enables recovery of aspects particular to that image given a part that does not contain them.
The recovered aspect must be beyond what can be inferred using only correlations in the data distribution.} 

% \textbf{Definition:} A model exhibits \emph{déjà vu memorization} when it retains information so specific to an individual training image, that it enables recovery of aspects particular to that image given a part that does not contain them.
% The recovered aspect must be beyond what can be inferred using only correlations in the data distribution.


 We intentionally kept the above definition broad enough to encompass different types of information that can be inferred about the training image, including but not restricted to object category, shape, color and position. For example, if one can infer that the foreground object is red given the background patch with accuracy significantly beyond correlation, we consider this an instance of \dejavu memorization as well. We mainly focus on object category to quantify \dejavu memorization in Section \ref{sec:quant} since the ground truth label can be easily obtained. We consider other types of information more qualitatively in the visual reconstruction experiments in Section \ref{sec:visualizing}.

\paragraph{Privacy implications of \dejavu memorization.} \Dejavu memorization can be a cause for concern when the training data contains privacy-sensitive information. As a motivating example, consider an SSL model trained on photos of individuals. If the model exhibits \dejavu memorization then, given the face of an individual, it may be possible to infer where the individual was or even visually reconstruct their location in the training image. Such information leakage raises privacy concerns, especially if there was no prior agreement that the trained model may reveal such information to third parties. This hypothetical scenario serves as a motivation that \dejavu memorization should be carefully examined to avoid unintended disclosure of private information in practical applications.

% \begin{figure*}[h]
%     \centering
%     \includegraphics[width = 0.85\textwidth]{figures/SSL_attack_cartoon.png}
%     \caption{We measure memorization by comparing the `target model' trained on the target image ($\SSL_A$ trained on $A_i$ in above example) with the `reference model' not trained on it ($\SSL_B$, above). \textbf{[Top Strip]} A cropping of the image disjoint from the labeled foreground object is embedded using the target model. This embedding is then labeled by a K-Nearest Neighbor (KNN) adversary built on a public set of labeled images, $X$, which it has also embedded using the target model. \textbf{[Bottom Strip]} To account for correlation, the same procedure is followed with the reference model. If the label is only extracted using the target model, it is counted as memorization. If it is extracted using either model, it is counted as correlation. We find that the KNN adversary's predictions using the target model (trained on attacked examples) are significantly more accurate than they are using the reference model, indicating routine memorization of training examples.}
%     \label{fig:ssl attack cartoon}
% \end{figure*}

\begin{figure}[t]
%%%
%SPIDER
%%%
     % \centering
     % \begin{subfigure}[b]{0.25\textwidth}
     %     \centering
     %     \includegraphics[width=\textwidth]{figures/data_split.png}
     %     % \caption{SimCLR correlated \textit{yellow garden spider} examples}
     %     \label{fig:data split}
     % \end{subfigure}
     % \hfill
     % \begin{subfigure}[b]{0.7\textwidth}
     %     \centering
     %     \includegraphics[width=\textwidth]{figures/pipeline_cartoon.png}
     %     \begin{minipage}{5cm}
     %        \vfill
     %    \end{minipage}
     %     % \caption{SimCLR memorized \textit{yellow garden spider} examples}
     %     \label{fig:pipeline cartoon}
     % \end{subfigure}
     \includegraphics[width=\textwidth]{figures/split_and_pipeline_cartoon.png}
\caption[Overview of testing methodology.]{
Overview of testing methodology. \textbf{Left:} Data is split into \emph{target set} $\calA$, \emph{reference set} $\calB$ and \emph{public set} $\calX$ that are pairwise disjoint. $\calA$ and $\calB$ are used to train two SSL models $\SSL_A$ and $\SSL_B$ in the same manner. $\calX$ is used for KNN decoding or for training an RCDM to reconstruct the input at test time. \textbf{Right:} Given a training image $A_i \in \calA$, we use $\SSL_A$ to embed $\crop{A_i}$ containing only the background, as well as the entire set $\calX$ and find the $k$-nearest neighbors of $\crop{A_i}$ in $\calX$ in the embedding space. These KNN samples can be used directly to infer the foreground object (\emph{i.e.}, class label) in $A_i$ using a KNN classifier, or their embeddings can be averaged as input to the trained RCDM to visually reconstruct the image $A_i$. For instance, the RCDM reconstruction results in Figure \ref{fig:black_swan} (left) when given $\SSL_A(\crop{A_i})$ and results in Figure \ref{fig:black_swan} (right) when given $\SSL_A(\crop{B_i})$ for an image $B_i \in \calB$.
%\textbf{Left:} illustration of the three datasets used in our tests. Two private data sets, $A$ and $B$, of equal size are used to train two SSL models, $\SSL_A$ and $\SSL_B$, respectively. A disjoint public set, $X$, is made available to the memorization test to help decode model embeddings. Memorization is only tested on examples $A_i \in A$ that are unique to set $A$. \textbf{Right:} illustration of inference pipeline used in tests. A periphery cropping that excludes the foreground object is taken from private image $A_i$. The KNN then finds the $k$ public set nearest neighbors of the periphery crop in the embedding space of $\SSL_A$. 
%The $\SSL_A$ representation of these $k$ neighbors and of the crop are used by the conditional generative model, RCDM, to reconstruct the foreground object. The labels of these $k$ neighbors are used to recover the foreground object label. (Not pictured) We repeat this process using reference model $\SSL_B$, not trained on image $A_i$, to determine whether the foreground object is still recoverable by learned correlations, e.g. if black swans were the only objects appearing near water in the data distribution. In this instance, the crop's public set neighbors in $\SSL_B$'s representation space include a variety of water animals like ducks, pelicans, and otters. Meanwhile, with $\SSL_A$, the neighbors are nearly all black swans in the same position as the swan of $A_i$.
}
\label{fig:split_and_pipeline_cartoon}
\end{figure}

\textbf{Distinguishing memorization from correlation.} When measuring \dejavu memorization, it is crucial to differentiate what the model associates through \emph{memorization} and what it associates through \emph{correlation}. Our testing methodology is based on the following intuitive definition.
\mybox{\textbf{Definition:} If an SSL model associates two parts in a training image, we say that it is due to \emph{correlation} if other SSL models trained on a similar dataset from $\mathcal{P}$ without this image would likely make the same association. Otherwise, we say that it is due to \emph{memorization}.}

Notably, such intuition forms the basis for differential privacy (DP; \cite{dwork2006calibrating, dwork2013algorithmic})---the most widely accepted notion of privacy in ML.

\subsection{Testing Methodology for Measuring \emph{Déjà Vu} Memorization}
\label{sec:notation and setup}

In this section, we use the above intuition to measure the extent of \dejavu memorization in SSL. Figure \ref{fig:split_and_pipeline_cartoon} gives an overview of our testing methodology.
\vspace{-0.75em}
\paragraph{Dataset splitting.} We focus on testing \dejavu memorization for SSL models trained on the ImageNet-1K dataset~\citep{imagenet}. Our test first splits the ImageNet training set into three independent and disjoint subsets $\calA$, $\calB$ and $\calX$. The dataset $\calA$ is called the \emph{target set} and $\calB$ is called the \emph{reference set}. The two datasets are used to train two separate SSL models, $\SSL_A$ and $\SSL_B$, called the \emph{target model} and the \emph{reference model}. Finally, the dataset set $\calX$ is used as an auxiliary public dataset to extract information from $\SSL_A$ and $\SSL_B$.
%\footnote{See Appendix \ref{sec:appx splits} for details on how the dataset splits are generated.}.
Our dataset splitting serves the purpose of distinguishing memorization from correlation in the following manner. Given a sample $A_i \in \calA$, if our test returns the same result on $\SSL_A$ and $\SSL_B$ then it is likely due to correlation because $A_i$ is not a training sample for $\SSL_B$. Otherwise, because $\calA$ and $\calB$ are drawn from the same underlying distribution, our test must have inferred some information unique to $A_i$ due to memorization. Thus, by comparing the difference in the test results for $\SSL_A$ and $\SSL_B$, we can measure the degree of \dejavu memorization\footnote{See Appendix \ref{sec:appx splits} for details on how the dataset splits are generated.}.
\vspace{-0.75em}
\paragraph{Extracting foreground and background crops.} Our testing methodology aims at measuring what can be inferred about the foreground object in an ImageNet sample given a background crop. This is made possible because ImageNet provides bounding box annotations for a subset of its training images---around 150K out of 1.3M samples. We split these annotated images equally between $\calA$ and $\calB$. Given an annotated image $A_i$, we treat everything inside the bounding box as the foreground object associated with the image label, denoted $\object{A_i}$. We take the largest possible crop that does not intersect with any bounding box as the background crop (or \emph{periphery crop}), denoted $\crop{A_i}$\footnote{We also present another heuristic in \cref{sec:appx corner crop} which takes a corner crop as the background crop, allowing our test to be run without bounding box annotations.}
%Since the labeled object tends to be at the image's center, the corner crop usually excludes it. }
%Because most images in ImageNet are object centric, an image's corner would not include the foreground object.}.
\vspace{-0.75em}
\paragraph{KNN-based test design.} Joint-embedding SSL approaches encourage the embeddings of random crops of a training image $A_i \in \calA$ to be similar. Intuitively, if the model exhibits \dejavu memorization, it is reasonable to expect that the embedding of $\crop{A_i}$ is similar to that of $\object{A_i}$ since both crops are from the same training image. In other words, $\SSL_A(\crop{A_i})$ encodes information about $\object{A_i}$ that cannot be inferred through correlation. However, decoding such information is challenging as these approaches do not learn a decoder associated with the encoder $\SSL_A$.

Here, we leverage the public set $\calX$ to decode the information contained in $\crop{A_i}$ about $\object{A_i}$. More specifically, we map images in $\calX$ to their embeddings using $\SSL_A$ and extract the $k$-nearest-neighbor (KNN) subset of $\SSL_A(\crop{A_i})$ in $\calX$. We can then decode the information contained in $\crop{A_i}$ in one of two ways:
\begin{itemize}
\item \emph{Label inference:} Since $\calX$ is a subset of ImageNet, each embedding in the KNN subset is associated with a class label. If $\crop{A_i}$ encodes information about the foreground object, its embedding will be close to samples in $\calX$ that have the same class label (\emph{i.e.}, foreground object category). We can then use a KNN classifier to infer the foreground object in $A_i$ given $\crop{A_i}$.
\item \emph{Visual reconstruction:} Following \citet{RCDM}, we train an RCDM---a conditional generative model---on $\calX$ to decode $\SSL_A$ embeddings into images. The RCDM reconstruction can recover qualitative aspects of an image remarkably well, such as recovering object color or spatial orientation using its SSL embedding. Given the KNN subset, we average their SSL embeddings and use the trained RCDM model to visually reconstruct $A_i$.
\end{itemize}
In Section \ref{sec:quant}, we focus on quantitatively measuring \dejavu memorization with label inference, and then use the RCDM reconstruction to visualize \dejavu memorization in Section \ref{sec:visualizing}.
%\section{Quantifying \emph{Déjà Vu} Memorization}
\label{sec:quant}

We apply our testing methodology to quantify a specific form of \dejavu memorization: inferring the foreground object (class label) given a crop of the background.

% \paragraph{Extracting model embeddings.} We test \dejavu memorization on two popular SSL algorithms, SimCLR~\citep{chen2020simclr} and VICReg~\citep{vicreg}.
% %\footnote{We present additional SSL models in \cref{sec:appx simclr results}} 
% As described in Section \ref{sec:related}, these algorithms produce two embeddings given an input image: a \emph{backbone} embedding and a \emph{projector} embedding that is derived by applying a small fully-connected network on top of the backbone embedding. Unless otherwise noted, all SSL embeddings refer to the projector embedding.
% To understand whether \dejavu memorization is particular to SSL, we also evaluate embeddings produced by a supervised model $\CLF_A$ trained on $\calA$. We apply the same set of image augmentations as those used in SSL and train $\CLF_A$ using the cross-entropy loss to predict ground truth labels. 
\vspace{-0.75em}
\paragraph{Extracting model embeddings.} We test \dejavu memorization on a variety of popular SSL algorithms, with a focus on VICReg~\citep{vicreg}. These algorithms produce two embeddings given an input image: a \emph{backbone} embedding and a \emph{projector} embedding that is derived by applying a small fully-connected network on top of the backbone embedding. Unless otherwise noted, all SSL embeddings refer to the projector embedding. 
To understand whether \dejavu memorization is particular to SSL, we also evaluate embeddings produced by a supervised model $\CLF_A$ trained on $\calA$. We apply the same set of image augmentations as those used in SSL and train $\CLF_A$ using the cross-entropy loss to predict ground truth labels. 
\vspace{-0.75em}
\paragraph{Identifying the most memorized samples.} Prior works have shown that certain training samples can be identified as more prone to memorization than others~\citep{feldman2020does, watson2021importance, ye2021enhanced}. Similarly, we provide a heuristic to identify the most memorized samples in our label inference test using confidence of the KNN prediction.
Given a periphery crop, $\crop{A_i}$, let $\KNN_A \big( \crop{A_i} \big) \subseteq \calX$ denote its $k$-nearest neighbors in the embedding space of $\SSL_A$. From this KNN subset we can obtain: \textbf{(1)} $\KNNprob_A \big( \crop{A_i} \big)$, the vector of class probabilities (normalized counts) induced by the KNN subset, and \textbf{(2)} $\KNNconf_A \big( \crop{A_i} \big)$, the negative entropy of the probability vector $\KNNprob_A \big( \crop{A_i} \big)$, as confidence of the KNN prediction. When entropy is low, the neighbors agree on the class of $A_i$ and hence confidence is high. 
% \begin{itemize}[noitemsep, leftmargin=*, topsep=0pt]
%     \item $\KNN_A \big( \crop{A_i} \big)$: The most prevalent class in the KNN subset as prediction for the class label $\cl(A_i)$. 
%     \item $\KNNprob_A \big( \crop{A_i} \big)$: The vector of class probabilities (normalized counts) induced by the KNN subset.
%     \item $\KNNconf_A \big( \crop{A_i} \big)$: Negative entropy of the probability vector $\KNNprob_A \big( \crop{A_i} \big)$ as confidence of the KNN prediction. When entropy is low, the neighbors agree on the class of $A_i$ and hence confidence is high. 
% \end{itemize}
We can sort the confidence score $\KNNconf_A \big( \crop{A_i} \big)$ across samples $A_i$ in decreasing order to identify the most confidently predicted samples, which likely correspond to the most memorized samples when $A_i \in \calA$.

\subsection{Population-level Memorization}
\label{sec:label inference accuracy}

%ORIGINAL FIGURE SETUP IN ARXIV: 
% \input{dejavu_training_epochs.tex}
% \input{dejavu_training_set_size.tex}
%PUT ORIGINAL FIGURES SIDE BY SIDE: 
% \input{dejavu_training_epochs_set_size.tex}
%PUT IN NEW FIGURES: 

\begin{wrapfigure}{r}{0.4\textwidth} 
    \centering
    \includegraphics[width=0.4\textwidth]{figures/dejavu_main.pdf}
    \caption{Accuracy of label inference using the target model (trained on $\calA$) vs. the reference model (trained on $\calB$) on the top $\%$ most confident examples $A_i \in \calA$ using only $\crop{A_i}$. For VICReg, there is a large accuracy gap between the two models, indicating a significant degree of \dejavu memorization.}
    \label{fig:dejavu main}
    \vspace{-2ex}
\end{wrapfigure}

Our first measure of \dejavu memorization is population-level label inference accuracy: \emph{What is the average label inference accuracy over a subset of SSL training images given their periphery crops?} 
To understand how much of this accuracy is due to $\SSL_A$'s \dejavu memorization, we compare with a correlation baseline using the reference model: $\KNN_B$'s label inference accuracy on images $A_i \in \calA$. 
In principle, this inference accuracy should be significantly above chance level ($1/1000$ for ImageNet) because the periphery crop may be highly indicative of the foreground object through correlation, \emph{e.g.}, if the periphery crop is a basketball player then the foreground object is likely a basketball.

Figure \ref{fig:dejavu main} compares the accuracy of $\KNN_A$ to that of $\KNN_B$ when inferring the labels of images in $A_i \in \calA$\footnote{The sets $\calA$ and $\calB$ are exchangeable, and in practice we repeat this test on images from $\calB$ using $\SSL_B$ as the target model and $\SSL_A$ as the reference model, and average the two sets of results.} using $\crop{A_i}$.
Results are shown for VICReg and the supervised model; trends for other models are shown in Appendix \ref{sec:appx simclr results}. For both VICReg and supervised models, inferring the class of $\crop{A_i}$ using $\KNN_B$ (dashed line) through correlation achieves a reasonable accuracy that is significantly above chance level. However, for VICReg, the inference accuracy using $\KNN_A$ (solid red line) is significantly higher, and the accuracy gap between $\KNN_A$ and $\KNN_B$ indicates the degree of \dejavu memorization. We highlight two observations: 
\begin{itemize}
    \item The accuracy gap of VICReg is significantly larger than that of the supervised model. This is especially notable when accounting for the fact that the supervised model is trained to associate randomly augmented crops of images with their ground truth labels. In contrast, VICReg has no label access during training but the embedding of a periphery crop can still encode the image label. 
    \item For VICReg, inference accuracy on the $1\%$ most confident examples is nearly $95\%$, which shows that our simple confidence heuristic can effectively identify the most memorized samples. This result suggests that an adversary can use this heuristic to identify vulnerable training samples to launch a more focused privacy attack.
\end{itemize}
\vspace{-.75em}
\paragraph{The \dejavu score. }
The curves of Figure \ref{fig:dejavu main} show memorization across confidence values for a single training scenario.  To study how memorization changes with different hyperparamters, we extract a single value from these curves: the \dejavu \emph{score} at confidence level $p$. In Figure \ref{fig:dejavu main}, this is the gap between the solid red (or gray) and dashed red (or gray) where confidence ($x$-axis) equal $p\%$. In other words, given the periphery crops of set $\calA$, $\KNN_A$ and $\KNN_B$ separately select and label their top $p\%$ most confident examples, and we report the difference in their accuracy. The \dejavu score captures both the degree of memorization by the accuracy gap and the \emph{ability to identify memorized examples} by the confidence level. If the score is 10\% for $p=33\%$, $\KNN_A$ has 10\% higher accuracy on its most confident third of $\calA$ than $\KNN_B$ does on its most confident third. In the following, we set $p = 20\%$, approximately the largest gap for VICReg (red lines) in Figure \ref{fig:dejavu main}. 
% Specifically, the \dejavu \emph{score} on the top $p\%$ most confident examples is,  
% \begin{equation}
%     \mathrm{DejaVu}(p) = \mathrm{Acc}_{\SSL_A}\big( \calA_{\SSL_A, p}  \big) - \mathrm{Acc}_{\SSL_B}\big( \calA_{\SSL_B, p}  \big) \ ,
%     \label{eqn:dejavu score}
% \end{equation}
% where $\calA_{\SSL_A, p}$
% Here we introduce a DejaVu memorization metric that quantify how much a target model is able to retrieve more class information from a crop than the reference model. We define it as:
% where $p$ is a function that take the $p$ purcent most confident samples.
%Figure \ref{fig:dejavu v. training epochs} shows how \dejavu memorization changes with the number of epochs used to train the embedding model (VICReg and supervised, respectively). The training set size is fixed to 300K samples, and label inference accuracy is computed on the top $20\%$ highest confidence examples. The number of epochs has a very strong influence on the degree of memorization for VICReg as the accuracy gap widens when number of epochs increases. We note that 1000 training epochs is used in several SSL works \citep{vicreg, simclr}. Remarkably, this trend in memorization is \emph{not} reflected in the standard metric for evaluating SSL representations: linear probe accuracy. The gray line in Figure \ref{fig:dejavu v. training epochs} shows the train-test accuracy gap of a linear classifier trained on top of the VICReg embeddings. Although there is a sizeable train-test gap, it does not grow significantly beyond 500 epochs. In contrast, \dejavu memorization (blue line) continues to worsen after 500 epochs. Thus, our test can be used as an alternative to linear probe accuracy to evaluate the memorization of SSL models.
% \vspace{-.75em}

% \paragraph{Comparison with the generalization gap} A network that perform very well on a training set while performing poorly on a test set (assuming the training set and test set sampled uniformly from the same distribution) is probably memorizing the training examples without being able to generalize on the test data. One could expect that measuring the difference in accuracy between the training and test set could give us insights on the degree of \dejavu memorization. However, we show in Figure  \ref{fig:dejavu v. training epochs} and \ref{fig:dejavu v. n} that this is not the case. In fact \dejavu memorization can significantly increase while the train-test gap decrease. In our experiments, we did not find a correlation between \dejavu and generalization.
\vspace{-0.75em}
\paragraph{Comparison with the linear probe train-test gap.} A standard method for measuring SSL performance is to train a linear classifier---what we call a `linear probe'---on its embeddings and compute its performance on a held out test set. From a learning theory standpoint, one might expect the linear probe's train-test accuracy gap to be indicative of memorization: the more a model overfits, the larger is the difference between train set and test set accuracy. However, as seen in Figure \ref{fig:dejavu epochs train set size}, the linear probe gap (dark blue) fails to reveal memorization captured by the \dejavu score (red) \footnote{See section \ref{sec:mitigation} for further discussion of the \dejavu score trends of Figure \ref{fig:dejavu epochs train set size}.}.

% \paragraph{Effect of training epochs.} 
% Figure \ref{fig:dejavu v. training epochs} shows how \dejavu memorization changes with training epochs for VICReg. The training set size is fixed to 300K samples. We observe that the number of epochs has a very strong influence on the degree of memorization for VICReg. From 250 to 1000 epochs, the \dejavu score (red curve) grows threefold: from under 10\% to over 30\%. Remarkably, this trend in memorization is \emph{not} reflected in the standard metric for evaluating SSL representations: linear probe accuracy. The dark blue curve shows the train-test linear probe accuracy gap. Although there is a sizeable train-test gap, it only changes by a few percent beyond 250 epochs. %Thus, our test can be used as an alternative to linear probe accuracy to evaluate the memorization of SSL models.
% \vspace{-.75em}
\begin{figure}[ht]
\label{fig:dejavu epochs and dataset}
\begin{minipage}[t]{0.49\textwidth}
\centering
     \begin{subfigure}[b]{0.48\textwidth}
         \centering
         \includegraphics[width=\textwidth]{figures/deja_vu_vs_epochs.png}
         \vspace{-1.5em}
         \caption{\dejavu vs. epochs}
         \label{fig:dejavu v. training epochs}
     \end{subfigure}
     \begin{subfigure}[b]{0.48\textwidth}
         \centering
         \includegraphics[width=\textwidth]{figures/deja_vu_vs_n.png}
         \vspace{-1.5em}
         \caption{\dejavu vs. train set size}
         \label{fig:dejavu v. n}
     \end{subfigure}~
     \vspace{-0.5em}
    \caption{
    Effect of training epochs and train set size with VICReg on \dejavu score (red) in comparison with linear probe accuracy train-test gap (dark blue). 
    \textbf{Left:} \dejavu score increases with training epochs, indicating growing memorization while the linear probe baseline decreases significantly.  
    \textbf{Right:} \dejavu score stays roughly constant with training set size suggesting that memorization may be problematic even for large datasets. %By comparison, the baseline \emph{declines} by half, spuriously suggesting less memorization. 
    %Both trends are not captured according to the linear probe train-test gap---a common method to evaluate generalization of SSL representations.}
    }
    \label{fig:dejavu epochs train set size}
\end{minipage}
\hfill
\begin{minipage}[t]{0.49\textwidth}
\centering
     \begin{subfigure}[b]{0.48\textwidth}
         \centering
         \includegraphics[width=\textwidth]{figures/vicreg_samples_epochs.pdf}
         \vspace{-1.5em}
         \caption{\dejavu vs. epochs}
         \label{fig:per sample v. training epochs}
     \end{subfigure}
     \begin{subfigure}[b]{0.48\textwidth}
         \centering
         \includegraphics[width=\textwidth]{figures/vicreg_samples_datasets.pdf}
         \vspace{-1.5em}
         \caption{\dejavu vs. train set size}
         \label{fig:per sample v. n}
     \end{subfigure}~
     \vspace{-0.5em}
    \caption{
    \definecolor{part_blue}{rgb}{0.2824, 0.4706, .8157}
	\definecolor{part_red}{rgb}{0.8392, 0.3725, 0.3725}
	\definecolor{part_orange}{rgb}{0.9333, 0.5216, 0.2902}
    Partition of samples $A_i \in \calA$ into the four categories: {\color{gray}unassociated} (not shown), {\color{part_orange}memorized}, {\color{part_red}misrepresented} and {\color{part_blue}correlated} for VICReg. The {\color{part_orange}memorized} samples---those whose labels are predicted by $\KNN_A$ but not by $\KNN_B$---occupy a significantly larger share of the training set than the {\color{part_red}misrepresented} samples---those predicted by $\KNN_B$ but not $\KNN_A$ by chance. %At 1000 epochs, $\approx 15\%$ of the training set is {\color{part_orange}memorized}. The trends across training epochs and training set sizes are consistent with those observed in Figure \ref{fig:dejavu epochs train set size}
    }
    \label{fig:partition attack main}
    \end{minipage}
\vspace{-1em} 
\end{figure}

\iffalse

\begin{minipage}[t]{0.49\textwidth}
\centering
     \begin{subfigure}[b]{0.48\textwidth}
         \centering
         \includegraphics[width=0.95\textwidth]{figures/deja_vu_vs_parameters.png}
         \vspace{-0.4em}
         \caption{\dejavu vs. capacity}
         \label{fig:dejavu v. capacity}
     \end{subfigure}
     \hfill
     \begin{subfigure}[b]{0.48\textwidth}
          \tiny
          \centering
          \setlength{\tabcolsep}{3pt}
          \begin{tabular}{|c|c|c|}
            \hline
            Criteria & DV & Acc P/B \\
            \hline
            Supervised & 8.9 & 55.3/61.1\\
            \hline
            Byol\citep{grill2020byol} & 8.0& 54.3/59.4\\
            \hline
            SimCLR\citep{chen2020simclr} & 10.0 & 44.2/54.1\\
            \hline
            Dino\citep{Dino} & 14.5 & 26.3/55.7 \\
            \hline
            Barlow T.\citep{zbontar2021barlow} & 30.5 & 33.7/54.4\\
            \hline
            VICReg\citep{vicreg} & \textbf{33.2} & 40.3/55.2\\
            \hline
          \end{tabular}
          \vspace{1.3em}
          % \caption{\dejavu (DV) vs. SSL Criterion}
          \caption{\dejavu (DV) vs. Criterion}
          \label{tab:dejavu vs. criterion}
    \end{subfigure}
    \vspace{-0.5em}
    \caption{
    Comparison of \dejavu score for different architectures and training criteria. \textbf{Left:} \dejavu score with VICReg for resnet (purple) and vision transformer (green) architectures versus number of model parameters. As expected, memorization grows with larger model capacity. This trend is more pronounced for convolutional (resnet) than transformer (ViT) architectures. \textbf{Right:} Comparison of \dejavu score and ImageNet validation accuracy (P: using projector embeddings, B: using backbone embeddings) for various SSL criteria. \textbf{Nearly all SSL models have more memorization than the supervised baseline.} 
    % Effect of training epochs and train set size on \dejavu score.
    % \textbf{Left:} \dejavu score increases with higher number of training epochs, indicating worsening memorization.
    % \textbf{Right:} \dejavu score stays roughly constant with training set size. Both trends are not captured according to the linear probe train-test gap---a common method to evaluate generalization of SSL representations.
    }
\end{minipage}
\vspace{-2em} 
\end{figure}

\begin{figure}[ht]
\begin{minipage}[t]{0.49\textwidth}
\centering
     \begin{subfigure}[b]{0.49\textwidth}
         \centering
         \includegraphics[width=\textwidth]{figures/epochs_lb_attk_epochs_acc_top1_legend.pdf}
         \caption{\dejavu vs. epochs}
         \label{fig:dejavu v. training epochs}
     \end{subfigure}
     \begin{subfigure}[b]{0.49\textwidth}
         \centering
         \includegraphics[width=\textwidth]{figures/epochs_lb_attk_datasets_acc_top1_legend.pdf}
         \caption{\dejavu vs. train set size}
         \label{fig:dejavu v. n}
     \end{subfigure}~
     \begin{subfigure}[b]{0.32\textwidth}
         \centering
         \includegraphics[width=0.8\textwidth]{figures/dejavu_vs_parameters.pdf}
         \caption{\dejavu vs. capacity}
         \label{fig:dejavu v. n}
     \end{subfigure}
    \caption{
    Effect of training epochs and train set size on \dejavu score.
    \textbf{Left:} \dejavu score increases with higher number of training epochs, indicating worsening memorization.
    \textbf{Right:} \dejavu score stays roughly constant with training set size. Both trends are not captured according to the linear probe train-test gap---a common method to evaluate generalization of SSL representations.}
    \end{minipage}
\vspace{-1em} 
\end{figure}

\begin{table}[ht]
  \footnotesize
  \centering
  \begin{tabular}{|c|c|}
    \hline
    Supervised & 8.9\\
    \hline
    SimCLR\citep{chen2020simclr} & 10.0\\
    \hline
    Byol\citep{grill2020byol} & 8.0\\
    \hline
    Dino\citep{Dino} & 14.5\\
    \hline
    Barlow T.\citep{zbontar2021barlow} & 30.5\\
    \hline
    VICReg\citep{vicreg} & \textbf{33.2}\\
    \hline
  \end{tabular}
  \caption{DejaVu Score 20\% Conf for various SSL methods.}
  \label{tab:two-row-table}
\end{table}
\vspace{-1em} 
\fi

\iffalse
\begin{figure}[ht]
\begin{minipage}[t]{.49\textwidth}
\centering
     \begin{subfigure}[b]{0.49\textwidth}
         \centering
         \includegraphics[width=\textwidth]{figures/epochs_lb_attk_epochs_acc_top1_legend.pdf}
         \caption{\dejavu vs. epochs}
         \label{fig:dejavu v. training epochs}
     \end{subfigure}
     \hfill
     \begin{subfigure}[b]{0.49\textwidth}
         \centering
         \includegraphics[width=\textwidth]{figures/epochs_lb_attk_datasets_acc_top1_legend.pdf}
         \caption{\dejavu vs. train set size}
         \label{fig:dejavu v. n}
     \end{subfigure}
\caption{
Effect of training epochs and train set size on \dejavu score.
\textbf{Left:} \dejavu score increases with higher number of training epochs, indicating worsening memorization.
\textbf{Right:} \dejavu score stays roughly constant with training set size. Both trends are not captured according to the linear probe train-test gap---a common method to evaluate generalization of SSL representations.}
\label{fig:dejavu epochs and dataset}
\end{minipage}
\hfill
\begin{minipage}[t]{.49\textwidth}
     \centering
     \begin{subfigure}[b]{0.49\textwidth}
         \centering
         \includegraphics[width=\textwidth]{figures/criteria_epochs.pdf}
         \caption{criteria comparison}
         \label{fig:dejavu v. criteria}
     \end{subfigure}
     \hfill
     \begin{subfigure}[b]{0.49\textwidth}
         \centering
         \includegraphics[width=\textwidth]{figures/architecture_epochs.pdf}
         \caption{architecture comparison}
         \label{fig:dejavu v. arch}
     \end{subfigure}
\caption{
Effect of SSL training criteria and model architectures on \dejavu score.
%the accuracy gap between target model (trained on $\calA$) and reference model (trained on $\calB$) making predictions on their 20\% most confident examples.
\textbf{Left:} \dejavu score for various training criteria.
%Barlow and VICReg have the heaviest degree of memorization, while SimCLR and BYOL have the least. 
%Note that we show detailed reconstructions of SimCLR's training data in Section \ref{sec:visualizing} despite its relatively low degree of \dejavu. 
%Regardless, Although SimCLR and BYOL have the least, we  visualize detailed reconstructions with SimCLR in section \ref{sec:mem v corr} 
All SSL models have significantly more \dejavu than the supervised baseline. \textbf{Right:} \dejavu score versus epochs for various training architectures. As expected, lower capacity architectures (Resnet18, Resnet34) reduce \dejavu but not completely. 
}
\label{fig:dejavu criteria and architecture}
\end{minipage}
\vspace{-1em} 
\end{figure}
\fi
% %\begin{figure}[ht]
%%%
%VICREG
%%%
     \centering
     \begin{subfigure}[b]{0.49\textwidth}
         \centering
         \includegraphics[width=\textwidth]{figures/sample_level_training_epochs.pdf}
         \caption{Categories of training samples vs. number of epochs}
         \label{fig:sample level epochs}
     \end{subfigure}
     \hfill
     \begin{subfigure}[b]{0.49\textwidth}
         \centering
         \includegraphics[width=\textwidth]{figures/sample_level_training_set_size.pdf}
         \caption{Categories of training samples vs. training set size}
         \label{fig:sample level training size}
     \end{subfigure}
\caption{
\definecolor{part_blue}{rgb}{0.2824, 0.4706, .8157}
\definecolor{part_red}{rgb}{0.8392, 0.3725, 0.3725}
\definecolor{part_orange}{rgb}{0.9333, 0.5216, 0.2902}
Partition of samples $A_i \in \calA$ into the four categories: {\color{gray}unassociated} (not shown), {\color{part_orange}memorized}, {\color{part_red}misrepresented} and {\color{part_blue}correlated}. The {\color{part_orange}memorized} samples---ones whose labels are predicted by $\KNN_A$ but not by $\KNN_B$---occupy a significantly larger share for VICReg compared to the supervised model, indicating that sample-level \dejavu memorization is more prevalent in VICReg. %The trends across number of training epochs and training set sizes are consistent with those observed in Figures \ref{fig:dejavu epochs and dataset} and \ref{fig:dejavu criteria and architecture}.
}
\label{fig:partition attack main appendix}
\end{figure}
% \paragraph{Effect of training set size.} 
% Figure \ref{fig:dejavu v. n} shows how \dejavu memorization responds to the model's training set size. The number of training epochs is fixed to 1000. Interestingly, training set size appears to have almost \emph{no} influence on the \dejavu score (red line), indicating that memorization is equally prevalent with a 100K dataset and a 500K dataset (which suggests that \dejavu memorization may be detectable for larger datasets). Meanwhile, the linear probe train-test accuracy gap \emph{declines} by half as the dataset size grows, failing to represent the memorization quantified by our test. 
% The trend is completely different according to linear probe accuracy (dark blue line), the train-test gap shrinks substantially when increasing the training set size from 100K to 500K. This highlights that the train-test gap is not able to capture \dejavu memorization. %Our evidence suggests that \dejavu memorization may be detectable even for large-scale training datasets. 
%\vspace{-.75em}

\vspace{-.75em} 
\subsection{Sample-level Memorization}
\label{sec:dissection}

% Section \ref{sec:label inference accuracy} shows the \emph{average} level of \dejavu memorization on a subset of the training set $\calA$. However, this average tell us only what the attacker success rate might be without explicitly describing how much of the datatset is \dejavu memorized.
The \dejavu score shows, \emph{on average}, how much better an adversary can select and classify images when using the target model trained on them. 
This average score does not tell us how many individual images have their label successfully recovered by $\KNN_A$ but not by $\KNN_B$. In other words, how many images are exposed by virtue of \emph{being in training set} $\calA$: a risk notion foundational to differential privacy. 
% However, from the perspective of an individual image $A_i \in \calA$, it is informative to know whether it was correctly classified 
To better quantify what fraction of the dataset is at risk, we perform a sample-level analysis by fixing a sample $A_i \in \calA$ and observing the label inference result of $\KNN_A$ vs. $\KNN_B$.
To this end, we partition samples $A_i \in \calA$ based on the result of label inference into four distinct categories: {\color{gray}\textbf{Unassociated}} - label inferred with neither KNN; {\color{part_orange}\textbf{Memorized}} - label inferred only with $\KNN_A$; {\color{part_red}\textbf{Misrepresented}} - label inferred only with $\KNN_B$; {\color{part_blue}\textbf{Correlated}} - label inferred with both KNNs. 
% \begin{multicols}{2}
% \begin{itemize}
%     \vspace{-.75em}
%     \setlength\itemsep{0.15em}
%     \item {\color{gray}Unassociated}: label inferred with neither KNN   
%     \item {\color{part_orange}Memorized}: label only inferred by $\KNN_A$
%     \item {\color{part_red}Misrepresented}: label only inferred with $\KNN_B$
%     \item {\color{part_blue}Correlated}: label inferred with both KNNs
%     \vspace{-.75em}
% \end{itemize}
% \end{multicols}
Intuitively, {\color{gray}unassociated} samples are ones where the embedding of $\crop{A_i}$ does not encode information about the label. {\color{part_blue}Correlated} samples are ones where the label can be inferred from $\crop{A_i}$ using correlation, \emph{e.g.}, inferring the foreground object is basketball given a crop showing a basketball player. Ideally, the {\color{part_red}misrepresented} set should be empty but contains a small portion of examples due to chance.
\emph{Déjà vu} memorization occurs for {\color{part_orange}memorized} samples where the embedding of $\SSL_B$ does not encode the label but the embedding of $\SSL_A$ does. To measure the pervasiveness of \dejavu memorization, we compare the size of the {\color{part_orange}memorized} and {\color{part_red}misrepresented} sets.
Figure \ref{fig:partition attack main} shows how the four categories of examples change with number of training epochs and training set size. The {\color{gray}unassociated} set is not shown since the total share adds up to one. The {\color{part_red}misrepresented} set remains under $5\%$ and roughly unchanged across all settings, consistent with our explanation that it is due to chance. In comparison, VICReg's {\color{part_orange}memorized} set surpasses $15\%$ at 1000 epochs. Considering that up to 5\% of these memorized examples could also be due to chance, we conclude that \textbf{at least 10\% of VICReg's training set is \dejavu memorized.} 
%is many times larger than its {\color{part_red}misrepresented} set, indicating substantial sample-level \dejavu memorization. 
%In fact, \textbf{it is 15\% of the training set that is \dejavu memorized with VICReg.}
%The trends across different number of training epochs and training set sizes match those observed in Section \ref{sec:label inference accuracy}. % On the other hand, the supervised model's {\color{part_orange}memorized} set is only marginally larger than its {\color{part_red}misrepresented} set.

% The trends across different number of training epochs and training set sizes match those observed in Section \ref{sec:label inference accuracy}: Increasing the number of epochs increases \dejavu memorization (Figure \ref{fig:per sample v. training epochs}), while increasing the training set size does not appear to reduce \dejavu memorization (Figure \ref{fig:per sample v. n}). 
%\section{Visualizing \emph{Déjà Vu} Memorization}
\label{sec:visualizing}
Beyond enabling label inference using a periphery crop, we show that \dejavu memorization allows the SSL model to encode other forms of information about a training image. Namely, we train an RCDM \citep{RCDM} on the public dataset $\calX$ and use it to visually reconstruct training images given their periphery crop.
We aim to answer the following two questions: \textbf{(1)} Can we visualize the distinction between correlation and \dejavu memorization? \textbf{(2)} What foreground object details can be extracted from the SSL model beyond class label? 
% \begin{enumerate}[noitemsep, leftmargin=*, topsep=0pt]
%     \item Can we visualize the distinction between correlation and \dejavu memorization? 
%     \item What foreground object details can be extracted from the SSL model beyond class label? 
% \end{enumerate}
\vspace{-0.5em}
\paragraph{Reconstruction pipeline.}
RCDM is a conditional generative model that is trained on the \emph{backbone embedding} of images $X_i \in \calX$ to generate an image that resembles $X_i$. All training images are first face-blurred for privacy purposes. \citet{RCDM} showed that the backbone embedding of SSL models contains more low-level information about the image, making them better suited for conditioning the RCDM.
At test time, following the pipeline in Figure \ref{fig:split_and_pipeline_cartoon}, we first use the projector embedding to find the KNN subset for the periphery crop, $\crop{A_i}$, and then average their backbone embeddings as input to the RCDM model. Ideally, when the public set contains enough representative images, the average representation of the KNN subset encodes objects present in $A_i$, and the RCDM model decodes this representation to visualize these objects.
% \begin{figure}[ht]
%%%
%VICREG
%%%
     \centering
     \begin{subfigure}[b]{0.49\textwidth}
         \centering
         \includegraphics[width=\textwidth]{figures/sample_level_training_epochs.pdf}
         \caption{Categories of training samples vs. number of epochs}
         \label{fig:sample level epochs}
     \end{subfigure}
     \hfill
     \begin{subfigure}[b]{0.49\textwidth}
         \centering
         \includegraphics[width=\textwidth]{figures/sample_level_training_set_size.pdf}
         \caption{Categories of training samples vs. training set size}
         \label{fig:sample level training size}
     \end{subfigure}
\caption{
\definecolor{part_blue}{rgb}{0.2824, 0.4706, .8157}
\definecolor{part_red}{rgb}{0.8392, 0.3725, 0.3725}
\definecolor{part_orange}{rgb}{0.9333, 0.5216, 0.2902}
Partition of samples $A_i \in \calA$ into the four categories: {\color{gray}unassociated} (not shown), {\color{part_orange}memorized}, {\color{part_red}misrepresented} and {\color{part_blue}correlated}. The {\color{part_orange}memorized} samples---ones whose labels are predicted by $\KNN_A$ but not by $\KNN_B$---occupy a significantly larger share for VICReg compared to the supervised model, indicating that sample-level \dejavu memorization is more prevalent in VICReg. %The trends across number of training epochs and training set sizes are consistent with those observed in Figures \ref{fig:dejavu epochs and dataset} and \ref{fig:dejavu criteria and architecture}.
}
\label{fig:partition attack main appendix}
\end{figure}
%\begin{figure*}[t!]
%%%
%DAM
%%%
     \centering
     \begin{subfigure}[b]{0.49\textwidth}
         \centering
         \includegraphics[width=\textwidth]{figures/dam_corr.png}
         \caption{A {\color{part_blue}correlated} dam example}
         \label{fig:dam correlated}
     \end{subfigure}
     \hfill
     \begin{subfigure}[b]{0.49\textwidth}
         \centering
         \includegraphics[width=\textwidth]{figures/dam_mem.png}
         \caption{A {\color{part_orange}memorized} dam example}
         \label{fig:dam memorized}
     \end{subfigure}
\caption{
{\color{part_blue}Correlated} and {\color{part_orange}Memorized} examples from the \emph{dam} class. Both $\SSL_A$ and $\SSL_B$ are SimCLR models.
\textbf{Left:} The periphery crop (pink square) contains a concrete structure that is often present in images of dams. Consequently, the trained RCDM can reconstruct the foreground object using representations from both $\SSL_A$ and $\SSL_B$ through this correlation.
\textbf{Right:} The periphery crop only contains a patch of water. The embedding produced by $\SSL_B$ only contains enough information to infer that the foreground object is related to water, as reflected by its KNN set and RCDM reconstruction. In contrast, the embedding produced by $\SSL_A$ memorizes the association of this patch of water with dam and the RCDM can visualize the embedding to produce images of dams.
}
\vspace{-1ex}
\label{fig:mem v corr dam}
\end{figure*}


\begin{figure*}[t!]
%%%
%DAM
%%%
     \centering
     \begin{subfigure}[b]{0.49\textwidth}
         \centering
         \includegraphics[width=\textwidth]{figures/dam_corr.png}
         \caption{A {\color{part_blue}correlated} dam example}
         \label{fig:dam correlated}
     \end{subfigure}
     \hfill
     \begin{subfigure}[b]{0.49\textwidth}
         \centering
         \includegraphics[width=\textwidth]{figures/dam_mem.png}
         \caption{A {\color{part_orange}memorized} dam example}
         \label{fig:dam memorized}
     \end{subfigure}
\caption[Correlated and Memorized examples from the \emph{dam} class.]{
Correlated and Memorized examples from the \emph{dam} class. Both $\SSL_A$ and $\SSL_B$ are SimCLR models.
\textbf{Left:} The periphery crop (pink square) contains a concrete structure that is often present in images of dams. Consequently, the trained RCDM can reconstruct the foreground object using representations from both $\SSL_A$ and $\SSL_B$ through this correlation.
\textbf{Right:} The periphery crop only contains a patch of water. The embedding produced by $\SSL_B$ only contains enough information to infer that the foreground object is related to water, as reflected by its KNN set and RCDM reconstruction. In contrast, the embedding produced by $\SSL_A$ memorizes the association of this patch of water with dam and the RCDM can visualize the embedding to produce images of dams.
}
\label{fig:mem v corr dam}
\end{figure*}


\begin{figure}[t!]
%%%
%BADGER
%%%
     \centering
     \begin{subfigure}[b]{0.49\textwidth}
         \centering
         \includegraphics[width=\textwidth]{figures/euro_badgers.png}
         \caption{{\color{part_orange}Memorized} European badgers}
         \label{fig:euro badgers}
     \end{subfigure}
     \hfill
     \begin{subfigure}[b]{0.49\textwidth}
         \centering
         \includegraphics[width=\textwidth]{figures/amer_badgers.png}
         \caption{{\color{part_orange}Memorized} American badgers}
         \label{fig:amer badgers}
     \end{subfigure}
\caption[Visualization of \dejavu memorization beyond class label.]{
Visualization of \dejavu memorization beyond class label. Both $\SSL_A$ and $\SSL_B$ are VICReg models. 
The four images shown belong to the memorized set of $\SSL_A$ from the \emph{badger} class. RCDM reconstruction using embeddings from $\SSL_A$ can reveal not only the correct class label, but also the specific badger species: \emph{European} (left) and \emph{American} (right). Such information does not appear to be memorized by the reference model $\SSL_B$.
} 
\label{fig:in class badger}
\end{figure}


% \subsection{Visualizing Correlation vs. Memorization}
\label{sec:mem v corr}
\vspace{-0.5em} 
\paragraph{Visualizing Correlation vs. Memorization.}
Figure \ref{fig:mem v corr dam} shows examples of dams from the {\color{part_blue}correlated} set (left) and the {\color{part_orange}memorized} set (right) as defined in Section \ref{sec:dissection}, along with the associated KNN set and RCDM reconstruction. Both $\SSL_A$ and $\SSL_B$ are SimCLR models. In Figure \ref{fig:dam correlated}, the periphery crop is represented by the pink square, which contains concrete structure attached to the dam's main structure. As a result, both $\SSL_A$ and $\SSL_B$ produce embeddings of $\crop{A_i}$ whose KNN set in $\calX$ consist of dams, \emph{i.e.}, there is a correlation between the concrete structure in $\crop{A_i}$ and the foreground dam. The RCDM reconstructions also consist of dams or structures that closely resemble dams. 
In Figure \ref{fig:dam memorized}, the periphery crop only contains a patch of water, which does not strongly correlate with dams in the ImageNet distribution. Evidently, the reference model $\SSL_B$ embeds $\crop{A_i}$ close to that of other objects commonly found in water, such as sea turtle and submarine. In contrast, the KNN set according to $\SSL_A$ all contain dams despite the vast number of alternative possibilities within the ImageNet classes, and the RCDM reconstruction outputs dams as well which highlight memorization in $\SSL_A$ between this specific patch of water and the dam. %\footnote{See Appendix \ref{sec:appx visualization} to see the same trend in the \emph{yellow garden spider} class.}


% \subsection{Visualizing Memorization Beyond Class Label}
% \label{sec:in class variation}
\vspace{-0.5em} 
\paragraph{Visualizing Memorization Beyond Class Label.}
We now use our reconstruction algorithm to show that \dejavu memorization can be exploited to reveal detailed information beyond class label. Figure \ref{fig:in class badger} shows four examples of badgers from the {\color{part_orange}memorized} set. In all four images, the periphery crop (pink square) does not contain any indication that the foreground object is a badger. Despite this, the KNN set and the RCDM reconstruction using $\SSL_A$ consistently produce images of badgers, while the same does not hold for $\SSL_B$.
More interestingly, reconstructions using $\SSL_A$ in Figure \ref{fig:euro badgers} all contain \emph{European} badgers, while reconstructions in Figure \ref{fig:amer badgers} all contain \emph{American} badgers, accurately reflecting the species of badger present in the respective training images. Since ImageNet-1K does \emph{not} differentiate between these two species of badgers, our reconstructions show that SSL models can memorize information that is highly specific to a training sample beyond its class label\footnote{See Appendix \ref{sec:appx visualization} for additional visualization experiments.}.%\footnote{See Appendix \ref{sec:appx visualization} for the same trend in the \emph{aircraft carrier} class.}.





%\vspace{-.5em} 
\section{Mitigation of \dejavu memorization}
\label{sec:mitigation}
% We do not have an understanding on why \dejavu occur so strongly in some SSL pretraining, however we present additional experiments that shed light on which parameters have the biggest impact on \dejavu memorization.
\begin{figure}[ht]
\label{fig:mitigations}
\begin{minipage}[t]{0.5\textwidth}
\centering
     \begin{subfigure}[b]{0.47\textwidth}
         \centering
         \includegraphics[width=\textwidth]{figures/dejavu_vicreg_param.png}
         \vspace{-1.5em}
         \caption{Loss hyper-parameter}
         \label{fig:dejavu v. invariance}
     \end{subfigure}
     \begin{subfigure}[b]{0.49\textwidth}
         \centering
         \includegraphics[width=\textwidth]{figures/deja_vu_vs_layer.png}
         \vspace{-1.5em}
         \caption{Guillotine regularization}
         \label{fig:dejavu v. guillotine}
     \end{subfigure}~
     \vspace{-0.5em}
    \caption[Effect of two kinds of hyper-parameters on VICReg memorization. ]{
    Effect of two kinds of hyper-parameters on VICReg memorization. \textbf{Left:} \dejavu score (red) versus the \emph{invariance} loss parameter, $\lambda$, used in the VICReg criterion (100k dataset). Larger $\lambda$ significantly reduces \dejavu, with minimal effect on linear probe validation performance (green). $\lambda = 25$ (near maximum \dejavu) is recommended in the original paper \textbf{Right:} \dejavu score versus projector layer---guillotine regularization \cite{Guillotine}---from projector to backbone. Removing the projector can significantly reduce \dejavu. Appendix \ref{sec:guillotine} shows that the backbone still can memorize, however; we demonstrate reconstructions using the SimCLR backbone.
    }
\end{minipage}
\hfill
\begin{minipage}[t]{0.48\textwidth}
\centering
     \begin{subfigure}[b]{0.46\textwidth}
         \centering
         \includegraphics[width=\textwidth]{figures/deja_vu_vs_parameters.png}
         \vspace{-1.3em}
         \caption{\dejavu vs. capacity}
         \label{fig:dejavu v. capacity}
     \end{subfigure}
     \hfill
     \begin{subfigure}[b]{0.52\textwidth}
          \tiny
          \centering
          \setlength{\tabcolsep}{3pt}
          \begin{tabular}{|c|c|c|}
            \hline
            Criteria & DV & Acc P/B \\
            \hline
            Supervised & 8.9 & 55.3/61.1\\
            \hline
            Byol\citep{grill2020byol} & 8.0& 54.3/59.4\\
            \hline
            SimCLR\citep{chen2020simclr} & 10.0 & 44.2/54.1\\
            \hline
            Dino\citep{Dino} & 14.5 & 26.3/55.7 \\
            \hline
            Barlow T.\citep{zbontar2021barlow} & 30.5 & 33.7/54.4\\
            \hline
            VICReg\citep{vicreg} & \textbf{33.2} & 40.3/55.2\\
            \hline
          \end{tabular}
          \vspace{1.3em}
          % \caption{\dejavu (DV) vs. SSL Criterion}
          \caption{\dejavu (DV) vs. Criterion}
          \label{tab:dejavu vs. criterion}
    \end{subfigure}
    \vspace{-1.4em}
    \caption[Effect of model architecture and criterion on \dejavu memorization.]{
    %Comparison of \dejavu score for different architectures and training criteria. 
    Effect of model architecture and criterion on \dejavu memorization. 
    \textbf{Left:} \dejavu score with VICReg for resnet (purple) and vision transformer (green) architectures versus number of model parameters. As expected, memorization grows with larger model capacity. This trend is more pronounced for convolutional (resnet) than transformer (ViT) architectures. \textbf{Right:} Comparison of \dejavu score 20\% conf. and ImageNet linear probe validation accuracy (P: using projector embeddings, B: using backbone embeddings) for various SSL criteria. %\textbf{Nearly all SSL models have more memorization than the supervised baseline.} 
    % Effect of training epochs and train set size on \dejavu score.
    % \textbf{Left:} \dejavu score increases with higher number of training epochs, indicating worsening memorization.
    % \textbf{Right:} \dejavu score stays roughly constant with training set size. Both trends are not captured according to the linear probe train-test gap---a common method to evaluate generalization of SSL representations.
    }
    \end{minipage}
\end{figure}
We cannot yet make claims on why \dejavu occurs so strongly for some SSL training settings and not for others. To gain some intuition for future work, we present additional observations that shed light on which parameters have the most salient impact on \dejavu memorization.
\vspace{-.75em}
\paragraph{Déjà vu memorization worsens by increasing number of training epochs.} 
Figure \ref{fig:dejavu v. training epochs} shows how \dejavu memorization changes with number of training epochs for VICReg. The training set size is fixed to 300K samples. From 250 to 1000 epochs, the \dejavu score (red curve) grows \emph{threefold}: from under 10\% to over 30\%. Remarkably, this trend in memorization is \emph{not} reflected by the linear probe gap (dark blue), which only changes by a few percent beyond 250 epochs. 

%\vspace{-.75em}
\paragraph{Training set size has minimal effect on \dejavu memorization.} Figure \ref{fig:dejavu v. n} shows how \dejavu memorization responds to the model's training set size. The number of training epochs is fixed to 1000. Interestingly, training set size appears to have almost \emph{no} influence on the \dejavu score (red line), indicating that memorization is equally prevalent with a 100K dataset and a 500K dataset. This result suggests that \dejavu memorization may be detectable even for large datasets. Meanwhile, the standard linear probe train-test accuracy gap \emph{declines} by more than half as the dataset size grows, failing to represent the memorization quantified by our test. 
% The trend is completely different according to linear probe accuracy (dark blue line), the train-test gap shrinks substantially when increasing the training set size from 100K to 500K. This highlights that the train-test gap is not able to capture \dejavu memorization. Our evidence suggests that \dejavu memorization may be detectable even for large-scale training datasets. 
\vspace{-0.5em}
\paragraph{Training loss hyper-parameter has a strong effect.} 
%We show in Figure \ref{fig:dejavu v. training epochs} that the number of training epochs is an important factor that can increase significantly \dejavu memorization. In contrast, the dataset size does not impact much \dejavu as shown in Figure \ref{fig:dejavu epochs train set size}. 
Loss hyper-parameters, like VICReg's invariance coefficient (Figure \ref{fig:dejavu v. invariance}) or SimCLR's temperature parameter (Appendix Figure \ref{fig:simclr temperature}) significantly impact \dejavu with minimal impact on the linear probe validation accuracy.

\vspace{-0.5em}
\paragraph{Some SSL criteria promote stronger \dejavu memorization.} Table \ref{tab:dejavu vs. criterion} demonstrates that the degree of memorization varies widely for different training criteria. VICReg and Barlow Twins have the highest \dejavu scores while SimCLR and Byol have the lowest.
%\footnote{We show detailed reconstructions of SimCLR's training data in Section \ref{sec:visualizing} despite its relatively low degree of \dejavu.}.
With the exception of Byol, all SSL models have more \dejavu memorization than the supervised model. Interestingly, different criteria can lead to similar linear probe validation accuracy and very different degrees of \dejavu as seen with SimCLR and Barlow Twins. Note that low degrees of \dejavu can still risk training image reconstruction, as exemplified by the SimCLR reconstructions in Figures \ref{fig:mem v corr dam} and \ref{fig:mem v corr spider}. 
%\vspace{-1em}
\vspace{-0.5em}
\paragraph{Larger models have increased \dejavu memorization.} Figure \ref{fig:dejavu v. capacity} validates the common intuition that lower capacity architectures (Resnet18/34) result in less memorization than their high capacity counterparts (Resnet50/101). 
% \begin{wrapfigure}{r}{0.25\textwidth} 
%     \centering
%     \includegraphics[width=0.25\textwidth]{figures/attk_layer_acc_top1_legend.pdf}
%     \caption{\dejavu memorization versus layer from backbone (0) to projector output (3).}
%     \label{fig:dejavu vs layer}
%     \vspace{-8ex}
% \end{wrapfigure}
We see the same trend for vision transformers as well. %This comes with a tradeoff, since reduced model capacity can result in a nontrivial degradation of representation quality\cite{vicreg, simclr}.  
\vspace{-0.5em}
\paragraph{Guillotine regularization can help reduce \dejavu memorization.} Previous experiments were done using the projector embedding. In Figure \ref{fig:dejavu v. guillotine}, we present how Guillotine regularization\citep{Guillotine} (removing final layers in a trained SSL model) impacts \dejavu with VICReg\footnote{Further experiments are available in Appendix \ref{sec:guillotine}.}. Using the backbone embedding instead of the projector embedding seems to be the most straightforward way to mitigate \dejavu memorization. However, as demonstrated in Appendix \ref{sec:appx backbone results}, backbone representation with low \dejavu score can still be leveraged to reconstruct some of the training images.

\section{Conclusion}
\label{sec:conclusion}

We defined and analyzed \dejavu memorization, a notion of unintended memorization of partial information in image data. As shown in Sections \ref{sec:quant} and \ref{sec:visualizing}, SSL models can largely exhibit \dejavu memorization on their training data, and this memorization signal can be extracted to infer or visualize image-specific information.
Since SSL models are becoming increasingly widespread as foundation models for image data, negative consequences of \dejavu memorization can have profound downstream impact and thus deserves further attention. 
Future work should focus on understanding how \dejavu emerges in the training of SSL models and why methods like Byol are much more robust to \dejavu than VICReg and Barlow Twins. In addition, trying to characterize which data points are the most at risk of \dejavu could be crucial to get a better understanding on this phenomenon. 

%\graphicspath{{./chapters/chapter5/}}
\chapter{Location Trace Privacy}

\newcommand{\inputTikZ}[2]{%  
     \scalebox{#1}{\input{#2}}  
}

\newcommand{\bG}{\mathbf{G}}
\newcommand{\bT}{\mathbf{T}}
\newcommand{\trace}{\mathbf{tr}}
\newcommand{\calF}{\mathcal{F}}
\newcommand{\calN}{\mathcal{N}}
\newcommand{\calQ}{\mathcal{Q}}
\newcommand{\calR}{\mathcal{R}}
\newcommand{\Zu}{{Z_{\mathbb{I}_U}}}
\newcommand{\Zs}{{Z_{\mathbb{I}_S}}}
\newcommand{\Zx}{{Z_{\mathbb{I}_x}}}
\newcommand{\Zy}{{Z_{\mathbb{I}_y}}}
\newcommand{\Xu}{{X_{\mathbb{I}_U}}}
\newcommand{\Xs}{{X_{\mathbb{I}_S}}}
\newcommand{\Xx}{{X_{\mathbb{I}_x}}}
\newcommand{\Xy}{{X_{\mathbb{I}_y}}}
\newcommand{\Gu}{{G_{\mathbb{I}_U}}}
\newcommand{\Gs}{{G_{\mathbb{I}_S}}}
\newcommand{\Gx}{{G_{\mathbb{I}_x}}}
\newcommand{\Gy}{{G_{\mathbb{I}_y}}}
\newcommand{\Spairs}{\calS_{\text{pairs}}}
\newcommand{\Sigmag}{\Sigma^{(g)}}
\newcommand{\Sigmaeff}{\Sigma_{\text{eff}}}
\newcommand{\maxeig}{{\max \text{eig}}}
\newcommand{\leff}{l_{\text{eff}}}
\newcommand{\Is}{{\mathbb{I}_S}}
\newcommand{\Iu}{\mathbb{I}_U}
\newcommand{\Ix}{\mathbb{I}_x}
\newcommand{\Iy}{\mathbb{I}_y}


\newtheorem{prop}[theorem]{Proposition}

\theoremstyle{definition}

%For moving around footnote at end 
\newcommand\blfootnote[1]{%
  \begingroup
  \renewcommand\thefootnote{}\footnote{#1}%
  \addtocounter{footnote}{-1}%
  \endgroup
}
\section{Introduction}
\label{sec:introduction}

%motivation + specification 
Location data is acutely sensitive information, detailing where we live, work, eat, shop, worship, and often when, too. Yet increasingly, location data is being uploaded for smartphone services such as ride hailing and weather forecasting and then being brokered in a thriving user location aftermarket to advertisers and even investors \citep{nyt}. Users share location `traces' when they release a sequence of locations, often across a short period of time. These traces are then used by central servers to monitor traffic trends, track individual fitness, target marketing, and even to study the effectiveness of social-distancing ordinances \citep{wash_post}. Here, we aim to provide a \emph{local} privacy guarantee, wherein traces are sanitized at the user level before being transmitted to a centralized service. Note that this requires different guarantees and mechanisms than in \emph{aggregate} applications making queries on large location trace databases. 

Specifically, we guarantee a radius $r$ of privacy at any sensitive time point or combination of time points within a given trace. This is challenging due to the fact that the locations within traces are highly inter-dependent. Informally, traces tend to follow relatively smooth trajectories in time. If not sanitized carefully, that knowledge alone may be exploited to infer actual locations from the released version of the trace. This work centers on designing meaningful privacy definitions and corresponding mechanisms that takes this dependence into account. 


Broadly speaking, the vast majority of prior work on rigorous data privacy can be divided into two classes that differ by the kind of guarantee offered: differential and inferential privacy. Differential privacy (DP) guarantees that the participation of a single person in a dataset does not change the probability of any outcome by much. In contrast, inferential privacy guarantees that an adversary who has a certain degree of prior knowledge cannot make certain sensitive inferences.

DP for releasing aggregate statistics of a spatio-temporal dataset has been well studied \citep{traffic_monitoring, quantifying_dp_cao, bayesian_DP, dependent_dp}. There, the idea is to add enough noise to released statistics such that the effect of any user's participation is obscured, even if their locations are highly correlated to each other or to those of other users. Here, such a guarantee does not apply since we aim to release a sanitized version of a single user's trace.

In this local case we cannot rule out the possibility that the data curator knows who each individual is and who participated. Instead, we want to guarantee that event level information \emph{about} each trace remains private. In this work, at any sensitive time $t$ we mask whether the user visited location A or location B for any A,B less than $r$ apart. Without \emph{ad hoc} modifications, standard DP tools are insufficient for achieving this for the primary reasons that 1) the domain of location is virtually unbounded and 2) locations are highly dependent across a short period of time. To see this, consider the following instinctual approaches to achieving location trace privacy. 

\paragraph{Approach A:} apply Local Differential Privacy (LDP) to each trace. Imagine a dataset of traces, each from a separate individual. Applying LDP implies that every trace has nearly the same probability of releasing the same sanitized version. This would be robust to arbitrary side information about dependence between locations in any one trace. Unfortunately, the amount of additive noise needed to achieve this would destroy nearly all utility: sanitized traces from California would have almost the same probability of showing up in Connecticut as do those from New York. Even if we constrained the domain to just Manhattan, this definition would not permit enough utility to perform e.g. traffic monitoring. 

\paragraph{Approach B:} apply LDP to each location within a trace. To preserve some utility, imagine a single trace as a dataset of $n$ locations, each of which enjoys $\varepsilon$-LDP guarantees. This alone is not robust to arbitrary dependence between locations. By the logic of group LDP, it does satisfy $k \varepsilon$-LDP regardless of the dependence between any $k$ locations. This approach has two setbacks. First, how to set $k$ is unclear. Technically, all points in the trace are correlated, so to ward off worst-case correlations one might set it to the length of the trace, which is identical to Approach A. Second, even if location is bounded to a single city or county, satisfying this definition would still destroy nearly all utility. We cannot use sanitized traces for traffic monitoring if locations from either side of town have about same probability of being sanitized to the same value. 

\paragraph{Approach C:} apply LDP guarantees to each location within a trace, but only within any region less than width $r$. This definition is known as Geo-Indistinguishability (GI) \citep{GI}. GI provides a substitute for restricting the domain of location allowing us to salvage some utility. Here, only locations within $r$ of each other are required to have $\varepsilon$-LDP guarantees. In DP parlance, we might say that `neighboring traces' have one location altered by $\leq r$ and are identical everywhere else. This gives us the guarantee we want for a trace with one location, but not with more than one location. To see why, compare with Approach B. Analogously, $(\varepsilon, r)$-GI along a trace provides $(k \varepsilon, r)$-GI to any subset of $k$ locations. Like Approach B, setting $k$ is unclear. Yet unlike Approach B, GI is not resistant to arbitrary dependence between any $k$ locations. Any dependence where a change in one or more location(s) by $r$ implies a change in some other location(s) by $\geq r$ breaks the GI guarantee. Even with the simplest models of dependence (e.g. if we know the true trace ought to move in a straight line) this is a problem. 

To reiterate, applying LDP to traces or to locations within traces (Approaches A \& B) does not provide a principled method for meaningful privacy with reasonable utility. GI adapts LDP by giving guarantees only within a radius $r$. But in relaxing LDP, GI compromises the standard DP tools for handling obvious dependences between data-points like group DP. In our eyes, this warrants an \emph{inferentially private} approach. Here, we continue to provide privacy within a radius $r$, thus allowing for utility. Yet instead of providing resistance to arbitrary dependence across any $k$ locations, we aim to provide resistance to natural models of dependence between all locations. One may view such models as an adversary's prior beliefs about what traces are likely, like the straight-line prior mentioned earlier. 


In contrast with differential privacy, providing inferential privacy guarantees is more complex, and has been less studied. It is however appropriate for applications such as ours, where information must be released based on a single person's data, the features of which are private and dependent. \cite{pufferfish} provide a formal inferential privacy framework called Pufferfish, and design mechanisms for specific Pufferfish instances. As these instances do not apply to our setting, we adapt the Pufferfish framework to location privacy and more broadly to releasing any sequence of real-valued private information. 

%GRAPHICAL MODEL
\begin{figure*}[h]
	\centering
	\begin{subfigure}[b]{.45\textwidth}
		\centering
		\inputTikZ{0.8}{chapters/chapter5/graphical_model.tex}
		\caption{}
		\label{fig:full model}
	\end{subfigure}
	\begin{subfigure}[b]{.45\textwidth}
		\centering
		\inputTikZ{0.8}{chapters/chapter5/graphical_model-1.tex}
		\caption{}
		\label{fig:condensed model}
	\end{subfigure}
	\caption{(a) An example graphical model of a four point trace $X$. (b) The more general grouped version of the model in (a), with the secret set $\Xs = \{X_1, X_2\}$ and the remaining set $\Xu = \{X_3, X_4\}$. 
	%$Z_u$, $\Zs$ are shown.
	}
	\label{fig:graphical models}
\end{figure*}

\paragraph{Contributions:}In this work, we propose an inferentially private approach to guaranteeing a radius $r$ of privacy for sensitive points in location traces in three parts: 
\begin{itemize}
	\item First, we propose an adaptable privacy framework tailored to sequences of highly dependent datapoints that adapts Pufferfish privacy \citep{Pufferfish} to use R\'enyi Differential Privacy (RDP) \citep{renyi}. Given a model of dependence between points, this framework more appropriately estimates the risk of inference within radius $r$ on points of interest than do vanilla LDP approaches. 
	\item We then demonstrate how to implement our framework for the highly flexible and expressive setting of Gaussian process (GP) priors. These nonparametric models capture the spatiotemporal aspect of location data \citep{PCS_GP, ATM_GP, Traffic_GP}. GPs have a natural synergy with R\'enyi privacy enabling an interpretable upper bound on privacy loss for additive Gaussian privacy mechanisms (that add Gaussian noise to each point). Using this, we design a semidefinite program (SDP) that optimizes the correlation of such mechanisms to minimize privacy loss without destroying utility, efficiently thwarting the inference of sensitive locations. 
	\item Finally, we provide experiments on both location trace and home temperature data to demonstrate the advantage of these techniques over Approach C mechanisms like GI. We find that our mechanisms successfully obscure sensitive locations while respecting utility constraints, even when the prior model is misspecified. 
\end{itemize}
 
Ultimately, by resisting only reasonable kinds of dependence in the data we are able to offer both meaningful privacy and utility. We show that our framework is robust to misspecification of this reasonable dependence and offers a privacy loss that is both tractable and interpretable. 



\section{Preliminaries and Problem Setting}
\label{sec:preliminaries}
%notation / definitions
A user transmits a sequence of $N$ 2-dimensional locations along with their corresponding timestamps, collectively forming a `trace'. We `unroll' the trace into $n$ real-valued random variables $X = \{X_1, X_2, \dots, X_n\}$. A trace of 10 2d locations has $n = 2 \times 10 = 20$ random variables $X_i$. Instead of releasing the raw trace $X$, the user releases a private version $Z = \{Z_1, Z_2, \dots, Z_n\}$, by way of an additive noise mechanism $Z = X + G$, where $G = \{G_1,G_2, \dots, G_n\}$ is random noise produced by a privacy mechanism.

%notation / definitions 
An adversary, receiving the obscured trace $Z$, then reasons about the true locations at some sensitive time(s). To reference the sensitive times, we use index set $\Is$. If the sensitive indices are $\Is = \{1,2\}$, the corresponding location values are $\Xs = \{X_{1}, X_{2}\}$ (e.g. referring to the two coordinates of one location). When inferring the true value of $\Xs$, the adversary makes use of the remaining points in the trace at indices $\mathbb{I}_U = [n] \backslash \Is$, denoted $\Xu$, with obscured values $\Zu$. This separation of points into $\Xs$ and $\Xu$ is represented in \textbf{Figure \ref{fig:graphical models}}. 

We use location as a guiding example, but such inter-dependent traces $X$ could take the form of home temperature time series data or spatial data like 3D facial maps used for identification. Going forward, we will continue to denote $X = \{X_1, X_2, \dots, X_n\}$ with the understanding that \emph{any} subsequence of $d$ points e.g. $\Xs = \{X_2, X_6, \dots\}$ could represent a $d$-dimensional sensitive value, or $Nd$ points could represent $N$ $d$-dimensional sensitive values. 

For the real-valued distributions considered here, $P_{\times}( \bullet )$ refers to a density of distribution $\times$ on r.v. $\bullet$ and $P_{\times}(\bullet | *)$ is its regular conditional density given $*$. 

\subsection{Background}
GI limits what can be inferred about the sensitive $\Xs$ from its corresponding $\Zs$, but not from the remaining locations $\Zu$. To do so we need a privacy definition that specifies what events of random variable $\Xs$ we wish to obscure, which realistic priors of inter-dependence to protect against, and a privacy loss. 

\subsection{Basic and Compound Secrets}

We borrow heavily from the Pufferfish framework \citep{pufferfish}, and specialize it for the setting of location traces. We define our own set of \emph{secrets} --- the collection of events we wish to obscure --- and \emph{discriminative pairs}, the pairs of secret events we do not want an adversary to tell between. 
%To give privacy within a radius $r$, we do not want an adversary to be able to infer whether one visited any two locations A vs. B within $r$ of each other. To formalize this, we define two classes of secrets: 

\paragraph{Basic Secrets \& Pairs} 
After releasing $Z$, we do not want an adversary with a reasonable prior on $X$, $\calP \in \Theta$, to have sharp posterior beliefs about the user's location at some sensitive time (e.g. one of the sensitive times in \textbf{Figure \ref{fig:nyc_example}} of Appendix \ref{apx: Illustrations}). As such, the adversary cannot distinguish whether the user visited location A or some nearby location B at that time. Let $x_s\in \R^2$ represent a possible assignments to $\Xs$, hypothesizing the true sensitive location. Any such assignment is secret, $\calS = \{ \Xs = x_s  : x_s \in \R^2\}$. Specifically, we want the posterior probability of any two assignments to $\Xs$ within a radius $r$ to be close: $\Spairs = \{(x_s, x_s'):\|x_s - x_s'\|_2 \leq r\}$. This protects a single time within a trace of locations. More generally, in the context of spatiotemporal data of any dimension, we call this a \emph{basic secret}. 

\paragraph{Compound Secrets \& Pairs} 
%\textbf{Figure \ref{fig:nyc_trace}} has not one but three time points that have sensitive locations in the trace. Let their index sets be ${\Is}_1$, ${\Is}_2$, and ${\Is}_3$. 
Suppose we have three sensitive times (again as in \textbf{Figure \ref{fig:nyc_example}}). A mechanism that blocks inference on each of these separately does not prevent inference on the combination of them simultaneously. To obscure hypotheses on \emph{all three} of these, we modify our set of secrets to any combination of assignments to each secret location: 
\begin{align*}
	\calS = \big\{ \{\Xs_1 = x_{s1}\} \cap \{\Xs_2 = x_{s2}\} \cap \{\Xs_3 = x_{s3} &\} \\
	: x_{si} \in \R^2, i \in [3] &\big\} \ .
\end{align*} 
Now, the set of discriminative pairs is any two assignments to all three secret locations: 
\begin{align*}
	\Spairs = \Big\{\big( \{x_{s1}, x_{s2}, x_{s3}\} &, \{x_{s1}', x_{s2}', x_{s3}'\}\big) \\
	&: \| x_{si} - x_{si}' \|_2 \leq r, \ i \in [3] \Big\}
\end{align*}
This protects against compound hypotheses: if daycare and work are within $r$ of each other, this keeps an adversary from inferring $\Xs_1 = $ `daycare' \emph{and} $\Xs_2 = $ `work' versus $\Xs_1 = $ `work' \emph{and} $\Xs_2 = $ `daycare'. More generally, in the context of spatiotemporal data of any dimension, we call this a \emph{compound secret}. Intuitively, a mechanism that protects a compound secret of locations close together in time prevents a Bayesian adversary from leveraging the remainder of the trace to infer direction of motion at those sensitive times. Note that bounding the privacy loss of a compound secret does not bound the privacy loss of its constituent basic secrets.

Going forward, we refer to $\Is$ as the `secret set'. 

\subsubsection{Gaussian Processes}
For the purpose of location privacy, it is important to choose a prior class $\Theta$ such that the conditional distribution $P_\calP(\Xu | \Xs)$ is simple to compute for any secret set $\Is$ and any prior $\calP \in \Theta$. Of course, it is also critical that the prior class naturally models the data, and thus consists of `reasonable assumptions' for adversaries. GPs satisfy both these requirements. We model a full $d$-dimensional trace sampled at $N$ times by `unrolling' it into a $n = dN$ dimensional GP. 
\begin{definition}\emph{Gaussian process} 
	A trace $X$ is a Gaussian process if $X_{\mathbb{I}_M}$ has a multivariate normal distribution for any set of indices $\mathbb{I}_M \subset [n]$. If $X$ is a gaussian process, then the function $i \rightarrow \E[X_i]$ is called the mean function and the function $(i,j) \rightarrow \text{Cov}(X_i, X_j)$ is called the kernel function. 
\end{definition}
In this work, the kernel uses locations' time stamps to compute their covariance $(t_i, t_j) \rightarrow \text{Cov}(X_i, X_j)$, but generally could use any side information provided with each location. 
%Let $(X_t^x)_{t \geq 0}$ be the continuous representation of one dimension of trace $X$ in time $t$. 
%\begin{definition}\emph{Gaussian process} 
%	A stochastic process $(X_t^x)_{t \geq 0}$ is considered a Gaussian process if $[X_{t_1}^x, X_{t_1}^x, \dots, X_{t_n}^x]$ has a multivariate normal distribution for any set of times $t_1, t_2, \dots t_n \geq 0$. If $(X_t^x)_{t \geq 0}$ is a gaussian process, then the function $t \rightarrow \E[X_t]$ is called the mean function and the function $(s,t) \rightarrow \text{Cov}(X_s, X_t)$ is called the kernel function. 
%\end{definition}

%This is particularly nice since, if different dimensions are independent, we may treat them as separate traces. If not, correlation across dimensions can be modeled into the covariance matrix. We will describe this in greater detail in the following sections. 

GPs have simple, closed form conditional distributions. Let $X \sim \calN(\mu, \Sigma)$, where $\mu \in \R^{n}$ and $\Sigma \in \R^{{n} \times {n}}$. Then, the random variable $\Xu | \{\Xs = x_s\} \sim \calN(\mu_{u|s}, \Sigma_{u|s})$, where $\mu_{u|s} = \mu_u + \Sigma_{us} \Sigma_{ss}^{-1} (x_s - \mu_s)$ and $\Sigma_{u|s} = \Sigma_{uu} - \Sigma_{us}\Sigma_{ss}^{-1} \Sigma_{su}$. Here, $\mu_s$ denotes the mean vector $\mu$ accessed at indices $\Is$ and $\Sigma_{su}$ denotes the covariance matrix $\Sigma$ accessed at rows $\Is$ and columns $\mathbb{I}_U$. 

For GP priors, we will use additive noise $G \sim \calN(\mathbf{0}, \Sigmag)$. Thus $Z = X + G$, too, is multivariate normal. Furthermore, the distribution of any set of variables conditioned on any other set of variables in \textbf{Figure \ref{fig:graphical models}} belongs to some multivariate normal distribution.

GPs have been shown to successfully model mobility \citep{Traffic_GP, PCS_GP, ATM_GP}, even in the domain of surveillance video \citep{surveillance_GP}.  Furthermore, although these non-parametric models are characterized by second order statistics, GPs are capable of complexity rivaling that of deep neural networks \citep{Deep_NN_GP}, allowing for scalability to more complex models and domains. Our proposed results and algorithms may be applied regardless of the complexity of the chosen GP. 


\subsubsection{R\'enyi Differential Privacy}
\label{sec:renyi_dp}
In the following section, we propose a privacy definition that adapts R\'enyi Differential Privacy (RDP) \citep{renyi} to the Pufferfish framework. RDP resembles Differential Privacy \citep{DP}, except instead of bounding the maximum probability ratio or \emph{max divergence} of the distribution on outputs for two neighboring databases, it bounds the \emph{R\'enyi divergence} of order $\lambda$, defined in Equation \eqref{eqn: renyi} for distributions $\calP_1$ and $\calP_2$. The R\'enyi divergence bears a nice synergy with Gaussian processes. If $\calP_1 = \calN(\mu_1, \Sigma)$ and $\calP_2 = \calN(\mu_2, \Sigma)$ --- two mean-shifted normal distributions --- the R\'enyi divergence takes on a simple closed form shown in Equation \eqref{eqn: normal renyi}. 
%For distributions $\calP_1$ and $\calP_2$ on random variable $X$, recall that the R\'enyi Divergence is defined as in Equation --- (left). Additionally, R\'enyi divergence bears nice synergy with Gaussian processes, since the R\'enyi divergence of any two mean-shifted multivariate normal distributions has a simple closed form. If  as seen below in Equation --- (right). 
\begin{align}
	\label{eqn: renyi}
	D_\lambda \binom{\calP_1}{\calP_2} 
	&= \frac{1}{\lambda - 1} \log \E_{x \sim \calP_2} \Big( \frac{P_{\calP_1}(X = x)}{P_{\calP_2}(X = x)} \Big)^\lambda \\
	\label{eqn: normal renyi}
	&= \frac{\lambda}{2} (\mu_1 - \mu_2)^\intercal \Sigma^{-1} (\mu_1 - \mu_2)
\end{align}
%As shown in \citep{subsampled_renyi}, $(\lambda, \varepsilon)$-RDP can be directly translated into the language of $(\varepsilon, \delta)$-DP, if the DP interpretation is preferred. 
%\begin{align*}
%	D_\lambda \binom{\calN(\mu_1, \Sigma)}{\calN(\mu_1, \Sigma)}
%	&= \frac{\lambda}{2} (\mu_1 - \mu_2)^\intercal \Sigma^{-1} (\mu_1 - \mu_2)
%\end{align*}
We will make use of this in defining and bounding privacy loss in the next section. 
\section{Conditional Inferential Privacy}
We now propose a privacy framework that is tailored to sequences of correlated data, Conditional Inferential Privacy (CIP). CIP guarantees a radius $r$ of indistinguishability for the basic or compound secrets associated with any secret set $\Is$. Specifically, CIP protects against any adversary with a specific prior on \emph{the shape} of the trace, and is agnostic to their prior on the absolute location of the trace. We call the set of such prior distributions a Conditional Prior Class.

\begin{definition} \emph{Conditional Prior Class}
\label{def: conditional prior class}
%	For $X = \{X_1, \dots, X_n\}$, prior distributions $\calP_i, \calP_j$ on $X$ are said to belong to the same conditional prior class $\Theta$ if the distribution of any subset of $X$ conditioned on the rest of $X$ is identical up to a mean shift. Formally, if conditional distributions $P_{\calP_i}(\Xu | \Xs = x_s) = P_{\calP_j}(c_{ij\Is} + \Xu | \Xs = x_s)$ for all $x_s$. Constant $c_{ij\Is}$ may change for different $\calP_i,\calP_j$ and partitions $\Is, \Iu$.
	For $X = \{X_1, \dots, X_n\}$, prior distributions $\calP_i, \calP_j$ on $X$ are said to belong to the same conditional prior class $\Theta$ if a constant shift in the conditioned $x_s$ results in a constant shift on the distribution of $\Xu$. Formally, if conditional distributions $P_{\calP_i}(\Xu | \Xs = x_s) = P_{\calP_j}(\Xu + c_{ij\Is}^u  | \Xs = x_s + c_{ij\Is}^s )$ for all $x_s$.
\end{definition}

For instance, prior $P_{\calP_i}$ may concentrate probability on traces passing through Los Angeles, while $P_{\calP_j}$ concentrates on traces passing through London. Conditioning on each secret in the pair $(x_s, x_s')$ in L.A. is analogous to conditioning on each secret in the pair $(x_s + c_{ij\Is}^s, x_s' + c_{ij\Is}^s)$ in London. The corresponding pair of conditional distributions on $\Xu$ in London ($P_{\calP_j}$) are copies of those in L.A. ($P_{\calP_i}$) shifted by $c_{ij\Is}^u$. What matters is that the set of all pairs of conditional distributions under $P_{\calP_i}$ induced by secret pairs $(x_s, x_s')$ is identical to those under $P_{\calP_j}$ up to a mean shift. See Appendix \ref{apx: GP prior class} for a more detailed discussion of conditional prior classes.  
 
% Two prior distributions $\calP_1$ and $\calP_2$ may have completely different marginal distributions on any $X_i$ but still belong to the same conditional prior class. For instance, $\calP_1$ could concentrate probability on the event that the middle point in the trace passes through Manhattan, and $\calP_2$ could have high uncertainty over which state in the U.S. the trace is in. As long as they have the same prior on the dependence between points, the trace is protected by the same CIP mechanism. 

\begin{definition} \emph{ $(\varepsilon, \lambda)$-Conditional Inferential Privacy $(\Spairs, r, \Theta)$}
	Given compound or basic discriminative pairs $\Spairs$ associated with $\Is$, a radius of privacy $r$, a conditional prior class, $\Theta$, and a privacy parameter, $\varepsilon > 0$, a privacy mechanism $Z = \calA(X)$ satisfies $(\varepsilon, \lambda)$-CIP$(\Spairs, r, \Theta)$ if for all $(s_i, s_j) \in \Spairs$, 
%	where 
%	\begin{align*}
%	\Spairs = \Big\{\big( \{x_{s1}, x_{s2}, \dots\}&, \{x_{s1}', x_{s2}', \dots\}\big) \\
%	 &: \| x_{sk} - x_{sk}' \|_2 \leq 2r, \forall \ k\Big\} 
%	\end{align*} 
	and all prior distributions $\calP \in \Theta$, where $P_\calP(s_i), P_\calP(s_j) > 0$, 
	\begin{align}
		\label{eqn:CIP loss}
		D_\lambda \binom{P_{\calA, \calP} (Z | \Xs = s_i)}{P_{\calA, \calP} (Z | \Xs = s_j)} &\leq \varepsilon
	\end{align}
\end{definition}

%Comparison with DP and PF
CIP departs from DP type notions of privacy like Approaches A$\rightarrow$C primarily by resisting only a restricted class of inter-dependence --- the conditional prior class --- as opposed to arbitrary dependence of any $k$ locations. Unlike approaches A and B, we are able to preserve utility for tasks like traffic monitoring. Unlike approach C, CIP is still resistant to realistic models of location inter-dependence. 

While this definition borrows heavily from the Pufferfish framework, it has a few key modifications. Pufferfish is generally described from a central, not local model. We specialize the kinds of secrets and discriminative pairs for the case of local location trace privacy. Additionally, we specialize the type of prior distribution class needed for this local setting: the conditional prior class. Finally, we relax the strict max divergence (max log odds) criterion of the Pufferfish definition to a R\'enyi divergence. This guarantees that --- with high probability on draws of \emph{realistic} traces $Z|\Xs$ --- the log odds will be bounded by $\varepsilon$. As $\lambda \rightarrow \infty$, the log odds are bounded for all traces, i.e. the max divergence is bounded. We formalize this in Theorem \ref{thm: prior-posterior}. 

The R\'enyi criterion of CIP greatly improves its flexibility. Unlike the standard DP Approaches A$\rightarrow$C which only take probabilities over the mechanism, we do not have full control over the randomness at play: it is partially from $\calA$ defined by us and from $\calP$ intrinsic to the data. Unlike max divergence, R\'enyi divergence is available in closed form for many distributions, allowing for a more flexible privacy framework. The $\lambda$ parameter helps us tune how strict a CIP definition is and how much noise we need to add. This allows us to design mechanisms that are resistant to natural models of dependence while preserving utility. 

%Second, providing a reasonable bound on the max divergence -- even for a restricted conditional prior class -- would require adding an unreasonable amount of noise and destroy utility.

\subsection{Properties}

We now identify key properties that make the CIP guarantee interpretable and robust. 

\paragraph{Interpretability:} CIP guarantees that a Bayesian adversary with any prior distribution on traces $\calP$ in the conditional prior class $\Theta$ does not learn much about basic or compound secrets from the released trace $Z$. For basic secrets, this means that the adversary's posterior beliefs regarding sensitive location $\Xs$ are not much sharper than their prior beliefs before witnessing $Z$.  
\begin{theorem} \emph{Prior-Posterior Gap:} 
\label{thm: prior-posterior}
	An $(\varepsilon, \lambda)$-CIP mechanism with conditional prior class $\Theta$ guarantees that for any event $O$ on sanitized trace $Z$
	\begin{align*}
		\bigg| \log \frac{P_{\calP, \calA}(s_i | Z \in O)}{P_{\calP, \calA}(s_j | Z \in O)} - \log \frac{P_{\calP}(s_i)}{P_{\calP}(s_j)} \bigg| \leq \varepsilon'
	\end{align*}
	for any $\calP \in \Theta$ with probability $\geq 1 - \delta$ over draws of $Z|\Xs=s_i$ or $Z|\Xs=s_j$, where $\varepsilon'$ and $\delta$ are related by
	\begin{align*}
		\varepsilon' = \varepsilon + \frac{\log \nicefrac{1}{\delta}}{\lambda - 1} \ .
	\end{align*}
	This holds under the condition that $Z|\Xs = s_i$ and $Z|\Xs = s_j$ have identical support. 
\end{theorem}
A CIP mechanism depends only on the conditional prior describing the data, not the data itself. Suppose an adversary's prior beliefs on $\Xs$ are uniform over some region. For $\lambda = 5$ and $\varepsilon = 0.1$, there is only a $\approx 1\%$ chance that their posterior odds on $s_i,s_j$ will be more than 3.5, and a $\approx 10\%$ chance that they will be more than 2. This `chance' is over draws of likely remaining locations $\Xu$ and the additive noise $G$.  Proofs of all results are in Appendix \ref{apx: proofs}.

For additive noise mechanisms like $\calA(X) = X + G = Z$, the CIP loss can be split into two terms: one accounting for the direct privacy loss of $\Zs$ on $\Xs$ and a second accounting for the inferential privacy loss of $\Zu$ on $\Xs$ via $\Xu$.

\begin{lemma}\emph{Conditional Independence}
	\label{lem: renyi additive loss}
	For an additive noise mechanism, a fully dependent trace as in \textbf{Figure \ref{fig:full model}}, and any prior $\calP$ on $X$ the CIP loss may be expressed as
	\begin{align}
	\label{eqn:two terms}
		&D_\lambda \binom{P_{\calA, \calP}(Z | \Xs = s_i)}{P_{\calA, \calP}(Z | \Xs = s_j)}  \\ 
		\vspace{2em}
		&= \sum_{i \in \Is} \bigg[ D_\lambda \binom{P_\calA(Z_i | X_i = s_i)}{P_\calA(Z_i | X_i = s_j)} \bigg]
		+ D_\lambda \binom{P_{\calA, \calP}(\Zu | \Xs = s_i)}{P_{\calA, \calP}(\Zu | \Xs = s_j)} \notag
	\end{align}
\end{lemma} 
One interpretation of GI is that it assumes all locations $X_i$ are independent. In this case, the second term vanishes and the privacy loss only depends on randomness of the mechanism, not the prior. 

\paragraph{Robustness:}
\cite{no_free_lunch} show that it is impossible to achieve both utility and privacy resistant to all priors. CIP provides resistance to a reasonable class of priors $\calP \in \Theta$, but it is possible that the true distribution $\calQ \notin \Theta$. In this case, the privacy guarantees degrade gracefully as the divergence between $\calQ$ and $\calP \in \Theta$ grows. 
\begin{theorem}\emph{Robustness to Prior Misspecification}
\label{thm: prior misspecification}
	Mechanism $\calA$ satisfies $\varepsilon(\lambda)$-CIP for prior class $\Theta$. Suppose the finite mean true distribution $\calQ$ is not in $\Theta$. The CIP loss of $\calA$ against prior $\calQ$ is bounded by 
	\begin{align*}
		D_\lambda \binom{P_{\calA, \calQ}(Z | \Xs = s_i)}{P_{\calA, \calQ}(Z | \Xs = s_j)} \leq \varepsilon'(\lambda)
	\end{align*}
	where
	\begin{align*}
		\varepsilon'(\lambda) 
		&= \frac{\lambda - \frac{1}{2}}{\lambda - 1} \ \Delta(2\lambda) + 
		\Delta(4\lambda - 3) +
		\frac{2\lambda - \frac{3}{2}}{2\lambda - 2} \ \varepsilon(4 \lambda -2)
	\end{align*}
	and where $\Delta(\lambda)$ is
	\begin{align*}
		\inf_{\calP \in \Theta} \sup_{s_i \in \calS} \max \bigg\{ 
		D_\lambda \binom{P_{ \calP}(\Xu | \Xs = s_i)}{P_{ \calQ}(\Xu | \Xs = s_i)}, 
		D_\lambda \binom{P_{ \calQ}(\Xu | \Xs = s_i)}{P_{ \calP}(\Xu | \Xs = s_i)}
		\bigg\}
	\end{align*}
\end{theorem}

As long as the conditional distribution on $\Xu|\Xs = s_i$ of prior $\calQ$ is close to that of some $\calP \in \Theta$, the privacy guarantees should change only marginally. This bound is tightest when $\varepsilon(\lambda)$ does not grow quickly with order $\lambda$.



%
%\begin{center}
%\begin{figure*}
%\begin{minipage}[t]{0.26\textwidth}
%	\begin{algorithm}[H]
%		\SetAlgoLined
%		\KwInput{$\Is, \Sigma, o_t$}
%		\KwOutput{$\Sigmag$}
%			\vskip 2mm
%			$\argmax_{\Sigmag \succeq 0} \  \beta^*$\;
%			\vskip 1mm
%			$ \text{s.t. } \tilde{A}^{-1} \tilde{B}^{-1} \tilde{A}^{-\intercal} \succeq \beta^* \mathbf{I}$\;
%			$ \quad \ \ \ \trace(\Sigmag) \leq n o_t$\;
%			\vskip 2mm
%		\Return $\Sigmag$\;
%		\caption{SIG OPT}
%	\label{alg: sig opt}
%	\end{algorithm}
%\end{minipage}%
%\hspace{0.04\textwidth}
%\begin{minipage}[t]{0.3\textwidth}
%	\begin{algorithm}[H]
%		\SetAlgoLined
%		\KwInput{$\calF$}
%		\KwOutput{$\Sigmag$}
%			\vskip 2mm
%			$\argmin_{\Sigmag } \  \trace(\Sigmag)$\;
%			\vskip 1mm
%			$ \text{s.t. } \Sigmag \succeq \Sigmag_i , \ \forall \Sigmag_i \in \calF$\;
%			\vskip 2mm
%		\Return $\Sigmag$\;
%		\caption{ALL SECRETS}
%	\label{alg: all secrets}
%	\end{algorithm}
%\end{minipage}
%\hspace{0.04\textwidth}
%\begin{minipage}[t]{0.34\textwidth}
%	\begin{algorithm}[H]
%		\SetAlgoLined
%		\KwInput{$\mathbb{I}_{\calS_b}, \Sigma, o_t$}
%		\KwOutput{$\Sigmag$}
%		\vskip 2mm
%		$\calF = \emptyset$\;
%		
%		\For{$\mathbb{I}_{S_i} \in \mathbb{I}_{\calS_b}$}
%			{
%				$\Sigmag_i =$ SIG OPT$(\mathbb{I}_{S_i}, \Sigma, o_t)$\; 
%				
%				$\calF = \calF \cup \Sigmag_i$\;
%			}
%		\vskip 2mm
%			\Return ALL SECRETS$(\calF)$\;
%	
%	\caption{Basic Mechanism}
%	\label{alg: basic mechanism}
%	\end{algorithm}
%\end{minipage}%	
%\end{figure*}
%\end{center}
%

\subsection{CIP for Gaussian Process Priors}
\label{sec: CIP for GP} 
A \emph{GP conditional prior class} is the set of all GP prior distributions with the same kernel function $(i,j) \rightarrow \text{Cov}(X_i, X_j)$ and any mean function $i \rightarrow \E[X_i]$. With an additive Gaussian mechanism $G \sim \calN(\mathbf{0}, \Sigmag)$, the CIP loss of Equation \eqref{eqn:two terms} can be bounded for any GP conditional prior class. See Appendix \ref{apx: GP prior class} for further discussion of the GP conditional prior class. 

\begin{theorem}\emph{CIP loss bound for GP conditional priors:}
\label{thm:GP bound}
	Let $\Theta$ be a GP conditional prior class. Let $\Sigma$ be the covariance matrix for $X$ produced by its kernel function. Let $\calS$ be the basic or compound secret associated with $\Is$, and $S$ be the number of unique times in $\Is$. The mechanism $\calA(X) = X + G = Z$, where $G \sim \calN(\mathbf{0}, \Sigmag)$, then satisfies $(\varepsilon, \lambda)$-Conditional Inferential Privacy $(\Spairs, r, \Theta)$, where 
	\begin{align}
		\varepsilon
		&\leq \frac{\lambda}{2} S r^2 \Big(  \frac{1 }{\sigma_s^2} + \alpha^*  \Big) 
		\label{eqn: priv bound}
	\end{align}
	\text{ }\vspace{1mm}\\
	where $\sigma_s^2$ is the variance of each $G_i \in \Gs$ (diagonal entries of $\Sigmag_{ss}$) and $\alpha^*$ is the maximum eigenvalue of $\Sigmaeff = \big(\Sigma_{us} \Sigma_{ss}^{-1}\big)^\intercal \big( \Sigma_{u | s} + \Sigma_{uu}^{(g)} \big)^{-1} \big(\Sigma_{us} \Sigma_{ss}^{-1}\big)$. 
\end{theorem}

The above bound is tight for basic secrets ($S = 1$). The two terms of Equation \eqref{eqn: priv bound} represent the direct $(\frac{1}{\sigma_s^2})$ and inferential $(\alpha^*)$ loss terms of Equation \eqref{eqn:two terms}. We assume that each diagonal entry of $\Sigmag_{ss}$ equals some $\sigma_s^2$, so that each $X_i \in \Xs$ experiences identical direct privacy loss, which is optimal under utility constraints. 

The above bound composes gracefully when multiple traces of an individual are released. 

\begin{corollary}\emph{Graceful Composition in Time}
\label{cor: composition}
	Suppose a user releases two traces $X$ and $\hat{X}$ with additive noise $G \sim \calN(\mathbf{0}, \Sigmag)$ and $\hat{G} \sim \calN(\mathbf{0}, \hat{\Sigma}^{(g)})$, respectively. Then basic or compound secret $\Xs$ of $X$ enjoys $(\bar{\varepsilon}, \lambda)$-CIP, where 
	\begin{align*}
		\bar{\varepsilon} \leq \frac{\lambda}{2} S r^2 \Big(  \frac{1 }{\sigma_s^2} + \bar{\alpha}^*  \Big) 
	\end{align*}
	and where $\bar{\alpha}^*$ is the maximum eigenvalue of $\bar{\Sigma}_{\text{eff}} = \big(\Sigma_{us} \Sigma_{ss}^{-1}\big)^\intercal \big( \Sigma_{u | s} + \bar{\Sigma}_{uu}^{(g)} \big)^{-1} \big(\Sigma_{us} \Sigma_{ss}^{-1}\big)$. $\Sigma$ is the covariance matrix of the joint distribution on $X, \hat{X}$ and 
	\begin{align*}
	\bar{\Sigma}^{(g)} =
		\begin{bmatrix}
			 \Sigmag & 0 \\
			 0 &  \hat{\Sigma}^{(g)} \ .
		\end{bmatrix}
	\end{align*}
\end{corollary}
\text{ } \vspace{2mm} \\
This bound is identical to that of Theorem \ref{thm:GP bound}, only using the joint distribution over $X$, $\hat{X}$ and $G, \hat{G}$. This provides some insight to the fact that, unlike DP, even parallel composition guarantees are not automatic. Composition depends on the conditional prior. In the GP setting, if the chosen kernel function decays over time, we can expect composition to have minimal effects on privacy for traces separated by long durations. 

To reduce the upper bound of Theorem \ref{thm:GP bound}, we optimize the correlation (off-diagonal) of $\Sigmag$ to minimize $\alpha^*$, and optimize its variance (diagonal) to balance a noise budget between lowering inferential ($\alpha^*$) and direct ($\frac{1}{\sigma_s^2}$) loss.
\section{Optimized Privacy Mechanisms}
\label{sec: algorithms}

Theorem \ref{thm:GP bound} characterizes the privacy loss for GP conditional priors. We next show how to use this Theorem to design mechanisms that can strategically reduce CIP loss given a utility constraint. We measure `utility loss' as the total mean squared error (MSE) between the released ($Z$) and true ($X$) traces: $\text{MSE}(\Sigmag) = \sum_{i=1}^n \E[Z_i - X_i] = \trace(\Sigmag)$. We bound the utility loss by $\trace(\Sigmag) \leq n o_t$, where $o_t$ is the average per-point utility loss.

It can be shown that optimizing the privacy loss under this utility constraint can be described by a semidefinite program (SDP) (formalization/derivation of SDPs in Appendix \ref{apx: algorithmns}). For a given trace $X$, define its covariance matrix $\Sigma$ using the the kernel of the GP conditional prior $\Sigma_{ij} = k(i,j)$. Then pass $\Sigma$, the secret set $\Is$, and the utility constraint $o_t$ to our first program, $\text{SDP}_\text{A}$, which returns noise covariance $\Sigmag$. This defines an additive noise mechanism $G \sim \calN(0, \Sigmag)$ that minimizes CIP loss to $\Is$. 
\begin{align*}
	\Sigmag = \text{SDP}_\text{A}(\Sigma, \Is, o_t)
\end{align*} 
We can thus use a SDP to minimize the CIP loss to any single compound or basic secret. However, a trace may contain multiple locations or combinations thereof that one wishes to protect. It remains to produce a single mechanism $\Sigmag$ that bounds the CIP loss to multiple basic and/or compound secrets in a single trace. 

For this we propose $\text{SDP}_\text{B}$, which uses the fact that if ${\Sigmag}' \succ \Sigmag$ it will have lower CIP loss (see Appendix \ref{apx: SDP B}). $\text{SDP}_\text{B}$ takes in a set of covariance matrices $\calF = \{\Sigmag_1, \dots, \Sigmag_m\}$, each designed to minimize CIP loss for a single compound or basic secret $\Is_i$. It then returns a single covariance matrix $\Sigmag \succeq \Sigmag_i, i \in [m]$ that maintains the privacy guarantee each $\Sigmag_i$ offered its corresponding $\Is_i$, while minimizing utility loss. 

\begin{figure*}[h]
    \centering
    \begin{subfigure}{.24\linewidth}
        \centering 
        \captionsetup{justification=centering}
        \includegraphics[width = 1\linewidth]{./images/figure_2a.png}
        \caption{}
        \label{fig: RBF basic}
    \end{subfigure}
    \begin{subfigure}{.24\linewidth}
        \centering 
        \captionsetup{justification=centering}
        \includegraphics[width = 0.95\linewidth]{./images/figure_2b.png}
        \caption{}
        \label{fig: RBF compound}
    \end{subfigure}
    \begin{subfigure}{.24\linewidth}
        \centering 
        \captionsetup{justification=centering}
        \includegraphics[width = 0.95\linewidth]{./images/figure_2c.png}
        \caption{}
        \label{fig: RBF all}
    \end{subfigure}
    \begin{subfigure}{.24\linewidth}
        \centering 
        \captionsetup{justification=centering}
        \includegraphics[width = 0.95\linewidth]{./images/figure_2d.png}
        \caption{}
        \label{fig: RBF misspec}
    \end{subfigure}
    \begin{subfigure}{.24\linewidth}
        \centering 
        \captionsetup{justification=centering}
        \includegraphics[width = 1\linewidth]{./images/figure_2e.png}
        \caption{}
        \label{fig: PER basic}
    \end{subfigure}
    \begin{subfigure}{.24\linewidth}
        \centering 
        \captionsetup{justification=centering}
        \includegraphics[width = 0.95\linewidth]{./images/figure_2f.png}
        \caption{}
        \label{fig: PER compound}
    \end{subfigure}
    \begin{subfigure}{.24\linewidth}
        \centering 
        \captionsetup{justification=centering}
        \includegraphics[width = 0.95\linewidth]{./images/figure_2g.png}
        \caption{}
        \label{fig: PER all}
    \end{subfigure}
    \begin{subfigure}{.24\linewidth}
        \centering 
        \captionsetup{justification=centering}
        \includegraphics[width = 0.95\linewidth]{./images/figure_2h.png}
        \caption{}
        \label{fig: PER misspec}
    \end{subfigure}
    \caption{$^1$Posterior uncertainty interval (higher$=$better privacy) on $\Xs$ of a GP Bayesian adversary. A larger $\leff$ corresponds to greater inter-dependence and reduces posterior uncertainty. The gray interval depicts the middle 50\% of the MLE $\leff$ among traces in each dataset, and the black dotted line the median $\leff$. \textbf{(a)}$\rightarrow$\textbf{(c)}, \textbf{(e)}$\rightarrow$\textbf{(g)} show SDP mechanisms (blue) maintaining relatively high uncertainty compared to two GI (Approach C) baselines of equal utility (MSE). \textbf{(d)}, \textbf{(h)} show the (minor) change in posterior uncertainty when the prior covariance $\Sigma$  used in $\text{SDP}_{\text{A}}$ is misspecified: when it is identical to the true covariance $\Sigma^*$ known to the adversary (blue), is more correlated (orange), or is less correlated (green).
    }
    \label{fig: experiments}
\end{figure*} 

\SetKwInput{KwInput}{Input}                % Set the Input
\SetKwInput{KwOutput}{Output}              % set the Output

\begin{algorithm}
		\SetAlgoLined
		\KwInput{$\Is_1, \dots, \Is_m, o_t, \Sigma$}
		\KwOutput{$\Sigmag$}
			\vskip 1mm
			$\calF = \emptyset$\;
			
			\For{$i \in [m]$}
			{
				$\Sigmag_i =$ $\text{SDP}_\text{A}(\Sigma, \Is_i, o_t)$\; 
				
				$\calF = \calF \cup \Sigmag_i$\;
			}
			\vskip 1mm
			$\Sigmag= \text{SDP}_{\text{B}}(\calF)$\;
			
			\Return $\Sigmag$\;
		\caption{Multiple Secrets}
\label{alg: Multiple Secrets}
\end{algorithm}
In our experiments, we use Algorithm \ref{alg: Multiple Secrets} to design a single mechanism that protects all locations in the trace --- all basic secrets --- while minimizing utility loss.





\section{Experiments}
\label{sec: experiments}
%\textcolor{blue}{
Here, we aim to empirically answer: \textbf{1)} 
Do our SDP mechanisms maintain high posterior uncertainty of sensitive locations? How do they compare to Approach C baselines of equal MSE? 
%What privacy improvements do our SDP-based mechanisms offer over Approach C baselines of identical MSE? 
\textbf{2)} How robust is the $\text{SDP}_\text{A}$ mechanism when the prior covariance $\Sigma$ is misspecified? 
%}

\paragraph{Methods} To answer these questions, we look at the range of conditional prior classes that fit real-world data. For location trace data, we use the GeoLife GPS Trajectories dataset \citep{geolife} containing 10k human mobility traces after preprocessing (see Appendix \ref{apx: experiments} for details). We also consider the privacy risk of room temperature data \citep{home_monitoring}, using the SML2010 dataset \citep{sml2010}, which contains approximately 40 days of room temperature data sampled every 15 minutes. 

For the location data, having observed that the correlation between latitude and longitude is low ($ \approx 0.06$) we treat each dimension as independent. By way of Corollary \ref{cor: independence}, this allows us to bound privacy loss and design mechanisms for each dimension separately. Furthermore, having observed that each dimension fits nearly the same conditional prior, we treat our dataset of 10k 2-dimensional traces as a dataset of 20k 1-dimensional traces, where each trace represents one dimension of a 2d location trajectory.

We model the location trace data with a Radial Basis Function (RBF) kernel GP and the temperature series data with a periodic kernel GP:
\begin{align*}
	k_{\text{RBF}}(t_i, t_j) 
	&=  \sigma_x^2 \exp \Big( -\frac{(t_i - t_j)^2}{2 l^2} \Big) \\
	k_{\text{PER}}(t_i, t_j) 
	&=  \sigma_x^2 \exp \Big(  \frac{-2 \sin^2(\pi |t_i - t_j| / p)}{l^2} \Big)
\end{align*}
In both kernels, the intrinsic degree of dependence between points is captured by the lengthscale $l$. However, the fact that sampling rates vary significantly between traces means that traces with equal length scales can have very different degrees of correlation. To encapsulate both of these effects, we study the empirical distribution of \emph{effective} length scale of each trace
\begin{align*}
	l_{\text{eff},x} = \frac{l_x}{P}
	\quad
	l_{\text{eff},y} = \frac{l_y}{P}
\end{align*}
where $P$ is the trace's sampling period and $l_x,l_y$ are the its optimal length scales for each dimension. 

$l_{\text{eff},x},l_{\text{eff},y}$ tell us the average number of neighboring locations that are highly correlated, instead of time period. For instance, a given trace with an optimal $l_{\text{eff},x} = 8$ tells us that every eight neighboring location samples in the $x$ dimension have correlation $> 0.8$. The empirical distribution of effective length scales across all traces describes -- over a range of logging devices (sampling rates), users, and movement patterns -- how many neighboring points are highly correlated in location trace data. After this preprocessing, we are able to use the kernels that take indices (not time) as arguments: 
\begin{align*}
	k_{\text{RBF}}(i, j) 
	&=  \exp \Big( -\frac{(i - j)^2}{2\leff^2} \Big) \\
	k_{\text{PER}}(i, j) 
	&=  \exp \Big(  \frac{-2 \sin^2(\pi |i - j| / p)}{\leff^2} \Big)
\end{align*}
See Appendix \ref{apx: experiments} for a more detailed discussion of how the empirical distribution of $\leff$ across traces is measured. 

To impart the range of realistic conditional priors the gray interval of each plot depicts the middle 50\% of the empirical $\leff$ among traces in each dataset. The dashed vertical line reports the median $\leff$. 



%Details of preprocessing and definition of $\leff$ are presented in Appendix \ref{apx: experiments}. 

%\begin{align}
%	\label{eqn: kernels}
%	k_{\text{RBF}}(i, j) 
%	&=  \exp \Big( -\frac{(i - j)^2}{2\leff^2} \Big), \notag \\
%	k_{\text{PER}}(i, j) 
%	&=  \exp \Big(  \frac{-2 \sin^2(\pi |i - j| / p)}{\leff^2} \Big)
%\end{align}
Each figure increases the degree of dependence, $\leff$, used by the kernel to compute the prior covariance $\Sigma(\leff)$. $\Sigma(\leff)$ is then used in one of the SDP routines of Section \ref{sec: algorithms} to produce a mechanism $\Sigmag(\leff)$ that protects a basic secret ($\text{SDP}_\text{A}$), a compound secret ($\text{SDP}_\text{A}$), or the union of all basic secrets (Multiple Secrets). We then observe the 68\% confidence interval of the Gaussian posterior on sensitive points $\Xs$ (blue line). This is the $2\sigma$ uncertainty of a Bayesian adversary with a GP prior represented by $\Sigma(\leff)$ (see Appendix \ref{apx: experiments} for how this is computed). As $\leff$ increases, their posterior uncertainty will reduce. Our aim is to mitigate this as much as possible with the given utility constraint. For scale, recall that prior variance $\textbf{diag}(\Sigma)$ is normalized to one. In the case of all basic secrets, we report the average posterior uncertainty over locations. 

We compare the SDP mechanisms with two mechanisms using the logic of Approach C (all three of equal MSE utility loss): \emph{independent/uniform} and \emph{independent/concentrated}. The uniform approach adds independent Gaussian noise evenly along the whole trace regardless of $\Is$, $\Sigmag = o_tI$. The concentrated approach allocates the entire noise budget to the sensitive set $\Is$. 
\paragraph{Results}
For our first question, see \textbf{Figures \ref{fig: RBF basic}$\rightarrow$\ref{fig: RBF all}, \ref{fig: PER basic}$\rightarrow$\ref{fig: PER all}}. For both location and temperature data, our SDP mechanisms maintain higher posterior uncertainty than the baselines with identical utility cost for a single basic secret, a compound secret, and all basic secrets. By actively considering the conditional prior class parametrized by $\Sigma$, the SDP mechanisms can strategize to both correlate noise samples and concentrate noise power such that posterior inference is thwarted at the sensitive set $\Is$. For an intuitive illustration of the chosen $\Sigmag$'s, see Appendix \ref{apx: juxtaposition}. 

To answer our second question, see \textbf{Figures \ref{fig: RBF misspec}} and \textbf{\ref{fig: PER misspec}}. When the prior covariance $\Sigma$ does not represent the true data distribution known to the adversary, a smaller posterior uncertainty may be achieved. The orange line indicates the uncertainty interval of an adversary who knows the data is \emph{less} correlated than we believe i.e. the true $\Sigma^* = \Sigma(0.5 \leff)$. The blue line represents an adversary who knows the data is \emph{more} correlated than we believe i.e. the true $\Sigma^* = \Sigma(1.5 \leff)$. Both plots confirm the robustness of our privacy guarantees stated by Theorem \ref{thm: prior misspecification}. Particularly around the median $\leff$ we see that the change in posterior uncertainty with this change in prior is indeed marginal. 

\section{Discussion}
\paragraph{Related Work}
Few works have proposed solutions to the \emph{local} guarantee when releasing individual traces. A mechanism offered in \cite{synthesizing_plausible_deniability} releases synthesized traces satisfying the notion of \emph{plausible deniability} \citep{plausible_deniability}, but this is distinctly different from providing a radius of privacy to sensitive locations. Meanwhile, the frameworks proposed in \cite{temporal} and \cite{priste} nicely characterize the risk of inference in location traces, but use only first-order Markov models of correlation between points, do not offer a radius of indistinguishability as in this work, and are not suited to continuous-valued spatiotemporal traces.

Perhaps more technically similar to this work, \cite{song_pufferfish_2017} provide a general mechanism that applies to any Pufferfish framework, as well as a more computationally efficient mechanism that applies when the joint distribution of an individual's features can be described by a graphical model. The first is too computationally intensive. The second is for discrete settings, and cannot accommodate spatiotemporal effects.
%The first mechanism is too computationally intensive for our setting, and the second only looks at discrete or categorical functions of data, and cannot (at least directly) accommodate spatiotemporal effects. 

\paragraph{Conclusion}
This work proposes a framework for both identifying and quantifying the \emph{inferential} privacy risk for highly dependent sequences of spatiotemporal data. As a starting point, we have provided a simple bound on the privacy loss for Gaussian process priors, and an SDP-based privacy mechanism for minimizing this bound without destroying utility. We hope to extend this work to other data domains with different conditional priors, and different sets of secrets.

\subsubsection*{Acknowledgements}
KC and CM would like to thank ONR under N00014-20-1-2334 and UC Lab Fees under LFR 18-548554  for research support. We would also like to thank our reviewers for their insightful feedback. 
\chapter*{Concluding Remarks}

This dissertation works towards addressing two specific issues in reliable machine learning, namely adversarial examples and data-copying. For adversarial-examples, we studied non-parametric classification along with linear classifiers and saw that in both cases, the robust setting leads to significant differences from the standard learning setting. For data-copying, we produced precise definition of what data-copying is along with an algorithm for detecting it. 

These two problems are two examples among many other issues (i.e. privacy, fairness) in reliable machine learning. This reflects that fact that word ``reliable" can many different things, some of which undoubtedly haven't been discovered. It is our belief that this necessitates careful conceptual work where the various sides of "reliable" and "trustworthy" are disentangled from each other into rigorous and measurable concepts, and we believe that our work provides progress towards doing this. 



%The above chapters provide a diverse set of examples of how privacy risks can be measured or mitigated in different scenarios. While highly different from each other, these examples all highlight the following three guiding principles for data privacy.
%
%\subsection*{No privacy definition is a `gold standard'}
%Differential privacy (DP) is often touted as the `gold standard' of data privacy. The cases studied in Chapters 3-5 challenge this by proposing entirely different privacy definitions for entirely different settings and risks. Chapter 3 proposes sentence privacy, which is DP-like but uses a different neighboring notion. Chapter 4, on the other hand, proposes a shuffling-based privacy definition that is almost orthogonal to DP. That is because the correlation adversary considered in that setting cannot be thwarted by DP alone. However, the broad population trends that we wish to learn are still accessible under our semi-random shuffling approach. Similarly, Chapter 5 analyzes the threat of correlation adversaries in the domain of location traces. Here, we see that a DP-based definition has to add \emph{more noise} in order to thwart these adversaries.
%
%Taken together, we see that sometimes DP definitions are effective, and other times they require one to choose between meaningful privacy and utility. If one `gold standard' definition were effective in all of these settings, we would not need to propose so many contrasting privacy definitions and methods. 
%
%\subsection*{No Free Lunch}
%The goal of data privacy is to allow the release of high-level information (\emph{e.g.} data distribution) while obscuring low-level information (\emph{e.g.} individuals' data features). It is natural to wonder whether it is possible to design a privacy definition under which we can release highly level information and defend against \emph{any} adversary. The answer is unequivocally, \emph{no}. This fact is known as the No-Free-Lunch theorem, made precise in \cite{Kifer}. The Theorem shows that releasing any information about a dataset that is useful to one person can be leveraged by an adversary to learn fine-grained information. The No-Free-Lunch theorem is instructive, because it shifts our attention from the question of whether we can provide air-tight privacy (impossible) to whether the adversaries our definition allows are \emph{realistic} in our setting. 
%
%The No-Free-Lunch principle is fundamental to the approaches of Chapters 4 and 5 in particular. Here, we propose novel privacy definitions that are adversary-focused. Note that in both of these papers, we consider limited classes of adversaries. As stated above, it is impossible to block the inferences of \emph{all adversaries} while still sharing useful information derived from the sensitive data. By practically evaluating what prior knowledge an adversary might have, like a correlation prior, we can formalize a privacy definition that gives strong guarantees in realistic settings. 
%
%\subsection*{Perfect is the enemy of good}
%Chapters 1 and 2 offer no formal privacy definition or provably private mechanism. Instead, they offer statistical tests to empirically evaluate a model's memorization of its training data, and thereby risk of exposing that data. In both chapters, we examine how model selection can effect the degree of memorization as detected by our tests. While our proposed test statistics do not confer any formal privacy guarantees, they guide practitioners towards models that memorize less. In many cases, our tests showed that it is possible to find models which have significantly less memorization at little to no cost in utility. 
%
%While formal privacy definitions are a valuable goal they make up only a small part of an ML practitioners privacy toolkit. To preserve privacy, we as researchers ought to put equal effort into methodical empirical privacy tests as we do formally private algorithms. These tests tend to be far more accessible to practitioners and allow them to significantly improve model privacy. Although empirical privacy tests are imperfect, the practical benefits to be gained by proposing them are undoubtedly a positive good. Do not let perfect privacy be the enemy of good privacy. 


\appendix            
%\Blinddocument
\graphicspath{{./chapters/chapter1/}}
\chapter{Appendix for Chapter 1}
\section{Proofs for $r$-separated distributions}

For any distribution $\D$ over $\X \times Y$, it will be convenient to use the following notation: for any measurable $S \subset \X$, let $\P_\D[S] = \P_{(x,y) \sim \D}[x \in S]$. The following definition will be central to our proofs. 

\begin{defn}
Let $\D$ be a distribution over $\X \times Y$. An \textbf{$(\epsilon, \gamma, \alpha)$-decomposition} of $\D$ is a finite set of closed balls $B_1, B_2, \dots, B_s \subset \X$ each with radius $\gamma$ such that $$\P_\D[\cup_1^s B_i] > 1 - \epsilon,$$ and such that $\P_\D[B_i] \geq \alpha > 0$ for $1 \leq i \leq s$. 
\end{defn}


\begin{lem}\label{lem_balls}
Let $\X$ be a totally bounded metric space. For any distribution $\D$, and $\epsilon, \gamma > 0$, there exists $\alpha > 0$ such that $\D$ admits a $(\epsilon, \gamma, \alpha)$-decomposition. 
\end{lem}

\begin{proof}
Fix any $x \in \X$ and $\epsilon, \gamma > 0$. Then the sequence of balls $\{S_i = B(x, i)\}$ has union equal to $\X$. Therefore, there exists $j$ such that $P_\D(S_j) > 1 - \epsilon$. Since $S_j$ is totally bounded and complete, it is compact. Let $B^o(x, a)$ denote the open ball centered at $x$ with radius $a$. Therefore, taking an open cover of $S_j$, $\{B^o(x, \gamma): x \in S_j\}$, we can take a finite subcover $\{B_1^o, B_2,^o, \dots, B_t^o\}$ that cover $S_j$. Discarding balls such that $\P_\D(B_i^o) = 0$ and taking the closure of each ball gives the desired result, with $\alpha = \min_{i}P_\D(B_i)$.  
\end{proof}

To prove Theorem \ref{thm_stone_cons}, we use the following lemma. 

\begin{lem}\label{lem_expectation}
Let $\D$ be a distribution over $\X \times \Y$, and let $B_1, B_2, \dots, B_s$ be a $(\epsilon, \gamma, \alpha)$-decomposition of $\D$, and let $r > 3\gamma$. If $W$ is a weight function satisfying the conditions of Theorem \ref{thm_stone_cons}, then for any $\delta > 0$ there exists $N$ such that for $n \geq N$, with probability $1-\delta$ over $S \sim \D^n$, and $w_1, w_2, \dots, w_n$ learned by $W$ from $S$, $$\sup_{\{x: d(x, \cup_1^s B_i) \leq r - 3\gamma\}} \sum_1^n w_i(x)I_{d(x_i, x) > r} < \frac{1}{3}.$$
\end{lem}

\begin{proof} 
Fix $\delta > 0$, and let $Y$ be the indicator variable defined as $$Y = \begin{cases} 1 & \text{ if }\sup_{\{x: d(x, \cup_1^s B_i) \leq r - 3\gamma\}} \sum_1^n w_i(x)I_{d(x_i, x) > r} \geq \frac{1}{3} \\ 0 & \text{ if }\sup_{\{x: d(x, \cup_1^s B_i) \leq r - 3\gamma\}} \sum_1^n w_i(x)I_{d(x_i, x)> r} < \frac{1}{3} \end{cases}.$$ It suffices to show that there exists $N$ such that for all $n \geq N$, $E_{S \sim \D}[Y] \leq \delta$. 

Fix $S \sim \D^n$ and suppose that $Y = 1$. Then there exists $x^*, B_i^*$ such that $d(x^*, B_i^*) \leq r - 3\gamma$ and such that $$\sum_1^n w_i(x^*)I_{d(x_i, x^*) > r} \geq \frac{1}{3}.$$ By definition, $B_i$ has radius $\gamma$, so by the triangle inequality, for any $x \in B_i^*$, $d(x, x^*) \leq 2\gamma + r - 3\gamma = r - \gamma$. This implies $x^* \in B(x, r-\gamma)$. Therefore, for any $x \in B_i^*$, $$\sup_{x' \in B(x, r-\gamma)} \sum_1^n w_i(x')I_{d(x', x_i) > r} \geq \sum_1^n w_i(x^*)I_{d(x^*, x_i) > r} \geq \frac{1}{3}.$$ By the definition of an $(\epsilon, \gamma, \alpha)$-decomposition, we have that $P_\D(B_i^*) \geq \alpha$. As a consequence, we have that $$\mathbb{E}_{X \sim \D_\X} \big [ \sup_{x' \in B(X, r-\gamma)} \sum_1^n w_i(x')I_{||x_i - x'|| > r} \big] \geq P_\D[B_i^*]\frac{1}{3} \geq \frac{\alpha}{3}.$$ Since the previous inequality is guaranteed to hold if $Y = 1$, taking the expectation over $S$ yields that $$\mathbb{E}_{S \sim \D^n} \mathbb{E}_{X \sim \D_\X} \big [ \sup_{x' \in B(X, r-\gamma)} \sum_1^n w_i(x')I_{||x_i - x'|| > r} \big] \geq \frac{\alpha E[Y]}{3}.$$ By the conditions of Theorem \ref{thm_stone_cons}, the left side of the equation must tend to $0$ as $n \to \infty$. This implies that the same must hold for the right side. Therefore, $E[Y]$ tends to $0$ as $n \to \infty$, and we can select $N$ such that $E[Y] < \delta$ for $n \geq n$, which completes the proof. 
\end{proof}

\begin{proof} (\textbf{Theorem \ref{thm_stone_cons}})
Let $W$ be a weight function that satisfies the condition of Theorem \ref{thm_stone_cons}. Fix $\epsilon, \delta > 0$, and $\gamma < r/3$. Applying Lemma \ref{lem_balls}, let $B_1, B_2, \dots, B_s$ be an $(\epsilon, \gamma, \alpha)$-decomposition of $\D$. Let $T^+$ and $T^-$ be subsets of $\X$ corresponding to the definition of $r$-separation for $\D$.  

For $S \sim \D^n$, let $A$ denote the event that $$\sup_{\{x: d(x, \cup_1^s B_i) \leq r - 3\gamma\}} \sum_1^n w_i(x)I_{d(x_i, x) > r} < \frac{1}{3}.$$
Suppose $A$ holds. Pick a $B_i$. Since $T^+$ and $T^-$ have distance greater than $2r$, and $diam(B_i) \leq 2\gamma < r$, either $B_i \cap T^+ = \emptyset$ or $B_i \cap T^- = \emptyset$. Note that for $n$ sufficiently large, both cannot be empty since $P_\D(B_i) \geq \alpha > 0$ and each $x$ in the support of $\D$ is either in $T^+$ or $T^-$. 

Without loss of generality, $B_i \cap T^- = \emptyset$. Then $B_i \cap T^+ \neq \emptyset$. $B_i$ has diameter $2\gamma$. Thus $d(B_i, T^-) > 2r - 2\gamma$. Let $x \in B(B_i, r-3\gamma)$. Then if $(x_j, -) \in S$, by the triangle inequality, $d(x, x_j) > 2r -2\gamma - (r - 3\gamma) = r+\gamma$. 

Substituting this and using event $A$, we have that $$\sum_1^n w_i^S(x)I_{(x_i, -) \in S} \leq \sum_1^n w_i^S(x)I_{d(x_i, x) > r} < \frac{1}{3}.$$ It follows that $W_S(x) = +1$. An analogous argument holds for $B_i \cap T^+ = \emptyset$. This implies that $W_S$ is astute with radius $r-3\gamma$ over all $B_i$.

$\cup B_i$ has measure at least $1-\epsilon$. By Lemma \ref{lem_expectation}, for any $\delta>0$ event $A$ holds with probability $1-\delta$ for $n$ sufficiently large. Therefore, for $n$ sufficiently large, we see that $A_{r-3\gamma}(W_S, \D) \geq 1-\epsilon$ with probabiltiy $1-\delta$. Because $\epsilon, \delta$ and $\gamma$ were arbitrary, it follows that $W$ is \rcons,\emph{ }as desired.

\end{proof}

\begin{proof} (\textbf{Corollary \ref{nn_sep_thm}})
For any $S = \{(x_1, y_1), (x_2, y_2), \dots, (x_n, y_n)\} \subset \X \times \Y$, let $w_i^S(x)$ be $1$ if and only if $x_i$ is one of the $k_n$ nearest neighbors of $x$ in the set $S_\X = \{x_1, x_2, \dots x_n\}$. Let $\D$ be a distribution over $\X \times \Y$. By Theorem \ref{thm_stone_cons}, it suffices to show that for any $0 < a < b$, $$\lim_{n \to \infty} \mathbb{E}_{X \sim \D_\X}[\mathbb{E}_{S \sim \D^n} [\sup_{x' \in B(x,a)} \sum_1^n w_i^S(x')I_{d(x_i, x') > b}]] = 0.$$ Fix $0 < a < b$, and let $\epsilon > 0$. 

Pick $\gamma > 0$ such that $a+ 2\gamma < b$. This is possible for any $a < b$. Let $B_1, B_2, \dots, B_s$ be an $(\epsilon, \gamma, \alpha)$-decomposition of $\D$. By applying a Chernoff bound followed by a union bound, for any $\delta > 0$ there exists $n$ such that with probability $1-\delta$ over $S \sim \D^n$, each $B_i$ satisfies $|B_i \cap S_\X| \geq \frac{n\alpha}{2}$. Furthermore, if $n$ is sufficiently large, then $\frac{n\alpha}{2} > k_n$ holds as well. 

Consider any $x \in B_i$, and$x' \in B(x,a)$. $B_i$ has radius $\gamma$ and also satisfies $|B_i \cap S_\X| > k_n$. Therefore, there are at least $k_n$ points within distance $a+2\gamma$ of $x$. Because $a + 2\gamma < b$, it follows that none of the $k_n$ nearest neighbors of $x'$ can have distance more than $b$ from $x'$. In particular, $$\sum_1^n w_i^S(x')I_{d(x_i, x') > b} = 0.$$ Since $B_i$, $x$ and $x'$ were arbitrary, we have that for all $x \in \cup B_i$, $$\sup_{x' \in B(x,a)} \sum_1^n w_i^S(x')I_{d(x_i, x') > b} \leq  \begin{cases} 0 & |B_i \cap S_\X| \geq \frac{n\alpha}{2}, 1 \leq i \leq s \\ 1 & \text{otherwise} \end{cases}$$

Since $X \in \cup_1^s B_i$ with probability at least $1-\epsilon$, and since $|B_i \cap S_\X| \geq \frac{n\alpha}{2}, 1 \leq i \leq s$ with probability at least $1-\delta$, it follows that $$\mathbb{E}_{X \sim \D}[\mathbb{E}_{S \sim \D^n} [\sup_{x' \in B(x,a)} \sum_1^n w_i^S(x')I_{d(x_i, x') > b}]] \leq (1- \delta - \epsilon)0 + \delta + \epsilon = \delta + \epsilon,$$ which can be made arbitrarily small as $\epsilon$ and $\delta$ were arbitrary. Therefore, the limit as $n$ approaches infinity is $0$, as desired.
\end{proof}

\begin{proof} (\textbf{Corollary \ref{thm_kernel}})
Let $\D$ be a distribution over $\X \times \Y$. By Theorem \ref{thm_stone_cons}, it suffices to show that for any $0 < a < b$, $$\lim_{n \to \infty} \mathbb{E}_{X \sim \D}[\mathbb{E}_{S \sim \D^n} [\sup_{x' \in B(x,a)} \sum_1^n w_i^S(x')I_{d(x_i, x') > b}]] = 0.$$ Fix $0 < a < b$, and let $\epsilon > 0$. 

Pick $\gamma > 0$ be such that $a+ 2\gamma < b$. Let $B_1, B_2, \dots, B_s$ be an $(\epsilon, \gamma, \alpha)$-decomposition of $\D$. By applying a Chernoff bound, for any $\delta > 0$ there exists $n$ such that with probability $1-\delta$ over $S \sim \D^n$, each $B_i$ satisfies $|B_i \cap S_\X| \geq \frac{n\alpha}{2}$.

Next, consider any $x_i, x_j \in S_\X$, and let $x$ be a point such that $d(x_i, x) \leq a+2\gamma$ and $d(x_j, x) > b$. Then we have that 
\begin{equation*}
\begin{split}
\frac{w_j^S(x)}{w_i^S(x)} = \frac{K(\frac{d(x_j, x)}{h_n})}{K(\frac{d(x_i, x)}{h_n})}.
\end{split}
\end{equation*}
Because $b > a + 2\gamma$, $\frac{d(x_j, x)}{d(x_i, x)} > 1$. Therefore, since $\lim_{n \to \infty} h_n = 0$ and $\lim_{x \to \infty} \frac{K(cx)}{K(x)} = 0$ for $c > 1$, it follows that for any $\beta > 0$, there exists $N$ such that for $n \geq N$, $$\frac{w_j^S(x)}{w_i^S(x)} \leq \frac{\alpha\beta}{2}.$$ 

Fix any such $\beta$, and consider any $x$ with $d(x, B_i) \leq a$. Then $d(x, x') \leq a+ 2\gamma < b$ for any $x' \in B_i$. Recall that $B_i$ contains at least $\frac{n\alpha}{2}$ points, and let $c = \min_{i, d(x_i, x) \leq a + 2\gamma} w_i(x)$. Then it follows that 
\begin{equation*}
\begin{split}
\sum_1^n w_i^S(x)I_{d(x_i, x) > b} &\stackrel{(a)}{=} \frac{\sum_1^n w_i^S(x)I_{d(x_i, x) > b}}{\sum_1^n w_i^S(x)} \\
&\stackrel{(b)}{\leq} \frac{\sum_1^n w_i^S(x)I_{d(x_i, x) > b}}{\sum_1^n w_i^S(x)I_{d(x_i, x) \leq a+2\gamma}} \\
&\stackrel{(c)}{\leq} \frac{nc\frac{\alpha\beta}{2}}{\frac{n\alpha}{2}c} \\
&= \beta
\end{split}
\end{equation*} $(a)$ holds because the weights always sum to $1$. $(b)$ holds because we are reducing the denominator. $(c)$ holds because there are at least $\frac{n\alpha}{2}$ points in $B_i$, with $c$ being the minimum weight (stated above). The numerator is a result of the inequality shown above in which $w_j^S(x)/w_i^S(x) \leq \alpha\beta/2$ if $d(x_j, x) > b$ and $d(x_i, x) \leq a+2\gamma$.

Using this, we get the following bound: $$\sup_{x' \in B(X,a)} \sum_1^n w_i^S(x')I_{d(x_i, x') > b} \leq  \begin{cases} \beta & x \in \cup_1^s B_i, |B_i \cap S_\X| \geq \frac{n\alpha}{2}, 1 \leq i \leq s \\ 1 & \text{otherwise} \end{cases}$$ 

Since $x \in \cup_1^s B_i$ with probability $1-\epsilon$, and since $|B_i \cap S_\X| \geq \frac{n\alpha}{2}, 1 \leq i \leq s$ with probability $1-\delta$, it follows that $$\mathbb{E}_{X \sim \D}[\mathbb{E}_{S \sim \D^n} [\sup_{x' \in B(x,a)} \sum_1^n w_i^S(x')I_{d(x_i, x') > b}]] \leq (1- \delta - \epsilon)\beta + \delta + \epsilon.$$ which can be made arbitrarily small as $\epsilon, \beta,$ and $\delta$ were arbitrary. Therefore, the limit as $n$ approaches infinity is $0$, as desired. 
\end{proof}

\section{Proofs for general distributions}

\begin{lem}\label{chernoff_max_lem}
Let $B_1, \dots, B_s$ be a $(\epsilon, \alpha, \gamma)$ decomposition of $\D$ over $\X \times \Y$. Let $U \subseteq [s]$. Then if $n \geq O(\frac{s2^{2s}\log(1/\delta)}{\epsilon^2})$, then with probability at least $1-\delta$, for all $U$ we have: $$|\P_{(x,y) \sim \D}[x \in \cup_{i \in U} B_i, y = +] - \P_{(x,y) \sim \D_S}[x \in \cup_{i \in U} B_i, y = +]| \leq \epsilon,$$$$|\P_{(x,y) \sim \D}[x \in \cup_{i \in U} B_i, y = -] - \P_{(x,y) \sim \D_S}[x \in \cup_{i \in U} B_i, y = -]| \leq \epsilon.$$ 
\end{lem}
\begin{proof}
For any given $U \subseteq [s]$, by a Chernoff bound we have that $$|\P_{(x,y) \sim \D}[x \in \cup_{i \in U} B_i, y = +] - \P_{(x,y) \sim \D_S}[x \in \cup_{i \in U} B_i, y = +]| > \epsilon$$ with probability at most $\frac{\delta}{2^{s+1}}$. Taking a union bound over all $U$, we see that with probability $1-\frac{\delta}{2}$, $$|\P_{(x,y) \sim \D}[x \in \cup_{i \in U} B_i, y = +] - \P_{(x,y) \sim \D_S}[x \in \cup_{i \in U} B_i, y = +]| \leq \epsilon$$ for all $U \subseteq [m]$. Applying the same to $y = -1$ and taking a union bound implies the result.
\end{proof}

\begin{lem}\label{gen_thm}
Let $M$ be a classification algorithm over $\X \times \Y$, $r >0$ be a radius, and $\D$ be a distribution over $\X \times \Y$. Then for any $\epsilon, \delta$ over $(0,1)$, and for all $\gamma$ over $(0, r/2)$, there exists $N$ such that for $n \geq N$, with probability $1-\delta$ over $S \sim \D^n$, $$A_{r - \gamma}(M_S, \D) \geq A_r(M_S, \D_S) - \epsilon,$$ where $\D_S$ denotes the uniform distribution over $S$.  
\end{lem}

\begin{proof} (\textbf{Lemma \ref{gen_thm}})
Fix $\epsilon, \delta > 0$ and $\gamma < r/2$. Applying Lemma \ref{lem_balls}, let $B_1, \dots, B_s$ be a $(\epsilon, \alpha, \gamma)$ decomposition of $\D$. 

Let $T$ be the subset of $S$ such that $M_S$ is astute at $T$ with radius $r$. Define: $$I_T^+ = \{i| (x_j, +) \in T, x_j \in B_i\}$$$$I_T^- = \{i| (x_j, -) \in T, x_j \in B_i\}.$$

Observe that $I_T^+ \cap I_T^- = \emptyset$. To see this, notice that $B_i$ has radius $\gamma < r/2$. This implies that any $(x_j, +), (x_k, -) \in B_i$ would force $M_S$ to not be astute at either of those points. Thus we an think of $I_T^+$ being the set of positively labeled balls, and $I_T^{-}$ being the set of negatively labeled balls.

Let $B^+ = \cup_{i \in I_T^+} B_i$ and $B^- = \cup_{i \in I_T^-} B_i$. Our strategy will be to argue that $M_S$ must be robust with radius $r-2\gamma$ at $B^+ \cup B^-$, and then to observe that $\P_\D[(B^+, +)] + \P_\D[(B^-, -)]$ must be close to $A_r(M_S, \D_S)$. 

Let $T_\X \subset \X$ denote the set of all $x_i$ such that $(x_i, y_i) \in T$. By the definitions of $\D_S$ and $T$, we have that 
\begin{equation*}
\begin{split}
A_r(M_S, \D_S) &= \frac{|T|}{n} \\
&= \frac{|T_\X \cap B^+|}{n} + \frac{|T_\X \cap B^-|}{n} + \frac{|T_\X \setminus (B^+ \cup B^-)|}{n}.
\end{split}
\end{equation*}

If $x_i \in \cup_1^s B_j$ and $x_i \in T_\X$, then by definition, $x \in (B^+ \cup B^-)$. Therefore, $T_\X \setminus (B^+ \cup B^-)$ consists of $x_i \notin \cup_1^s B_j$. Using this, we see that 
\begin{equation*}
\begin{split}
A_r(M_S, \D_S) &= \frac{|T_\X \cap B^+|}{n} + \frac{|T_\X \cap B^-|}{n} + \frac{|T_\X \setminus (B^+ \cup B^-)|}{n} \\
&\leq \P_{(x,y) \sim \D_S}[x \in B^+, y=+]+\P_{(x,y) \sim \D_S}[x \in B^-, y=-] + \P_{(x,y) \sim \D_S}[x \notin \cup_1^s B_j].
\end{split}
\end{equation*}

If $n$ is sufficiently large, then by Lemma \ref{chernoff_max_lem}, each term on the right is within $\epsilon$ of its corresponding probability over $\D$. Thus we see that with probability $1-\delta$, 
\begin{equation}\label{astute_bound_eqn}
A_r(M_S, \D_S) \leq \P_{(x,y) \sim \D}[x \in \cup_{i \in I_T^+}B_i, y=+]+\P_{(x,y) \sim \D}[x \in \cup_{i \in I_T^-}, y=-] + 4\epsilon.
\end{equation} 

Observe that if $M_S$ is robust with radius $r$ at $x_j \in B_i$, then it is robust with radius $r-2\gamma$ at all $x \in B_i$. Furthermore, for $x_j \in \cup_{i \in I_T^+}B_i$, $M_S$ is astute at $(x_j, +1)$ with radius $r$. Therefore $M_S(x) = +1$ for all $x \in \cup_{i \in I_T^+}B_i$. Consequently, 
\begin{equation*}
\begin{split}
A_{r-2\gamma}(M_S, \D) &\geq \P_{(x,y) \sim \D}[x \in \cup_{i \in I_T^+}B_i, y=+]+\P_{(x,y) \sim \D}[x \in \cup_{i \in I_T^-}B_i, y=-] \\
&\geq A_r(M_S, \D_S) - 4\epsilon \text{ }(\text{by equation }\ref{astute_bound_eqn}).
\end{split}
\end{equation*}
Since this equation holds with probability $1 - \delta$, and since $\epsilon$ and $\gamma$ were arbitrary, the result follows. 
\end{proof}

\begin{proof}(\textbf{Theorem \ref{thm_weight_general}})
For convenience, we let $W'$ represent the weight function described by $\ga(S,W, r)$. In particular, $W'_S$ and $W_{S_r}$ are the same classifier, where $S_r$ denotes the largest $r$-separated subset of $S$.

Fix $\epsilon, \delta >0$, and let $0 < \gamma < r$. For convenience, let $$Z_ i = \sup_{x \in B(x_i, r-\gamma)} \sum_{j=1}^m w_j^{S_r}(x)I_{||x_j - x|| > r}.$$ Because $W$ fulfills the conditions of Theorem \ref{thm_weight_general}, there exists $N$ such that for $n > N$, with probability $1-\delta$ over $S \sim \D^n$, $ \frac{1}{m} \sum_{i = 1}^m Z_i < \epsilon.$ Therefore, there exist at most $3m\epsilon$ values of $i$ for which $Z_i > \frac{1}{3}$. 

Since $S_r$ is $r$-separated, it follows that $$\sup_{x \in B(x_i, r-\gamma)} \sum_1^m w_j^{S_r}(x)I_{y_j \neq y_i} \leq Z_i.$$ Consequently, if $Z_i \leq \frac{1}{3}$, then $W_{S_r}(x) = y_i$ for all $x \in B(x_i, r-\gamma)$. Let $\D_S$ denote the uniform distribution over $S$. Then we have that $$A_{r-\gamma}(W'_S, \D_S) = A_{r-\gamma}(W_{S_r}, \D_S) \geq \frac{|S_r|}{n} - 3\epsilon.$$ 
Observe that for $n$ sufficiently large, with probability $1-\delta$, $|A_r(\b_r, \D) - A_r(\b_r, \D_S)| \leq \epsilon$. The maximum possible astuteness over $\D_S$ is $\frac{|S_r|}{n}$ since no classifier can be astute at 2 oppositely labeled points with distance at most $2r$. Therefore, with probability $1-2\delta$, $$A_{r - \gamma}(W'_S, \D_S) \geq A_r(\b_r, \D) - 4\epsilon.$$ By Lemma \ref{gen_thm}, for $n$ sufficiently large, with probability $1-\delta$ $$A_{r-2\gamma}(W'_S, \D) \geq A_{r-\gamma}(W'_S, \D_S) - \epsilon.$$ Therefore, for $n$ sufficiently large, with probability $1-3\delta$ over $S \sim \D$, $$A_{r - 2\gamma}(W'_S, \D) \geq A_r(\b_r, \D) - 5\epsilon.$$ Since $\epsilon, \delta,$ and $\gamma$ were arbitrary, we are done. 
\end{proof}

The following two quick lemmas are used for the proofs of Corollaries \ref{thm_NN_gen} and \ref{thm_kern_gen}.

\begin{lem}\label{lem_point_count}
Let $B_1, B_2, \dots, B_s \subset \X$ denote $s$ balls. Let $T \subset \X$ satisfy $|T \cap \cup_1^s B_i| = m$. Let $$I_k \subseteq [s] = \{i: |B_i \cap T| \geq k\}.$$ Then $|\cup_{i \in I_k} B_i \cap T| \geq m - ks$. 
\end{lem}

\begin{proof}
For any $j \notin I_k$, $|B_j \cap T| < k$. Since there are at most $s$ such $j$, it follows that $|\cup_{i \notin I_k} B_i \cap T| < ks$. Taking the complement implies the result. 
\end{proof}

\begin{lem}\label{lem_half}
Let $S$ be a finite subset of $\X \times \Y$. For any $r > 0$, let $S_r$ denote the largest $r$-separated subset of $S$. Then $|S_r| \geq \frac{|S|}{2}$. 
\end{lem}

\begin{proof}
Let $S = \{(x_1, y_1), (x_2, y_2), \dots (x_n, y_n)\}$. 
Define: $$S_+ = \{(x_i, y_i): y_i = +1\}$$ $$S_- = \{(x_i, y_i): y_i = -1\}.$$ Observe that $S_+$ and $S_-$ are both $r$-separated and have union $S$. Therefore one must have cardinality at least $\frac{|S|}{2}$, which implies the same about $|S_r|$.
\end{proof}

\begin{proof}(\textbf{Corollary  \ref{thm_NN_gen}}) 
For convenience, we let $W'$ represent the weight function described by $\ga(S,W, r)$. In particular, $W'_S$ and $W_{S_r}$ are the same classifier, where $S_r$ denotes the largest $r$-separated subset of $S$.

Relabel the points in $S$ so that $$S_r = \{(x_1, y_1), (x_2, y_2), \dots, (x_m, y_m)\},$$ with $ m \leq n$. We will also let $S_r^\X = \{x_1, x_2, \dots, x_m\}$. 

By Theorem \ref{thm_weight_general}, it suffices to show that for any $0 < a < b$, $$\lim_{n \to \infty}\mathbb{E}_{S \sim \D^n} [\frac{1}{m} \sum_{i=1}^{m}\sup_{x \in B(x_i,a)} \sum_{j=1}^{m} w_j^{S_r}(x)I_{d(x_i, x) > b}] = 0,$$ where $w_j$ denote the weight functions corresponding to $W$. Fix $0 < a < b$, and let $\epsilon > 0$. 

Pick $\gamma > 0$ be such that $a+ 2\gamma < b$. Let $B_1, B_2, \dots, B_s$ be a $(\epsilon, \gamma, \alpha)$ decomposition of $\D$. By applying a Chernoff bound, for any $\delta > 0$ there exists $n_0$ such that for $n \geq n_0$, with probability $1-\delta$ over $S \sim \D^n$, $$|S_\X \cap \cup_1^s B_i| \geq (1-2\epsilon)n.$$  By Lemma \ref{lem_half}, $\frac{m}{n} \geq \frac{1}{2}$. It follows that $|S_r^\X \cap \cup_1^s B_i| \geq m(1-4\epsilon)$. 

Let $$J = \{i: |B_i \cap S_r^\X| \geq m\frac{\epsilon}{s}\}.$$ By Lemma \ref{lem_point_count} it follows that  $|S_r^\X \cap \cup_{i \in J} B_i| \geq m(1 - 4\epsilon) - m\epsilon = m(1 - 5\epsilon)$.

Next, observe that if $n$ is sufficiently large, then $$\frac{k_n}{m} \leq \frac{2k_n}{n} \leq \frac{\epsilon}{s}.$$ Therefore, $|B_i \cap S_r^X|_r \geq k_n$ for $i \in J$.

Fix any $B_j$ with $j \in J$, and consider $x$ with $d(x, B_j) \leq a$. Then $d(x, x') \leq a+ 2\gamma < b$ for any $x' \in B_j$. Therefore, since $|S_r^X \cap B_i| \geq k_n$, all $k_n$-nearest neighbors of $x$ have distance at most $b$ to $x$. This implies that $$\sum_1^m w_i^{S_r}(x)I_{d(x_i, x) > b} = 0.$$ 

For convenience, let $$f(x_i) =\sup_{x \in B(x_i, a)}\sum_{j=1}^m w_j^{S_r}(x)I_{d(x, x_j) > b}.$$ For $x_i \in \cup_{j \in J} B_j$, any $x \in B(x_i,a)$ trivially satisfies $d(x, B_i) \leq a$. Therefore, $f(x_i) = 0$ Since $|S_r^\X \cap \cup_{j \in J} B_j| \geq m(1-5\epsilon)$, and $f(x_i) \leq 1$ for all $1 \leq i \leq m$, we have that 
\begin{equation*}
\begin{split}
\frac{1}{m}\sum_1^m f(x_i) &= \frac{1}{m}(\sum_{x_i \in \cup_{i \in J} B_i} f(x_i) + \sum_{x_i \notin \cup_{i \in J}B_i} f(x_i))\\
&\leq  \frac{1}{m}(0 + 5\epsilon m (1)) \\
&= 5\epsilon.
\end{split}
\end{equation*}
Since all of our equations hold with probability $1-\delta$ over $S$ for sufficiently large $n$, this last one does as well. Since this entirely expression is always at most $1$ (regardless of $S$), and  since $\delta, \epsilon$ were arbitrary, we have that $$\lim_{n \to \infty} E_{S \sim \D^n}[\frac{1}{m}\sum_1^m f(x_i)] = 0,$$ which completes the proof.  
\end{proof}

\begin{proof}(\textbf{Corollary \ref{thm_kern_gen}})
For convenience, we let $W'$ represent the weight function described by $\ga(S,W, r)$. In particular, $W'_S$ and $W_{S_r}$ are the same classifier, where $S_r$ denotes the largest $r$-separated subset of $S$.

Relabel the points in $S$ so that $$S_r = \{(x_1, y_1), (x_2, y_2), \dots, (x_m, y_m)\},$$ with $ m \leq n$. We will also let $S_r^\X = \{x_1, x_2, \dots, x_m\}$. 

By Theorem \ref{thm_weight_general}, it suffices to show that for any $0 < a < b$, $$\lim_{n \to \infty}\mathbb{E}_{S \sim \D^n} [\frac{1}{m} \sum_{i=1}^{m}\sup_{x \in B(x_i,a)} \sum_{j=1}^{m} w_j^{S_r}(x)I_{d(x_i, x) > b}] = 0,$$ where $w_j$ are the weight functions corresponding to $W$. Fix $0 < a < b$, and let $\epsilon > 0$. 

Pick $\gamma > 0$ be such that $a+ 2\gamma < b$. Let $B_1, B_2, \dots, B_s$ be a $(\epsilon, \gamma, \alpha)$ decomposition of $\D$. By applying a Chernoff bound, for any $\delta > 0$ there exists $n_0$ such that for $n \geq n_0$, with probability $1-\delta$ over $S \sim \D^n$, $$|S_\X \cap \cup_1^s B_i| \geq (1-2\epsilon)n.$$  By Lemma \ref{lem_half}, $\frac{m}{n} \geq \frac{1}{2}$. It follows that $|S_r^\X \cap \cup_1^s B_i| \geq m(1-4\epsilon)$. 

Let $$J = \{i: |B_i \cap S_r^\X| \geq \frac{m\epsilon}{s}\}.$$ By Lemma \ref{lem_point_count}, $|S_r^\X \cap \cup_{i \in J} B_i| \geq m(1 - 4\epsilon) - m\epsilon = m(1 - 5\epsilon)$.

Next, consider any $x_i, x_j \in S_r^\X$, and let $x$ be a point such that $d(x_i, x) \leq a+2\gamma$ and $d(x_j, x) > b$. Recall that $W$ is constructed from kernel function $K$ and window parameter $h_n$. We then have that 
\begin{equation}\label{eqn_kern_ratio}
\begin{split}
\frac{w_j^S(x)}{w_i^S(x)} = \frac{K(\frac{d(x_j, x)}{h_n})}{K(\frac{d(x_i, x)}{h_n})}.
\end{split}
\end{equation}
Because $b > a + 2\gamma$, $\frac{d(x_j, x)}{d(x_i, x)} > 1$. Fix any $\beta > 0$. Because $\lim_{n \to \infty} h_n = 0$ and $\lim_{x \to \infty} \frac{K(cx)}{K(x)} = 0$ for $c > 1$, there exists $N$ such that for $n \geq N$, $$\frac{w_j^S(x)}{w_i^S(x)} \leq \frac{\beta\epsilon}{s}.$$ 

Fix $B_j$ with $j \in J$, and consider $x$ with $d(x, B_j) \leq a$. By the triangle inequality, $d(x, x') \leq a+ 2\gamma$ for all $x' \in B_j$. Then we have the following,
\begin{equation}\label{eqn_weight_bound}
\begin{split}
\sum_1^m w_i^{S_r}(x)I_{d(x_i, x) > b} & \stackrel{(a)}{=} \frac{\sum_1^m w_i^{S_r}(x)I_{d(x_i, x) > b}}{\sum_1^m w_i^{S_r}(x)} \\
&\stackrel{(b)}{\leq} \frac{\sum_1^m w_i^{S_r}(x)I_{d(x_i, x) > b}}{\sum_{x_i \in B_j} w_i^{S_r}(x)} \\
&\stackrel{(c)}{\leq} \frac{m\sup_{x_i: d(x_i, x)>b}w_i^{S_r}(x)}{m\epsilon/s \inf_{x_i \in B_j} w_i^{S_r}(x)} \\
&\stackrel{(d)}{\leq} \frac{\beta\epsilon/s}{\epsilon/s} = \beta.
\end{split}
\end{equation}

Equation $(a)$ holds because the total sum of weights is always 1, $(b)$ because all weights are nonnegative, $(c)$ because $|B_j \cap S_r^\X| \geq m\epsilon/s$, and $(d)$ because of equation \ref{eqn_kern_ratio}.

Let $$Z_i=\sup_{x \in B(x_i, a)}\sum_{j=1}^m w_j^{S_r}(x)I_{d(x, x_j) > b}.$$ For $x_i \in \cup_1^t B_j$, any $x \in B(x_i,a)$ trivially satisfies $d(x, B_i) \leq a$. By equation \ref{eqn_weight_bound}, it follows that $Z_i \leq \beta.$ Since $|\cup_{j \in J} B_j \cap S_r^\X| \geq m(1-5\epsilon)$ and $Z_i \leq 1$ for all $1 \leq i \leq m$, we have that 
\begin{equation*}
\begin{split}
\frac{1}{m}\sum_1^m Z_i &= \frac{1}{m}(\sum_{x_i \in \cup_{j \in J} B_j} Z_i+ \sum_{x_i \notin \cup_{j \in J} B_j} Z_i)\\
&\leq (1-5\epsilon)\beta + 5\epsilon.
\end{split}
\end{equation*}
Since all of our equations hold with probability $1-\delta$ over $S$ for sufficiently large $n$, this last one does as well. Since this entire expression is always at most $1$ (regardless of $S$), and  since $\delta, \epsilon, \beta$ were arbitrary, we have that $$\lim_{n \to \infty} E_{S \sim \D^n}[\frac{1}{m}\sum_1^m Z_i] = 0,$$ which completes the proof. 

\end{proof}

\section{Experimental Details}
%\begin{figure}[ht]
%\vskip 0.2in
%\begin{center}
%\subfloat[][Training Size = 20]{\includegraphics[width=.45\textwidth]{visual20}}\quad
%   \subfloat[][Training Size = 50]{\includegraphics[width=.45\textwidth]{visual50}}\\
%   \subfloat[][Training Size = 500]{\includegraphics[width=.45\textwidth]{visual500}}\quad
%   \subfloat[][Training Size = 3000]{\includegraphics[width=.45\textwidth]{visual3000}}
%\end{center}
%\caption{A visualization of histograms learned with training data sampled from noiseless halfmoon. As training size grows, the histogram classifier becomes increasingly susceptible to adversarial examples in the blue regions.}
%\label{fig:appendix_fig}
%\vskip -0.2in
%\end{figure}

\begin{figure}
\begin{subfigure}{0.45\textwidth}
\includegraphics[width=\linewidth]{visual20}
\caption{Training Size = 20} \label{fig:a}
\end{subfigure}\hspace*{\fill}
\begin{subfigure}{0.45\textwidth}
\includegraphics[width=\linewidth]{visual50}
\caption{Training Size = 50} \label{fig:b}
\end{subfigure}

\medskip
\begin{subfigure}{0.45\textwidth}
\includegraphics[width=\linewidth]{visual500}
\caption{Training Size = 500} \label{fig:a}
\end{subfigure}\hspace*{\fill}
\begin{subfigure}{0.45\textwidth}
\includegraphics[width=\linewidth]{visual3000}
\caption{Training Size = 3000} \label{fig:b}
\end{subfigure}
%\caption{Sixth subfigure} \label{fig:f}

\caption{Empirical accuracy/astuteness of different classifiers as a function of training sample size. Accuracy is shown in green, astuteness in purple. Left : Noiseless Setting. Right: Noisy Setting. Top Row: Histogram Classifier, Bottom Row: 1-Nearest Neighbor} \label{fig:1}
\end{figure}

\subsection{Optimal attacks against histogram classifiers}

Let $H$ be a histogram classifier, and let $(x,y)$ be any labeled example. Let $r > 0$ be some fixed robustness radius. Recall that an \textit{adversarial example} against $H$ at $(x,y)$ is any $x'$ such that $x' \in B(x,r)$ and $H(x') \neq y$. Note that if $H(x) \neq y$, then $x$ itself is an adversarial example. Conversely, if $H$ is astute at $(x,y)$ with radius $r$, then no adversarial example exists.

For arbitrary classifiers, finding adversarial examples at a given point can be challenging. However, recent work (Yang et. al. 2019) has shown that for non-parametric classifiers, there are tractable methods for doing so. The key insight is that non-parametric classifiers can be construed as a partitioning of input space into convex cells, with each cell having a given label. For example, Figure \ref{fig:appendix_fig} gives a visualization for these cells in a histogram classifier. 

Because these cells are convex, finding an adversarial example for $H$ at $(x,y)$ (here $x$ is a point in $\R^2$, and $y$ is a label) amounts to finding the closest cell $c \in H$ to $x$ such that $H(c) \neq y$. While Yang et. al. (Yang et. al. 2019) presents convex programming algorithms for doing this, the case of histograms in the $\ell_\infty$ metric is much simpler. 

As stated in definition 10, a histogram partitions the input space into hypercubes by iteratively splitting each cube into $2^d$ cubes with half the length. Therefore, the cells of a histogram are all hypercubes of varying sizes. For cell $c$, let $s(c)$ denote the length of the cube that $c$ corresponds to, and let $H(c)$ denote the label $H$ assigns to $c$. The key observation is that $c$ contains an adversarial example for $(x,y)$ if and only if $d(c, x) \leq s(c)/2 + r$, and $H(c) \neq y$. This yields the following algorithm:


Algorithm \ref{alg_hist_attack} was further optimized by utilizing nearest-neighbor type algorithms to find the ``closest" cells to $x$. This was done by grouping cells by their radii, and utilizing a separate nearest-neighbor data structure for all cells of a given radius. 

Although this algorithm doesn't have the same performance metrics as those presented in (Yang et. al. 2019), it was easily sufficient for computing the empirical astuteness for our experiments.

%\begin{algorithm}[tb]
%   \caption{Optimal attack algorithm for Histogram Classifiers}
%   \label{alg_hist_attack}
%\begin{algorithmic}
%   \STATE {\bfseries Input:} Histogram $H$, labeled point $(x,y) \in \R^2 \times \{\pm 1\}$, robustness radius $r$
%   \FOR{cell $c \in H$}
%   \IF{$d(c,x) \leq s(c)/2 + r$ and $H(c) \neq y$}
%   \STATE{Return $c$}
%   \ENDIF
%   \ENDFOR
%\end{algorithmic}
%\end{algorithm}

\begin{algorithm}[H]
    \SetAlgoLined
    {\bfseries Input:} Histogram $H$, labeled point $(x,y) \in \R^2 \times \{\pm 1\}$, robustness radius $r$\;
    
 	\For{cell $c \in H$}{
 		\If{$d(c,x) \leq s(c)/2 + r$ and $H(c) \neq y$}{
 			Return $c$
 		}
 	}
    

\caption{Optimal attack algorithm for Histogram Classifiers}
\end{algorithm}
 
\graphicspath{{./chapters/chapter1/}}

\def\supp{supp}
%\newtheorem{cor}[thm]{Corollary}
%\newtheorem{defn}[thm]{Definition}
\def\D{{\mathcal D}}
\def\U{{\mathcal U}}
\def\V{{\mathcal V}}
\def\X{\mathcal X}
\def\R{\mathbb R}
\def\Y{\{\pm 1\}}
\def\d{\rho}
\def\E{\mathbb{E}}
\def\N{\mathbb{N}}
\def\g{g}
\def\A{\mathcal{A}}
\def\nat{g_{neighbor}}
\def\bad{\D_{1/2}^{-}}
\def\natural{neighborhood preserving}
\def\Natural{Neighborhood preserving}
\def\ncons{neighborhood}
\def\Ncons{Neighborhood}

\def\calD{\mathcal{D}}
\def\calU{\mathcal{U}}
\def\calV{\mathcal{V}}

\chapter{Appendix for Chapter 2}

\section{Further Details of Definitions and Theorems}

\subsection{Non-Parametric Classifiers}

In this section, we precisely define weight functions, histogram classifiers and kernel classifiers.

\begin{defn} \label{def:weight_chap_1} 
\cite{devroye96} A \textbf{weight function} $W$ is a non-parametric classifier with the following properties.
\begin{enumerate}
	\item Given input $S = \{(x_1, y_1), (x_2, y_2,), \dots, (x_n, y_n)\} \sim \D^n$, $W$ constructs functions $w_1^S, w_2^S, \dots, w_n^S: \R^d \to [0, 1]$ such that for all $x \in \R^d$, $\sum_1^n w_i^S(x) = 1$. The functions $w_i^S$ are allowed to depend on $x_1, x_2, \dots x_n$ but must be independent of $y_1, y_2, \dots, y_n$. 
	\item $W$ has output $W_S$ defined as \[ W_S(x) = \begin{cases} 
      +1 & \sum_1^n w_i^S(x)y_i > 0 \\
      -1 & \sum_1^n w_i^S(x)y_i \leq 0 \\
   \end{cases}
\]
As a result, $w_i^S(x)$ can be thought of as the weight that $(x_i, y_i)$ has in classifying $x$.
\end{enumerate}
\end{defn}

\begin{defn}
A \textbf{histogram classifier}, $H$, is a non-parametric classification algorithm over $\R^d \times \Y$ that works as follows. For a distribution $\D$ over $\R \times \Y$, $H$ takes $S = \{(x_i, y_i): 1 \leq i \leq n\} \sim \D^n$ as input. Let $k_i$ be a sequence with $\lim_{i \to \infty} k_i = \infty$ and $\lim_{i \to \infty} \frac{k_i}{i} = 0$. $H$ constructs a set of hypercubes $C = \{c_1, c_2, \dots, c_m\}$ as follows:
\begin{enumerate}
	\item Initially $C = \{c\}$, where $S \subset c$.
	\item For $c \in C$, if $c$ contains more than $k_n$ points of $S$, then partition $c$ into $2^d$ equally sized hypercubes, and insert them into $C$.
	\item Repeat step $2$ until all cubes in $C$ have at most $k_n$ points. 
\end{enumerate}
For $x \in \R$ let $c(x)$ denote the unique cell in $C$ containing $x$. If $c(x)$ doesn't exist, then $H_S(x) = -1$ by default. Otherwise, \[ H_S(x) = \begin{cases} 
      +1 & \sum_{x_i \in c(x)} y_i > 0 \\
      -1 & \sum_{x_i \in c(x)}y_i \leq 0 \\
   \end{cases}.
\]
\end{defn}

\begin{defn}
A \textbf{partitioning rule} is a weight function $W$ over $\X \times \Y$ constructed in the following manner. Given $S = \{(x_i, y_i)\} \sim \D^n$, as a function of $\{x_1, \dots, x_n\}$, we partition $\R^d$ into regions with $A(x)$ denoting the region containing $x$.  Then, for any $x \in \R^d$ we have $$w_i^S(x) = \begin{cases}1 & x_i \in A(x) \\ 0 & \text{ otherwise}\end{cases}.$$To achieve $\sum w_i^S(x) = 1$, we can simply normalize weights for any $x$ by $\sum_1^n w_i^S(X)$.
\end{defn}

\begin{defn}
A \textbf{kernel classifier} is a weight function $W$ over $\R^d \times \Y$ constructed from function $K: \R^+ \cup \{0\} \to \R^+$ and some sequence $\{h_n\} \subset \R^+$ in the following manner. Given $S = \{(x_i, y_i)\} \sim \D^n$, we have $$w_i^S(x) = \frac{K(\frac{\d(x, x_i)}{h_n})}{\sum_{j = 1}^n K(\frac{\d(x, x_j)}{h_n})}.$$ Then, as above, $W$ has output \[ W_S(x) = \begin{cases} 
      +1 & \sum_1^n w_i^S(x)y_i > 0 \\
      -1 & \sum_1^n w_i^S(x)y_i \leq 0 \\
   \end{cases}
\]
\end{defn}

\subsection{Splitting Numbers}

We refer to definitions \ref{defn:prob_radius} and \ref{defn:splitting_number}.



The main idea behind splitting numbers is that they allow us to ensure uniform convergence properties over a weight function. To prove neighborhood consistency, it is necessary for a classifier to be correct at \textit{all} points in a given region. Consequently, techniques that consider a single point will be insufficient. The splitting number provides a mechanism for studying entire regions simultaneously. For clarity, we include a quick example in which we bound the splitting number for a given weight function.

\paragraph{Example:} Let $W$ denote any kernel classifier corresponding such that $K: \R_{\geq 0} \to \R_{\geq 0}$ is a decreasing function. For any $S \sim \D^n$, observe that the condition $w_i^S(x) \geq \beta$ precisely corresponds to $\d(x, x_i) \leq \gamma$ for some value of $\gamma$. This is because $w_i^S(x) > w_j^S(x)$ if and only if $\d(x, x_i) < \d(x, x_j)$. Thus, the regions $W_{x, \alpha, \beta}$ correspond to $\{i: \d(x, x_i) \leq \gamma\}$, where $\gamma$ is a positive real number that depends on $x, \alpha, \beta$. These sets precisely correspond to subsets of $S$ that are contained within $B(x, \gamma)$. Since balls have VC dimension at most $d+2$, by  Sauer's lemma, the number of subsets of $S$ that can be obtained in this manner is $O(n^{d+2})$. Therefore, we have that $T(W,S) = O(n^{d+2})\text{ for all }S \sim \D^n.$

\subsection{Stone's Theorem}

\begin{thm}\label{thm_stone_chapter_1}
\cite{Stone77} Let $W$ be weight function over $\R^d \times \Y$. Suppose the following conditions hold for any distribution $\D$ over $\R^d \times \Y$.  Let $X$ be a random variable with distribution $\D_{\R^d}$, and $S = \{(x_1, y_1), (x_2, y_2), \dots, (x_n, y_n)\} \sim \D^n$. All expectations are taken over $X$ and $S$. 

1. There is a constant $c$ such that, for every nonnegative measurable function $f$ satisfying $\mathbb{E} [f(X)] < \infty$, and $\mathbb{E} [\sum_1^n w_i^S(X)f(x_i)] \leq c \mathbb{E} [f(x)].$

2. $\forall a > 0$, $\lim_{n \to \infty} \mathbb{E}[\sum_1^n w_i^S(x)I_{||x_i - X|| > a||}] = 0.$ 

3. $\lim_{n \to \infty} \mathbb{E}[\max_{1 \leq i \leq n} w_i^S(X)] = 0.$

Then $W$ is consistent. 
\end{thm}

\section{Proofs}

\paragraph{Notation:} \begin{itemize}
	\item We let $\d$ denote our distance metric over $\R^d$. For sets $X_1, X_2 \subset \R^d$, we let $\d(X_1, X_2) = \inf_{x_1 \in X_1, x_2 \in X_2} \d(x_1, x_2)$. 
	\item For any $x \in \R^d$, $B(x, a) = \{x: \d(x, x') \leq a\}$.
	\item For any measure over $\R^d$, $\mu$, we let $supp(\mu) = \{x: \mu(B(x,a)) > 0\text{ for all }a > 0\}.$ 
	\item Given some measure $\mu$ over $\R^d$ and some $x \in \R^d$, we let $r_p(x)$ denote the probability radius (Definition \ref{defn:prob_radius}) of $x$ with probability $p$. that is, $r_p(x) = \inf \{r: \mu(B(x,r)) \geq p\}.$
	\item For weight function $W$ and training sample $S$, we let $W_S$ denote the weight function learned by $W$ from $S$.
\end{itemize}

\subsection{Proofs of Theorems \ref{thm:accuracy_margin} and \ref{thm:robust_margin}}

\begin{proof}
(Theorem \ref{thm:accuracy_margin}) Let $\D = (\mu, \eta)$ be a data distribution, and let $\mu^+, \mu^-$ be as described in Definition \ref{defn:includes_mu_plus_and_minus}. Observe that for any $x \in \mu^+$, the Bayes optimal classifier and the \natural\emph{ }Bayes optimal both have the same output, and furthermore the \natural\emph{ }Bayes gives this output (by definition) throughout the entirety of $V_x$, the \natural\emph{ }robustness region of $x$. It follows that the \natural\emph{ }Bayes optimal has optimal astuteness, as desired. 
\end{proof}

\begin{proof}
(Theorem \ref{thm:robust_margin}) Let $\D = (\mu, \eta)$ be a data distribution, and assume towards a contradiction that there exists classifier $f$ which has maximal astuteness with respect towards some set of robustness regions $\U = \{U_x\}$ such that $V_x \subseteq U_x$ for all $x$. The key observation is that because $f$ has maximal astuteness, we must have $f(x) = g(x)$ for almost all points $x \sim \mu$ (where $g$ is the Bayes optimal classifier). Furthermore, for those values of $x$, we must have $g$ be robust at $x$ (meaning it uniformly outputs the same output through $U_x$).

In order for $U_x$ to be strictly larger than $V_x$ for some $x$, it \textit{necessarily} must intersect with $U_{x'}$ for some $x'$ with $g(x') \neq g(x)$, and this is what causes the contradiction: $f$ cannot be astute at both $x$ and $x'$ if they are differently labeled and their robustness regions intersect. 
\end{proof}

\subsection{Proof of Theorem \ref{thm:lower_bound}}

Let $\D = (\mu, \eta)$ be the distribution with $\mu$ being the uniform distribution over $[0, 1]$ and $\eta: [0, 1] \to [0, 1]$ be $\eta(x) = x$. For example, if $(x, y) \sim \D$, then $\Pr[y = 1| x = 0.3] = 0.3$. 

We desire to show that $k_n$-nearest neighbors is not neighborhood consistent with respect to $\D$. We begin with the following key lemma.

\begin{lem}\label{cl:delta}
For any $n > 0$, let $f_n$ denote the $k_n$-nearest neighbor classifier learned from $S \sim \D^n$. There exists some constant $\Delta > 0$ such that for all sufficiently large $n$, with probability at least $\frac{1}{2}$ over $S \sim \D^n$, there exists $x \in [0,1]$ with $\frac{1}{2} - \Delta \leq x \leq \frac{1}{2} - \frac{3\Delta}{4}$ and $f_n(x) = +1$.
\end{lem}

\begin{proof}
Let $C$ be a constant such that $k_n \leq C\log n$ for all $2 \leq n < \infty$. Set $\Delta$ as \begin{equation}\label{eqn:kl}\frac{1}{2}\log_2\frac{1}{1 - 2\Delta} + \frac{1}{2}\log_2\frac{1}{1 + 2\Delta} < \frac{1}{C}.\end{equation} Let $A \subset [0,1]$ denote the interval $[\frac{1}{2} - \Delta, \frac{1}{2} - \frac{3\Delta}{4}]$. For $S \sim \D^n$, with high probability, there exist at least $\frac{\Delta n}{8}$ instances $x_i$ that are in $A$. Let us relabel these $x_i$ as $x_1, x_2, \dots, x_m$ as $$\frac{1}{2} - \Delta \leq x_1 < x_2 < \dots < x_m \leq \frac{1}{2} - \frac{3\Delta}{4}.$$

Next, suppose that for some $i$, at least half of $y_i, y_{i+1}, \dots, y_{i + k_n - 1}$ are $+1$. Then it follows that $f_n(x) = +1$ for $x = \frac{x_{i+k_n} + x_i}{2}$ because the $k_n$ nearest neighbors of $x$ are precisely $x_i, x_{i+1}, \dots x_{i + k_n - 1}$ (as a technical note we make $x$ just slightly smaller to break the tie between $x_i$ and $x_{i + k_n}$). To lower bound the probability that this occurs for some $i$, we partition $y_1, y_2, \dots y_m$ into at least $\frac{m}{2k_n}$ disjoint groups each containing $k_n$ consecutive values of $y_i$. We then bound the probability that each group will have at least $k_n/2$ $+1$s.

Consider any group of $k_n$ $y_i$s. We have that $\Pr[y_i] = +1 = \eta(x_i) = x_i \geq \frac{1}{2} - \Delta$. Since the variables $y_i$ are independent (even conditioning on $x_i$), it follows that the probability that at least half of them are $+1$ is at least  $\Pr[\text{Bin}(k_n, \frac{1}{2} - \Delta) \geq \frac{k_n}{2}].$ For simplicity, assume that $k_n$ is even. Then using a standard lower bound for the tail of a binomial distribution (see, for example, Lemma 4.7.2 of \cite{Ash90}), we have that $$\Pr[\text{Bin}(k_n, \frac{1}{2} - \Delta) \geq \frac{k_n}{2}] \geq \frac{1}{\sqrt{2k_n}}\exp(-k_nD(\frac{1}{2}||(\frac{1}{2} - \Delta)),$$ where $D(\frac{1}{2}||(\frac{1}{2} - \Delta)) =  \frac{1}{2}\log_2\frac{1}{1 - 2\Delta} + \frac{1}{2}\log_2\frac{1}{1 + 2\Delta}$. 

To simplify notation, let $D_\Delta = D(\frac{1}{2}||(\frac{1}{2} - \Delta))$. Then because we have $\frac{m}{2k_n}$ independent groups of $y_i$s, we have that
\begin{equation*}
\begin{split}
\Pr_{S \sim \D^n}[\exists x \in [\frac{1}{2} - \Delta, \frac{1}{2} - \frac{3\Delta}{4}]\text{ s.t. }f_n(x) = +1] &\geq 1 - (1 - \frac{1}{\sqrt{2k_n}}\exp(-k_nD_\Delta))^{\frac{m}{2k_n}} \\
&\geq 1 - \exp(-\frac{m}{2k_n\sqrt{2k_n}}e^{-k_nD_\Delta}) \\
&\geq 1 - \exp(-\frac{n\Delta}{(16C\log n)^{3/2}}e^{-CD_\Delta\log n}),
\end{split}
\end{equation*}
with the inequalities holding because $m \geq \frac{n\Delta}{8}$ and $k_n \leq C \log n$. By equation \ref{eqn:kl}, $CD_\Delta < 1$. Therefore, $\lim_{n \to \infty} \frac{n}{(2C \log n)^{3/2}}e^{-CD_\Delta\log n} = \infty$, which implies that for $n$ sufficiently large, $$\Pr_{S \sim \D^n}[\exists x \in [\frac{1}{2} - \Delta, \frac{1}{2} - \frac{3\Delta}{4}]\text{ s.t. }f_n(x) = +1] \geq \frac{1}{2},$$ as desired.
\end{proof}

We now complete the proof of Theorem \ref{thm:lower_bound}.

\begin{proof}
(Theorem \ref{thm:lower_bound}) Let $\Delta$ be as described in Lemma \ref{cl:delta}, and let $\kappa = \frac{1}{2}$. For all $x < \frac{1}{2}$, we have that $[x, \frac{2x}{3} + \frac{1}{6}] \subseteq V_x^{\kappa}$. This is because we can easily verify that all points inside that interval are closer to $x$ than they are to $\frac{1}{2}$ (and consequently all points in $\mu^+ \cup \mu^{1/2}$) by factor of $2$. It follows that for all $x \in [\frac{1}{2} - \frac{7\Delta}{8}, \frac{1}{2} - \Delta]$, $$[\frac{1}{2} - \Delta, \frac{1}{2} - \frac{3\Delta}{4}] \subseteq V_x^{\kappa}.$$ However, applying Lemma \ref{cl:delta}, we know that with probability at least $\frac{1}{2}$, there exists some point $x' \in [\frac{1}{2} - \Delta, \frac{1}{2} - \frac{3\Delta}{4}]$ such that $f_n(x') = +1$. It follows that with probability at least $\frac{1}{2}$, $f_n$ lacks astuteness at \textit{all} $x \in [\frac{1}{2} - \frac{7\Delta}{8}, \frac{1}{2} - \Delta]$. Since this set of points has total probability mass $\Delta/8$, it follows that with probability at least $\frac{1}{2}$, there is a fixed gap between $A_{\V^\kappa}(f_n, \D)$ and $A(g, \D)$ (as they differ in a region of probability mass at least $\Delta/8$). This implies that $k_n$-nearest neighbors is not \ncons\emph{ }consistent. 
\end{proof}

\subsection{Proof of Theorem \ref{thm:main}}

Let $\D = (\mu, \eta)$ is a distribution over $\R^d \times \{\pm 1\}$. We will use the following notation: let $\D^+ = \{x: \eta(x) > \frac{1}{2}\}$, $\D^- = \{x: \eta(x) < \frac{1}{2}$ and $\D_{1/2} = \{x: \eta(x) = \frac{1}{2}\}$. In particular, we have that $\D^+ = \mu^+, \D^- = \mu^-$ and $\D_{1/2} = \mu^{1/2}$. This notation serve will be convenient throughout this section since it allows us to avoid overloading the symbol $\mu$. 

To show that an algorithm is \ncons\emph{ }consistent with respect to $\D$, we must show that for any $0 < \kappa < 1$, the astuteness with respect to $\V^\kappa$ converges towards the accuracy of the Bayes optimal. To this end, we fix any $0 < \kappa < 1$ and consider $\V^\kappa$. 

For our proofs, it will be useful to have the additional assumption that the robustness regions, $V_x^\kappa$ are \textit{closed}. To obtain this, we let $\U = \{U_x\}$ where $U_x = \overline{V_x^\kappa}$. Each $U_x$ is the closure of the corresponding $V_x^{\kappa}$, and in particular we have $V_x^{\kappa} \subset U_x$. Because of this, it will suffice for us to consider $A_\U$ as opposed to $A_{\V^\kappa}$ since $A_\U(f, \D) \leq A_{\V^\kappa}(f, \D)$ for all classifiers $f$.

We now begin by first proving several useful properties of $\U$ that we will use throughout this entire section. 

\begin{lem}\label{lem:u_is_npr}
The collection of sets $\U = \{U_x\}$ defined as $U_x = \overline{V_x^\kappa}$ satisfies the following properties. 
\begin{enumerate}
	\item $U_x$ is closed for all $x$. 
	\item if $x \in \D^+$, for all $x' \in U_x$, $\d(x, x') < \d(\D^+ \cup \D_{1/2}, x')$.
	\item if $x \in \D^-$, for all $x' \in U_x$, $\d(x, x') < \d(\D^- \cup \D_{1/2}, x')$. 
	\item $U_x = \{x\}$ for all $x \in \D_{1/2}$. 
	\item $U_x$ is bounded for all $x$.
\end{enumerate}
Here $\mu^+, \mu^-, \mu^{1/2}$ are as described in Definition \ref{defn:includes_mu_plus_and_minus}. 
\end{lem}

\begin{proof}
Property (1) is given the by definition, and properties (2), (3) follow from the fact that $\kappa$ is strictly less than $1$. In particular, the distance function $\rho$ is continuous and consequently all limit points of a set have distances that are limits of distances within the set. Property (4) is since $V_x^\kappa = \{x\}$ for all $x \in \D_{1/2}$. 

Finally, property (5) follows from the fact that $\kappa < 1$. As $x$ gets arbitrarily far away from $x$ the ratio of its distance to $x$ with its distance to $\mu^-$ gets arbitrarily close to $1$, and consequently there is some maximum radius $R$ so that $V_x^\kappa \subset B(x, R)$. Since $B(x, R)$ is closed, it follows that $U_x \subset B(x, R)$ as well. 
\end{proof}

Next, fix $W$ as a weight function and $t_n$ is a sequence of positive integers such that the conditions of Theorem \ref{thm:main} hold, that is: 
\begin{enumerate}
	\item $W$ is consistent (with resp. to accuracy) with resp. to $\D$.
	\item For any $0 < p < 1$, $\lim_{n \to \infty} E_{S \sim \D^n} [\sup_{x \in \R^d} \sum_1^n w_i^S(x)1_{\d(x, x_i) > r_p(x)}] = 0.$
	\item $\lim_{n \to \infty} E_{S \sim D^n}[t_n \sup_{x \in \R^d} w_i^S(x)] = 0$.
	\item $\lim_{n \to \infty} E_{S \sim D^n}\frac{\log T(W,S)}{t_n} = 0$.
\end{enumerate}

Finally, we will also make the additional assumption that $\D$ has infinite support. Cases where $\D$ has finite support can be somewhat trivially handled: when the sample size goes to infinity, we will have perfect labels for every point in the support, and consequently condition 2. will ensure that any $x' \in V_x^\kappa$ is labeled according to the label of $x$. 


We also use the following notation. For any classifier $f: \R^d \to \{\pm 1\}$, we let \begin{equation}\label{eqn:cool_sets}\D_f^+ = \{x: f(x' = +1\text{ for all }x' \in U_x\},\text{ and }\D_f^- = \{x: f(x' = -1\text{ for all }x' \in U_x\}.\end{equation} These sets represent the examples that $f$ robustly labels as $+1$ and $-1$ respectively. These sets are useful since they allows us to characterize the astuteness of $f$, which we do with the following lemma.

\begin{lem}\label{lem:conversion_to_measure_thing}
For any classifier $f: \R^d \to \{\pm 1\}$, we have $$A_\U(f, \D) \geq A(g, \D) - \mu(\D^+ \setminus \D_f^+) - \mu(D^- \setminus \D_f^-),$$ where $g$ denotes the Bayes optimal classifier.
\end{lem}

\begin{proof}
By property 4 of Lemma \ref{lem:u_is_npr}, $U_x = \{x\}$ for all $x \in \D_{1/2}$. Consequently, if $x \in \D_{1/2}$, there is a $\frac{1}{2}$ chance that any classifier is astute at $(x,y)$. Using this along with the definition of astuteness, we see that 
\begin{equation*}
\begin{split}
A_\U(f, \D) &= \Pr_{(x,y) \sim \D} [f(x') = y\text{ for all }x' \in U_x] \\
&= \Pr_{(x,y) \sim \D}[y = +1\text{ and }x \in (D^+ \cap D_f^+)] + \Pr_{(x,y) \sim \D}[y = -1\text{ and }x \in (D^- \cap D_f^-)] + \frac{1}{2}\Pr_{(x,y) \sim \D}[x \in \D_{1/2}]
\end{split}
\end{equation*}
However, observe by the definitions of $\D^+, \D^-$ and $\D_{1/2}$ that $$A(g, \D) = \Pr_{(x,y) \sim \D}[y = +1\text{ and }x \in D^+] + \Pr_{(x,y) \sim \D}[y = -1\text{ and }x \in D^-] + \frac{1}{2}\Pr_{(x,y) \sim \D}[x \in \D_{1/2}].$$ Substituting this, we find that 
\begin{equation*}
\begin{split}
A_\U(f, \D) &\geq A(g, \D) - \Pr_{(x,y) \sim \D}[x \in (D^+ \setminus D_f^+)] - \Pr_{(x,y) \sim \D}[x \in (D^- \setminus D_f^-)] \\
&= A(g, \D) - \mu(\D^+ \setminus \D_f^+) - \mu(D^- \setminus \D_f^-),
\end{split}
\end{equation*}
as desired. 
\end{proof}

Lemma \ref{lem:conversion_to_measure_thing} shows that to understand how $W_S$ converges in astuteness, it suffices to understand how the regions $\D_{W_S}^+$ and $\D_{W_S}^-$ converge towards $D^+$ and $D^-$ respectively. This will be our main approach for proving Theorem \ref{thm:main}. Due to the inherent symmetry between $+$ and $-$, we will focus on showing how the region $\D_{W_S}^+$ converges towards $D^+$. The case for $-$ will be analogous. To that end, we have the following key definition. 

\begin{defn}\label{defn:covered}
Let $p, \Delta > 0.$ We say $x \in \D^+$ is $(p, \Delta)$-\textbf{covered} if for all $x' \in U_x$ and for all $x'' \in B(x', r_p(x')) \cap supp(\mu)$, $\eta(x'') > \frac{1}{2} + \Delta.$ Here $r_p$ denotes the probability radius (Definition \ref{defn:prob_radius}). We also let $\D_{p, \Delta}^+$ denote the set of all $x \in \D^+$ that are $(p, \Delta)$-covered. 
\end{defn}

If $x$ is $(p, \Delta)$-covered, it means that for all $x' \in U_x$, there is a set of points with measure $p$ around $x'$ that are both close to $x'$, and likely (with at least probability $\frac{1}{2} + \Delta$) to be labeled as $+1$. Our main idea will be to show that if $x$ is $(p, \Delta)$ covered and $n$ is sufficiently large, $x$ is likely to be in $\D_{W_S}^+$. 

We begin this process by first showing that all $x$ are $(p, \Delta)$-covered for some $p, \Delta$. To do so, it will be useful to have one more piece of notation which we will also use throughout the rest of the section. We let $$\D_{1/2}^{-} = \D^- \cup \D_{1/2} = \supp(\mu) \setminus \D^+.$$ This set will be useful, since Lemma \ref{lem:u_is_npr} implies that for all $x \in \D^+$ and for all $x' \in U_x$, $\d(x, x') < \d(\D_{1/2}^{-}, x').$ We now return to showing that all $x$ are $(p, \Delta$-covered for some $p, \Delta$.

\begin{lem}\label{lem:everything_covered}
For any $x \in \D^+$, there exists $p, \Delta > 0$ such that $x$ is $(p, \Delta)$-covered.
\end{lem}

\begin{proof}
Fix any $x$. Let $f: U_x \to \R$ be the function defined as $f(x') = \d(x', \bad) - \d(x', x)$. Observe that $f$ is continuous. By assumption, $U_x$ is closed and bounded, and consequently must attain its minimum. However, by Lemma \ref{lem:u_is_npr}, we have that $f(x') > 0$ for all $x' \in U_x$. it follows that $\min_{x' \in U_x} f(x') = \gamma$ where $\gamma > 0$.

Next, let $p = \mu(B(x, \gamma/2))$. $p > 0$ since $x \in supp(\mu)$. Observe that for any $x' \in U_x$, $r_p(x') \leq \d(x, x') + \gamma/2$, where, $r_p(x')$ denotes the probability radius of $x'$. This is because $B(x', (\d(x, x') + \gamma/2))$ contains $B(x, \gamma/2)$ which has probability mass $p$. It follows that for any $x' \in U_x$, $\d(x', \bad) \geq r_p(x') + \gamma/2$. Motivated by this observation, let $A$ be the region defined as $$A = \bigcup_{x' \in U_x} B(x', r_p(x')).$$ Then by our earlier observation, we have that $\d(A, \bad) \geq \frac{\gamma}{2}$. Since distance is continuous, it follows that $\d(\overline{A}, \bad) \geq \frac{\gamma}{2}$ as well, where $\overline{A}$ denotes the closure of $A$. 

This means that for any $x'' \in \overline{A} \cap supp(\mu)$, $\eta(x'') > \frac{1}{2}$, since otherwise $\d(\overline{A}, \bad)$ would equal $0$ (as the two sets would literally intersect). Finally, $supp(\mu)$ is a closed set (see Appendix \ref{sec:distribution_details}), and thus $\overline{A} \cap supp(\mu)$ is closed as well. Since $\eta$ is continuous (by assumption from Definition \ref{definition:neighborhood_consistent}), it follows that $\eta$ must maintain its minimum value over $\overline{A} \cap supp(\mu)$. It follows that there exists $2\Delta > 0$ such that $\eta(x'') \geq \frac{1}{2} + 2\Delta > \frac{1}{2} + \Delta$ for all $x'' \in \overline{A} \cap supp(\mu)$. 

Finally, by the definition of $A$, for all $x' \in U_x$, $B(x', r_p(x')) \subset A$. It consequently follows from the definition that $x$ is $(p, \Delta)$-covered, as desired. 
\end{proof}

While the previous lemma show that some $p, \Delta$ cover any $x \in \D^+$, this does not necessarily mean that there are some fixed $p, \Delta$ that cover \textit{all} $x \in \D^+$. Nevertheless, we can show that this is almost true, meaning that there are some $p, \Delta$ that cover \textit{most} $x \in \D^+$. Formally, we have the following lemma.

\begin{lem}\label{lem:most_covered}
For any $\epsilon > 0$, there exists $p, \Delta$ such that $\mu(\D^+ \setminus \D_{p, \Delta}^+) < \epsilon$, where $\D_{p, \Delta}^+$ is as defined in Definition \ref{defn:covered}. 
\end{lem}

\begin{proof}
Observe that if $x$ is $(p, \Delta)$-covered, then it is also $(p', \Delta')$-covered for any $p' < p$ and $\Delta' < \Delta$. This is because $B(x', r_{p'}(x')) \subset B(x', r_p(x))$ and $\frac{1}{2} + \Delta > \frac{1}{2} + \Delta'$. Keeping this in mind, define $$\mathcal{A} = \{\D_{1/i, 1/j}^+: i, j \in \N\}.$$ For any $x \in \D^+$, by Lemma \ref{lem:everything_covered} and our earlier observation, there exists $A \in \mathcal{A}$ such that $x \in A$. It follows that $\cup_{A \in \mathcal{A}} A = \D^+$. By applying Lemma \ref{lem:measure_lemma}, we see that there exists a finite subset of $\mathcal{A}$, $\{A_1, \dots, A_m\}$ such that $$\mu(A_1 \cup \dots \cup A_m\}) > \mu(\D^+) - \epsilon.$$ Let $A_k = \D_{1/i_k, 1/j_k}^+$ for $1 \leq k \leq m$. From our previous observation once again, we see that $\cup A_i \subset \D_{1/I, 1/J}^+$ where  $I = \max(i_k)$ and $J = \max(j_k)$. It follows that setting $p = 1/I$ and $\Delta = 1/J$ suffices. 
\end{proof}

Recall that our overall goal is to show that if $x$ is $(p, \Delta)$-covered, $n$ is sufficiently large, then $x$ is very likely to be in $\D_{W_S}^+$ (defined in equation \ref{eqn:cool_sets}). To do this, we will need to find sufficient conditions on $S$ for $x$ to be in $W_S$. This requires the following definitions, that are related to \textit{splitting numbers} (Definition \ref{defn:splitting_number}). 

\begin{defn}\label{defn:Delta_particular_split_thing}
Let $x \in \R^d$ be a point, and let $S = \{(x_1, y_1), \dots, (x_n, y_n)\}$ be a training set sampled from $\D^n$. For $0 \leq \alpha$, $0 \leq \beta \leq 1$, and $0 < \Delta < \frac{1}{2}$, we define $$W_{x, \alpha, \beta}^{\Delta, S} = \{i: \d(x, x_i) \leq \alpha, w_i^S(x) \geq \beta, \eta(x_i) > \frac{1}{2} + \Delta\}.$$
\end{defn}

\begin{defn}
Let $0 < \Delta < \frac{1}{2}$, and let $S = \{(x_1, y_1), \dots, (x_n, y_n)\}$ be a training set sampled from $\D^n$. Then we let $$W^{\Delta, S} = \{W_{x, \alpha, \beta}^{\Delta, S}: x \in \R^d, 0 \leq \alpha, 0 \leq \beta \leq 1\}.$$
\end{defn}

These convoluted looking sets will be useful for determining the behavior of $W_s$ at some $x \in \D_{p, \Delta}^+$. Broadly speaking, the idea is that if every set of indices $R \subset W^{\Delta, S}$ is relatively well behaved (i.e. the number of $y_i$s that are $+1$ is close to $(|R|(\frac{1}{2} + \Delta)$, the expected amount), then $W_s(x') = +1$ for all $x' \in U_x$. Before showing this, we will need a few more lemmas.

\begin{lem}\label{lem:vc_mimicry}
Fix any $\delta > 0$ and let $0 <  \Delta < \frac{1}{2}$. There exists $N$ such that for all $n > N$ the following holds. With probability $1 - \delta$ over $S \sim \D^n$, for all $R \in W^{\Delta, S}$ with $|R| > t_n$, $\frac{1}{|R|} \sum_{i \in R} y_i \geq \Delta$ 
\end{lem}

\begin{proof}
The key idea is to observe that the set $W^{\Delta, S}$ and the value $T(W, S)$ are completely determined by $\{x_1, \dots, x_n\}$. This is because weight functions choose their weights only through dependence on $x_1, \dots, x_n$. Consequently, we can take the equivalent formulation of first drawing $x_1, \dots, x_n \sim \mu^n$, and then drawing $y_i$ independently according to $y_i = 1$ with probability $\eta(x_1)$ and $0$ with probability $1 - \eta(x_i)$. In particular, we can treat $y_1, \dots, y_n$ as independent from $W^{\Delta, S}$ and $T(W, S)$ conditioning on $x_1, \dots, x_n$. 

Fix any $x_1, \dots, x_n$. First, we see that $|W^{\Delta, S}| \leq T(W, S)$. This is because $W_{x, \alpha, \beta}^{\Delta, S}$ is a subset that is uniquely defined by $W_{x, \alpha, \beta}$ (see Definitions \ref{defn:Delta_particular_split_thing} and \ref{defn:splitting_number}). Second, for any $R \in W^{\Delta, S}$, observe that for all $i \in R$, $y_i$ is a binary variable in $[-1, 1]$ with expected value at least $(\frac{1}{2} + \Delta) - (\frac{1}{2} - \Delta) = 2\Delta$ (again by the definition). It follows that if $|R| \geq t_n$, by Hoeffding's inequality $$\Pr_{y_1 \dots y_n} [\sum_{i \in R} y_i < \Delta] \leq \exp \left( -\frac{2|R|^2\Delta^2}{4|R|} \right) \leq \exp \left( -\frac{t_n\Delta^2}{2} \right).$$ Since there at most $T(W, S)$ sets $R$, it follows that $$\Pr_{y_1 \dots y_n}[\sum_{i \in R} y_i < \Delta\text{ for some }R \in W^{\Delta, S}\text{ with }|R| > t_n] \leq T(W, S)\exp \left( -\frac{t_n\Delta^2}{2} \right).$$ However, by condition 4. of Theorem \ref{thm:main}, it is not difficult to see that this quantity has expectation that tends to $0$ as $n \to \infty$ (unless $T(W, S)$ uniformly equals $1$, but this degenerate case can easily be handled on its own). Thus, for any $\delta > 0$, it follows that there exists $N$ such that for all $n > N$, with probability at least $1 - \frac{\delta}{2}$, $T(W, S)\exp \left( -\frac{t_n\Delta^2}{2} \right) \leq \frac{\delta}{2}$. This value of $N$ consequently suffices for our lemma. 
\end{proof}

We now relate $\D_{W_S}^+$ (Equation \ref{eqn:cool_sets}) to $W^{\Delta, S}$ as well as the conditions of Theorem \ref{thm:main}.

\begin{lem}\label{lem:proving_it_works}
Let $S = \{(x_1, y_1), \dots, (x_n, y_n)\}$ and let $0 < \Delta \leq \frac{1}{2}$ and $0 < p < 1$ such that the following conditions hold. 
\begin{enumerate}
	\item For all $R \in W^{\Delta, S}$ with $|R| > t_n$, $\frac{1}{|R|} \sum_{i \in R} y_i \geq \Delta$. 
	\item $\sup_{x \in \R^d} \sum_1^n w_i^S(x)1_{\d(x, x_i) > r_p(x)} < \frac{\Delta}{5}$.
	\item $t_n\sup_{x \in \R^d} w_i^S(x) < \frac{\Delta}{5}$. 
\end{enumerate}
Then $\D_{p, \Delta}^+ \subseteq \D_{W_S}^+$. 
\end{lem}

\begin{proof}
Let $x \in \D_{p, \Delta}^+$, and let $x' \in U_x$ be arbitrary. It suffices to show that $W_S(x') = +1$ (as $x, x'$ were arbitrarily chosen). From the definition of $W_S$, this is equivalent to showing that $\sum_1^n w_i^S(x')y_i > 0.$ Thus, our strategy will be to lower bound this sum using the conditions given in the lemma statement. 

We first begin by simplifying notation. Since $S$ and $x'$ are both fixed, we use $w_i$ to denote $w_i^S(x')$. Since $n$ is fixed, we will also use $t$ to denote $t_n$. Next, suppose that $|\{x_1, \dots, x_n\} \cap B(x', r_p(x'))| = k$. Without loss of generality, we can rename indices such that $\{x_1, \dots, x_n\} \cap B(x', r_p(x')) \cap B(x', r_p(x')) = \{x_1, \dots, x_k\}$, and $w_1 \geq w_2 \geq \dots \geq w_k.$ 

Let $Y_j = \sum_{i=1}^j y_i$. Our main idea will be to express the sum in terms of these $Y_j$s as follows.  
\begin{equation*}
\begin{split}
\sum_1^n w_iy_i &= \sum_1^k w_iy_i + \sum_{k+1}^n w_iy_i \\
&= w_kY_k + (w_{k-1} - w_k)Y_{k-1} + \dots + (w_{t+1} -w_{t+2})Y_{t+1} + \sum_{i = 1}^t (w_i - w_{t+1})y_i + \sum_{k+1}^n w_iy_i \\
&= \underbrace{w_kY_k + \sum_{i = t+1}^{k-1} (w_i - w_{i+1})Y_i}_{\alpha} + \underbrace{\sum_{i = 1}^t (w_i - w_{t+1})y_i}_\beta + \underbrace{\sum_{k+1}^n w_iy_i}_\tau. 
\end{split}
\end{equation*}

We now bound $\alpha, \beta$ and $\tau$ in terms of $\Delta$ by using the conditions given in the lemma. We begin with $\beta$ and $\tau$, which are considerably easier to handle.

For $\beta$, we have that 
\begin{equation*}
\begin{split}
\beta = \sum_{i=1}^t (w_i - w_{t+1})y_i \geq \sum_{i=1}^t (w_i - w_{t+1})(-1) \geq -tw_1. 
\end{split}
\end{equation*}
By condition 2 of the lemma, we see that $tw_1 < \frac{\Delta}{5}$, which implies that $\beta \geq  -\frac{\Delta}{5}$.

For $\gamma$, we have that $\gamma = \sum_{k+1}^n w_iy_i \geq -\sum_{k+1}^n w_i$. However, for all $k+1 \leq i \leq n$, by definition of $k$, $\d(x', x_i) > r_p(x')$. It follows from condition 3 of the lemma that $\gamma \geq -\frac{\Delta}{5}$.

Finally, we handle $\alpha$. Recall that $x$ is $(p, \Delta)$-covered. It follows that for all $x'' \in supp(\mu) \cap B(x', r_p(x'))$, $\eta(x'') > \frac{1}{2} + \Delta$. Thus, by the definition of $k$, $\eta(x_i) > \frac{1}{2} + \Delta$ for $1 \leq i \leq k$. It follows that if $w_i > w_{i+1}$ or $i = k$, then 
\begin{equation*}
\begin{split}
W_{x', r_p(x'), w_i}^{\Delta, S} &= \{j: \d(x', x_j) \leq r_p(x'), w_j \geq w_i, \eta(x_j) > \frac{1}{2} + \Delta\} \\
&= \{1, \dots, i\}.
\end{split}
\end{equation*}

This implies that $\{1, \dots, i\} \in W^{\Delta, S}$, and consequently that $Y_i \geq i\Delta$, from condition 1 of the lemma. It follows that for all $t < i \leq k$, $(w_{i} - w_{i+1})Y_i \geq i(w_i - w_{i+1})\Delta$, and that $w_kY_k \geq kw_k\Delta$. Substituting these, we find that 
\begin{equation*}
\begin{split}
\alpha &= w_kY_k + \sum_{i = t+1}^{k-1} (w_i - w_{i+1})Y_i \\
&\geq kw_k\Delta + \sum_{i = t + 1}^{k-1} i(w_i - w_{i+1})\Delta \\
&= w_k\Delta + w_{k-1}\Delta + \dots + w_{t+1}\Delta + (t+1)w_{t+1}\Delta. \\
&\geq (1 - \sum_{1^t} w_i - \sum_{k+1}^n w_i)\Delta \\
&\geq (1 - \frac{2\Delta}{5})\Delta \\
&\geq (\frac{4\Delta}{5}),
\end{split}
\end{equation*}
with the last inequalities holding from the arguments given for $\beta$ and $\gamma$ along with the fact that $0 < \Delta \leq  \frac{1}{2}$. Finally, substituting these, we find that $\alpha + \beta + \gamma \geq \frac{4\Delta}{5} - \frac{2\Delta}{5} = \frac{2\Delta}{5} > 0$, as desired. 
\end{proof}

We are now ready to prove the key lemma that forms one half of the main theorem (the other half corresponding to $\D_{W_S}^-$). 

\begin{lem}
Let $\delta, \epsilon > 0$. There exists $N$ such that for all $n > N$, with probability $1 - \delta$ over $S \sim \D^n$, $\mu(\D^+ \setminus \D_{W_S}^+) < \epsilon$. 
\end{lem}

\begin{proof}
First, by Lemma \ref{lem:most_covered}, let $0 < p$ and $0 < \Delta$ be such that $\mu(\D^+ \setminus \D_{p, \Delta}^+) < \epsilon$. By combining Lemma \ref{lem:vc_mimicry}, condition 3 of Theorem \ref{thm:main}, and condition 2 of Theorem \ref{thm:main} respectively,  we see that there exists $N$ such that for all $n > N$, the following hold:
\begin{enumerate}
	\item With probability at least $1 - \frac{\delta}{3}$ over $S \sim \D^n$, for all $R \in W^{\Delta, S}$ with $|R| > t_n$, $\frac{1}{|R|} \sum_{i \in R} y_i \geq \Delta$. 
	\item With probability at least $1 - \frac{\delta}{3}$ over $S \sim \D^n$, $\sup_{x \in \R^d} \sum_1^n w_i^S(x)1_{\d(x, x_i) > r_p(x)} < \frac{\Delta}{5}$.
	\item With probability at least $1 - \frac{\delta}{3}$ over $S \sim \D^n$, $t_n\sup_{x \in \R^d} w_i^S(x) < \frac{\Delta}{5}$. 
\end{enumerate}
By a union bound, this implies that $p, \Delta, S$ satisfy the conditions of Lemma \ref{lem:proving_it_works} with probability at least $1 - \delta$.  Thus, applying the Lemma, we see that with probability $1 - \delta$, $\D_{p, \Delta}^+ \subset \D_{W_S}^+$. This immediately implies our claim. 
\end{proof}

By replicating all of the work in this section for $\D^-$ and $\D_{p, \Delta}^-$, we can similarly show the following: 

\begin{lem}
Let $\delta, \epsilon > 0$. There exists $N$ such that for all $n > N$, with probability $1 - \delta$ over $S \sim \D^n$, $\mu(\D^- \setminus \D_{W_S}^-) < \epsilon$. 
\end{lem}

Combining these two lemmas with Lemma \ref{lem:conversion_to_measure_thing} immediately implies that for all $\delta, \epsilon > 0$, there exists $N$ such that for all $n > N$, with probability $1- \delta$ over $S \sim \D^n$, $$A_\U(W_S, \D) \geq A(g, \D) - \epsilon.$$ Since $V_x^\kappa \subset U_x$ and since $\kappa$ was arbitrary, this implies Theorem \ref{thm:main}, which completes our proof. 

\subsection{Proof of Corollary \ref{cor:nn}}

Recall that $k_n$-nearest neighbors can be interpreted as a weight function, in which $w_i^S(x) = \frac{1}{k_n}$ if $x_i$ is one of the $k_n$ closest points to $x$, and $0$ otherwise. Therefore, it suffices to show that the conditions of Theorem \ref{thm:main} are met. 

We let $W$ denote the weight function associated with $k_n$-nearest neighbors.

\begin{lem}
$W$ is consistent.
\end{lem}

\begin{proof}
It is well known (for example \cite{Dasgupta14}) that $k_n$-nearest neighbors is consistent for $\lim_{n \to \infty} k_n = \infty$ and $\lim_{n \to \infty} \frac{k_n}{n} = 0$. These can easily be verified for our case.
\end{proof}

\begin{lem}
For any $0 < p < 1$, $\lim_{n \to \infty} \E_{S \sim \D^n}[ \sup_{x \in \R^d} \sum_1^n w_i^S(x)1_{\d(x, x_i) > r_p(x)}] = 0.$ 
\end{lem}

\begin{proof}
It suffices to show that for $n$ sufficiently large, all $k_n$-nearest neighbors of $x$ are located inside $B(x, r_p(x))$ for all $x \in \R^d$. We do this by using a VC-dimension type argument to show that all balls $B(x, r)$ contain a number of points from $S \sim \D^n$ that is close to their expectation. 

For $x \in \R^d$ and $r \geq 0$, let $f_{x, r}$ denote the $0-1$ function defined as $f_{x,r}(x') = 1_{x' \in B(x,r)}$. Let $F = \{f_{x, r}: x \in \R^d, r \geq 0\}$ denote the class of all such functions. It is well known that the VC dimension of $F$ is at most $d+2$. 

For $f \in F$, let $\E f$ denote $\E_{(x', y) \sim \D}f(x')$ and $\E_nf$ denote $\frac{1}{n}\sum_1^n f(x_i)$, where $\E_nf$ is defined with respect to some sample $S \sim \D^n$. By the standard generalization result of Vapnik and Chervonenkis (see \cite{Dasgupta07} for a proof), we have that with probability $1- \delta$ over $S \sim \D^n$, \begin{equation}\label{eqn:vc}-\beta_n\sqrt{\E f} \leq \E f - \E_nf \leq \beta_n\sqrt{\E f}\end{equation} holds for all $f \in F$, where $\beta_n = \sqrt{(4/n)((d+2)\ln 2n + \ln(8/\delta)}.$ 

Suppose $n$ is sufficiently large so that $\beta_n \leq \frac{p}{2}$ and $\frac{k_n}{n} < \frac{p}{2}$, and suppose that equation \ref{eqn:vc} holds. Pick any $x \in \R^d$ and consider $f_{x, r}$ where $r > r_p(x)$. This implies $\E f_{x, r} \geq p$. Then by equation \ref{eqn:vc}, we see that $\E_nf \geq \frac{p}{2}$. This implies that all $k_n$ nearest neighbors of $x$ are in the ball $B(x,r)$, and that consequently $\sum_1^n w_i^S(x)1_{\d(x, x_i) > r} = 0$. Because this holds for all $x, r$ with $x \in \R^d$ and $r > r_p(x)$, it follows that equation $2$ implies that $$\sup_{x \in X} \sum_1^n w_i^S(x) 1_{\d(x, x_i) > r_p(x)} = 0.$$ Because equation \ref{eqn:vc} holds with probability at least $1 - \delta$, and $\delta$ can be made arbitrarily small, the desired claim follows. 
\end{proof}

Let $t_n = \sqrt{d k_n\log n}$.

\begin{lem}
 $\lim_{n \to \infty} E_{S \sim D^n}[t_n \sup_{x \in \R^d} w_i^S(x)] = 0$.
\end{lem}

\begin{proof}
Let $S \sim \D^n$. By the definition of $k_n$ nearest neighbors, $\sup_{x \in \R^d}w_i^S(x) = \frac{1}{k_n}$. Therefore, $t_n\sup_{x \in \R^d} w_i^S(x) = \sqrt{\frac{d\log n}{k_n}}$. By assumption 2. of corollary \ref{cor:nn}, $\lim_{n \to \infty} \frac{d \log n}{k_n} = 0$, which implies that $$\lim_{n \to \infty} \E_{S \sim D^n}[t_n\sup_{x \in \R^d} w_i^S(x)] = \lim_{n \to \infty} \sqrt{\frac{d\log n}{k_n}} = \lim_{n \to \infty} \frac{d \log n}{k_n} = 0,$$ as desired.
\end{proof}

\begin{lem}
$\lim_{n \to \infty} E_{S \sim D^n}\frac{\log T(W,S)}{t_n} = 0$.
\end{lem}

\begin{proof}
For $S \sim \D^n$, recall that $T(W, S)$ was defined as $$T(W, S) |\{W_{x, \alpha, \beta}: x \in \R^d, 0 \leq \alpha, 0 \leq \beta \leq 1\}|,$$ where $W_{x, \alpha, \beta}$ denotes $$W_{x, \alpha, \beta} = \{i: \d(x, x_i) \leq \alpha, w_i^S(x) \geq \beta\}.$$ Our goal will to be upper bound $\log T(W, S)$. 

To do so, we first need a tie-breaking mechanism for $k_n$-nearest neighbors. For each $x_i \in S$, we independently sample $z_i \in [0, 1]$ from the uniform distribution. We then tie break based upon the value of $z_i$, i.e. if $\d(x, x_i) = \d(x, x_j)$, we say that $x_i$ is closer to $x$ than $x_j$ if $z_i < z_j$. With probability $1$, no two values $z_i ,z_j$ will be equal, so this ensures that this method always works.

Let $A_{x, \alpha} = \{i: \d(x, x_i) \leq \alpha\}$ and let $B_{x, c} = \{i: z_i \leq c\}.$ The key observation is that for any $\alpha, \beta$, $W_{x, \alpha, \beta} = A_{x, \alpha} \cap B_{x, c}$ for some value of $c$. This can be seen by noting that the nearest neighbors of $x$ are uniquely determined by $\d(x, x_i)$ and $z_i$. Therefore, it suffices to bound $|A = A_{x, \alpha}: x \in \R^d, \alpha \geq 0\}|$ and $|B = \{B_{x, c}: x \in \R^d, c \geq 0\}|$. 

To bound $|A|$, observe that the set of closed balls in $\R^d$ has VC-dimension at most $d+2$. Thus by Sauer's lemma, there are at most $O(n^{d+2}$ subsets of $\{x_1, x_2, \dots, x_n\}$ that can be obtained from closed balls. Thus $|A| \leq O(n^{d+2}$. 

To bound $|B|$, we simply note that $B_{x, c}$ consists of all $i$ for which $z_i \leq c$. Since the $z_i$ can be sorted, there are at most $n+1$ such sets. Thus $|B| \leq n+1$.

Combining this, we see that $T(W, S) \leq |A||B| \leq O(n^{d+3})$. Finally, we see that $$\lim_{n \to \infty} \frac{\log T(W,S)}{t_n} = \lim_{n \to \infty} \frac{O(d\log n)}{\sqrt{k_n d \log n}} = \lim_{n \to \infty} \sqrt{\frac{O(d \log n)}{k_n}} = 0,$$ with the last inequality holding by condition 2. of Corollary \ref{cor:nn}. 

\end{proof}


Finally, we note that Corollary \ref{cor:nn} is an immediate consequence of the previous 4 lemmas as we can simply apply Theorem \ref{thm:main}.

\subsection{Proof of Corollary \ref{cor:kern}}

Let $W$ be a kernel classifier constructed from $K$ and $h_n$ such that the conditions of Corollary \ref{cor:kern} hold: that is, 
\begin{enumerate}
	\item $K: [0, \infty) \to [0, \infty)$ is decreasing and satisfies $\int_{\R^d}K(x)dx < \infty.$
	\item $\lim_{n \to \infty} h_n = 0$ and $\lim_{n \to \infty} nh_n^d = \infty$.
	\item For any $c > 1$, $\lim_{x \to \infty} \frac{K(cx)}{K(x)} = 0$.
	\item For any $x \geq 0$, $\lim_{n \to \infty} \frac{n}{\log n}K(\frac{x}{h_n}) = \infty$.
\end{enumerate}

It suffices to show that the conditions of Theorem \ref{thm:main} are met for $W$. Before doing this, we will describe one additional assumption we make for this case.

\paragraph{Additional Assumption:} We assume that $\D, \U$ are such that there exists some compact set $\X \subset \R^d$ such that for all $x \in supp(\mu)$, $U_x \subset \X$. This is primarily for convenience: observe that any distribution can be approximated arbitrarily closely by distributions satisfying these properties (as each $U_x$ is bounded by assumption). Importantly, because of this, we will note that it is possible for conditions 2. and 3. of Theorem \ref{thm:main} to be relaxed to taking supremums over $\X$ rather than $\R^d$. This is because in our proof, we only ever used these conditions in their restriction to $\bigcup_{x \in supp(\mu)} \bigcup{x' \in U_x} B(x', r_p(x'))$.

Using this assumption, we return to proving the corollary. 

\begin{lem}\label{cl:kern_consistency}
$W$ is consistent with respect to $\D$. 
\end{lem}

\begin{proof}
Condition 1. of Corollary \ref{cor:kern} imply that $K$ is a regular kernel. This together with Condition 2. implies that $W$ is consistent: a proof can be found in \cite{devroye96}. 
\end{proof}

To verify the second condition, it will be useful to have the following definition. 

\begin{defn}\label{defn:probability_epsilon_radius}
For any $p, \epsilon > 0$ and $x \in \X$, define $r_p^\epsilon$ as $$r_p^\epsilon(x) = \sup \{r: \mu(B(x, r)) - \mu(B(x, r_p(x)) \leq \epsilon\}.$$ 
\end{defn}

\begin{lem}\label{cl:rad}
For any $p, \epsilon > 0$, there exists a constant $c_p^\epsilon > 1$ such that $\frac{r_p^\epsilon(x)}{r_p(x)} \geq c_p^{\epsilon}$ for all $x \in \X$, where we set $\frac{r_p^\epsilon(x)}{r_p(x)} = \infty$ if $r_p(x) = 0$. 
\end{lem}

\begin{proof}
The basic idea is to use the fact that $\X$ is compact. Our strategy will be to analyze the behavior of $\frac{r_p^{\epsilon}(x)}{r_p(x)}$ over small balls $B(x_0, r)$ centered around some fixed $x_0$, and then use compactness to pick some finite set of balls $B(x_0, r)$. This must be done carefully because the function $x \to \frac{r_p^\epsilon(x)}{r_p(x)}$ is not necessarily continuous. 

Fix any $x_0 \in \X$. First, observe that $r_p^{\epsilon}(x_0) > r_p(x_0)$. This is because $B(x_0, r_p(x_0)) = \cap_{r > r_p(x_0)} B(x_0,r)$, and consequently $\lim_{r \downarrow r_p(x_0)} \mu(B(x_0, r)) = \mu(B(x_0, r_p(x_))).$ 

Next, define $$s_p^\epsilon(x) = \inf\{r: \mu(B(x, r_p(x)) - \mu(B(x,r)) \leq \epsilon\}.$$ We can similarly show that $r_p(x_0) > s_p^{\epsilon}(x_0)$. 

Finally, define $$r_0 = \frac{1}{3}\min(r_p^{\epsilon}(x_0) - r_p(x_0), r_p(x_0) - s_p^{\epsilon}(x_0)).$$ Consider any $x \in B^o(x_0, r_0)$ where $B^o$ denotes the open ball, and let $\alpha = \d(x_0, x)$. Then we have the following. 
\begin{enumerate}
	\item $r_p(x) \leq r_p(x_0) + \alpha$. This holds because $B(x, r_p(x_0) + \alpha)$ contains $B(x_0, r_p(x_0))$, which has probability mass at least $p$. 
	\item $r_p(x) \geq r_p(x_0) - \alpha$. This holds because if $r_p(x) < r_p(x_0) - \alpha$, then there would exists $r < r_p(x_0)$ such that $\mu(B(x_0, r)) \geq p$ which is a contradiction.
	\item $B(x_0, s_p^{\epsilon}(x_0)) \subset B(x, r_p(x)).$ This is just a consequence of the definition of $r_0$ and the previous observation.
\end{enumerate}
By the definitions of $r_p^\epsilon$ and $s_p^\epsilon$, we see that $\mu(B(x_0, r_p^{\epsilon}(x_0)) - \mu(B(x_0, s_p^\epsilon(x_0)) \leq 2\epsilon$. By the triangle inequality, $B(x, r_p^{\epsilon}(x_0) - \alpha) \subset B(x_0, r_p^\epsilon(x_0))$ and $B(x_0, s_p^{\epsilon}(x_0)) \subset B(x, r_p(x))$. it follows that $$\mu(B(x, r_p^{\epsilon}(x_0) - \alpha)) - \mu(B(x, r_p(x))) \leq 2\epsilon,$$ which implies that $r_p^{2\epsilon}(x) \geq r_p^{\epsilon}(x_0) - \alpha$. Therefore we have the for all $x \in B(x_0, r_0)$, $$\frac{r_p^{2\epsilon}(x) }{r_p(x)} \geq \frac{r_p^{\epsilon}(x_0) - \alpha}{r_p(x_0) + \alpha} \geq \frac{2r_p^\epsilon(x_0) + r_p(x_0)}{r_p^\epsilon(x_0) + 2r_p(x_0)}.$$ Notice that the last expression is a constant that depends only on $x_0$, and moreover, since $r_p^\epsilon(x_0) > r_p(x_0)$, this constant is strictly larger than $1$. Let us denote this as $c(x_0)$. Then we see that $\frac{r_p^{2\epsilon}(x)}{r_p(x)} \geq c(x_0)$ for all $x \in B^o(x_0, r_0)$. 

Finally, observe that $\{B^o(x_0, r_0): x_0 \in \X\}$ forms an open cover of $\X$ and therefore has a finite sub-cover $C$. Therefore, taking $c = \min_{B^o(x_0, r_0) \in C}c(x_0)$, we see that $\frac{r_p^{2\epsilon}(x)}{r_p(x)} \geq c > 1$ for all $x \in \X$. Because $\epsilon$ was arbitrary, the claim holds.
\end{proof}


\begin{lem}\label{cl:kern_radius}
For any $0 < p < 1$, $\lim_{n \to \infty} \E_{S \sim \D^n}[ \sup_{x \in \X} \sum_1^n w_i^S(x)1_{\d(x, x_i) > r_p(x)}] = 0.$ 
\end{lem}


\begin{proof}
Fix $p > 0$, and fix any $\epsilon, \delta > 0$. Pick $n$ sufficiently large so that the following hold.
\begin{enumerate}
	\item Let $c_p^\epsilon$ be as defined from Lemma \ref{cl:rad}. \begin{equation}\label{eqn:ratio}\sup_{x \in \X} \frac{K(c_p^\epsilon r_p(x)/h_n)}{K(r_p(x) / h_n)} < \delta.\end{equation} This is possible because of conditions 2. and 3. of Corollary \ref{cor:kern}, and because the function $x \to r_p(x)$ is continuous.
	\item With probability at least $1 - \delta$ over $S \sim \D^n$, for all $r > 0$, and $x \in \X$, \begin{equation}\label{eqn:uniform_vc}|\mu(B(x,r)) - \frac{1}{n}\sum_1^n 1_{x_i \in B(x, r)}| \leq \epsilon.\end{equation} This is possible because the set of balls $B(x,r)$ has VC dimension at most $d+2$.
\end{enumerate}
We now bound $\E_{S \sim \D^n}[ \sup_{x \in \X} \sum_1^n w_i^S(x)1_{\d(x, x_i) > r_p(x)}]$ by dividing into cases where $S$ satisfies and doesn't satisfy equation \ref{eqn:uniform_vc}. 

Suppose $S$ satisfies equation \ref{eqn:uniform_vc}. By condition 1. of Corollary \ref{cor:kern}, $K$ is decreasing, and by Lemma \ref{cl:rad}, $r_p^\epsilon(x) \geq c_p^\epsilon r_p(x)$. Therefore, we have that for any $x \in \X$,
\begin{equation*}
\begin{split}
\sum_1^n K(\d(x, x_i)/h_n)1_{\d(x, x_i) \geq r_p^\epsilon(x)} &\leq \sum_1^n K(c_p^\epsilon r_p(x)/h_n)\\
&\leq n\delta K(r_p(x)/h_n)),
\end{split}
\end{equation*}
where the second inequality comes from equation \ref{eqn:ratio}. 

Next, by the definition of $r_p^\epsilon(x)$, we have that $\mu(B(x, r_p^\epsilon(x)) - \mu(B(x,r_p(x))) \leq \epsilon$. Therefore, by applying equation \ref{eqn:uniform_vc} two times, we see that for any $x \in \X$ $$\sum_1^n K(\d(x, x_i)/h_n)1_{r_p(x) < \d(x, x_i) \leq r_p^\epsilon(x)} \leq 3n\epsilon K(r_p(x)/h_n).$$ Finally, we have that $$\sum_1^n w_i^S(x) \geq \sum_1^n K(r_p(x)/h_n)1_{\d(x, x_i) \leq r_p(x)} \geq n(p - \epsilon)K(r_p(x)/h_n).$$ Therefore, using all three of our inequalities, we have that for any $x \in \X$
\begin{equation*}
\begin{split}
\sum_1^n w_i^S(x)1_{\d(x, x_i) > r_p(x)} &= \sum_1^n w_i^S(x)1_{\d(x, x_i) > r_p^\epsilon(x)} + \sum_1^n w_i^S(x)1_{r_p^\epsilon \geq \d(x, x_i) > r_p(x)} \\
&= \frac{\sum_1^n K(\d(x, x_i)/h_n)1_{\d(x, x_i) > r_p^\epsilon(x)} + \sum_1^n K(\d(x, x_i)/h_n)1_{r_p^\epsilon \geq \d(x, x_i) > r_p(x)}}{\sum_1^n K(\d(x, x_i)/h_n)} \\
&\leq \frac{ n\delta K(r_p(x)/h_n)) + 3n\epsilon K(r_p(x)/h_n)}{n(p - \epsilon)K(r_p(x)/h_n).} \\
&= \frac{\delta + 3\epsilon}{p - \epsilon}.
\end{split}
\end{equation*}
If $S$ does \textit{not} satisfy equation \ref{eqn:uniform_vc}, then we simply have $\sup_{x \in \X} \sum_1^n w_i^S(x)1_{\d(x, x_i) > r_p(x)} \leq 1$. Combining all of this, we have that 
$$E_{S \sim \D^n} \sum_1^n w_i^S(x)1_{\d(x, x_i) > r_p(x)} \leq \delta(1) + (1-\delta)\frac{\delta + 3\epsilon}{p - \epsilon}.$$ Since $\delta, \epsilon$ can be made arbitrarily small, the result follows. 
\end{proof}

By assumption, $\X$ is compact and therefore has diameter $D < \infty$. Define $$t_ n = \sqrt{n\log nK(\frac{D}{h_n})}\text{ for }1 \leq n < \infty.$$

\begin{lem}\label{cl:kern_tn}
$\lim_{n \to \infty} E_{S \sim D^n}[t_n \sup_{x \in \X} w_i^S(x)] = 0$.
\end{lem}

\begin{proof}
Because $K$ is a decreasing function, we have that $K(D / h_n) \leq K(\d(x, x_i) / h_n) \leq K(0)$. As a result, we have that for any $x \in \X$, 
\begin{equation*}
\begin{split}
t_n\sup_{1 \leq i \leq n}w_i^S(x) &= \frac{t_n\sup_{1 \leq i \leq n}K(\d(x, x_i)/h_n)}{\sum_1^n K(\d(x, x_i)/h_n)} \\
&\leq \frac{t_n K(0)}{nK(D/h_n)} \\
&= K(0)\sqrt{\frac{n\log nK(D/h_n)}{n^2K(D/h_n)^2}} \\
&= K(0)\sqrt{\frac{\log n}{nK(D/h_n)}}.
\end{split}
\end{equation*}
However, by condition 4. of Corollary \ref{cor:kern}, $\lim_{n \to \infty} \frac{n}{\log n}K(D/h_n) = \infty$. Therefore, since the above inequality holds for all $ x\in \X$, we have that $$\lim_{n \to \infty} E_{S \sim D^n}[t_n \sup_{x \in \X} w_i^S(x)] \leq \lim_{n \to \infty} K(0)\sqrt{\frac{\log n}{nK(D/h_n)}} = 0.$$
\end{proof}

\begin{lem}\label{cl:kern_vc}
$\lim_{n \to \infty} E_{S \sim D^n}\frac{\log T(W,S)}{t_n} = 0$.
\end{lem}

\begin{proof}
For $S \sim \D^n$, recall that $T(W, S)$ was defined as $$T(W, S) |\{W_{x, \alpha, \beta}: x \in \X, 0 \leq \alpha, 0 \leq \beta \leq 1\}|,$$ where $W_{x, \alpha, \beta}$ denotes $$W_{x, \alpha, \beta} = \{i: \d(x, x_i) \leq \alpha, w_i^S(x) \geq \beta\}.$$ Our goal will to be upper bound $\log T(W, S)$. 

The key observation is that $W_{x, \alpha, \beta}$ is precisely the set of $x_i$ for which $\d(x, x_i) \leq r$ where $r$ is some threshold. This is because the restriction that $w_i^S(x) \geq \beta$ can be directly translated into $\d(x, x_i) \leq r$ for some value of $r$, as $K$ is a monotonically decreasing function. Thus, $T(W,S)$ is the number of subsets of $S$ that can be obtained by considering the interior of some ball $B(x,r)$ centered at $x$ with radius $r$.

We now observe that the set of closed balls in $\R^d$ has VC-dimension at most $d+2$. Thus by Sauer's lemma, there are at most $O(n^{d+2}$ subsets of $\{x_1, x_2, \dots, x_n\}$ that can be obtained from closed balls. Thus $T(W,S) \leq O(n^{d+2}$. 

Finally, we see that $$\lim_{n \to \infty} \frac{\log T(W,S)}{t_n} = \lim_{n \to \infty} \frac{O(d\log n)}{\sqrt{n\log nK(\frac{D}{h_n})}} \leq \lim_{n \to \infty} \sqrt{\frac{O(d \log n)}{nK(\frac{D}{h_n})}} = 0,$$ with the last equality holding by condition 4. of Corollary \ref{cor:kern}. 
\end{proof}

Finally, we note that Corollary \ref{cor:kern} is an immediate consequences of Lemmas \ref{cl:kern_consistency}, \ref{cl:kern_radius}, \ref{cl:kern_tn}, and \ref{cl:kern_vc}, as we can simply apply Theorem \ref{thm:main}.

\section{Useful Technical Definitions and Lemmas}\label{sec:useful_lemmas}

\begin{lem}\label{lem:measure_lemma}
Let $\mu$ be a measure over $\R^d$, and let $\mathcal{A}$ denote a countable collections of measurable sets $A_i$ such that $\mu(\bigcup_{A \in \mathcal{A}} A) < \infty$. Then for all $\epsilon > 0$, there exists a finite subset of $\mathcal{A}$, $\{A_1, \dots, A_m\}$ such that $$\mu(A_1 \cup A_2 \cup \dots \cup A_m) > \mu(\bigcup_{A \in \mathcal{A}} A) - \epsilon.$$ 
\end{lem}

\begin{proof}
Follows directly from the definition of a measure.
\end{proof}

\subsection{The support of a distribution}\label{sec:distribution_details}

Let $\mu$ be a probability measure over $\R^d$.

\begin{defn}
The \textbf{support} of $\mu$, $\supp(\mu)$, is defined as all $x \in \R^d$ such that for all $r > 0$, $\mu(B(x, r)) > 0$. 
\end{defn}

From this definition, we can show that $supp(\mu)$ is closed.

\begin{lem}
$supp(\mu)$ is closed.
\end{lem}

\begin{proof}
Let $x$ be a point such that $B(x, r) \cap \supp(\mu) \neq \emptyset$ for all $r > 0$. It suffices to show that $x \in supp(\mu)$, as this will imply closure. 

Let $x$ be such a point, and fix $r > 0$. Then there exists $x' \in B(x, r/2)$ such that $x' \in supp(\mu)$. By definition, we see that $\mu(B(x', r/3)) > 0$. However, $B(x', r/3) \subset B(x, r)$ by the triangle inequality. it follows that $\mu(B(x,r)) > 0$. Since $r$ was arbitrary, it follows that $x \in supp(\mu)$.
\end{proof}

\section{Experiment Details}\label{sec:experiment_details}

\begin{figure}
    \centering
        \includegraphics[scale=0.34] {the_distribution.png}
    \caption{Our data distribution $\D = (\mu, \eta)$ with $\mu^+$ shown in blue and $\mu^-$ shown in red. Observe that this simple distribution captures varying distances between the red and blue regions, which necessitates having varying sizes for robustness regions. }
    \label{fig:distribution}
\end{figure}

\paragraph{Data Distribution} Our data distribution $\D = (\mu, \eta)$ is over $\R^2 \times \{\pm 1\}$, and is defined as follows. We let $\mu^+$ consist of a uniform distribution over the circle $x^2 + y^2 = 1$, and $\mu^-$ consist of the uniform distribution over the circle $(x-0.5)^2 + y^2 = 0.04$. The two distributions are weighted so that we draw a point from $\mu^+$ with probability 0.7, and $\mu^-$ with probability $0.3$. Finally, we utilize label noise $0.2$ meaning that the label $y$ matches that given by the Bayes optimal with probability $0.2$. In summary, $\D$ can be described with the following 4 cases:
\begin{enumerate}
	\item With probability $0.7 \times 0.8$, we select $(x,y)$ with $x \in \mu^+$ and $y = +1$.
	\item With probability $0.7 \times 0.2$, we select $(x,y)$ with $x \in \mu^+$ and $y = -1$. 
	\item With probability $0.3 \times 0.8$, we select $(x,y)$ with $x \in \mu^-$ and $y = -1$.
	\item With probability $0.3 \times 0.2$, we select $(x,y)$ with $x \in \mu^-$ and $y = +1$.
\end{enumerate}
We also include a drawing (Figure \ref{fig:distribution}) of the support of $\D$, with the positive portion $\mu^+$ shown in blue and the negative portion, $\mu^-$ shown in red. 

\paragraph{Computing Robustness Regions} 

Recall that in order to measure robustness, we utilize the so-called partial \natural\emph{ }regions $V_x^\kappa$ (Definition \ref{def:partial_nat_region}) for varying values of $\kappa$. In the case of our data distribution $\D$, $V_x^\kappa$ consists of points closer to $x$ by a factor of $\kappa$ than they are to $\mu^-$ (resp. $\mu^+$) when $x \in \mu^+$ (resp. $\mu^-$). To represent a region $V_x^\kappa$, we simply use a function $f$ that verifies whether a given point $x' \in V_x^\kappa$. While this methodology is not sufficient for training general classifiers (for a whole litany of reasons: to begin with it assumes full knowledge of the distribution), it will suffice for our toy synthetic experiments. 


\paragraph{Trained Classifiers} We train two classifiers, both of which are kernel classifiers. 

The first classifier is an exponential kernel classifier with bandwidth function $h_n = \frac{1}{10\sqrt{\log n}}$ and kernel function $K(x) = e^{-x}$. 

The second classifier is a polynomial kernel classifier with bandwidth function $h_n = \frac{1}{10n^{1/3}}$ and kernel function $K(x) = \frac{1}{1 + x^2}$. 

Both of these kernels are regular kernels, and both bandwidths satisfy sufficient conditions for consistency with respect to accuracy.  In other words, both of these classifiers will converge towards the accuracy of the Bayes optimal.

However, the first classifier is selected to satisfy the criterion of Corollary \ref{cor:kern}, whereas the second is not. This distinction is reflected in our experiments. 


\paragraph{Verifying Robustness} To verify the robustness of classifier $f$ at point $x$ (with respect to $V_x^\kappa$), we simply do a grid search with grid parameter 0.01. We grid the entire regions into points with distance at most $0.01$ between them, and then verify that $f$ has the desired value at all of those points. To ensure proper robustness, we also simply verify that $f$ cannot change enough within a distance of $0.01$ by constructing an upper bound on how much $f$ can possibly change. For kernel classifiers, this is simple to do as there is a relatively straightforward upper bound on the gradient of a Kernel classifier.


 
\graphicspath{{./chapters/chapter2/}}
\newcommand\numeq[1]%
  {\stackrel{\scriptscriptstyle(\mkern-1.5mu#1\mkern-1.5mu)}{=}}
 \newcommand\numgeq[1]%
  {\stackrel{\scriptscriptstyle(\mkern-1.5mu#1\mkern-1.5mu)}{\geq}}
  
\def\D{{\mathcal D}}
\def\Pphi{\overline{\Phi}}
\def\F{{\mathcal F}}
\def\N{{\mathcal N}}
%\def\R{{\mathbb R}}
\def\E{{\mathbb E}}
\def\A{\Pi}
\def\B{\Sigma}
\def\diam{\text{diam}}
\def\c{\mathcal L}
\def\l{\ell}
\def\seq{seq}
\def\R{\mathbb{R}}
\def\C{\mathcal C}
\def\p{p}
\def\s{size}
\def\L{\mathcal L}
\def\o{opt}
\def\H{\mathcal H}
\def\calH{\mathcal H}
\def\of{approxCluster}
\def\on{onlineCluster}
\def\R{\mathbb R}
\def\Y{\{\pm 1\}}
\def\U{\mathbb U}
\def\dd{\Delta}
\def\simp{{U\Delta}}
\def\g{g}
\def\rr{R}
\def\f{f}


\chapter{Appendix for Chapter 3}

\section{Expanded summary of \cite{ravikumar20}}\label{sec:appendix_comparison}

In this section, we derive the formulation of Theorem \ref{thm:ravikumar} directly from their results. In particular, their results are not stated in terms of $L_{rob}$ and $L_{std}$, and are instead framed in terms of different parameters. To account for this, we first review these alternative parameters, and then show how the statements in Theorem \ref{thm:ravikumar} can be 

Recall, that \cite{ravikumar20} consider the setting in which the data distribution $\D_{\mu, \Sigma}$ can be characterized as a pair of Gaussians in $\R^d$, $\N(\mu, \Sigma)$ and $\N(-\mu, \Sigma)$, that are symmetric about the origin with each of them representing one label class. They consider robustness measured in any normed metric in $\R^d$, including the $\ell_p$ norm for $p \in [1, \infty]$. 

For any such distribution (and robustness radius $r$), they introduce parameters $s_{rob}(\mu, \Sigma)$ and $s_{std}(\mu, \Sigma)$, which they refer to as the robust and standard signal-to-noise ratios respectively, that are defined as follows:

$$s_{std}(\mu, \Sigma) = 2\sqrt{\mu^t\Sigma^{-1}\mu},$$ $$s_{rob}(\mu, \Sigma) = \min_{||z||_p \leq r} 2\sqrt{(\mu - z)^t\Sigma^{-1}(\mu - z)},$$ where $r$ represents the robustness radius and $\ell_p$ is the distance norm under which adversarial perturbations are measured. 

They then show that these parameters fully characterize the sample complexity for robust and standard learning respectively. They express this through the following results:
\begin{enumerate}
	\item Let $\Phi$ denote the cumulative density function of the standard normal distribution, and let $\overline{\Phi}(x) = 1 - \Phi(x)$. Then for any $\D_{\mu, \Sigma}$, 
		\begin{itemize}
			\item the optimally accurate classifier has standard loss $\Pphi(\frac{1}{2}s_{std})$.
			\item the optimally robust classifier has robust loss $\Pphi(\frac{1}{2}s_{rob})$.
		\end{itemize}		 
	\item For any learning algorithm, there exists some mixture of $\D_{\mu, \Sigma}$ such that the expected robust loss is at least $\Omega(e^{(-\frac{1}{8} + o(1))s_{rob}^2}\frac{d}{n})$.
	\item By contrast, for any distribution $\D_{\mu , \Sigma}$, it is possible to learn a classifier with expected standard loss at most $O(s_{std}e^{-\frac{1}{8}s_{std}^2}\frac{d}{n})$.
	\item Thus, by (2.) and (3.), the gap between the robust sample complexity and the standard complexity can be bounded as $$gap \geq \Omega\left(\frac{e^{(-\frac{1}{8} + o(1))s_{rob}^2}\frac{d}{n}}{s_{std}e^{-\frac{1}{8}s_{std}^2}\frac{d}{n}}\right) \simeq \Omega(e^{\frac{-1}{8}(s_{std}^2 - s_{rob}^2)}).$$ They then qualitatively analyze this gap, and observe that for large values of $\mu$ and large values of $r$, this gap can be arbitrarily large, even as a function of $d$, the dimension.
\end{enumerate}

We now show how to convert (2.), (3.), and (4.) into the statements appearing in Theorem \ref{thm:ravikumar}. As before, let us define $L_{std}$ and $L_{rob}$ as the best possible standard and robust losses for $\D_{\mu , \Sigma}$ respectively. In particular, by (1.), we have $$L_{std} = \Pphi(\frac{1}{2}s_{std}^2),\text{ and }L_{rob} = \Pphi(\frac{1}{2}s_{rob}^2).$$ We now express the bounds in (2.) and (4.) in terms of $L_{std}$ and $L_{rob}$. To do so, we use the well known inequality bounding $\Pphi(x)$ as $$\Omega(\frac{x}{x^2 + 1}e^{-x^2/2}) < \Phi(x) <  O(\frac{e^{-x^2/2}}{x}).$$ Substituting this into (2.) through (4.) imply the following, alternative forms.

\begin{enumerate}
	\item[2.] For any learning algorithm, there exists some mixture of Gaussians, $\D_{\mu, \Sigma}$ such that the expected robust loss is at least $\Omega(L_{rob}\frac{d}{n}).$
	\item[3.] For any distribution $\D_{\mu, \Sigma}$, it is possible to learn a classifier with expected standard loss at most $O(L_{std}\frac{d}{n})$.
	\item[4.] By (2.) and (3.), the gap between robust sample complexity and standard sample complexity can be expressed as $$gap \geq \Omega(\frac{L_{rob}}{L_{std}}).$$
\end{enumerate}

Together, these three statements comprise Theorem \ref{thm:ravikumar}. 

\subsection{The limiting case}

While a core difference between our works is that we consider separated distributions whereas Gaussians are non-separated, we now consider the limiting case in which a pair of Gaussians \textit{appear} separated. To do this, we will consider a case in which $L_{rob}$ is small, and $n \sim O(\frac{1}{L_{rob}})$. In this case, with high probability, a sample of size $n$ will \textit{appear} linearly $r$-separated. Examining the bound in part 1 of Theorem \ref{thm:ravikumar}, we see that their lower bound on the expected robust loss reduces to $O(\frac{1}{n}\frac{d}{n}) = O(\frac{d}{n^2})$, which is significantly weaker than ours (Theorem \ref{thm:lower}). Thus, considering Gaussians that appear linearly $r$-separated does not generalize to the general, linearly $r$-separated case. 

\section{Proof of Theorem \ref{thm:lower}}

We begin by broadly outlining our proof of Theorem \ref{thm:lower}. Let $\A$ be a probability distribution over $\F_{r, \rho}$, and let $A$ be a learning algorithm that returns a linear classifier.

\begin{enumerate}
	\item Sample $\D \sim \A$.
	\item Sample $S \sim \D^n$.
	\item Learn the classifier $A_S$ using algorithm $A$ and training sample $S$.
	\item Evaluate $A_S$ on $\D$. That is, compute $\L_r(A_S, \D)$. 
\end{enumerate}

The basic idea of our proof is to show that for an appropriate choice of $\A$, the overall expected loss of this procedure, $\L_r(A_S, \D)$, satisfies  $$\E_{D \sim \A}[\E_{S \sim \D^n}[\L_r(A_S, \D)]] \geq \Omega(\frac{d}{n}).$$ Our primary method for doing this is switching expectations. In particular, observe that $$\E_{D \sim \A}[\E_{S \sim \D^n}[\L_r(A_S, \D)]] = \E_{S \sim \B}[\E_{\D \sim \Pi|S}[\L_r(A_S, \D)]],$$ where $\B$ denotes the distribution over all $S$ obtained from first sampling $\D \sim \A$ and then sampling $S \sim \D^n$, and $\Pi|S$ denotes the posterior distribution of $\D$ after observing $S$. It then suffices to bound the quantity $\E_{\D \sim \Pi|S}[\L_r(A_S, \D)]$, which is a significantly more tractable problem since we no longer need to deal with any specifics of the Algorithm $A$. In particular, $S$ is fixed in this expectation and consequently $A_S$ is just a fixed linear classifier. This bound subsequently follows from the distribution $\Pi|S$ having enough ``variation" for this expectation to be sufficient large. 

Our proof will have the following main steps, each of which is given its own subsection.

\begin{enumerate}
	\item In section \ref{subsec:constructing_A}, we construct the distribution $\A$, and prove several important properties about it. 
	\item In section \ref{subsec:bound_expectation}, we show that the desired property of $\A$ holds, by bounding $\E_{\D \sim \A|S}[\L_r(A_S, \D)].$
\end{enumerate}


\subsection{Constructing $\A$}\label{subsec:constructing_A}

We let $r$ be a fixed robustness radius, and $\ell_p$ be our norm with which we measure robustness. Our construction of $\A$ is a somewhat technical and lengthy process. We will organize this construction into 4 subsections, outlined here:
\begin{itemize}
	\item In section \ref{subsubsec:D_a}, we define the distribution $\D_a$, characterized by parameter $a \in [0,1]^d$. This forms the basis for constructing $\A$, which will comprise of distributions $\D_a$ for certain choices of $a$. We also show that $\D_a$ is linearly $r$-separated.
	\item In section \ref{subsubsec:dd}, we define the constant $\dd$, which will be essential for specifying which values of parameter $a$ are permissible. 
	\item In section \ref{subsubsec:g1g2}, we define functions $g_1, g_2: [0, \frac{\dd}{3}] \to [0, \frac{\dd}{3}]$ that will be used to construct $\A$. 
	\item In section \ref{subsubsec:finalA}, we finally put together the previous 3 sections and construct $\A$. We also show that any $\D_a \sim \A$ satisfies $\rho(\D_a) \leq C$. 
\end{itemize}

\subsubsection{Defining $\D_a$}\label{subsubsec:D_a}

Let $e_1, e_2, \dots, e_d$ denote the standard normal basis in $\R^d$. Define $v_i = R e_i$ and $u = \frac{\rr}{\sqrt{d}} \sum_1^d e_i$, where $\rr = \frac{9rd^{1/q}}{2\sqrt{d}}$. It will also be convenient to define the following function, which we will frequently use throughout the entirety of the appendix.
\begin{defn}\label{defn:function_f}
For $1 \leq l \leq \infty$, let $\f_l: [0,1]^d \to \R^+$ be the function defined as $$\f_l(a) = \sqrt[l]{\sum_1^d|\frac{1}{\sqrt{d}} + \overline{a} - a_i|^l},$$ where $\overline{a} = \frac{1}{d}\sum_1^d a_i$. For $l = \infty$, we take the convention that $\sqrt[\infty]{\sum_1^d |x_i|^\infty} = \max_{1 \leq i \leq d} |x_i|.$ 
\end{defn}


To define $\D_a$, we first define the concept of a line segment in $\R^d$.
\begin{defn}\label{defn:line_segment}
Let $x_1, x_2 \in \R^d$ be two points. A \textbf{line segment} joining $x_1, x_2$ is defined as one of the following four sets. 
\begin{itemize}
	\item $(x_1, x_2) = \{tx_1 + (1-t)x_t: 0 < t < 1\}$.
	\item $[x_1, x_2) = \{tx_1 + (1-t)x_t: 0 \leq t < 1\}$.
	\item $(x_1, x_2] = \{tx_1 + (1-t)x_t: 0 < t \leq 1\}$.
	\item $[x_1, x_2] = \{tx_1 + (1-t)x_t: 0 \leq t \leq 1\}$.
\end{itemize}
We will always distinguish which set we mean by using the notation above. In all cases, $x_1, x_2$ are said to be the endpoints of the line segment. 
\end{defn}
We now define $\D_a$.
\begin{defn}\label{def:w_dist}
Let $a \in [0,1]^d$ be a vector, and let $\overline{a} = \frac{1}{d}\sum_1^d a_i$. Set $\lambda_a = \frac{r}{\rr}f_q(a)$, where $q$ is the dual norm of $p$. Assume that for all $1 \leq i \leq d$, $a_i > \lambda_a$ (i.e. we only $\D_a$ for $a$ for which this holds). Let $S^-$ and $S^+$ be two sets of $d$ disjoint line segments (as defined in Definition \ref{defn:line_segment}) defined as $$S^- = \{[v_i, v_i + (a_i - \lambda_a)u): 1 \leq i \leq d\},$$ $$S^+ = \{(v_i + (a_i + \lambda_a)u, v_i + u]: 1 \leq i \leq d\}.$$ Then $D_a$ is defined as the probability distribution of random variables $(X,Y)$ where 
\begin{itemize}
	\item $X$ is chosen by the following random procedure. First, sample an arbitrary segment from $S^+ \cup S^-$ with each segment chosen with probability proportional to its $\ell_2$ length. Next, $X$ is selected from the uniform distribution over the chosen line segment. In particular, the probability that $X$ lies on any interval on any line segment contained within $S^+ \cup S^-$ is directly proportional to the length of the interval. 
	\item $Y$ is $-1$ if $X \in \cup S^-$ and $+1$ if $X \in \cup S^+$.
\end{itemize}
\end{defn}

\begin{figure}[h]
\vspace{.3in}
\includegraphics[scale=0.5]{d_a_pic}
\vspace{.3in}
\caption{An illustration of $\D_a$ in two dimensions. $S^-$ is shown in red, and $S^+$ is shown in blue. The decision boundary, $H_a$, of the optimal linear classifier, $f_{w^a, 1}$, is shown in purple. }
\label{fig:d_a_illustration}
\end{figure}

We include an example of such a distribution in Figure \ref{fig:d_a_illustration}. Next, we explicitly compute a linear classifier that linearly $r$-separates $\D_a$.

\begin{defn}\label{def:normal_vector}
Let $a \in [0,1]^d$, and let $\overline{a} = \sum_{i=1}^d a_i.$ Then let $w^a$ be defined as $$w_i^a = \frac{1}{\rr} - \frac{da_i}{\rr\sqrt{d} + d\rr\overline{a}}.$$ 
\end{defn}

\begin{lem}\label{lem:normal_vector_works}
$w^a$ satisfies $\langle w^a, u\rangle = \frac{d}{\sqrt{d} + d\overline{a}}$ and $\langle w^a, v_i + a_iu \rangle = 1,$ for all $1 \leq i \leq d.$ 
\end{lem}

\begin{proof}
By the definitions of $v_i, u$, we have that 
\begin{equation*}
\begin{split}
\langle w^a, u \rangle &= \langle w^a, \frac{1}{\sqrt{d}}\sum_1^d v_i \rangle \\
&= \frac{1}{\sqrt{d}} \sum_1^d Rw_i^a \\
&= \frac{1}{\sqrt{d}} \sum_1^d 1 - \frac{da_i}{\sqrt{d} + d\overline{a}} \\
&= \frac{1}{\sqrt{d}} \sum_1^d \frac{\sqrt{d} + d\overline{a} - da_i}{\sqrt{d} + d\overline{a}} \\
&= \frac{1}{\sqrt{d}} \frac{d\sqrt{d}}{\sqrt{d} + d\overline{a}} = \frac{d}{\sqrt{d} + d\overline{a}},
\end{split}
\end{equation*}
Which proves the first claim. Next, we also have that $\langle w^a, v_i \rangle = Rw_i^a$. Summing these, we get $$Rw_i^a + \frac{da_i}{\sqrt{d} + d\overline{a}} = 1 - \frac{da_i}{\sqrt{d} + d\overline{a}} + \frac{da_i}{\sqrt{d} + d\overline{a}} = 1,$$ as desired. 
\end{proof}

We now prove that $\D_a$ is linearly $r$-separated.

\begin{lem}\label{lem:separation}
$\D_a$ is linearly $r$-separated by the classifier $f_{w_a, 1}$.
\end{lem}

\begin{proof}
Let $H_a$ denote the hyperplane passing through $\{v_i + a_iu: 1 \leq i \leq d\}$. By Lemma \ref{lem:normal_vector_works}, $H_a$ is the decision boundary of $f_{w_a, 1}$.  Referring to Figure \ref{fig:d_a_illustration}, we see that $\cup S^+$ lies entirely above $H_a$ while the set $\cup S^-$ lies entirely below the hyperplane $H_a$, which the classifier $f_{w^a, 1}$ has accuracy $1$ with respect to $\D_a$. It suffices to show that $f_{w^a, 1}$ is robust everywhere. In order to do this, we must show that all points in the support of $\D_a$ have $\ell_p$ distance at least $r$ from $H_a$. 

Fix any $1 \leq i \leq d$. Since the $\ell_p$ distance metric is invariant under translation and scales linearly with dilations, it follows that the point $x_i = v_i + (a_i - \lambda_a)u$ is the closest point on the segment $[v_i, v_i+(a_i - \lambda_a)u)$ to $H_a$. Suppose $x_i$ has distance $D$ under the $\ell_p$ norm to $H_a$. Then the key observation is that the $\ell_p$ ball, $B_p(x_i, D)$, must be tangent to $H_a$. Expressing this as an equation, we have $\max_{z \in B_p(x_i, D)} \langle z, w^a \rangle = 1,$ which can be re-written as $$\max_{||z - x_i||_p \leq D} \langle z - x_i, w^a \rangle = 1 - \langle x_i, w^a \rangle.$$ By Lemma \ref{lem:normal_vector_works} , $\langle w^a, u \rangle = \frac{d}{\sqrt{d} + d\overline{a}}$ and $\langle w^a, v_i + a_iu \rangle = 1$. Substituting this, we see that 
\begin{equation*}
\begin{split}
1 - \langle x_i, w^a \rangle &= 1 - \langle v_i + a_iu - \lambda_a u, w^a \rangle \\
&= 1 - \langle v_i + a_iu, w^a \rangle + \langle \lambda_a u, w^a \rangle \\
&= \langle \lambda_a u, w^a \rangle \\
&= \frac{d\lambda_a}{\sqrt{d} + d\overline{a}}.
\end{split}
\end{equation*}

However, by using the dual norm, we see that $\max_{||z - x_i||_p \leq D} \langle z - x_i, w^a \rangle = D||w^a||_q$. Thus it follows that
\begin{equation*}
\begin{split}
D &= \frac{d\lambda_a}{(\sqrt{d} + d\overline{a})||w^a||_q} \\
&= \frac{d\frac{r}{\rr}f_q(a)}{(\sqrt{d} + d\overline{a})||w^a||_q} \\
&= \frac{d\frac{r}{\rr}\sqrt[q]{\sum_1^d |\frac{1}{\sqrt{d}} + \overline{a} - a_i|^q}}{(\sqrt{d} + d\overline{a})||w^a||_q} \\
&= \frac{r\sqrt[q]{\sum_1^d |\frac{1}{\rr}\frac{\sqrt{d} + d\overline{a} - da_i}{(\sqrt{d} + d\overline{a})}|^q}}{||w^a||_q} \\
&= \frac{r||w^a||_q}{||w^a||_q} = r.
\end{split}
\end{equation*}
We can use an analogous argument holds for $v_i + (a_i + r_a)u$, the closest point to $H_a$ in $S^+$. Thus each point in the support of $D^a$ has distance strictly larger than $r$ (as the endpoints were not included) to $H_a$. Consequently $f_{w^a, 1}$ linearly $r$-separates $D^a$, as desired. 
\end{proof}

\subsubsection{Defining $\dd$}\label{subsubsec:dd}

Now that we have defined $\D_a$, we turn our attention to defining $\A$, which requires us to specify a distribution over valid choices of $a$. In particular, although $\D_a$ is defined for $a \in [0, 1]^d$, we will require a more stringent condition on $a$ for our construction to work. To this end, we begin by defining $\Delta$, a key parameter that characterizes the domain of $a$. To define $\dd$, we use the following lemma.

\begin{lem}\label{lem:dd}
There exists a real number $\dd > 0$ such that for all $l \in \{2, q\}$, and for all $a \in [\frac{1}{2} - \dd, \frac{1}{2} + \dd]^d$, $$||\nabla f_l(a)||_2 \leq \frac{1}{d^2\sqrt{d}},$$ where $f_l$ is as defined in Definition \ref{defn:function_f}.
\end{lem}

\begin{proof}
Since $1 \leq q < \infty$, we see that for both choices of $l$, the function $h_l(x) = (\frac{1}{\sqrt{d}} - x)^l$ is a convex function for $x \in [-\frac{1}{2\sqrt{d}}, \frac{1}{2\sqrt{d}}]$. Thus, if $\sum_1^d x_i = 0$, then by Jensen's inequality, $\sum_1^d h_l(x_i) \geq \sum_1^d h_l(0)$. Applying this, we see that for all $l \in \{2, q\}$ and for all $a \in [\frac{1}{2} - \frac{1}{4\sqrt{d}}, \frac{1}{2} + \frac{1}{4\sqrt{d}}]^d$, 
\begin{equation*}
\begin{split}
f_l(a) &= \sqrt[l]{\sum_1^d |\frac{1}{\sqrt{d}} + \overline{a} - a_i|^l} \\
&= \sqrt[l]{\sum_1^d \left(\frac{1}{\sqrt{d}} + \overline{a} - a_i\right)^l} \\
&= \sqrt[l]{\sum_1^d h_l(a_i - \overline{a})} \\
&\geq \sqrt[l]{\sum_1^d h_l(0)} \\
&= f_l((\frac{1}{2}, \frac{1}{2}, \dots, \frac{1}{2})),
\end{split}
\end{equation*}
with the first equality holding since $\overline{a} - a_i < \frac{1}{\sqrt{d}}$ and the first inequality holding since $\sum_1^d a_i - \overline{a} = 0$. Thus $f_l(a)$ must be locally minimized when $a = (\frac{1}{2}, \frac{1}{2}, \dots, \frac{1}{2})$, and it follows that $$||\nabla f_l(\frac{1}{2}, \frac{1}{2}, \dots, \frac{1}{2})||_2 = 0, \text{ for } l = 2, q.$$ Now observe that the map $H(a) = \max_{l \in \{2, q\}} ||\nabla f_l(a)||_2$ is a continuous map as long as $|a_i - \overline{a}| < \frac{1}{\sqrt{d}}$ for all $1 \leq i \leq d$. Thus there exists an open neighborhood $U$ about $(\frac{1}{2}, \frac{1}{2}, \dots, \frac{1}{2})$ such that $H(a) \leq \frac{1}{d^2\sqrt{d}}$ for all $a \in U$. Taking $\dd$ so that $[\frac{1}{2} - \dd, \frac{1}{2} + \dd]^d \subseteq U$ suffices. 
\end{proof}

\begin{defn}\label{defn:Delta}
Let $\Delta$ be any constant for which Lemma \ref{lem:dd} holds. In particular, $\Delta$ only depends on $\ell_p$, the robustness norm, and $d$, the dimension.
\end{defn}

\subsubsection{Defining $\g_1$ and $\g_2$}\label{subsubsec:g1g2}

In this section, we define functions $\g_1, \g_2: [0, \frac{\Delta}{3}] \to [0, \frac{\Delta}{3}]$ which we will use to specify $\A$. Before defining $\g_1$ and $\g_2$, we will first prove several technical lemmas.

\begin{lem}\label{lem:phi_s}
Let $I \subseteq \R$ be an interval, and $\Phi:I \to \R$ be a strictly convex function.  For any $s \in \R$ and $t \geq 0$, let $\Phi_s(t) = \Phi(s-t) + \Phi(s + t)$. Then $\Phi_s$ is a strictly increasing function.
\end{lem}

\begin{proof}
Fix $s$, and let $0 \leq t_1 < t_2$. Then we see that by Jensen's inequality (for strictly convex functions), $$\Phi(s + t_1) < \frac{(t_2-t_1)\Phi(s+t_2)}{t_1 + t_2} + \frac{2t_1\Phi(s-t_1)}{t_1 + t_2},$$ and $$\Phi(s - t_1) < \frac{(t_2-t_1)\Phi(s-t_2)}{t_1 + t_2} + \frac{2t_1\Phi(s+t_1)}{t_1 + t_2}.$$ Summing these inequalities, we see that 
\begin{equation*}
\begin{split}
\Phi_s(t_1) &= \Phi(s - t_1) + \Phi(s + t_1) \\
&< \frac{(t_2-t_1)\Phi(s+t_2)}{t_1 + t_2} + \frac{2t_1\Phi(s-t_1)}{t_1 + t_2} + \frac{(t_2-t_1)\Phi(s-t_2)}{t_1 + t_2} + \frac{2t_1\Phi(s+t_1)}{t_1 + t_2} \\
&= \frac{t_2 - t_1}{t_1 + t_2}(\Phi(s +t_2) + \Phi(s - t_2)) + \frac{2t_1}{t_1 + t_2}(\Phi(s - t_1) + \Phi(s + t_1)) \\
&= \frac{t_2 - t_1}{t_1 + t_2}\Phi_s(t_2) + \frac{2t_1}{t_1 + t_2}\Phi_s(t_1).
\end{split}
\end{equation*}
Rearranging this yields $\Phi_s(t_1) < \Phi_s(t_2)$, as desired.
\end{proof}

\begin{lem}\label{lem:lipschitz}
Let $I \subseteq \R$ be an interval, $\Phi:I \to \R$ be a strictly convex continuous function, and  $x, y, z \in I$ be real numbers with $x < y < z$. Let $\epsilon >0$ be such that $x - \epsilon \in I$ and $y + \epsilon \leq z - \epsilon$. Then there exist unique $\delta, \gamma >0$ such that the following hold: $$\delta + \gamma = \epsilon,$$ $$\Phi(x-\delta) + \Phi(y + \epsilon) + \Phi(z - \gamma) = \Phi(x) + \Phi(y) + \Phi(z)$$
\end{lem}

\begin{proof}
Fix any $\epsilon$ satisfying the desired conditions, and define $\Theta: [0, \epsilon] \to \R$ as $\Theta(t) = \Phi(x - t) + \Phi(y + \epsilon) + \Phi(z + t - \epsilon)$. Then, utilizing the definition of $\Phi_s$ from Lemma \ref{lem:phi_s}, we see that $$\Theta(t) = \Phi_{\frac{x + z - \epsilon}{2}}(\frac{z - x - \epsilon}{2} + t) + \Phi(y + \epsilon).$$ By Lemma \ref{lem:phi_s}, it follows that $\Theta$ is strictly increasing in $t$, and since $\Phi$ is continuous, so is $\Theta$. Next, we bound $\Theta(0)$ and $\Theta(\epsilon)$ to put us in the configuration to apply the intermediate value theorem. To bound $\Theta(0)$, we have
\begin{equation*}
\begin{split}
\Theta(0) &= \Phi(x) + \Phi(y + \epsilon) + \Phi(z - \epsilon) \\
&= \Phi(x) + \Phi_{\frac{y+z}{2}}(\frac{z - y}{2} - \epsilon) \\
&< \Phi(x) + \Phi_{\frac{y+z}{2}}(\frac{z - y}{2}) \\
&= \Phi(x) + \Phi(y) + \Phi(z),
\end{split}
\end{equation*}
and to bound $\Theta(\epsilon)$, we have 
\begin{equation*}
\begin{split}
\Theta(\epsilon) &= \Phi(x - \epsilon) + \Phi(y + \epsilon) + \Phi(z) \\
&= \Phi_{\frac{x+y}{2}}(\frac{y- x}{2} + \epsilon) + \Phi(z)\\
&> \Phi_{\frac{x+y}{2}}(\frac{y - x}{2}) + \Phi(z) \\
&= \Phi(x) + \Phi(y) + \Phi(z).
\end{split}
\end{equation*}
Together, these equations imply $\Theta(0) < \Phi(x) + \Phi(y) + \Phi(z) < \Theta(\epsilon)$. Since $\Theta$ is strictly increasing and continuous, there exists a unique $\delta \in [0, \epsilon]$ such that $\Theta(\delta) = \Phi(x) + \Phi(y) + \Phi(z)$. Setting $\gamma = \epsilon - \delta$, we see that $$\Theta(\delta) = \Phi(x - \delta) + \Phi(y + \epsilon) + \Phi(z - \gamma) = \Phi(x) + \Phi(y) + \Phi(z),$$ as desired.


\end{proof}



Next, we define a function that will be useful for simplifying notation, both in this section and subsequent ones.

\begin{defn}\label{defn:F_the_function}
Let $\Delta$ be as in definition \ref{defn:Delta}. For $x, y, z \in [0, \frac{\dd}{3}]$, let $$F(x, y, z) = \sqrt[q]{\left(\frac{1}{\sqrt{d}} - x\right)^q + \left(\frac{1}{\sqrt{d}} - \frac{2\Delta}{3} + y\right)^q + \left(\frac{1}{\sqrt{d}} + \frac{2\Delta}{3} + z\right)^q}.$$
\end{defn}

We now define $\g_1, \g_2$.
\begin{cor}\label{cor:lipschitz_maps}
Let $\Delta$ be as in definition \ref{defn:Delta}. There exist $1$-Lipshitz, monotonically non-decreasing functions $\g_1, \g_2: [0, \frac{\dd}{3}] \to [0, \frac{\dd}{3}]$ such that for all $t \in [0, \frac{\dd}{3}]$, $\g_1(t) + \g_2(t) = t$ and $F(t, \g_1(t), \g_2(t)) = F(0, 0, 0)$. 
\end{cor}
\begin{proof}
We have two cases.
\paragraph{Case 1: $1 < q < \infty$:} Let $\Phi: [-\Delta, \Delta] \to \R$ be defined as $\Phi(x) = (\frac{1}{\sqrt{d}} - x)^q$. Since $q > 1$, and $\Delta < \frac{1}{\sqrt{d}}$, $\Phi$ is strictly convex. Observe that $$F(x, y, z)^q = \Phi(x) + \Phi(2\frac{\dd}{3} - y) + \Phi(-2\frac{\dd}{3} - z).$$ Next, fix any $t \in [0, \frac{\Delta}{3}]$. Then observe that $-\frac{2\Delta}{3} \geq -\Delta$ and that $\frac{2\Delta}{3} - t \geq 0 + t$. This puts us in the configuration to apply Lemma \ref{lem:lipschitz}. In particular, there exist unique reals $\delta_t, \gamma_t > 0$ such that $$\delta_t + \gamma_t = t,$$ $$\Phi(-\frac{2\Delta}{3} - \delta_t) + \Phi(t) + \Phi(\frac{2\Delta}{3} - \gamma_t) = \Phi(-\frac{2\Delta}{3}) + \Phi(0) + \Phi(\frac{2\Delta}{3}).$$ We now define $g_1, g_2: [0, \frac{\Delta}{3}] \to [0, \frac{\Delta}{3}]$ as $$g_1(t) = \gamma_t\text{ and }g_2(t) = \delta_t.$$ Then it is clear that $F(0,0,0) = F(t, g_1(t), g_2(t))$ and $g_1(t) + g_2(t)$ (by directly substituting into the equations above). All that remains is to show that $g_1$ and $g_2$ are 1-Lipschitz. 

Fix any $0 \leq t_1 < t_2 \leq \frac{\Delta}{3}$, and let $t_2 - t_1 = \epsilon$. The key idea is to apply Lemma \ref{lem:lipschitz} to $-\frac{2\Delta}{3} - g_2(t_1)<  t_1 < \frac{2\Delta}{3} - g_1(t_1)$ and $\epsilon$. To do so, we first check the conditions of the lemma. 

We have that $$-\frac{2\Delta}{3} - g_2(t_1) - \epsilon \geq -\frac{2\Delta}{3} - t_1 - \epsilon = -\frac{2\Delta}{3} - t_2 \geq -\Delta,$$ and 
\begin{equation*}
\begin{split}
t_1 + \epsilon &= t_2 \\
&\leq \frac{\Delta}{3} \\
&\leq \frac{2\Delta}{3} - t_2 \\
&= \frac{2\Delta}{3} - t_1 - \epsilon \\
&\leq \frac{2\Delta}{3} - g_1(t_1) - \epsilon.
\end{split}
\end{equation*}

Thus $\epsilon$ satisfies the necessary conditions for Lemma \ref{lem:lipschitz}. Since $\Phi$ is strictly convex, by Lemma \ref{lem:lipschitz}, there exist unique $\delta, \gamma > 0$ with $\delta+ \gamma = \epsilon$ such that 
\begin{equation*}
\begin{split}
&\Phi(-\frac{2\Delta}{3} - g_2(t_1) - \delta) + \Phi(t_1 + \epsilon) + \Phi(\frac{2\Delta}{3} - g_1(t_1) - \gamma)\\ &= \Phi(-\frac{2\Delta}{3} - g_2(t_1)) + \Phi(t_1) + \Phi(\frac{2\Delta}{3} - g_1(t_1)).
\end{split}
\end{equation*}
However, by the definition of $g_1, g_2$, we see that both of these quantities are equal to $F(0,0,0)^q$. Moreover, again by the definition of $g_1, g_2$, we also have that $g_1(t_2)$ and $g_2(t_2)$ are the unique real numbers in $[0, \frac{\Delta}{3}$ that satisfy $$\Phi(-\frac{2\Delta}{3} - g_2(t_2)) + \Phi(t_2) + \Phi(\frac{2\Delta}{3}+g_1(t_2)) = F(0,0,0)^q.$$ Thus, it follows that $g_2(t_2) = g_2(t_1) + \delta$ and $g_1(t_2) = g_1(t_1) + \gamma$. However, $t_2 - t_1 = \epsilon$, and $\delta, \gamma < \epsilon$ (since they sum to $\epsilon$). Thus, we see that $|g_1(t_2) - g_1(t_1)| \leq |t_2 - t_1|$ and $|g_2(t_2) - g_2(t_1)| \leq |t_2 - t_1|$. Since $t_1$ and $t_2$ were arbitrary, it follows that $g_1$ and $g_2$ are both $1$-Lipschitz, as desired. 

Finally, since $\delta, \gamma > 0$, it follows that $g_2(t_2) > g_2(t_1)$ and $g_1(t_2) > g_1(t_1)$. Since $t_1, t_2$ were arbitrary, it follows that $g_1, g_2$ are monotonically non-decreasing.

\paragraph{Case 2: $q = 1$} In this case, since $\Delta < \frac{1}{\sqrt{d}}$ (Lemma \ref{lem:dd}), we see that $F(x, y, z) = \frac{3}{\sqrt{d}} + y + z - x$. Setting $\g_1(t) = \g_2(t) = \frac{t}{2}$ suffices, and clearly satisfies the desired properties. 
\end{proof}

\begin{defn}\label{defn:g_1_and_g_2}
Let $\Delta$ be as defined in Definition \ref{defn:Delta}. We let $g_1, g_2: [0, \frac{\Delta}{3}] \to [0, \frac{\Delta}{3}]$ be defined as any function satisfying the conditions of Corollary \ref{cor:lipschitz_maps}.
\end{defn}

\subsubsection{Putting it all together: defining $\A$}\label{subsubsec:finalA}

We are now ready to define $\A$. For convenience, we assume $d$ is a multiple of $3$.
\begin{defn}\label{defn:A}
Let $\Delta, g_1$, and $g_2$ be as defined in Definitions \ref{defn:Delta} and \ref{defn:g_1_and_g_2}. Then $\A$ is defined as the distribution of distributions $\D_a$ where $a$ is a random vector constructed as follows. Let $t_1, t_2, \dots t_{d/3}$ be drawn i.i.d from the uniform distribution over $[0, \frac{\dd}{3}]$. Then for $1 \leq i \leq d/3$, we let
	\begin{itemize}
		\item $a_i = \frac{1}{2} + t_i$.
		\item $a_{i+d/3} = \frac{1}{2} + 2\frac{\dd}{3} - g_1(t_i)$.
		\item $a_{i + 2d/3} = \frac{1}{2} - 2\frac{\dd}{3} - g_2(t_i).$
	\end{itemize}
Together the variables $a_1, a_2, \dots, a_d$ compose $a$. Thus a random distribution $\D \sim \A$ can be constructed by sampling $a$ as above and setting $\D = \D_a$.
\end{defn}

We now show that for all $\D_a \sim \A$, $\lambda_a$ (Definition \ref{def:w_dist}) is constant.

\begin{lem}\label{lem:cons_lambda}
There exists a constant $\Lambda$ such that for all $\D_a \sim \A$, $\lambda_a = \Lambda$. 
\end{lem}

\begin{proof}
Let $\D_a \sim \A$ be arbitrary. By Lemma \ref{cor:lipschitz_maps}, for all $1 \leq i \leq d$, $g_1(t_i) + g_2(t_i) = t_i$. Substituting this, we see that 
\begin{equation*}
\begin{split}
\overline{a} &= \frac{1}{d}\sum_1^d a_i \\
&= \frac{1}{d}\sum_1^{d/3} (\frac{1}{2} + t_i) + (\frac{1}{2} + \frac{2\Delta}{3} - g_1(t_i)) + (\frac{1}{2} - \frac{2\Delta}{3} - g_2(t_i)) \\
&= \frac{1}{d}\sum_1^{d/3} \frac{3}{2} \\
&= \frac{1}{2}.
\end{split}
\end{equation*}

Recall that $\lambda_a = \frac{r}{R}f_q(a) = \frac{r}{R}\sqrt[q]{\sum_1^d |\frac{1}{\sqrt{d}} + \overline{a} - a_i|^q}$. By substituting that $\overline{a} = \frac{1}{2}$ and expressing each $a_i$ in terms of $t_i$, we see that 
\begin{equation*}
\begin{split}
\lambda_a &=  \frac{r}{R}\sqrt[q]{\sum_1^d |\frac{1}{\sqrt{d}} + \overline{a} - a_i|^q}\\
 &= \frac{r}{R}\sqrt[q]{\sum_1^{d/3} \left|\frac{1}{\sqrt{d}} - t_i\right|^q + \left|\frac{1}{\sqrt{d}} + g_1(t_i) - \frac{2\Delta}{3}\right|^q + \left|\frac{1}{\sqrt{d}} + g_2(t_i) + \frac{2\Delta}{3}\right|^q} \\
&= \frac{r}{R}\sqrt[q]{\sum_1^{d/3}F(t_i, g_1(t_i), g_2(t_i))^q}, \\
\end{split}
\end{equation*}
where $F$ is defined as in Definition \ref{defn:F_the_function}. Next, by Corollary \ref{cor:lipschitz_maps}, $F(t_i, g_1(t_i), g_2(t_i)) = F(0,0,0)$ for all $1 \leq i \leq \frac{d}{3}$. Thus, if we set $\Lambda = \frac{r}{R}(\frac{d}{3})^{1/q}F(0,0,0)$, we have 
\begin{equation*}
\begin{split}
\lambda_a &= \frac{r}{R}\sqrt[q]{\sum_1^{d/3} F(t_i, g_1(t_i), g_2(t_i))^q} \\
&= \frac{r}{R}\sqrt[q]{\sum_1^{d/3}F(0,0,0)^q} \\
&= \frac{r}{R}\sqrt[q]{\frac{d}{3}F(0,0,0)^q} \\
&=  \frac{r}{R}(\frac{d}{3})^{1/q}F(0,0,0) = \Lambda,
\end{split}
\end{equation*}
proving the claim.
\end{proof}

\begin{defn}\label{defn:big_lambda}
We define $\Lambda = \frac{r}{R}(\frac{d}{3})^{1/q}F(0,0,0)$, where $F$ is defined as in Definition \ref{defn:F_the_function}.
\end{defn}

Next, we compute upper and lower bounds on $\Lambda$, both of which will be useful for subsequent lemmas. 
\begin{lem}\label{lem:lambda_bounds}
$\frac{1}{9} < \Lambda < \frac{1}{3}$. 
\end{lem}

\begin{proof}
By definition, $\Lambda = \frac{d}{3}^{1/q}F(0, 0, 0)$. Substituting the definition of $f$, we see that  $F(0, 0, 0) = \sqrt[q]{|\frac{1}{\sqrt{d}}|^q + |\frac{1}{\sqrt{d}} - \frac{2\Delta}{3}|^q + |\frac{1}{\sqrt{d}} + \frac{2\Delta}{3}|^q},$ and consequently, $$3^{1/q}|\frac{1}{\sqrt{d}} - \frac{2\Delta}{3}| \leq F(0, 0, 0) \leq 3^{1/q}|\frac{1}{\sqrt{d}} + \frac{2\Delta}{3}|.$$ By definition, $\frac{2\Delta}{3} < \frac{1}{2\sqrt{d}}$. It follows that $$\frac{r}{R}\frac{d^{1/q}}{2\sqrt{d}} < \Lambda < \frac{r}{R}\frac{3d^{1/q}}{2\sqrt{d}}.$$ Finally, since $\frac{r}{R} = \frac{2\sqrt{d}}{9d^{1/q}}$, substituting this yields $\frac{1}{9} < \Lambda < \frac{1}{3}$, as desired.
\end{proof}

Next, we show that for all $\D_a \in \A$, the aspect ratio (Definition \ref{defn:aspect_ratio}), $\rho(\D_a)$, is bounded by a constant.

\begin{lem}\label{lem:large_margin}
For all $\D_a \in \A$, we have $\rho(\D_a) \leq 18\sqrt{3}$. 
\end{lem}

\begin{proof}
We first bound the $\ell_2$ margin, $\gamma(\D_a)$ (Definition \ref{defn:margin}). Recall that the margin, $\gamma(\D_a)$ is described as the largest possible $\ell_2$ distance from the support of $\D_a$ to the decision boundary of a linear classifier. Thus, we can lower bound $\gamma(\D_a)$ by computing the distance from the support of $\D_a$ to $H_a$, the decision boundary of $f_{w^a, 1}$ (Definition \ref{def:normal_vector}).

By referring to Figure \ref{fig:d_a_illustration} (in Section \ref{subsubsec:D_a}), it becomes clear that the closest point (under the $\ell_2$ margin) from $S^-$ to $H_a$ is the point $v_i + (a_i - \lambda_a)u$, for some value of $i$. Thus it suffices to compute the $\ell_2$ distance from this point to the plane $H_a$. 

Recall that by Lemma \ref{lem:normal_vector_works}, the point $v_i + a_iu$ satisfies $\langle w^a, v_i + a_iu \rangle = 1$, and consequently must lie on the hyperplane $H_a$. Let $D$ denote the $\ell_2$ distance from $v_i + (a_i - \lambda_a)u$ to $H_a$. Since $w^a$ is the normal vector to $H_a$, it follows that
\begin{equation*}
\begin{split}
D &= \langle v_i + a_iu - (v_i + (a_i - \lambda_a)u), \frac{w^a}{||w^a||_2} \rangle \\
&= \frac{\langle \lambda_a u, w^a \rangle}{||w^a||_2} \\
&\numeq{1} \frac{\langle \Lambda u, w^a \rangle}{||w^a||_2} \\
&\numeq{2} \frac{\Lambda \frac{d}{\sqrt{d} + d\overline{a}}}{||w^a||_2}  \\
&\numeq{3} \frac{\Lambda \frac{d}{\sqrt{d} + d\overline{a}}}{\sqrt{\sum_1^d \left(\frac{\sqrt{d} + d\overline{a} - da_i}{R(\sqrt{d} + d\overline{a}}\right)^2}} \\
&= \frac{R\Lambda}{\sqrt{\sum_1^d (\frac{1}{\sqrt{d}} + \overline{a} - a_i)^2}} \\
&\numeq{4} \frac{R\Lambda}{f_2(a)}.
\end{split}
\end{equation*}
Here, (1) holds by Lemma \ref{lem:cons_lambda}, (2) holds by Lemma \ref{lem:normal_vector_works}, (3) holds by Definition \ref{def:normal_vector}, and (4) holds by Definition \ref{defn:function_f}.

Next, observe that since $\D_a \sim \A$, we must have $a \in [\frac{1}{2} - \Delta, \frac{1}{2} + \Delta]^d$. Thus it follows that $||a - (\frac{1}{2}, \frac{1}{2}, \dots, \frac{1}{2})||_2 \leq \Delta\sqrt{d}$. However, by applying Lemma \ref{lem:dd}, we also see that $f_2$ is $\frac{1}{d^2\sqrt{d}}$-Lipschitz over $[\frac{1}{2} - \Delta, \frac{1}{2} + \Delta]^d$. Thus, it follows that $$f_2(a) \leq f_2(\frac{1}{2}, \frac{1}{2}, \dots, \frac{1}{2}) + \Delta\sqrt{d} \frac{1}{d^2\sqrt{d}} \leq 2,$$ with the latter inequality holding from the definition of $\Delta$. 

Substituting this and applying Lemma \ref{lem:lambda_bounds}, we see that $$\gamma(\D_a) \geq \frac{R\Lambda}{2} \geq \frac{R}{18}.$$ Next, to bound the aspect ratio, $\rho(\D_a)$, we must also bound the $\ell_2$ diameter of $\D_a$. However, the $\ell_s$ diameter of $\D_a$ is $R\sqrt{3}$, since it is the distance from $v_i + u$ to $v_j$ for $i \neq j$. Thus, it follows that $$\rho(\D_a) = \frac{diam_2(\D_a)}{\gamma(\D_a)} \leq \frac{R\sqrt{3}}{R/18} = 18\sqrt{3},$$ as desired. 
\end{proof}

Note that a tighter analysis (and selection of $\Delta$) can give a smaller bound for $\rho(\D_a)$, but the most important fact is that $\rho(\D_a) = O(1)$. 

\subsection{Bounding the expected robust loss}\label{subsec:bound_expectation}

In this section, we finally prove our lower bound, Theorem \ref{thm:lower}. This will require a few important steps, which we have separated into the following subsections. 
\begin{itemize}
	\item In section \ref{subsubsec:loss_bounding}, we give a useful lower bound for the loss $\L_r(f, \D_a)$ where $f$ is an arbitrary linear classifier. 
	\item In section \ref{subsubsec:posterior}, we give an explicit computation for the posterior distribution $\A|S$ where $S \sim \D_a^n$ is the observed training sample. 
	\item Finally, in section \ref{subsubsec:proof}, we present the proof of Theorem \ref{thm:lower}.
\end{itemize}

\subsubsection{Bounding the loss $\L_r(f, \D_a)$}\label{subsubsec:loss_bounding}

In this section, we find a lower bound on the loss $\L_r(f, \D_a)$ where $f$ is a linear classifier. We begin by first restricting $f$ to be in the set of classifiers $$f \in \{f_{w^b, 1}: b \in [0, 1]^d\},$$ where $w^b$ is as defined in Definition \ref{def:normal_vector}. These are precisely the classifiers that have a decision boundary that passes through some point on every line segment in $\{[v_i, v_i + u]: 1 \leq i \leq d\}$. We are able to only consider these classifiers since all other linear classifiers clearly have a very high loss with respect to $\D_a$ as they necessarily misclassify at least half the points on the line segment $[v_i, v_i + u]$ for some value of $i$. 

We now find an initial lower bound on $\L_r(f_{w^b, 1}, \D_a)$.

\begin{lem}\label{lem:loss_bound_general}
Fix any $\D_a \in \A$, and let $b \in [0,1]^d$ be arbitrary. Let $w^b$ be the vector defined as in Definition \ref{def:normal_vector}, and $\lambda_b = \frac{r}{\rr}\f_q(b)$ where $f$ is as defined in Definition \ref{defn:function_f}. Then $$\L_r(f_{w^b, 1}, \D_a) \geq \frac{d(\lambda_b - \lambda_a) + \sum_1^d|a_i - b_i|}{d - 2d\Lambda}.$$ 
\end{lem}

\begin{proof}
By Lemma \ref{lem:separation}, $f_{w^b, 1}$ precisely $r$-separates $\D_b$. This implies that for all $1 \leq i \leq d$,
$$f_{w^b, 1}(x) = \begin{cases} 1 & x \in (v_i + (b_i + \lambda_b)u, v_i + u] \\-1 & x \in [v_i, v_i + (b_i - \lambda_b)u) \\ \text{not robust} & x \in [v_i + (b_i - \lambda_b)u, v_i + (b_i + \lambda_b)u] \end{cases}.$$ Without loss of generality, suppose that $b_i \geq a_i$. The key observation is that for all $1 \leq i \leq d$, if $x \in [v_i + (a_i + \lambda_a)u, v_i + (b_i + \lambda_b)u]$, then $f_{w^b, 1}(x) = -1$ for $f_{w^b, 1}$ is not robust at $x$. In both cases, we see that $f_{w^b, 1}$ is either inaccurate or not robust for all points in $[v_i + (a_i + \lambda_a)u, v_i + (b_i + \lambda_b)u]$. 

This interval has $\ell_2$ length at least $(|a_i - b_i| + (\lambda_b - \lambda_a))||u||_2$. Note that in the case that $a_i \leq b_i$ we can get an identical expression. Thus,  combining this for all $i$, we see that $f_{w^b, 1}$ is either inaccurate or not robust for a total length of $[d(\lambda_b - \lambda_a) + \sum_1^d |a_i - b_i|]||u||_2$. Dividing by the total length of the support of $\D_a$, we find that
\begin{equation*}
\begin{split}
\L_r(f_{w^b, 1}, \D_a) &\geq \frac{[d(\lambda_b - \lambda_a) + \sum_1^d |a_i - b_i|]||u||_2}{\sum_1^d ||[v_i, v_i + (a_i - \lambda_a)u) + (v_i + (a_i + \lambda_a)u, v_i + u]||_2} \\
&= \frac{[d(\lambda_b - \lambda_a) + \sum_1^d |a_i - b_i|]||u||_2}{\sum_1^d ||u_2||(1 - 2\lambda_a)} \\
&= \frac{d(\lambda_b - \lambda_a) + \sum_1^d |a_i - b_i|}{d(1 - 2\lambda_a)} \\
&= \frac{d(\lambda_b - \lambda_a) + \sum_1^d |a_i - b_i|}{d - 2d\Lambda},
\end{split}
\end{equation*}
with the last equality holding since by Lemma \ref{lem:cons_lambda}, $\lambda_a = \Lambda$. 
\end{proof}

\begin{lem}\label{lem:loss_bound_clever}
For all $\D_a \in \A$ and $b \in [0,1]^d$, $d(\lambda_a - \lambda_b) \leq \frac{1}{2}\sum_1^d |a_i - b_i|.$ 
\end{lem}

\begin{proof}
We have two cases.
\paragraph{Case 1:} $b \in [\frac{1}{2} - \Delta, \frac{1}{2} + \Delta]^d$.

Observe that $\lambda_b = \frac{r}{R}f_q(b)$ and $\lambda_a = \frac{r}{R}f_q(a)$. By Lemma \ref{lem:dd}, we see that $f_q$ is $\frac{1}{d^2\sqrt{d}}$-Lipschitz over the domain $[\frac{1}{2} - \Delta, \frac{1}{2} + \Delta]^d$. It follows that
\begin{equation*}
\begin{split}
\lambda_a - \lambda_b &= \frac{r}{R}(f_q(a) - f_q(b)) \\
&\leq \frac{r}{R}||a - b||_2\frac{1}{d^2\sqrt{d}} \\
&= \frac{2\sqrt{d}}{9d^{1/q}}||a - b||_2\frac{1}{d^2\sqrt{d}} \\
&< \frac{||a - b||_1}{2d},
\end{split}
\end{equation*}
with the last inequality following since the $\ell_2$ norm is smaller than the $\ell_1$ norm. Rearranging this gives the statement of the Lemma as desired.

\paragraph{Case 2: } $b \notin [\frac{1}{2} - \Delta, \frac{1}{2} + \Delta]^d$.

The main idea in this case will be to find $b' \in [\frac{1}{2} - \Delta, \frac{1}{2} + \Delta]^d$ such that $\lambda_b \geq \lambda_{b'}$ and such that $||b' - a||_1 \leq ||b - a||_1$. We will then apply Case 1 to get the desired result.

Without loss of generality, assume that $b_1 \geq b_2 \geq \dots \geq b_d$, and that $b_1, b_2, \dots b_k > \frac{1}{2} + \Delta$, $b_{k+1}, \dots, b_l \in [\frac{1}{2} - \Delta, \frac{1}{2} + \Delta]$, and $b_{l+1}, \dots b_d < \frac{1}{2} - \Delta$ for some values of $k$ and $l$. 

We will construct $b'$ in four steps. In each of these steps, we will change the values of $b_i$ such that neither $||a - b||_1$ nor $\lambda_b$ are increased. At each step, we let $b_i$ refer to its value at the end of the previous step. 


Finally, for reference, recall that $$\lambda_b = \frac{r}{R}f_q(b) = \frac{r}{R}\sqrt[q]{\sum_1^d |\frac{1}{\sqrt{d}} + \overline{b} - b_i|^q}.$$

\paragraph{Step 1:} We set $$b_i \leftarrow  \begin{cases} \frac{1}{k}\sum_{j=1}^k b_j & 1 \leq i \leq k \\b_i & k+1 \leq i \leq l \\  \frac{1}{d-l}\sum_{j=l+1}^d b_j& l+1 \leq i \leq d \end{cases}.$$ Since $a \in [\frac{1}{2} - \Delta, \frac{1}{2} + \Delta]^d$, we see that these operations do not change $||a - b||_1$, as $\sum_1^k |b_i - a_i| = \sum_1^k b_i - a_i$ and $\sum_{l+1}^d |b_i - a_i| = \sum_1^k a_i - b_i$. Also, observe that this operation preserves $\overline{b}$, and consequently since the function $f(x) = |\frac{1}{\sqrt{d}} + \overline{b} - x|^q$ is convex, we see that by Jensen's inequality that $\lambda_b$ is not increased by this operation.

\paragraph{Step 2:} Let $\beta = \sum_1^k(b_i - \frac{1}{2} - \Delta) - \sum_{l+1}^d (\frac{1}{2} - \Delta - b_i)$. Then we set $$b_i \leftarrow  \begin{cases} \begin{cases} \frac{1}{2} + \Delta + \frac{\beta}{k} & 1 \leq i \leq k \\ b_i & k+1 \leq i \leq l \\ \frac{1}{2} - \Delta & l+1 \leq i \leq d \end{cases} & \beta \geq 0 \\\begin{cases} \frac{1}{2} + \Delta & 1 \leq i \leq k \\ b_i & k+1 \leq i \leq l \\ \frac{1}{2} - \Delta + \frac{\beta}{d - l} & l+1 \leq i \leq d \end{cases} & \beta < 0\end{cases}.$$

Observe that this operation cannot increase $||a - b||_1$, since it doesn't increase $|a_i - b_i|$ for any value of $i$. Furthermore, this operation also does not change $\overline{b}$, and a similar convexity argument on the function $f(x) = |\frac{1}{\sqrt{d}} + \overline{b} - x|^q$ can show that this does not increase $\lambda_b$. 

Finally, if $\beta = 0$, we set $b' = b$, since we have reached a state such that $b \in [\frac{1}{2} - \Delta, \frac{1}{2} + \Delta]^d$. 

\paragraph{Step 3a:} We run this step if $\beta > 0$. Let $\alpha = \frac{\sum_{k+1}^d (\frac{1}{2} + \Delta - b_i)}{\beta}$. We then set  $$b_i \leftarrow  \begin{cases} \begin{cases} \frac{1}{2} + \Delta & 1 \leq i \leq k \\ (\frac{1}{2} + \Delta)(\frac{\alpha - 1}{\alpha}) + \frac{b_i}{\alpha} & k+1 \leq i \leq d \end{cases} & \alpha \geq 1 \\\begin{cases} \frac{1}{2} + \Delta + \frac{\beta}{k}(1 - \alpha) & 1 \leq i \leq k \\ \frac{1}{2} + \Delta & k+1 \leq i \leq d \end{cases} & \alpha < 1\end{cases}.$$ In this step, we can similarly verify that $||a - b||_1$ does not increase (as $|a_i - b_i|$ is strictly reduced for $1 \leq i \leq k$ by an exact amount to offset the possible increases in $|a_i - b_i|$ for $k+1 \leq i \leq d$). We also see by the same convexity argument as usual that this operation reduces $\lambda_b$. 

\paragraph{Step 3b:} We run this step if $\beta < 0$. Let $\alpha = \frac{\sum_{k+1}^d (b_i - \frac{1}{2} + \Delta)}{\beta}$. We then set  $$b_i \leftarrow  \begin{cases} \begin{cases} (\frac{1}{2} - \Delta)(\frac{\alpha - 1}{\alpha}) + \frac{b_i}{
\alpha} & 1 \leq i \leq l \\ \frac{1}{2} - \Delta & k+1 \leq i \leq d \end{cases} & \alpha \geq 1 \\\begin{cases} \frac{1}{2} - \Delta & 1 \leq i \leq l \\ \frac{1}{2} - \Delta + \frac{\beta}{d-l}(1-\alpha) & l+1 \leq i \leq d \end{cases} & \alpha < 1\end{cases}.$$ The justification for this step is analogous to 3a.

\paragraph{Step 4:} We only run this step if $\alpha < 1$. Observe that if $\alpha \geq 1$, then both Step 3a and Step 3b result with $b \in [\frac{1}{2} - \Delta, \frac{1}{2} + \Delta]^d$, which we set as $b'$. Observe that in this case, either $b_i \geq a_i$ for all $i$, or $b_i \leq a_i$ for all $i$. Thus we simply set $$b_i \leftarrow \overline{b}.$$ This operation does not change $||a - b||_1$, and it also reduces $\lambda_b$ (by a convexity argument). 

\paragraph{Step 5:} Finally, for all $1 \leq i \leq d\Delta$, we set $b_i = \frac{1}{2}-\Delta$ if $\overline{b} < \frac{1}{2} - \Delta$ and otherwise set $b_i = \frac{1}{2} - \Delta$ if $\overline{b} > \frac{1}{2} + \Delta$. In both cases, $\lambda_b$ is not changed, and $||a-b||_1$ is strictly reduced. In this step, we finally set $b' = b$. Note that we do not always reach this step, as it was possible in any of the previous steps to reach some $b \in [\frac{1}{2} - \Delta, \frac{1}{2} + \Delta]^d$, at which point we would have simply terminated.

\paragraph{Conclusion: } Through steps $1$ through $5$, we have found $b' \in [\frac{1}{2} - \Delta, \frac{1}{2} + \Delta]^d$ such that $\lambda_{b'} \leq \lambda_b$ and $||a - b'||_1 \leq ||a - b||_1$. By applying Case 1 to $b'$, we see that $d(\lambda_a - \lambda_{b'}) \leq \frac{1}{2}||a - b'||_1$. Thus, we have that $$\frac{1}{2}||a - b||_1 \geq \frac{1}{2}||a - b'||_1 \geq d(\lambda_a - \lambda_{b'}) \geq d(\lambda_a - \lambda_b),$$ which implies the result by the transitive property.


\end{proof}

From the previous two lemmas, we immediately have the following:
\begin{cor}\label{cor:l_1distancebound}
For all $\D_a \in \A$ and $b \in [0,1]^d$, $$\L_r(f_{w^b, 1}, \D_a) \geq \frac{1}{2d}\sum_1^d |a_i - b_i|.$$
\end{cor}

\begin{proof}
We have that
\begin{equation*}
\begin{split}
\L_r(f_{w^b, 1}, \D_a) &\numgeq{a} \frac{d(\lambda_b - \lambda_a) + \sum_1^d|a_i - b_i|}{d - 2d\Lambda} \\
&\geq \frac{\sum_1^d|a_i - b_i| - d(\lambda_a - \lambda_b) + }{d} \\
&\numgeq{b} \frac{\sum_1^d|a_i - b_i| - \frac{1}{2}\sum_1^d |a_i - b_i|}{d} \\
&= \frac{1}{2d}\sum_1^d |a_i - b_i|,
\end{split}
\end{equation*}
where (a) holds by Lemma \ref{lem:loss_bound_general} and (b) holds by Lemma \ref{lem:loss_bound_clever}. 
\end{proof}

\subsubsection{Computing the posterior distribution, $\A|S$}\label{subsubsec:posterior}

Recall that our ultimate goal is to show that $$\E_{\D \sim \A}[\E_{S \sim \D^n}[ \L_r(A_S, \D)]] \geq \Omega(\frac{d}{n}),$$ where $A$ denotes any learning algorithm returning a linear classifier.  The main idea for showing this is to ``switch expectations" and realize that $$\E_{\D \sim \A}[\E_{S \sim \D^n} [\L_r(A_S, \D)]] = \E_{S \sim \B}[\E_{\D \sim \A|S}[\L_r(A_S, \D)]],$$ where $\A|S$ denotes the posterior distribution over $\A$ after observing $S$. In this section, we fully characterize the distribution $\A|S$, and prove several important properties about it.

Recall (Definition \ref{defn:A}) that $\D_a \sim \A$ is generated by first choosing $t_1, t_2, \dots, t_{d/3} \sim \U[0, \frac{\dd}{3}]$ i.i.d, and then letting $a = (a_1, a_2, \dots, a_d)$ be a function of $t = (t_1, \dots, t_{d/3})$. Thus, to compute the posterior $\A|S$, it suffices to focus on the posterior distribution of $t|S$ for any $1 \leq i \leq \frac{d}{3}$. We begin by first defining the likelihood of observing $S$ given that it is generated from parameter $t$.

\begin{defn}\label{defn:L(S|t)}
Let $S = \{(x_1, y_1), (x_2, y_2), \dots, (x_n, y_n)\}$ be any set of $n$ points in $\R^d \times \{\pm 1\}$, and let $t \in [0, \frac{\Delta}{3}]^{d/3}$ be a vector. Let $a \in [\frac{1}{2} - \Delta, \frac{1}{2} + \Delta]^d$ be defined as in Definition \ref{defn:A}. That is, let 
\begin{itemize}
	\item $a_i = \frac{1}{2} + t_i$.
	\item $a_{i + d/3} = \frac{1}{2} + \frac{2\Delta}{3} - g_1(t_i)$.
	\item $a_{i + 2d/3} = \frac{1}{2} - \frac{2\Delta}{3} - g_2(t_i)$. 
\end{itemize}
Then we define $L(S|t)$ as the likelihood of observing the set $S$ from $\D_a^n$. In particular, for any measurable region of points $R \subseteq (\R^d \times \{\pm 1\})^n$, we have that $$\mathbb{P}_{S \sim \D_a^n}[S \in R] = \int_{x \in R}L(x|t)dx.$$
\end{defn}

\begin{lem}\label{lem:binary}
Let $S \subset \R^d \times \{\pm 1\}$ be a set with $n$ points. Then for all $t \in [0, \frac{\dd}{3}]^{d/3}$, $$L(S|t) \in \left\{0, \left(\frac{1}{(d - 2\Lambda)||u||_2}\right)^n\right\},$$ where $\Lambda$ is as defined in Definition \ref{defn:big_lambda} and $L(S|t)$ is as defined in Definition \ref{defn:L(S|t)}. 
\end{lem}

\begin{proof}
Let $\D_a$ be an arbitrary distribution in $\A$. Observe that $\D_a$ is uniform over the set of all points in its support. Thus for every point in its support, we have that the likelihood $L(x|t)$ satisfies $L(x|t) = \frac{1}{(d - 2\Lambda)||u||_2}$. 

Taking the product of this over all points in $S$, we get the desired result. Note that if $S$ contains some point not in the support of $\D_a$, then the likelihood becomes $0$, since the likelihood of observing some point not in the support of $\D_a$ is $0$.
\end{proof}


\begin{defn}\label{defn:permissible_set}
For any dataset $S$, let $P_S$ denote the set of all ``permissible" $t$, that is $t \in [0, \frac{\dd}{3}]^d$ such that $L(S|t) \neq 0$. Formally, $$P_S = \{t: L(S|t) >0\}.$$
\end{defn}

We now fully characterize $P_S$ when $S$ is drawn from some $\D \sim \A$.

\begin{lem}\label{lem:intervals}
Fix $n > 0$. For all $\D \sim \A$ and $S \sim \D^n$, there exist intervals (possibly open, closed, half open) $I_1^S, I_2^S, \dots, I_{d/3}^S \subseteq [0, \frac{\dd}{3}]$ such that $P_S = \prod_1^{d/3} I_i^S$.
\end{lem}

\begin{proof}
Let $S = \{(x_1, y_1), (x_2, y_2), \dots, (x_n, y_n)\}$. Since $S \sim \D^n$, we see that for $1 \leq j \leq n$, $x_j$ must satisfy $x_j \in [v_i, v_i + u]$ for some $1 \leq j \leq d$. Using this, for $1 \leq i \leq d$ let $$s_i^- = \argmax_{\{x_j: x_j \in [v_i, v_i + u], y_j = -1\}} ||x_j - v_i||_2,$$ and $$s_i^+ = \argmax_{\{x_j: x_j \in [v_i, v_i + u], y_j = +1\}} ||x_j - (v_i+ u)||_2.$$ $s_i^-$ and $s_i^+$ can be thought of as the points from $S$ on segment $[v_i, v_i + u]$ that are closest to each other and labeled as $-$ and $+$ respectively. As a default, if no such points exist, we set $s_i^- = v_i$ and $s_i^+ = v_i + u$. 

Next, consider any $t \in [0, \frac{\Delta}{3}]^{d/3}$, let $a \in [\frac{1}{2} - \Delta, \frac{1}{2} + \Delta]^d$ be defined as in Definition \ref{defn:A}. That is, let 
\begin{itemize}
	\item $a_i = \frac{1}{2} + t_i$.
	\item $a_{i + d/3} = \frac{1}{2} + \frac{2\Delta}{3} - g_1(t_i)$.
	\item $a_{i + 2d/3} = \frac{1}{2} - \frac{2\Delta}{3} - g_2(t_i)$. 
\end{itemize}
The key idea of this lemma is that $t \in P_S$ (i.e. $L(S|t) > 0$) if and only if for all $1 \leq i \leq d$, $$[v_i + (a_i - \Lambda)u, v_i + (a_i + \Lambda)u] \subseteq (s_i^-, s_i^+).$$  To see this, observe that if the claim above holds, then we must have that $s_i^- \in [v_i, v_i + (a_i - \Lambda)u)$ and $s_i^+ \in (v_i + (a_i + \Lambda)u, v_i + u]$, and it consequently follows that all points in $S$ are elements of the support of $\D_a$ (Definition \ref{def:w_dist}), as all other points in $S$ are ``further" from the interval $[v_i + (a_i - \Lambda)u, v_i + (a_i + \Lambda)u]$ than the points $s_i^+$ and $s_i^-$. Conversely, if $L(S|t) > 0$, we must have that $S \subseteq supp(\D_a)$, which immediately translates to the statement above. Thus, it suffices to find all $t$ such that this condition holds.

To do this, observe that the interval $[v_i + (a_i - \Lambda)u, v_i + (a_i + \Lambda)u]$ is a line segment of length $2\Lambda||u||_2$ that is centered at the point $v_i + a_iu$. Thus, in order for this to be a sub-segment of $(s_i^-, s_i^+)$, we only need that $a_i$ satisfy $v_i + a_iu \in (s_i^- + \Lambda u, s_i^- - \Lambda u)$. This condition is equivalent to the condition that $a_i \in J_i^S$ for some open interval $J_i^S \subseteq [0, 1]$, where $J_i^S$ is only dependent on $s_i^-, s_i^+$ and $\Lambda$ (which is a constant). In summary, there exist interval $J_1^S, J_2^S, \dots, J_d^S$ such that $t \in P_S$ if and only if $a_i \in J_i^S$ for $1 \leq i \leq d$.

Finally, note that for $1 \leq i \leq d/3$, $a_i, a_{i+d/3}, a_{i + 2d/3}$ are all functions of $t_i$, and moreover these functions are $1$-lipschitz, and monotonic. As a consequence, by taking the intersections of the pre-images of these functions, we find that this condition holds if and only if $t_i \in I_i^S$ where $I_i^S$ is some interval that is a subset of $[0, \frac{\Delta}{3}]^{d/3}$. This proves the claim.
\end{proof}

\begin{cor}\label{cor:posterior}
For any $S \sim \D$ where $\D \sim \A$, let $I_i^S$ be defined as in Lemma \ref{lem:intervals} for $1 \leq i \leq d/3$. Then the posterior distribution $t|S$ is equal to the uniform distribution over the set $\prod_{1 \leq i \leq d/3} I_i^S$, where $t_i$ is sampled from $I_i^S$. 
\end{cor}

\begin{proof}
First, recall that our prior on $t$ is $\U([0, \frac{\Delta}{3}]^d)$, where $\U$ denotes the uniform distribution. By Lemma \ref{lem:binary}, we see that for all $t \in P_S$, $L(S|t) = \left(\frac{1}{(d - 2\Lambda)||u||_2}\right)^n$, and for all other $t$, $L(S|t) = 0$. Furthermore, by Lemma \ref{lem:intervals}, we see that $P_S = \prod_1^{1 \leq i \leq d/3} I_i^S$. Thus, applying Bayes rules gives the desired result. 
\end{proof}

We conclude this section by lower bounding the expected length of the interval $I_i^S$, denoted $\ell(I_i^S)$. 
\begin{lem}\label{lem:expected_length}
For an interval $(c, d) \subset \R$, we let its length, denoted $\ell((c,d))$ be defined as $\ell((c,d)) = d - c$. Then for $1 \leq k \leq d/3$, the expected length (taken over $\D_a \sim \A$ and $S \sim \D_a^n$) of the interval $I_k^S$ is at least $\Omega(\frac{d}{n})$. That is, $$\E_{\D_a \sim \A}\E_{S \sim \D_a^n}[\ell(I_k^S)]] \geq \Omega(\frac{d}{n}).$$
\end{lem}

\begin{proof}
Fix any $\D_{a^*} \sim \Pi$, and let $t^*$ denote the value of $t$ used to generate $a$ (as in Definition \ref{defn:A}). We will show that $\E_{S \sim \D_{a^*}^n}[\ell(I_k^S)]] \geq \Omega(\frac{d}{n}),$ for all $1 \leq k \leq d/3$. We begin by explicitly computing the interval $I_k^S$. 

Fix $1 \leq k \leq d/3$. Then $t_k* \in [0, \frac{\Delta}{3}]$. Assume that $t_k^* > 0$; we will handle the case $t_k^* = 0$ separately. Recall from the proof of Lemma \ref{lem:intervals} that for $1 \leq i \leq d$, we defined $$s_i^- = \argmax_{\{x_j: x_j \in [v_i, v_i + u], y_j = -1\}} ||x_j - v_i||_2,$$ and $$s_i^+ = \argmax_{\{x_j: x_j \in [v_i, v_i + u], y_j = +1\}} ||x_j - (v_i+ u)||_2.$$ for $1 \leq i \leq d$. 

Next let $t \in [0, \frac{\Delta}{3}]^{d/3}$ be a vector, and let $a \in [\frac{1}{2} - \Delta, \frac{1}{2} + \Delta]^d$ be defined as $a_k = \frac{1}{2} + t_k$, $a_{k + d/3} = \frac{1}{2} + \frac{2\Delta}{3} - g_1(t_k)$ and $a_{k + 2d/3} = \frac{1}{2} - \frac{2\Delta}{3} - g_2(t_k)$, for $1 \leq k \leq d/3$. Note that $g_1, g_2$ are the functions defined in Definition \ref{defn:g_1_and_g_2}.

As we argued in the proof of Lemma \ref{lem:intervals}, it then follows that $t_k \in I_k^S$ if and only if $$[v_i + (a_i - \Lambda)u, v_i + (a_i + \Lambda)u] \subseteq (s_i^-, s_i^+),$$ for $i = k, k+d/3, k+2d/3$. Finally, as we did in Lemma \ref{lem:intervals}, for each $1 \leq i \leq d$, we define intervals $J_i^S \subseteq [\frac{1}{2} - \Delta, \frac{1}{2} + \Delta]$ such that $a_i \in J_i^S$ if and only if $[v_i + (a_i - \Lambda)u, v_i + (a_i + \Lambda)u] \subseteq (s_i^-, s_i^+)$. 


We now have the following three claims.

\paragraph{Claim 1:} Let $\alpha = \min \left(\frac{||s_k^- -  (v_k + (a_k^* - \Lambda)u)||_2}{||u||_2}, t_k^*\right)$. If $t_k \in (t_k^* - \alpha, t_k^*]$, then $$[v_k + (a_k - \Lambda)u, v_k + (a_k + \Lambda)u] \subseteq (s_k^-, s_k^+).$$ 

\textit{Proof: } First, observe that since $s_k^+$ and $s_k^-$ were sampled from $\D_{a^*}$, it follows that $$[v_k + (a_k^* - \Lambda)u, v_k + (a_k^* + \Lambda)u] \subseteq (s_i^-, s_i^+).$$ Consider any $t_k \in [t_k^* - \alpha, t_k^*]$. Then substituting the definitions of $a_k, a_k^*$ imply that $a_k \in [a_k^* - \alpha, a_k^*]$. Because of this, it follows that 
\begin{equation*}
\begin{split}
||(v_k + (a_k - \Lambda)u) - (v_k + (a_k^* - \Lambda)u)||_2 &= ||(a_k - a_k^*)u||_2 \\
&< \alpha||u||_2 \\
&\leq ||s_k^- - (v_k + (a_k^* - \Lambda)u)||_2,
\end{split}
\end{equation*}
which implies that $v_k + (a_k - \Lambda)u \in (s_i^-, v_k + (a_k^* - \Lambda)u]$. Furthermore, the fact that $a_k \leq a_k^*$ implies that $v_k + (a_k + \Lambda)u \in (v_k + (a_k - \Lambda)u, v_k + (a_k^* + \Lambda)u]$. 

Together, these observations imply the desired result, as it follows that $$[v_k + (a_k - \Lambda)u, v_k + (a_k + \Lambda)u] \subset (s_k^-, v_k + (a_k^* + \Lambda)u] \subset (s_k^-, s_k^+).$$ $\blacksquare$

\paragraph{Claim 2:} Let $\beta = \min \left(\frac{||s_{k+d/3}^+ -  (v_{k+d/3} + (a_{k+d/3}^* + \Lambda)u)||_2}{||u||_2}, g_1(t_{k}^*)\right)$. If $t_k \in (g_1^{-1}(g_1(t_k^*) - \beta), t_k^*]$, then $$[v_{k+d/3} + (a_{k+d/3} - \Lambda)u, v_k + (a_{k+d/3} + \Lambda)u] \subseteq (s_{k+d/3}^-, s_{k+d/3}^+).$$ 

\textit{Proof: } First, we observe that $\beta$ is well defined since $g_1$ is a monotonic $1$-Lipschitz function, and consequently has an inverse. Next, we also see that $0 \leq g_1(t_k^*) - g_1(t_k) \leq \beta$. Substituting the definitions of $a_k^*, a_k$, it follows that $0 \leq a_k - a_k^* \leq \beta$ (notice the order switch). At this point, we can apply the same argument as in Claim 1 to get the desired result.  $\blacksquare$.

\paragraph{Claim 3:} Let $\tau = \min \left(\frac{||s_{k+2d/3}^+ -  (v_{k+2d/3} + (a_{k+2d/3}^* + \Lambda)u)||_2}{||u||_2}, g_2(t_k^*)\right)$. If $t_k \in (g_2^{-1}(g_2(t_k^*) - \tau), t_k^*]$, then $$[v_{k+2d/3} + (a_{k+2d/3} - \Lambda)u, v_{k+2d/3} + (a_{k+2d/3} + \Lambda)u] \subseteq (s_{k+2d/3}^-, s_{k+2d/3}^+).$$  

\textit{Proof: }  Completely analogous to Claim 2. $\blacksquare$.

Combining these claims, we see that $$t_k \in (t_k^* - \alpha, t_k^*] \cap (g_1^{-1}(g_1(t_k^*) - \beta), t_k^*] \cap  (g_2^{-1}(g_2(t_k^*) - \tau), t_k^*] \implies t_k \in I_k^S.$$ Since these three intervals all have an endpoint in $t_k^*$, it follows that there is an interval with length $\eta$ that is a subset of $I_k^S$, where $$\eta = \min(\ell((t_k^* - \alpha, t_k^*])), \ell((g_1^{-1}(g_1(t_k^*) - \beta), t_k^*]), \ell((g_2^{-1}(g_2(t_k^*) - \tau), t_k^*])).$$ However, by substituting that $g_1, g_2$ are $1$-Lipschitz, we see that $\ell((g_1^{-1}(g_1(t_k^*) - \beta), t_k^*]) \geq \beta$ and $\ell((g_2^{-1}(g_2(t_k^*) - \tau), t_k^*])) \geq \tau$. Thus, it follows that $$\ell(I_k^S) \geq \eta \geq \min(\alpha, \beta, \tau).$$ Thus it suffices to show that $\E_{S \sim \D_{a^*}}[\min(\alpha, \beta, \tau)] \geq \Omega(\frac{d}{n})$. 

To do this, observe that
\begin{itemize}
	\item $\alpha||u||_2$ is the distance from the closest point labeled $-$ on the segment $[v_k, v_k + u]$ to the point $v_k + (a_k^* - \Lambda)u$
	\item  $\beta||u||_2$ is the distance from the closest point labeled $+$ on the segment $[v_{k+d/3}, v_{k+d/3} + u]$ to the point $v_{k+d/3} + (\Lambda + a_{k+d/3}^*)u$
	\item $\tau||u||_2$is the distance from the closest point labeled $+$ on the segment $[v_{k + 2d/3}, v_{k + 2d/3} + u]$ to the point $v_{k+2d/3} + (\Lambda + a_{k+2d/3}^*)u$.
\end{itemize}

Finally, it is not difficult to see that for sufficiently large $n$, with high probability each of these distances will be $\Omega(\frac{d}{n})$. This is because with high probability there will be $\Theta(\frac{n}{d})$ points on each of the respective line segments, and we are considering the closest point among them to some reference point. Thus, it follows that with high probability $\E_{S \sim \D_{a^*}}[\min(\alpha, \beta, tau)] \geq \Omega(\frac{d}{n}),$ as desired.
\end{proof}

\subsubsection{Putting it all together, the proof}\label{subsubsec:proof}

We prove the following key lemma, which directly implies Theorem \ref{thm:lower}.

\begin{lem}\label{lem:lower_bound}
Let $M$ be any learning algorithm that outputs a linear classifier. For any training sample of points $S = \{(x_1, y_1), (x_2, y_2), \dots, (x_n, y_n)\}$, we let $M_S$ denote the classifier learned by $M$ from $S \sim \D$. Then it follows that $$\E_{\D \sim \A} \E_{S \sim \D^n}[\L_r(M_S, \D)]] \geq \Omega(\frac{d}{n}).$$ 
\end{lem}

\begin{proof}
Let $\F_n$ denote the distribution over $(\R^d \times \{\pm 1\})^n$ defined as the composition $\D \sim \A$ and $S \sim \D^n$. That is, $S \sim \F_n$ follows the same distribution as $\D \sim \A, S \sim \D^n$. Then we can write the expectation above as 
\begin{equation*}
\begin{split}
\E_{\D \sim \A} \E_{S \sim \D^n}[\L_r(A_S, \D)]] = \E_{S \sim \F_n} \E_{\D \sim (\A|S)}[\L_r(M_S, \D)]],
\end{split}
\end{equation*}
where $\A|S$ denotes the posterior distribution of $\D$ conditioned on observing $S$. First, fix any such $S$. We will bound $\E_{\D \sim (\A|S)}[\L_r(M_S, \D)].$ First, by reparametrizing in terms of $t \in [0,\frac{\dd}{3}]^{d/3}$ and applying Corollary \ref{cor:posterior}, we have that $$\E_{D \sim (\A|S)}[\L_r(M_S, \D)] = \E_{t_1 \sim \U(I_1^S)}[\dots [\E_{t_n \sim \U(I_{d/3})}[\L_r(M_S, \D_a)]\dots ],$$ where $I_1^S, I_2^S, \dots, I_{d/3}^S \subset [0, \frac{\dd}{3}]$ are the intervals defined in Lemma \ref{lem:intervals}, and $a$ is defined as in Definition \ref{defn:A}. 

Next, let $b \in [0, 1]^d$ be such that $M_S = f_{w^b, 1}$, where $w^b$ is defined as in Definition \ref{def:normal_vector}. Then it follows from Corollary \ref{cor:l_1distancebound} that 
\begin{equation*}
\begin{split}
\L_r(M_S, \D_a)] &\geq \frac{1}{20d}\sum_1^d |a_i - b_i| \\
&\geq \frac{1}{20d}\sum_1^{d/3} |\frac{1}{2} + t_i - b_i|
\end{split}
\end{equation*}
with the last inequality coming from substituting the definition of $a_i$ and (and ignoring $a_i$ for $i > d/3$). We now take the expectation of this inequality over $t_1, t_2, \dots, t_{d/3}$. To do so, observe that by simple algebra, $\E_{t_i \sim \U(I_i^S)} |\frac{1}{2} + t_i - b_i| \geq \frac{\ell(I_i^S)}{4}$. Substituting this, we see that $$E_{t_1 \sim \U(I_1^S)}[\dots [\E_{t_n \sim \U(I_{d/3}^S)}[\L_r(M_S, \D_a)]\dots ] \geq \frac{1}{80d} \sum_{i=1}^{d/3} \ell(I_i^S).$$ Finally, by taking expectations over $S \sim \F_n$, we see that 
\begin{equation*}
\begin{split}
\E_{\D \sim \A} \E_{S \sim \D^n}[\L_r(A_S, \D)]] &= \E_{S \sim \F_n} \E_{\D \sim (\A|S)}[\L_r(M_S, \D)]] \\
&\geq \E_{S \sim \F_n} \frac{1}{80d} \sum_{i=1}^{d/3} \ell(I_i^S) \\
&= \frac{1}{80d}\sum_1^{d/3}\E_{S \sim \F}[\ell(I_i^S)] \\
&= \frac{1}{80d}\sum_1^{d/3}\E_{\D \sim \A}\E_{S \sim \D^n}[\ell(I_i^S)] \\
&\geq \frac{1}{80d}\sum_1^{d/3}\Omega(\frac{d}{n}) = \Omega(\frac{d}{n}),
\end{split}
\end{equation*}
where the last step follows from Lemma \ref{lem:expected_length}. 
\end{proof}

Finally, we can prove Theorem \ref{thm:lower}.

\begin{proof}
(Theorem \ref{thm:lower}). First, by Lemmas \ref{lem:separation} and \ref{lem:large_margin}, we see that $\A \subseteq \F_{r, \rho}$ (provided $\rho > 10$). Next, by Lemma \ref{lem:lower_bound}, for any $n$ there must exists some $\D \sim \A$ such that $\E_{S \sim \D^n}[\L_r(M_S, \D)] \geq \Omega(\frac{d}{n})$. Thus selecting this distribution suffices. This concludes the proof.
\end{proof}

\section{Proofs for Algorithm \ref{alg:upper_bound}}\label{sec:upper_bound_details}

This section is divided into 2 parts. In section \ref{sec:upper_bound_origin}, we show that for the case in which our data distribution $\D$ is linearly $r$-separated by some hyperplane through the origin, the desired error bound holds. That is, we prove Theorem \ref{thm:upper_bound} under this assumption.

Next, in section \ref{sec:upper_bound_general}, we show how to generalize Algorithm \ref{alg:upper_bound} to arbitrary linearly $r$-separated distributions, and subsequently prove Theorem \ref{thm:upper_bound} in the general case.

\subsection{Origin Case}\label{sec:upper_bound_origin}

We begin by precisely stating the conditions required in the ``origin" case. We assume the following properties hold for our data distribution $\D$. We let $S_r^+$ and $S_r^-$ be defined as in section \ref{sec:upper_bound}.

\begin{enumerate}
	\item There exists $R > 0$ such that for all $x \in S_r^+ \cup S_r^-$, $||x||_2 \leq R$.
	\item There exists a unit vector $u \in \R^d$ and $\gamma_r > 0$ such that 
	\begin{itemize}
		\item $\L_r(f_{u, 0}, \D) = 0$, where $f_{u, 0}$ denotes the linear classifier with decision boundary $\langle u, x \rangle = 0$. 
		\item $S_r^+ \cup S_r^-$ has distance at least $\gamma_r$ from the decision boundary of $f_w$. That is, $||S_r^+ \cup S_r^- - H_{u, 0}||_2 \geq \gamma_r$.
	\end{itemize}
	\item By the previous conditions, it follows that $\langle u, yx' \rangle \geq \gamma_r$ for all $(x,y) \sim \D$, and $x' \in B_p(x, r)$. This is because $u$ is a unit vector. 
\end{enumerate}

Next, before analyzing Algorithm \ref{alg:upper_bound}, we will first give a slight modification of the algorithm that lends itself to better analysis. The only difference is that in this new algorithm, we first randomly sample $k \sim \{1, 2, \dots, n\}$, and then only train on the first $r$ data-points of our training sample.

\begin{algorithm}[H]
   \caption{Modified-Adversarial-Perceptron}
   \label{alg:upper_bound_modified}

    \textbf{Input}:  $S = \{(x_1, y_1), \dots, (x_n, y_n)\} \sim \D^n,$
    
    $w \leftarrow 0$ 
    
    $k \sim \U(\{0, 1, 2, \dots, n\})$
    
    \For{$i = 1 \dots k$}{
    	$z = \argmin_{||z - x_i||_p \leq r}  y_i\langle w, z \rangle$ 
        \If{$\langle w, y_iz \rangle \leq 0$}{
            $w \leftarrow w + y_iz$
        }
     }
     
    return $f_{w, 0}$
\end{algorithm}

We will show that Algorithm \ref{alg:upper_bound_modified} satisfies the guarantees of Theorem \ref{thm:upper_bound_origin}. We begin with the following, key lemma.

\begin{lem}\label{lem:update_count}
Under the assumptions above about $\D$, Algorithm \ref{alg:upper_bound_modified} makes at most $\frac{R^2}{\gamma_r^2}$ updates to $w$.
\end{lem}

\begin{proof}
Let $w_t$ denote our weight vector after we make $t$ updates. Observe that $w_t = w_{t-1} + y_tx_t + z'$ where $(x_t, y_t)$ denotes the point we made a mistake on, and $z' = \argmin_{|z|_p \leq r} \langle w, z \rangle$. Letting $x_t' = x_t + y_tz'$, we see that $w_t = w_{t-1} + y_tx_t'$. Now the key observation is that $(x_t', y_t) \in S_r^+ \cup S_r^-$, and as a result, it follows that $\langle u, y_tx_t' \rangle \geq \gamma_r$. Using this, we see that
\begin{equation*}
\begin{split}
\langle u, w_t \rangle &= \langle u, w_{t-1} + y_tx_t' \rangle \\
&= \langle u, w_{t-1} \rangle + \langle u, y_tx_t' \rangle \\
&\geq \langle u, w_{t-1} \rangle + \gamma_r.
\end{split}
\end{equation*}
Thus, by a simple proof by induction, we see that $\langle w_t, u \rangle \geq t\gamma_r$. 

Next, observe that we must have $\langle w_{t-1}, y_tx_t' \rangle \leq 0$. This is because $w_{t-1}$ must missclassify $(x_t', y_t)$ (thus failing to be astute at $(x_t, y_t)$) in order for it to be updated. Substituting this, we see that
\begin{equation*}
\begin{split}
||w_t||_2 &= \sqrt{\langle w_t, w_t \rangle} \\
&= \sqrt{\langle w_{t-1} + x_t'y_t, w_{t-1} + x_t'y_ \rangle} \\
&= \sqrt{\langle w_{t-1}, w_{t-1} \rangle + 2\langle w_{t-1}, x_t'y_t \rangle + \langle x_t', x_t' \rangle} \\
&\leq \sqrt{||w_{t-1}||_2^2 +  0 + R^2},
\end{split}
\end{equation*}
with the last inequality holding since $|x_t'|_2 \leq R$. Thus, by a simple proof by induction, we see that $||w_t||_2 \leq R\sqrt{t}$. 

Finally, since $u$ is a unit vector, it follows that $||w_t||_2 \geq \langle w_t, u$. Substituting our inequalities, we find that $R\sqrt{t} \geq \gamma_r t$ which implies that $t \leq \frac{R^2}{\gamma_r^2}$. Since $t$ is the number of mistakes we make, the result follows. 
\end{proof}

\begin{lem}\label{thm:upper_bound_origin}
Let $\D$ be a distribution with the assumptions above. For any $S \sim \D^n$, let $f_S$ denote the classifier learned by Algorithm \ref{alg:upper_bound_modified}. Then $$\E_{S \sim \D^n}\L_r(f_S, \D) \leq \frac{R^2}{\gamma_r^2 (n+1)}.$$
\end{lem}

This Theorem directly follows from the classic online to offline result (Theorem 3 of \cite{Freund99}). For completeness, we include a proof in our context.

\begin{proof}
Fix any $n$ and consider running Algorithm \ref{alg:upper_bound_modified} on $S \sim \D^n$. Let $L_t$ denote the expected robust loss of our classifier conditioning on $k = t$, and let $L^*$ denote the expected overall loss of our classifier. It follows that $$\E_{S \sim \D^n} L^* = \frac{1}{n+1}\sum_{t=0}^n \E_{S \sim \D^n}[L^*|k = t] = \frac{1}{n+1}\sum_{t=0}^n \E_{S \sim \D^n}[L_t].$$

Next, let $T \sim \D^{n+1}$ be a separate i.i.d drawn sample, and suppose we run the adversarial perceptron algorithm on the entirety of $T$ (i.e. rung Algorithm \ref{alg:upper_bound_modified} on $T$ by setting $k = n+1$). For $1 \leq t \leq n + 1$, let $X_t$ be the indicator variable for whether the $t$th point in $T$ requires an update on $w$ (i.e. the classifier is not astute at $w$). There are two important observations to make.

First, we have that $\E_{T \sim \D^{n+1}}[X_t] = \E_{S \sim \D^n}[L_{t-1}]$. This is because $X_t$ is an indicator variable for a classifier trained on precisely $t-1$ i.i.d training examples lacking astuteness for a randomly drawn point from $\D$. Second, we have that $\sum_{t = 1}^{n+1} X_t \leq \frac{R^2}{\gamma_r^2}$. This is because each $\sum X_t$ is precisely the number of updates that perceptron makes on $T$, which is bounded by Lemma \ref{lem:update_count}. By combining these two observations, we see that 
\begin{equation*}
\begin{split}
\E_{S \sim \D^n}[L^*] &= \frac{1}{n+1}\sum_{t=0}^n \E_{S \sim \D^n}[L_t] \\
&= \frac{1}{n+1}\sum_{t=0}^n \E_{T \sim \D^{n+1}}[X_{t+1}] \\
&= \frac{1}{n+1}\E_{T \sim \D^{n+1}}[\sum_{t = 1}^{n+1} X_{t}] \\
&\leq \frac{R^2}{\gamma_r^2(n+1)},
\end{split}
\end{equation*}
as desired. 
\end{proof}

\subsection{General Case}\label{sec:upper_bound_general}

In general case, we no longer assume that the optimal classifier $f_{u, b}$ passes through the origin. To account for this, we will need to first adapt our algorithm. The basic idea is to simply append a $1$ to the vectors $x$ and increase the dimension $d$ by $1$. We are then left with solving a $d+1$ dimensional problem in which the data is once-again separated by a hyperplane passing through the origin. 

We begin with two useful sets of notation.


\begin{defn}
We use the following notation:
\begin{itemize}
	\item For any $x \in \R^d$ and $R \in \R$, we let $x|R \in \R^{d+1}$ denote the $d+1$ dimensional vector obtained by appending the value $R$ to $x$. 
	\item For $w \in \R^{d+1}$, let $||w||_q^*$ denote the $\ell_q$ norm of the first $d$ coordinates of $w$.
	\item For $x \in \R^{d+1}$, let $B_p^*(x, r)$ denote all $z \in \R^{d+1}$ such that $||z - x||_p \leq r$ and such that $z$ and $x$ both share the same last coordinate.
	\item For $S = \{(x_1, y_1), \dots, (x_n, y_n)\} \subset \R^{d+1} \times \{\pm 1\}$, let $R_S$ denote $\max_{i \neq j} ||x_i - x_j||_2$. 
\end{itemize}


 
\end{defn}

We now propose the following modified version of Algorithm \ref{alg:upper_bound}, that is capable of handling any dataset, including ones that aren't separated by a hyperplane through the origin.

\begin{algorithm}[H]
    \caption{General-Adversarial-Perceptron}
    \label{alg:gen_upper_bound}
    
    \textbf{Input}:  $S = \{(x_1, y_1), \dots, (x_n, y_n)\} \sim \D^n,$
    
    $x_i' \leftarrow x_i - x_1$. 
    
    $R_S = diam_2(S)$
    
    $w \leftarrow 0 \in \R^{d+1}$
    
    Randomly permute $S$
    
    Randomly choose $k \in \{1, 2, 3, \dots, n\}$.
     
    \For{$t = 1 \dots k$}{   
        \If{$\langle w, y_t(x_t|R_S) \rangle \leq r||w||_q^*$}{
        
            $z' = \argmin_{|z|_p \leq r} \langle w, z|0 \rangle$
            
            $w \leftarrow w + y_t(x_t|R_S) + z'|0$
        }
     }
     
    $w^* \leftarrow$ first $d$ coordinates of $w$
    
    $b \leftarrow$ the last element of $w$
    
    Return $f_{w^*, \langle w^*, x_1 \rangle -bR_S}$

\end{algorithm}

The basic idea of the algorithm is to first translate $S$ so that one point is the origin, and then append $R_S$ to every vector in $S$ so that each vector is now $d+1$ dimensional. After doing this, we apply Algorithm \ref{alg:upper_bound} as before with one important difference: for our adversarial attacks, we make sure to not change the last coordinate. 

We now show that this algorithm has a similar performance to our old algorithm. We first prove a helpful lemma.

\begin{lem}\label{lem:general_upper_bound}
Let $\D$ be any linearly $r$-separated distribution, and let $S \sim \D^n$ such that $S$ has positively and negatively labeled examples. Let $x_i' = x_i - x_1$ for $1 \leq i \leq n$. Then the following hold.
\begin{itemize}
	\item There exists a unit vector $u \in \R^{d+1}$ such that for all $(x_i, y_i) \in S$, $\min_{z \in B_p^*(x_i')} \langle u, y_i(z|R_S) \rangle \geq \frac{\gamma_r(\D)}{\sqrt{2}}.$
	\item For all $(x_i, y_i) \in S$, $||x_i'|R_S||_2 \leq \sqrt{2}diam_2(\D)$. 
\end{itemize}
\end{lem}

\begin{proof}
Without loss of generality, we will assume $x_1 = 0$ so that we can safely ignore the differences between $x_i'$ and $x_i$. Since $\D$ is $r$-separated, there exist $w, b$ (with $w$ a unit vector) such that $$\langle w, zy \rangle \geq by + \gamma_r(\D),$$ for all $(x,y) \sim \D$ and $z \in B_p(x, r)$. Furthermore, since $x_1 = 0$, it follows that $||x||_2 \leq \diam_2(\D)$ for all $(x, y) \sim \D$. This immediately implies that $||x_i|R_S||_2 \leq \sqrt{\diam_2(\D)^2 + R_S^2} \leq \sqrt{2}\diam_2(\D)$, yielding the second part of the lemma.

For the first part, observe that we can rearrange the equation above, we see that $$\langle w|-\frac{b}{R_S}, zy | R_S \rangle \geq \gamma_r(\D).$$ The key observation is that the first equation implies that $b \leq R_S$. This is because $S$ contains positively and negatively labeled examples, and consequently $\langle w, x_i \rangle \geq b + \gamma_r(\D) > b$ for some $x_i$ such that $|x_i| = R_S$. Thus, it follows that the unit vector $u = \frac{w|\frac{-b}{R_S}}{\sqrt{1 + b^2/R_S^2}}$ has the desired property, by observing that $\sqrt{1 + b^2/R_S^2} \leq \sqrt{2}$. 
\end{proof}

Lemma \ref{lem:general_upper_bound} allows us to analyze the performance of Algorithm \ref{alg:gen_upper_bound}. The basic idea is that our performance on the transformed data in $\R^{d+1}$ is isomorphic to its performance on the data in $\R^d$. As a consequence, we can apply the same argument as in Theorem \ref{thm:upper_bound_origin} to get a bound on the error estimate. However, this bound must be given in terms of the diameter and robust margin of the \textit{transformed data}: quantities that have been bounded in Lemma \ref{lem:general_upper_bound}. Thus, putting this all together, Theorem \ref{thm:upper_bound} follows.


\section{Details for Kernel Algorithm}\label{sec:kernel_appendix}

Next, we find analogs of linear $r$-separability and the robust margin when considering kernels. First, we define an embedding function.

\begin{defn}\label{defn:embedding_function}
Let $K: \R^d \times \R^d \to \R^+$ be a kernel similarity function. Then there exists a Hilbert space $H$ and map $\phi: \R^d \to H$ such that for all $x_1, x_2 \in \R^d,$ we have $$K(x_1, x_2) = \langle \phi(x_1), \phi(x_2) \rangle.$$ We call $\phi$ the \textbf{embedding function} and $H$ the \textbf{embedding space}.
\end{defn}

The key idea of this section is that Kenrel classifiers correspond to linear classifiers in embedded space. This is the essence of the ``kernel trick." Formally, we have the following, well-known theorem. 

\begin{thm}\label{thm:kernel_trick}
Let $T = \{(x_1, y_1), \dots, (x_m, y_m)\} \subset \R^d \times \{\pm 1\}$ be a set of labeled points. Let $K:\R^d \times \R^d \to \R^+$ be a kernel similarity function, and $\alpha \in \R^m$ be a vector of $m$ real numbers. Then for all $x \in \R^d$, we have that $$\sum_{i = 1}^m \alpha_iy_iK(x_i, x) = \big \langle \sum_{i= 1}^m \alpha_iy_i\phi(x_i), \phi(x) \big \rangle.$$ Because of this, if we let $w = \sum_{i=1}^m \alpha_iy_i\phi(x_i)$, then the kernel classifier $f_{T, \alpha}^k$ satisfies $f_{T, \alpha}^k(x) = f_{w, 0}(\phi(x))$, where the latter classifier is the linear classifier in $H$ with weight vector $w$. 
\end{thm}

The main idea behind Algorithm \ref{alg:upper_bound_kernel}, is that it corresponds to running Algorithm \ref{alg:upper_bound} inside the embedded space of the kernel $K$. In particular, the kernel-perceptron update step precisely corresponds to the dual-form of the perceptron-update step inside embedded space. It follows from Theorem \ref{thm:kernel_trick} that the following algorithm is identical to Algorithm \ref{alg:upper_bound_kernel}.

\begin{algorithm}[H]
   \caption{Adversarial-Kernel-Perceptron}
   \label{alg:upper_bound_kernel_nice}

    \textbf{Input}:  $S = \{(x_1, y_1), \dots, (x_n, y_n)\} \sim \D^n,$ Similarity function, $K$,
    
    $w \leftarrow 0$
    
    \For{$i = 1 \dots n$}{
    
    	$z = \argmin_{||z - x||_p \leq r}  y_i\langle w, \phi(z) \rangle$ 
    	
        \If{$\langle y_iw, \phi(z) \rangle \leq 0$}{
            $w = w + y_i\phi(z)$ 
        }
     }
    return $f_{w,0} \circ \phi$

\end{algorithm}

In particular, by comparing Algorithms \ref{alg:upper_bound_kernel} and \ref{alg:upper_bound_kernel_nice}, we have by Theorem \ref{thm:kernel_trick} that for all time steps $t$, $$w = \sum_{(z,y) \in T} y\phi(z).$$ Therefore, to analyze the performance of Algorithm \ref{alg:upper_bound_kernel}, it suffices to analyze Algorithm \ref{alg:upper_bound_kernel_nice}. However, we already have built to the tools for doing this: all of the results from Section \ref{sec:upper_bound_origin} apply to Algorithm \ref{alg:upper_bound_kernel_nice} since the only difference is replacing $\R^d$ with $H$, the embedding space of $K$. 

We now proceed by giving the corresponding assumptions on $\D$ needed for Theorem \ref{thm:upper_bound_kernel}. We begin by first defining $(K, r)$-separability and $K$-robust margin, $\gamma_{r, K}$, the Kernel analogs of linear $r$-separability (Definition \ref{defn:r_separability}) and the robust margin (Definition \ref{def:robust_margin}). 


\begin{defn}\label{defn:ker_r_separability}
For any $r > 0$, a distribution $\D$ over $\R^d \times \{\pm 1\}$ is $(K, r)$-\textbf{separable} if there exists a kernel classifier $f_{S, \alpha}^K$ such that $\L_r(f_{S, \alpha}^K, \D) = 0$.
\end{defn}

To define the $K$-robust margin, we will once again need the sets $S_r^+$ and $S_r^-$ defined in equation \ref{eqn:s_plus_s_minus} (top right of page 7). Recall that these sets denote the positively and negatively labeled elements from $supp(\D)$ \textit{including} all adversarial perturbations of those points. 

\begin{defn}\label{defn:k_rob_margin}
Let $\D$ be a $(K, r)$-separable distribution over $\R^d \times \{ \pm 1\}$. Then $\D$ has $K$-robust margin $\gamma_r$ if $\gamma_r$ is the largest real number such that there exists a kernel classifier $f_{T, \alpha}^K$, such that the following conditions hold.

\begin{enumerate}
	\item $\L_r(f_{T, \alpha}^K, \D) = 0$. 
	\item Let $\phi, H$ be the embedding function/space of $K$, let $w = \sum_{(z, y) \in T} y\phi(z)$, and let $H_w = \{z \in H, \langle z, w \rangle = 0\}$ be the decision boundary in $H$ of $f_{T, \alpha}^K$. Then for all $x \in S_r^+ \cup S_r^-$, $\phi(x)$ has $\ell_2$ distance at least $\gamma_r^K$ from $H_w$ inside $H$. That is, $$\inf_{x \in S_r^+ \cup S_r^-} \inf_{z \in H_w} \sqrt{\langle \phi(x) - z, \phi(x) - z \rangle} = \gamma_r^K.$$ 
\end{enumerate}
\end{defn}

We now state the main theorem giving the performance of Algorithm \ref{alg:upper_bound_kernel}. 

\begin{thm}
Let $\D$ be a distribution over $\R^d \times \{\pm 1\}$ such that the following conditions hold. 
\begin{enumerate}
	\item There exists $R > 0$ such that for all $x \in S_r^+ \cup S_r^-$, $\langle \phi(x), \phi(x) \rangle \leq R^2$.
	\item $\D$ is $K, r$-separable, and has $K$-robust margin $\gamma_r^K > 0$.
\end{enumerate}
Then for any $S \sim D^n$, if $f_{T, \alpha}^k$ denotes the classifier learned by Algorithm \ref{alg:upper_bound_kernel}, then $$\E_{S \sim \D^n}[\L_r(f_{T, \alpha}^k, \D)] = O\left(\frac{(\gamma_r^K)^2}{R^2(n+1)} \right).$$
\end{thm}

\begin{proof}
The key idea is to observe that Lemmas \ref{lem:update_count} and \ref{thm:upper_bound_origin} both directly translate from Algorithm \ref{alg:gen_upper_bound} to Algorithm \ref{alg:upper_bound_kernel_nice}. In particular, neither proof used the dimension, $d$, of $\R^d$, and consequently would equally apply to even an infinite dimensional Hilbet Space, $H$. Thus, the proof is completely analogous to the proof of Theorem \ref{thm:upper_bound_origin}.
\end{proof}







 
\graphicspath{{./chapters/chapter3/}}
\chapter{ }

\label{sec:stc priv appendix} 

\subsection{Privacy Mechanism}
We now describe in detail our instance of the exponential mechanism $\mname$. Recall from Definition \ref{def: exp mech} that the exponential mechanism samples candidate $f_i \in F$ with probability
\begin{align*}
	\Pr[\calM(x) = f_i] \propto \exp\big( \frac{\epsilon u(x, f_i)}{2 \Delta u} \big) \ .
\end{align*}
Thus, $\mname$ is fully defined by its utility function, which, as listed in Equation \eqref{eqn:utility}, is approximate Tukey Depth, 
\begin{align*}
u(x, f_i) = \tdappx_{S_x}(f_i) \quad.
\end{align*}
We now describe our approximation algorithm of Tukey Depth $\tdappx_{S_x}(f_i)$, which is an adaptation of the general median hypothesis algorithm proposed by \citet{median_hyp}. 

\begin{algorithm}
    \SetKwFunction{isOddNumber}{isOddNumber}
    % \SetKwInput{Input}{Input}
    % \SetKwInput{Output}{Output}
    \SetKwInOut{KwIn}{Input}
    \SetKwInOut{KwOut}{Output}

    \KwIn{$m$ candidates $F$, \\sentence embs. $S_x = (s_1, \dots, s_k)$,\\ number of projections $p$}
    \KwOut{probability of sampling each candidate $P_F = [P_{f_1}, \dots, P_{f_m}]$}
    
    $v_1, \dots, v_p \gets $ random vecs. on unit sphere 
    
    \tcp{Project all embeddings}
  
    \For{$i \in [k]$}{
    \For{$j \in [p]$}{
    $s_i^j \gets s_i^\intercal v_j$
    }
    }
    
    \For{$i \in [m]$}{
    \For{$j \in [p]$}{
    $f_i^j \gets f_i^\intercal v_j$
    
    \tcc{Compute depth of $f_i$ on projection $v_j$}
    
    $h_j(x,f_i) \gets \#\{s_l^j : s_l^j \geq  f_i^j, l \in [k]\}$
    
    $u_j(x,f_i) \gets -\big| h_j(x,f_i) - \frac{k}{2} \big|$ 
    }
    $u(x,f_i) \gets \max_{j \in [p]} u_j(x,f_i)$
    $\hat{P}_{f_i} \gets \exp(\epsilon u(x,f_i) / 2)$
    }
    
    $\Psi \gets \sum_{i=1}^{m} \hat{P}_{f_i}$
    
    \For{$i \in [m]$}{
    $P_{f_i} \gets \frac{1}{\Psi} \hat{P}_{f_i}$
    }
    
    \KwRet{$P_F$}
    \caption{$\mname$ compute probabilities}
    \label{alg:main alg}
\end{algorithm}

Note that we can precompute the projections on line 10. The runtime is $O(mkp)$: for each of $m$ candidates and on each of $p$ projections, we need to compute the scalar difference with $k$ sentence embeddings. Sampling from the multinomial distribution defined by $P_F$ then takes $O(m)$ time. 

Additionally note from lines 13 and 15 that utility has a maximum of 0 and a minimum of $-\frac{k}{2}$, which is a semantic change from the main paper where maximum utility is $\frac{k}{2}$ and minimum is 0. 

\subsection{Proof of Privacy}

\textbf{Theorem \ref{thm:mainthm}} \emph{
	$\mname$ satisfies $\epsilon$-Sentence Privacy
}
\begin{proof}
%\begin{lemma}
%	The maximum change in the minimum $h_j$ for neighboring documents is 1. 
%	\begin{proof}
%		\max_{x, x', f_i} \big| 
%	\end{proof}
%\end{lemma}

	It is sufficient to show that the sensitivity, 
	\begin{align*}
		\Delta u = \max_{x, x', f_i} | u(x,f_i) - u(x', f_i)| \leq 1 \quad . 
	\end{align*} 
	Let us expand the above expression using the terms in Algorithm \ref{alg:main alg}. 
	\begin{align*}
		\Delta u &= \max_{x, x', f_i} | \max_{j \in [p]} u_j(x,f_i)  - \max_{j' \in [p]} u_{j'}(x',f_i)| \\ 
		&= \max_{x, x', f_i} | \min_{j \in [p]} \big| h_j(x,f_i) - \frac{k}{2} \big|  \\
		&- \min_{j' \in [p]} \big| h_{j'}(x',f_i) - \frac{k}{2} \big|| \\
		&\leq \max_{ f_i} | \min_{j \in [p]} \big| h_j(x,f_i) - \frac{k}{2} \big|  \\
		&- \big( \min_{j' \in [p]} \big| h_{j'}(x,f_i) - \frac{k}{2} \big|-1\big) | \\
		&\leq 1
	\end{align*}
	The last step follows from the fact that $|h_j(x, f_i) - h_j(x', f_i)| \leq 1$ for all $j \in [p]$. In other words, by modifying a single sentence embedding, we can only change the number of embeddings greater than $f_i^j$ on projection $j$ by 1. So, the distance of $h_j(x, f_i)$ from $\frac{k}{2}$ can only change by 1 on each projection. In the `worst case', the distance $\big| h_j(x,f_i) - \frac{k}{2} \big|$ reduces by 1 on every projection $v_j$. Even then, the minimum distance from $\frac{k}{2}$ across projections (the worst case depth) can only change by 1, giving us a sensitivity of 1. 
\end{proof}





%$\Delta u = \max_{D, D', o} | u(D,o) - u(D', o)|$. 

\subsection{Experimental Details}

Here, we provide an extended, detailed version of section \ref{sec:experiments}. 

For the general encoder, $G:\calS \rightarrow \R^{768}$, we use SBERT \cite{sbert}, a version of BERT fine-tuned for sentence encoding. Sentence embeddings are generated by mean-pooling output tokens. In all tasks, we freeze the weights of SBERT. The cluster-preserving recoder, $H$, as well as every classifier is implemented as an instance of a 4-layer MLP taking $768$-dimension inputs and only differing on output dimension. We denote an instance of this MLP with output dimension $o$ as \MLP{o}. We run 5 trials of each experiment with randomness taken over the privacy mechanisms, and plot the mean along with a $\pm$ 1 standard deviation envelope. 

\paragraph{Non-private:} For our non-private baseline, we demonstrate the usefulness of sentence-mean document embeddings. First, we generate the document embeddings $\overline{g}(x_i)$ for each training, validation, and test set document using SBERT, $G$. We then train a classifier $C_{\text{nonpriv}} = $ \MLP{r} to predict each document's topic or sentiment, where $r$ is the number of classes. The number of training epochs is determined with the validation set. 

\paragraph{\technique :} We first collect the candidate set $F$ by sampling 5k document embeddings from the subset of the training set containing at least 8 sentences. We run $k$-means with $n_c = 50$ cluster centers, and label each training set document embedding $t_i \in T_G$ with its cluster. The sentence recoder, $H = $ \MLP{768} is trained on the training set along with the linear model $L$ with the Adam optimizer and cross-entropy loss. For a given document $x$, its  sentence embeddings $S_x$ are passed through $H$, averaged together, and then passed to $L$ to predict $x$'s cluster. $L$'s loss is then back-propagated through $H$. A classifier $C_{\text{dc}} = $ \MLP{r} is trained in parallel using a separate instance of the Adam optimizer to predict class from the recoded embeddings, where $r$ is the number of classes (topics or sentiments). The number of training epochs is determined using the validation set. At test time, (generating private embeddings using $\mname$), the optimal number of projections $p$ is empirically chosen for each $\epsilon$ using the validation set. 

\paragraph{Truncation:} The truncation baseline \cite{clifton} requires first constraining the embedding instance space. We do so by computing the 75\% median interval on each of the 768 dimensions of training document embeddings $T_G$. Sentence embeddings are truncated at each dimension to lie in this box. In order to account for this distribution shift, a new classifier $C_{\text{trunc}} = $ \MLP{r} is trained on truncated mean embeddings to predict class. The number of epochs is determined with the validation set. At test time, a document's sentence embeddings $S_x$ are truncated and averaged. We then add Laplace noise to each dimension with scale factor $\frac{768 w}{k \epsilon}$, where $w$ is the width of the box on that dimension (\emph{sensitivity} in DP terms). Note that the standard deviation of noise added is inversely proportional to the number of sentences in the document, due to the averaging operation reducing sensitivity. 

\paragraph{Word Metric-DP:} Our next baseline satisfies $\epsilon$-word-level metric DP and is adopted from \cite{metricdp}. The corresponding mechanism $\text{MDP}: \calX \rightarrow \calX$ takes as input a document $x$ and returns a private version, $x'$, by randomizing each word individually. For comparison, we generate document embeddings by first randomizing the document $x' = \text{MDP}(x)$ as prescribed by \cite{metricdp}, and then computing its document embedding $\overline{g}(x')$ using SBERT. At test time, we classify the word-private document embedding using $C_{\text{nonpriv}}$. 

\paragraph{Random Guess:} To set a bottom-line, we show the theoretical performance of a random guesser. The guesser chooses class $i$ with probability $q_i$ equal to the fraction of $i$ labels in the training set. The performance is then given by $\sum_{i = 1}^{r} q_i^2$. 



\subsection{Reproducability Details}
We plan to publish a repo of code used to generate the exact figures in this paper (random seeds have been set) with the final version. Since we do not train the BERT base model $G$, our algorithms and training require relatively little computational resouces. Our system includes a single Nvidia GeForce RTX 2080 GPU and a single Intel i9 core. All of our models complete an epoch training on all datasets in less than one minute. We never do more than 20 epochs of training. All of our classifier models train (including linear model) have less than 11 million parameters. The relatively low amount of parameters is due to the fact that we freeze the underlying language model. The primary hyperparameter tuned is the number of projections $p$. We take the argmax value on the validation set between 10 and 100 projections. We repeat this for each value of $\epsilon$. 

\paragraph{Dataset preprocessing:} For all datasets, we limit ourselves to documents with at least 2 sentences. 

\imdb: This dataset has pre-defined train/test splits. We use the entire training set and form the test set by randomly sampling 4,000 from the test set provided. We do this for efficiency in computing the Metric-DP baseline, which is the slowest of all algorithms performed. Since the Metric-DP baseline randomizes first, we cannot precompute the sentence embeddings $G(s_i)$ -- we need to compute the sentence embeddings every single time we randomize. Since we randomize for each sentence of each document at each $\epsilon$ and each $k$ over 5 trials -- this takes a considerable amount of time. 

\goodreads: This dataset as provided is quite large. We randomly sample 15000 documents from each of 4 classes, and split them into 12K training examples, 2K validation examples, and 1K test examples per class. 

\tnews: We preprocess this dataset to remove all header information, which may more directly tell information about document class, and only provide the model with the sentences from the main body. We use the entire dataset, and form the Train/Val/Test splits by random sampling. 

















% \twocolumn
% [\textbf{Local Inferential Privacy through Data Shuffling -- Supplementary Material }]

\graphicspath{{./chapters/chapter4/}}
\chapter{ }


\section{Appendix}\label{app}
\subsection{Background Cntd.}\label{app:background}

\subsection{Local Inferential Privacy}  \vspace{-0.2cm}
%introduce pufferfish inferntial log loss 
%In this section, we introduce some context for inferential privacy in the \ldp setting. 
%Inferential privacy captures the privacy loss in the face of an informed adversary in a Bayesian framework. 
Local inferential privacy captures what information a Bayesian adversary \cite{Pufferfish}, with some prior, can learn in the \ldp setting. 
Specifically, it measures the largest possible ratio between the adversary's posterior and prior beliefs about an individual’s data after observing a mechanism's output .%\footnote{This quantity is identical to the \ldp parameter of the mechanism when\textit{individuals’ data are independent}\cite{sok,}.}.
\begin{defn}(Local Inferential Privacy Loss \cite{Pufferfish}) Let $\bx=\langle x_1, \cdots, x_n\rangle$ and let $\by=\langle y_1, \cdots, y_n \rangle$ denote the input (private) and output sequences (observable to the adversary) in the \ldp setting. Additionally, the adversary's auxiliary knowledge is modeled by a prior distribution $\mathcal{P}$ on $\mathbf{x}$. The inferential privacy loss for the input sequence $\mathbf{x}$ is given by
% \vspace{0cm} 
\begin{equation}
% \vspace{-0.1cm}
\small \mathbb{L}_{\calP}(\mathbf{x})=\max_{\substack{i\in [n]\\ a,b \in \calX}}\Bigg(\log\frac{\mathrm{Pr}_{\calP}[\mathbf{y}|x_i=a]}{\mathrm{Pr}_{\calP}[\mathbf{y}|x_i=b]}\Bigg)
=\max_{\substack{i\in [n]\\ a,b \in \calX}}\Bigg (	\bigg| \log \frac{\mathrm{Pr}_{\calP}[x_i = a | \bf{y} ]}{\mathrm{Pr}_{\calP}[x_i = b | \bf{y}]}
	- \log \frac{\mathrm{Pr}_{\calP}[x_i = a]}{\mathrm{Pr}_{\calP}[x_i = b]} \bigg|\Bigg)
\end{equation}
\label{def:ip}
\vspace{-1em}
\end{defn}
% Using Bayes' theorem, we have\vspace{-0.2cm}
% \begin{gather*}\small\vspace{-0.4cm} \mathbb{L}_{\calP}(\mathbf{x})=\max_{\substack{i\in [n]\\ a,b \in \calX}}\Bigg (	\bigg| \log \frac{\mathrm{Pr}_{\calP}[x_i = a | \bf{y} ]}{\mathrm{Pr}_{\calP}[x_i = b | \bf{y}]}
% 	- \log \frac{\mathrm{Pr}_{\calP}[x_i = a]}{\mathrm{Pr}_{\calP}[x_i = b]} \bigg|\Bigg)\vspace{-0.7cm}\end{gather*}
Bounding  $\mathbb{L}_{\calP}(\mathbf{x})$  would imply
 that the adversary's belief about the value of any $x_i$ does not change by much even after observing the output sequence $\bf{y}$. This means that an informed adversary does not learn much about the individual $i$'s private input upon observation of the entire private dataset $\by$.

Here we define two rank distance measures \begin{defn}[Kendall's $\tau$ Distance] For any two permutations, $\sigma, \pi \in \mathrm{S}_n$, the Kendall's $\tau$
distance $\textswab{d}_{\tau}(\sigma, \pi)$ counts the number of pairwise disagreements between $\sigma$ and $\pi$, i.e., the
number of item pairs that have a relative order in one permutation and a different order in
the other. Formally, 
\begin{equation}
\textswab{d}_{\tau}(\sigma,\pi)=\Big| \ \big\{(i,j) : i < j,  \big[\sigma(i) > \sigma(j) \wedge \pi(i) < \pi(j) \big]\\\hspace{2cm} \\
		\vee \big[\sigma(i) < \sigma(j) \wedge \pi(i) > \pi(j)\big] \big\} \ \Big|\label{eq:kendalltau}
\end{equation}  
\label{def:kendall} 
\end{defn}
%Equivalently, $d_{\tau}(\sigma, \pi)$ is defined as the number of adjacent swaps to convert$\sigma^{-1}$into $\pi^{-1}$.
For example, if $\sigma=(1 \:\: 2 \: \: 3 \: \: 4 \:  \: 5 \: \: 6 \: \: 7 \: \: 8 \: 9 \: 10)$ and  $\pi=(1\:2\:3\:\underline{6} \: 5 \: \underline{4}\:7\:8\:9\:10)$, then $\textswab{d}_{\tau}(\sigma,\pi)=3$.

 Next, Hamming distance measure is defined as follows.
 
\begin{defn}[Hamming Distance] 
For any two permutations, $\sigma, \pi \in \mathrm{S}_n$, the Hamming distance $\textswab{d}_{H}(\sigma, \pi)$ counts the number of positions in which the two permutations disagree. Formally, 
\begin{align*}
    \textswab{d}_H(\sigma, \pi)
    &= \Big| \big\{ i \in [n] : \sigma(i) \neq \pi(i) \big\} \Big| 
\end{align*}
Repeating the above example, if $\sigma=(1 \:\: 2 \: \: 3 \: \: 4 \:  \: 5 \: \: 6 \: \: 7 \: \: 8 \: 9 \: 10)$ and  $\pi=(1 \: 2 \: 3 \: \underline{6} \: 5 \: \underline{4} \: 7 \: 8 \: 9 \: 10)$, then $\textswab{d}_{H}(\sigma,\pi)=2$.
\end{defn}

\subsection{\name-privacy and the De Finetti attack}
\label{app:de finetti}
We now show that a strict instance of \name privacy is sufficient for thwarting any de Finetti attack \cite{definetti} on individuals. The de Finetti attack involves a Bayesian adversary, who, assuming some degree of correlation between data owners, attempts to recover the true permutation from the shuffled data. As written, the de Finetti attack assumes the sequence of sensitive attributes and side information $(x_1, t_1), \dots, (x_n, t_n)$ are \emph{exchangeable}: any ordering of them is equally likely. By the de Finetti theorem, this implies that they are i.i.d. conditioned on some latent measure $\theta$. To balance privacy with utility, the $\bx$ sequence is non-uniformly randomly shuffled w.r.t. the $\bt$ sequence producing a shuffled sequence $\bz$, which the adversary observes. Conditioning on $\bz$ the adversary updates their posterior on $\theta$ (i.e. posterior on a model predicting $x_i | t_i$), and thereby their posterior predictive on the true $\bx$. The definition of privacy in \cite{definetti} holds that the adversary's posterior beliefs are close to their prior beliefs by some metric on distributions in $\calX$, $\delta(\cdot, \cdot)$: 
\begin{align*}
    \delta\Big( \Pr[x_i], \Pr[x_i | \bz] \Big) \leq \alpha 
\end{align*}

We now translate the de Finetti attack to our setting. First, to align notation with the rest of the paper we provide privacy to the sequence of $\ldp$ values $\by$ since we shuffle those instead of the $\bx$ values as in \cite{definetti}. We use max divergence (multiplicative bound on events used in \DP) for $\delta$: 
\begin{align*}
    \Pr[y_i \in O] &\leq e^\alpha \Pr[y_i \in O | \bz] \\
    \Pr[y_i \in O | \bz] &\leq e^\alpha \Pr[y_i \in O]
\end{align*}
which, for compactness, we write as 
\begin{align}
    \Pr[y_i \in O] \approx_\alpha \Pr[y_i \in O | \bz] \quad. 
    \label{eq:definetti privacy}
\end{align}
We restrict ourselves to shuffling mechanisms, where we only randomize the order of sensitive values. By learning the unordered values $\{y\}$ alone, an adversary may have arbitrarily large updates to its posterior (e.g. if all values are identical), breaking the privacy requirement above. With this in mind, we assume the adversary already knows the unordered sequence of values $\{y\}$ (which they will learn anyway), and has a prior on permutations $\sigma$ allocating values from that sequence to individuals. We then generalize the de Finetti problem to an adversary with an \emph{arbitrary} prior on the true permutation $\sigma$, and observes a randomize permutation $\sigma'$ from the shuffling mechanism. We require that the adversary's prior belief that $\sigma(i) = j$ is close to their posterior belief for all $i,j \in [n]$: 
\begin{align}
    \Pr[\sigma \in \Sigma_{i,j} ] \approx_\alpha \Pr[\sigma \in \Sigma_{i,j} | \sigma'] \quad \forall i,j \in [n], \forall \sigma' \in S_n \quad ,
    \label{eq:definetti privacy II}
\end{align}
where $\Sigma_{i,j} = \{\sigma \in S_n : \sigma(i) = j\}$, the set of permutations assigning element $j$ to $\DO_i$. Conditioning on any unordered sequence $\{y\}$ with all unique values, the above condition is necessary to satisfy Eq. \eqref{eq:definetti privacy} for events of the form $O = \{y_i = a\}$, since $\{y_i = a\} = \{\Sigma_{i,j}\}$ for some $j \in [n]$. For any $\{y\}$ with repeat values, it is sufficient since $\Pr[y_i = a]$ is the sum of probabilities of disjoint events of the form $\Pr[\sigma \in \Sigma_{i,k}]$ for various $k \in [n]$ values. 

We now show that a strict instance of \name-privacy satisfies Eq. \eqref{eq:definetti privacy II}. Let $\widehat{\calG}$ be any group assignment such that at least one $G_i \in \widehat{\calG}$ includes all data owners, $G_i = \{1, 2, \dots, n\}$. 

\begin{prope}
A $(\widehat{\calG}, \alpha)$-\name-private shuffling mechanism $\sigma' \sim \calA$ satisfies  
\begin{align*}
    \Pr[\sigma \in \Sigma_{i,j} ] \approx_\alpha \Pr[\sigma \in \Sigma_{i,j} | \sigma']
\end{align*}
for all $i,j \in [n]$ and all priors on permutations $\Pr[\sigma]$. 
\end{prope}

\begin{proof}

\begin{lemma}
\label{lem:definetti equivalent}
    For any prior $\Pr[\sigma]$, Eq. \eqref{eq:definetti privacy II} is equivalent to the condition
    \begin{align}
        \frac{\sum_{\hat{\sigma} \in \overline{\Sigma}_{i,j}} \Pr[\hat{\sigma}] \Pr[\sigma' | \hat{\sigma}] }{
        \sum_{\hat{\sigma} \in {\Sigma_{i,j}}} \Pr[\hat{\sigma}] \Pr[\sigma' | \hat{\sigma}] } 
        \approx_\alpha 
        \frac{\sum_{\hat{\sigma} \in \overline{\Sigma}_{i,j}} \Pr[\hat{\sigma}] }{
        \sum_{\hat{\sigma} \in {\Sigma_{i,j}}} \Pr[\hat{\sigma}] }
        \label{eq:definetti privacy III}
    \end{align}
    where the set $\overline{\Sigma}_{i,j}$ is the complement of ${\Sigma}_{i,j}$. 
\end{lemma}
Under grouping $\hat{\calG}$, every permutation $\sigma_a \in {\Sigma}_{i,j}$ neighbors every permutation $\sigma_b \in \overline{\Sigma}_{i,j}$, $\sigma_a \approx_{\hat{\calG}} \sigma_b$, for any $i,j$. By the definition of \name-privacy, we have that for any observed permutation $\sigma'$ output by the mechanism: 
\begin{align*}
    \Pr[\sigma' | \sigma = \sigma_a] \approx_\alpha \Pr[\sigma' | \sigma = \sigma_b]
    \quad \forall \sigma_a \in {\Sigma}_{i,j}, \sigma_b \in \overline{\Sigma}_{i,j}, \sigma' \in S_n 
    \quad .
\end{align*}
This implies Eq. \ref{eq:definetti privacy III}. Thus, $(\widehat{\calG}, \alpha)$-\name-privacy implies Eq. \ref{eq:definetti privacy III}, which implies Eq. \ref{eq:definetti privacy II}, thus proving the property. 
\end{proof}

Using Lemma \ref{lem:definetti equivalent}, we may also show that this strict instance of \name-privacy is \emph{necessary} to block all de Finetti attacks: 

\begin{prope}
A $(\widehat{\calG}, \alpha)$-\name-private shuffling mechanism $\sigma' \sim \calA$ is necessary to satisfy 
\begin{align*}
    \Pr[\sigma \in \Sigma_{i,j} ] \approx_\alpha \Pr[\sigma \in \Sigma_{i,j} | \sigma']
\end{align*}
for all $i,j \in [n]$ and all priors on permutations $\Pr[\sigma]$. 
\end{prope}

\begin{proof}
If our mechanism $\calA$ is not $(\widehat{\calG}, \alpha)$-\name-private, then for some pair of true (input) permutations $\sigma_a \neq \sigma_b$ and some released permutation $\sigma' \sim \calA$, we have that 
\begin{align*}
    \Pr[\sigma' | \sigma_b] \geq e^\alpha \Pr[\sigma' | \sigma_a]\quad. 
\end{align*}
Under $\hat{\calG}$, all permutations neighbor each other, so $\sigma_a \approx_{\hat{\calG}} \sigma_b$. Since $\sigma_a \neq \sigma_b$, then for some $i,j \in [n]$, $\sigma_a \in \Sigma_{i,j}$ and $\sigma_b \in \overline{\Sigma}_{i,j}$: one of the two permutations assigns some $j$ to some $\DO_i$ and the other does not. Given this, we may construct a bimodal prior on the true $\sigma$ that assigns half its probability mass to $\sigma_a$ and the rest to $\sigma_b$, 
\begin{align*}
    \Pr[\sigma_a] = \Pr[\sigma_b] = \frac{1}{2} \quad .
\end{align*}
Therefore, for released permutation $\sigma'$, the RHS of Eq. \ref{eq:definetti privacy III} is 1, and the LHS is 
\begin{align*}
    \frac{\sum_{\hat{\sigma} \in \overline{\Sigma}_{i,j}} \Pr[\hat{\sigma}] \Pr[\sigma' | \hat{\sigma}] }{
        \sum_{\hat{\sigma} \in {\Sigma_{i,j}}} \Pr[\hat{\sigma}] \Pr[\sigma' | \hat{\sigma}] }
        &= \frac{\nicefrac{1}{2} \Pr[\sigma' | \sigma_b]}{\nicefrac{1}{2} \Pr[\sigma' | \sigma_a]} \\
        &\geq e^\alpha \quad , 
\end{align*}
violating Eq. \ref{eq:definetti privacy III}, thus violating Eq. \ref{eq:definetti privacy II}, and failing to prevent de Finetti attacks against this bimodal prior. 
\end{proof}

Ultimately, unless we satisfy \name-privacy shuffling the entire dataset, there exists some prior on the true permutation $\Pr[\sigma]$ such that after observing the shuffled $\bz$ permuted by $\sigma'$, the adversary's posterior belief on one permutation is larger than their prior belief by a factor $\geq e^\alpha$. If we suppose that the set of values $\{y\}$ are all distinct, this means that for some $a \in \{y\}$, the adversary's belief that $y_i = a$ is signficantly larger after observing $\bz$ than it was before. 

Now to prove Lemma \ref{lem:definetti equivalent}: 
\begin{proof}
\begin{align*}
    \Pr[\sigma \in \Sigma_{i,j} ] 
    &\approx_\alpha \Pr[\sigma \in \Sigma_{i,j} | \sigma'] \\
    \Pr[\sigma \in \Sigma_{i,j} ]
    &\approx_\alpha \frac{\Pr[\sigma' | \sigma \in \Sigma_{i,j}] \Pr[\sigma \in \Sigma_{i,j} ]}{\sum_{\hat{\sigma} \in S_n} \Pr[\hat{\sigma}] \Pr[\sigma' | \hat{\sigma}]} \\
    \sum_{\hat{\sigma} \in S_n} \Pr[\hat{\sigma}] \Pr[\sigma' | \hat{\sigma}] 
    &\approx_\alpha \Pr[\sigma' | \sigma \in \Sigma_{i,j}] \\
    \sum_{\hat{\sigma} \in S_n} \Pr[\hat{\sigma}] \Pr[\sigma' | \hat{\sigma}] 
    &\approx_\alpha \Pr[\sigma \in \Sigma_{i,j}]^{-1} \sum_{\hat{\sigma} \in \Sigma_{i,j}} \Pr[\hat{\sigma}] \Pr[\sigma' | \hat{\sigma}] \\
    \sum_{\hat{\sigma} \in \Sigma_{i,j}} \Pr[\hat{\sigma}] \Pr[\sigma' | \hat{\sigma}] +
    \sum_{\hat{\sigma} \in \overline{\Sigma}_{i,j}} \Pr[\hat{\sigma}] \Pr[\sigma' | \hat{\sigma}] 
    &\approx_\alpha \Pr[\sigma \in \Sigma_{i,j}]^{-1} \sum_{\hat{\sigma} \in \Sigma_{i,j}} \Pr[\hat{\sigma}] \Pr[\sigma' | \hat{\sigma}] \\
    \sum_{\hat{\sigma} \in \overline{\Sigma}_{i,j}} \Pr[\hat{\sigma}] \Pr[\sigma' | \hat{\sigma}] 
    &\approx_\alpha \sum_{\hat{\sigma} \in \Sigma_{i,j}} \Pr[\hat{\sigma}] \Pr[\sigma' | \hat{\sigma}] 
    \Big( \frac{1}{\Pr[\sigma \in \Sigma_{i,j}]} - 1 \Big)  \\
    \frac{\sum_{\hat{\sigma} \in \overline{\Sigma}_{i,j}} \Pr[\hat{\sigma}] \Pr[\sigma' | \hat{\sigma}] }{
    \sum_{\hat{\sigma} \in {\Sigma_{i,j}}} \Pr[\hat{\sigma}] \Pr[\sigma' | \hat{\sigma}] } 
    &\approx_\alpha 
    \frac{\sum_{\hat{\sigma} \in \overline{\Sigma}_{i,j}} \Pr[\hat{\sigma}] }{
    \sum_{\hat{\sigma} \in {\Sigma_{i,j}}} \Pr[\hat{\sigma}] }
\end{align*}
\end{proof}

As such, a strict instance of \name-privacy can defend against any de Finetti attack (i.e. for any prior $\Pr[\sigma]$ on permutations), wherein at least one group $G_i \in \calG$ includes all data owners. Furthermore, it is necessary. This makes sense. In order to defend against any prior, we need to significantly shuffle the entire dataset. Without a restriction of priors as in Pufferfish \cite{Pufferfish}, the de Finetti attack (i.e. uninformed Bayesian adversaries) is an indelicate metric for evaluating the privacy of shuffling mechanisms: to achieve significant privacy, we must sacrifice all utility. This in many regards is reminiscent of the no free lunch for privacy theorem established in \cite{Kifer}. As such, there is a need for more flexible privacy definitions for shuffling mechanisms.  

% Consider that an adversary with a prior on $\bx$ can similarly be written as a prior on the unordered set of values $\{x\}$ and on the permutation allocating values from that set to individuals $\sigma | \{x\}$. Any event may be written as 
% \begin{align*}
%     \Pr[\bx \in O] = \Pr[\sigma \in \Sigma | \{x\}] \Pr[\{x\} \in X] 
% \end{align*}
% for any event $O \subseteq \calX^n$ and corresponding events $\Sigma \subseteq S_n$ and $X \subseteq \calE$, where $\calE$ is the sigma field generated by all sets of $n$ elements in $\calX$. For shuffling mechanisms, we assume the adversary already knows the set of values $\{x\}$ since they will learn them anyway. Formally, the posterior $\Pr[\{x\} \in X | \bz]$ concentrates on the single outcome of $\{x\} = \{z\}$

% De Finetti's privacy definition asks that the prior $\Pr[y_i]$ is close to the posterior in some distance (I use ldp values here which is analagous to the $x_i$ values used in the paper, since they shuffled the $x_i$ directly): 
% \begin{align*}
%     \delta( \Pr[y_i], \Pr[y_i | \bz] ) \leq b
% \end{align*}

% if we use max divergence as this distance, we get that 
% \begin{align*}
%     \Pr[y_i = a] \leq \Pr[y_i = a | \bz]
% \end{align*}
% for all $i$, all $a \in \calX$, all $\bz \in \calZ^n$. 

% I'll make this next part brief, but I need to write out a longer bit to make it clear. The point is that we can equivalently have an arbitrary prior/posterior on permutations as the De Finetti paper has a prior posterior on models. I have a more mathematical way of writing this but it take a little effort. De Finetti assumes the sequence of underlying $(y_i, t_i)$ are exchangeable. There is some latent model $f_\theta: \calT \rightarrow \calY$ and a prior on $t_i$'s. Conditioning on knowing $f_\theta$ and the prior on $t_i$'s all the data is iid. Otherwise, by observing a few partially shuffled entries, we can update our posterior belief on the model $\theta$, which then updates our posterior predictive on the true permutation of the remaining points. 

% For shuffling mechanisms, we may assume the adversary knows the orderless set of $\by$, $\{y\}$, since they will see it anyway, and then make our proof for all possible $\{y\}$. Our adversary has a prior on true permutations $\sigma$ and they observe a noisy permutation $\sigma'$. We want their prior to be close to their posterior. 

% \begin{align*}
%     \Pr[\sigma] \leq e^\epsilon \Pr[\sigma | \sigma']
% \end{align*}
% and
% \begin{align*}
%     e^\epsilon \Pr[\sigma] \geq \Pr[\sigma | \sigma']
% \end{align*}
% for all true permutations $\sigma$ and observed permutations $\sigma'$. Note that this is equivalent to the original posterior/prior requirement on $y_i$: a prior on permutations implies that the belief on $y_i$ is a mixture over the believed permutuations. If we have a large divergence on permutations, we can construct a $\{y\}$ to make the divergence on $y_i$ equally large. Can write a more formal version. 
% From here the rest is relatively simple: 

% \begin{align*}
%     \Pr[\sigma] &\approx_{\epsilon} \Pr[\sigma | \sigma'] \\
%     \Pr[\sigma] &\approx_{\epsilon} \Pr[\sigma' | \sigma] \Pr[\sigma] \big/ \Pr[\sigma'] \\
%     \Pr[\sigma'] &\approx_{\epsilon} \Pr[\sigma' | \sigma] \\
%     \sum_{\sigma^*} \Pr[\sigma^*] \Pr[\sigma' | \sigma^*] &\approx_{\epsilon} \Pr[\sigma' | \sigma] \\
% \end{align*}

% So, if we don't have $\Pr[\sigma' | \sigma] \leq e^\epsilon \Pr[\sigma' | \sigma^*]$ for all $\sigma', \sigma, \sigma^*$, then we can construct a prior $\Pr[\sigma^*]$ that will break the above $\approx_\epsilon$. But also, if we have $\Pr[\sigma' | \sigma] \leq e^\epsilon \Pr[\sigma' | \sigma^*]$, then we \emph{can't} construct a prior $\Pr[\sigma^*]$ to break the above $\approx_\epsilon$. 

% The requirement $\Pr[\sigma' | \sigma] \leq e^\epsilon \Pr[\sigma' | \sigma^*]$ is just \name-privacy for $\alpha = \epsilon$ and with at least one group $G_i \in \calG$ being all data owners. 

% The main part that I need to write out more here is explaining how you need a low divergence on permutations to have a low divergence on beliefs on $y_i$. The proof is by showing that one's beliefs on $y_i$ is just a mixture over these permutations. If I can have a high divergence on permutations, I can construct some $\{y\}$ and some pair of distributions on permutations to make the divergence on $y_i$ equally large. 
 
\subsection{ Additional Properties of \name-privacy} \label{app:post-processing}

\begin{lemma}[Convexity] \label{lem:convexity}
Let $\calA_1, \dots \calA_k: \mathcal{Y}^n \mapsto \mathcal{V}$ be a collection of $k$ $(\alpha,\calG)$-\name private mechanisms for a given group assignment $\calG$ on a set of $n$ entities. Let $\calA: \mathcal{Y}^n \mapsto \mathcal{V}$ be a convex combination of these $k$ mechanisms, where the probability of releasing the output of mechanism $\calA_i$ is $p_i$, and $\sum_{i=1}^k p_i = 1$. $\calA$ is also $(\alpha,\calG)$-\name private w.r.t. $\calG$. 
\end{lemma}
\begin{proof}
For any $(\sigma, \sigma') \in \mathrm{N}_\calG$ and $\by \in \calY$: 
\begin{align*}
    \mathrm{Pr} [\calA \big( \sigma( \by) \big) \in O]
    &= \sum_{i=1}^k p_i \mathrm{Pr} [\calA_i \big( \sigma( \by) \big) \in O] \\
    &\leq e^\alpha \sum_{i=1}^k p_i \mathrm{Pr} [\calA_i \big( \sigma'( \by) \big) \in O] \\
    &=  \mathrm{Pr} [\calA \big( \sigma'( \by) \big) \in O]
\end{align*}
\end{proof}

% For a given group assignment $\calG$ on a set of $n$ entities and a privacy parameter $\alpha \in \R_{\geq0}$, a randomized  mechanism $\calA: \mathcal{Y}^n \mapsto \mathcal{V} $ is $(\alpha,\mathcal{G})$-\name~private if for all $\mathbf{y} \in \mathcal{Y}^n$ and neighboring permutations $\sigma, \sigma' \in \mathrm{N}_\calG$ and any subset of output $O\subseteq \mathcal{V}$, we have\vspace{-0.2cm} 
% \begin{gather*} \vspace{-0.5cm}
%     \mathrm{Pr}[\calA\big(\sigma(\mathbf{y})\big) \in O] \leq e^\alpha \cdot \mathrm{Pr}\big[\calA\big(\sigma'(\mathbf{y})\big) \in O \big] \numberthis \label{eq:privacy} \vspace{-0.2cm}
% \end{gather*}
% %where $\bz=\pi(\by)=\langle y_{\pi(1)},\cdots, y_{\pi(n)}\rangle, \pi \in \mathrm{S}_n$
%  $\sigma(\mathbf{y})$ and $\sigma'(\mathbf{y})$  are defined to be \textit{neighboring sequences}. 

\begin{thm}[Post-processing]\label{theorem:post}
Let \scalebox{0.9}{$\mathcal{A}: \mathcal{Y}^n \mapsto \mathcal{V}$} be  $(\alpha,\calG)$-\name private for a given group assignment $\calG$ on a set of $n$ entities. Let \scalebox{0.9}{$f : \mathcal{V} \mapsto \mathcal{V}'$} be an
arbitrary randomized mapping. Then \scalebox{0.9}{$f \circ \mathcal{A} : \mathcal{Y}^n \mapsto \mathcal{V}'$} is also $(\alpha,\calG)$-\name private. \vspace{-0.2cm}\end{thm}

\begin{proof}
Let $g: \mathcal{V}\rightarrow \mathcal{V}'$ be a deterministic, measurable function. For any output event $\mathcal{Z}\subset \mathcal{V}'$, let $\mathcal{W}$ be its preimage: \newline $\mathcal{W}=\{v \in \mathcal{V}| g(v) \in \mathcal{Z}\}$. Then, for any $(\sigma, \sigma') \in \mathrm{N}_\calG$,
\begin{align*}
    \mathrm{Pr}\Big[g\Big(\calA\big(\sigma(\by)\big)\Big)\in \mathcal{Z}\Big] 
    &= \mathrm{Pr}\Big[\calA\big(\sigma(\by)\big)\in \mathcal{W}\Big] \\ 
    &\leq e^{\alpha} \cdot \mathrm{Pr}\Big[\calA\big(\sigma'(\by)\big)\in \mathcal{W}\Big]\\ 
    &=e^{\alpha}\cdot \mathrm{Pr}\Big[g\Big(\calA\big(\sigma'(\by)\big)\Big)\in \mathcal{Z}\Big] 
\end{align*}
This concludes our proof because any randomized mapping
can be decomposed into a convex combination of measurable, deterministic functions \cite{Dwork}, and as Lemma \ref{lem:convexity} shows, a convex combination of $(\alpha,\calG)$-\name private mechanisms is also $(\alpha,\calG)$-\name private. 
\end{proof}

\begin{thm}[Sequential Composition] \label{theorem:seq}
If $\calA_1$ and $\calA_2$ are $(\alpha_1, \calG)$- and $(\alpha_2, \calG)$-\name private mechanisms, respectively, that use independent randomness, then releasing the outputs $\big( \calA_1(\by), \calA_2(\by) \big)$ satisfies $(\alpha_1+\alpha_2, \calG)$-\name privacy. 
\end{thm}
\begin{proof}
We have that $\calA_1: \calY^n \rightarrow \mathcal{V}'$ and $\calA_1: \calY^n \rightarrow \mathcal{V}''$ each satisfy \name-privacy for different $\alpha$ values. Let $\calA:\calY^n \rightarrow (\mathcal{V}' \times \mathcal{V}'')$ output $\big(\calA_1(\by), \calA_2(\by)\big)$. Then, we may write any event $\mathcal{Z} \in (\mathcal{V}' \times \mathcal{V}'')$ as $\mathcal{Z}' \times \mathcal{Z}''$, where $\mathcal{Z}' \in \mathcal{V}'$ and $\mathcal{Z}'' \in \mathcal{V}''$. We have for any $(\sigma, \sigma') \in \mathrm{N}_\calG$, 
\begin{align*}
    \mathrm{Pr} \big[ \calA\big( \sigma(\by) \big)  &\in \mathcal{Z} \big]  
    = \mathrm{Pr} \big[ \big(\calA_1\big( \sigma(\by) \big) , \calA_2\big( \sigma(\by) \big) \big) \in \mathcal{Z} \big] \\
    &= \mathrm{Pr} \big[ \{\calA_1\big( \sigma(\by) \big)  \in \mathcal{Z}'\} \cap \{\calA_2\big( \sigma(\by) \big)   \in \mathcal{Z}'' \} \big]  \\
    &= \mathrm{Pr} \big[ \{\calA_1\big( \sigma(\by) \big)  \in \mathcal{Z}'\} \big] 
    \mathrm{Pr} \big[ \{\calA_2\big( \sigma(\by) \big)   \in \mathcal{Z}'' \} \big] \\
    &\leq e^{\alpha_1 + \alpha_2}
    \mathrm{Pr} \big[ \{\calA_1\big( \sigma'(\by) \big)  \in \mathcal{Z}'\} \big] 
    \mathrm{Pr} \big[ \{\calA_2\big( \sigma'(\by) \big)   \in \mathcal{Z}'' \} \big] \\
    &= e^{\alpha_1 + \alpha_2} \cdot 
    \mathrm{Pr} \big[ \calA\big( \sigma'(\by) \big)  \in \mathcal{Z} \big] 
\end{align*}
\end{proof}

% Proof of Lemma \ref{lemma:LDP} \\

% \textbf{Lemma \ref{lemma:LDP}} \emph{
% An $\epsilon$-\ldp mechanism is $(k\epsilon, \calG)$-\name~ private for any group assignment $\calG$ such that $
%         k \geq \max_{G_i \in \calG} |G_i|
% $
% }
% \begin{proof}
% This follows from $k$-group privacy \cite{Dwork}. $\by$ are $\varepsilon$-LDP outputs $\calA_{\text{LDP}}(\bx)$ from input sequence $\bx$. For any $\sigma \approx_{G_i} \sigma'$, we know by definition that $\sigma(j) = \sigma'(j)$ for all $j \notin G_i$. As such, the permuted sequences $\sigma(\bx)_j = \sigma'(\bx)_j$ for all $j \notin G_i$, and differ in at most $|G_i|$ entries. In other words, 
% \begin{align*}
%     \textswab{d}_H \big(\sigma(\bx), \sigma'(\bx)\big) \leq |G_i| 
% \end{align*}
% Using this fact, we have from the $k$-group property of LDP that 
% \begin{align*}
%     \mathrm{Pr}\big[ \calA_{\text{LDP}} \big( \sigma(\bx) \big) \in O \big] 
%     \leq e^{|G_i| \epsilon} \mathrm{Pr}\big[ \calA_{\text{LDP}} \big( \sigma'(\bx) \big) \in O \big] 
% \end{align*}
% and thus if $k \geq \max_{G_i \in \calG} |G_i|$, 
% \begin{align*}
%     \mathrm{Pr}\big[ \calA_{\text{LDP}} \big( \sigma(\bx) \big) \in O \big] 
%     \leq e^{k \epsilon} \mathrm{Pr}\big[ \calA_{\text{LDP}} \big( \sigma'(\bx) \big) \in O \big] 
% \end{align*}
% for all $(\sigma, \sigma') \in \mathrm{N}_\calG$. 
% \end{proof}

\subsection{Proof for Thm. \ref{thm: semantic guarantee}}\label{app:thm:semantic}
\label{app:bayesian proof} 
\textbf{Theorem \ref{thm: semantic guarantee}} 
\emph{
For a given group assignment $\calG$ on a set of $n$ data owners, if a shuffling mechanism $\calA:\calY^n\mapsto \calY^n$ is $(\alpha,\calG)$-\name private, then for each data owner $\DO_i, i \in [n]$, %\vspace{-0.1cm}
\begin{align*}
   \max_{\substack{i\in [n]\\ a,b \in \calX}} \bigg|\log \frac{\Pr_\calP [x_i = a | \bz, \{y_{G_i}\},\by_{\overline{G}_i}]}{\Pr_\calP [x_i = b | \bz, \{y_{G_i}\},\by_{\overline{G}_i}]} - \log \frac{\Pr_\calP [x_i = a | \{y_{G_i}\},\by_{\overline{G}_i}]}{\Pr_\calP [x_i = b | \{y_{G_i}\},\by_{\overline{G}_i}]} \bigg| \leq \alpha  %\vspace{-0.5cm}
\end{align*}
for a prior distribution $\calP$, where \scalebox{0.9}{$\bz=\calA(\by)$} and \scalebox{0.9}{$\by_{\overline{G}_i}$} is the noisy sequence for data owners outside \scalebox{0.9}{$G_i$}.
}
\begin{proof}
We prove the above by bounding the following equivalent expression for any $i \in [n]$ and $a, b \in \calX$. 
\begin{align*}
     &\frac{\Pr_\calP[\bz | x_i=a, \{y_{G_i}\}, \by_{\overline{G}_i}]}{\Pr_\calP [\bz | x_i=b, \{y_{G_i}\}, \by_{\overline{G}_i}]}\\
     &= \frac{\int \Pr_\calP [\by | x_i = a, \{y_{G_i}\}, \by_{\overline{G}_i}] \Pr_\calA[\bz | \by] d\by}{\int \Pr_\calP [\by | x_i = b, \{y_{G_i}\}, \by_{\overline{G}_i}] \Pr_\calA[\bz | \by] d\by} \\
     &= \frac{\sum_{\sigma \in \mathrm{S}_r} \Pr_\calP[ \sigma(\by_{G_i}^*) | x_i = a, \by_{\overline{G}_i}] \Pr_\calA [\bz | \sigma(\by_{G_i}^*), \by_{\overline{G}_i}] }
     {\sum_{\sigma \in \mathrm{S}_r} \Pr_\calP[ \sigma(\by_{G_i}^*) | x_i = b, \by_{\overline{G}_i}] \Pr_\calA [\bz | \sigma(\by_{G_i}^*), \by_{\overline{G}_i}]} \\
     &\leq \max_{\{ \sigma, \sigma' \in \mathrm{S}_r\}}
     \frac{ \Pr_\calA [\bz | \sigma(\by_{G_i}^*), \by_{\overline{G}_i}]}{ \Pr_\calA [\bz | \sigma'(\by_{G_i}^*), \by_{\overline{G}_i}]}  \\
     &\leq \max_{\{ \sigma, \sigma' \in \mathrm{N}_{G_i}\}}
     \frac{\Pr_\calA[\bz | \sigma(\by)]}{\Pr_\calA[\bz | \sigma'(\by)]}  \\
     &\leq e^\alpha 
\end{align*}
The second line simply marginalizes out the full noisy sequence $\by$. The third line reduces this to a sum over permutations of of $\by_{G_i}$, where $r = |G_i|$ and $\by^*_{G_i}$ is any fixed permutation of values $\{y_{G_i}\}$. This is possible since we are given the values outside the group, $\by_{\overline{G}_i}$, and the unordered set of values inside the group, $\{y_{G_i}\}$. Note that the permutations $\sigma$ written here are possible permutations of the \ldp input, not permutations output by the mechanism applied to the input as sometimes written in other parts of this document. 

The fourth line uses the fact that the numerator and denominator are both convex combinations of $\Pr_\calA [\bz | \sigma(\by_{G_i}^*), \by_{\overline{G}_i}]$ over all $\sigma \in \mathrm{S}_r$. 

The fifth line uses the fact that for any $\by_{\overline{G}_i}$, $$(\sigma(\by_{G_i}^*), \by_{\overline{G}_i}) \approx_{G_i} (\sigma'(\by_{G_i}^*), \by_{\overline{G}_i}) \ . $$
This allows a further upper bound over all neighboring sequences w.r.t. $G_i$, and thus over any permutation of $\by_{\overline{G}_i}$, as long as it is the same in the numerator and denominator. 
\end{proof}

\paragraph{Discussion}
The above Bayesian analysis measures what can be learned about $\DO_i$'s $x_i$ from observing the private release $\bz$ relative to some other known information (the conditioned information). 
%With \ldp alone, we condition on every other data owner's private value $x_j$. This implies that releasing the private sequence $\by$ cannot provide much more information about $x_i$ than releasing every other $\DO_j$'s $x_j$ would. So, only modest information unique to $x_i$ can be garnered by any Bayesian adversary. For Alice, this may be a concern, since making inferences on her disease state from those of her household is indeed a privacy violation. 
Under \name-privacy, we condition on the bag of \ldp values in Alice's group $\{y_{G_i}\}$ as well as the sequence (order and value) of $\ldp$ values outside her group $\by_{\overline{G_i}}$. This implies that releasing the shuffled sequence $\bz$ cannot provide much more information about Alice's $x_i$ than would releasing the \ldp values outside her neighborhood (her group) and the unordered bag of \ldp values inside her neighborhood, regardless of the adversary's prior knowledge $\calP$. This is a communicable guarantee: if Alice feels comfortable with the data collection knowing that her entire neighborhood's responses will be uniformly shuffled together (including those of her household), then she ought to be comfortable with \name-privacy. Now, we have to provide this guarantee to Bob, a neighbor of Alice, as well as Luis, a neighbor of Bob but \textit{not} of Alice. Thus, Bob, Alice and Luis have \textit{distinct} and \textit{overlapping} groups (neighborhoods). Hence, the trivial solution of uniformly shuffling the noisy responses of every group separately does not work in this case. % Of course, we could instead simply shuffle Alice's neighborhood uniformly, but we could not do this for each data owner's group while still maintaining utility -- the analyst can still estimate disease prevalence within neighborhoods and districts. With overlapping groups this may require shuffling the entire dataset uniformly. 
\name-privacy, however, offers the above guarantee to each user (knowing that their entire neighborhood is \textit{nearly} uniformly shuffled) while still maintaining utility (estimate disease prevalence within neighborhoods). Semantically, this is very powerful, since it implies that the noisy responses specific to one's household cannot be leveraged to infer one's disease state $x_i$. %The adversary can learn about $x_i$ from the disease prevalence outside Alice's neighborhood and, on average, inside her neighborhood, but not much beyond that. 

% For a given group assignment $\calG$ on a set of $n$ data owners, if a shuffling mechanism $\calA:\calY^n\mapsto \calY^n$ is $(\alpha,\calG)$-\name private, then for each data owner $\DO_i, i \in [n]$, we have 
% \begin{align}
%     \mathbb{L}^{\sigma}_{\mathcal{P}}(\bx)
%     =\max_{\substack{i\in [n]\\ a,b \in \calX}}\Big|\log \frac{\Pr_\calP[\bz | x_i=a, \{y_{G_i}\}]}{\Pr_\calP [\bz | x_i=b, \{y_{G_i}\}]} \Big| 
%     \leq \alpha  
% \end{align}
% for any prior distribution $\calP$ such that $(x_{i},\{y_{G_i}\}) \perp \!\!\! \perp \by_{\overline{G}_i}|\calP$ where $\bz=\calA(\by)$ and $\by_{\overline{G}_i}$ is the (ordered and noisy) data sequence for all data owners outside $G_i$. 

% \begin{align*}
%      &\frac{\Pr_\calP[\bz | x_i=a, \{y_{G_i}\}]}{\Pr_\calP [\bz | x_i=b, \{y_{G_i}\}]}\\
%      &= \frac{\int \Pr_\calP [\by | x_i = a, \{y_{G_i}\}] \Pr_\calA[\bz | \by] d\by}{\int \Pr_\calP [\by | x_i = b, \{y_{G_i}\}] \Pr_\calA[\bz | \by] d\by} \\
%      &= \frac{\int \textcolor{red}{\Pr_\calP [\by_{\overline{G}_i} | x_i = a, \by_{G_i}]} 
%      \Pr_\calP [\by_{G_i} | x_i = a, \{y_{G_i}\}] 
%      \Pr_\calA[\bz | \by] d\by}
%      {\int \textcolor{red}{\Pr_\calP [\by_{\overline{G}_i} | x_i = b, \by_{G_i}]}
%      \Pr_\calP [\by_{G_i} | x_i = b, \{y_{G_i}\}]
%      \Pr_\calA[\bz | \by] d\by} \\
%      &= \frac{\int  \textcolor{red}{\Pr_\calP [\by_{\overline{G}_i} ]} \sum_{\sigma \in \mathrm{S}_m} \ \Pr_\calP [\by_{G_i} = \sigma(\{y_{G_i}\}) | x_i = a]  \ \Pr_\calA[\bz | \by] d\by_{\overline{G}_i}}
%      {\int  \textcolor{red}{\Pr_\calP [\by_{\overline{G}_i} ]} \sum_{\sigma \in \mathrm{S}_m} \ \Pr_\calP [\by_{G_i} = \sigma(\{y_{G_i}\}) | x_i = b]  \ \Pr_\calA[\bz | \by]  d\by_{\overline{G}_i}} \\
%      &\leq \sup_{ \by_{\overline{G}_i}} \frac{\sum_{\sigma \in \mathrm{S}_m} \ \Pr_\calP [\by_{G_i} = \sigma(\{y_{G_i}\}) | x_i = a]  \ \Pr_\calA[\bz | \by]}
%      {\sum_{\sigma \in \mathrm{S}_m} \ \Pr_\calP [\by_{G_i} = \sigma(\{y_{G_i}\}) | x_i = b]  \ \Pr_\calA[\bz | \by]} \\
%      &\leq \max_{\{ \sigma(\by), \sigma'(\by) : \sigma, \sigma' \in N_{G_i}\}}
%      \frac{\Pr_\calA[\bz | \sigma(\by)]}{\Pr_\calA[\bz | \sigma'(\by)]} \leq e^\alpha 
% \end{align*}

% The terms in red highlight the step where we can play with independence. The simplest assumption to add is that $\bx_{G_i} \perp \bx_{\overline{G}_i} | \calP$. We have to change what we currently have -- I had forgotten why I had changed it previously. This is close, though. 

% It's straightforward to see how we could also just condition on knowing $\by_{\overline{G}_i}$, and that would take care of the red terms. 

\subsection{Proof of Theorem \ref{thm: decision theoretic}} 

% \textbf{Theorem \ref{thm: decision theoretic}} \emph{
%   For $\mathcal{A}(\mathcal{M}(\bx))=\bz$ where $\mathcal{M}(\cdot)$ is $\epsilon$-\ldp and $\mathcal{A}(\cdot)$ is $\alpha$ - \name private, we have  
% %  \begin{align*}
% %      \Pr[\calA(\bx) = \bz, \sigma(H) \cap I < l]
% %      \geq \beta(r,k, l) e^{-(2k \epsilon + \alpha)} \Pr[\calA(\bx) = \bz, \sigma(I) \geq H]
% %  \end{align*}
%  \begin{align*}
%      \Pr[\mathcal{D}_{Adv} \text{ loses}] \geq \binom{r-k}{k} e^{-(2k\epsilon+\alpha)} \cdot \Pr[\mathcal{D}_{Adv} \text{ wins}]
%  \end{align*}
%  for any input subgroup $I \subset G_i, r = |G_i|$ and  $k < r/2$. 
%  }
 
%  \begin{proof}
%  We first focus on deterministic adversaries and then expand to randomized adversaries afterwards using the fact that randomized adversaries are mixtures of deterministic ones. 

% Our adversary $\mathcal{D}_{Adv}$ is then defined by a deterministic decision function $\eta: \calY^n \rightarrow [n]^k$. Upon observing $\bz$, $\eta(\bz)$ selects $k$ indices in $\bz$ which it believes originated from $I \subset G_i$. 
 
%  In the following, let $\Pr_{\bz}$ be the probability of events conditioned on the shuffled output sequence $\bz$, where randomness is over the $\epsilon$-\ldp mechanism $\mathcal{M}$ and the $\alpha$-\name-private shuffling mechanism $\calA$. \footnote{As an abuse of notation, we assume the output space of the \ldp randomizers, $\calY$, have outcomes with non-zero measure e.g. randomized response. The following analysis can be expanded to continuous outputs (with outcomes of zero measure) by simply replacing the output sequence $\bz \in \calY^n$ with an output event $\mathbf{Z} \subseteq \calY^n$.}
 
%  The adversary wins if it reidentifies all of the \ldp values originating from $I$. Letting $H = \eta(\bz)$ be the indices of elements in $\bz$ selected by $\eta$, we have that 
%  \begin{align*}
%      \Pr_{\bz} [\eta(\bz) \text{ wins}] = \Pr_{\bz} [\sigma(H) = I]
%  \end{align*}
%  where $\sigma$ is the shuffling permutation produced by $\calA$, $\bz = \sigma(\by)$ i.e. $z_i = y_{\sigma(i)}$. Concretely, this is equivalent to $\DO_i$ releasing $\DO_{\sigma(i)}$'s \ldp response. Since the permutation and \ldp outptus are randomized, there remains a significant probability that $\sigma(H) \cap I = \emptyset$ i.e. not a single element in $H$ originated from $I$ and $\calA (\mathcal{M}(\bx))$ still output the sequence $\bz$. 
 
%  We may rewrite the above probability by marginalizing out all possible $\ldp$ sequences $\by$. Conditioning on the output sequence $\bz$, the only possible \ldp sequences $\by$ are permutations of $\bz$. Note that the probability of any sequence $\by$ is determined by the input $\bx$ and the \ldp mechanism $\mathcal{M}$:  
%  \begin{align*}
%      \Pr_{\bz} [\sigma(H) = I]
%      &= \sum_{\sigma \in \mathrm{S}^n}
%      \Pr [\calA(\bx) = \by = \sigma^{-1}(\bz)]  \Pr [ \sigma | \by ] \Pr[\sigma(H) = I  | \by, \sigma] / \Pr[\bz] \\
%      &= \sum_{\sigma \in \mathrm{S}^n}
%      \Pr [\calA(\bx) = \by = \sigma^{-1}(\bz)]  \Pr [ \sigma | \by ] \mathbf{1}\{ \sigma(H) = I  \} / \Pr[\bz]
%  \end{align*}
%  Note that $\Pr_\bz [\sigma | \by] = \Pr_\bz [\sigma]$ for the mallows mechanism, which chooses its permutations independently of $\by$. Now consider when $\eta(\bz)$ loses. By similar arguments as above: 
% \begin{align*}
%     \Pr_\bz[\eta(\bz) \text{ loses}]
%     &= \Pr_\bz[\sigma(\bz) \cap I = \emptyset] \\
%     &= \sum_{\sigma \in \mathrm{S}^n}
%      \Pr [\calA(\bx) = \by = \sigma^{-1}(\bz)]  \Pr [ \sigma | \by ] \mathbf{1}\{ \sigma(H) \cap I = \emptyset  \} / \Pr[\bz]
% \end{align*}
% % Now, consider the set of permutations $\sigma$ such that $\eta$ reidentifies exactly $s$ \ldp responses from subgroup $I$. Let 
% % \begin{align*}
% %     C(s) &= \{ \sigma : | \sigma(H) \cap I | = s \} \\
% %     C_\emptyset &= \{ \sigma : \sigma(H) \cap I = \emptyset \} 
% % \end{align*}
% % Our argument centers on the following lemma

% Proof of the main theorem follows from the simple fact that there are many more permutations where $\eta(\bz)$ loses and they are close in probability to those where $\eta(\bz)$ wins. 

% \begin{lemma}
% Let $C = \{\sigma \in \mathrm{S}^n : \sigma(H) = I\}$. For every $\sigma \in C$, we may construct a set of $\binom{r-k}{k}$ permutations $E(\sigma)$ such that
% \begin{enumerate}
%     \item $E(\sigma_a) \cap E(\sigma_b) = \emptyset$ for any pair of permutations $\sigma_a, \sigma_b \in C$
%     \item $\sigma^{'-1} \approx_{G_i} \sigma^{-1}$ for all $\sigma' \in E(\sigma)$ 
%     \item $\Pr [\calA(\bx) = \by = \sigma^{-1}(\bz)] \leq e^{2k\epsilon} \Pr [\calA(\bx) = \by = \sigma^{'-1}(\bz)]$ for any $\sigma' \in E(\sigma)$
% \end{enumerate}
% \end{lemma}
% \begin{proof}
% For any permutation $\sigma \in C$, we construct $E(\sigma)$ by first taking its inverse permutation $\sigma^{-1}$. We know that $\sigma^{-1}(I) = H$ by definition. We then construct the permutations of $E(\sigma)$ by swapping the elements $\sigma^{-1}(I)$ with the elements $\sigma^{-1}(J)$ for any subset of $k$ elements outside $I$ but still in $G_i$, $J \subset G_i : J \cap I = \emptyset$, while preserving the relative ordering within $\sigma^{-1}(I)$ and within $\sigma^{-1}(J)$. There are $\binom{r-k}{k}$ such subsets. 

% On the first point, we know that no one permutation could be in both $E(\sigma_a)$ and in $E(\sigma_b)$ since the above operation is reversible (swap the elements $H$ back into position $I$ while preserving order). 

% On the second point, we only swapped elements within group $G_i$ to determine $\sigma^{'-1}$ from $\sigma^{-1}$, so by definition they are neighboring on $G_i$. 

% On the third point, the sequence $\by = \sigma^{-1}(\bz)$ only differs from $\sigma^{'-1}(\bz)$ on at most $2k$ indices.  
% \end{proof}

% Using the above lemma we have that 
% \begin{align*}
%     \frac{\Pr_\bz[\eta(\bz) \text{ loses}]}{\Pr_\bz[\eta(\bz) \text{ wins}]}
%     &\geq \frac{ \sum_{\sigma \in \mathrm{S}^n} \sum_{\sigma' \in E(\sigma)} \Pr [\calA(\bx) = \by = \sigma^{'-1}(\bz)]  \Pr [ \sigma' | \by ] }
%     { \sum_{\sigma \in \mathrm{S}^n} \Pr [\calA(\bx) = \by = \sigma^{-1}(\bz)]  \Pr [ \sigma | \by ] } \\
%     &\geq \max_{\sigma \in \mathrm{S}^n} \frac{\sum_{\sigma' \in E(\sigma)} \Pr [\calA(\bx) = \by = \sigma^{'-1}(\bz)]  \Pr [ \sigma' | \by ]}
%     {\Pr [\calA(\bx) = \by = \sigma^{-1}(\bz)]  \Pr [ \sigma | \by ]} \\
%     &\geq \binom{r-k}{k}e^{-(2k\epsilon + \alpha)}
% \end{align*}
% Where the final line comes from the fact 3 of the Lemma and from the fact that $\sigma^{-1}(\by)$ and $\sigma^{'-1}(\by)$ are neighboring (fact 2 of the Lemma), so the probability of the \name-private shuffler producing the permutations needed to achieve $\bz$, $\sigma$ and $\sigma'$, are close by a factor of $e^-\alpha$. 

% Since this holds conditionally for each $\bz$, it holds marginally across all $\bz$. Furthermore, we may state any randomized adversary as a mixture of deterministic adversaries $\eta(\bz)$, preserving the above bound. This completes the proof. 

% \end{proof} 

\textbf{Theorem \ref{thm: decision theoretic}} 

\emph{
  For $\mathcal{A}(\mathcal{M}(\bx))=\bz$ where $\mathcal{M}(\cdot)$ is $\epsilon$-\ldp and $\mathcal{A}(\cdot)$ is $\alpha$ - \name private, we have  
 \begin{align*}
     \Pr[\mathcal{D}_{Adv} \text{ loses}] \geq \lfloor \frac{r-k}{k} \rfloor e^{-(2k\epsilon+\alpha)} \cdot \Pr[\mathcal{D}_{Adv} \text{ wins}]
 \end{align*}
 for any input subgroup $I \subset G_i, r = |G_i|$ and  $k < r/2$. 
 }
 
 \begin{proof}
 
  We first focus on deterministic adversaries and then expand to randomized adversaries afterwards using the fact that randomized adversaries are mixtures of deterministic ones. 

Our adversary $\mathcal{D}_{Adv}$ is then defined by a deterministic decision function $\eta: \calY^n \rightarrow [n]^k$. Upon observing $\bz$, $\eta(\bz)$ selects $k$ elements in $\bz$ which it believes originated from $I \subset G_i$. 
 
 In the following, let $\Pr_{\bz}$ be the probability of events conditioned on the shuffled output sequence $\bz$, where randomness is over the $\epsilon$-\ldp mechanism $\mathcal{M}$ and the $\alpha$-\name-private shuffling mechanism $\calA$. \footnote{As an abuse of notation, we assume the output space of the \ldp randomizers, $\calY$, have outcomes with non-zero measure e.g. randomized response. The following analysis can be expanded to continuous outputs (with outcomes of zero measure) by simply replacing the output sequence $\bz \in \calY^n$ with an output event $\mathbf{Z} \subseteq \calY^n$.}
 
 The adversary wins if it reidentifies $> \frac{k}{2}$ of the \ldp values originating from $I$. Let $H = \eta(\bz)$ be the indices of elements in $\bz$ selected by $\eta$. Let $W = \{\sigma \in \mathrm{S}^n : |\sigma(H) \cap I| > \frac{k}{2}\}$ be the set of permutations where the adversary wins and let $L = \{\sigma \in \mathrm{S}^n :\sigma(H) \cap I| \leq \frac{k}{2}\} $ be the set of permutations where the adversary loses. 
 \begin{align*}
     \Pr_{\bz} [\eta(\bz) \text{ wins}] &= \Pr_{\bz} [\sigma \in W] \\
     \Pr_{\bz} [\eta(\bz) \text{ loses}] &= \Pr_{\bz} [\sigma \in L] 
 \end{align*}
 where $\sigma$ is the shuffling permutation produced by $\calA$, $\bz = \sigma(\by)$ i.e. $z_i = y_{\sigma(i)}$. Concretely, this is equivalent to $\DO_i$ releasing $\DO_{\sigma(i)}$'s \ldp response. Since the permutation and \ldp outputs are randomized, many subgroups of size $k$ in $G_i$ could have produced the \ldp values $(z_{H_1}, \dots, z_{H_k})$ and then been mapped to $H$ by a permutation. Concretely, there is a reasonable probability that Alice's household output the \ldp values of another $k$-member household in her neighborhood and they output her household's \ldp values. In the worst case, this is $e^{-2k \epsilon}$ less likely than without swapping values, by group \DP guarantees. Since both households are part of the same group $G_i$, the permutation that maps her household to elements $H$ in the output is close in probability to that which maps the other household to elements $H$ in the output. As such, we have in the worst case a $e^{-(2k\epsilon + \alpha)}$ reduction in probability of the other household having swapped \ldp values with Alice's and permuting to subset $H$. 
 
 The above provides intuition on how we could get the same output $\bz$ many different ways, and how Alice's household could or could not contribute to elements $H$. It does not, however, explain why an adversary who is given output $\bz$ has limited advantage in choosing a subset $H$ such that they recover \emph{most} of Alice's household's values. We formalize this fact as follows. 
 
 We may rewrite the probabilities of winning or losing by marginalizing out all possible $\ldp$ sequences $\by$. Conditioning on the output sequence $\bz$, the only possible \ldp sequences $\by$ are permutations of $\bz$. Note that the probability of any sequence $\by$ is determined by the input $\bx$ and the \ldp mechanism $\mathcal{M}$:  
 \begin{align*}
    \Pr_\bz[\eta(\bz) \text{ loses}]
     &= \Pr_{\bz} [\sigma \in W] \\
     &= \sum_{\sigma \in W}
     \Pr [\calA(\bx) = \by = \sigma^{-1}(\bz)]  \Pr [ \sigma | \by ] / \Pr[\bz] 
 \end{align*}
 Note that $\Pr_\bz [\sigma | \by] = \Pr_\bz [\sigma]$ for the mallows mechanism, which chooses its permutations independently of $\by$. Now consider when $\eta(\bz)$ loses. By similar arguments as above: 
\begin{align*}
    \Pr_\bz[\eta(\bz) \text{ loses}]
    &= \Pr_\bz[\sigma \in L] \\
    &= \sum_{\sigma \in L}
     \Pr [\calA(\bx) = \by = \sigma^{-1}(\bz)]  \Pr [ \sigma | \by ] / \Pr[\bz]
\end{align*}
The odds of losing versus winning is given by 
 \begin{align*}
    \frac{\Pr_\bz[\eta(\bz) \text{ loses}]}{\Pr_\bz[\eta(\bz) \text{ wins}]}
    &= \frac{ \sum_{\sigma' \in L} \Pr [\calA(\bx) = \by = \sigma^{'-1}(\bz)]  \Pr [ \sigma' | \by ] }
    { \sum_{\sigma \in W} \Pr [\calA(\bx) = \by = \sigma^{-1}(\bz)]  \Pr [ \sigma | \by ] } \\
\end{align*}
We now show that for each $\sigma$ in the denominator, we may construct $m = \lfloor \frac{r-k}{k} \rfloor$ distinct permutations $\sigma'$ in the numerator that are close in probability to it. 
\begin{lemma}
For every $\sigma \in W$ there exists a set of $m = \lfloor \frac{r-k}{k} \rfloor$ permutations, $E(\sigma)$, such that 
\begin{enumerate}
    \item $E(\sigma) \subseteq L$
    \item $\sigma^{-1} \approx_{G_i} \sigma^{'-1}$ 
    \item $E(\sigma_a) \cap E(\sigma_b) = \emptyset$ for any pair $\sigma_a, \sigma_b \in W$
    \item $\Pr [\calA(\bx) = \by = \sigma^{-1}(\bz)] \leq e^{2k\epsilon} \Pr [\calA(\bx) = \by = \sigma^{'-1}(\bz)]$ for any $\bx \in \calX^n$ and any $\bz \in \calY^n$
\end{enumerate}
\end{lemma}
\begin{proof}
Given $\sigma \in W$, we construct $E(\sigma)$ by first taking the inverse $\sigma^{-1}$. Recall that, since $\sigma \in W$, we have that $|\sigma^{-1}(I) \cap H| > \frac{k}{2}$. ($\sigma^{-1}(i) = j$ could be interpreted as data owner $i$'s \ldp value will be output at position $j$). We then divide the remainder of the group $G_i \backslash I$ into $m$ disjoint subsets of size $k$ each, $J_1, J_2, \dots, J_m$. These represent the other distinct subsets of size $k$ that Alice's household could swap \ldp values with. We then produce $m$ permutations, $\sigma^{'-1}_1, \dots, \sigma^{'-1}_m$, by making $\sigma^{'-1}_i(I) = \sigma^{-1}(J_i)$ and $\sigma^{'-1}_i(J_i) = \sigma^{-1}(I)$ (preserving order within those subsets) and $\sigma^{'-1} = \sigma^{-1}$ everywhere else. 

On the first point, we know that every $\sigma' \in E(\sigma)$ is also in $L$. We know this because $\sigma^{'-1}_i(I) = \sigma^{-1}(J_i)$. Since $\sigma \in W$, we have that $|\sigma^{-1}(J_i) \cap H| < \frac{k}{2}$ since $|\sigma^{-1}(I) \cap H| \geq \frac{k}{2}$ and $I \cap J_i = \emptyset$ by definition. Thus, $|\sigma^{'-1}_i(I) \cap H| < \frac{k}{2}$, so $|\sigma'_i(H) \cap I| < \frac{k}{2}$ and $\sigma'_i \in L$. 

On the second point, we know that the inverse permutations are neighboring $\sigma^{-1} \approx_{G_i} \sigma^{'-1}$ simply by construction -- they only differ on elements in $G_i$. 

On the third point, we know that the sets $E(\sigma_a)$ and $E(\sigma_b)$ are distinct since we can map any permutation $\sigma' \in E(\sigma_a)$ uniquely back to $\sigma_a$ for any $\sigma_a \in W$. We do so by taking its inverse $\sigma^{'-1}$, finding which subset $J_i$ has majority elements from $H$ i.e. $|\sigma^{'-1}(J_i) \cap H| > \frac{k}{2}$. Swap elements back: $\sigma^{'-1}(J_i)$ with $\sigma^{'-1}(I)$. Invert back to $\sigma_a$. 

On the fourth point, we know that $\sigma^{-1}(\bz)$ and $\sigma^{'-1}(\bz)$ differ on at most $2k$ indices. As such, by group \DP guarantees, we know that their probabilities must be close to a factor of $e^{-2k\epsilon}$ regardless of $\bz$ and $\bx$. 
\end{proof}

Using the above Lemma we may bound the odds of losing vs. winning. 

\begin{align*}
    \frac{\Pr_\bz[\eta(\bz) \text{ loses}]}{\Pr_\bz[\eta(\bz) \text{ wins}]}
    &= \frac{ \sum_{\sigma' \in L} \Pr [\calA(\bx) = \by = \sigma^{'-1}(\bz)]  \Pr [ \sigma' | \by ] }
    { \sum_{\sigma \in W} \Pr [\calA(\bx) = \by = \sigma^{-1}(\bz)]  \Pr [ \sigma | \by ] } \\
    &\geq \frac{\sum_{\sigma \in W} \sum_{\sigma' \in E(\sigma)} \Pr [\calA(\bx) = \by = \sigma^{'-1}(\bz)]  \Pr [ \sigma' | \by ]}
    {\sum_{\sigma \in W} \Pr [\calA(\bx) = \by = \sigma^{-1}(\bz)]  \Pr [ \sigma | \by ]} \\
    &\geq \min_{\sigma \in W} \frac{\sum_{\sigma' \in E(\sigma)} \Pr [\calA(\bx) = \by = \sigma^{'-1}(\bz)]  \Pr [ \sigma' | \by ]}{\Pr [\calA(\bx) = \by = \sigma^{-1}(\bz)]  \Pr [ \sigma | \by ]} \\
    &\geq \lfloor \frac{r-k}{k} \rfloor e^{-(2k\epsilon + \alpha)}
\end{align*}
where the last line follows from the fourth point of the above Lemma (for the $2k\epsilon$ term) and the fact that the inverse permutations $\sigma'^{-1}, \sigma^{-1}$ are neighboring (second point of the Lemma) so the probabilities of the mechanism to produce $\sigma$ vs. $\sigma'$ to reach $\bz$ from these neighboring permutations must be close by a factor of $e^{\alpha}$. 

Since the above holds for any $\bz$ and $\bx$, the bound holds on average across all outcomes $\bz$, thus 
\begin{align*}
    \Pr[\eta \text{ loses}] \geq \lfloor \frac{r-k}{k} \rfloor e^{-(2k\epsilon+\alpha)} \cdot \Pr[\eta \text{ wins}]
\end{align*}
for any deterministic adversary with decision function $\eta$. Finally, we may write any probabilistic adversary as mixture of decision functions. By convexity (same argument used in Lemma \ref{lem:convexity}), the above bound still holds. As such, 

\begin{align*}
    \Pr[\mathcal{D}_{Adv} \text{ loses}] \geq \lfloor \frac{r-k}{k} \rfloor e^{-(2k\epsilon+\alpha)} \cdot \Pr[\mathcal{D}_{Adv} \text{ wins}]
\end{align*}

 \end{proof}
 
 \subsection{Utility of Shuffling Mechanism}\label{app:utility}
 We now introduce a novel metric, $(\eta,\delta)$-preservation, for assessing the utility of any shuffling mechanism. Let $S\subseteq [n]$ correspond to a set of indices in $\by$. The metric is defined as follows.
%representing data owners in Alice's neighborhood for instance.  
%$(\eta,\delta)$-preservation measures how well the shuffling mechanism preserves the original indices in $S$ after shuffling, i.e. the fraction of data owners in Alice's neighborhood that still correspond to datapoints from the neighborhood after shuffling:
\begin{defn}($(\eta,\delta)$-preservation) A shuffling mechanism $\calA:\calY^n\mapsto\calY^n$ is defined to be $(\eta,\delta)$-preserving $(\eta, \delta 
\in [0,1])$ w.r.t to a given subset $S\subseteq [n]$, if \begin{gather}\Pr\big[|S_{\sigma}\cap S|\geq \eta\cdot|S|\big]\geq 1-\delta,  \sigma \in \mathrm{S}_n\end{gather} where $\bz=\calA(\by)=\sigma(\by)$ and $S_{\sigma}=\{\sigma(i)|i \in S\}$. \label{def:utility} 
% \vspace{-0.2cm}
\end{defn}
For example, consider \scalebox{0.9}{$S=\{1,4,5,7,8\}$}. If \scalebox{0.9}{$\calA(\cdot)$} permutes the output according to  \scalebox{0.9}{$\sigma=(\underline{5}\:3\:2\:\underline{6}\:\underline{7}\:9\:\underline{8}\:\underline{1}\:4\:10)$}, then  \scalebox{0.9}{$S_{\sigma}=\{5,6,7,8,1\}$}  which preserves \scalebox{0.9}{$4$} or \scalebox{0.9}{$80\%$} of its original indices.  This means that for any data sequence $\by$, at least \scalebox{0.9}{$\eta$} fraction of its data values corresponding to the subset \scalebox{0.9}{$S$} overlaps with that of shuffled sequence $\bz$ with high probability \scalebox{0.9}{$(1-\delta)$}. Assuming, \scalebox{0.9}{$\{y_S\}=\{y_{i}|i
\in S\}$} and \scalebox{0.9}{$\{z_S\}=\{z_i|i \in S\}=\{y_{\sigma(i)}| i \in S\}$} denotes the set of data values corresponding to $S$ in data sequences $\by$ and $\bz$ respectively, we  have \scalebox{0.9}{$\Pr\big[|\{y_S\}\cap \{z_S\}|\geq \eta \cdot |S|\big]\geq 1-\delta, \: \forall \by $}.
For example, let $S$ be the set of individuals from Nevada. Then, for a shuffling mechanism that provides \scalebox{0.9}{$(\eta =0.8, \delta=0.1)$}-preservation to $S$, with probability \scalebox{0.9}{$\geq 0.9$}, \scalebox{0.9}{$\geq 80\%$} of the values that are reported to be from Nevada in $\bz$ are genuinely from Nevada. The rationale behind this metric is that it captures the utility of the learning allowed by \name-privacy -- if $S$ is equal to some group \scalebox{0.9}{$G \in \calG$}, \scalebox{0.9}{$(\eta, \delta)$} preservation allows overall statistics of \scalebox{0.9}{$G$} to be captured. Note that this utility metric is \textit{agnostic of both the data distribution and the analyst's query}. Hence, it is a conservative analysis of utility which serves as a lower bound for learning from $\{z_S\}$. We suspect that with the knowledge of the data distribution and/or the query, a tighter utility analysis is possible. \\
A formal utility analysis of Alg. \ref{algo:main} is presented in App. \ref{app:utility:formal}. Empirical evaluation of $(\eta,\delta)$ - preservation is presented in App. \ref{app:extraresults}. 
\subsection{Discussion on Properties of Mallows Mechanism}\label{app:prop}

% \textbf{Property \ref{prop:1}} 
\begin{prope}
\label{prop:1}
% \emph{
For group assignment $\calG$, a  mechanism $\calA(\cdot)$ that shuffles according to a permutation sampled from the Mallows model $\mathbb{P}_{\theta,\textswab{d}}(\cdot)$, satisfies $(\alpha, \calG)$-\name privacy where
\begin{align*}
 \Delta(\sigma_0 : \textswab{d}, \calG) &= \max_{(\sigma, \sigma') \in N_\calG} |\textswab{d}(\sigma_0 \sigma, \sigma_0) - \textswab{d}(\sigma_0 \sigma', \sigma_0)|\\
 \text{and}&\\
    \alpha 
    &= \theta \cdot \Delta(\sigma_0 : \textswab{d}, \calG)
\end{align*} 
We refer to $\Delta(\sigma_0 : \textswab{d}, \calG) $ as the sensitivity of the rank-distance measure $\textswab{d}(\cdot)$
\end{prope}
% }
\begin{proof}
% Start with a useful lemma. 
% \begin{lemma}
% Consider any pair of neighboring permutations $\sigma \approx_{G_i} \sigma'$ w.r.t. group $G_i$. Consider any third permutation $\sigma^* \in \mathrm{S}_n$. Then, the permutations that turn $\sigma$ and $\sigma'$ into $\sigma^*$ are also neighboring w.r.t. $G_i$. Formally, if 
% \begin{align*}
%     \sigma^*(\by) &= \sigma_a \big( \sigma(\by) \big) \\
%     \sigma^*(\by) &= \sigma_b \big( \sigma'(\by) \big)  
% \end{align*}
% then, $\sigma_a \approx_{G_i} \sigma_b$.

Consider two permutations of the initial sequence $\by$, $\sigma_1(\by), \sigma_2(\by)$ that are neighboring w.r.t. some group $G_i \in \calG$, $\sigma_1 \approx_{G_i} \sigma_2$. Additionally consider any fixed released shuffled sequence $\bz$. Let $\Sigma_1, \Sigma_2$ be the set of permutations that turn $\sigma_1(\by), \sigma_2(\by)$ into $\bz$, respectively: 
\begin{align*}
    \Sigma_1 
    & = \{\sigma \in \mathrm{S}_n : \sigma \sigma_1(\by) = \bz \} \\
    \Sigma_2 
    & = \{\sigma \in \mathrm{S}_n : \sigma \sigma_2(\by) = \bz \} \quad .
\end{align*}
In the case that $\{y\}$ consists entirely of unique values, $\Sigma_1, \Sigma_2$ will each contain exactly one permutation, since only one permutation can map $\sigma_i(\by)$ to $\bz$. 

\begin{lemma}
For each permutation $\sigma_1' \in \Sigma_1$ there exists a permutation in $\sigma_2' \in \Sigma_2$ such that 
\begin{align*}
    \sigma_1' \approx_{G_i} \sigma_2' \quad . 
\end{align*}
\end{lemma}
Proof follows from the fact that --- since only the elements $j \in G_i$ differ in $\sigma_1(\by)$ and $\sigma_2(\by)$ --- only those elements need to differ to achieve the same output permutation. In other words, we may define $\sigma_1', \sigma_2'$ at all inputs $i \notin G_i$ identically, and then define all inputs $i \in G_i$ differently as needed. As such, they are neighboring w.r.t. $G_i$. 

Recalling that Alg. 1 applies $\sigma_0^{-1}$ to the sampled permutation, we must sample $\sigma_0\sigma_1'$ (for some $\sigma_1' \in \Sigma_1$) for the mechanism to produce $\bz$ from $\sigma_1(\by)$. Formally, since $\sigma_1' \sigma_1 (\by) = \bz$ we must sample $\sigma_0 \sigma_1'$ to get $\bz$ since we are going to apply $\sigma_0^{-1}$ to the sampled permutation. 
\begin{align*}
    \Pr\big[ \calA \big( \sigma_1(\by) \big) = \bz \big] 
    &= \mathbb{P}_{\theta,\textswab{d}}\big(\sigma_0\sigma', \sigma' \in \Sigma_1 : \sigma_0\big) \\
    \Pr\big[ \calA \big( \sigma_2(\by) \big) = \bz \big] 
    &= \mathbb{P}_{\theta,\textswab{d}}\big(\sigma_0\sigma', \sigma' \in \Sigma_2 : \sigma_0\big) 
\end{align*}

Taking the odds, we have
\begin{align*}
     \frac{\mathbb{P}_{\theta,\textswab{d}}\big(\sigma_0\sigma', \sigma' \in \Sigma_1 : \sigma_0\big)}{
     \mathbb{P}_{\theta,\textswab{d}}\big(\sigma_0\sigma'', \sigma'' \in \Sigma_2 : \sigma_0\big)} 
    &= \frac{\sum_{\sigma' \in \Sigma_1}\mathbb{P}_{\Theta,\textswab{d}}(\sigma_0\sigma' : \sigma_0)}{
    \sum_{\sigma'' \in \Sigma_2}\mathbb{P}_{\Theta,\textswab{d}}(\sigma_0\sigma'' : \sigma_0)} \\
    &=\frac{\sum_{\sigma' \in \Sigma_1} e^{-\theta \textswab{d}(\sigma_0\sigma', \sigma_0)}}{
    \sum_{\sigma'' \in \Sigma_2} e^{-\theta \textswab{d}(\sigma_0\sigma'', \sigma_0)}}\\
    &\leq \frac{e^{-\theta \textswab{d}(\sigma_0\sigma_a, \sigma_0)}}{e^{-\theta \textswab{d}(\sigma_0\sigma_b, \sigma_0)}} \\
    &\leq e^{\theta|  \textswab{d}(\sigma_0\sigma_a, \sigma_0) - \textswab{d}(\sigma_0\sigma_b, \sigma_0) |} \\
    &\leq e^{\theta \Delta}
\end{align*}
where 
\begin{align*}
    \sigma_a &= \arg \max_{\sigma' \in \Sigma_1} e^{-\theta \textswab{d}(\sigma_0 \sigma', \sigma_0)} 
    \text{ and } \\
    \sigma_a &= \arg \min_{\sigma'' \in \Sigma_2} e^{-\theta \textswab{d}(\sigma_0 \sigma'', \sigma_0)} ~.
\end{align*}
Therefore, setting $\alpha = \theta \cdot \Delta$, we achieve $(\alpha, \calG)$-\name privacy. 
\end{proof}

% \textbf{Property \ref{prop:2}}
% \emph{
\begin{prope}
\label{prope:2}
The sensitivity of a rank-distance is an increasing function of the width $\omega_{\calG}^{\sigma_0}$. For instance, for Kendall's $\tau$ distance $\textswab{d}_\tau(\cdot )$, we have 
$\Delta(\sigma_0 : \textswab{d}_\tau, \calG)
    =\frac{\omega_{\calG}^{\sigma_0}(\omega_{\calG}^{\sigma_0} + 1)}{2}$. 
% }
\end{prope}
%  \begin{lemma}
%  For any two neighboring permutations $\sigma \approx_{\calG} \sigma'$: 
%  \begin{align*}
%      \bigg| \log \frac{\mathbb{P}_{\Theta,\textswab{d}}(\sigma:\sigma_0)}{\mathbb{P}_{\Theta,\textswab{d}}(\sigma':\sigma_0)} \bigg| 
%      &\leq \theta \Delta(\sigma_0 : \textswab{d}, \calG) 
%  \end{align*}\label{lem:prop1}
%  \end{lemma}
 
%  The above lemma essentially gives the proof for Prop. \ref{prop:1}.
 
%  \begin{lemma}For Kendall's $\tau$ distance,
%  \begin{align*}
%      \bigg| \log \frac{\mathbb{P}_{\Theta,\textswab{d}}(\sigma_s:\sigma_0)}{\mathbb{P}_{\Theta,\textswab{d}}(\sigma_s':\sigma_0)} \bigg| \leq \theta \Delta(\sigma_0 : \textswab{d}, \calG)
%  \end{align*}\end{lemma}

To show the sensitivity of Kendall's $\tau$, we make use of its triangle inequality. 

 \begin{proof}
 Recall from the proof of the previous property that the expression $\textswab{d}(\sigma, \sigma_0) = \textswab{d}\big( \sigma_0\sigma, \sigma_0 \big)$, where $\textswab{d}$ is the actual rank distance measure e.g. Kendall's $\tau$. As such, we require that 
 
 \begin{align*}
     \big|  \textswab{d}(\sigma_0\sigma_a, \sigma_0) - \textswab{d}(\sigma_0\sigma_b, \sigma_0) \big|
     &\leq \frac{\omega_{\calG}^{\sigma_0}(\omega_{\calG}^{\sigma_0} + 1)}{2}
 \end{align*}

for any pair of permutations $(\sigma_a, \sigma_b) \in N_\calG$. 
 
 For any group $G_i \in \calG$, let $W_i \subseteq n$ represent the smallest contiguous subsequence of indices in $\sigma_0$ that contains all of $G_i$. 

For instance, if $\sigma_0 = [2,4,6,8,1,3,5,7]$ and $G_i = \{2,6,8\}$, then $W_i = \{2,4,6,8\}$. Then the group width width is $\omega_i = |W_i| - 1 = 3$. Now consider two permutations neighboring w.r.t. $G_i$, $\sigma_a \approx_{G_i} \sigma_b$, so only the elements of $G_i$ are shuffled between them. We want to bound 
\begin{align*}
    \big|  \textswab{d}(\sigma_0\sigma_a, \sigma_0) - \textswab{d}(\sigma_0\sigma_b, \sigma_0) \big| 
\end{align*}
For this, we use a pair of triangle inequalities: 
\begin{align*}
    \textswab{d}(\sigma_0\sigma_a, \sigma_0\sigma_b) 
    &\geq \textswab{d}(\sigma_0\sigma_a, \sigma_0) - \textswab{d}(\sigma_0\sigma_b, \sigma_0) 
    \quad \& \quad 
    \textswab{d}(\sigma_0\sigma_a, \sigma_0\sigma_b) 
    &\geq \textswab{d}(\sigma_0\sigma_b, \sigma_0) - \textswab{d}(\sigma_0\sigma_a, \sigma_0)
\end{align*}
so, 
\begin{align*}
     \big|  \textswab{d}(\sigma_0\sigma_a, \sigma_0) - \textswab{d}(\sigma_0\sigma_b, \sigma_0) \big| &\leq \textswab{d}(\sigma_0\sigma_a, \sigma_0\sigma_b)
\end{align*}


% We make use of the fact that by only permuting the contents of $W_i$ in $\sigma_0$, we can construct some $\sigma_0'$ such that $\textswab{d}(\sigma , \sigma_0) = \textswab{d}(\sigma' , \sigma_0')$. By the triangle inequality, we have
% \begin{align*}
% \textswab{d}(\sigma' , \sigma_0) &\leq \textswab{d}(\sigma' , \sigma_0') + \textswab{d}(\sigma_0 , \sigma_0') \\
% &= \textswab{d}(\sigma , \sigma_0) + \textswab{d}(\sigma_0 , \sigma_0') \\
% \textswab{d}(\sigma' , \sigma_0) - \textswab{d}(\sigma , \sigma_0) 
% &\leq \textswab{d}(\sigma_0 , \sigma_0')
% \end{align*}
% Thus,
% \begin{align*}
%     |\textswab{d}(\sigma , \sigma_0) - \textswab{d}(\sigma' , \sigma_0) |
%     &\leq |\textswab{d}(\sigma_0 , \sigma_0')|
% \end{align*}
Since $\sigma_0\sigma_a$ and $\sigma_0\sigma_b$ only differ in the contiguous subset $W_i$, the largest number of discordant pairs between them is given by the maximum Kendall's $\tau$ distance between two permutations of size $\omega_i + 1$:  
\begin{align*}
    |\textswab{d}(\sigma_0\sigma_a , \sigma_0\sigma_b)|
    &\leq \frac{\omega_i(\omega_i + 1)}{2}
\end{align*}
Since $\omega_{\calG}^{\sigma_0} \geq \omega_i$ for all $G_i \in \calG$, we have that 
\begin{align*}
    \Delta(\sigma_0 : \textswab{d}, \calG) \leq 
    \frac{\omega_{\calG}^{\sigma_0}(\omega_{\calG}^{\sigma_0} + 1)}{2}
\end{align*}
\end{proof}
 



\subsection{Hardness of Computing The Optimum Reference Permutation}\label{app:NP}
\begin{thm} The problem of finding the optimum reference permutation, i.e., $\sigma^*_0=\arg\min_{\sigma\in \mathrm{S}_n}\omega_{\calG}^{\sigma}$ is NP-hard. \label{thm:NP} \end{thm}
\begin{proof} We start with the formal
representation of the problem as follows.

\textit{Optimum Reference Permutation Problem.} Given n subsets $\calG=\{G_i\in 2^{[n]}, i \in [n]\}$,  find the permutation $\sigma^*_0=\arg\min_{\sigma\in \mathrm{S}_n}\omega_{\calG}^{\sigma}$.  

Now, consider the following job-shop scheduling problem.

\textit{Job Shop Scheduling.} There is one job $J$ with $n$ operations $o_i, i \in [n]$ and $n$ machines such that $o_i$ needs to run on machine $M_i$.  Additionally, each machine has a sequence dependent processing time $p_i$. Let $S$ be the sequence till 
There are $n$ subsets $S_i \subseteq [n]$, each corresponding to a set of operations that need to occur in contiguous machines, else the processing times incur penalty as follows. Let $p_i$ denote the processing time for the machine running the $i$-th operation scheduled. Let $\mathbb{S}_i$ be the prefix sequence  with $i$ schedulings. For instance, if the final scheduling is $ 1\:3\:4\:5\:9\:8\:10\:6\:7\:2$ then $\mathbb{S}_4=1345$. Additionally, let $P^j_{\mathbb{S}_i}$ be the shortest subsequence such of $\mathbb{S}_i$ such that it contains all the elements in $S_j \cap \{\mathbb{S}_{i}\}$. For example for $S_1=\{3,5,7\}$, $P^1_{\mathbb{S}_{4}}=345$. 
\begin{gather}p_i=\max_{i\in [n]}(|P^j_{\mathbb{S}_i}|-|S_j \cap \{\mathbb{S}_i\}|)\end{gather}
 The objective is to find a scheduling for $J$ such that it minimizes the makespan, i.e., the completion time of the job. Note that $p_n=\max_i{p_i}$, hence the problem reduces to minimizing $p_n$.

\begin{lemma}The aforementioned  job shop scheduling problem with sequence-dependent processing time is NP-hard.\end{lemma}
\begin{proof} Consider the following instantiation of the sequence-dependent job shop scheduling problem where the processing time is given by $p_i$=$p_{i-1}+w_{kl}, p_1=0$ where $\mathbb{S}_i[i-1]=k$, $\mathbb{S}_i[i]=l$ and $w_{ij}, j\in S_i$  represents some associated weight.    This problem is equivalent to the travelling salesman problem (TSP) \cite{TSP} and is therefore, NP-hard. Thus, our aforementioned job shop scheduling problem is also clearly NP-hard. \end{proof}

\textit{Reduction:}
Let the $n$ subsets $S_i$ correspond to the groups in $\calG$. Clearly, minimizing $\omega^{\sigma}_{\calG}$ minimizes $p_n$. Hence, the optimal reference permutation gives the solution to the scheduling problem as well. 

 \end{proof}
 
 
 \begin{figure}[h]
\begin{subfigure}[b]{\columnwidth}\centering
    \includegraphics[width=0.7\linewidth]{./figures/BFS_graph.png}
        \caption{Group graph}
        \label{fig:BFS:graph}
    \end{subfigure}\\
    \begin{subfigure}[b]{\columnwidth}\centering
    \includegraphics[width=0.5\columnwidth]{./figures/BFS_order.png}
        \caption{BFS reference permutation $\sigma_0$}
        \label{fig:BFS:order}
    \end{subfigure}
    \caption{Illustration of Alg. 1}
    \label{fig:alg:example}
\end{figure}
 
 
 \subsection{Illustration of Alg. 1}\label{app:alg:illustration}
 We now provide a small-scale step-by-step example of how Alg. 1 operates. 
 
 Fig. \ref{fig:BFS:graph} is an example of a grouping $\calG$ on a dataset of $n = 8$ elements. The group of $\DO_i$ includes $i$ and its neighbors. For instance, $G_8 = \{8,3,5\}$. To build a reference permutation, Alg. 1 starts at the index with the largest group, $i = 5$ (highlighted in purple), with $G_5 = \{5,2,3,8,4\}$. As shown in Figure \ref{fig:BFS:order}, the $\sigma_0$ is then constructed by following a BFS traversal from $i=5$. Each $j \in G_5$ is visited, queuing up the neighbors of each $j \in G_5$ that haven't been visited along the way, and so on. The algorithm completes after the entire graph has been visited. 
 
 The goal is to produce a reference permutation in which the width of each group in the reference permutation $\omega_i$ is small. In this case, the width of the largest group $G_5$ is as small as it can be $\omega_5 =5-1 = 4$. However, the width of $G_4 = \{4,5,7\}$ is the maximum possible since $\sigma^{-1}(5) = 1$ and $\sigma^{-1}(7) = 8$, so $\omega_4 = 7$. This is difficult to avoid when the maximum group size is large as compared to the full dataset size $n$. Realistically, we expect $n$ to be significantly larger, leading to relatively smaller groups. 
 
 With the reference permutation in place, we compute the sensitivity: 
 \begin{align*}
     \Delta(\sigma_0 : \textswab{d}, \calG)
     &= \frac{\omega_4 (\omega_4 + 1)}{2} \\
     &= 28
 \end{align*}
 Which lets us set $\theta = \frac{\alpha}{28}$ for any given $\alpha$ privacy value. To reiterate, lower $\theta$ results in more randomness in the mechanism. 
 
 We then sample the permutation $\hat{\sigma} = \mathbb{P}_{\theta, \textswab{d}}(\sigma_0)$. Suppose 
 \begin{align*}
     \hat{\sigma}
     &= [3 \ 2\ 5\ 4\ 8\ 1\ 7\ 6]
 \end{align*}
 Then, the released $\bz$ is given as 
  \begin{align*}
     \bz = \sigma^* &= \sigma^{-1} \hat{\sigma} (\by)\\
     &= [y_1 \ y_2\ y_5\ y_8\ y_3\ y_7\ y_6\ y_4]
 \end{align*}
 One can think of the above operation as follows. What was 5 in the reference permutation ($\sigma_0(1) = 5$) is 3 in the sampled permutation $(\hat{\sigma}(1) = 3)$. So, index 5 corresponding to $\DO_5$ now holds $\DO_3$'s noisy data $y_3$. As such, we shuffle mostly between members of the same group, and minimally between groups. 

%\newcommand{\calO}{\mathcal{O}}

 The runtime of this mechanism is dominated by the Repeated Insertion Model sampler \cite{RIM}, which takes $\calO(n^2)$ time. It is very possible that there are more efficient samplers available, but RIM is a standard and simple to implement for this first proposed mechanism. Additionally, the majority of this is spent computing sampling parameters which can be stored in advanced with $\calO(n^2)$ memory. Furthermore, sampling from a Mallows model with some reference permutation $\sigma_0$ is equivalent to sampling from a Mallows model with the identity permutation and applying it to $\sigma_0$. As such, permutations may be sampled in advanced, and the runtime is dominated by computation of $\sigma_0$ which takes $\calO(|V| + |E|)$ time (the number of vertices and edges in the graph). 
 
 A future direction could be exploring even better heuristics for computing $\sigma_0$. One possibility could be based on ranked enumeration of the permutations \cite{deep2021ranked,deep2021enumeration}.
 \subsection{Proof of Thm. \ref{thm:privacy}}\label{app:thm:privacy}
 \textbf{Theorem \ref{thm:privacy}}
 \emph{Alg. 1 is $(\alpha,\calG)$-\name~private. }
 \begin{proof} The proof follows from Prop. \ref{prop:1}. Having computed the sensitivity of the reference permutation $\sigma_0$, $\Delta$, and set $\theta = \alpha / \Delta$, we are guaranteed by Property \ref{prop:1} that shuffling according to the permutation $\hat{\sigma}$ guarantees $(\alpha, \calG)$-\name privacy. 
 
%  In the algorithm, we permute by $\sigma_0^{-1} \hat{\sigma}$. Since this is equivalent to first permuting by $\hat{\sigma}$ and then permuting by $\sigma_0^{-1}$, this too guarantees $(\alpha, \calG)$-\name privacy by the immunity to post-processing property (Thm. \ref{theorem:post}).
 \end{proof}.
% \arc{rearranged the proof a bit, refer to lemma \ref{lem:prop1} to prove this, should be straighforward just need to argue that $\sigma^*$ and $\hat{\sigma}$ have same probabilistic distribution (they have 1:1 mapping) which I think you try to show below. Also no need to refer to Kendall's tau here since the Alg. 1 is generic}
%  Start with a Lemma: 

%  (Proof essentially by definition of $\Delta$) 
 
%  Add a prop
%  \begin{prop}
%  If $\sigma \approx_{\calG} \sigma'$, then $\gamma \sigma \approx_{\calG} \gamma \sigma'$ for any $\gamma \in \mathrm{S}$. 
%  \end{prop}
%  \begin{proof}
%  By def'n, for some group $G_i \in \calG$, $\sigma(j) = \sigma'(j)$ for all $j \notin G_i$, then $\gamma(\sigma(j)) = \gamma(\sigma(j))$ for all $j \notin G_i$.   
%  \end{proof}
 
%  Main proof of \ref{thm: privacy} (sketch). We want to show that $\Pr[\bz | \by_{\sigma}] \leq e^\alpha \Pr[\bz | \by_{\sigma'}]$, where $\sigma \approx_{\calG} \sigma'$. Recall from the algorithm that we sample some $\sigma_s$ from Mallows given ref permutation $\sigma_0$, and apply $\sigma_0^{-1} \sigma_s$ to $\by$. Define $\sigma_a$ as the permutation that goes from $\by_{\sigma}$ to $\bz$, $\sigma_z = \sigma_a \sigma$ and $\sigma_z = \sigma_b \sigma'$ for $\by_{\sigma'}$. We then have that 
%  \begin{align*}
%      \sigma_0 \sigma_a &= \sigma_s \\
%      \sigma_0 \sigma_b &= \sigma_s'
%  \end{align*}
%  So, $\sigma_s$ must be sampled to get from $\by_\sigma$ to $\bz$ and $\sigma_s'$ needs to be sampled to get from $\by_{\sigma'}$ to $\bz$. From the above proposition we have that $\sigma_s \approx_{\calG} \sigma_s'$. It follows from the above lemma that 
%  \begin{align*}
%      \bigg| \log \frac{\mathbb{P}_{\Theta,\textswab{d}}(\sigma_s:\sigma_0)}{\mathbb{P}_{\Theta,\textswab{d}}(\sigma_s':\sigma_0)} \bigg| \leq \theta \Delta(\sigma_0 : \textswab{d}, \calG)
%  \end{align*}
% So, setting $\theta = \frac{\alpha}{\Delta(\sigma_0: \textswab{d}, \calG)}$ satisifies our privacy definition. 

% It remains to show that for Kendall $\tau$ that 
% \begin{align*}
%     \Delta(\sigma_0 : \textswab{d}, \calG)
%     &= \frac{w (w+1)}{2}
% \end{align*}

%  \begin{proof}
%  \begin{align*}
%     \bigg| \log \frac{\mathbb{P}_{\Theta,\textswab{d}}(\sigma:\sigma_0)}{\mathbb{P}_{\Theta,\textswab{d}}{(\sigma':\sigma_0)} \bigg| 
%     &\leq \theta | \textswab{d}(\sigma, \sigma_0) - \textswab{d}(\sigma', \sigma_0) |
%  \end{align*}
%  \end{proof}


 
%  The key component of the proof is that the ratio of probabilities 
%  \begin{align*}
%      \frac{\Pr[\bz | \by_{\sigma'}]}{\Pr[\bz | \by_\sigma]}
%  \end{align*}
% for any neighboring permutations $\sigma, \sigma'$ is given by 
%  \begin{align*}
%     \log \frac{\Pr[\bz | \by_{\sigma'}]}{\Pr[\bz | \by_\sigma]}
%      &= \log \frac{\exp (-\theta \textswab{d}(\sigma, \sigma_0))} 
%      {\exp (-\theta \textswab{d}(\sigma', \sigma_0)} \\
%      &\leq \theta \big| \textswab{d}(\sigma, \sigma_0) - \textswab{d}(\sigma', \sigma_0) \big| 
%  \end{align*}
 
\subsection{Proof of Thm. \ref{thm:generalized:privacy}} \label{app:thm:generalized}

\textbf{Theorem \ref{thm:generalized:privacy}} \emph{
Alg. 1 satisfies $(\alpha',\calG')$-\name privacy for any group assignment $\calG'$ where  $ \alpha'=\alpha\frac{\Delta(\sigma_0 : \textswab{d}, \calG')}{\Delta(\sigma_0 : \textswab{d}, \calG)}$ 
}
\begin{proof}
Recall from Property \ref{prop:1} that we satisfy $(\alpha, \calG)$ \name-privacy by setting $\theta = \alpha / \Delta(\sigma_0:\textswab{d}, \calG)$. Given alternative grouping $\calG'$ with sensitivity $\Delta(\sigma_0:\textswab{d}, \calG')$, this same mechanism provides 
\begin{align*}
    \alpha' 
    &= \frac{\theta}{\Delta(\sigma_0:\textswab{d}, \calG')} \\
    &= \frac{\alpha / \Delta(\sigma_0:\textswab{d}, \calG)}{\Delta(\sigma_0:\textswab{d}, \calG')} \\
    &= \alpha \frac{\Delta(\sigma_0:\textswab{d}, \calG')}{\Delta(\sigma_0:\textswab{d}, \calG)}
\end{align*}
\end{proof}


\subsection{Formal Utility Analysis of Alg. 1}\label{app:utility:formal}
\begin{thm}For a given set $S \subset [n]$ and Hamming distance metric,  $\textswab{d}_H(\cdot)$,   Alg. 1 is $(\eta,\delta)$-preserving for $\delta=\frac{1}{\psi(\theta, \textswab{d}_H)}\sum_{h=2k+1}^{n} (e^{-\theta\cdot h} \cdot c_h)$ where \scalebox{0.9}{$k=\lceil(1-\eta)\cdot |S|\rceil$} and $c_h$ is the number of permutations with hamming distance $h$ from the reference permutation that do not preserve \scalebox{0.9}{$\eta\%$} of $S$ and is given by
\begin{gather*}c_h=\sum_{j=k+1}^{\max(l_s,\lfloor h/2\rfloor)}\binom{l_s}{j}\cdot \binom{n-l_s}{j}\cdot \Bigg[\sum_{i=0}^{\min(l_s-j,h-2j)}\binom{l_s-j}{i}\\\cdot \binom{i+j}{j}\cdot f(i,j)\cdot\binom{n-l_s-j}{h-2j-i} \cdot f(h-2j-i,j)!\Bigg]\end{gather*}\begin{gather*}
f(i,0)=!i, f(0,q)=q!\\
f(i,j)=\sum_{q=0}^{\min(i,j)}\Bigg[\binom{i}{q}\cdot\binom{j}{j-q}\cdot j! \cdot f(i-q,q)\Bigg]\\l_s=|S|, k=(1-\eta)\cdot l_s, !n=\lfloor \frac{n!}{e}+\frac{1}{2}\rfloor\end{gather*}\end{thm}
\begin{proof} Let $l_s=|S|$ denote the size of the set $S$ and $k=\lceil (1-\eta)\cdot l_S\rceil$ denote the maximum number of correct values that can be missing from $S$. Now, for a given permutation $\sigma \in \mathrm{S}_n$, let $h$ denote its Hamming distance from the reference permutation $\sigma_0$, i.e, $h=\textswab{d}_H(\sigma,\sigma_0)$. This means that $\sigma$ and $\sigma_0$ differ in $h$ indices.  Now, $h$ can be analysed in the  the following two cases, 
\par
\textbf{Case I. $h\leq 2k+1$}

For $(1-\eta)$ fraction of indices to be removed from $S$, we need at least $k+1$ indices from $S$ to be replaced by $k+1$ values from outside $S$. This is clearly not possible for $h\leq 2k+1$. Hence, here $c_h=0$. 
\par
\textbf{Case II. $h > 2k$}

For the following analysis we consider we treat the permutations as strings (multi-digit numbers are treated as a single string character). Now,   
 Let $\mathbb{S}_{\sigma_0}$ denote the non-contiguous substring of $\sigma_0$ such that it consists of all the  elements of $S$, i.e., \begin{gather}|\mathbb{S}|=l_S\\
 \forall i \in [l_S], \mathbb{S}_{\sigma_0}[i] \in S \end{gather}  Let $\mathbb{S}_{\sigma}$ denote the substring corresponding to the positions occupied by $\mathbb{S}_{\sigma_0}$ in $\sigma$. Formally,
 \begin{gather}|\mathbb{S}_{\sigma}|=l_S\\
 \forall i \in [l_S], \mathbb{S}_{\sigma_0}[i] = \sigma(\sigma_0^{-1}(\mathbb{S}_{\sigma_0}[i])) \end{gather}  For example, for $\sigma_0=(1\:2\:3\:5\:4\:7\:8\:10\:9\:6), \sigma=(1\:3\:2\:7\:8\:5\:4\:6\:10\:9)$ and $S=\{2,4,5,8\}$, we have $\mathbb{S}_{\sigma_0}=2548$ and $S_{\sigma}=3784$ where $h=\textswab{d}_H(\sigma,\sigma_0)=9$.  Let $\{\mathbb{S}_{\sigma}\}$ denote the set of the elements of string $\mathbb{S}_{\sigma}$.
 Let $A$ be the set of characters in  $\mathbb{S}_{\sigma}$ such that they do not belong to $S$, i.e, $A=\{\mathbb{S}_{\sigma}[i]|\mathbb{S}_{\sigma}[i] \not \in S, i \in [l_S]\}$. Let $B$ be the set of characters in $\mathbb{S}_{\sigma}$ that belong to $S$ but differ from $\mathbb{S}_{\sigma_0}$ in position, i.e., $B=\{\mathbb{S}_{\sigma}[i]|\mathbb{S}_{\sigma}[i] \in S, \mathbb{S}_{\sigma}[i]\neq \mathbb{S}_{\sigma_0}[i], i \in [l_S]\}$. Additionally, let $C=S-\{\mathbb{S}_{\sigma}\}$. For instance, in the above example, $A=\{3,7\}, B=\{4,8\}, C=\{2,5\}$. Now consider  an initial arrangement of $p+m$ distinct objects that are subdivided into two types -- $p$ objects of Type A and m objects of Type B. Let $f(p,m)$ denote the number of permutations of these $p+m$ objects such that the $m$ Type B objects can occupy any position but no object of Type A can occupy its original position. For example, for $f(p,0)$ this becomes the number of derangements \cite{derangement} denoted as $!p=\lfloor \frac{p!}{e} +\frac{1}{2} \rfloor$. Therefore, $f(|B|,|A|)$ denotes the number of permutations of $\mathbb{S}_\sigma$ such that $\textswab{d}_H(\mathbb{S}_{\sigma_0},\mathbb{S}_{\sigma})=|A|+|B|$. This is because if elements of $B$ are allowed to occupy their original position then this will reduce the Hamming distance.  
 \par Now, let $\bar{\mathbb{S}}_{\sigma}$ ($\bar{\mathbb{S}}_{\sigma_0}$) denote the substring left out after extracting from $\mathbb{S}_{\sigma}$ ($\mathbb{S}_{\sigma_0}$) from $\sigma$ ($\sigma_0$). For example, $\bar{\mathbb{S}}_{\sigma}=1256 10 9$ and $\bar{\mathbb{S}}_{\sigma_0}=13710 9 6$ in the above example. Let $D$ be the set of elements outside of $S$  and $A$ that occupy different positions in $\bar{\mathbb{S}}_\sigma$ and $\bar{\mathbb{S}}_{\sigma_0}$ (thereby contributing to the hamming distance), i.e., $D=\{\bar{\mathbb{S}}_{\sigma_0[i]}|\bar{\mathbb{S}}_{\sigma_0[i]} \not \in S, \bar{\mathbb{S}}_{\sigma_0[i]} \neq \bar{\mathbb{S}}_{\sigma[i]}, i \in [n-l_S]\}$. For instance, in the above example $D=\{9,6,10\}$. Hence, $h=\textswab{d}_{H}(\sigma,\sigma_0)=|A|+|B|+|C|+|D|$ and clearly $f(|D|,|C|)$ represents the number of permutations of $\bar{\mathbb{S}}_{\sigma}$ such that $\textswab{d}_H(\bar{\mathbb{S}}_{\sigma},\bar{\mathbb{S}}_{\sigma_0})=|C|+|D|$. Finally, we have 
\begin{gather*}c_h=\sum_{j=k+1}^{\max(l_s,\lfloor h/2\rfloor)}\underbrace{\binom{l_s}{j}}_{\text{\# ways of selecting set $C$}}\cdot \underbrace{\binom{n-l_s}{j}}_{\text{\# ways of selecting set $A$}}\cdot \Bigg[\\\sum_{i=0}^{\min(l_s-j,h-2j)}\underbrace{\binom{l_s-j}{i}}_{\text{\# ways of selecting set $B$}}\cdot f(i,j)\\\cdot\underbrace{\binom{n-l_s-j}{h-2j-i}}_{\text{\# ways of selecting set $D$}} \cdot f(h-2j-i,j)\Bigg]\end{gather*}
Now, for $f(i,j)$ let $E$ be the set of original positions of Type A that are occupied by Type B objects in the resulting permutation. Additionally, let $F$ be the set of the original positions of Type B objects that are still occupied by some Type B object. Clearly, Type B objects can occupy these $|E
|+|F|=m$ in any way they like. However, the type A objects can only result in $f(p-q,q)$ permutations. Therefore, $f(p,m)$ is given by the following recursive function \begin{gather*}
f(p,0)=!p\\
f(0,m)=m!\\
f(p,m)=\sum_{q=0}^{\min{p,m}}\Bigg(\underbrace{\binom{p}{q}}_{\text{\# ways of selecting set $E$}}\cdot\underbrace{\binom{m}{m-q}}_{\text{\# ways of selecting set $F$}}\\\cdot m! \cdot f(p-q,q)\Bigg)\end{gather*}

Thus, the total probability of failure is given by 
\begin{gather}\delta=\frac{1}{\psi(\theta, \textswab{d}_H)}\sum_{h=2k+2}^{n} (e^{-\theta\cdot h} \cdot c_h)\end{gather}
\end{proof}






%newpage
\subsection{Additional Experimental Details}\label{app:extraresults}
\subsubsection{Evaluation of $(\eta,\delta)$-preservation}\label{app:numerical}

\begin{figure*}[ht]
    \begin{subfigure}[b]{0.33\linewidth}\centering
    \includegraphics[width=\linewidth]{./figures/eta_alpha.png}
        \caption{Variation with $\alpha$}
        \label{fig:eta:alpha}
    \end{subfigure}
    \begin{subfigure}[b]{0.33\linewidth}\centering
    \includegraphics[width=\linewidth]{./figures/eta_width.png}
        \caption{Variation with $\omega$; $\alpha = 3$}
        \label{fig:eta:width}
    \end{subfigure}
    \begin{subfigure}[b]{0.33\linewidth}\centering
    \includegraphics[width=\linewidth]{./figures/eta_subset.png}
        \caption{Variation with $l_S$; $\alpha = 3$}
        \label{fig:eta:subset}
    \end{subfigure}
        \caption{$(\eta,\delta)$-Preservation Analysis}
        \label{fig: eta delta preservation}
\end{figure*}
% \subsubsection{Twitch Dataset Details}
% \begin{figure}[h]
% \begin{subfigure}[b]{\columnwidth}\centering
%     \includegraphics[width=0.8\columnwidth]{Twitch_attack.png}
%         \caption{Twitch: Attack}
%         \label{fig:Twitch:attack:1}
%     \end{subfigure}\\
%     \begin{subfigure}[b]{\columnwidth}\centering
%     \includegraphics[width=0.8\columnwidth]{Twitch_utility.png}
%         \caption{Twitch: Utility}
%         \label{fig:Twitch:utility:1}
%     \end{subfigure}
%     \caption{Twitch dataset results.}
%     \label{fig:Twitch:dataset}
% \end{figure}


% Here, we present experimental results on an additional dataset based on the \emph{Twitch} social media platform \cite{twitch}. Here, the publicly available information is a social media graph, wherein each node represents a user and each edge represents a mutual friendship. So $t_i$ is the set of mutual friendships for data owner $\DO_i$; the $i$'th row of the graph's adjacency matrix. We have binary sensitive attributes of each user $x_i$ that indicates whether or not the user uses explicit language. The dataset includes 9,498 edges. 

% We let the distance between data owners, $d(t_i, t_j)$, be the path length between them. So if data owners $\DO_i$ and $\DO_j$ mutually follow each other, then $d(t_i, t_j) = 1$. For this dataset, we construct the reference permutation using Alg. 1 with $r = 1$, so the reference permutation $\sigma_0$ is constructed to place $\DO_i$'s friends as close as possible in the reference permutation (and thus are most likely to shuffle together). 

% We formalize the group assignment based on the distance threshold $r$ in Sec. \ref{sec:privacy:def} for the ease of exposition. Notably, the actual privacy definition (Def. \ref{def:privacy}) does not dependent on this formalization -- any arbitrary group assignment for $\calG$ is admissible. Hence, exploring other systematic group assignment policies is an interesting future direction. To this end, in this experiment we present another alternative group assignment policy, where for a fixed group size $g$,
% \begin{gather*}G_i=\{\text{Top $g$ closest neighbors of $\DO_i$ as measured via $d(\cdot)$} \}\end{gather*}
% Thus, each data owner $\DO_i$ is assigned an equal sized group $|G_i|$ consisting of its closest `friends' on Twitch. 

% The results are presented in Fig. \ref{fig:Twitch:dataset}. We see that, as we increases the group size afforded to each user, the attack efficacy and utility both gradually decline. Note that, instead of the GBDT calibration model used in the previous experiments, we simply report the empirical distribution of user $i$'s group after shuffling, $\bz_{G_i}$. The utility offered by small to medium group sizes suggests that we can maintain the distribution of sensitive attributes as it varies across the graph.




In this section, we evaluate the characteristics of the  $(\eta,\delta)$-preservation for Kendall's $\tau$ distance $\textswab{d}_\tau(\cdot, \cdot)$.

Each sweep of Fig. \ref{fig: eta delta preservation} fixes $\delta = 0.01$, and observes $\eta$. We consider a dataset of size $n = 10K$ and a subset $S$ of size $l_S$ corresponding to the indices in the middle of the reference permutation $\sigma_0$ (the actual value of the reference permutation is not significant for measuring preservation). For the rest of the discussion, we denote the width of a permutation by $\omega$ for notational brevity. For each value of the independent axis, we generate $50$ trials of the permutation $\sigma$ from a Mallows model with the appropriate $\theta$ (given the $\omega$ and $\alpha$ parameters). We then report the largest $\eta$ (fraction of subset preserved) that at least 99\% of trials satisfy. 

In Fig. \ref{fig:eta:alpha}, we see that preservation is highest for higher $\alpha$ and increases gradually with declining width $\omega$ and increasing subset size $l_s$. 

Fig. \ref{fig:eta:width} demonstrates that preservation declines with increasing width. $\Delta$ increases quadratically with width $\omega$ for $\textswab{d}_\tau$, resulting in declining $\theta$ and increasing randomness. We also see that larger subset sizes result in a more gradual decline in $\eta$. This is due to the fact that the worst-case preservation (uniform random shuffling) is better for larger subsets. i.e. we cannot do worse than $80\%$ preservation for a subset that is $80\%$ of indices. 

Finally, Fig. \ref{fig:eta:subset} demonstrates how preservation grows rapidly with increasing subset size. For large widths, we are nearly uniformly randomly permuting, so preservation will equal the size of the subset relative to the dataset size. For smaller widths, we see that preservation offers diminishing returns as we grow subset size past some critical $l_s$. For $\omega = 30$, we see that subset sizes much larger than a quarter of the dataset gain little in preservation. 

\subsubsection{Adult Dataset}
\label{app:adult experiments}
\begin{figure*}[ht]
    \begin{subfigure}[b]{0.33\linewidth}\centering
        \includegraphics[width=\linewidth]{./figures/Adult_attack_new.png}
        %\vspace{-0.15cm}
        \caption{\textit{Adult}: Attack }%($r$)}
        \label{fig:Adult:attack}
    \end{subfigure}
    \begin{subfigure}[b]{0.33\linewidth}\centering
    \includegraphics[width=\linewidth]{./figures/Adult_alpha.png}
        %   \vspace{-0.15cm} 
        \caption{\textit{Adult}: Attack ($\alpha$)}
        \label{fig:Adult:alpha}
    \end{subfigure}
    \begin{subfigure}[b]{0.33\linewidth}\centering
   \includegraphics[width=\linewidth]{./figures/Adult_utility_1.png}
        %\vspace{-0.15cm}
        \caption{\textit{Adult}: Learnability}
        \label{fig:Adult:utility}
    \end{subfigure}
        \caption{Adult dataset experiments}
        \label{fig: adult experiments}
\end{figure*}






\subsection{Additional Related Work}\label{app:related}
In this section, we discuss the relevant existing work. \par  The anonymization of noisy responses  to improve differential privacy was first proposed by Bittau et al. \cite{Bittau2017} who proposed a principled system architecture for shuffling. This model was formally studied later in \cite{shuffling1, shuffle2}. Erlingsson et al. \cite{shuffling1} showed that for arbitrary $\epsilon$-\ldp randomizers, random shuffling results in privacy amplification. Cheu et al. \cite{shuffle2} formally defined the shuffle \DP model
and analyzed the privacy guarantees of the binary randomized response in this model.
The shuffle \DP model differs from our approach in  two ways. First, it focuses completely on the \DP guarantee. %\ldp is characterized by a privacy parameter (see Section \ref{sec:background}) $\epsilon$, lower the value of $\epsilon$ stronger is the guarantee achieved. 
The privacy amplification is manifested in the from of a lower $\epsilon$ (roughly a factor of $\sqrt{n}$) when viewed in an alternative \DP model known as the central \DP model. \cite{shuffling1,shuffle2,blanket,feldman2020hiding,Bittau2017,Balcer2020SeparatingL}.  
However, our result caters to local inferential privacy. Second, the shuffle model involves an uniform random shuffling of the entire dataset. In contrast, our approach the granularity at which the data is shuffled is tunable which delineates a threshold for the learnability of the data. 
\par A steady line of work has sudied the inferential privacy setting \cite{semantics, Kifer,  IP, Dalenius:1977, dwork2010on, sok}. Kifer et al. \cite{Kifer} formally studied privacy degradation in the face of data correlations and later proposed a  privacy framework, Pufferfish \cite{Pufferfish, Song,Blowfish}, for analyzing inferential privacy. Subsequently, several other privacy definitions have also been proposed for the inferential privacy setting \cite{DDP,BDP,correlated,correlated2,CWP}. For instance, Gehrke et al.  proposed a zero-knowledge privacy \cite{ZKPrivacy,crowd} which is based on simulation semantics. Bhaskar et al. proposed  noiseless privacy \cite{noiseless, TNP} by restricting the set of prior
distributions that the adversary may have access to.  A recent work by Zhang et al. proposes attribute privacy \cite{AP} which focuses on the sensitive properties of a whole dataset. In another recent work, Ligett et al. study a relaxation of \DP that accounts for mechanisms that leak
some additional, bounded information about the database 
\cite{bounded}. Some early work in local inferential privacy include profile-based privacy \cite{profile} by Gehmke et al. where the problem setting comes with a graph of data generating distributions, whose edges encode sensitive pairs of distributions that should be made indistinguishable. In another work by Kawamoto et al., the authors propose distribution privacy \cite{takao} -- local differential privacy for probability distributions.    The major difference between our work and prior research is that we provide local inferential privacy through a new angle -- data shuffling. 

Finally, older works such as $k$-anonymity \cite{kanon},  $l$-diversity \cite{ldiv}, and Anatomy \cite{anatomy} and other \cite{older1, older2, older3, older4, older5} have studied the privacy risk of non-sensitive auxiliary information, or `quasi identifiers' (QIs). In practice, these works focus on the setting of dataset release, where we focus on dataset collection. As such, QIs can be manipulated and controlled, whereas we place no restriction on the amount or type of auxiliary information accessible to the adversary, nor do we control it. Additionally, our work offers each individual formal inferential guarantees against informed adversaries, whereas those works do not. We emphasize this last point since formalized guarantees are critical for providing meaningful privacy definitions. As established by Kifer and Lin in \emph{An Axiomatic View of Statistical Privacy and Utility} (2012), privacy definitions ought to at least satisfy post-processing and convexity properties which our formal definition does.



 
%Relations between differential privacy andthat has given rise to a set of privacy definitions \cite{DDP,BDP,correlated, correlated2, CWP} (not an exhaustive list)  including zero-knowledge privacy \cite{ZKPrivacy}, crowd-blending privacy \cite{crowd}, noiseless privacy \cite{noiseless, TNP}, Pufferfish framework \cite{Pufferfish, Blowfish, Song} and attribute privacy \cite{AP}. 
\newpage


\subsection{Evaluation of Heuristic}
\label{apx:heuristic eval}

\begin{figure}[h]
    \centering
    \includegraphics[width = \linewidth]{./figures/heuristic_optimal.png}
    \caption{Comparison of our heuristic's performance with that of an optimal reference permutation $\sigma^*_0$. An optimal $\sigma^*_0$ is generated with every group having size $w$. A graph is generated from this optimal $\sigma^*_0$ from which our heuristic (blue) attempts to reconstruct the optimal permutation. For baselining, the performance of a random $\sigma_0$ selection is plotted (orange). We observe that at worst, our heuristic picks a reference permutation with width $2.5\times$ that of the optimal reference permutation (green). See Section \ref{sec:mechanism} for definition of terms.}
    \label{fig:heuristic optimal}
\end{figure}

Algorithm \ref{algo:main} is designed to find a reference permutation $\sigma_0$ with low width $\omega_\calG^\sigma$ w.r.t. the given grouping $\calG$. A low width is desirable, since it leads to low sensitivity $\Delta(\sigma_0 : \textswab{d}, \calG)$, which in turn leads to higher dispersion parameter $\theta = \alpha / \Delta$, and thus less randomness over permutations (higher utility). Theorem \ref{thm:NP} proves that computing the optimal reference permutation (minimum width) is NP-hard. As such, we propose a BFS-based heuristic. 

\textbf{Comparison with optimal reference permutation}\\
To demonstrate the value of the heuristic used in Alg. \ref{algo:main}, we provide two evaluations of its performance. 
For our first evaluation, we compare the performance of our heuristic BFS reference permutation selection ($\sigma_0$) with that of the optimal reference permutation  and that of a random reference permutation. As identified by Theorem \ref{thm:NP}, finding the optimal reference permutation for a given grouping $\calG$ is NP-hard. For these experiments, we first create an optimal reference permutation, where each group $G_i \in \calG$ is equally sized $w$ and maximally compact. The optimal width, $\omega_\calG^\sigma$, is then $\min(n, w)$. We then generate a graph from this optimal reference permutation. Finally, we run the BFS reference permutation computation described in Alg. \ref{algo:main} attempting to approximate the optimal $\sigma^*_0$, and compute its width. 

To compare with a naive approach, we also plot the performance of a randomly chosen reference permutation. We expect the maximum width across groups $\omega_\calG^\sigma$ to be large for this technique. If one of the $n$ groups has a single entry low (near 0) in $\sigma_0$ and a single entry high (near $n$) in $\sigma_0$, the width will be near $n$. The random baseline is averaged over 10 trials with a 1 standard deviation envelope plotted (but difficult to see, since the variance is low). 

Figure \ref{fig:heuristic optimal} depicts our findings. Each plot has a different group size $w$, listed at the top, used in the optimal reference permutation. We find that the random baseline (orange) consistently chooses a reference permutation such that $\omega_\calG^\sigma$ is near $n$, as expected. Our method (blue), on the other hand, closely tracks the optimal solution (green). We find that in the worst case, our algorithm's solution has a width $\leq 2.5 \times$ larger than the optimal. Note that for $r = 0$ (upper left), all methods trivially have a width of one, since the corresponding graph has no edges. While there may be room for improvement, we find this to be sufficient for the present work. 

% \newpage


\begin{figure}[h]
    \centering
    \includegraphics[width = \linewidth]{./figures/heuristic_random.png}
    \caption{example of our heuristic's performance on randomly generated graphs. As $r$ increases, so does the connectivity of the random graphs and the average group size (green). As shown by Theorem \ref{thm:NP}, computing the optimal $\omega_\calG^\sigma$ is NP-hard. The average group size (green) in $\calG$ is a loose lower bound on the optimal $\omega_\calG^\sigma$. The performance of a random $\sigma_0$ assignment (orange) is also plotted for reference. Our heuristic BFS algorithm (blue) consistently outperforms the random baseline.}
    \label{fig:heuristic random}
\end{figure}

\textbf{Performance on randomly generated graphs}\\
For our second evaluation, we observe how well our BFS heuristic (in Algorithm \ref{algo:main}) performs on randomly generated graphs. Here, we sample $n$ points uniformly on the unit interval. We then say that the $i$th point's group, $G_i$, consists of all other points within $r$ of it. As $r$ increases, so does the groups size. Since computing the optimal reference permutation is NP-hard (Theorem \ref{thm:NP}), we do not show the optimal width. Instead, we show a loose lower bound of the optimal width (green) by plotting the average group size for a given $r$ (recall that the width is greater than or equal to the largest group size, so we expect this to be a loose lower bound, solely for reference). For comparison, we evaluate the performance of a random $\sigma_0$ choice as well. For both of these methods, we run 10 trials of generating a random graph (and picking a random $\sigma_0$) at each value of $n$ and plot the mean along with a 1 standard deviation envelope, which is difficult to see due to low variance. 

Figure \ref{fig:heuristic random} depicts our findings. We find that --- across values of $n$ and $r$ --- our heuristic (blue) significantly outperforms the random baseline (orange). Additionally, we observe the trends we expect. For a low $r$ values, our heuristic BFS algorithm chooses a $\sigma_0$ with width close to the lower bound (green) of the optimal width $\omega_{\calG}^\sigma$. As $r$ increases, the graph become significantly more connected. Both the lower bound and our heuristic move closer to the width of the random baseline. Note that for $r = 0$ (upper left), all methods trivially have a width of one, since the corresponding graph has no edges. Ultimately, these findings indicate that our heuristic for computing $\sigma_0$ significantly outperforms a naive random choice, and follows the same trend as the lower bound of the optimal. 
%\graphicspath{{./chapters/chapter5/}}
\chapter{ }

\section{Appendix}
For documented code demonstrating our SDP mechanisms used to generate the plots of \textbf{Figure \ref{fig: experiments}} please visit our repo: \url{https://github.com/casey-meehan/location_trace_privacy} 

The following sections will include proofs of results, derivations of algorithms, and explanations of experimental procedures. 
\subsection{Illustrations}
\label{apx: Illustrations}

\subsubsection{NYC Mayoral Staff Member Location Trace}

\begin{figure*}[h]
	\centering
	\begin{subfigure}[b]{.32\textwidth}
		\centering
		\includegraphics[width = \linewidth]{./images/nyc_trace.png}
		\caption{}
		\label{fig:nyc_trace}
	\end{subfigure}
	\begin{subfigure}[b]{.32\textwidth}
		\centering
		\includegraphics[width = \linewidth]{./images/nyc_trace_isotropic.png}
		\caption{}
		\label{fig:nyc_trace_iso}
	\end{subfigure}
	\begin{subfigure}[b]{.32\textwidth}
		\centering
		\includegraphics[width = \linewidth]{./images/nyc_trace_opt.png}
		\caption{}
		\label{fig:nyc_trace_opt}
	\end{subfigure}
	\caption[Example of sensitive location trace of NYC mayoral staff member exposed by \citep{nyt}.]{Example of sensitive location trace of NYC mayoral staff member exposed by \citep{nyt}. (b) and (c) depict the posterior uncertainty (green) $P_{\calA,\calP}(X_i | Z)$ for each 2d location. (a) depicts three sensitive times (red with blue outline): Gracie Mansion (Mayor's home), an event on Staten Island that the mayor attended, and finally the staff member's home on long island. (b) provides an example of Approach C: adding independent Gaussian noise to each location (red dotted line). A GP posterior still maintains high confidence within a small radius along the trace, including at the sensitive times. (c) provides an example of the optimized noise of Multiple Secrets of identical aggregate MSE as (b). By focusing \textit{correlated} noise around the three sensitive times, there is high uncertainty at sensitive times and high confidence elsewhere.}
	\label{fig:nyc_example}
\end{figure*}

\subsubsection{Juxtaposition of Mechanisms' Covariance Matrices}
\label{apx: juxtaposition}
The following figures aim to illustrate the difference between the covariance matrices used in the experimental baselines (indep./uniform and indep./concentrated) and those chosen by our SDP algorithms for both the RBF and periodic prior. Note that here we presume the different dimensions of location to be independent and --- by Corollary \ref{cor: independence} --- are able to treat a 2d location trace as two 1d traces. As such, the following examples are demonstrating mechanism covariance matrices and additive noise samples used for either a single dimension of location data (for RBF kernel) or for the one dimension of temperature data (for periodic kernel). 

The first figure \textbf{(a)} shows the covariance of the Approach C baselines used in the experiments. The second figure \textbf{(b)} shows the covariance of our SDP mechanisms for the RBF kernel used on location data. The third figure \textbf{(c)} shows the covariance of our SDP mechanisms for the periodic kernel used for temperature data. 

In each figure the covariance matrix is depicted as a heat map with warmer colors indicating higher values (normalized to largest and smallest value in the covariance matrix). The drawn noise samples $G$ are plotted against their time index. So, the sequence of plotted $(x,y)$ values is $\big[(1, G_1), (2, G_2), \dots, (n, G_n)\big]$, where $n = 50$ for the RBF case and $n = 48$ for the periodic case. 

\begin{figure*}[h]
	\centering
	\begin{subfigure}[b]{1\textwidth}
		\centering
		\includegraphics[width = \linewidth]{./images/cov_table_header.png}
	\end{subfigure}
	\begin{subfigure}[b]{1\textwidth}
		\centering
		\includegraphics[width = \linewidth]{./images/cov_table_baselines.png}
		\caption[Covariance matrices and mechanism samples for the baselines used in experiments.]{Covariance matrices and mechanism samples for the baselines used in experiments. 
		\vspace{2mm}\\
		The first figure demonstrates the uniform approach that distributes the independent Gaussian noise budget along the entire trace, regardless of $\Is$. 
		\vspace{2mm}\\
		The second and third show the concentrated approach that allocates the entire noise budget to only the sensitive locations in $\Is$: first for a basic secret (one location) and then for a compound secret of 3 evenly spaced locations.} 
		\label{fig: cov table baselines}
	\end{subfigure}
\end{figure*}

\begin{figure*}[h] \ContinuedFloat
	\begin{subfigure}[b]{1\textwidth}
		\centering
		\includegraphics[width = \linewidth]{./images/cov_table_header.png}
	\end{subfigure}
	\begin{subfigure}[b]{1\textwidth}
		\centering
		\includegraphics[width = \linewidth]{./images/cov_table_RBF_1.png}
	\end{subfigure}
	\begin{subfigure}[b]{1\textwidth}
		\centering
		\includegraphics[width = \linewidth]{./images/cov_table_RBF_2.png}
		\caption[Covariance matrices and mechanism samples for the median RBF prior ($\leff \approx 6$).]{
			Covariance matrices and mechanism samples for the median RBF prior ($\leff \approx 6$). 
			\vspace{2mm} \\
			The first noise mechanism (Mech. basic) demonstrates the covariance matrix chosen by $\text{SDP}_\text{A}$ for a basic secret of a single location $X_i$ in the middle of the trace. The uncorrelated dot in the middle of the covariance matrix, $\Sigmag_{ii}$, represents the independent noise $G_i$ added at the sensitive location to mitigate \emph{direct} loss. To mitigate \emph{inferential} loss, the SDP optimizes the remainder of the matrix to be positively correlated with maximum variance allocated to locations near $X_i$ in time. This thwarts GP inference of the true location at time $t_i$. 
			\vspace{2mm} \\
			The second mechanism (Mech. comp.) depicts the covariance chosen by $\text{SDP}_\text{A}$ to protect a compound secret of two adjacent locations in the trace (visible as the uncorrelated `$+$' through the middle consuming 2 rows/columns). Recall that a compound secret ought to protect directional information: \emph{did the user visit B first and then A, or A and then B?} That is precisely what this mechanism does by randomizing the angle of approach to the two locations in the middle with positively and negatively correlated noise. Also note that the SDP does not allocate a large share of noise budget to the actual locations themselves. This highlights the fact that protecting a compound secret does not protect its constituent basic secrets.
			\vspace{2mm} \\
			The third and final mechanism (Mech. all basic) is the noise covariance chosen by $\text{SDP}_\text{B}$ in the Multiple Secrets algorithm. To protect all basic secrets with a utility constraint, the SDP converges to a mechanism that looks similar to the uniform baseline. However, this mechanism adds a subtle degree of off-diagonal correlation along with greater noise power towards the beginning and end of the trace. The off-diagonal correlation is noticeable when the samples are compared to those of the uniform baseline in the previous figure. While this change appears to be minor, it makes a significant change in the posterior confidence of a GP adversary (as seen in \textbf{Figure \ref{fig: RBF all}}). 
			}
		\label{fig: cov table rbf}
	\end{subfigure}
\end{figure*}

\begin{figure*}[h] \ContinuedFloat
	\begin{subfigure}[b]{1\textwidth}
		\centering
		\includegraphics[width = \linewidth]{./images/cov_table_header.png}
	\end{subfigure}
	\begin{subfigure}[b]{1\textwidth}
		\centering
		\includegraphics[width = \linewidth]{./images/cov_table_PER_1.png}
	\end{subfigure}
	\begin{subfigure}[b]{1\textwidth}
		\centering
		\includegraphics[width = \linewidth]{./images/cov_table_PER_2.png}
		\caption[Covariance matrices and mechanism samples for the median periodic prior ($\leff \approx 1.1$), and a period of half the trace length.]{
			Covariance matrices and mechanism samples for the median periodic prior ($\leff \approx 1.1$), and a period of half the trace length. 
			\vspace{2mm} \\
			The first noise mechanism (Mech. Basic) shows the covariance chosen by $\text{SDP}_\text{A}$ to protect a single location (temperature) in the middle of the trace. As in the RBF case, significant noise power is allocated to the sensitive location itself, $X_i$, to limit \emph{direct} privacy loss. However, the noise added to the remainder of the trace is significantly different. It is tailored to thwart inference by a periodic prior, wherein the location one period away has correlation 1. 
			\vspace{2mm}\\
			The second noise mechanism (Mech. comp.) shows the covariance chosen by $\text{SDP}_\text{A}$ to protect a compound secret of two locations, $X_i, X_j$, 16 timesteps apart (not quite a full period). Here, we see the SDP randomize the phase of the additive noise such that periodic inference cannot tell directional information like $X_i > X_j$ or vice versa. 
			\vspace{2mm}\\
			The third noise mechanism (Mech. all basic) is identical to the all basic secrets mechanism chosen for the RBF case above, except using a periodic prior $\Sigma$. The mechanism chosen looks similar to the uniform baseline, except with slightly periodic off-diagonal correlation imitating the prior covariance. Additionally, noise power is mitigated towards the middle and ends of the trace. Again, \textbf{Figure \ref{fig: PER all}} indicates that this subtle change makes a significant difference in thwarting Bayesian adversaries. 
			}
		\label{fig: cov table rbf}
	\end{subfigure}
\end{figure*}

\clearpage

\subsection{Proof of results}
\label{apx: proofs} 
\subsubsection{Proof of Theorem \ref{thm: prior misspecification}} 
\textbf{Theorem \ref{thm: prior misspecification}} Prior-Posterior Gap:
\textit{
An $(\varepsilon, \lambda)$-CIP mechanism with conditional prior class $\Theta$ guarantees that for any event $O$ on sanitized trace $Z$
	\begin{align*}
		\bigg| \log \frac{P_{\calP, \calA}(s_i | Z \in O)}{P_{\calP, \calA}(s_j | Z \in O)} - \log \frac{P_{\calP}(s_i)}{P_{\calP}(s_j)} \bigg| \leq \varepsilon'
	\end{align*}
	for any $\calP \in \Theta$ with probability $\geq 1 - \delta$ over draws of $Z|\Xs=s_i$ or $Z|\Xs=s_j$, where $\varepsilon'$ and $\delta$ are related by
	\begin{align*}
		\varepsilon' = \varepsilon + \frac{\log \nicefrac{1}{\delta}}{\lambda - 1} \ .
	\end{align*}
	This holds under the condition that $Z|\Xs = s_i$ and $Z|\Xs = s_j$ have identical support. 
}

\begin{proof}
	This result makes use of a R\'enyi divergence property identified in \cite{renyi}: 
	\begin{lemma}
		\label{lem: renyi to eps delt}
		Let $\calP,\calQ$ be two distributions on $X$ of identical support such that  
		\begin{align*}
			\max \bigg\{ D_\lambda \binom{P_\calP(X)}{P_\calQ(X)}, 
			D_\lambda \binom{P_\calQ(X)}{P_\calP(X)} \bigg\}
			\leq \varepsilon 
		\end{align*}
		Then for any event $O$,
		\begin{align*}
			P_\calP(X \in O) \leq \max \big\{ e^{\varepsilon'} P_\calQ(X \in S), \delta \big\}
		\end{align*} 
		and
		\begin{align*}
			P_\calQ(X \in O) \leq \max \big\{ e^{\varepsilon'} P_\calP(X \in S), \delta \big\}
		\end{align*} 
		where 
		\begin{align*}
			\varepsilon' = \varepsilon + \frac{\log \nicefrac{1}{\delta}}{\lambda - 1}
		\end{align*}
	\end{lemma}
	CIP guarantees that for all $\calP \in \Theta$ and all discriminative pairs $(s_i, s_j) \in \Spairs$ (which also includes $(s_j, s_i)$) 
	\begin{align*}
		D_\lambda \binom{P_{\calP, \calA}(Z | \Xs = s_i)}{P_{\calP,\calA}(Z | \Xs = s_j)} \leq \varepsilon
	\end{align*}
	and thus by Lemma \ref{lem: renyi to eps delt} we have for any event $O$ on $Z$
	\begin{align*}
		P_{\calP, \calA}(Z \in O | \Xs = s_i) 
		\leq \max \big\{ e^{\varepsilon'} P_{\calP, \calA}(Z \in O | \Xs = s_j), \delta \big\}
	\end{align*}
	and
	\begin{align*}
		P_{\calP, \calA}(Z \in O | \Xs = s_j) 
		\leq \max \big\{ e^{\varepsilon'} P_{\calP, \calA}(Z \in O | \Xs = s_i), \delta \big\}
	\end{align*}
	As such, given that $\Xs = s_i$ the probability of some event $\{Z \in W\}$ such that 
	\begin{align*}
		P_{\calP, \calA}(Z \in W | \Xs = s_i) 
		\geq  e^{\varepsilon'} P_{\calP, \calA}(Z \in W | \Xs = s_j)
	\end{align*}
	is no more than $\delta$. The same is true swapping $s_j$ for $s_i$. So, over draws of $Z | \Xs = s_i$ or $Z | \Xs = s_j$ we have that 
	\begin{align*}
		 \frac{P_{\calP, \calA}(Z \in O | \Xs = s_i)}{P_{\calP,\calA}(Z \in O | \Xs = s_j)} \leq e^{\varepsilon'}
		 \quad \text{and} \quad
		 \frac{P_{\calP, \calA}(Z \in O | \Xs = s_j)}{P_{\calP,\calA}(Z \in O | \Xs = s_i)} \leq e^{\varepsilon'}
	\end{align*}
	with probability $\geq 1 - \delta$, which is equivalent to the statement that 
	\begin{align*}
		-\varepsilon' 
		\leq \log  \frac{P_{\calP, \calA}(Z \in O | \Xs = s_i)}{P_{\calP,\calA}(Z \in O | \Xs = s_j)}
		&\leq \varepsilon' \\
		\bigg| \log \frac{P_{\calP, \calA}(s_i | Z \in O)}{P_{\calP, \calA}(s_j | Z \in O)} - \log \frac{P_{\calP}(s_i)}{P_{\calP}(s_j)} \bigg| 
		&\leq \varepsilon'
	\end{align*}
\end{proof}

\subsubsection{Proof of Lemma \ref{lem: renyi additive loss}}
\textbf{Lemma \ref{lem: renyi additive loss}} (CIP loss for additive mechanisms)
\textit{
	For an additive noise mechanism, a fully dependent trace as in \textbf{Figure \ref{fig:condensed model}}, and any prior $\calP$ on $X$ the CIP loss may be expressed as
	\begin{align}
		&D_\lambda \binom{P_{\calA, \calP}(Z | \Xs = s_i)}{P_{\calA, \calP}(Z | \Xs = s_j)}  
		&= \sum_{i \in \Is} \bigg[ D_\lambda \binom{P_\calA(Z_i | X_i = s_i)}{P_\calA(Z_i | X_i = s_j)} \bigg]
		+ D_\lambda \binom{P_{\calA, \calP}(\Zu | \Xs = s_i)}{P_{\calA, \calP}(\Zu | \Xs = s_j)} \notag
	\end{align}
%	where $\Xu_{|x_s} \sim P_\calP(\Xu | \Xs = x_s)$ and $\Xu_{|x_s'} \sim P_\calP(\Xu | \Xs = x_s')$. 
}
\begin{proof}
\begin{align}
	D_\lambda \binom{P_{\calA, \calP}(Z | \Xs = x_s)}{P_{\calA, \calP}(Z | \Xs = x_s')} 
	&= D_\lambda \binom{P_{\calA}(\Zs | \Xs = x_s)P_{\calA, \calP}(\Zu | \Xs = x_s)}{P_{\calA}(\Zs | \Xs = x_s')P_{\calA, \calP}(\Zu | \Xs = x_s')} \tag{1} \\
	&=  D_\lambda \binom{P_{\calA}(\Zs | \Xs = x_s)}{P_{\calA}(\Zs | \Xs = x_s')} + D_\lambda \binom{P_{\calA, \calP}(\Zu | \Xs = x_s)}{P_{\calA, \calP}(\Zu | \Xs = x_s')} \tag{2} \\
	&= D_\lambda \binom{\prod_{i \in \Is } P_{\calA}(Z_i | X_i = x_i)}{\prod_{i \in \Is } P_{\calA}(Z_i | X_i = x_i')} + D_\lambda \binom{P_{\calA, \calP}(\Zu | \Xs = x_s)}{P_{\calA, \calP}(\Zu | \Xs = x_s')} \tag{3} \\
	&= \sum_{i \in \Is} \bigg[ D_\lambda \binom{P_\calA(Z_i | X_i = x_i)}{P_\calA(Z_i | X_i = x_i')} \bigg]
	+ D_\lambda \binom{P_{\calA, \calP}(\Zu | \Xs = x_s)}{P_{\calA, \calP}(\Zu | \Xs = x_s')} \tag{4}
\end{align}
Where line (1) uses the conditional independence seen in the graphical model of \textbf{Figure \ref{fig:graphical models}}. Line (2) is due to the fact that the two terms in line (1) are conditionally independent, allowing for separating into the sum of two separate divergences (which is an easily verifiable property of R\'enyi divergence evident from its definition in Equation \ref{eqn: renyi}). Line (3) is again from the conditional independence between the $Z_i$ for each $i \in \Is$ when conditioned on $\Xs$. Line (4) uses the same property of R\'enyi divergence used in Line (2): the terms in the product are conditionally independent allowing for the separation into the sum of multiple divergences. 

%We now rewrite the conditional distribution of the second term as the marginal distribution of the sum of two independent random variables: $P_{\calA, \calP}(\Zu | \Xs = x_s) = P_{\calA, \calP}(\Xu_{|x_s} + \Gu)$ where \\$\Xu_{|x_s} \sim P_\calP(\Xu | \Xs = x_s)$. Effectively this says that the distribution of the conditional random variable $\Zu | \{\Xs = x_s\}$ is identical to the distribution of independently drawing $\Xu_{|x_s} \sim P_\calP(\Xu | \Xs = x_s)$ and $\Gu \sim \calN(\mathbf{0}, \Sigma_{uu})$ and adding them together. 
%\begin{align}
%	P_{\calA, \calP}(\Zu = z_u | \Xs = x_s) 
%	&= \int_{\R^{|\Iu|}} P_{\calA, \calP}(\Zu = z_u, \Xu = x_u | \Xs = x_s) \ dx_u  \tag{5} \\
%	&=  \int_{\R^{|\Iu|}} P_{\calP}(\Xu = x_u | \Xs = x_s) P_{\calA}(\Zu = z_u | \Xu = x_u) \ dx_u  \tag{6} \\
%	&=  \int_{\R^{|\Iu|}} P_{\calP}(\Xu = x_u | \Xs = x_s) P_{\calA}(\Gu = z_u - x_u) \ dx_u \tag{7} \\
%	&= \big( P_{\calP}(\Xu | \Xs = x_s) * P_{\calA}(\Gu) \big)(z_u) \tag{8} \\
%	&= P_{\calA, \calP}(\Xu_{|x_s} + \Gu = z_u) \tag{9}
%\end{align}
%Where lines (5) and (6) are also due to the structure of conditional independence, and line (7) is simply rewriting $P_{\calA}(\Zu = z_u | \Xu = x_u)$ in terms of the density of $\Gu$. Line (8) is by definition of a convolution, and line (9) is due to the fact that the convolution of the densities of two independent random variables is the distribution of their sum. Thus, $P_{\calA, \calP}(\Zu | \Xs = x_s) = P_{\calA, \calP}(\Xu_{|x_s} + \Gu)$. Substituting this back into the second divergence in line (4), we get 
%\begin{align*}
%	D_\lambda \binom{P_{\calA, \calP}(Z | \Xs = x_s)}{P_{\calA, \calP}(Z | \Xs = x_s')} &= 
%	\sum_{i \in \Is} \bigg[ D_\lambda \binom{P_\calA(Z_i | X_i = x_i)}{P_\calA(Z_i | X_i = x_i')} \bigg]
%	+ D_\lambda \binom{P_{\calA, \calP}(\Xu_{|x_s} + \Gu)}{P_{\calA, \calP}(\Xu_{|x_s'} + \Gu)}
%\end{align*}

\end{proof}

\subsubsection{Proof of Theorem \ref{thm: prior misspecification}}
\label{apx: prior misspecification proof}
\textbf{Thoerem \ref{thm: prior misspecification}}
Robustness to Prior Misspecification 
\textit{
	Mechanism $\calA$ satisfies $\varepsilon(\lambda)$-CIP for prior class $\Theta$. Suppose the finite mean true distribution $\calQ$ is not in $\Theta$. The CIP loss of $\calA$ against prior $\calQ$ is bounded by 
	\begin{align*}
		D_\lambda \binom{P_{\calA, \calQ}(Z | \Xs = s_i)}{P_{\calA, \calQ}(Z | \Xs = s_j)} \leq \varepsilon'(\lambda)
	\end{align*}
	where
	\begin{align*}
		\varepsilon'(\lambda) 
		&= \frac{\lambda - \frac{1}{2}}{\lambda - 1} \ \Delta(2\lambda) + 
		\Delta(4\lambda - 3) +
		\frac{2\lambda - \frac{3}{2}}{2\lambda - 2} \ \varepsilon(4 \lambda -2)
	\end{align*}
	and where $\Delta(\lambda)$ is
	\begin{align*}
		\inf_{\calP \in \Theta} \sup_{s_i \in \calS} \max \bigg\{ 
		D_\lambda \binom{P_{ \calP}(\Xu | \Xs = s_i)}{P_{ \calQ}(\Xu | \Xs = s_i)}, 
		D_\lambda \binom{P_{ \calQ}(\Xu | \Xs = s_i)}{P_{ \calP}(\Xu | \Xs = s_i)}
		\bigg\}
	\end{align*}
}
\begin{proof}
By `finite mean' distribution $\calQ$, we mean that all conditionals of $\calQ$ given some $\Xs$ have finite mean. Since a conditional prior class contains conditionals of one distribution with any offset (any mean value), this guarantees that $\Delta(\lambda)$ is achieved for some $\calP \in \Theta$. Intuitively, this prevents the pathological case of $\inf_{\calP \in \Theta}$ being a limit as the mean of $\calP \rightarrow \infty$, only asymptotically approaching $\Delta(\lambda)$. If the mean of $\calQ$ is finite, then the closest $\calP \in \Theta$ (in R\'enyi divergence) must also have finite mean, since any mean is attainable in a conditional prior class $\Theta$.

With this in mind, we make use of the following triangle inequality provided in \cite{renyi}: 
\begin{lemma}
	For distributions $\calP$, $\calQ$, $\calR$ on $X$ with common support we have
	\begin{align*}
		D_\lambda \binom{P_\calP(X)}{P_\calQ(X)} \leq 
		\frac{\lambda - \frac{1}{2}}{\lambda - 1} D_{2 \lambda} \binom{P_\calP(X)}{P_\calR(X)} 
		+ D_{2\lambda - 1} \binom{P_\calR(X)}{P_\calQ(X)}
	\end{align*}
\end{lemma}
In our case, we assume that the mechanism $\calA$ gives $Z|\Xs = x_s$ identical support for all $\Is, x_s$. Using this, we have 
\begin{align*}
	D_\lambda \binom{P_{\calA, \calQ}(\Zu | \Xs = x_s)}{P_{\calA, \calQ}(\Zu | \Xs = x_s')} 
	\leq \frac{\lambda - \frac{1}{2}}{\lambda - 1} D_{2\lambda} \binom{P_{\calA, \calQ}(\Zu | \Xs = x_s)}{P_{\calA, \calP}(\Zu | \Xs = x_s)}
	+  D_{2\lambda - 1} \binom{P_{\calA, \calP}(\Zu | \Xs = x_s)}{P_{\calA, \calQ}(\Zu | \Xs = x_s')} \ \ . \\
\end{align*}
By a data processing inequality, the divergence of the first term is bounded by $\Delta(2\lambda)$ and the blue term may be bounded by a second application of the triangle inequality: 
\begin{align*}
	D_{2\lambda - 1} \binom{P_{\calA, \calP}(\Zu | \Xs = x_s)}{P_{\calA, \calQ}(\Zu | \Xs = x_s')}
	&\leq \frac{2\lambda - \frac{3}{2}}{2\lambda - 2} D_{4\lambda - 2} \binom{P_{\calA, \calP}(\Zu | \Xs = x_s)}{P_{\calA, \calP}(\Zu | \Xs = x_s')}
	+ D_{4\lambda - 3} \binom{P_{\calA, \calP}(\Zu | \Xs = x_s')}{P_{\calA, \calQ}(\Zu | \Xs = x_s')}
\end{align*}
The first divergence is bounded by $\varepsilon(4\lambda - 2)$ and the second divergence is bounded by $\Delta(4\lambda - 3)$. Putting all this together we have the following upper bound 
\begin{align*}
	D_\lambda \binom{P_{\calA, \calQ}(\Zu | \Xs = x_s)}{P_{\calA, \calQ}(\Zu | \Xs = x_s')}
	\leq 
	\frac{\lambda - \frac{1}{2}}{\lambda - 1} \ \Delta(2\lambda) + 
		\Delta(4\lambda - 3) +
		\frac{2\lambda - \frac{3}{2}}{2\lambda - 2} \ \varepsilon(4 \lambda -2)
\end{align*}
\end{proof}

\subsubsection{Proof of Theorem \ref{thm:GP bound}}
\label{apx: GP bound proof}
\textbf{Theorem \ref{thm:GP bound}}
CIP loss bound for GP conditional priors:
\emph{
Let $\Theta$ be a GP conditional prior class. Let $\Sigma$ be the covariance matrix for $X$ produced by its kernel function. Let $\calS$ be the basic or compound secret associated with $\Is$, and $S$ be the number of unique times in $\Is$. The mechanism $\calA(X) = X + G = Z$, where $G \sim \calN(\mathbf{0}, \Sigmag)$, then satisfies $(\varepsilon, \lambda)$-Conditional Inferential Privacy $(\Spairs, r, \Theta)$, where 
\begin{align*}
	\varepsilon
	&\leq \frac{\lambda}{2} S r^2 \Big(  \frac{1 }{\sigma_s^2} + \alpha^*  \Big) 
\end{align*}
where $\sigma_s^2$ is the variance of each $G_i \in \Gs$ (diagonal entries of $\Sigmag_{ss}$) and $\alpha^*$ is the maximum eigenvalue of $\Sigmaeff = \big(\Sigma_{us} \Sigma_{ss}^{-1}\big)^\intercal \big( \Sigma_{u | s} + \Sigma_{uu}^{(g)} \big)^{-1} \big(\Sigma_{us} \Sigma_{ss}^{-1}\big)$. 
}

\begin{proof}
Again, the conditional prior class $\Theta$ is defined by a kernel function $i,j \rightarrow \text{Cov}(i,j)$, which -- given the indices of the trace $X$ -- induces a covariance matrix $\Sigma$ between all $X_i, X_j$. In practice, when the sampling rate of locations is non-uniform the kernel function may use the time-stamps of the points in the trace to assign high correlation to $X_i$ that are close in time and low correlation to $X_i$ that are far apart in time. Of course, correlation between $X_i$ that are different dimension (e.g. latitude and longitude) must be designed for the given application and may be completely independent. The kernel function can encode this as well. 

Recall from Equation \ref{eqn: renyi} that the R\'enyi divergence between two mean-shifted multivariate normal distributions, $\calP_1 = \calN(\mu_1, \Sigma)$ and $\calP_2 = \calN(\mu_2, \Sigma)$ is 
\begin{align*}
	D_\lambda \binom{\calP_1}{\calP_2} = \frac{\lambda}{2} (\mu_1 - \mu_2)^\intercal \Sigma^{-1} (\mu_1 - \mu_2)
\end{align*}
Now, for any prior $\calP \in \Theta$, we have that $X \sim \calN(\mu, \Sigma)$ for some $\mu$ and for $\Sigma$ defined by the kernel function. Again, $G \sim \calN(\mathbf{0}, \Sigmag)$. $\Is$ encodes the indices of a single location basic secret or a multi-location compound secret. Then, the divergence to bound for $(\varepsilon, \lambda)$-CIP$(\Spairs, r, \Theta)$ is 
\begin{align*}
	D_\lambda \binom{P_{\calA, \calP}(Z | \Xs = s_i)}{P_{\calA, \calP}(Z | \Xs = s_j)}
\end{align*}
for any 
\begin{align*} 
	(s_i, s_j) \in \Spairs = \{(x_s, x_s'):\|x_s - x_s'\|_2 \leq 2r\}
\end{align*}
if $\Is$ encodes a basic secret, or for any
\begin{align*}
	(s_i, s_j) \in \Spairs = \Big\{\big( \{x_{s1}, x_{s2}, \dots\}, \{x_{s1}', x_{s2}', \dots\}\big): \| x_{sk} - x_{sk}' \|_2 \leq 2r, \forall \ k\Big\} 
\end{align*} 
if $\Is$ encodes a compound secret. A discriminative pair $(s_i,s_j)$ is two real valued vectors $\in \R^{|\Is|}$, representing two hypotheses about the true values of $\Xs$. We denote the $m^\text{th}$ element as ${s_i}_m, {s_j}_m$. Let $f:\Is \rightarrow [|\Is|]$ be a mapping from each index $w \in \Is$ to its corresponding position in the vector $s_i$ or $s_j$ (where the value of $X_w$ is hypothesized). By Lemma \ref{lem: renyi additive loss}, the divergence can be written as  
\begin{align*}
	D_\lambda \binom{P_{\calA, \calP}(Z | \Xs = s_i)}{P_{\calA, \calP}(Z | \Xs = s_j)}
	&= \sum_{w \in \Is} \bigg[ D_\lambda \binom{P_\calA(Z_w | X_w = {s_i}_{f(w)})}{P_\calA(Z_w | X_w = {s_j}_{f(w)})} \bigg]
	+ D_\lambda \binom{P_{\calA, \calP}(\Zu | \Xs = x_s)}{P_{\calA, \calP}(\Zu | \Xs = x_s')} 
\end{align*}
where $P_\calA(Z_w | X_w = x) = \calN(x, \sigma_s^2)$ for all $w \in \Is$. Recall from the statement of the Theorem that we assume the diagonal entries of $\Sigma_{ss}$ all equal some value $\sigma_s^2$: we add the same noise variance to each point in the secret set, which is optimal under MSE constraints. Additionally, note that for the hypothesis $\Xs = x_s$, we know the distribution of $\Xu | \Xs = x_s \sim \calN(\mu_{u|s}, \Sigma_{u|s})$, where $\mu_{u|s} = \mu_u + \Sigma_{us} \Sigma_{ss}^{-1} (x_s - \mu_s)$ and $\Sigma_{u|s} = \Sigma_{uu} - \Sigma_{us}\Sigma_{ss}^{-1} \Sigma_{su}$. Notice that only $\mu_{u|s}$ depends on the actual value of $x_s$, and $\Sigma_{u|s}$ depends only on the indices of $\Is$. Being the sum of two normally distributed variables, we have that $(\Zu | \Xs = x_s) \overset{d}{=} (\Xu|\Xs = x_s) + \Gu = \calN(\mu_{u|s}, \Sigma_{u|s} + \Sigmag_{uu})$. Substituting this into the divergences above sum of divergences: 
\begin{align}
	&D_\lambda \binom{P_{\calA, \calP}(Z | \Xs = s_i)}{P_{\calA, \calP}(Z | \Xs = s_j)}
	= \sum_{m =1}^{|\Is|} \bigg[ D_\lambda \binom{\calN({s_i}_m, \sigma_s^2)}{\calN({s_j}_m, \sigma_s^2)} \bigg]
	+ D_\lambda \binom{\calN(\mu_{u|s_i}, \Sigma_{u|s} + \Sigmag_{uu})}{\calN(\mu_{u|s_j}, \Sigma_{u|s} + \Sigmag_{uu})} \tag{1} \\
	&=  \frac{\lambda}{2} \sum_{m = 1}^{|\Is|}  \frac{1}{\sigma_s^2} ({s_i}_m - {s_j}_m)^2 
	+  \frac{\lambda}{2} (\mu_{u|s_i} - \mu_{u|s_j})^\intercal (\Sigma_{u|s} + \Sigmag_{uu})^{-1} (\mu_{u|s_i} - \mu_{u|s_j})  \tag{2} \\
	&=  \frac{\lambda}{2 \sigma_s^2}   ({s_i} - {s_j})^\intercal({s_i} - {s_j}) 
	+  \frac{\lambda}{2} \big( \Sigma_{us} \Sigma_{ss}^{-1}(s_i - s_j) \big)^\intercal (\Sigma_{u|s} + \Sigmag_{uu})^{-1} \big( \Sigma_{us} \Sigma_{ss}^{-1}(s_i - s_j) \big)  \tag{3}  \\
	&= \frac{\lambda}{2 \sigma_s^2}   ({s_i} - {s_j})^\intercal({s_i} - {s_j}) 
	+  \frac{\lambda}{2} (s_i - s_j)^\intercal \Sigma_{ss}^{-1} \Sigma_{su}  (\Sigma_{u|s} + \Sigmag_{uu})^{-1} \Sigma_{us} \Sigma_{ss}^{-1} (s_i - s_j) \tag{4} 
\end{align}
Line (1) substitutes in the normal distributions given by our mechanism and conditional prior class. Line (2) substitutes in the closed-form expression for R\'enyi divergence between two mean-shifted normal distributions given in Equation \ref{eqn: renyi}. Line (3) substitutes in the expression for $\mu_{u|s}$ given above, and simplifies. To expand out this simplification in explicit steps: 
\begin{align*}
	(\mu_{u|s_i} - \mu_{u|s_j})
	&= \big(  \mu_u + \Sigma_{us} \Sigma_{ss}^{-1} (s_i - \mu_s) -  [\mu_u + \Sigma_{us} \Sigma_{ss}^{-1} (s_j - \mu_s)] \big) \\
	&= \big(  \Sigma_{us} \Sigma_{ss}^{-1} s_i -  \Sigma_{us} \Sigma_{ss}^{-1} s_j \big) \\
	&= \Sigma_{us} \Sigma_{ss}^{-1} (s_i - s_j)
\end{align*}
Line (4) distributes the transpose in the right term of line (3): 
\begin{align*}
	\big( \Sigma_{us} \Sigma_{ss}^{-1}(s_i - s_j) \big)^\intercal
	&= (s_i - s_j)^\intercal \big(  \Sigma_{us} \Sigma_{ss}^{-1} \big)^\intercal \\
	&=  (s_i - s_j)^\intercal  \big( \Sigma_{ss}^{-1} \big)^\intercal \Sigma_{us}^\intercal   \\
	&= (s_i - s_j)^\intercal \Sigma_{ss}^{-1}  \Sigma_{su}
\end{align*}
where that final step is a consequence of $\Sigma$ being symmetric. $\Sigma_{ss}$ is also a symmetric matrix (so its inverse is symmetric) and $\Sigma_{us}^\intercal = \Sigma_{su}$. 

Returning to line (4) above, simplify this expression by substituting $\Delta = s_i - s_j$: 
\begin{align}
	D_\lambda \binom{P_{\calA, \calP}(Z | \Xs = s_i)}{P_{\calA, \calP}(Z | \Xs = s_j)}
	&= \frac{\lambda}{2 \sigma_s^2}   \Delta^\intercal \Delta 
	+  \frac{\lambda}{2} \Delta^\intercal \Sigma_{ss}^{-1} \Sigma_{su}  (\Sigma_{u|s} + \Sigmag_{uu})^{-1} \Sigma_{us} \Sigma_{ss}^{-1} \Delta \tag{5} \\
	&= \frac{\lambda}{2 \sigma_s^2}  \| \Delta \|_2^2 
	+  \frac{\lambda}{2} \Delta^\intercal \Sigmaeff \Delta \tag{6} 
\end{align}
Where $\Sigmaeff = \Sigma_{ss}^{-1} \Sigma_{su}  (\Sigma_{u|s} + \Sigmag_{uu})^{-1} \Sigma_{us} \Sigma_{ss}^{-1}$. The left term of line (6) attributes the direct loss of $\Zs$ on $\Xs$ and the right term attributes the indirect loss of $\Zu$ on $\Xs$. 

We are interested in bounding the expression of line (6) for all $(s_i, s_j) \in \Spairs$. We do this by bounding it for all vectors $\Delta \in \calD$ 
\begin{align*}
	\calD = \{ s_i - s_j : \| s_i - s_j \|_2 \leq  \sqrt{S}\  r \}
\end{align*}  
, where $S$ is the number of basic secrets (locations) contained in $\Is$ which may be a basic or compound secret set. For a basic secret ($S = 1$), this bound is tight, since $\calD = \{s_i - s_j: (s_i, s_j) \in \Spairs\}$. The set of $\Delta \in \calD$ is exactly any two hypothesis $(s_i, s_j)$ that are within any circle of radius $r$. For a compound secret, this bound is not guaranteed to be tight. Recall once again that the set of $\Spairs$ for a compound secret is given by the set of $(s_i, s_j)$ in 
\begin{align*}
	\Spairs = \Big\{\big( \{x_{s1}, x_{s2}, \dots\}, \{x_{s1}', x_{s2}', \dots\}\big): \| x_{sk} - x_{sk}' \|_2 \leq r, \forall \ k\Big\} 
\end{align*} 
For concreteness, consider the 2d location trace example in \textbf{Figure \ref{fig:nyc_example}}, where we have a compound secret of $S = 3$ locations. Here, $s_i, s_j \in \R^{6}$, where 6 comes from the fact that we have three 2d locations. So, $(s_i, s_j)$ represents a pair of hypotheses on all three locations. $s_i$'s hypothesis of the first secret location --- written as ${x_s}_1 \in \R^2$ above --- is within $r$ of the $s_j$'s hypothesis of the first secret location --- written as ${x_s}_1' \in \R^2$ above. The same goes for the second and third locations. So, the $L_2$ norm of $\Delta = s_i - s_j$ is no greater than
\begin{align*}
	\sup_{(s_i, s_j) \in \Spairs} \|s_i - s_j\|_2 
	&=  \sup_{(s_i, s_j) \in \Spairs} \sqrt{\sum_{m=1}^6 ({s_i}_m - {s_j}_m)^2} \\
	&=  \sup_{(s_i, s_j) \in \Spairs} \sqrt{\sum_{k=1}^3 \|{x_s}_k - {x_s}_k'\|_2^2} \\
	&= \sqrt{\sum_{k=1}^3 r^2} \\
	&= \sqrt{3} \ r
\end{align*}
For compound secrets, $\calD$ represents the $L_2$ ball enclosing all $\Delta \in \{s_i - s_j : (s_i, s_j) \in \Spairs \}$. However, $\calD$ also includes some values of $\Delta = s_i - s_j$ not covered by $\Spairs$. Suppose an adversary considers the hypotheses 
\begin{align*}
s_i = \{x_{s1}, x_{s2}, x_{s3}\}, s_j = \{x_{s1}', x_{s2}', x_{s3}'\}
\end{align*} 
where ${x_s}_1 = 0, {x_s}_1' = \sqrt{3} \ r, {x_s}_2 = {x_s}_2', {x_s}_3 = {x_s}_3'$. Since ${x_s}_1, {x_s}_1'$ are not within $r$ of each other, this is not in $\Spairs$. However, it is covered by $\calD$, and thus is covered by our bound on CIP loss and our mechanisms. 

With $\calD$ defined, we may return to bounding the expression in line (6): 
\begin{align}
	D_\lambda \binom{P_{\calA, \calP}(Z | \Xs = s_i)}{P_{\calA, \calP}(Z | \Xs = s_j)}
	&\leq \sup_{\Delta \in \calD} \bigg( \frac{\lambda}{2 \sigma_s^2}  \| \Delta \|_2^2 
	+  \frac{\lambda}{2} \Delta^\intercal \Sigmaeff \Delta \bigg) \tag{7} \\
	&\leq  \frac{\lambda}{2}\bigg( \frac{1}{\sigma_s^2} S r^2 + S r^2 \text{maxeig}(\Sigmaeff) \bigg) \tag{8} \\
	&= \frac{\lambda}{2} S r^2 \big( \frac{1}{\sigma_s^2} + \alpha^* \big) \tag{9}
\end{align}
where line (8) distributes the supremum. For the right term, this is given by the maximum magnitude of all $\Delta \in \calD$ times the maximum eigenvalueof $\Sigmaeff$ which equals $S r^2 \text{maxeig}(\Sigmaeff)$. Line (9) simply substitutes $\alpha^* = \text{maxeig}(\Sigmaeff)$. 

%(explain how you went from $\Sigma_{us}^\intercal$ to $\Sigma_{su}$. Also explain why $(\Sigma_{ss}^{-1})^\intercal = \Sigma_{ss}^{-1}$ (cuz inverse of symmetric matrix is symmetric). Then move to $\Delta s$ notation. Also explain that Lemma 3 isnt needed. Can show this operating on distributions of $Z|s$, $Z|s'$ alone. 
\end{proof}

\subsubsection{Proof of Corollary \ref{cor: composition}}
\textbf{Corollary \ref{cor: composition}}
Graceful Composition in Time
\textit{
	Suppose a user releases two traces $X$ and $\hat{X}$ with additive noise $G \sim \calN(\mathbf{0}, \Sigmag)$ and $\hat{G} \sim \calN(\mathbf{0}, \hat{\Sigma}^{(g)})$, respectively. Then basic or compound secret $\Xs$ of $X$ enjoys $(\bar{\varepsilon}, \lambda)$-CIP, where 
	\begin{align*}
		\bar{\varepsilon} \leq \frac{\lambda}{2} S r^2 \Big(  \frac{1 }{\sigma_s^2} + \bar{\alpha}^*  \Big) 
	\end{align*}
	and where $\bar{\alpha}$ is the maximum eigenvalue of $\bar{\Sigma}_{\text{eff}} = \big(\Sigma_{us} \Sigma_{ss}^{-1}\big)^\intercal \big( \Sigma_{u | s} + \bar{\Sigma}_{uu}^{(g)} \big)^{-1} \big(\Sigma_{us} \Sigma_{ss}^{-1}\big)$. $\Sigma$ is the covariance matrix of the joint distribution on $X, \hat{X}$ and 
	\begin{align*}
	\bar{\Sigma}^{(g)} =
		\begin{bmatrix}
			 \Sigmag & 0 \\
			 0 &  \hat{\Sigma}^{(g)} \ .
		\end{bmatrix}
	\end{align*}
}

\begin{proof}
Here, we record two traces (presumably) far apart in time 
\begin{align*}
	(X_1, \dots, X_n) \text{ and } (\hat{X}_1, \dots, \hat{X}_m)
\end{align*}
And release
\begin{align*}
	(Z_1, \dots, Z_n) = (X_1, + G_1, \dots, X_n + G_n) \text{ and } (\hat{Z}_1, \dots, \hat{Z}_m) = (\hat{X}_1, + \hat{G}_1, \dots, \hat{X}_m, + \hat{G}_m)
\end{align*}
the first trace protects secret locations $\Xs$ and the second protects $\widehat{\Xs}$, so we have that 
\begin{align*}
	D_\lambda \binom{P_{\calA, \calP}(Z | \Xs = s_i)}{P_{\calA, \calP}(Z | \Xs = s_j)} &\leq \varepsilon \\
	D_\lambda \binom{P_{\calA, \calP}(\hat{Z} | \widehat{\Xs} = \hat{s}_i)}{P_{\calA, \calP}(\hat{Z} | \widehat{\Xs} = \hat{s}_j)} &\leq \hat{\varepsilon}
\end{align*}
We aim to update the losses: 
\begin{align*}
	D_\lambda \binom{P_{\calA, \calP}(Z, \hat{Z} | \Xs = s_i)}{P_{\calA, \calP}(Z, \hat{Z} | \Xs = s_j)} &\leq \varepsilon' \\
	D_\lambda \binom{P_{\calA, \calP}(\hat{Z}, Z | \widehat{\Xs} = \hat{s}_i)}{P_{\calA, \calP}(\hat{Z}, Z | \widehat{\Xs} = \hat{s}_j)} &\leq \hat{\varepsilon}'
\end{align*}
Fortunately, our framework is pretty friendly to figuring this out, and can be done simply by updating the `inferential loss term' $\alpha^*$ and $\hat{\alpha}^*$ of each, the max eigenvalues used to compute each of $\varepsilon$ and $\hat{\varepsilon}$, respectively. Let's focus on $\varepsilon'$, since the same analysis follows for $\hat{\varepsilon}'$.  

Recall that $\alpha^*$ is given by the max eigenvalue of $\Sigmaeff$ which is 
\begin{align*}
	\Sigmaeff 
	&= \big(\Sigma_{us} \Sigma_{ss}^{-1}\big)^\intercal \big( \Sigma_{u | s} + \Sigma_{uu}^{(g)} \big)^{-1} \big(\Sigma_{us} \Sigma_{ss}^{-1}\big)
\end{align*}
Where $\Sigma$ is the covariance matrix of $X_1, \dots, X_n$ and $\Sigmag$ is the noise covariance matrix added. Simply augment $\Sigma$ to become the joint covariance matrix $\Sigma_J$ of $X, \hat{X}$, and augment $\Sigmag$ to become 
\begin{align*}
	\Sigmag_J
	&= 
	\begin{bmatrix}
		\Sigmag & 0 \\
		0 & \hat{\Sigma}^{(g)}
	\end{bmatrix}
\end{align*}
then update $\Sigmaeff$ to $\Sigma_{\text{eff}, J}$ which uses both $\Sigma_J$ and $\Sigmag_J$. Using the corresponding max eigenvalue $\alpha^*_J$ in the loss expression of Theorem 3.2 gives us $\varepsilon'$. 

Note that for kernels like RBF, $\varepsilon' \rightarrow \varepsilon$ as the traces $X$ and $\hat{X}$ move apart further and further in time. This is not the case for traces using a purely periodic kernel with not time decay, and we should expect much worse composition. 
\end{proof}


\subsubsection{Traces with Independent Dimensions}
In many cases, the different dimensions of the trace may be probabilistically independent, and it may be more convenient to make separate privacy mechanisms for each. For a 2d trace $X$, suppose $\Ix$ and $\Iy$ store the indices of the latitude points $\Xx$ and longitude points $\Xy$, such that $X = \Xx \cup \Xy$. If latitude and longitude are independent, it may be more convenient to characterize the conditional priors of $\Xx$ abd $\Xy$ separately. The question is whether privacy guarantees remain for the full trace $X$. To answer this, we provide the following corollary: 

\begin{corollary}\emph{CIP loss of independent dimensions} 
\label{cor: independence}
	Let $\Theta$ be a GP conditional prior class on a 2d trace $X$ such that the dimensions are independent. Let $\Is$ be some secret set of time indices corresponding to some basic or compound secret. For the trace $X = \Xx \cup \Xy$, the Gaussian mechanism $\calA(X) = \Zx \cup \Zy$ where $\Zx = \calA_x(\Xx) = \Xx + \Gx$ and $\Zy = \calA_y(\Xy) = \Xy + \Gy$ satisfies $(\varepsilon, \lambda)$-CIP where
	\begin{align*}
		\varepsilon \leq \frac{\lambda}{2} S r^2 \big( \frac{1}{\sigma_s^2} + \alpha^*_x + \alpha^*_y \big) 
	\end{align*} 
	when $\calA_x$ and $\calA_y$ provide $\frac{\lambda}{2} S r^2 \big( \frac{1}{\sigma_s^2} + \alpha^*_x)$ and $\frac{\lambda}{2} S r^2 \big( \frac{1}{\sigma_s^2} + \alpha^*_y)$ to $\Is \cap \Ix$ and $\Is \cap \Iy$, respectively. 
\end{corollary}
The gist of this corollary is that a mechanism can be designed to achieve the bound of Theorem \ref{thm:GP bound} to each dimension independently and released with still-meaningful privacy guarantees. The reason is that this still includes all secret pairs $\Spairs$ 
\begin{proof}
	By independence, $\Xx$ and $\Xy$ can be treated as two unconnected traces of the type seen in \textbf{Figure \ref{fig:graphical models}}. As such the privacy guarantee of Theorem \ref{thm:GP bound} can be upheld for each. The question is whether bounding CIP loss to the one-dimensional basic or compound secret associated with secret sets $\Is \cap \Ix$ and $\Is \cap \Iy$ still provides guarantees for the full secret set $\Is$. 
	
	Without loss of generality, we will demonstrate for a basic and a compound secret. Consider the basic secret set $\Is = \{X_{10}, X_{11}\}$, where $\Is \cap \Ix = \{X_{10}\}$ (latitude) and $\Is \cap \Iy = \{X_{11}\}$ (longitude). We again assume that independent gaussian noise of variance $\sigma_s^2$ is added to all $\Xs$, since this is optimal under utility constraints. We have now bounded the R\'enyi divergence when conditioning on pairs of hypotheses on latitude and longitude separately. 
	\begin{align*} 
	{\Spairs}_x = {\Spairs}_y = \{(x_s, x_s'):x_s \in \R,  \|x_s - x_s'\|_2 \leq r\}
	\end{align*}
	By independence, this also bounds the R\'enyi divergence conditioning on pairs of hypotheses on latitude and longitude jointly: 
	\begin{align*} 
	{\Spairs}_{xy} = \{(x_s, x_s'):x_s \in \R^2,  \|x_s - x_s'\|_2 \leq r\}
	\end{align*}
	In effect, we have guaranteed privacy for any pair of hypotheses $(s_i, s_j)$ in the square circumscribing the circle of radius $r$ that we with to provide. The analysis on the direct privacy loss is exactly the same as it was in the more general case. Since the R\'enyi divergences of $\Xu \cap \Xx$ and of $\Xu \cap \Xy$ add, the $\alpha^*$'s add. 
	
	The same goes for a compound secret. Consider three location compound secret pairs given by 
	\begin{align*}
		{\Spairs}_{xy} = \Big\{\big( \{x_{s1}, x_{s2}, \dots\}, \{x_{s1}', x_{s2}', \dots\}\big): x_{si} \in \R^2, \| x_{sk} - x_{sk}' \|_2 \leq r, \forall \ k\Big\} 
	\end{align*} 
	Instead, we bound privacy loss for 
	\begin{align*}
		{\Spairs}_x = {\Spairs}_y = \Big\{\big( \{x_{s1}, x_{s2}, \dots\}, \{x_{s1}', x_{s2}', \dots\} \big): x_{si} \in \R, \| x_{sk} - x_{sk}' \|_2 \leq r, \forall \ k \Big\}
	\end{align*}
	Separately, giving us $\alpha_x^*$ and $\alpha_y^*$. This again includes any two hypotheses on the three locations such that each pair of $x_{sk}, x_{sk}'$ is within a square circumscribing a circle of radius $r$. We achieve this by bounding privacy loss for all $\Delta_x$ in a 3d $L_2$ ball of radius $\sqrt{S}  \ r$, as with $\Delta_y$. 
	
	This corollary can be extended to all traces of all dimensions that are probabilistically independent. 
\end{proof}

We make use of the above proof in the Experiments section. 

\subsection{Derivation of Algorithms}
\label{apx: algorithmns}
In this section, we derive the three SDP-based algorithms of Section \ref{sec: algorithms} and their properties. 

\subsubsection{Derivation of $\text{SDP}_\text{A}$}

$\text{SDP}_\text{A}$ minimizes the privacy loss bound of Theorem \ref{thm:GP bound} for any compound or basic secret encoded by secret set $\Is$. As is clarified in its proof (Appendix \ref{apx: GP bound proof}), the bound is tight when $\Is$ encodes a basic secret. If $\Is$ encodes a compound secret, the tightness depends on the conditional prior class $\Theta$. 

Our variable for minimizing this bound is the noise covariance matrix $\Sigmag$. Due to the conditional independence exhibited by Lemma \ref{lem: renyi additive loss}, $\Gs$ and $\Gu$ may be independent. The additive noise $G_i \in \Gs$ are all independent Gaussian with variance $\sigma_s^2$. This is because --- conditioning on $\{\Xs = x_s\}$ --- $\Zs$ is independent of $\Xu$ and $\Zu$. So, $\Gs \sim \calN(\mathbf{0}, \sigma_s^2 I)$, and $\Sigmag_{ss} = \sigma_s^2 I$. The additive noise $G_i \in \Gu$ are all dependent as described by $\Sigmag_{uu}$, and $\Gu \sim \calN(\mathbf{0}, \Sigmag_{uu})$. Consequently, $\Sigmag$ is completely characterized by $\Sigmag_{uu}$ and $\sigma_s^2$. 

To see how the bound of Theorem \ref{thm:GP bound} can be redrafted as an SDP, first notice that its two terms may be written as the maximum eigenvalue of a matrix product. Here, $\Sigmaeff = A^\intercal B A$, where $A = \Sigma_{us} \Sigma_{ss}^{-1}$ and $B = \big( \Sigma_{u | s} + \Sigmag_{uu} \big)^{-1}$
\begin{align*}
	\frac{1}{\sigma_s^2} + \alpha^*
	= \text{maxeig} \big( 
	\frac{1}{\sigma_s^2} I + A^\intercal B A \big)
	= \text{maxeig} \bigg(  
	\begin{bmatrix}
		I \  A
	\end{bmatrix} 
	\begin{bmatrix}
		\frac{1}{\sigma_s^2} I \ \ \  0 \\
		\quad 0 \quad  B
	\end{bmatrix}
	\begin{bmatrix}
		I \\ A
	\end{bmatrix}
	\bigg) 
	= \text{maxeig} \big( \tilde{A}^\intercal \tilde{B} \tilde{A} \big) 
\end{align*}
This expression uses all parameters of $\Sigmag$: $\sigma_s^2$ parametrizes $\Sigmag_{ss}$ and $\Sigmag_{uu} = B^{-1} - \Sigma_{u|s}$, where $\Sigma_{u|s}$ is given by the kernel function of $\Theta$. 

Before casting this as an SDP, we provide a formal definition from \cite{SDPs}: 

\begin{definition}\emph{Semidefinite Program} 
	\label{def: SDP}
	The problem of minimizing a linear function of a variable $x \in \R^n$ subject to a matrix inequality: 
	\begin{align*}
		\min_{x \in \R^n} \ &c^\intercal x \\
		&\text{s.t. } F_0 + \sum_{i=1}^n x_i F_i \succeq 0 \\
		& \quad \ \  Ax = b
	\end{align*}
	where the $F_i \in \R^{n \times n}$ are all symmetric and $A \in \R^{p \times n}$ is a \emph{semidefinite program}, or SDP. 
\end{definition}

The task of minimizing $\text{maxeig} \big( \tilde{A}^\intercal \tilde{B} \tilde{A} \big)$ under MSE constraints can almost be formulated as an SDP: 
\begin{align*}
	\min_{B \succeq 0 , \nicefrac{1}{\sigma_s^2} \geq 0} \ &\beta^* \\
	&\text{s.t. } \beta^* I  \succeq \tilde{A}^\intercal \tilde{B} \tilde{A} \\
	& \quad \ \ B \preceq \Sigma_{u|s}^{-1} \\
	&\quad \ \ \trace(\Sigmag_{uu}) + |\Is| \sigma_s^2 \leq n o_t 
\end{align*}
Here, the first constraint guarantees that the maximum eigenvalue of $\tilde{A}^\intercal \tilde{B} \tilde{A}$ is bounded by $\beta^*$, which the objective minimizes. At program completion, we set $\Sigmag_{uu} = B^{-1} - \Sigma_{u|s}$, and the second constraints ensures that this is still PSD. The final constraint bounds the MSE of the mechanism $\Sigmag$. Note that $\trace(\Sigmag_{uu}) + |\Is| \sigma_s^2 = \trace(\Sigmag)$. The trouble lies the last constraint. Our program variable is $B$, but the final linear constraint requires $\Sigmag$, which is expressed using the inverse of $B$. This is not immediately available in the SDP framework. 

To make the final linear constraint available, we invert the above program using the observation that the maximum eigenvalue of $\tilde{A}^\intercal \tilde{B} \tilde{A}$ is the inverse of the minimum eigenvalue of $(\tilde{A}^\intercal \tilde{B} \tilde{A})^{-1}$. Instead of optimizing over $B$ and $\nicefrac{1}{\sigma_s^2}$, we optimize over $B^{-1}$ and $\sigma_s^2$. Since $B^{-1} = \Sigma_{u|s} + \Sigmag_{uu}$, we may now have a utility constraint directly on the trace of $\Sigmag$. To make $B^{-1}$ our program variable, we approximate $(\tilde{A}^\intercal \tilde{B} \tilde{A})^{-1}$ with $\tilde{A}^{-1} \tilde{B}^{-1} \tilde{A}^{-\intercal}$. First note that $\tilde{A} \in \R^{n \times |\Is|}$, and has full column rank for the covariances we work with. So, $\tilde{A}^{-1} = (\tilde{A}^\intercal \tilde{A})^{-1}\tilde{A}^\intercal \in \R^{(|\Is| \times n)}$ is the left inverse of $\tilde{A}$ and is the least squares solution to $\tilde{A}^{-1} \tilde{A} = \tilde{A}^\intercal \tilde{A}^{-\intercal}  = I$ (we denote its transpose as $\tilde{A}^{-\intercal}$). It is also the least squares solution to $\tilde{A} \tilde{A}^{-1} = \tilde{A}^{-\intercal} \tilde{A}^\intercal = I$. Thus, we have an approximation of the inverse $(\tilde{A}^\intercal \tilde{B} \tilde{A})^{-1}$: 
\begin{align*}
	(\tilde{A}^\intercal \tilde{B} \tilde{A}) \ (\tilde{A}^{-1} \tilde{B}^{-1} \tilde{A}^{-\intercal})
	&\approx \tilde{A}^\intercal \tilde{B} \tilde{B}^{-1} \tilde{A}^{-\intercal} \\
	&= \tilde{A}^\intercal \tilde{A}^{-\intercal} \\
	&\approx I
\end{align*}

We now can optimize in terms of $B^{-1}$ with the augmented matrix $\tilde{B}^{-1}$: 
\begin{align*}
	\tilde{B}^{-1} = 
	\begin{bmatrix}
		\sigma_s^2 I \ \ \  0 \\
		\quad 0 \quad  B^{-1}
	\end{bmatrix}
\end{align*}

We then optimize the following SDP: 

\begin{align*}
	\max_{B^{-1} \succeq 0 , \sigma_s^2 \geq 0} \ &\beta^* \\
	&\text{s.t. } \beta^* I  \preceq \tilde{A}^{-1} \tilde{B}^{-1} \tilde{A}^{-\intercal} \\
	& \quad \ \ B^{-1} \succeq \Sigma_{u|s} \\
	&\quad \ \ \trace(\tilde{B}) -  \trace{(\Sigma_{u|s})} \leq n o_t 
\end{align*}
Upon program completion we recover $\sigma_s^2$ and $\Sigmag_{uu} = B^{-1} - \Sigma_{u|s}$ which we know is PSD due to the second constraint. The first constraint guarantees that the minimum eigenvalue of the approximated inverse is $\geq \beta^*$, which the objective maximizes. If the minimum eigenvalue of the approximate inverse is close to that of the true inverse, then we successfully minimize the maximum eigenvalue of $\tilde{A}^\intercal \tilde{B} \tilde{A}$, and thus minimize the direct and indirect privacy loss. The third constraint limits the MSE of $\Sigmag$ since $\trace(\tilde{B}) - \trace(\Sigma_{u|s}) = (\trace(\Sigmag_{uu}) + |\Is| \sigma_s^2 + \trace(\Sigma_{u|s})) - \trace(\Sigma_{u|s}) = \trace(\Sigmag)$. By inverting $\tilde{A}^\intercal \tilde{B} \tilde{A}$, this constraint is available in the SDP framework. 

By expressing the above program in terms of the variable $\Sigmag$ instead of indirectly via $B^{-1}$ and $\sigma_s^2$, we get $\text{SDP}_\text{A}$: 

\begin{align*}
	\textbf{SDP}_\textbf{A}: \quad 
	\argmax_{\Sigmag \succeq 0}& \ \beta^* \\
	\text{s.t. }& \tilde{A}^{-1} \tilde{B}^{-1} \tilde{A}^{-\intercal} \succeq \beta^* \mathbf{I} \\
	&\trace(\Sigmag) \leq n o_t
\end{align*}
It is straightforward to write this SDP in the form seem in Definition \ref{def: SDP}. The program variables $x$ would be the diagonal and upper or lower triangular part of $\Sigmag$ along with $\beta^*$. With some linear algebra, the first constraint can be written in the form of $F_0 + \sum_{i=1}^n x_i F_i \succeq 0$, and the second constraint can be written as $Ax = b$. With the use of contemporary convex programming tools like CVXOPT \citep{cvxopt} rewriting into this form is unnecessary. 

%With the derivation of the above program, the proof of Theorem \ref{thm: SDP optimal} is clear. 
%
%\textbf{Theorem \ref{thm: SDP optimal}} SIG OPT versus isotropic:
%\emph{
%For a basic or compound secret denoted by indices $\Is$, the CIP loss bound of Equation \ref{eqn: priv bound} provided by a Gaussian noise mechanism with covariance \\$\Sigmag =$ SIG OPT$(\Is, \Sigma, o_t)$ is less than or equal to that of an isotropic mechanism of equal MSE $\Sigmag = o_t I$ if the minimum eigenvalue of $\tilde{A}^{-1} \tilde{B}^{-1} \tilde{A}^{-\intercal}$ equals that of $ (\tilde{A}^\intercal \tilde{B} \tilde{A})^{-1}$ for all $\tilde{B}$.
%}
%\begin{proof}
%	The proof is nearly by construction. If the minimum eigenvalue of $\tilde{A}^{-1} \tilde{B}^{-1} \tilde{A}^{-\intercal}$ equals that of $ (\tilde{A}^\intercal \tilde{B} \tilde{A})^{-1}$ for all $\tilde{B}$ then so do the maximum eigenvalues of their inverses. So, the $\tilde{B}$ that maximizes the minimum eigenvalue of our approximation $\tilde{A}^{-1} \tilde{B}^{-1} \tilde{A}^{-\intercal}$ also minimizes the maximum eigenvalue of $\tilde{A}^\intercal \tilde{B} \tilde{A}$ which equals the the privacy loss bound $\frac{1}{\sigma_s^2} + \alpha^*$ (constants $2 \lambda S r^2$ aside).
%	
%	Since the isotropic mechanism $\Sigmag = o_t I$ is in the feasible set of solutions, we are guaranteed that the covariance chosen by SIG OPT produces a smaller lower bound on CIP loss. 
%\end{proof}
%
%The intuition of the theorem is that if $\tilde{A}^{-1} \tilde{B}^{-1} \tilde{A}^{-\intercal}$ is a good approximation of $(\tilde{A}^\intercal \tilde{B} \tilde{A})^{-1}$, then the SDP is optimal. To show how `good' the approximation must be, consider the following. Let $f(\Sigmag) = \text{mineig}\big((\tilde{A}^\intercal \tilde{B} \tilde{A})^{-1}\big)$. Let our approximation to $f$ be $\hat{f}(\Sigmag) = \text{mineig} \big( \tilde{A}^{-1} \tilde{B}^{-1} \tilde{A}^{-\intercal} \big)$. Let the true optimal noise covariance be $\Sigmag_{\text{opt}} = \argmax_{\Sigmag \in \mathcal{T}} f(\Sigmag)$, where $\mathcal{T}$ is the set of all covariance matrices with MSE bounded by $n o_t$. Then, if for all $\Sigmag \in \mathcal{T}$ 
%\begin{align*}
%	|f(\Sigmag) - \hat{f}(\Sigmag)| \leq \delta 
%	\quad \quad \text{where} \quad \quad 
%	\delta = |f(\Sigmag_{\text{opt}}) - f(o_t I)|
%\end{align*}
%SIG OPT will return a covariance matrix that reduces the bound of Equation \ref{eqn: priv bound} better than an isotropic mechanism.

\subsubsection{Derivation of $\text{SDP}_\text{B}$ }
\label{apx: SDP B}
$\text{SDP}_\text{B}$ takes a set of covariance matrices $\calF = \{\Sigma_1, \dots, \Sigma_k\}$, each of which is designed to protect some secret set ${\Is}_i$, and returns a covariance matrix $\Sigmag$ that preserves the privacy loss bound of each $\Sigma_i$ to each ${\Is}_i$. It does so while minimizing the utility loss of $\Sigmag$. This algorithm is also expressed as an SDP. It is based on the following corollary, which we have omitted from the main text: 
\begin{corollary}\emph{More PSD, More Private: }
\label{cor: more_psd}
	For a basic or compound secret denoted by indices $\Is$, the CIP loss bound of Equation \ref{eqn: priv bound} provided by a Gaussian noise mechanism with covariance $\Sigmag$ is lower than it would be for any ${\Sigmag}' \prec \Sigmag$. 
\end{corollary}
\begin{proof}
	First note that if $\Sigmag \succ {\Sigmag}' $, then the same is true for its sub-matrices: 
	\begin{align*}
		\Sigmag_{ss} \succ {\Sigmag_{ss}}'
		\quad \quad
		\Sigmag_{uu} \succ {\Sigmag_{uu}}'
	\end{align*}
	Recall the privacy loss bound of Equation \ref{eqn: priv bound}: 
	\begin{align*}
		\varepsilon \leq \frac{\lambda}{2} S r^2 \Big(  \frac{1 }{\sigma_s^2} + \alpha^*  \Big)
	\end{align*}
	Also recall that $\Sigmag_{ss} = \sigma_s^2 I$ and ${\Sigmag_{ss}}' = {\sigma_s^2}' I$. Since $\Sigmag_{ss} \succ {\Sigmag_{ss}}'$, we already know that $\sigma_s^2 > {\sigma_s^2}'$, and thus the first term of Equation \ref{eqn: priv bound} is lower for $\Sigmag$.
	
	It remains to show that the second term is also lower, $\alpha^* < {\alpha^*}'$. Starting with what we're given, 
	\begin{align*}
		\Sigmag_{uu} &\succ {\Sigmag_{uu}}' \\
		\Sigmag_{uu} + \Sigma_{u|s} &\succ {\Sigmag_{uu}}' + \Sigma_{u|s} \\
		(\Sigmag_{uu} + \Sigma_{u|s})^{-1} &\prec ({\Sigmag_{uu}}' + \Sigma_{u|s})^{-1} \\
		B &\prec B' \\
		A^\intercal B A &\prec A^\intercal B' A \\
		\maxeig(A^\intercal B A) &< \maxeig(A^\intercal B' A) \\
		\alpha^* &< {\alpha^*}'
	\end{align*}
	Therefore $\frac{1}{\sigma_s^2} + \alpha^* < \frac{1}{{\sigma_s^2}'} + {\alpha^*}'$, and the CIP bound of Equation \ref{eqn: priv bound} is lower for $\Sigmag$ than it is for ${\Sigmag}'$. 
\end{proof}
With Corollary \ref{cor: more_psd} in mind, $\text{SDP}_\text{B}$ is natural: 

\begin{align*}
	\textbf{SDP}_\textbf{B}: \quad 
	\argmin_{\Sigmag } \  &\trace(\Sigmag) \\
	\text{s.t. }& \Sigmag \succeq \Sigmag_i , \ \forall \Sigmag_i \in \calF
\end{align*}

$\text{SDP}_\text{B}$ attempts to minimize, but does not constrain, the utility loss of the chosen $\Sigmag$. To provide an upper bound on the resulting utility loss, we provided the following claim in the main text: 

\textbf{Claim} Utility loss of $\text{SDP}_\text{B}$: 
\emph{
	The utility loss of $\Sigmag = \text{SDP}_\text{B}(\calF)$ is no greater than $\sum_{\Sigma_i \in \calF} \trace(\Sigma_i)$. 
}
\begin{proof}
	The covariance ${\Sigmag}' = \sum_{\Sigmag_i \in \calF} \Sigmag_i$ with MSE $\sum_{\Sigmag_i \in \calF} \trace(\Sigmag_i)$ is in the feasible set of $\text{SDP}_\text{B}$ problem since ${\Sigmag}' \succeq \Sigmag_i, \ \forall \Sigmag_i \in \calF$. Unless ${\Sigmag}'$ has the lowest MSE of all $\Sigmag$ in the feasible set, a covariance matrix with better utility will be chosen. 
\end{proof}

\subsubsection{Derivation of Algorithm \ref{alg: Multiple Secrets}, Multiple Secrets}

Multiple Secrets combines $\text{SDP}_\text{A}$ and $\text{SDP}_\text{B}$ to minimize the privacy loss to each basic secret within a trace. The basic mechanism is useful in cases when inferences at each time within the trace --- each basic secret --- is sensitive. 

Let ${\Is}_i$ be the secret set representing basic secret $i$, of which there are $N$ (e.g. if location is sampled at $N$ times). Then $\mathbb{I}_{\calS_b} = \{{\Is}_1, \dots, {\Is}_N\}$ contains the indices corresponding to each. Multiple Secrets works by first producing $N$ covariance matrices, $\Sigmag_i$ = $\text{SDP}_\text{A}$$({\Is}_i, \Sigma, o_t)$ on each basic secret. It then uses $\text{SDP}_\text{B}$($\calF = \{\Sigmag_1, \dots, \Sigmag_N\}$) to produce a single covariance matrix $\Sigmag$ that preserves the privacy loss to each basic secret (note that, being basic secrets, the privacy loss bound that SIG OPT optimizes is tight). 

By virtue of using $\text{SDP}_\text{B}$, the MSE of the resultant $\Sigmag$ is minimized but not constrained. To bound the MSE of the Basic Mechanism by $O$, we may simply bound the MSE of each $\Sigmag_i$ by $o_t = \nicefrac{O}{N}$. Then, by the above Claim, the MSE of the solution cannot be greater than $O$. In practice, this bound may be too loose. We hope to tighten it in future work. 

\subsection{Experimental details}
\label{apx: experiments}

We use a 2d location trace and a 1d home temperature dataset. For the location data, having observed that the correlation between latitude and longitude is low ($ \approx 0.06$) we treat each dimension as independent. By way of Corollary \ref{cor: independence}, this allows us to bound privacy loss and design mechanisms for each dimension separately. Furthermore, having observed that each dimension fits the nearly the same conditional prior, we treat our dataset of 10k 2-dimensional traces as a dataset of 20k 1-dimensional traces, where each trace represents one dimension of a 2d location trajectory. 

The one-dimensional traces of temperature and location are indexed by timestamps, for which we would use the following kernel functions: 

\begin{align}
	k_{\text{RBF}}(t_i, t_j) 
	=  \sigma_x^2 \exp \Big( -\frac{(t_i - t_j)^2}{2 l^2} \Big) 
	\quad \quad 
	k_{\text{PER}}(t_i, t_j) 
	=  \sigma_x^2 \exp \Big(  \frac{-2 \sin^2(\pi |t_i - t_j| / p)}{l^2} \Big)
\end{align}

to determine the covariance between two points sampled at times $t_i$ and $t_j$. The parameters including variance $\sigma_x^2$ and length scale $l$. The lengthscale determines the window of time in which two sampled points are highly correlated. 

\paragraph{Preprocessing of location data} We first limit the dataset to traces of under 50 locations that are between 4.5 and 5.5 minutes in duration. Caring only about the conditional dependence between locations, we then de-mean each trace and normalize its variance to one. Normalizing the variance of traces implicitly sets $\sigma_x^2 = 1$ in the above RBF kernel, in essence assuming that the adversary has a decent prior for the user's average speed in a given trace, and could do the same operation. 

\paragraph{Fitting of location data} We then find the maximum likelihood RBF kernel for each distinct trace. Having fixed the variance $\sigma_x^2$, this amounts to fitting only the length scale for each dimension, $l_x$ and $l_y$, individually. The length scale represents the average window of time during which neighboring locations are highly correlated (i.e. correlation $ > 0.8$). Relatively smooth traces will have large length scales and chaotic traces will have low length scales. However, the fact that sampling rates vary significantly between traces means that traces with equal length scales can have very different degrees of correlation. To encapsulate both of these effects, we study the empirical distribution of \emph{effective} length scale of each trace
\begin{align*}
	l_{\text{eff},x} = \frac{l_x}{P}
	\quad
	l_{\text{eff},y} = \frac{l_y}{P}
\end{align*}
where $P$ is the trace's sampling period and $l_x,l_y$ are the its optimal length scales. $l_{\text{eff},x},l_{\text{eff},y}$ tell us the average number of neighboring locations that are highly correlated, instead of time period. For instance, a given trace with an optimal $l_{\text{eff},x} = 8$ tells us that every eight neighboring location samples in the $x$ dimension have correlation $> 0.8$. The empirical distribution of effective length scales across all traces describes -- over a range of logging devices (sampling rates), users, and movement patterns -- how many neighboring points are highly correlated in location trace data. After this preprocessing, we are able to use the kernels that take indices (not time) as arguments. 

\begin{align*}
	\label{eqn: kernels}
	k_{\text{RBF}}(i, j) 
	=  \exp \Big( -\frac{(i - j)^2}{2\leff^2} \Big) 
	\quad \quad 
	k_{\text{PER}}(i, j) 
	=  \exp \Big(  \frac{-2 \sin^2(\pi |i - j| / p)}{\leff^2} \Big)
\end{align*}

In each plot we then observed a spectrum of conditional priors by sweeping the effective length scale and plotting posterior uncertainty for various noise mechanisms of equal utility loss. This ranges from a prior assuming nearly independent location samples (chaotic trace) on the left up to highly dependent location samples (traveling in a straight line or standing still) on the right. To understand how realistic these conditional prior parameters are, we displayed the middle 50\% of the empirical distribution of $l_{\text{eff}}$ ($x$ and $y$ together) from the GeoLife dataset. Note that the distribution of ${\leff}_x$ and ${\leff}_y$ are nearly identical. 

To compute posterior uncertainty, we consider a 50-point one-dimensional location trace. The basic secret is a single index in the middle of the trace, and the compound secret consists of two neighboring indices also in the middle of trace. For each value of $l_{\text{eff}}$, we compute the $\R^{50 \times 50}$ conditional prior covariance matrix $\Sigma$ using the RBF kernel above. We then compare the posterior uncertainty when $\Sigmag$ is an Approach C baseline, or an optimized covariance matrix using one of the three algorithms. We re-optimize $\Sigmag$ for each $\leff$, since each $\leff$ represents a different conditional prior class. The MSE is fixed in all figures except the two exhibiting ``All Basic Secrets'', where $\text{SDP}_\text{B}$ is used. Recall that this algorithm minimizes utility loss while maintaining a series of privacy guarantees. Here, the MSE is identical across mechanisms for each $\leff$, but changes from one $\leff$ to another. 

For the temperature data, our preprocessing steps were nearly identical, except we use the periodic kernel instead of the RBF kernel, and we did not need to remove any traces from the dataset, as the data was much cleaner. 

\paragraph{Computation of Posterior Uncertainty Interval}
Each of the plots in \textbf{Figure \ref{fig: experiments}} shows the $2\sigma$ uncertainty interval on $\Xs$ of a Gaussian process Bayesian adversary with prior covariance $\Sigma$ and any mean function

The posterior covariance is computed using standard formulas for linear Gaussian systems. Knowing that $Z = X + G$, we may write the joint precision matrix $\Lambda$ (inverse of covariance matrix) of $(X,Z)$ as 
\begin{align*}
	\Lambda^{(X,Z)}
	&= 
	\begin{bmatrix}
		\Sigma^{-1} + {\Sigmag}^{-1} & -{\Sigmag}^{-1} \\
		-{\Sigmag}^{-1} & {\Sigmag}^{-1} 
	\end{bmatrix}
\end{align*}

It is then a well known result that the conditional covariance matrix is given by 
\begin{align*}
	\Sigma_{x|z} &= \Lambda_{xx}^{-1}  \\
	&= \big(\Sigma^{-1} + {\Sigmag}^{-1}\big)^{-1}
\end{align*}
This provides the posterior covariance of all locations $X$ given any released trace $Z$ that uses a Gaussian mechanism with covariance $\Sigmag$. Note that the CIP guarantee naturally keeps posterior uncertainty large since the posterior density at any two $x_s$ close together must be similar. For these Gaussian posteriors, $2 \sigma$ tells us the adversary's 68\% confidence interval on $\Xs$ after obvserving $Z$. 

For basic secrets (one location), we simply report twice the posterior standard deviation at the sensitive index $i$, given by 
\begin{align*}
2 \sqrt{ \Sigma_{{x|z, ii}} } \ .
\end{align*}  
For compound secrets involving multiple locations the posterior distribution is a length $|\Is|$ multivariate normal with covariance $\Sigma_{x|z, ss}$. Intuitively, we wish to find the direction of the vector $\Xs$ in which the posterior interval is the \emph{shortest}. This is the worst case posterior interval on the compound secret. We do this by reporting 
\begin{align*}
	2 \sqrt{\text{mineig}\  \Sigma_{{x|z, ss}}} \ .
\end{align*}

\subsection{Discussion of GP Conditional Prior Class}
\label{apx: GP prior class}

Recall that a conditional prior class requires for any $P_{\calP_i}, P_{\calP_j} \in \Theta$ that 
\begin{align*}
	P_{\calP_i}(\Xu | \Xs = x_s)
	&= P_{\calP_j}(\Xu + c_{ij\Is}^u | \Xs = x_s + c_{ij\Is}^s)
\end{align*}
for all $x_s$. Notice that the mapping $(x_s, x_s') + c_{ij\Is}^s$ is a bijection from $\Spairs$ onto itself. As such, each pair of conditional distributions, 
\begin{align*}
	\Big(P_{\calP_j}(\Xu | \Xs = x_s), P_{\calP_j}(\Xu | \Xs = x_s')\Big)
\end{align*}
induced by $(x_s, x_s') \in \Spairs$ is a mean-shifted version of the pair of distributions 
\begin{align*}
	\Big(P_{\calP_i}(\Xu | \Xs = x_s - c_{ij\Is}^s), P_{\calP_i}(\Xu | \Xs = x_s' - c_{ij\Is}^s)\Big)
\end{align*}
induced by $(x_s , x_s' ) - c_{ij\Is}^s \in \Spairs$. Since the R\'enyi divergence between two distributions and two mean-shifted versions thereof is unchanged, we may use one additive noise mechanism for all priors in class $\Theta$.  

To see how this applies to the GP prior class, recall the formula for a conditional multivariate Gaussian distribution: 
\begin{align*}
	P(\Xu | \Xs = x_s)
	&= \calN(\mu_{u|s} , \Sigma_{u|s})
\end{align*}
where, 
\begin{align*}
		\mu_{u|s} &= \mu_u + \Sigma_{us} \Sigma_{ss}^{-1} (x_s - \mu_s) \\
		\Sigma_{u|s} &= \Sigma_{uu} - \Sigma_{us}\Sigma_{ss}^{-1} \Sigma_{su}
\end{align*}
A GP prior class includes all GP distributions with a fixed kernel $k(t_i, t_j)$ and any mean function $\mu(t)$. For a fixed set of time points, this corresponds to a fixed covariance matrix $\Sigma$ and any mean parameters $\bmu$: 
\begin{align*}
	X \sim \calN(\bmu, \Sigma)
\end{align*}

Let $P_{\calP_i} = \calN(\bar{\bmu}, \Sigma)$ and $P_{\calP_j} = \calN(\hat{\bmu}, \Sigma)$, then conditioned on some sensitive points $\Xs$ the distribution on $\Xu$ has the same covariance $\Sigma_{u|s}$ and conditional means 
\begin{align*}
	\bar{\mu}_{u|s}
	&= \bar{\mu}_u + \Sigma_{us} \Sigma_{ss}^{-1} (x_s - \bar{\mu}_s) \\
	&= (\bar{\mu}_u - \Sigma_{us} \Sigma_{ss}^{-1}\bar{\mu}_s) + \Sigma_{us} \Sigma_{ss}^{-1} x_s \\
	\hat{\mu}_{u|s}
	&= \hat{\mu}_u + \Sigma_{us} \Sigma_{ss}^{-1} (x_s - \hat{\mu}_s) \\
	&= (\hat{\mu}_u - \Sigma_{us} \Sigma_{ss}^{-1}\hat{\mu}_s) + \Sigma_{us} \Sigma_{ss}^{-1} x_s 
\end{align*}
which implies that the conditional distributions are identical up to a mean shift for the \emph{same} $x_s$ value. 
\begin{align*}
	P_{\calP_i}(\Xu | \Xs = x_s)
	&= P_{\calP_j}(\Xu + c_{ij\Is}^u | \Xs = x_s)
\end{align*}
for all $x_s$. Here, $c_{ij\Is}^u = (\bar{\mu}_u - \Sigma_{us} \Sigma_{ss}^{-1}\bar{\mu}_s) - (\hat{\mu}_u - \Sigma_{us} \Sigma_{ss}^{-1}\hat{\mu}_s)$, and $c_{ij\Is}^s = 0$. 

To see how this allows a single additive mechanism to work for all mean functions, notice that we also have 
\begin{align*}
	P_{\calP_i}(\Xu | \Xs = x_s')
	&= P_{\calP_j}(\Xu + c_{ij\Is}^u | \Xs = x_s')
\end{align*}
for $x_s'$, so the divergences 
\begin{align*}
	D_\lambda \binom{P_{\calP_i}(\Xu | \Xs = x_s)}{P_{\calP_i}(\Xu | \Xs = x_s')}
	&= D_\lambda \binom{P_{\calP_j}(\Xu + c_{ij\Is}^u | \Xs = x_s)}{P_{\calP_j}(\Xu + c_{ij\Is}^u | \Xs = x_s')} \\
	&= D_\lambda \binom{P_{\calP_j}(\Xu  | \Xs = x_s)}{P_{\calP_j}(\Xu  | \Xs = x_s')}
\end{align*}
are equal. The same goes for the noisy trace $\Xu + \Zu | \Xs = x_s$, when $Z$ is drawn independently of $X$, allowing us to bound privacy loss for all $P \in \Theta$. 









% Stuff at the end of the dissertation goes in the back matter
\backmatter
\bibliographystyle{unsrtnat} % Or whatever style you want like plainnat
\bibliography{references}

\end{document}
