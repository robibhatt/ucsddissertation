\begin{figure*}[t!]
%%%
%DAM
%%%
     \centering
     \begin{subfigure}[b]{0.49\textwidth}
         \centering
         \includegraphics[width=\textwidth]{figures/dam_corr.png}
         \caption{A {\color{part_blue}correlated} dam example}
         \label{fig:dam correlated}
     \end{subfigure}
     \hfill
     \begin{subfigure}[b]{0.49\textwidth}
         \centering
         \includegraphics[width=\textwidth]{figures/dam_mem.png}
         \caption{A {\color{part_orange}memorized} dam example}
         \label{fig:dam memorized}
     \end{subfigure}
\caption{
{\color{part_blue}Correlated} and {\color{part_orange}Memorized} examples from the \emph{dam} class. Both $\SSL_A$ and $\SSL_B$ are SimCLR models.
\textbf{Left:} The periphery crop (pink square) contains a concrete structure that is often present in images of dams. Consequently, the trained RCDM can reconstruct the foreground object using representations from both $\SSL_A$ and $\SSL_B$ through this correlation.
\textbf{Right:} The periphery crop only contains a patch of water. The embedding produced by $\SSL_B$ only contains enough information to infer that the foreground object is related to water, as reflected by its KNN set and RCDM reconstruction. In contrast, the embedding produced by $\SSL_A$ memorizes the association of this patch of water with dam and the RCDM can visualize the embedding to produce images of dams.
}
\vspace{-1ex}
\label{fig:mem v corr dam}
\end{figure*}