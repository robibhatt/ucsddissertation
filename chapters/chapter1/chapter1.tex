\graphicspath{{./chapters/chapter1/}}
%\newtheorem{thm}{Theorem}
%\newtheorem{lem}[thm]{Lemma}
\def\supp{supp}
%\newtheorem{cor}[thm]{Corollary}
%\newtheorem{defn}[thm]{Definition}
\def\D{{\mathcal D}}
\def\U{{\mathcal U}}
\def\V{{\mathcal V}}
\def\X{\mathcal X}
\def\R{\mathbb R}
\def\Y{\{\pm 1\}}
\def\d{\rho}
\def\E{\mathbb{E}}
\def\N{\mathbb{N}}
\def\g{g}
\def\A{\mathcal{A}}
\def\nat{g_{neighbor}}
\def\bad{\D_{1/2}^{-}}
\def\natural{neighborhood preserving}
\def\Natural{Neighborhood preserving}
\def\ncons{neighborhood}
\def\Ncons{Neighborhood}

\def\calD{\mathcal{D}}
\def\calU{\mathcal{U}}
\def\calV{\mathcal{V}}

\chapter{Consistent Non-Parametric Methods for Maximizing Robustness} 

\section{Introduction}
Adversarially robust classification, that has been of much recent interest, is typically formulated as follows. We are given data drawn from an underlying distribution $D$, a metric $d$, as well as a pre-specified robustness radius $r$. We say that a classifier $c$ is $r$-robust at an input $x$ if it predicts the same label on a ball of radius $r$ around $x$. Our goal in robust classification is to find a classifier $c$ that maximizes astuteness, which is defined as accuracy on those examples where $c$ is also $r$-robust. 

While this formulation has inspired a great deal of recent work, both theoretical and empirical \cite{Carlini17, Liu17, Papernot17, Papernot16,Szegedy14, Hein17,Schmidt18,Wu16,Steinhardt18, Sinha18, Yang20}, a major limitation is that enforcing a pre-specified robustness radius $r$ may lead to sub-optimal accuracy {\em{and}} robustness. To see this, consider what would be an ideally robust classifier the example in Figure~\ref{fig:intro}. For simplicity, suppose that we know the data distribution. In this case, a classifier that has an uniformly large robustness radius $r$ will misclassify some points from the blue cluster on the left, leading to lower accuracy. This is illustrated in panel (a), in which large robustness radius leads to intersecting robustness regions. On the other hand, in panel (b), the blue cluster on the right is highly separated from the red cluster, and could be accurately classified with a high margin. But this will not happen if the robustness radius is set small enough to avoid the problems posed in panel (a). Thus, enforcing a fixed robustness radius that applies to the entire dataset may lead to lower accuracy and lower robustness.

In this work, we propose an alternative formulation of robust classification that ensures that in the large sample limit, there is no robustness-accuracy trade off, and that regions of space with higher separation are classified more robustly. An extra advantage is that our formulation is achievable by existing methods. In particular, we show that two very common non-parametric algorithms -- nearest neighbors and kernel classifiers -- achieve these properties in the large sample limit.


\begin{figure}
\begin{subfigure}{0.45\textwidth}
\includegraphics[width=\linewidth]{big_radius_robustness.png}
\caption{Large robustness radii} 
\end{subfigure}\hspace*{\fill}
\begin{subfigure}{0.45\textwidth}
\includegraphics[width=\linewidth]{small_radius_robustness.png}
\caption{Small robustness radii} 
\end{subfigure}

\caption{A data distribution demonstrating the difficulties with fixed radius balls for robustness regions. The red represents negatively labeled points, and the blue positive. If the robustness radius is set too large (panel (a)), then the regions of A and B intersect leading to a loss of accuracy. If the radius is set too small (panel (b)), this leads to a loss of robustness at point C where in principle it should be possible to defend against a larger amount of adversarial attacks.}\label{fig:intro}
\end{figure}

%\begin{figure}[ht]
%	
%	\subfloat[Large robustness radii]{\includegraphics[width=.45\textwidth]{big_radius_robustness.png}}\hfill
%	\subfloat[Small robustness radii]{\includegraphics[width=.45\textwidth]{small_radius_robustness.png}}
%	\caption{A data distribution demonstrating the difficulties with fixed radius balls for robustness regions. The red represents negatively labeled points, and the blue positive. If the robustness radius is set too large (panel (a)), then the regions of A and B intersect leading to a loss of accuracy. If the radius is set too small (panel (b)), this leads to a loss of robustness at point C where in principle it should be possible to defend against a larger amount of adversarial attacks.}
%	
%	\label{fig:intro}
%\end{figure}

Our formulation is built on the notion of a new large-sample limit. In the standard statistical learning framework, the large-sample ideal is the Bayes optimal classifier that maximizes accuracy on the data distribution, and is undefined outside. Since this is not always robust with radius $r$, prior work introduces the notion of an $r$-optimal classifier~\cite{YRWC19} that maximizes accuracy on points where it is also $r$-robust. However, this classifier also suffers from the same challenges as the example in Figure~\ref{fig:intro}. 

We depart from both by introducing a new limit that we call the \natural\emph{ }Bayes optimal classifier, described as follows. Given an input $x$ that lies in the support of the data distribution $D$, it predicts the same label as the Bayes optimal. On an $x$ outside the support, it outputs the prediction of the Bayes Optimal on the nearest neighbor of $x$ {\em{within}} the support of $D$. The first property ensures that there is no loss of accuracy -- since it always agrees with the Bayes Optimal within the data distribution. The second ensures higher robustness in regions that are better separated. Our goal is now to design classifiers that converge to the \natural\emph{ }Bayes optimal in the large sample limit; this ensures that with enough data, the classifier will have accuracy approaching that of the Bayes optimal, as well as higher robustness where possible without sacrificing accuracy. 

We next investigate how to design classifiers with this convergence property. Our starting point is classical statistical theory~\cite{Stone77} that shows that a class of methods known as weight functions will converge to a Bayes optimal in the large sample limit provided certain conditions hold; these include $k$-nearest neighbors under certain conditions on $k$ and $n$, certain kinds of decision trees as well as kernel classifiers. Through an analysis of weight functions, we next establish precise conditions under which they converge to the \natural\emph{ }Bayes optimal in the large sample limit. As expected, these are stronger than standard convergence to the Bayes optimal. In the large sample limit, we show that $k_n$-nearest neighbors converge to the \natural\emph{ }Bayes optimal provided $k_n = \omega(\log n)$, and kernel classifiers converge to the \natural\emph{ }Bayes optimal provided certain technical conditions (such as the bandwidth shrinking sufficiently slowly). By contrast, certain types of histograms do not converge to the \natural\emph{ }Bayes optimal, even if they do converge to the Bayes optimal.  We round these off with a lower bound that shows that for nearest neighbor, the condition that $k_n = \omega(\log n)$ is tight. In particular, for $k_n = O(\log n)$, there exist distributions for which $k_n$-nearest neighbors provably fails to converge towards the \natural\emph{ }Bayes optimal (despite converging towards the standard Bayes optimal). 


%For $r$-robustness, the story is a little different, and in the general case -- that is, when the classes are close together or overlapping in space -- none of these classifiers may converge to the $r$-optimal. Instead\cite{Bhattacharjee20} shows that the Adversarial Pruning procedure of~\cite{YRWC19} that removes a subset of training examples to make data well-separated, followed by $k$-nearest neighbors or kernel classifiers, does converge to the $r$-optimal in the large sample limit.


In summary, the contributions of the paper are as follows. First, we propose a new large sample limit the \natural\emph{ }Bayes optimal and a new formulation for robust classification. We then establish conditions under which weight functions, a class of non-parametric methods, converge to the \natural\emph{ }Bayes optimal in the large sample limit. Using these conditions, we show that $k_n$-nearest neighbors satisfy these conditions when $k_n = \omega(\log n)$, and kernel classifiers satisfy these conditions provided the kernel function $K$ has faster than polynomial decay, and the bandwidth parameter $h_n$ decreases sufficiently slowly. 

To complement these results, we also include negative examples of non-parametric classifiers that do not converge. We provide an example where histograms do not converge to the \natural\emph{ }Bayes optimal with increasing $n$. We also show a lower bound for nearest neighbors, indicating that $k_n = \omega(\log n)$ is both necessary and sufficient for convergence towards the \natural\emph{ }Bayes optimal. 

Our results indicate that the \natural\emph{ }Bayes optimal formulation shows promise and has some interesting theoretical properties. We leave open the question of coming up with other alternative formulations that can better balance both robustness and accuracy for all kinds of data distributions, as well as are achievable algorithmically. We believe that addressing this would greatly help address the challenges in adversarial robustness.

\section{Preliminaries}


We consider binary classification over $\R^d \times \Y$, and let $\d$ denote any distance metric on $\R^d$. We let $\mu$ denote the measure over $\R^d$ corresponding to the probability distribution over which instances $x \in \R^d$ are drawn. Each instance $x$ is then labeled as $+1$ with probability $\eta(x)$  and $-1$ with probability $1 - \eta(x)$. Together, $\mu$ and $\eta$ comprise our data distribution $\D = (\mu, \eta)$ over $\R^d \times \Y$. 


For comparison to the robust case, for a classifier $f: \R^d \to \{\pm 1\}$ and a distribution $\D$ over $\R^d \times \{\pm 1\}$, it will be instructive to consider its  \textbf{accuracy},  denoted $A(f, \D)$, which is defined as the fraction of examples from $\D$ that $f$ labels correctly. Accuracy is maximized by the \textbf{Bayes Optimal classifier}: which we denote by $\g$. It can be shown that for any $x \in \supp(\mu)$, $\g(x) = 1$ if $\eta(x) \geq \frac{1}{2}$, and $\g(x) = -1$ otherwise. 

Our goal is to build classifiers $\R^d \to \Y$ that are both accurate and robust to small perturbations. For any example $x$, perturbations to it are constrained to taking place in the \textbf{robustness region} of $x$, denoted $U_x$. We will let $\U = \{U_x: x \in \R^d\}$ denote the collections of all robustness regions. 

We say that a classifier $f: \R^d \to \{\pm 1\}$ is \textbf{robust} at $x$ if for all $x' \in U_x$, $f(x') = f(x)$. Combining robustness and accuracy, we say that classifier is \textbf{astute} at a point $x$ if it is both accurate and robust. Formally, we have the following definition. 

\begin{defn}
A classifier $f: \R^d \to \Y$ is said to be \textbf{astute} at $(x,y)$ with respect to robustness collection $\U$ if $f(x) = y$ and $f$ is robust at $x$ with respect to $\U$. If $\D$ is a data distribution over $\R^d \times \Y$, the \textbf{astuteness} of $f$ over $\D$ with respect to $\U$, denoted $A_\U(f, \D)$, is the fraction of examples $(x,y) \sim \D$ for which $f$ is astute at $(x,y)$ with respect to $\U$. Thus $$A_\U(f, \D) = P_{(x, y) \sim \D}[f(x') = y, \forall x' \in \U_x].$$
\end{defn}




\paragraph{Non-parametric Classifiers} We now briefly review several kinds of non-parametric classifiers that we will consider throughout this paper. We begin with \textit{weight functions}, which are a general class of non-parametric algorithms that encompass many classic algorithms, including nearest neighbors and kernel classifiers.

\textbf{Weight functions} are built from training sets, $S = \{(x_1, y_1), (x_2, y_2,), \dots, (x_n, y_n)\}$ by assigning a function $w_i^S: \R^d \to [0, 1]$ that essentially scores how relevant the training point $(x_i, y_i)$ is to the example being classified. The functions $w_i^S$ are allowed to depend on $x_1, \dots, x_n$ but must be independent of the labels $y_1, \dots, y_n$. Given these functions, a point $x$ is classified by just checking whether $\sum y_iw_i^S(x) \geq 0$ or not. If it is nonnegative, we output $+1$ and otherwise $-1$. A complete description of weight functions is included in the appendix. 

Next, we enumerate several common Non-parametric classifiers that can be construed as weight functions. Details can be found in the appendix.

\textbf{Histogram classifiers} partition the domain $\R^d$ into cells recursively by splitting cells that contain a sufficiently large number of points $x_i$. This corresponds to a weight function in which $w_i^S(x) = \frac{1}{k_x}$ if $x_i$ is in the same cell as $x$, where $k_x$ denotes the number of points in the cell containing $x$.

$k_n$-\textbf{nearest neighbors} corresponds to a weight function in which $w_i^S(x) = \frac{1}{k_n}$ if $x_i$ is one of the $k_n$ nearest neighbors of $x$, and $w_i^S(x) = 0$ otherwise.

\textbf{Kernel-Similarity classifiers} are weight functions built from a kernel function $K:\R_{\geq 0} \to \R_{\geq 0}$ and a window size $(h_n)_1^\infty$ such that $w_i^S(x) \propto  K(\d(x, x_i)/h_n)$ (we normalize by dividing by $\sum_1^n K((\d(x, x_i)/h_n))$).
\section{The \Natural\emph{ }Bayes optimal classifier}

Robust classification is typically studied by setting the robustness regions,  $\mathcal{U} = \{U_x\}_{x \in \R^d}$, to be balls of radius $r$ centered at $x$, $U_x = \{x': \d(x, x') \leq r\}$. The quantity $r$ is the robustness radius, and is typically set by the practitioner (before any training has occurred). 

This method has a limitation with regards to trade-offs between accuracy and robustness. To increase the margin or robustness, we must have a large robustness radius (thus allowing us to defend from larger adversarial attacks). However, with large robustness radii, this can come at a cost of accuracy, as it is not possible to robustly give different labels to points with intersecting robustness regions. 

For an illustration, consider Figure \ref{fig:intro}. Here we consider a data distribution $D = (\mu, \eta)$ in which the blue regions denote all points with $\eta(x) > 0.5$ (and thus should be labeled $+$), and the red regions denote all points with $\eta(x) < 0.5$ (and thus should be labeled $-$). Observe that it is not possible to be simultaneously accurate and robust at points $A, B$ while enforcing a large robustness radius, as demonstrated by the intersecting balls. While this can be resolved by using a smaller radius, this results in losing out on potential robustness at point $C$. In principal, we should be able to afford a large margin of robustness about $C$ due to its relatively far distance from the red regions. 

Motivated by this issue, we seek to find a formalism for robustness that allows us to simultaneously avoid paying for any accuracy-robustness trade-offs and \textit{adaptively} size robustness regions (thus allowing us to defend against a larger range of adversarial attacks at points that are located in more homogenous zones of the distribution support). To approach this, we will first provide an ideal limit object: a classifier that has the same accuracy as the Bayes optimal (thus meeting our first criteria) that has good robustness properties. We call this the the \natural\emph{ }Bayes optimal classifier, defined as follows.

\begin{defn}\label{defn:includes_mu_plus_and_minus}
Let $\D = (\mu, \eta)$ be a distribution over $\R^d \times \{\pm 1\}$. Then the \textbf{\natural\emph{ }Bayes optimal classifier of $\D$}, denoted $\nat$, is the classifier defined as follows. Let $\mu^+ = \{x: \eta(x) \geq \frac{1}{2}\}$ and $\mu^- = \{x: \eta(x) < \frac{1}{2}\}$. Then for any $x \in \R^d$, $\nat(x) = +1$ if $\d(x, \mu^+) \leq \d(x, \mu^-)$, and $\nat(x) = -1$ otherwise. 
\end{defn}

This classifier can be thought of as the most robust classifier that matches the accuracy of the Bayes optimal. We call it \textit{\natural} because it extends the Bayes optimal classifier into a local neighborhood about every point in the support. For an illustration, refer to Figure \ref{fig:decision_boundary2}, which plots the decision boundary of the \natural\emph{ }Bayes optimal for an example distribution. 

\begin{figure}
    \centering
        \includegraphics[scale=0.30] {decision_boundary.png}
    \caption{The decision boundary of the \natural\emph{ }Bayes optimal classifier is shown in green,  and the \natural\emph{ }robust region of $x$ is shown in pink. The former consists of points equidistant from $\mu^+, \mu^-$, and the latter consists of points equidistant from $x$, $\mu^+$.}
    \label{fig:decision_boundary2}
\end{figure}

Next, we turn our attention towards measuring its robustness, which must be done with respect to some set of robustness regions $\mathcal{U} = \{U_x\}$. While these regions $U_x$ can be nearly arbitrary, we seek regions $U_x$ such that $A_\U(g_{max}, \D) = A(g_{bayes}, \D)$ (our astuteness equals the maximum possible accuracy) and $U_x$ are ``as large as possible" (representing large robustness). To this end, we propose the following regions.

\begin{defn}\label{def:nat_region}
Let $\D = (\mu, \eta)$ be a data distribution over $\R^d \times \{\pm 1\}$. Let $\mu^+ = \{x: \eta(x) > \frac{1}{2}\}$, $\mu^- = \{x: \eta(x) < \frac{1}{2}\}$, and $\mu^{1/2} = \{x: \eta(x) = \frac{1}{2}\}$. For $x \in \mu^+$, we define the \textbf{\natural\emph{ }robustness region}, denoted $V_x$, as $$V_x = \{x': \rho(x, x') < \rho(\mu^- \cup \mu^{\frac{1}{2}}, x')\}.$$ It consists of all points that are closer to $x$ than they are to $\mu^- \cup \mu^{1/2}$ (points oppositely labeled from $x$). We can use a similar definition for $x \in \mu^{-}$. Finally, if $x \in \mu^{1/2}$, we simply set $V_x = \{x\}$. 
\end{defn}

These robustness regions take advantage of the structure of the \natural\emph{ }Bayes optimal. They can essentially be thought of as regions that maximally extend from any point $x$ in the support of $\D$ to the decision boundary of the \natural\emph{ }Bayes optimal. We include an illustration of the regions $V_x$ for an example distribution in Figure \ref{fig:decision_boundary2}. 

As a technical note, for $x \in supp(\D)$ with $\eta(x) = 0.5$, we give them a trivial robustness region. The rational for doing this is that $\eta(x) = 0.5$ is an edge case that is arbitrary to classify, and consequently enforcing a robustness region at that point is arbitrary and difficult to enforce. 

We now formalize the robustness and accuracy guarantees of the max-margin Bayes optimal classifier with the following two results.

\begin{thm}\label{thm:accuracy_margin}
(Accuracy) Let $\D$ be a data distribution. Let $\V$ denote the collection of \natural\emph{ }robustness regions, and let $g$ denote the Bayes optimal classifier. Then the \natural\emph{ }Bayes optimal classifier, $\nat$, satisfies $A_\V(\nat, \D) = A(g, \D)$, where $A(g, \D)$ denotes the accuracy of the Bayes optimal. Thus, $\nat$ maximizes accuracy.
\end{thm}

\begin{thm}\label{thm:robust_margin}
(Robustness) Let $\D$ be a data distribution, let $f$ be a classifier, and let $\U$ be a set of robustness regions. Suppose that $A_\U(f, \D) = A(g, \D)$, where $g$ denotes the Bayes optimal classifier. Then there exists $x \in \supp(\D)$ such that $V_x \not \subset U_x$, where $V_x$ denotes the \natural\emph{ }robustness region about $x$. In particular, we cannot have $V_x$ be a strict subset of $U_x$ for all $x$. 
\end{thm}

Theorem \ref{thm:accuracy_margin} shows that the \natural\emph{ }Bayes classifier achieves maximal accuracy, while Theorem \ref{thm:robust_margin} shows that achieving a strictly higher robustness (while maintaining accuracy) is not possible; while it is possible to make accurate classifiers which have higher robustness than $\nat$ in some regions of space, it is not possible for this to hold across all regions. Thus, the \natural\emph{ }Bayes optimal classifier can be thought of as a local maximum to the constrained optimization problem of maximizing robustness subject to having maximum (equal to the Bayes optimal) accuracy.

\subsection{\Ncons\emph{ }Consistency}

Having defined the \natural\emph{ }Bayes optimal classifier, we now turn our attention towards building classifiers that converge towards it. Before doing this, we must precisely define what it means to converge. Intuitively, this consists of building classifiers whose robustness regions ``approach" the robustness regions of the \natural\emph{ }Bayes optimal classifier. This motivates the definition of \textit{partial \natural\emph{ }robustness regions}.
\begin{defn}\label{def:partial_nat_region}
Let $0 < \kappa < 1$ be a real number, and let $\D = (\mu, \eta)$ be a data distribution over $\R^d \times \{\pm 1\}$. Let $\mu^+ = \{x: \eta(x) > \frac{1}{2}\}$, $\mu^- = \{x: \eta(x) < \frac{1}{2}\}$, and $\mu^{1/2} = \{x: \eta(x) = \frac{1}{2}\}$. For $x \in \mu^+$, we define the \textbf{\natural\emph{ }robustness region}, denoted $V_x$, as $$V_x = \{x': \rho(x, x') < \kappa\rho(\mu^- \cup \mu^{\frac{1}{2}}, x')\}.$$ It consists of all points that are closer to $x$ than they are to $\mu^- \cup \mu^{1/2}$ (points oppositely labeled from $x$) by a factor of $\kappa$. We can use a similar definition for $x \in \mu^{-}$. Finally, if $\eta(x) = \frac{1}{2}$, we simply set $V_x^\kappa = \{x\}$.
\end{defn}

Observe that $V_x^{\kappa} \subset V_x$ for all $0 < \kappa < 1$, and thus being robust with respect to $V_x^{\kappa}$ is a milder condition than $V_x$. Using this notion, we can now define margin consistency.

\begin{defn}\label{definition:neighborhood_consistent}
A learning algorithm $A$ is said to be \textbf{\ncons\emph{ }consistent} if the following holds for any data distribution $\D = (\mu, \eta)$ where $\eta$ is continuous on its support. For any $0 < \epsilon, \delta, \kappa < 1$, there exists $N$ such that for all $n \geq N$, with probability at least $1- \delta$ over $S \sim \D^n$, $$A_{\V^\kappa}(A_S, D) \geq A(g, \D) - \epsilon,$$  where $g$ denotes the Bayes optimal classifier and $A_S$ denotes the classifier learned by algorithm $A$ from dataset $S$. 
\end{defn}

This condition essentially says that the astuteness of the classifier learned by the algorithm converges towards the accuracy of the Bayes optimal classifier. Furthermore, we stipulate that this holds as long as the astuteness is measured with respect to some $\V^\kappa$. Observe that as $\kappa \to 1$, these regions converge towards the \natural\emph{ }robustness regions, thus giving us a classifier with robustness effectively equal to that of the \natural\emph{ }Bayes optimal classifier. 

\section{\Ncons\emph{ }Consistent Non-Parametric Classifiers}

Having defined \ncons\emph{ }consistency, we turn to the following question: which non-parametric algorithms are \ncons\emph{ }consistent? Our starting point will be the standard literature for the convergence of non-parametric classifiers with regard to accuracy. We begin by considering the standard conditions for $k_n$-nearest neighbors to converge (in accuracy) towards the Bayes optimal.

$k_n$-nearest neighbors is \textit{consistent} if and only if the following two conditions are met:  $\lim_{n \to \infty} k_n = \infty$, and $\lim_{n \to \infty} \frac{k_n}{n} = 0$. The first condition guarantees that each point is classified by using an increasing number of nearest neighbors (thus making the probability of a misclassification small), and the second condition guarantees that each point is classified using only points very close to it. We will refer to the first condition as \textit{precision}, and the second condition as \textit{locality.}  A natural question is whether the same principles suffice for \ncons\emph{ }consistency as well. We began by showing that without any additional constraints, the answer is no.

\begin{thm}\label{thm:lower_bound}
Let $\D = (\mu, \eta)$ be the data distribution where $\mu$ denotes the uniform distribution over $[0,1]$ and $\eta$ is defined as: $\eta(x) = x$. Over this space, let $\d$ be the euclidean distance metric. Suppose $k_n = O(\log n)$ for $1 \leq n < \infty$. Then $k_n$-nearest neighbors is not \ncons\emph{ }consistent with respect to $\D$. 
\end{thm}

The issue in the example above is that for smaller $k_n$, $k_n$-nearest neighbors lacks sufficient precision. For \ncons\emph{ }consistnecy, points must be labeled using even more training points than are needed accuracy. This is because the classifier must be uniformly correct across the entirety of $V_x^\kappa$. Thus, to build \ncons\emph{ }consistent classifiers, we must bolster the precision from the standard amount used for standard consistency. To do this, we begin by introducing \textit{splitting numbers}, a useful tool for bolstering the precision of weight functions.

\subsection{Splitting Numbers}

We will now generalize beyond nearest neighbors to consider weight functions. Doing so will allow us to simultaneously analyze nearest neighbors and kernel classifiers. To do so, we must first rigorously substantiate our intuitions about increasing precision into concrete requirements. This will require several technical definitions.

\begin{defn}\label{defn:prob_radius}
Let $\mu$ be a probability measure over $\R^d$. For any $x \in \R^d$, the \textbf{probability radius} $r_p(x)$ is the smallest radius for which $B(x, r_p(x))$ has probability mass at least $p$. More precisely, $r_p(x) = \inf\{r: \mu(B(x,r)) \geq p\}.$ 
\end{defn}

\begin{defn}\label{defn:splitting_number}
Let $W$ be a weight function and let $S = \{x_1, x_2, \dots, x_n\}$ be any finite subset of $\R^d$. For any $x \in \R^d$, $\alpha \geq 0$, and $0 \leq \beta \leq 1$, let $W_{x, \alpha, \beta}  = \{i: \d(x, x_i) \leq \alpha, w_i^S(x) \geq \beta\}.$ Then the \textbf{splitting number} of $W$ with respect to $S$, denoted as $T(W, S)$ is the number of distinct subsets generated by $W_{x, \alpha \beta}$ as $x$ ranges over $\R^d$, $\alpha$ ranges over $[0, \infty)$, and $\beta$ ranges over $[0,1]$. Thus $T(W, S) = |\{W_{x, \alpha, \beta}: x \in \R^d, 0 \leq \alpha, 0 \leq \beta \leq 1\}|.$
\end{defn}

Splitting numbers allow us to ensure high amounts of precision over a weight function. To prove \ncons\emph{ }consistency, it is necessary for a classifier to be correct at \textit{all} points in a given region. Consequently, techniques that consider a single point will be insufficient. The splitting number provides a mechanism for studying entire regions simultaneously. For more details on splitting numbers, we include several examples in the appendix.

\subsection{Sufficient Conditions for Neighborhood Consistency}

We now state our main result.
\begin{thm}\label{thm:main}
Let $W$ be a weight function, $\D$ a distribution over $\R^d \times \{\pm 1\}$, $\U$ a neighborhood preserving collection, and $(t_n)_1^{\infty}$ be a sequence of positive integers such that the following four conditions hold. 

1. $W$ is consistent (with resp. to accuracy) with resp. to $\D$.

2. For any $0 < p < 1$, $\lim_{n \to \infty} E_{S \sim \D^n} [\sup_{x \in \R^d} \sum_1^n w_i^S(x)1_{\d(x, x_i) > r_p(x)}] = 0.$

3. $\lim_{n \to \infty} E_{S \sim D^n}[t_n \sup_{x \in \R^d} w_i^S(x)] = 0$.

4. $\lim_{n \to \infty} E_{S \sim D^n}\frac{\log T(W,S)}{t_n} = 0$.

Then $W$ is \ncons\emph{ }consistent with respect to $\D$.
\end{thm}

\textbf{Remarks:} Condition 1 is necessary because \ncons\emph{ }consistency implies standard consistency -- or, convergence in accuracy to the Bayes Optimal. Standard consistency has been well studied for non-parametric classifiers, and there are a variety of results that can be used to ensure it -- for example, Stone's Theorem (included in the appendix). 

Conditions 2. and 3. are stronger version of conditions 2. and 3. of Stone's theorem. In particular, both include a supremum taken over all $x \in \R^d$ as opposed to simply considering a random point $x \sim \D$. This is necessary for ensuring correct labels on entire regions of points simultaneously. We also note that the dependence on $r_p(x)$ (as opposed to some fixed $r$) is a key property used for adaptive robustness. This allows the algorithm to adjust to potential differing distance scales over different regions in $\R^d$. This idea is reminiscent of the analysis given in \cite{Dasgupta14}, which also considers probability radii.

Condition 4. is an entirely new condition which allows us to simultaneously consider all $T(W,S)$ subsets of $S$. This is needed for analyzing weighted sums with arbitrary weights.

Next, we apply Theorem \ref{thm:main} to get specific examples of margin consistent non-parametric algorithms.

\subsection{Nearest Neighbors and Kernel Classifiers}

We now provide sufficient conditions for $k_n$-nearest neighbors to be  \ncons\emph{ }consistent.

\begin{cor}\label{cor:nn}
Suppose $(k_n)_1^{\infty}$ satisfies (1) $\lim_{n \to \infty} \frac{k_n}{n} = 0$, and (2) $\lim_{n \to \infty} \frac{\log n}{k_n} = 0$. Then $k_n$-nearest neighbors is \ncons\emph{ }consistent.
\end{cor}

As a result of Theorem \ref{thm:lower_bound}, corollary \ref{cor:nn} is tight for nearest neighbors. Thus $k_n$ nearest neighbors is \ncons\emph{ }consistent if and only if $k_n = \omega(\log n)$. 

Next, we give sufficient conditions for a kernel-similarity classifier.
\begin{cor}\label{cor:kern}
Let $W$ be a kernel classifier over $\R^d \times \Y$ constructed from $K: \R^+ \to \R^+$ and $h_n$. Suppose the following properties hold.

1. $K$ is decreasing, and satisfies $\int_{\R^d}K(||x||)dx < \infty.$

2. $\lim_{n \to \infty} h_n = 0$ and $\lim_{n \to \infty} nh_n^d = \infty$.

3. For any $c > 1$, $\lim_{x \to \infty} \frac{K(cx)}{K(x)} = 0$.

4. For any $x \geq 0$, $\lim_{n \to \infty} \frac{n}{\log n}K(\frac{x}{h_n}) = \infty$.

Then $W$ is \ncons\emph{ }consistent.
\end{cor}

Observe that conditions 1. 2. and 3. are satisfied by many common Kernel functions such as the Gaussian or Exponential kernel ($K(x) = \exp(-x^2)$/ $K(x) = \exp(-x)$). Condition 4. can be similarly satisfied by just increasing $h_n$ to be sufficiently large. Overall, this theorem states that Kernel classification is \ncons\emph{ }consistent as long as the bandwidth shrinks slowly enough.

\begin{figure}
    \centering
        \includegraphics[scale=0.25] {histogram_pic.png}
    \caption{we have a histogram classifier being applied to the blue and red regions. The classifier will be unable to construct good labels in the cells labeled $A, B, C$, and consequently will not be robust with respect to $V_x^{\kappa}$ for sufficiently large $\kappa$.}
    \label{fig:histogram1}
\end{figure}

\subsection{Histogram Classifiers}

Having discussed \ncons\emph{ }consistent nearest-neighbors and kernel classifier, we now turn our attention towards another popular weight function, histogram classifiers. Recall that histogram classifiers operate by partitioning their input space into increasingly small cells, and then classifying each cell by using a majority vote from the training examples within that cell (a detailed description can be found in the appendix). We seek to answer the following question: is increasing precision sufficient for making histogram classifiers \ncons\emph{ }consistent? Unfortunately, the answer this turns out not to be no. The main issue is that histogram classifiers have no mechanism for performing classification outside the support of the data distribution. 

For an example of this, refer to Figure \ref{fig:histogram1}. Here we see a distribution being classified by a histogram classifier. Observe that the cell labeled $A$ contains points that are strictly closer to $\mu^+$ than $\mu^-$, and consequently, for sufficiently large $\kappa$, $V_x^{\kappa}$ will intersect $A$ for some point $x \in \mu^+$. A similar argument holds for the cells labeled $B$ and $C.$. However, since $A, B, C$ are all in cells that will never contain any data, they will never be labeled in a meaningful way. Because of this, histogram classifiers are not \ncons\emph{ }consistent.

\section{Validation}

%\begin{figure}[ht]
%	\centering
%	\subfloat[exponential kernel]{\includegraphics[width=.4\textwidth]{exponential.png}}
%	\subfloat[polynomial kernel]{\includegraphics[width=.4\textwidth]{polynomial.png}}
%	\caption{Plots of astuteness against the training sample size. In both panels, accuracy is plotted in red, and the varying levels of robustness regions $(\kappa = 0.1, 0.3, 0.5)$ are givne in blue, green and purple. In panel (a), observe that as sample size increases, every measure of astuteness converges towards $0.8$ which is as predicted by Corollary \ref{cor:kern}. In panel (b), although the accuracy appears to converge, none of the robustness measure. In fact, they get progressively worse the larger $\kappa$ gets. 
%	}
%	\label{fig:validation}
%\end{figure}

\begin{figure}
\begin{subfigure}{0.45\textwidth}
\includegraphics[width=\linewidth]{exponential.png}
\caption{exponential kernel} 
\end{subfigure}\hspace*{\fill}
\begin{subfigure}{0.45\textwidth}
\includegraphics[width=\linewidth]{polynomial.png}
\caption{polynomial kernel} 
\end{subfigure}

\caption{Plots of astuteness against the training sample size. In both panels, accuracy is plotted in red, and the varying levels of robustness regions $(\kappa = 0.1, 0.3, 0.5)$ are givne in blue, green and purple. In panel (a), observe that as sample size increases, every measure of astuteness converges towards $0.8$ which is as predicted by Corollary \ref{cor:kern}. In panel (b), although the accuracy appears to converge, none of the robustness measure. In fact, they get progressively worse the larger $\kappa$ gets.}
	\label{fig:validation}
\end{figure}

To complement our theoretical large sample results for non-parametric classifiers, we now include several experiments to understand their behavior for finite samples. We seek to understand how quickly non-parametic classifiers converge towards the \natural\emph{ }Bayes optimal. 

We focus our attention on kernel classifiers and use two different kernel similarity functions: the first, an exponential kernel, and the second, a polynomial kernel.  These classifiers were chosen so that the former meets the conditions of Corollary \ref{cor:kern}, and the latter does not. Full details on these classifiers can be found in the appendix.

To be able to measure performance with increasing data size, we look at a simple synthetic dataset over overlayed circles (see Figure \ref{fig:distribution} for an illustration) with support designed so that the data is intrinsically multiscaled. In particular, this calls for different levels of robustness in different regions. For simplicity, we use a global label noise parameter of $0.2$, meaning that any sample drawn from this distribution is labeled differently than its support with probability $0.2$. Further details about our dataset are given in section \ref{sec:experiment_details}. 

\textbf{Performance Measure.} For a given classifier, we evaluate its astuteness at a test point $x$ with respect to the robustness region $V_x^{\kappa}$ (Definition \ref{def:partial_nat_region}). While these regions are not computable in practice due to their dependency on the  support of the data distribution, we are able to approximate them for this synthetic example due to our explicit knowledge of the data distribution. Details for doing this can be found in the appendix. To compute the empirical astuteness of a kernel classifier $W_K$ about test point $x$, we perform a grid search over all points in $V_x^{\kappa}$ to ensure that all points in the robustness region are labeled correctly.  

For each classifier, we measure the empirical astuteness by using three trials of $20$ test points and taking the average. While this is a relatively small amount of test data, it suffices as our purpose is to just verify that the algorithm roughly converges towards the optimal possible astuteness. Recall that for any \ncons\emph{ }consistent algorithm, as $n \to \infty$, $A_{\mathcal{V}^\kappa}$ should converge towards $A^*$, the accuracy of the Bayes optimal classifier, for \textit{any} $0 < \kappa < 1$. Thus, to verify this holds, we use $\kappa = 0.1, 0.3, 0.5$. For each of these values, we plot the empirical astuteness as the training sample size $n$ gets larger and larger. As a baseline, we also plot their standard accuracy on the test set. 

\textbf{Results and Discussion:} The results are presented in Figure~\ref{fig:validation}; the left panel is for the exponential kernel, while the right one is for the polynomial kernel. As predicted by our theory, we see that in all cases, the exponential kernel converges towards the maximum astuteness regardless of the value of $\kappa$: the only difference is that the rate of convergence is slower for larger values of $\kappa$. This is, of course, expected because larger values of $\kappa$ entail larger robustness regions. 

By contrast, the polynomial kernel performs progressively worse for larger values of $\kappa$. This kernel was selected specifically to violate the conditions of Corollary \ref{cor:kern}, and in particular fails criteria 3. However, note that the polynomial kernel nevertheless performs will with respect to accuracy thus giving another example demonstrating the added difficulty of \ncons\emph{ }consistency.

Our results bridge the gap between our asymptotic theoretical results and finite sample regimes. In particular, we see that kernel classifiers that meet the conditions of Corollary \ref{cor:kern} are able to converge in astuteness towards the \natural\emph{ }Bayes optimal classifier, while classifiers that do not meet these conditions fail.

\section{Related Work}\label{sec:rel_work}

There is a wealth of literature on robust classification, most of which impose the same robustness radius $r$ on the entire data.  \cite{Carlini17, Liu17, Papernot17, Papernot16,Szegedy14, Hein17,Katz17,Schmidt18,Wu16,Steinhardt18, Sinha18}, among others, focus primarily on neural networks, and robustness regions that are $\ell_1, \ell_2, $ or $\ell_\infty$ norm balls of a given radius $r$. 

\cite{Cheng20} and~\cite{Ding20} show how to train neural networks with different robustness radii at different points by trading off robustness and accuracy; their work differ from ours in that they focus on neural networks, their robustness regions are still norm balls, and that their work is largely empirical.

Our framework is also related to large margin classification -- in the sense that the robustness regions $\U$ induce a {\em{margin constraint}} on the decision boundary. The most popular large margin classifier is the Support Vector Machine\cite{cortes95, Bennett00, Freund99} -- a large margin linear classifier that minimizes the worst-case margin over the training data. Similar ideas have also been used to design classifiers that are more flexible than linear; for example, \cite{Luxburg03} shows how to build large margin Lipschitz classifiers by rounding globally Lipschitz functions. Finally, there has also been purely empirical work on achieving large margins for more complex classifiers -- such as~\cite{Samy18} for deep neural networks that minimizes the worst case margin, and~\cite{Weinberger05} for metric learning to find large margin nearest neighbors. Our work differs from these in that our goal is to ensure a high enough local margin at each $x$, (by considering the \natural\emph{ }regions $V_x$) as opposed to optimizing a global margin. 

%However, in all of these cases, the goal is substantially different from building astute classifiers. Broadly speaking, this area is centered around minimizing \textit{global} margin, whereas our work can be thought of us optimizing local margin, since the regions $\U_x$ vary considerably in size based on $x$. 


%The relationship here is that building a robust classifier for a neighborhood preserving collection $\U$, bears some resemblance to building a large margin classifier.. In particular, the conditions on $\U$ indicates that the decision boundary for the data distribution, $\D$, should ideally be fairly far away from the support. 





%Nearly all of these works consider robustness regions $\U_x$ that are balls of radius $r$, typically using the $\ell_1, \ell_2, $ and $\ell_\infty$ norms. 

Finally, our analysis builds on prior work on robust classification for non-parametric methods in the standard framework. \cite{Amsaleg17, Sitawarin19, WJC18, YRWC19} provide adversarial attacks on non-parametric methods. Wang et. al. \cite{WJC18} develops a defense for $1$-NN that removes a subset of the training set to ensure higher robustness. Yang et. al~\cite{YRWC19} proposes the $r$-optimal classifier -- which is the maximally astute classifier in the standard robustness framework -- and proposes a defense called Adversarial Pruning. 


%that relies on \textbf{adversarial pruning}, which is process in which points are removed from the training set so that any two elements with opposite labels have distance at least $2r$, where $r$ is the robustness radius.  They show that their algorithm converges towards the same robustness as the Bayes optimal classifier. This was subsequently improved upon in Yang et. al. \cite{YRWC19} which proposes the $r$-optimal classifier, which is the maximally robust classifier. 

Theoretically, \cite{Bhattacharjee20} provide conditions under which weight functions converge towards the $r$-optimal classifier in the large sample limit. They show that for $r$-separated distributions, where points from different classes are at least distance $2r$ or more apart, nearest neighbors and kernel classifiers satisfy these conditions. In the more general case, they use Adversarial Pruning as a preprocessing step to ensure that the training data is $r$-separated, and show that this preprocessing step followed by nearest neighbors or kernel classifiers leads to solutions that are robust and accurate in the large sample limit. Our result fundamentally differs from theirs in that we analyze a different algorithm, and our proof techniques are quite different. In particular, the fundamental differences between the $r$-optimal classifier and the \natural\emph{ }Bayes optimal classifier call for different algorithms and different analysis techniques.

In concurrent work, \cite{Chowdhury21} proposes a similar limit to the neighborhood preserving Bayes optimal which they refer to as the margin canonical Bayes. However, their work then focuses on a data augmentation technique that leads to convergence whereas we focus on proving the neighborhood consistency of classical non-parametric classifiers.



%\newcommand{\crop}[1]{\mathrm{crop}({#1})}
\newcommand{\object}[1]{\mathrm{object}({#1})}
\newcommand{\ba}{A_i}
\newcommand{\bb}{B_i}
\newcommand{\calA}{\mathcal{A}}
\newcommand{\calB}{\mathcal{B}}
\newcommand{\calX}{\mathcal{X}}
\newcommand{\masked}[1]{\mathrm{masked}({#1})}
\newcommand{\bx}{\mathbf{x}}
\newcommand{\SSL}{\textsc{SSL}}
\newcommand{\SSLbb}{\SSL^\mathrm{back}}
\newcommand{\SSLpj}{\SSL^\mathrm{proj}}
\newcommand{\CLF}{\textsc{CLF}}
\newcommand{\CLFbb}{\CLF^\mathrm{back}}
\newcommand{\CLFpj}{\CLF^\mathrm{proj}}
\newcommand{\SUP}{\textsc{SUP}}
\newcommand{\KNN}{\textsc{KNN}}
\newcommand{\KNNset}{\textsc{KNN}^\mathrm{set}}
\newcommand{\KNNprob}{\textsc{KNN}^\mathrm{prob}}
\newcommand{\KNNcl}{\textsc{KNN}^\mathrm{cl}}
\newcommand{\KNNconf}{\textsc{KNN}^\mathrm{conf}}
\newcommand{\RCDM}{\textsc{RCDM}}
\newcommand{\cl}{\mathrm{cl}}
\newcommand{\clpred}{\tilde{\mathrm{cl}}}
\newcommand{\Abox}{\overline{\calA}}
\newcommand{\Bbox}{\overline{\calB}}
\newcommand{\dejavu}{\emph{déjà vu }}
\newcommand{\Dejavu}{\emph{Déjà vu }}

\newcommand{\citations}{{\color{green}[CITE]}}

\definecolor{part_blue}{rgb}{0.2824, 0.4706, .8157}
\definecolor{part_red}{rgb}{0.8392, 0.3725, 0.3725}
\definecolor{part_orange}{rgb}{0.9333, 0.5216, 0.2902}

\DeclareRobustCommand{\mybox}[2][gray!20]{%
\begin{tcolorbox}[   %% Adjust the following parameters at will.
        % breakable,
        left=0pt,
        right=0pt,
        top=0pt,
        bottom=0pt,
        colback=#1,
        colframe=#1,
        width=\dimexpr\columnwidth\relax, 
        % width=\textwidth, 
        enlarge left by=0mm,
        boxsep=5pt,
        arc=0pt,outer arc=0pt,
        ]
        #2
\end{tcolorbox}
}
%\section{Introduction}
\label{sec:intro}
Self-supervised learning (SSL)~\citep{chen2020simclr, chen2020simsiam, zbontar2021barlow, vicreg, caron2020swav, MAE} aims to learn general representations of content-rich data without explicit labels by solving a \textit{pretext task}. In many recent works, such pretext tasks rely on joint-embedding architectures whereby randomized image augmentations are applied to create multiple views of a training sample, and the model is trained to produce similar representations for those views. When using cropping as random image augmentation, the model learns to associate objects or parts (including the background scenery) that co-occur in an image.
However, doing so also arguably exposes the training data to higher privacy risk as objects in training images can be explicitly memorized by the SSL model. For example, if the training data contains the photos of individuals, the SSL model may learn to associate the face of a person with their activity or physical location in the photo. This may allow an adversary to extract such information from the trained model for targeted individuals.

\begin{figure}[t]
    \centering
    \includegraphics[width=1.0\columnwidth]{figures/new_black_swan.pdf}
    \caption{\textbf{Left:} Reconstruction of an SSL training image from a crop containing only the background. The SSL model memorizes the association of this \emph{specific} patch of water (pink square) to this \emph{specific} foreground object (a black swan) in its embedding, which we decode to visualize the full training image. \textbf{Right:} The reconstruction technique fails on a public test image that the SSL model has not seen before.}
    \label{fig:black_swan}
\end{figure}

In this work, we aim to evaluate to what extent SSL models memorize the association of specific objects in training images or the association of objects and their specific backgrounds, and whether this memorization signal can be used to reconstruct the model's training samples. Our results demonstrate that SSL models memorize such associations beyond simple correlation. For instance, in Figure \ref{fig:black_swan} (\textbf{left}), we use the SSL representation of a \emph{training image crop containing only water} and this enables us to reconstruct the object in the foreground with remarkable specificity---in this case a black swan.
By contrast, in Figure \ref{fig:black_swan} (\textbf{right}), when using the \emph{crop from the background of a test set image} that the SSL model \emph{has not seen before}, its representation only contains enough information to infer, through correlation, that the foreground object was likely some kind of waterbird --- but not the specific one in the image.

Figure \ref{fig:black_swan} shows that SSL models suffer from the unintended memorization of images in their training data---a phenomenon we refer to as \emph{déjà vu memorization}
%\footnote{The French loanword \emph{déjà vu} means already-seen, which reflects the type of unintended memorization of objects that the SSL model saw during training.}.
\footnote{The French loanword \emph{déjà vu} means `already-seen', just as an image is seen and memorized in training.}
Beyond visualizing \emph{déjà vu} memorization through data reconstruction, we also design a series of experiments to quantify the degree of memorization for different SSL algorithms, model architectures, training set size, \emph{etc.} We observe that \emph{déjà vu} memorization is exacerbated by the atypically large number of training epochs often recommended in SSL training, as well as certain hyperparameters in the SSL training objective. Perhaps surprisingly, we show that \emph{déjà vu} memorization occurs even when the training set is large---as large as half of ImageNet~\citep{imagenet}---and can continually worsen even when standard techniques for evaluating learned representation quality (such as linear probing) do not suggest increased overfitting. Our work serves as the first systematic study of unintended memorization in SSL models and motivates future work on understanding and preventing this behavior. Specifically, we: 
\begin{itemize}
    \vspace{-0.5em}
    \item Elucidate how SSL representations memorize aspects of individual training images, what we call \emph{déjà vu} memorization;
    \item Design a novel training data reconstruction pipeline for non-generative vision models. This is in contrast to many prominent reconstruction algorithms like \citep{carlini2021extracting, google_diffusion}, which rely on the model itself to generate its own memorized samples and is not possible for SSL models or classifiers;
    \item Propose metrics to quantify the degree of \dejavu memorization committed by an SSL model. This allows us to observe how \dejavu changes with training epochs, dataset size, training criteria, model architecture and more. 
\end{itemize}

%\section{Preliminaries and Related Work}
\label{sec:related}

\textbf{Self-supervised learning} (SSL) is a machine learning paradigm that leverages unlabeled data to learn representations. Many SSL algorithms rely on \emph{joint-embedding} architectures (\emph{e.g.}, SimCLR~\citep{chen2020simclr}, Barlow Twins~\citep{zbontar2021barlow}, VICReg~\citep{vicreg} and Dino~\citep{Dino}), which are trained to associate different augmented views of a given image. For example, in SimCLR, given a set of images $\calA = \{A_1,\ldots,A_n\}$ and a randomized augmentation function $\mathrm{aug}$, the model is trained to maximize the cosine similarity of draws of $\SSL(\mathrm{aug}(A_i))$ with each other and minimize their similarity with $\SSL(\mathrm{aug}(A_j))$ for $i \neq j$. The augmentation function $\mathrm{aug}$ typically consists of operations such as cropping, horizontal flipping, and color transformations to create different views that preserve an image's semantic properties. 

\paragraph{SSL representations.} Once an SSL model is trained, its learned representation can be transferred to different downstream tasks. This is often done by extracting the representation of an image from the \emph{backbone model}\footnote{SSL methods often use a trick called \emph{guillotine regularization}~\citep{Guillotine}, which decomposes the model into two parts: a \emph{backbone model} and a \emph{projector} consisting of a few fully-connected layers. Such trick is needed to handle the misalignment between the pretext SSL task and the downstream task.} and either training a linear probe on top of this representation or finetuning the backbone model with a task-specific head~\citep{Guillotine}.
%Compared to representations learned by supervised learning, SSL representations are often more robust and transferable~\citep{hendrycks2019using, ericsson2021self}, leading to state-of-the-art result on many downstream tasks. To understand the effectiveness of SSL algorithms, several prior works investigated what kind of information the SSL model has learned~\citep{jing2021understanding, ericsson2021self, kalibhat2022towards, RCDM}. In particular, \citet{RCDM} trained a conditional generative model on SSL representations and showed that they encode richer visual details about the input image compared to supervised learning. 
%However, from a privacy perspective, this may be a cause for concern as the model also has more potential to overfit and memorize precise details about the training data compared to supervised learning. We show concretely that this privacy risk can indeed be realized by defining and measuring \emph{déjà vu} memorization.
It has been shown that SSL representations encode richer visual details about input images than supervised models do \cite{RCDM}. However, from a privacy perspective, this may be a cause for concern as the model also has more potential to overfit and memorize precise details about the training data compared to supervised learning. We show concretely that this privacy risk can indeed be realized by defining and measuring \emph{déjà vu} memorization.
\vspace{-0.5em} 
% \paragraph{Privacy risks in ML.} Overfitting in ML occurs when a model memorizes information specific to its training data rather than general population-level information. When the model is trained on privacy-sensitive data, overfitting is especially harmful as an adversary can infer private information about the training data when given access to the model~\citep{yeom2018privacy, feldman2020does}. The simplest and most well-studied form of privacy risk in ML is susceptibility to \emph{membership inference attacks}~\citep{shokri2017membership, salem2018ml, sablayrolles2019white}, where the adversary infers whether an individual is part of the training set or not. More sophisticated privacy attacks include \emph{attribute inference}~\citep{fredrikson2014privacy, mehnaz2022your, jayaraman2022attribute}, where specific attributes about an individual are inferred given others, and \emph{data reconstruction}~\citep{carlini2021extracting, balle2022reconstructing, guo2022bounding}, where entire training samples are recovered from the trained model. Our study of \emph{déjà vu} memorization is similar to both attribute inference and data reconstruction, leveraging SSL representations of the training image background to infer and reconstruct the foreground object.
% \vspace{-0.5em} 
% \paragraph{Training data extraction in NLP.} Our study of \dejavu memorization in SSL models is inspired by similar work in the natural language processing (NLP) domain. \citet{carlini2019secret} first showed that language models exhibit unintended memorization, where given a context string present in its training data, the model can generate the remaining text at test time. This unintended memorization has been further exploited in \citet{carlini2021extracting} to extract training data from GPT-2~\citep{radford2019language} and, more recently, extended to extract memorized images from Stable Diffusion \citep{google_diffusion}. The way by which these works exploit unintended memorization is similar to ours: given partial information about a training sample, the model is prompted to reveal the rest of the sample. In our case, however, since the SSL model is not generative, extraction is significantly harder and requires careful design.

\paragraph{Privacy risks in ML.} When a model is overfit on privacy-sensitive data, it memorizes specific information about its training examples, allowing an adversary with access to the model to learn private information~\citep{yeom2018privacy, feldman2020does}. Privacy attacks in ML range from the simplest and best-studied \emph{membership inference attacks}~\citep{shokri2017membership, salem2018ml, sablayrolles2019white} to \emph{attribute inference}~\citep{fredrikson2014privacy, mehnaz2022your, jayaraman2022attribute} and \emph{data reconstruction}~\citep{carlini2021extracting, balle2022reconstructing, guo2022bounding} attacks. In the former, the adversary only infers whether an individual participated in the training set. Our study of \emph{déjà vu} memorization is most similar to the latter: we leverage SSL representations of the training image background to infer and reconstruct the foreground object. Our approach reflects similar work in the NLP domain \citep{carlini2019secret, carlini2021extracting}: when prompted with a context string present in the training data, a large language model is shown to generate the remainder of string at test time, revealing sensitive text like home addresses. This method was recently extended to extract memorized images from Stable Diffusion \citep{google_diffusion}.  We exploit memorization in a similar manner: given partial information about a training sample, the model is prompted to reveal the rest of the sample. In our case, however, since the SSL model is not generative, extraction is significantly harder and requires careful design.

%\section{Defining \emph{Déjà Vu} Memorization}
\label{sec:definition}

\paragraph{What is \dejavu memorization?} At a high level, the objective of SSL is to learn general representations of objects that occur in nature. This is often accomplished by associating different parts of an image with one another in the learned embedding. Returning to our example in Figure \ref{fig:black_swan}, given an image whose background contains a patch of water, the model may learn that the foreground object is a water animal such as duck, pelican, otter, \emph{etc.}, by observing different images that contain water from the training set. We refer to this type of learning as \emph{correlation}: the association of objects that tend to co-occur in images from the training data distribution.

A natural question to ask is \emph{``Can the reconstruction of the black swan in Figure \ref{fig:black_swan} be reasoned as correlation?''} The intuitive answer may be no, since the reconstructed image is qualitatively very similar to the original image. However, this reasoning implicitly assumes that for a random image from the training data distribution containing a patch of water, the foreground object is unlikely to be a black swan. Mathematically, if we denote by $\mathcal{P}$ the training data distribution and $A$ the image, then
\begin{equation*}
\label{eq:p_corr}
p_\text{corr} := \mathbb{P}_{A \sim \mathcal{P}}(\mathrm{object}(A) = \texttt{black swan} ~|~ \mathrm{crop}(A) = \texttt{water})
\end{equation*}
is the probability of inferring that the foreground object is a black swan through \emph{correlation}. This probability may be naturally high due to biases in the distribution $\mathcal{P}$, \emph{e.g.}, if $\mathcal{P}$ contains no other water animal except for black swans. In fact, such correlations are often exploited to learn a model for image inpainting with great success~\citep{yu2018generative, ulyanov2018deep}.

Despite this, we argue that reconstruction of the black swan in Figure \ref{fig:black_swan} is \emph{not} due to correlation, but rather due to \emph{unintended memorization}: the association of objects unique to a single training image. As we will show in the following sections, the example in Figure \ref{fig:black_swan} is not a rare success case and can be replicated across many training samples. More importantly, failure to reconstruct the foreground object in Figure \ref{fig:black_swan} (\textbf{right}) on test images hints at inferring through correlation is unlikely to succeed---a fact that we verify quantitatively in Section \ref{sec:label inference accuracy}. Motivated by this discussion, we give a verbal definition of \dejavu memorization below, and design a testing methodology to quantify \dejavu memorization in Section \ref{sec:notation and setup}.
\mybox{\textbf{Definition:} A model exhibits \emph{déjà vu memorization} when it retains information so specific to an individual training image, that it enables recovery of aspects particular to that image given a part that does not contain them.
The recovered aspect must be beyond what can be inferred using only correlations in the data distribution.} 

% \textbf{Definition:} A model exhibits \emph{déjà vu memorization} when it retains information so specific to an individual training image, that it enables recovery of aspects particular to that image given a part that does not contain them.
% The recovered aspect must be beyond what can be inferred using only correlations in the data distribution.


 We intentionally kept the above definition broad enough to encompass different types of information that can be inferred about the training image, including but not restricted to object category, shape, color and position. For example, if one can infer that the foreground object is red given the background patch with accuracy significantly beyond correlation, we consider this an instance of \dejavu memorization as well. We mainly focus on object category to quantify \dejavu memorization in Section \ref{sec:quant} since the ground truth label can be easily obtained. We consider other types of information more qualitatively in the visual reconstruction experiments in Section \ref{sec:visualizing}.

\paragraph{Privacy implications of \dejavu memorization.} \Dejavu memorization can be a cause for concern when the training data contains privacy-sensitive information. As a motivating example, consider an SSL model trained on photos of individuals. If the model exhibits \dejavu memorization then, given the face of an individual, it may be possible to infer where the individual was or even visually reconstruct their location in the training image. Such information leakage raises privacy concerns, especially if there was no prior agreement that the trained model may reveal such information to third parties. This hypothetical scenario serves as a motivation that \dejavu memorization should be carefully examined to avoid unintended disclosure of private information in practical applications.

% \begin{figure*}[h]
%     \centering
%     \includegraphics[width = 0.85\textwidth]{figures/SSL_attack_cartoon.png}
%     \caption{We measure memorization by comparing the `target model' trained on the target image ($\SSL_A$ trained on $A_i$ in above example) with the `reference model' not trained on it ($\SSL_B$, above). \textbf{[Top Strip]} A cropping of the image disjoint from the labeled foreground object is embedded using the target model. This embedding is then labeled by a K-Nearest Neighbor (KNN) adversary built on a public set of labeled images, $X$, which it has also embedded using the target model. \textbf{[Bottom Strip]} To account for correlation, the same procedure is followed with the reference model. If the label is only extracted using the target model, it is counted as memorization. If it is extracted using either model, it is counted as correlation. We find that the KNN adversary's predictions using the target model (trained on attacked examples) are significantly more accurate than they are using the reference model, indicating routine memorization of training examples.}
%     \label{fig:ssl attack cartoon}
% \end{figure*}

\begin{figure}[t]
%%%
%SPIDER
%%%
     % \centering
     % \begin{subfigure}[b]{0.25\textwidth}
     %     \centering
     %     \includegraphics[width=\textwidth]{figures/data_split.png}
     %     % \caption{SimCLR correlated \textit{yellow garden spider} examples}
     %     \label{fig:data split}
     % \end{subfigure}
     % \hfill
     % \begin{subfigure}[b]{0.7\textwidth}
     %     \centering
     %     \includegraphics[width=\textwidth]{figures/pipeline_cartoon.png}
     %     \begin{minipage}{5cm}
     %        \vfill
     %    \end{minipage}
     %     % \caption{SimCLR memorized \textit{yellow garden spider} examples}
     %     \label{fig:pipeline cartoon}
     % \end{subfigure}
     \includegraphics[width=\textwidth]{figures/split_and_pipeline_cartoon.png}
\caption[Overview of testing methodology.]{
Overview of testing methodology. \textbf{Left:} Data is split into \emph{target set} $\calA$, \emph{reference set} $\calB$ and \emph{public set} $\calX$ that are pairwise disjoint. $\calA$ and $\calB$ are used to train two SSL models $\SSL_A$ and $\SSL_B$ in the same manner. $\calX$ is used for KNN decoding or for training an RCDM to reconstruct the input at test time. \textbf{Right:} Given a training image $A_i \in \calA$, we use $\SSL_A$ to embed $\crop{A_i}$ containing only the background, as well as the entire set $\calX$ and find the $k$-nearest neighbors of $\crop{A_i}$ in $\calX$ in the embedding space. These KNN samples can be used directly to infer the foreground object (\emph{i.e.}, class label) in $A_i$ using a KNN classifier, or their embeddings can be averaged as input to the trained RCDM to visually reconstruct the image $A_i$. For instance, the RCDM reconstruction results in Figure \ref{fig:black_swan} (left) when given $\SSL_A(\crop{A_i})$ and results in Figure \ref{fig:black_swan} (right) when given $\SSL_A(\crop{B_i})$ for an image $B_i \in \calB$.
%\textbf{Left:} illustration of the three datasets used in our tests. Two private data sets, $A$ and $B$, of equal size are used to train two SSL models, $\SSL_A$ and $\SSL_B$, respectively. A disjoint public set, $X$, is made available to the memorization test to help decode model embeddings. Memorization is only tested on examples $A_i \in A$ that are unique to set $A$. \textbf{Right:} illustration of inference pipeline used in tests. A periphery cropping that excludes the foreground object is taken from private image $A_i$. The KNN then finds the $k$ public set nearest neighbors of the periphery crop in the embedding space of $\SSL_A$. 
%The $\SSL_A$ representation of these $k$ neighbors and of the crop are used by the conditional generative model, RCDM, to reconstruct the foreground object. The labels of these $k$ neighbors are used to recover the foreground object label. (Not pictured) We repeat this process using reference model $\SSL_B$, not trained on image $A_i$, to determine whether the foreground object is still recoverable by learned correlations, e.g. if black swans were the only objects appearing near water in the data distribution. In this instance, the crop's public set neighbors in $\SSL_B$'s representation space include a variety of water animals like ducks, pelicans, and otters. Meanwhile, with $\SSL_A$, the neighbors are nearly all black swans in the same position as the swan of $A_i$.
}
\label{fig:split_and_pipeline_cartoon}
\end{figure}

\textbf{Distinguishing memorization from correlation.} When measuring \dejavu memorization, it is crucial to differentiate what the model associates through \emph{memorization} and what it associates through \emph{correlation}. Our testing methodology is based on the following intuitive definition.
\mybox{\textbf{Definition:} If an SSL model associates two parts in a training image, we say that it is due to \emph{correlation} if other SSL models trained on a similar dataset from $\mathcal{P}$ without this image would likely make the same association. Otherwise, we say that it is due to \emph{memorization}.}

Notably, such intuition forms the basis for differential privacy (DP; \cite{dwork2006calibrating, dwork2013algorithmic})---the most widely accepted notion of privacy in ML.

\subsection{Testing Methodology for Measuring \emph{Déjà Vu} Memorization}
\label{sec:notation and setup}

In this section, we use the above intuition to measure the extent of \dejavu memorization in SSL. Figure \ref{fig:split_and_pipeline_cartoon} gives an overview of our testing methodology.
\vspace{-0.75em}
\paragraph{Dataset splitting.} We focus on testing \dejavu memorization for SSL models trained on the ImageNet-1K dataset~\citep{imagenet}. Our test first splits the ImageNet training set into three independent and disjoint subsets $\calA$, $\calB$ and $\calX$. The dataset $\calA$ is called the \emph{target set} and $\calB$ is called the \emph{reference set}. The two datasets are used to train two separate SSL models, $\SSL_A$ and $\SSL_B$, called the \emph{target model} and the \emph{reference model}. Finally, the dataset set $\calX$ is used as an auxiliary public dataset to extract information from $\SSL_A$ and $\SSL_B$.
%\footnote{See Appendix \ref{sec:appx splits} for details on how the dataset splits are generated.}.
Our dataset splitting serves the purpose of distinguishing memorization from correlation in the following manner. Given a sample $A_i \in \calA$, if our test returns the same result on $\SSL_A$ and $\SSL_B$ then it is likely due to correlation because $A_i$ is not a training sample for $\SSL_B$. Otherwise, because $\calA$ and $\calB$ are drawn from the same underlying distribution, our test must have inferred some information unique to $A_i$ due to memorization. Thus, by comparing the difference in the test results for $\SSL_A$ and $\SSL_B$, we can measure the degree of \dejavu memorization\footnote{See Appendix \ref{sec:appx splits} for details on how the dataset splits are generated.}.
\vspace{-0.75em}
\paragraph{Extracting foreground and background crops.} Our testing methodology aims at measuring what can be inferred about the foreground object in an ImageNet sample given a background crop. This is made possible because ImageNet provides bounding box annotations for a subset of its training images---around 150K out of 1.3M samples. We split these annotated images equally between $\calA$ and $\calB$. Given an annotated image $A_i$, we treat everything inside the bounding box as the foreground object associated with the image label, denoted $\object{A_i}$. We take the largest possible crop that does not intersect with any bounding box as the background crop (or \emph{periphery crop}), denoted $\crop{A_i}$\footnote{We also present another heuristic in \cref{sec:appx corner crop} which takes a corner crop as the background crop, allowing our test to be run without bounding box annotations.}
%Since the labeled object tends to be at the image's center, the corner crop usually excludes it. }
%Because most images in ImageNet are object centric, an image's corner would not include the foreground object.}.
\vspace{-0.75em}
\paragraph{KNN-based test design.} Joint-embedding SSL approaches encourage the embeddings of random crops of a training image $A_i \in \calA$ to be similar. Intuitively, if the model exhibits \dejavu memorization, it is reasonable to expect that the embedding of $\crop{A_i}$ is similar to that of $\object{A_i}$ since both crops are from the same training image. In other words, $\SSL_A(\crop{A_i})$ encodes information about $\object{A_i}$ that cannot be inferred through correlation. However, decoding such information is challenging as these approaches do not learn a decoder associated with the encoder $\SSL_A$.

Here, we leverage the public set $\calX$ to decode the information contained in $\crop{A_i}$ about $\object{A_i}$. More specifically, we map images in $\calX$ to their embeddings using $\SSL_A$ and extract the $k$-nearest-neighbor (KNN) subset of $\SSL_A(\crop{A_i})$ in $\calX$. We can then decode the information contained in $\crop{A_i}$ in one of two ways:
\begin{itemize}
\item \emph{Label inference:} Since $\calX$ is a subset of ImageNet, each embedding in the KNN subset is associated with a class label. If $\crop{A_i}$ encodes information about the foreground object, its embedding will be close to samples in $\calX$ that have the same class label (\emph{i.e.}, foreground object category). We can then use a KNN classifier to infer the foreground object in $A_i$ given $\crop{A_i}$.
\item \emph{Visual reconstruction:} Following \citet{RCDM}, we train an RCDM---a conditional generative model---on $\calX$ to decode $\SSL_A$ embeddings into images. The RCDM reconstruction can recover qualitative aspects of an image remarkably well, such as recovering object color or spatial orientation using its SSL embedding. Given the KNN subset, we average their SSL embeddings and use the trained RCDM model to visually reconstruct $A_i$.
\end{itemize}
In Section \ref{sec:quant}, we focus on quantitatively measuring \dejavu memorization with label inference, and then use the RCDM reconstruction to visualize \dejavu memorization in Section \ref{sec:visualizing}.
%\section{Quantifying \emph{Déjà Vu} Memorization}
\label{sec:quant}

We apply our testing methodology to quantify a specific form of \dejavu memorization: inferring the foreground object (class label) given a crop of the background.

% \paragraph{Extracting model embeddings.} We test \dejavu memorization on two popular SSL algorithms, SimCLR~\citep{chen2020simclr} and VICReg~\citep{vicreg}.
% %\footnote{We present additional SSL models in \cref{sec:appx simclr results}} 
% As described in Section \ref{sec:related}, these algorithms produce two embeddings given an input image: a \emph{backbone} embedding and a \emph{projector} embedding that is derived by applying a small fully-connected network on top of the backbone embedding. Unless otherwise noted, all SSL embeddings refer to the projector embedding.
% To understand whether \dejavu memorization is particular to SSL, we also evaluate embeddings produced by a supervised model $\CLF_A$ trained on $\calA$. We apply the same set of image augmentations as those used in SSL and train $\CLF_A$ using the cross-entropy loss to predict ground truth labels. 
\vspace{-0.75em}
\paragraph{Extracting model embeddings.} We test \dejavu memorization on a variety of popular SSL algorithms, with a focus on VICReg~\citep{vicreg}. These algorithms produce two embeddings given an input image: a \emph{backbone} embedding and a \emph{projector} embedding that is derived by applying a small fully-connected network on top of the backbone embedding. Unless otherwise noted, all SSL embeddings refer to the projector embedding. 
To understand whether \dejavu memorization is particular to SSL, we also evaluate embeddings produced by a supervised model $\CLF_A$ trained on $\calA$. We apply the same set of image augmentations as those used in SSL and train $\CLF_A$ using the cross-entropy loss to predict ground truth labels. 
\vspace{-0.75em}
\paragraph{Identifying the most memorized samples.} Prior works have shown that certain training samples can be identified as more prone to memorization than others~\citep{feldman2020does, watson2021importance, ye2021enhanced}. Similarly, we provide a heuristic to identify the most memorized samples in our label inference test using confidence of the KNN prediction.
Given a periphery crop, $\crop{A_i}$, let $\KNN_A \big( \crop{A_i} \big) \subseteq \calX$ denote its $k$-nearest neighbors in the embedding space of $\SSL_A$. From this KNN subset we can obtain: \textbf{(1)} $\KNNprob_A \big( \crop{A_i} \big)$, the vector of class probabilities (normalized counts) induced by the KNN subset, and \textbf{(2)} $\KNNconf_A \big( \crop{A_i} \big)$, the negative entropy of the probability vector $\KNNprob_A \big( \crop{A_i} \big)$, as confidence of the KNN prediction. When entropy is low, the neighbors agree on the class of $A_i$ and hence confidence is high. 
% \begin{itemize}[noitemsep, leftmargin=*, topsep=0pt]
%     \item $\KNN_A \big( \crop{A_i} \big)$: The most prevalent class in the KNN subset as prediction for the class label $\cl(A_i)$. 
%     \item $\KNNprob_A \big( \crop{A_i} \big)$: The vector of class probabilities (normalized counts) induced by the KNN subset.
%     \item $\KNNconf_A \big( \crop{A_i} \big)$: Negative entropy of the probability vector $\KNNprob_A \big( \crop{A_i} \big)$ as confidence of the KNN prediction. When entropy is low, the neighbors agree on the class of $A_i$ and hence confidence is high. 
% \end{itemize}
We can sort the confidence score $\KNNconf_A \big( \crop{A_i} \big)$ across samples $A_i$ in decreasing order to identify the most confidently predicted samples, which likely correspond to the most memorized samples when $A_i \in \calA$.

\subsection{Population-level Memorization}
\label{sec:label inference accuracy}

%ORIGINAL FIGURE SETUP IN ARXIV: 
% \input{dejavu_training_epochs.tex}
% \input{dejavu_training_set_size.tex}
%PUT ORIGINAL FIGURES SIDE BY SIDE: 
% \input{dejavu_training_epochs_set_size.tex}
%PUT IN NEW FIGURES: 

\begin{wrapfigure}{r}{0.4\textwidth} 
    \centering
    \includegraphics[width=0.4\textwidth]{figures/dejavu_main.pdf}
    \caption{Accuracy of label inference using the target model (trained on $\calA$) vs. the reference model (trained on $\calB$) on the top $\%$ most confident examples $A_i \in \calA$ using only $\crop{A_i}$. For VICReg, there is a large accuracy gap between the two models, indicating a significant degree of \dejavu memorization.}
    \label{fig:dejavu main}
    \vspace{-2ex}
\end{wrapfigure}

Our first measure of \dejavu memorization is population-level label inference accuracy: \emph{What is the average label inference accuracy over a subset of SSL training images given their periphery crops?} 
To understand how much of this accuracy is due to $\SSL_A$'s \dejavu memorization, we compare with a correlation baseline using the reference model: $\KNN_B$'s label inference accuracy on images $A_i \in \calA$. 
In principle, this inference accuracy should be significantly above chance level ($1/1000$ for ImageNet) because the periphery crop may be highly indicative of the foreground object through correlation, \emph{e.g.}, if the periphery crop is a basketball player then the foreground object is likely a basketball.

Figure \ref{fig:dejavu main} compares the accuracy of $\KNN_A$ to that of $\KNN_B$ when inferring the labels of images in $A_i \in \calA$\footnote{The sets $\calA$ and $\calB$ are exchangeable, and in practice we repeat this test on images from $\calB$ using $\SSL_B$ as the target model and $\SSL_A$ as the reference model, and average the two sets of results.} using $\crop{A_i}$.
Results are shown for VICReg and the supervised model; trends for other models are shown in Appendix \ref{sec:appx simclr results}. For both VICReg and supervised models, inferring the class of $\crop{A_i}$ using $\KNN_B$ (dashed line) through correlation achieves a reasonable accuracy that is significantly above chance level. However, for VICReg, the inference accuracy using $\KNN_A$ (solid red line) is significantly higher, and the accuracy gap between $\KNN_A$ and $\KNN_B$ indicates the degree of \dejavu memorization. We highlight two observations: 
\begin{itemize}
    \item The accuracy gap of VICReg is significantly larger than that of the supervised model. This is especially notable when accounting for the fact that the supervised model is trained to associate randomly augmented crops of images with their ground truth labels. In contrast, VICReg has no label access during training but the embedding of a periphery crop can still encode the image label. 
    \item For VICReg, inference accuracy on the $1\%$ most confident examples is nearly $95\%$, which shows that our simple confidence heuristic can effectively identify the most memorized samples. This result suggests that an adversary can use this heuristic to identify vulnerable training samples to launch a more focused privacy attack.
\end{itemize}
\vspace{-.75em}
\paragraph{The \dejavu score. }
The curves of Figure \ref{fig:dejavu main} show memorization across confidence values for a single training scenario.  To study how memorization changes with different hyperparamters, we extract a single value from these curves: the \dejavu \emph{score} at confidence level $p$. In Figure \ref{fig:dejavu main}, this is the gap between the solid red (or gray) and dashed red (or gray) where confidence ($x$-axis) equal $p\%$. In other words, given the periphery crops of set $\calA$, $\KNN_A$ and $\KNN_B$ separately select and label their top $p\%$ most confident examples, and we report the difference in their accuracy. The \dejavu score captures both the degree of memorization by the accuracy gap and the \emph{ability to identify memorized examples} by the confidence level. If the score is 10\% for $p=33\%$, $\KNN_A$ has 10\% higher accuracy on its most confident third of $\calA$ than $\KNN_B$ does on its most confident third. In the following, we set $p = 20\%$, approximately the largest gap for VICReg (red lines) in Figure \ref{fig:dejavu main}. 
% Specifically, the \dejavu \emph{score} on the top $p\%$ most confident examples is,  
% \begin{equation}
%     \mathrm{DejaVu}(p) = \mathrm{Acc}_{\SSL_A}\big( \calA_{\SSL_A, p}  \big) - \mathrm{Acc}_{\SSL_B}\big( \calA_{\SSL_B, p}  \big) \ ,
%     \label{eqn:dejavu score}
% \end{equation}
% where $\calA_{\SSL_A, p}$
% Here we introduce a DejaVu memorization metric that quantify how much a target model is able to retrieve more class information from a crop than the reference model. We define it as:
% where $p$ is a function that take the $p$ purcent most confident samples.
%Figure \ref{fig:dejavu v. training epochs} shows how \dejavu memorization changes with the number of epochs used to train the embedding model (VICReg and supervised, respectively). The training set size is fixed to 300K samples, and label inference accuracy is computed on the top $20\%$ highest confidence examples. The number of epochs has a very strong influence on the degree of memorization for VICReg as the accuracy gap widens when number of epochs increases. We note that 1000 training epochs is used in several SSL works \citep{vicreg, simclr}. Remarkably, this trend in memorization is \emph{not} reflected in the standard metric for evaluating SSL representations: linear probe accuracy. The gray line in Figure \ref{fig:dejavu v. training epochs} shows the train-test accuracy gap of a linear classifier trained on top of the VICReg embeddings. Although there is a sizeable train-test gap, it does not grow significantly beyond 500 epochs. In contrast, \dejavu memorization (blue line) continues to worsen after 500 epochs. Thus, our test can be used as an alternative to linear probe accuracy to evaluate the memorization of SSL models.
% \vspace{-.75em}

% \paragraph{Comparison with the generalization gap} A network that perform very well on a training set while performing poorly on a test set (assuming the training set and test set sampled uniformly from the same distribution) is probably memorizing the training examples without being able to generalize on the test data. One could expect that measuring the difference in accuracy between the training and test set could give us insights on the degree of \dejavu memorization. However, we show in Figure  \ref{fig:dejavu v. training epochs} and \ref{fig:dejavu v. n} that this is not the case. In fact \dejavu memorization can significantly increase while the train-test gap decrease. In our experiments, we did not find a correlation between \dejavu and generalization.
\vspace{-0.75em}
\paragraph{Comparison with the linear probe train-test gap.} A standard method for measuring SSL performance is to train a linear classifier---what we call a `linear probe'---on its embeddings and compute its performance on a held out test set. From a learning theory standpoint, one might expect the linear probe's train-test accuracy gap to be indicative of memorization: the more a model overfits, the larger is the difference between train set and test set accuracy. However, as seen in Figure \ref{fig:dejavu epochs train set size}, the linear probe gap (dark blue) fails to reveal memorization captured by the \dejavu score (red) \footnote{See section \ref{sec:mitigation} for further discussion of the \dejavu score trends of Figure \ref{fig:dejavu epochs train set size}.}.

% \paragraph{Effect of training epochs.} 
% Figure \ref{fig:dejavu v. training epochs} shows how \dejavu memorization changes with training epochs for VICReg. The training set size is fixed to 300K samples. We observe that the number of epochs has a very strong influence on the degree of memorization for VICReg. From 250 to 1000 epochs, the \dejavu score (red curve) grows threefold: from under 10\% to over 30\%. Remarkably, this trend in memorization is \emph{not} reflected in the standard metric for evaluating SSL representations: linear probe accuracy. The dark blue curve shows the train-test linear probe accuracy gap. Although there is a sizeable train-test gap, it only changes by a few percent beyond 250 epochs. %Thus, our test can be used as an alternative to linear probe accuracy to evaluate the memorization of SSL models.
% \vspace{-.75em}
\begin{figure}[ht]
\label{fig:dejavu epochs and dataset}
\begin{minipage}[t]{0.49\textwidth}
\centering
     \begin{subfigure}[b]{0.48\textwidth}
         \centering
         \includegraphics[width=\textwidth]{figures/deja_vu_vs_epochs.png}
         \vspace{-1.5em}
         \caption{\dejavu vs. epochs}
         \label{fig:dejavu v. training epochs}
     \end{subfigure}
     \begin{subfigure}[b]{0.48\textwidth}
         \centering
         \includegraphics[width=\textwidth]{figures/deja_vu_vs_n.png}
         \vspace{-1.5em}
         \caption{\dejavu vs. train set size}
         \label{fig:dejavu v. n}
     \end{subfigure}~
     \vspace{-0.5em}
    \caption{
    Effect of training epochs and train set size with VICReg on \dejavu score (red) in comparison with linear probe accuracy train-test gap (dark blue). 
    \textbf{Left:} \dejavu score increases with training epochs, indicating growing memorization while the linear probe baseline decreases significantly.  
    \textbf{Right:} \dejavu score stays roughly constant with training set size suggesting that memorization may be problematic even for large datasets. %By comparison, the baseline \emph{declines} by half, spuriously suggesting less memorization. 
    %Both trends are not captured according to the linear probe train-test gap---a common method to evaluate generalization of SSL representations.}
    }
    \label{fig:dejavu epochs train set size}
\end{minipage}
\hfill
\begin{minipage}[t]{0.49\textwidth}
\centering
     \begin{subfigure}[b]{0.48\textwidth}
         \centering
         \includegraphics[width=\textwidth]{figures/vicreg_samples_epochs.pdf}
         \vspace{-1.5em}
         \caption{\dejavu vs. epochs}
         \label{fig:per sample v. training epochs}
     \end{subfigure}
     \begin{subfigure}[b]{0.48\textwidth}
         \centering
         \includegraphics[width=\textwidth]{figures/vicreg_samples_datasets.pdf}
         \vspace{-1.5em}
         \caption{\dejavu vs. train set size}
         \label{fig:per sample v. n}
     \end{subfigure}~
     \vspace{-0.5em}
    \caption{
    \definecolor{part_blue}{rgb}{0.2824, 0.4706, .8157}
	\definecolor{part_red}{rgb}{0.8392, 0.3725, 0.3725}
	\definecolor{part_orange}{rgb}{0.9333, 0.5216, 0.2902}
    Partition of samples $A_i \in \calA$ into the four categories: {\color{gray}unassociated} (not shown), {\color{part_orange}memorized}, {\color{part_red}misrepresented} and {\color{part_blue}correlated} for VICReg. The {\color{part_orange}memorized} samples---those whose labels are predicted by $\KNN_A$ but not by $\KNN_B$---occupy a significantly larger share of the training set than the {\color{part_red}misrepresented} samples---those predicted by $\KNN_B$ but not $\KNN_A$ by chance. %At 1000 epochs, $\approx 15\%$ of the training set is {\color{part_orange}memorized}. The trends across training epochs and training set sizes are consistent with those observed in Figure \ref{fig:dejavu epochs train set size}
    }
    \label{fig:partition attack main}
    \end{minipage}
\vspace{-1em} 
\end{figure}

\iffalse

\begin{minipage}[t]{0.49\textwidth}
\centering
     \begin{subfigure}[b]{0.48\textwidth}
         \centering
         \includegraphics[width=0.95\textwidth]{figures/deja_vu_vs_parameters.png}
         \vspace{-0.4em}
         \caption{\dejavu vs. capacity}
         \label{fig:dejavu v. capacity}
     \end{subfigure}
     \hfill
     \begin{subfigure}[b]{0.48\textwidth}
          \tiny
          \centering
          \setlength{\tabcolsep}{3pt}
          \begin{tabular}{|c|c|c|}
            \hline
            Criteria & DV & Acc P/B \\
            \hline
            Supervised & 8.9 & 55.3/61.1\\
            \hline
            Byol\citep{grill2020byol} & 8.0& 54.3/59.4\\
            \hline
            SimCLR\citep{chen2020simclr} & 10.0 & 44.2/54.1\\
            \hline
            Dino\citep{Dino} & 14.5 & 26.3/55.7 \\
            \hline
            Barlow T.\citep{zbontar2021barlow} & 30.5 & 33.7/54.4\\
            \hline
            VICReg\citep{vicreg} & \textbf{33.2} & 40.3/55.2\\
            \hline
          \end{tabular}
          \vspace{1.3em}
          % \caption{\dejavu (DV) vs. SSL Criterion}
          \caption{\dejavu (DV) vs. Criterion}
          \label{tab:dejavu vs. criterion}
    \end{subfigure}
    \vspace{-0.5em}
    \caption{
    Comparison of \dejavu score for different architectures and training criteria. \textbf{Left:} \dejavu score with VICReg for resnet (purple) and vision transformer (green) architectures versus number of model parameters. As expected, memorization grows with larger model capacity. This trend is more pronounced for convolutional (resnet) than transformer (ViT) architectures. \textbf{Right:} Comparison of \dejavu score and ImageNet validation accuracy (P: using projector embeddings, B: using backbone embeddings) for various SSL criteria. \textbf{Nearly all SSL models have more memorization than the supervised baseline.} 
    % Effect of training epochs and train set size on \dejavu score.
    % \textbf{Left:} \dejavu score increases with higher number of training epochs, indicating worsening memorization.
    % \textbf{Right:} \dejavu score stays roughly constant with training set size. Both trends are not captured according to the linear probe train-test gap---a common method to evaluate generalization of SSL representations.
    }
\end{minipage}
\vspace{-2em} 
\end{figure}

\begin{figure}[ht]
\begin{minipage}[t]{0.49\textwidth}
\centering
     \begin{subfigure}[b]{0.49\textwidth}
         \centering
         \includegraphics[width=\textwidth]{figures/epochs_lb_attk_epochs_acc_top1_legend.pdf}
         \caption{\dejavu vs. epochs}
         \label{fig:dejavu v. training epochs}
     \end{subfigure}
     \begin{subfigure}[b]{0.49\textwidth}
         \centering
         \includegraphics[width=\textwidth]{figures/epochs_lb_attk_datasets_acc_top1_legend.pdf}
         \caption{\dejavu vs. train set size}
         \label{fig:dejavu v. n}
     \end{subfigure}~
     \begin{subfigure}[b]{0.32\textwidth}
         \centering
         \includegraphics[width=0.8\textwidth]{figures/dejavu_vs_parameters.pdf}
         \caption{\dejavu vs. capacity}
         \label{fig:dejavu v. n}
     \end{subfigure}
    \caption{
    Effect of training epochs and train set size on \dejavu score.
    \textbf{Left:} \dejavu score increases with higher number of training epochs, indicating worsening memorization.
    \textbf{Right:} \dejavu score stays roughly constant with training set size. Both trends are not captured according to the linear probe train-test gap---a common method to evaluate generalization of SSL representations.}
    \end{minipage}
\vspace{-1em} 
\end{figure}

\begin{table}[ht]
  \footnotesize
  \centering
  \begin{tabular}{|c|c|}
    \hline
    Supervised & 8.9\\
    \hline
    SimCLR\citep{chen2020simclr} & 10.0\\
    \hline
    Byol\citep{grill2020byol} & 8.0\\
    \hline
    Dino\citep{Dino} & 14.5\\
    \hline
    Barlow T.\citep{zbontar2021barlow} & 30.5\\
    \hline
    VICReg\citep{vicreg} & \textbf{33.2}\\
    \hline
  \end{tabular}
  \caption{DejaVu Score 20\% Conf for various SSL methods.}
  \label{tab:two-row-table}
\end{table}
\vspace{-1em} 
\fi

\iffalse
\begin{figure}[ht]
\begin{minipage}[t]{.49\textwidth}
\centering
     \begin{subfigure}[b]{0.49\textwidth}
         \centering
         \includegraphics[width=\textwidth]{figures/epochs_lb_attk_epochs_acc_top1_legend.pdf}
         \caption{\dejavu vs. epochs}
         \label{fig:dejavu v. training epochs}
     \end{subfigure}
     \hfill
     \begin{subfigure}[b]{0.49\textwidth}
         \centering
         \includegraphics[width=\textwidth]{figures/epochs_lb_attk_datasets_acc_top1_legend.pdf}
         \caption{\dejavu vs. train set size}
         \label{fig:dejavu v. n}
     \end{subfigure}
\caption{
Effect of training epochs and train set size on \dejavu score.
\textbf{Left:} \dejavu score increases with higher number of training epochs, indicating worsening memorization.
\textbf{Right:} \dejavu score stays roughly constant with training set size. Both trends are not captured according to the linear probe train-test gap---a common method to evaluate generalization of SSL representations.}
\label{fig:dejavu epochs and dataset}
\end{minipage}
\hfill
\begin{minipage}[t]{.49\textwidth}
     \centering
     \begin{subfigure}[b]{0.49\textwidth}
         \centering
         \includegraphics[width=\textwidth]{figures/criteria_epochs.pdf}
         \caption{criteria comparison}
         \label{fig:dejavu v. criteria}
     \end{subfigure}
     \hfill
     \begin{subfigure}[b]{0.49\textwidth}
         \centering
         \includegraphics[width=\textwidth]{figures/architecture_epochs.pdf}
         \caption{architecture comparison}
         \label{fig:dejavu v. arch}
     \end{subfigure}
\caption{
Effect of SSL training criteria and model architectures on \dejavu score.
%the accuracy gap between target model (trained on $\calA$) and reference model (trained on $\calB$) making predictions on their 20\% most confident examples.
\textbf{Left:} \dejavu score for various training criteria.
%Barlow and VICReg have the heaviest degree of memorization, while SimCLR and BYOL have the least. 
%Note that we show detailed reconstructions of SimCLR's training data in Section \ref{sec:visualizing} despite its relatively low degree of \dejavu. 
%Regardless, Although SimCLR and BYOL have the least, we  visualize detailed reconstructions with SimCLR in section \ref{sec:mem v corr} 
All SSL models have significantly more \dejavu than the supervised baseline. \textbf{Right:} \dejavu score versus epochs for various training architectures. As expected, lower capacity architectures (Resnet18, Resnet34) reduce \dejavu but not completely. 
}
\label{fig:dejavu criteria and architecture}
\end{minipage}
\vspace{-1em} 
\end{figure}
\fi
% %\begin{figure}[ht]
%%%
%VICREG
%%%
     \centering
     \begin{subfigure}[b]{0.49\textwidth}
         \centering
         \includegraphics[width=\textwidth]{figures/sample_level_training_epochs.pdf}
         \caption{Categories of training samples vs. number of epochs}
         \label{fig:sample level epochs}
     \end{subfigure}
     \hfill
     \begin{subfigure}[b]{0.49\textwidth}
         \centering
         \includegraphics[width=\textwidth]{figures/sample_level_training_set_size.pdf}
         \caption{Categories of training samples vs. training set size}
         \label{fig:sample level training size}
     \end{subfigure}
\caption{
\definecolor{part_blue}{rgb}{0.2824, 0.4706, .8157}
\definecolor{part_red}{rgb}{0.8392, 0.3725, 0.3725}
\definecolor{part_orange}{rgb}{0.9333, 0.5216, 0.2902}
Partition of samples $A_i \in \calA$ into the four categories: {\color{gray}unassociated} (not shown), {\color{part_orange}memorized}, {\color{part_red}misrepresented} and {\color{part_blue}correlated}. The {\color{part_orange}memorized} samples---ones whose labels are predicted by $\KNN_A$ but not by $\KNN_B$---occupy a significantly larger share for VICReg compared to the supervised model, indicating that sample-level \dejavu memorization is more prevalent in VICReg. %The trends across number of training epochs and training set sizes are consistent with those observed in Figures \ref{fig:dejavu epochs and dataset} and \ref{fig:dejavu criteria and architecture}.
}
\label{fig:partition attack main appendix}
\end{figure}
% \paragraph{Effect of training set size.} 
% Figure \ref{fig:dejavu v. n} shows how \dejavu memorization responds to the model's training set size. The number of training epochs is fixed to 1000. Interestingly, training set size appears to have almost \emph{no} influence on the \dejavu score (red line), indicating that memorization is equally prevalent with a 100K dataset and a 500K dataset (which suggests that \dejavu memorization may be detectable for larger datasets). Meanwhile, the linear probe train-test accuracy gap \emph{declines} by half as the dataset size grows, failing to represent the memorization quantified by our test. 
% The trend is completely different according to linear probe accuracy (dark blue line), the train-test gap shrinks substantially when increasing the training set size from 100K to 500K. This highlights that the train-test gap is not able to capture \dejavu memorization. %Our evidence suggests that \dejavu memorization may be detectable even for large-scale training datasets. 
%\vspace{-.75em}

\vspace{-.75em} 
\subsection{Sample-level Memorization}
\label{sec:dissection}

% Section \ref{sec:label inference accuracy} shows the \emph{average} level of \dejavu memorization on a subset of the training set $\calA$. However, this average tell us only what the attacker success rate might be without explicitly describing how much of the datatset is \dejavu memorized.
The \dejavu score shows, \emph{on average}, how much better an adversary can select and classify images when using the target model trained on them. 
This average score does not tell us how many individual images have their label successfully recovered by $\KNN_A$ but not by $\KNN_B$. In other words, how many images are exposed by virtue of \emph{being in training set} $\calA$: a risk notion foundational to differential privacy. 
% However, from the perspective of an individual image $A_i \in \calA$, it is informative to know whether it was correctly classified 
To better quantify what fraction of the dataset is at risk, we perform a sample-level analysis by fixing a sample $A_i \in \calA$ and observing the label inference result of $\KNN_A$ vs. $\KNN_B$.
To this end, we partition samples $A_i \in \calA$ based on the result of label inference into four distinct categories: {\color{gray}\textbf{Unassociated}} - label inferred with neither KNN; {\color{part_orange}\textbf{Memorized}} - label inferred only with $\KNN_A$; {\color{part_red}\textbf{Misrepresented}} - label inferred only with $\KNN_B$; {\color{part_blue}\textbf{Correlated}} - label inferred with both KNNs. 
% \begin{multicols}{2}
% \begin{itemize}
%     \vspace{-.75em}
%     \setlength\itemsep{0.15em}
%     \item {\color{gray}Unassociated}: label inferred with neither KNN   
%     \item {\color{part_orange}Memorized}: label only inferred by $\KNN_A$
%     \item {\color{part_red}Misrepresented}: label only inferred with $\KNN_B$
%     \item {\color{part_blue}Correlated}: label inferred with both KNNs
%     \vspace{-.75em}
% \end{itemize}
% \end{multicols}
Intuitively, {\color{gray}unassociated} samples are ones where the embedding of $\crop{A_i}$ does not encode information about the label. {\color{part_blue}Correlated} samples are ones where the label can be inferred from $\crop{A_i}$ using correlation, \emph{e.g.}, inferring the foreground object is basketball given a crop showing a basketball player. Ideally, the {\color{part_red}misrepresented} set should be empty but contains a small portion of examples due to chance.
\emph{Déjà vu} memorization occurs for {\color{part_orange}memorized} samples where the embedding of $\SSL_B$ does not encode the label but the embedding of $\SSL_A$ does. To measure the pervasiveness of \dejavu memorization, we compare the size of the {\color{part_orange}memorized} and {\color{part_red}misrepresented} sets.
Figure \ref{fig:partition attack main} shows how the four categories of examples change with number of training epochs and training set size. The {\color{gray}unassociated} set is not shown since the total share adds up to one. The {\color{part_red}misrepresented} set remains under $5\%$ and roughly unchanged across all settings, consistent with our explanation that it is due to chance. In comparison, VICReg's {\color{part_orange}memorized} set surpasses $15\%$ at 1000 epochs. Considering that up to 5\% of these memorized examples could also be due to chance, we conclude that \textbf{at least 10\% of VICReg's training set is \dejavu memorized.} 
%is many times larger than its {\color{part_red}misrepresented} set, indicating substantial sample-level \dejavu memorization. 
%In fact, \textbf{it is 15\% of the training set that is \dejavu memorized with VICReg.}
%The trends across different number of training epochs and training set sizes match those observed in Section \ref{sec:label inference accuracy}. % On the other hand, the supervised model's {\color{part_orange}memorized} set is only marginally larger than its {\color{part_red}misrepresented} set.

% The trends across different number of training epochs and training set sizes match those observed in Section \ref{sec:label inference accuracy}: Increasing the number of epochs increases \dejavu memorization (Figure \ref{fig:per sample v. training epochs}), while increasing the training set size does not appear to reduce \dejavu memorization (Figure \ref{fig:per sample v. n}). 
%\section{Visualizing \emph{Déjà Vu} Memorization}
\label{sec:visualizing}
Beyond enabling label inference using a periphery crop, we show that \dejavu memorization allows the SSL model to encode other forms of information about a training image. Namely, we train an RCDM \citep{RCDM} on the public dataset $\calX$ and use it to visually reconstruct training images given their periphery crop.
We aim to answer the following two questions: \textbf{(1)} Can we visualize the distinction between correlation and \dejavu memorization? \textbf{(2)} What foreground object details can be extracted from the SSL model beyond class label? 
% \begin{enumerate}[noitemsep, leftmargin=*, topsep=0pt]
%     \item Can we visualize the distinction between correlation and \dejavu memorization? 
%     \item What foreground object details can be extracted from the SSL model beyond class label? 
% \end{enumerate}
\vspace{-0.5em}
\paragraph{Reconstruction pipeline.}
RCDM is a conditional generative model that is trained on the \emph{backbone embedding} of images $X_i \in \calX$ to generate an image that resembles $X_i$. All training images are first face-blurred for privacy purposes. \citet{RCDM} showed that the backbone embedding of SSL models contains more low-level information about the image, making them better suited for conditioning the RCDM.
At test time, following the pipeline in Figure \ref{fig:split_and_pipeline_cartoon}, we first use the projector embedding to find the KNN subset for the periphery crop, $\crop{A_i}$, and then average their backbone embeddings as input to the RCDM model. Ideally, when the public set contains enough representative images, the average representation of the KNN subset encodes objects present in $A_i$, and the RCDM model decodes this representation to visualize these objects.
% \begin{figure}[ht]
%%%
%VICREG
%%%
     \centering
     \begin{subfigure}[b]{0.49\textwidth}
         \centering
         \includegraphics[width=\textwidth]{figures/sample_level_training_epochs.pdf}
         \caption{Categories of training samples vs. number of epochs}
         \label{fig:sample level epochs}
     \end{subfigure}
     \hfill
     \begin{subfigure}[b]{0.49\textwidth}
         \centering
         \includegraphics[width=\textwidth]{figures/sample_level_training_set_size.pdf}
         \caption{Categories of training samples vs. training set size}
         \label{fig:sample level training size}
     \end{subfigure}
\caption{
\definecolor{part_blue}{rgb}{0.2824, 0.4706, .8157}
\definecolor{part_red}{rgb}{0.8392, 0.3725, 0.3725}
\definecolor{part_orange}{rgb}{0.9333, 0.5216, 0.2902}
Partition of samples $A_i \in \calA$ into the four categories: {\color{gray}unassociated} (not shown), {\color{part_orange}memorized}, {\color{part_red}misrepresented} and {\color{part_blue}correlated}. The {\color{part_orange}memorized} samples---ones whose labels are predicted by $\KNN_A$ but not by $\KNN_B$---occupy a significantly larger share for VICReg compared to the supervised model, indicating that sample-level \dejavu memorization is more prevalent in VICReg. %The trends across number of training epochs and training set sizes are consistent with those observed in Figures \ref{fig:dejavu epochs and dataset} and \ref{fig:dejavu criteria and architecture}.
}
\label{fig:partition attack main appendix}
\end{figure}
%\begin{figure*}[t!]
%%%
%DAM
%%%
     \centering
     \begin{subfigure}[b]{0.49\textwidth}
         \centering
         \includegraphics[width=\textwidth]{figures/dam_corr.png}
         \caption{A {\color{part_blue}correlated} dam example}
         \label{fig:dam correlated}
     \end{subfigure}
     \hfill
     \begin{subfigure}[b]{0.49\textwidth}
         \centering
         \includegraphics[width=\textwidth]{figures/dam_mem.png}
         \caption{A {\color{part_orange}memorized} dam example}
         \label{fig:dam memorized}
     \end{subfigure}
\caption{
{\color{part_blue}Correlated} and {\color{part_orange}Memorized} examples from the \emph{dam} class. Both $\SSL_A$ and $\SSL_B$ are SimCLR models.
\textbf{Left:} The periphery crop (pink square) contains a concrete structure that is often present in images of dams. Consequently, the trained RCDM can reconstruct the foreground object using representations from both $\SSL_A$ and $\SSL_B$ through this correlation.
\textbf{Right:} The periphery crop only contains a patch of water. The embedding produced by $\SSL_B$ only contains enough information to infer that the foreground object is related to water, as reflected by its KNN set and RCDM reconstruction. In contrast, the embedding produced by $\SSL_A$ memorizes the association of this patch of water with dam and the RCDM can visualize the embedding to produce images of dams.
}
\vspace{-1ex}
\label{fig:mem v corr dam}
\end{figure*}


\begin{figure*}[t!]
%%%
%DAM
%%%
     \centering
     \begin{subfigure}[b]{0.49\textwidth}
         \centering
         \includegraphics[width=\textwidth]{figures/dam_corr.png}
         \caption{A {\color{part_blue}correlated} dam example}
         \label{fig:dam correlated}
     \end{subfigure}
     \hfill
     \begin{subfigure}[b]{0.49\textwidth}
         \centering
         \includegraphics[width=\textwidth]{figures/dam_mem.png}
         \caption{A {\color{part_orange}memorized} dam example}
         \label{fig:dam memorized}
     \end{subfigure}
\caption[Correlated and Memorized examples from the \emph{dam} class.]{
Correlated and Memorized examples from the \emph{dam} class. Both $\SSL_A$ and $\SSL_B$ are SimCLR models.
\textbf{Left:} The periphery crop (pink square) contains a concrete structure that is often present in images of dams. Consequently, the trained RCDM can reconstruct the foreground object using representations from both $\SSL_A$ and $\SSL_B$ through this correlation.
\textbf{Right:} The periphery crop only contains a patch of water. The embedding produced by $\SSL_B$ only contains enough information to infer that the foreground object is related to water, as reflected by its KNN set and RCDM reconstruction. In contrast, the embedding produced by $\SSL_A$ memorizes the association of this patch of water with dam and the RCDM can visualize the embedding to produce images of dams.
}
\label{fig:mem v corr dam}
\end{figure*}


\begin{figure}[t!]
%%%
%BADGER
%%%
     \centering
     \begin{subfigure}[b]{0.49\textwidth}
         \centering
         \includegraphics[width=\textwidth]{figures/euro_badgers.png}
         \caption{{\color{part_orange}Memorized} European badgers}
         \label{fig:euro badgers}
     \end{subfigure}
     \hfill
     \begin{subfigure}[b]{0.49\textwidth}
         \centering
         \includegraphics[width=\textwidth]{figures/amer_badgers.png}
         \caption{{\color{part_orange}Memorized} American badgers}
         \label{fig:amer badgers}
     \end{subfigure}
\caption[Visualization of \dejavu memorization beyond class label.]{
Visualization of \dejavu memorization beyond class label. Both $\SSL_A$ and $\SSL_B$ are VICReg models. 
The four images shown belong to the memorized set of $\SSL_A$ from the \emph{badger} class. RCDM reconstruction using embeddings from $\SSL_A$ can reveal not only the correct class label, but also the specific badger species: \emph{European} (left) and \emph{American} (right). Such information does not appear to be memorized by the reference model $\SSL_B$.
} 
\label{fig:in class badger}
\end{figure}


% \subsection{Visualizing Correlation vs. Memorization}
\label{sec:mem v corr}
\vspace{-0.5em} 
\paragraph{Visualizing Correlation vs. Memorization.}
Figure \ref{fig:mem v corr dam} shows examples of dams from the {\color{part_blue}correlated} set (left) and the {\color{part_orange}memorized} set (right) as defined in Section \ref{sec:dissection}, along with the associated KNN set and RCDM reconstruction. Both $\SSL_A$ and $\SSL_B$ are SimCLR models. In Figure \ref{fig:dam correlated}, the periphery crop is represented by the pink square, which contains concrete structure attached to the dam's main structure. As a result, both $\SSL_A$ and $\SSL_B$ produce embeddings of $\crop{A_i}$ whose KNN set in $\calX$ consist of dams, \emph{i.e.}, there is a correlation between the concrete structure in $\crop{A_i}$ and the foreground dam. The RCDM reconstructions also consist of dams or structures that closely resemble dams. 
In Figure \ref{fig:dam memorized}, the periphery crop only contains a patch of water, which does not strongly correlate with dams in the ImageNet distribution. Evidently, the reference model $\SSL_B$ embeds $\crop{A_i}$ close to that of other objects commonly found in water, such as sea turtle and submarine. In contrast, the KNN set according to $\SSL_A$ all contain dams despite the vast number of alternative possibilities within the ImageNet classes, and the RCDM reconstruction outputs dams as well which highlight memorization in $\SSL_A$ between this specific patch of water and the dam. %\footnote{See Appendix \ref{sec:appx visualization} to see the same trend in the \emph{yellow garden spider} class.}


% \subsection{Visualizing Memorization Beyond Class Label}
% \label{sec:in class variation}
\vspace{-0.5em} 
\paragraph{Visualizing Memorization Beyond Class Label.}
We now use our reconstruction algorithm to show that \dejavu memorization can be exploited to reveal detailed information beyond class label. Figure \ref{fig:in class badger} shows four examples of badgers from the {\color{part_orange}memorized} set. In all four images, the periphery crop (pink square) does not contain any indication that the foreground object is a badger. Despite this, the KNN set and the RCDM reconstruction using $\SSL_A$ consistently produce images of badgers, while the same does not hold for $\SSL_B$.
More interestingly, reconstructions using $\SSL_A$ in Figure \ref{fig:euro badgers} all contain \emph{European} badgers, while reconstructions in Figure \ref{fig:amer badgers} all contain \emph{American} badgers, accurately reflecting the species of badger present in the respective training images. Since ImageNet-1K does \emph{not} differentiate between these two species of badgers, our reconstructions show that SSL models can memorize information that is highly specific to a training sample beyond its class label\footnote{See Appendix \ref{sec:appx visualization} for additional visualization experiments.}.%\footnote{See Appendix \ref{sec:appx visualization} for the same trend in the \emph{aircraft carrier} class.}.





%\vspace{-.5em} 
\section{Mitigation of \dejavu memorization}
\label{sec:mitigation}
% We do not have an understanding on why \dejavu occur so strongly in some SSL pretraining, however we present additional experiments that shed light on which parameters have the biggest impact on \dejavu memorization.
\begin{figure}[ht]
\label{fig:mitigations}
\begin{minipage}[t]{0.5\textwidth}
\centering
     \begin{subfigure}[b]{0.47\textwidth}
         \centering
         \includegraphics[width=\textwidth]{figures/dejavu_vicreg_param.png}
         \vspace{-1.5em}
         \caption{Loss hyper-parameter}
         \label{fig:dejavu v. invariance}
     \end{subfigure}
     \begin{subfigure}[b]{0.49\textwidth}
         \centering
         \includegraphics[width=\textwidth]{figures/deja_vu_vs_layer.png}
         \vspace{-1.5em}
         \caption{Guillotine regularization}
         \label{fig:dejavu v. guillotine}
     \end{subfigure}~
     \vspace{-0.5em}
    \caption[Effect of two kinds of hyper-parameters on VICReg memorization. ]{
    Effect of two kinds of hyper-parameters on VICReg memorization. \textbf{Left:} \dejavu score (red) versus the \emph{invariance} loss parameter, $\lambda$, used in the VICReg criterion (100k dataset). Larger $\lambda$ significantly reduces \dejavu, with minimal effect on linear probe validation performance (green). $\lambda = 25$ (near maximum \dejavu) is recommended in the original paper \textbf{Right:} \dejavu score versus projector layer---guillotine regularization \cite{Guillotine}---from projector to backbone. Removing the projector can significantly reduce \dejavu. Appendix \ref{sec:guillotine} shows that the backbone still can memorize, however; we demonstrate reconstructions using the SimCLR backbone.
    }
\end{minipage}
\hfill
\begin{minipage}[t]{0.48\textwidth}
\centering
     \begin{subfigure}[b]{0.46\textwidth}
         \centering
         \includegraphics[width=\textwidth]{figures/deja_vu_vs_parameters.png}
         \vspace{-1.3em}
         \caption{\dejavu vs. capacity}
         \label{fig:dejavu v. capacity}
     \end{subfigure}
     \hfill
     \begin{subfigure}[b]{0.52\textwidth}
          \tiny
          \centering
          \setlength{\tabcolsep}{3pt}
          \begin{tabular}{|c|c|c|}
            \hline
            Criteria & DV & Acc P/B \\
            \hline
            Supervised & 8.9 & 55.3/61.1\\
            \hline
            Byol\citep{grill2020byol} & 8.0& 54.3/59.4\\
            \hline
            SimCLR\citep{chen2020simclr} & 10.0 & 44.2/54.1\\
            \hline
            Dino\citep{Dino} & 14.5 & 26.3/55.7 \\
            \hline
            Barlow T.\citep{zbontar2021barlow} & 30.5 & 33.7/54.4\\
            \hline
            VICReg\citep{vicreg} & \textbf{33.2} & 40.3/55.2\\
            \hline
          \end{tabular}
          \vspace{1.3em}
          % \caption{\dejavu (DV) vs. SSL Criterion}
          \caption{\dejavu (DV) vs. Criterion}
          \label{tab:dejavu vs. criterion}
    \end{subfigure}
    \vspace{-1.4em}
    \caption[Effect of model architecture and criterion on \dejavu memorization.]{
    %Comparison of \dejavu score for different architectures and training criteria. 
    Effect of model architecture and criterion on \dejavu memorization. 
    \textbf{Left:} \dejavu score with VICReg for resnet (purple) and vision transformer (green) architectures versus number of model parameters. As expected, memorization grows with larger model capacity. This trend is more pronounced for convolutional (resnet) than transformer (ViT) architectures. \textbf{Right:} Comparison of \dejavu score 20\% conf. and ImageNet linear probe validation accuracy (P: using projector embeddings, B: using backbone embeddings) for various SSL criteria. %\textbf{Nearly all SSL models have more memorization than the supervised baseline.} 
    % Effect of training epochs and train set size on \dejavu score.
    % \textbf{Left:} \dejavu score increases with higher number of training epochs, indicating worsening memorization.
    % \textbf{Right:} \dejavu score stays roughly constant with training set size. Both trends are not captured according to the linear probe train-test gap---a common method to evaluate generalization of SSL representations.
    }
    \end{minipage}
\end{figure}
We cannot yet make claims on why \dejavu occurs so strongly for some SSL training settings and not for others. To gain some intuition for future work, we present additional observations that shed light on which parameters have the most salient impact on \dejavu memorization.
\vspace{-.75em}
\paragraph{Déjà vu memorization worsens by increasing number of training epochs.} 
Figure \ref{fig:dejavu v. training epochs} shows how \dejavu memorization changes with number of training epochs for VICReg. The training set size is fixed to 300K samples. From 250 to 1000 epochs, the \dejavu score (red curve) grows \emph{threefold}: from under 10\% to over 30\%. Remarkably, this trend in memorization is \emph{not} reflected by the linear probe gap (dark blue), which only changes by a few percent beyond 250 epochs. 

%\vspace{-.75em}
\paragraph{Training set size has minimal effect on \dejavu memorization.} Figure \ref{fig:dejavu v. n} shows how \dejavu memorization responds to the model's training set size. The number of training epochs is fixed to 1000. Interestingly, training set size appears to have almost \emph{no} influence on the \dejavu score (red line), indicating that memorization is equally prevalent with a 100K dataset and a 500K dataset. This result suggests that \dejavu memorization may be detectable even for large datasets. Meanwhile, the standard linear probe train-test accuracy gap \emph{declines} by more than half as the dataset size grows, failing to represent the memorization quantified by our test. 
% The trend is completely different according to linear probe accuracy (dark blue line), the train-test gap shrinks substantially when increasing the training set size from 100K to 500K. This highlights that the train-test gap is not able to capture \dejavu memorization. Our evidence suggests that \dejavu memorization may be detectable even for large-scale training datasets. 
\vspace{-0.5em}
\paragraph{Training loss hyper-parameter has a strong effect.} 
%We show in Figure \ref{fig:dejavu v. training epochs} that the number of training epochs is an important factor that can increase significantly \dejavu memorization. In contrast, the dataset size does not impact much \dejavu as shown in Figure \ref{fig:dejavu epochs train set size}. 
Loss hyper-parameters, like VICReg's invariance coefficient (Figure \ref{fig:dejavu v. invariance}) or SimCLR's temperature parameter (Appendix Figure \ref{fig:simclr temperature}) significantly impact \dejavu with minimal impact on the linear probe validation accuracy.

\vspace{-0.5em}
\paragraph{Some SSL criteria promote stronger \dejavu memorization.} Table \ref{tab:dejavu vs. criterion} demonstrates that the degree of memorization varies widely for different training criteria. VICReg and Barlow Twins have the highest \dejavu scores while SimCLR and Byol have the lowest.
%\footnote{We show detailed reconstructions of SimCLR's training data in Section \ref{sec:visualizing} despite its relatively low degree of \dejavu.}.
With the exception of Byol, all SSL models have more \dejavu memorization than the supervised model. Interestingly, different criteria can lead to similar linear probe validation accuracy and very different degrees of \dejavu as seen with SimCLR and Barlow Twins. Note that low degrees of \dejavu can still risk training image reconstruction, as exemplified by the SimCLR reconstructions in Figures \ref{fig:mem v corr dam} and \ref{fig:mem v corr spider}. 
%\vspace{-1em}
\vspace{-0.5em}
\paragraph{Larger models have increased \dejavu memorization.} Figure \ref{fig:dejavu v. capacity} validates the common intuition that lower capacity architectures (Resnet18/34) result in less memorization than their high capacity counterparts (Resnet50/101). 
% \begin{wrapfigure}{r}{0.25\textwidth} 
%     \centering
%     \includegraphics[width=0.25\textwidth]{figures/attk_layer_acc_top1_legend.pdf}
%     \caption{\dejavu memorization versus layer from backbone (0) to projector output (3).}
%     \label{fig:dejavu vs layer}
%     \vspace{-8ex}
% \end{wrapfigure}
We see the same trend for vision transformers as well. %This comes with a tradeoff, since reduced model capacity can result in a nontrivial degradation of representation quality\cite{vicreg, simclr}.  
\vspace{-0.5em}
\paragraph{Guillotine regularization can help reduce \dejavu memorization.} Previous experiments were done using the projector embedding. In Figure \ref{fig:dejavu v. guillotine}, we present how Guillotine regularization\citep{Guillotine} (removing final layers in a trained SSL model) impacts \dejavu with VICReg\footnote{Further experiments are available in Appendix \ref{sec:guillotine}.}. Using the backbone embedding instead of the projector embedding seems to be the most straightforward way to mitigate \dejavu memorization. However, as demonstrated in Appendix \ref{sec:appx backbone results}, backbone representation with low \dejavu score can still be leveraged to reconstruct some of the training images.

\section{Conclusion}
\label{sec:conclusion}

We defined and analyzed \dejavu memorization, a notion of unintended memorization of partial information in image data. As shown in Sections \ref{sec:quant} and \ref{sec:visualizing}, SSL models can largely exhibit \dejavu memorization on their training data, and this memorization signal can be extracted to infer or visualize image-specific information.
Since SSL models are becoming increasingly widespread as foundation models for image data, negative consequences of \dejavu memorization can have profound downstream impact and thus deserves further attention. 
Future work should focus on understanding how \dejavu emerges in the training of SSL models and why methods like Byol are much more robust to \dejavu than VICReg and Barlow Twins. In addition, trying to characterize which data points are the most at risk of \dejavu could be crucial to get a better understanding on this phenomenon. 
