\graphicspath{{./chapters/chapter2/}}
%\newtheorem{thm}{Theorem}
%\newtheorem{lem}[thm]{Lemma}
%\DeclareMathOperator*{\argmax}{arg\,max}
%\DeclareMathOperator*{\argmin}{arg\,min}

\def\D{{\mathcal D}}
\def\Pphi{\overline{\Phi}}
\def\F{{\mathcal F}}
\def\N{{\mathcal N}}
%\def\R{{\mathbb R}}
\def\E{{\mathbb E}}
\def\A{\Pi}
\def\B{\Sigma}
\def\diam{\text{diam}}
\def\c{\mathcal L}
\def\l{\ell}
\def\seq{seq}
\def\R{\mathbb{R}}
\def\C{\mathcal C}
\def\p{p}
\def\s{size}
\def\L{\mathcal L}
\def\o{opt}
\def\H{\mathcal H}
\def\calH{\mathcal H}
\def\of{approxCluster}
\def\on{onlineCluster}
\def\R{\mathbb R}
\def\Y{\{\pm 1\}}
\def\U{\mathbb U}
\def\dd{\Delta}
\def\simp{{U\Delta}}
\def\g{g}
\def\rr{R}
\def\f{f}


\chapter{Sample Complexity of Robust Linear Classification on Separated Data} 

\section{Introduction}

Motivated by the use of machine learning in safety-critical settings, adversarially robust classification has been of much recent interest. Formally, the problem is as follows. A learner is given training data drawn from an underlying distribution $D$, a hypothesis class $\calH$, a robustness metric $d$, and a radius $r$. The learner's goal is to find a classifier $h \in \calH$ which has the lowest robust loss at radius $r$. The robust loss of a classifier is the expected fraction of examples where either $f(x) \neq y$ or where there exists an $x'$ at distance $d(x, x') \leq r$ such that $f(x) \neq f(x')$.  Robust classification thus aims to find a classifier that maximizes accuracy on examples that are distance $r$ or more from the decision boundary, where distances are measured according to the metric $d$.


In this work, we ask: how many samples are needed to learn a classifier with low robust loss when $\calH$ is the class of linear classifiers, and $d$ is an $\ell_p$-metric? Prior work has provided both upper~\cite{bartlett19, ravikumar20} as well as lower bounds~\cite{Schmidt18, ravikumar20} on the sample complexity of the problem. However, almost all look at settings where the data distribution itself is not separated --  data from different classes overlap or are close together in space. In this case, the classifier that minimizes robust loss is quite different from the one that minimizes error, which often leads to strong sample complexity gaps. Many real tasks where robust solutions are desired however tend to involve well-separated data~\cite{Yang20}, and hence it is instructive to look at what happens in these cases.

With this motivation, we consider in this work robust classification of data that is linearly $r$-separable. Specifically, there exists a linear classifier which has zero robust loss at robustness radius $r$. This case is thus the analog of the realizable case for robust classification, and we consider both upper and lower bounds in this setting.

For lower bounds, prior work \cite{Cullina18} shows that both standard and robust linear classification have VC-dimension $O(d)$, and consequently have similar bounds on the expected loss in the worst case. However, these results do not apply to this setting since we are specifically considering well-separated data, which greatly restricts the set of possible worst-case distributions.  For our lower bound, we provide a family of distributions that are linearly $r$-separable and where the maximum margin classifier, given $n$ independent samples, has error $O(1/n)$. In contrast, any algorithm for finding the minimum robust loss classifier has robust loss at least $\Omega(d/n)$, where $d$ is the data dimension. These bounds hold for all $\ell_p$-norms provided $p > 1$, including $p=2$ and $p=\infty$. Unlike prior work, our bounds do not rely on the difference in loss between the solutions with optimal robust loss and error, and hence cannot be obtained by prior techniques. Instead, we introduce a new geometric construction that exploits the fact that learning a classifier with low robust loss when data is linearly $r$-separated requires seeing a certain number of samples close to the margin.

For upper bounds, prior work \cite{bartlett19} provides a bound on the Rademacher complexity of adversarially robust learning, and show that it can be worse than the standard Rademacher complexity by a factor of $d^{1/q}$ for $\ell_p$-norm robustness where $1/p + 1/q = 1$. Thus, an interesting question is whether dimension-independent bounds, such as those for the accuracy under large margin classification, can be obtained for robust classification as well. Perhaps surprisingly, we show that when data is really well-separated, the answer is yes. Specifically, if the data distribution is linearly $r + \gamma$-separable, then there exists an algorithm that will find a classifier with robust loss $O(\Delta^2/\gamma^2 n)$ at radius $r$ where $\Delta$ is the diameter of the instance space. Observe that much like the usual sample complexity results on SVM and perceptron, this upper bound is independent of the data dimension and depends only on the excess margin (over $r$). This establishes that when data is really well-separated, finding robust linear classifiers does not require a very large number of samples. 

While the main focus of this work is on linear classifiers, we also show how to generalize our upper bounds to Kernel Classification, where we find a similar dynamic with the loss being governed by the excess margin in the embedded kernel space. However, we defer a thorough investigation of robust kernel classification as an avenue for future work.

Our results imply that while adversarially robust classification may be more challenging than simply accurate classification when the classes overlap, the story is different when data is well-separated. Specifically, when data is linearly (exactly) $r$-separable, finding an $r$-separated solution to robust loss $\epsilon$ may require $\Omega(d/\epsilon)$ samples for some distribution families where finding an accurate solution is easier. Thus in this case, there is a gap between the sample complexities of robust and simply accurate solutions, and this is true regardless of the $\ell_p$ norm in which robustness is measured. In contrast, if data is even more separated -- linearly $r + \gamma$-separable --  then we can obtain a dimension-independent upper bound on the sample complexity, much like the sample complexity of SVMs and perceptron. Thus, how separable the data is matters for adversarially robust classification, and future works in the area should consider separability while discussing the sample complexity.

\subsection{Related Work}

There is a large body of work \cite{Carlini17, Liu17, Papernot17, Papernot16, Szegedy14, Hein17, Katz17, Wu16,Steinhardt18, Sinha18} empirically studying adversarial examples primarily in the context of neural networks. Several works \cite{Schmidt18, Raghunathan20, Tsipras19} have empirically investigated trade-offs between robust and standard classification.

On the theoretical side, this phenomenon has been studied in both the parametric and non-parametric settings. On the parametric side, several works \cite{loh18, attias19, Srebro19, bartlett19, pathak20} have focused on finding distribution agnostic bounds of the sample complexity for robust classification. In \cite{Srebro19}, Srebro et. al. showed through an example that the VC dimension of robust learning may be much larger than standard or accurate learning indicating that the sample complexity bounds may be higher. However, their example did not apply to linear classifiers. 

\cite{Kane20} considers learning linear classifiers robustly, but is primarily focused on computational complexity as opposed to sample complexity.

In \cite{bartlett19}, Bartlett et. al. investigated the Rademacher complexity of robustly learning linear classifiers as well as neural networks. They showed that in both cases, the robust Rademacher complexity can be bounded in terms of the dimension of the input space -- thus indicating a possible gap between standard and robust learning. However, as with the works considering VC dimension, this work is fundamentally focused on upper bounds  -- they do not show true lower bounds on data requirements.

Because of it's simplicity and elegance, the case where the data distribution is a mixture of Gaussians has been particularly well-studied. The first such work was \cite{Schmidt18}, in which Schmidt et. al. showed an $\Omega(\sqrt{d})$ gap between the standard and robust sample complexity for a mixture of two Gaussians using the $\ell_\infty$ norm. This was subsequently expanded upon in \cite{Bhagoji19}, \cite{robey20} and  \cite{ravikumar20}. \cite{Bhagoji19} introduces a notion of ``optimal transport," which they subsequently apply to the Gaussian case, deriving a closed form expression for the optimally robust linear classifier. Their results apply to any $\ell_p$ norm. \cite{robey20} applies expands upon \cite{Schmidt18} by consider mixtures of three Gaussians in both the $\ell_2$ and $\ell_\infty$ norms. Finally, \cite{ravikumar20} fully generalizes the results of \cite{Schmidt18} providing tight upper and lower bounds on the standard and robust sample complexities of a mixture of two Gaussians, in any norm (including $\ell_p$ for $p \in [1, \infty]$). \cite{Schmidt18} and \cite{ravikumar20} bear the most relevance with our work, and we consequently carefully compare our results in section \ref{sec:comparison}.

Another approach for lower and upper bounds on sample complexities for linear classifiers can be found in \cite{Cullina18}, which examines the robust VC dimension of learning linear classifiers. They show that the VC dimension is $d+1$, just as it is in the standard case. This implies that the bounds in the robust case match the bounds in the standard case and in particular shows a lower bound of $\Omega(d/n)$ on the expected loss of learning a robust linear classifier from $n$ samples.

While this result appears to match our lower bound, there is a crucial distinction between the bounds. Our bound implies that there exists some distribution with a large $\ell_2$ margin for which the expected robust loss must be $\Omega(d/n)$. On the other hand, standard results about learning linear classifiers on large margin data implies that the expected standard loss will be $O(1/n)$ (when running the max-margin algorithm). For this reason, our paper provides a case in the well-separated setting in which learning linear classifiers is provably more difficult (in terms of sample complexity) in the robust setting than in the standard setting. By contrast, \cite{Cullina18} does not show this. Their paper only implies (through standard VC constructions) the existence of \textit{some} distribution that is difficult to learn, and the standard PAC bounds cannot ensure that such a distribution also has a large $\ell_2$ margin.

In the non-parametric setting, there are several works which contrast standard learning with robust learning. \cite{WJC18} considers the nearest neighbors algorithm, and shows how to adapt it for converging towards a robust classifier. In \cite{YRWC19}, Yang et. al. propose the $r$\textit{-optimal classifier}, which is the robust analog of the Bayes optimal classifier. Through several examples they show that it is often a fundamentally different classifier - which can lead to different convergence behavior in the standard and robust settings. \cite{Bhattacharjee20} unified these approaches by specifying conditions under which non-parametric algorithms can be adapted to converge towards the $r$-optimal classifier, thus introducing $r$-consistency, the robust analog of consistency.

\section{Preliminaries}
We consider binary classification over $\R^d \times \Y$. Our metric of choice is the $\ell_p$ norm, where $p > 1$ (including $p = \infty$) is arbitrary. For $x \in \R^d$, we will use $||x||_p$ to denote the $\ell_p$ norm of $x$, and consequently will use $||x - y ||_p$ to denote the $\ell_p$ distance between $x$ and $y$. We will also let $\ell_q$ denote the dual norm to $\ell_p$ - that is, $\frac{1}{q} + \frac{1}{p}= 1$.

 We use $B_p(x,r)$ to denote the closed $\ell_p$ ball with center $x$ and radius $r$. For any $S \subset \R^d$, we let $diam_p(S)$ denote its diameter: that is, $diam_p(S) = \sup_{x, y \in S} ||x - y||_p.$

\subsection{Standard and Robust Loss}

In classical statistical learning, the goal is to learn an accurate classifier, which is defined as follows:

\begin{defn}
Let $\D$ be a distribution over $\R^d \times \Y$, and let $f \in \Y^{\R^d}$ be a classifier. Then the \textbf{standard loss} of $f$ over $\D$, denoted $\L(f, \D)$, is the fraction of examples $(x,y) \sim \D$ for which $f$ is not accurate. Thus $$\L(f, \D) = P_{(x,y) \sim \D}[f(x) \neq y].$$
\end{defn}

Next, we define robustness, and the corresponding robust loss.

\begin{defn}
A classifier $f \in \Y^{\R^d}$ is said to be \textbf{robust} at $x$ with radius $r$ if $f(x) = f(x')$ for all $x' \in B_p(x,r)$. 
\end{defn}


\begin{defn}
The \textbf{robust loss} of $f$ over $\D$, denoted $\L_r(f, \D)$, is the fraction of examples $(x,y) \sim \D$ for which $f$ is either inaccurate at $(x,y)$, or $f$ is not robust at $(x,y)$ with radius $r$. Observe that this occurs if and only if there is some $x' \in B_p(x,r)$ such that $f(x') \neq y$. Thus $$\L_r(f, \D) = P_{(x, y) \sim \D}[\exists x' \in B_p(x,r)\text{ s.t. }f(x') \neq y].$$ 
\end{defn}

\subsection{Expected Loss and  Sample Complexity}

The most common way to characterize the performance of a learning algorithm is through an $(\epsilon, \delta)$ guarantee, which computes $\epsilon_n, \delta_n$ such that an algorithm trained over $n$ samples has loss at most $\epsilon_n$ with probability at least $1 - \delta_n$. 

In this work, we use the simpler notion of \textit{expected loss}, which is defined as follows: 

\begin{defn}
Let $A$ be a learning algorithm and let $\D$ be a distribution over $\R^d \times \{\pm 1\}$. For any $S \sim \D^n$, we let $A_S$ denote the classifier learned by $A$ from training data $S$. Then the \textbf{expected standard loss} of $A$ with respect to $\D$, denoted $EL^n(A, \D)$ where $n$ is the number of training samples, is defined as $$E\L^n(A, \D) = \E_{S \sim \D^n} \L(A_S, \D).$$ Similarly, we define the \textbf{expected robust loss} of $A$ with respect to $\D$ as $$E\L_r^n(A , \D) = \E_{S \sim \D^n}\L_r(A_S, \D).$$ 
\end{defn}

Our main motivation for using this criteria is simplicity. Our primary goal is to compare and contrast the performances of algorithms in the standard and robust cases, and this contrast clearest when the performances are summarized as a single number (namely the expected loss) rather than an $(\epsilon, \delta)$ pair. 

Next, we address the notion of sample complexity. As above, sample complexity is typically defined as the minimum number of samples needed to guarantee $(\epsilon, \delta)$ performance. In this work, we will instead define it solely with respect to $\epsilon$, the expected loss. 

\begin{defn}
Let $\D$ be a distribution over $\R^d \times \{\pm 1\}$ and $A$ be a learning algorithm. Then the \textbf{standard sample complexity} of $A$ with respect to $\D$, denoted $m^\epsilon(A, \D)$, is the minimum number of training samples needed such that $A$ has  expected standard loss at most $\epsilon$. Formally, $$m^\epsilon(A, \D) = \min(\{n: E\L^n(A, D) \leq \epsilon\}).$$ Similarly, we can define the \textbf{robust sample complexity} as $$m_r^\epsilon(A, \D) = \min(\{n: E\L^n(A, D) \leq \epsilon\}).$$
\end{defn}

\subsection{Linear classifiers}

In this work, we consider linear classifiers, formally defined as follows:
\begin{defn}
Let $w \in \R^d$ be a vector. Then the \textbf{linear classifier} with parameters $w \in \R^d$ and $b \in \R$ over $\R^d \times {\pm 1}$, denoted $f_{w, b}$, is defined as , $$f_{w,b}(x) = \begin{cases} +1 & \langle w, x \rangle  \geq b \\ -1 & \langle w, x \rangle < b  \end{cases}.$$ 
\end{defn}

Learning linear classifiers is well understood in the standard classification setting. We now consider the linearly \textit{separable} case, in which some linear classifier has perfect accuracy. We will later define linear $r$-separability as the robust analog of separability.

\begin{defn}
A distribution $\D$ over $\R^d \times Y$ is \textbf{linearly separable} if its support can be partitioned into sets $S^+$ and $S^-$ such that:

1. $S^+$ and $S^-$ correspond to the positively and negatively labeled subsets of $\R^d$. In particular, $P_{(x,y) \sim \D}[x \in S^y] = 1.$

2. There exists a linear classifier, $f_{w, b}$, that has perfect accuracy. That is, $\L(f_{w, b}, \D) = 0$. 
\end{defn}

The standard sample complexity for linearly separable distributions can be characterized through their margin, which is defined as follows.

\begin{defn}\label{defn:margin}
Let $\D$ be a linearly separable distribution over $\R^d \times \{\pm 1\}$. Let $S^+$ and $S^-$ be as above. Then $\D$ has \textbf{margin} $\gamma$ if $\gamma$ is the largest real number such that there exists a linear classifier $f_{w,b}$ with the following properties:

1. $f_{w,b}$ has perfect accuracy. That is, $\L(f_{w,b}, \D) = 0$.

2. Let $H_{w,b} = \{x: \langle x, w \rangle = b\}$ denote the decision boundary of $f_{w,b}$. Then for all $x \in (S^+ \cup S^-)$, $x$ has $\ell_2$ distance at least $\gamma$ from $H_{w,b}$. That is, $$\inf_{x \in S^+ \cup S^-, z \in H_{w,b}} ||x - z||_2 \geq \gamma.$$ We let $\gamma(\D)$ denote the margin of $\D$.
\end{defn}

Observe that although we use a general norm, $\ell_p$, to measure robustness, the margin is always measured in $\ell_2$. This is because the $\ell_2$ norm plays a fundamental role in  bounding the number of samples needed to learn a linear classifier. 

The basic idea is that when the $\ell_2$ margin is large relative to the $\ell_2$ diameter of the distribution, the max margin algorithm requires fewer samples needed to learn a linear classifier. In particular, the ratio between the $\ell_2$ margin and the $\ell_2$ diameter fully characterizes the standard sample complexity of the max margin algorithm. To further simplify our notation, we define this ratio as the aspect ratio.

\begin{defn}\label{defn:aspect_ratio}
Let $\D$ be a linearly separable distribution over $\R^d \times \{\pm 1\}$. Then the \textbf{aspect ratio} of $\D$, $\rho(\D)$ is defined as, $$\rho(\D) = \frac{diam_2(S^+ \cup S^-)}{\gamma(\D)},$$ where $\diam_2(S^+ \cup S^-)$ denotes its diameter in the $\ell_2$ norm.
\end{defn}

We now have the following well-known result, which characterizes the expected standard loss with the aspect ratio.
\begin{thm}\label{thm:standard}
\emph{(Chapter 10 in \cite{vapnik1998})} Let $M$ denote the hard margin SVM algorithm. If $\D$ is a distribution with aspect ratio $\rho = \rho(\D)$, then for any $n > 0$ we have $\E_{S \sim \D^n}\L(M_S, \D) \leq O(\frac{\rho^2}{n}),$ where $M_S$ denotes the classifier learned by $M$ from training data $S$. 
\end{thm}

We can also express this result in terms of standard sample complexity.
\begin{cor}\label{cor:standard}
Let $M$ denote the hard margin SVM algorithm. If $\D$ is a distribution with aspect ratio $\rho = \rho(\D)$, then for any $\epsilon > 0$ we have $m^\epsilon(M_S, \D) \leq O(\frac{\rho^2}{\epsilon}),$ where $M_S$ denotes the classifier learned by $M$ from training data $S$. 
\end{cor}

Theorem \ref{thm:standard} and Corollary \ref{cor:standard} will serve as a benchmark for comparison with the robust sample complexity.  
\subsection{Linear $r$-separability}

Finally, we introduce linear $r$-separability, which is the key characteristic of distributions considered in this paper. This can be thought of as the robust analog of linear separability.
\begin{defn}\label{defn:r_separability}
For any $r > 0$, a distribution $\D$ over $\R^d \times \{\pm 1\}$ is \textbf{linearly} $r$-\textbf{separable} if there exists a linear classifier $f_{w, b}$ such that $\L_r(f_{w, b}, \D) = 0$.
\end{defn}
This definition is the fundamental property considered in this paper. Our goal is to understand the sample complexity required for learning robust linear classifiers on linearly $r$-separable distributions, and compare it with the standard sample complexity given in Theorem \ref{thm:standard}.

\section{Lower Bounds}\label{sec:lower_bounds}

In this section, we consider $r$-separated distributions whose aspect ratio is constant. By Theorem \ref{thm:standard}, the standard sample complexity for learning them is independent of $d$. We will show that in contrast, the robust sample complexity has a linear dependence on $d$, and consequently establish a substantial gap between the standard and robust cases.

We begin by defining the family of such distributions.
\begin{defn}
For any $\rho, r$, the set $\F_{r, \rho}$ is defined as the set of all distributions $\D$ over $\R^d \times \{\pm 1\}$ such that $\D$ is $r$-separated and has aspect ratio at most $\rho$.
\end{defn} 

We now state our main result.
\begin{thm}\label{thm:lower}
Let $r > 0$ and $\rho > 20$. Then the following hold.
\begin{enumerate}
	\item For every learning algorithm $A$, and any $n > 0$, there exists $\D \in \F_{r, \rho}$ such that the expected robust loss when $A$ is trained on a sample of size $n$ from $\D$ is at least $\Omega(\frac{d}{n})$. Formally, there exists a constant $c > 0$ such that $\E_{S \sim \D^n}[\L_r(A_S, \D)] \geq \frac{cd}{n}.$
	\item  In contrast, by Theorem \ref{thm:standard}, for \textit{any} $\D \in \F_{r, D}$, the max margin algorithm has expected standard loss $O(\frac{\rho^2}{n})$, when trained on a sample of size $n$ from $\D$. Formally, there exists a constant $c' > 0$ such that $\E_{S \sim \D^n}[\L(A_S, \D)] \leq \frac{c'\rho^2}{n}.$
\end{enumerate}
\end{thm}
The condition $\rho > 20$ is required to rule out degenerate cases. This is because for small values of $\rho$, the $\ell_2$ diameter of $\D$ is not much larger than the $\ell_2$ margin of $\D$. This forces $\D$ to be mostly clustered around a line which leads to more complicated behavior.

Observe that when $\rho$ is a constant independent of $d$, the expected  standard loss is $O(\frac{1}{n})$ while the expected robust loss is $\Omega(\frac{d}{n})$. Thus, the ratio between the expected robust loss and the expected standard loss is $\Omega(d)$, leading to a dimensional dependent gap between the robust and standard cases. 

We also note that these bounds hold regardless of which $\ell_p$ ($p \in (1, \infty])$ norm is being used. This is because our construction of $\D \in \F_{r, \rho}$ for which the lower bound holds is given in terms of the norm $p$. More generally, the family $\F_{r, \rho}$ is implicitly defined with respect to $p$.

Furthermore, our lower bound differs from the lower bound of $\Omega(\frac{d}{n})$ shown in prior work \cite{Cullina18} because it specifically holds for $\F_{r, \rho}$, a linearly $r$-separated family of distributions with constant aspect ratio. Thus, while \cite{Cullina18} has shown the existence of distributions satisfying the first condition of Theorem \ref{thm:lower}, our result is the first to exhibit a distribution satisfying both conditions.


Finally, we note that Theorem \ref{thm:lower} can also be expressed in terms of sample complexities. We include this in the following corollary.

\begin{cor}
Let $r > 0$ and $\rho > 20$. Then the following hold.

1. For every learning algorithm $A$, and any $\epsilon > 0$, there exists $\D \in \F_{r, \rho}$ such that the robust sample complexity of $A$ with respect to $\D$ is at least $\Omega(\frac{d}{\epsilon})$. Formally, there exists a constant $c > 0$ such that $m_r^\epsilon(A, \D) \geq \frac{cd}{\epsilon}.$

2. In contrast, by Theorem \ref{thm:standard}, for \textit{any} $\D \in \F_{r, D}$, the max margin algorithm has standard sample complexity $O(\frac{\rho^2}{\epsilon})$. Formally, there exists a constant $c' > 0$ such that $m^\epsilon(A, \D) \leq \frac{c'\rho^2}{\epsilon}.$

\end{cor}

\subsection{Comparison with \cite{ravikumar20} and~\cite{Schmidt18}}\label{sec:comparison}

The first work to provide a robust sample complexity lower bound that applied to linear classifiers is~\cite{Schmidt18}; they showed a gap of $\Omega(\sqrt{d})$ between the robust and accuracy loss for a specific mixture of two Gaussians. This was later generalized to mixtures of any two Gaussians by~\cite{ravikumar20}, who also established more general lower bounds for any $\ell_p$ norm. Since \cite{ravikumar20} is a strict generalization of \cite{Schmidt18}, we next explain how our lower bounds differ from~\cite{ravikumar20}, and why their techniques do not lead to our results. We begin by summarizing their results.

\paragraph{Summary of \cite{ravikumar20}} \cite{ravikumar20} considers data distributions $\D$ that are parametrized by $\mu \in \R^d$ and $\Sigma \in \R^{d \times d}$, $\Sigma \succcurlyeq 0$. $\D_{\mu, \Sigma}$ is the mixture of two Gaussians, $\N(\mu, \Sigma)$ and $\N(-\mu, \Sigma)$, with equal mass, where instances drawn from $\N(\mu, \Sigma)$ are labeled as $+$, and instances drawn from $\N(-\mu, \Sigma)$ are labeled as $-$. They consider robustness measured in any normed metric in $\R^d$, including the $\ell_p$ norm for $p \in (1, \infty]$. Although their bounds apply to any classifier, this effectively deals with linear classifiers since it can be shown that the optimally robust and accurate classifiers are both linear.

For any distribution $\D_{\mu, \Sigma}$, let $L_{rob}$ denote the optimal robust loss of any classifier on $\D_{\mu,\Sigma}$, and let $L_{std}$ denote the optimal standard loss. Then the bounds shown in \cite{ravikumar20} can restated as follows (a detailed derivation from \cite{ravikumar20} appears in Appendix \ref{sec:appendix_comparison}). 

\begin{thm}\label{thm:ravikumar}
\cite{ravikumar20}
\begin{enumerate}
	\item For any learning algorithm $A$ and any $n > 0$, there exists some mixture of Gaussians, $\D_{\mu, \Sigma}$ such that the expected \textit{excess} robust loss is at least $\Omega(L_{rob}\frac{d}{n}),$ when $A$ is trained on a sample of size $n$ from $\D$. 
	\item For any distribution $\D_{\mu, \Sigma}$, it is possible to learn a classifier with expected \textit{excess} standard loss at most $O(L_{std}\frac{d}{n})$.
	\item By (1.) and (2.), the ratio between the expected excess loss and expected excess standard loss can be expressed as $ratio \geq \Omega(\frac{L_{rob}}{L_{std}}).$
\end{enumerate}
\end{thm}

Observe that their bounds are given through \textit{excess} losses, which is the amount by which the loss exceeds to the optimal loss. This is necessary because in their setting, the optimal classifiers do not have $0$ loss. 

\paragraph{Comparison with our bounds} Recall that in our work, we are concerned with the \textit{linearly $r$-separated case}, which occurs precisely when the optimal robust and standard losses both equal $0$. However, from Theorem \ref{thm:ravikumar}, we see that although \cite{ravikumar20} proves a gap between standard and robust sample complexity, this gap is predicated on distributions for which the optimal robust loss, $L_{rob}$ and optimal standard loss, $L_{std}$ differ. Furthermore, in the case where they obtain a gap of $\Omega(d)$, we see that this requires $\frac{L_{rob}}{L_{std}} = \Omega(d)$ which is a substantial difference. By contrast, our results characterize a gap exclusively in the case that this does not occur. 

Finally, in the limiting case where the Gaussians they consider are sufficiently far apart, their data will begin to appear linearly $r$-separated, meaning both $L_{rob}$ and $L_{std}$ are close to $0$. However, even in this case, it can be shown that the ratio $\frac{L_{rob}}{L_{std}}$ diverges towards infinity, meaning that their lower bound characterizes a very different dynamic from ours. Precise details on this comparison can be found in appendix \ref{sec:appendix_comparison}.

\subsection{Intuition behind Theorem \ref{thm:lower}}

The proof idea for Theorem \ref{thm:lower} can be summarized with a simple example (Figure \ref{fig:small_margin_robust}). In this example, we seek to learn a linear classifier for a linearly $r$-separated distribution in $\R^2$. The key idea is to contrast the necessary conditions for learning a robust classifier, and the necessary conditions for learning an accurate classifier. 

Observe that the distribution is \textit{precisely} linearly $r$-separated, that is, it is not possible to achieve robustness for radii larger than $r$. Because of this, there is a unique linear classifier $f_{rob}$ that has perfect robustness. In order to learn this classifier, we must see examples from $S^+ \cup S^-$ that are close to the ``boundary" of $S^+ \cup S^-$. In our figure, this consists of points that are close to the dotted blue and red lines. Moreover, it can be shown that the number of such examples we must see is related to $d$, the dimension.

By contrast, any classifier that separates $S^+$ from $S^-$ has perfect accuracy (take for example $f_{std}$ shown in the figure). It is possible to exploit this by using margin based algorithms for learning linear classifiers. In particular, we no longer need to see points that are extremely close to the boundary of $S^+ \cup S^-$.

\begin{figure}[h]
\centering
\vspace{.3in}
\includegraphics[scale=0.55]{small_margin_robust}
\vspace{.3in}
\caption{An example of a linearly $r$-separated distribution, with positively and negatively labeled examples in $S^+$ and $S^-$ respectively. The optimally robust classifier, $f_{rob}$ is shown in purple, while the (not necessarily unique) optimally accurate classifier, $f_{std}$, is shown in green.}
\label{fig:small_margin_robust}
\end{figure}

\paragraph{General Hypothesis Classes:} We now briefly consider how to extend our methods to other hypothesis classes. For any hypothesis class $\H$ and distribution $\D$ let $$\H_{\D, \alpha} = \{h: h \in \H,\L(h, \D) \leq \alpha\}$$ and let $$\H_{\D, \alpha}^r = \{h: h \in \H,\L_r(h, \D) \leq \alpha\}.$$ $\H_{\D, \alpha}$ can be thought of as the set of accurate classifiers while $\H_{\D, \alpha}^r$ can be thought of as the set of astute classifiers. By their definitions, it is clear that $\H_{\D, \alpha}^r \subseteq \H_{\D, \alpha}$. However, in the case when $\H$ is the set of linear classifiers, we see that for small $\alpha$, $\H_{\D, \alpha}^r$ is a much ``smaller" set than $\H_{\D, \alpha}$. By exploiting the geometric structure inherent to $\H$, we can much more efficiently search for some $h \in \H_{\D, \alpha}$ than we can in $\H_{\D, \alpha}^r$. This dynamic is the crux of our lower bound: as we essentially show that there are far more critical points (i.e. points near the decision boundary) that we must see for learning $\H_{\D, \alpha}^r$ that aren't required for $\H_{\D, \alpha}$. 

Thus, for our methods to extend to an arbitrary hypothesis class, we would require a similar dynamic. We need two properties to hold: (1) $\H_{\D, \alpha}^r$ must be a very strict subset of $\H_{\D, \alpha}$ for sufficiently small alpha. (2) We must have some kind of exploitable geometric structure about $\H$ which allows us to exploit this gap. For the case of linear classifiers, this was the $\ell_2$ measured aspect ratio, $\gamma(\D)$. 

\paragraph{Kernel Classifiers: } A natural choice of a more general hypothesis class would be Kernel Classifiers, which are linear classifiers that operate in an embedded space, $H$. The main difficulty in expanding our lower bound to this more general setting comes from the behavior near the margin: the effects of the robustness radius in the embedded space are considerably less behaved than they are in the standard linear case. Nevertheless, we leave this as an important avenue for future work.

%\begin{algorithm}[H]
%   \caption{Adversarial-Perceptron}
%   \label{alg:upper_bound}
%\begin{algorithmic}[1]
%    \STATE \textbf{Input}:  $S = \{(x_1, y_1), \dots, (x_n, y_n)\} \sim \D^n,$
%    \STATE $w \leftarrow 0$ 
%    \FOR{$i = 1 \dots n$}
%    	\STATE $z = \argmin_{||z - x_i||_p \leq r}  y_i\langle w, z \rangle$ \COMMENT{\rbb{finds adv. ex.}}
%        \IF{$\langle w, y_iz \rangle \leq 0$ \COMMENT{\rbb{checks label}}}
%            \STATE $w \leftarrow w + y_iz$ \COMMENT{\rbb{perceptron update}}
%        \ENDIF           
%    \ENDFOR
%    \STATE return $f_{w, 0}$
%\end{algorithmic}
%\end{algorithm}

\begin{algorithm}[H]
    \SetAlgoLined
    {\bfseries Input:} $S = \{(x_1, y_1), \dots, (x_n, y_n)\} \sim \D^n,$\;
    
    $w \leftarrow 0$ \;
    
    \For{$i = 1 \dots n$}{
    	$z = \argmin_{||z - x_i||_p \leq r}  y_i\langle w, z \rangle$ {\color{red} finds adversarial example}\;
    	
    	\If{$\langle w, y_iz \rangle \leq 0$ {\color{red} checks label}}{
    	
	     	 $w \leftarrow w + y_iz$ {\color{red} perceptron update}\;  	
	     	 
    	}
    }
    
    {\bfseries Return:} $f_{w, 0}$\;
    

\caption{Adversarial-Perceptron}\label{alg:upper_bound}
\end{algorithm}

\section{Upper Bounds}\label{sec:upper_bound}

In the previous section, we showed that for any algorithm, there is some distribution $\D \in \F_{r, \rho}$ that is difficult (i.e. requires high sample complexity) to learn robustly. A natural follow-up question is: what about distributions for which the margin, $\gamma$ is very large compared to $r$. 

Observe that in Figure \ref{fig:small_margin_robust} the robustness radius $r$ is very close to the margin. In particular, we can find adversarial examples from $S^+$ and $S^-$ that are very close to the decision boundary $f_{rob}$. By contrast, if $\gamma >> r$, then this no longer holds which suggests that better robust sample complexities might be possible.

In this section, we will describe a subset of $\F_{r, \rho}$ that can be learned with expected loss $O(\frac{1}{n})$, thus matching the standard sample complexity up to a constant factor. To do so, we will introduce a novel concept: the \textit{robust margin}. The basic intuition is that distributions for which the margin greatly exceeds the robustness radius are precisely distributions with a large robust margin. We use the following notation.

Observe that if $\D$ is a linearly $r$-separated distribution, then $\D$ must also be linearly separable. As earlier, let $S^+, S^- \subset \R^d$ denote the positively and negatively labeled examples from $\D$. We now define \begin{equation}\label{eqn:s_plus_s_minus} S_r^+ = \cup_{s \in S^+} B_p(s, r)\text{ and }S_r^{-} = \cup_{s \in S^-} B_p(s,r).\end{equation} It follows that the decision boundary of any linear classifier with perfect robustness over $\D$ must separate $S_r^+$ and $S_r^-$. We now define the robust margin as a measurement of this separation.

\begin{defn}\label{def:robust_margin}
Let $\D$ be a linearly $r$-separable distribution over $\R^d \times \{\pm 1\}$. Let $S_r^+$ and $S_r^-$ be as above. Then $\D$ has \textbf{robust margin} $\gamma_r$ if $\gamma_r$ is the largest real number such that there exists a linear classifier $f_{w,b}$ with the following properties: 

1. $f_{w,b}$ has perfect astuteness. That is, $\L_r(f_{w,b}, \D) = 0$. 

2. Let $H_{w,b} = \{x: \langle x, w \rangle = b\}$ denote the decision boundary of $f_{w,b}$. Then for all $x \in (S_r^+ \cup S_r^-)$, $x$ has $\ell_2$ distance at least $\gamma$ from $H_{w,b}$. That is, $$\inf_{x \in S_r^+ \cup S_r^-} \inf_{z \in H_{w,b}} ||x - z||_2 \geq \gamma.$$ We let $\gamma_r(\D)$ denote the margin of $\D$, and say that such a distribution is $r, \gamma_r$-separated. 
\end{defn}

It is crucial to note that although adversarial perturbations are measured in $\ell_p$, the robust margin is measured in $\ell_2$. This is because while the metric $\ell_p$ plays a role in constructing $B(x,r)$, it can be completely disregarded once the sets $S_r^+$ and $S_r^-$ are considered, as any hyperplane separating $S_r^+$ and $S_r^-$ will have perfect robustness.  

We now define the robust aspect ratio, which is the robust analog of standard aspect ratio.

\begin{defn}
Let $\D$ be a distribution over $\R^d \times \{\pm 1\}$. Then the \textbf{robust aspect ratio} of $\D$, $\rho_r(\D)$ is defined as $$\rho_r(\D) = \frac{diam_2(S_r^+ \cup S_r^-)}{\gamma_r(\D)},$$ where as before, $diam_2(S_r^+ \cup S_r^-)$ denotes its diameter in the $\ell_2$ norm.
\end{defn}

We will now show that just as the aspect ratio, $\rho(\D)$, characterized the sample complexity for standard classification, the robust aspect ratio, $\rho_r(\D)$ will characterize the sample complexity for robust learning. To do so, we present a perceptron-inspired algorithm (Algorithm \ref{alg:upper_bound}) for learning a robust classifier on $r$-separated data with robust aspect ratio $\rho_r$. 

The basic idea behind Algorithm \ref{alg:upper_bound} is to combine the standard perceptron algorithm with adversarial training. In particular, we iterate through the training set and do the following on each point (refer to Algorithm \ref{alg:upper_bound} for precise details). 

1. Find an adversarial example $(z, y_i)$ by attacking our classifier, $f_{w, 0}$, at $(x_i, y_i)$ (line 4). This is a straightforward convex optimization problem for linear classifiers.

2. If $f_{w,0}(z) \neq y_i$, we update our weight vector with $(z, y_i)$ by using the standard perceptron update (lines 5-6).

We have the following upper bound on the expected robust loss of our algorithm.

\begin{thm}\label{thm:upper_bound}
Let $\D$ be a distribution with robust aspect ratio $\rho_r(\D)$. Then for any $n > 0$, we have $$\E_{S \sim \D^n} [\L_r(A_S, \D)] \leq O(\frac{\rho_r(\D)^2}{n}),$$ where $A_S$ denotes the classifier learned by Algorithm \ref{alg:upper_bound} from training data $S$. 
\end{thm}

Observe that this expected loss is still larger than the expected standard loss in  Theorem \ref{thm:standard} as $\rho_r(\D) > \rho(\D)$ for any $\D$. We also note that this result is not contradictory with our lower bound; there exist distributions $\D \in \F_{r, \rho}$ such that $\gamma_r(\D) = 0$, and these are precisely the distributions for which our lower bounds hold.

\subsection{Generalization to Kernel Classifiers}

Algorithm \ref{alg:upper_bound} can be thought of as the robust analog to the perceptron algorithm. We now generalize this algorithm to obtain a robust variant of the \textit{kernel perceptron algorithm}. We first briefly review kernel classifiers. A detailed explanation of our generalized algorithm along with requisite background material can be found in Appendix \ref{sec:kernel_appendix}

\begin{defn}\label{defn:kernel}
Let $K: \R^d \times \R^d \to \R$ be a kernel similarity function, $T = \{(x_1, y_1), \dots, (x_m, y_m)\} \subset \R^d \times \{\pm 1\}$ be a set of labeled points, and $\alpha \in \R^m$ be a vector of $m$ real numbers. Then the \textbf{kernel classifier} with similarity function $K$, parameters $T, \alpha$, and denoted by $f_{T, K}^\alpha$ is defined as $$f_{T, \alpha}^K(x) = \begin{cases} +1 &  \sum_1^m \alpha_iy_iK(x_i, x) \geq 0\\ -1 &  \sum_1^m \alpha_iy_iK(x_i, x) < 0  \end{cases}.$$
\end{defn}

Conceptually, kernel classifiers are linear classifiers operating in embedded space. With each kernel similarity function $K$, there is a map $\phi: \R^d \to H$ (where $H$ is some Hilbert space) such that $K(x, x') = \langle \phi(x), \phi(x') \rangle$. Thus we can think of kernel classifiers as having a linear decision boundary in $H$. 

We now present an analog of Algorithm \ref{alg:upper_bound} that we call the Adversarial Kernel-Perceptron. The essence of this algorithm has not changed. For each $(x_t, y_t)$ in our training set, we do the following.

1. Find an adversarial example $(z, y_i)$ by attacking our classifier, $f_{T, \alpha}^K$, at $(x_i, y_i)$ (line 4). 

2. If $f_{T, \alpha}^K(z) \neq y_i$, we update our weight vector with $(z, y_i)$ by appending $(z, y_i)$ to $T$ lines (5-6). This corresponds to a kernel-perceptron update that uses $(z, y_i)$ instead of $(x_i, y_i)$.

\begin{algorithm}[H]

\textbf{Input}:  $S = \{(x_1, y_1), \dots, (x_n, y_n)\} \sim \D^n,$ Similarity function, $K$

$T \leftarrow \emptyset$, $\alpha \leftarrow 0$

\For{$i = 1 \dots n$}{
    	$z = \argmin_{||z - x||_p \leq r}  y_if_{T, \alpha}^K(z)$ {\color{red}finds adv. ex.}
    	
        \If{$f_{T, \alpha}^k(z) \leq 0$ {\color{red} checks label}}{
        
            $T = T \cup \{(z, y_i)\}$ {\color{red} kern. percep. update}
            
            $\alpha = (1, \dots, 1)_{|T|}$
            
         }
}
        
Return $f_{T, \alpha}^K$
   \caption{Adversarial-Kernel-Perceptron}
   \label{alg:upper_bound_kernel}
\end{algorithm}

One challenging aspect of this algorithm is minimizing $f_{T, \alpha}^k(z)$. For linear classifiers, this has a closed form solution that utilizes the dual norm. For arbitrary Kernel classifiers, this is a somewhat more challenging problem. However, we note that this can be solved using standard optimization techniques, and in some cases (when $K$ is particularly simple), it can be solved with basic gradient descent.

Finally, we show that this Algorithm has similar performance to the linear case. Instead of using the robust aspect ratio, $\rho_r(\D)$, to bound the performance, we will require the \textbf{robust $K$-aspect ratio}, which is the kernel analog of this quantity. It can be thought of as the robust aspect ratio in the embedded space $H$. Details about this quantity (along with the proof of the theorem) can be found in Appendix \ref{sec:kernel_appendix}.

\begin{thm}\label{thm:upper_bound_kernel}
Let $\D$ be a distribution with robust $K$-aspect ratio $\rho_r^K(\D)$. Then for any $n > 0$, we have $$\E_{S \sim \D^n} [\L_r(A_S, \D)] \leq O(\frac{\rho_r^K(\D)^2}{n}),$$ where $A_S$ denotes the classifier learned by Algorithm \ref{alg:upper_bound_kernel} from training data $S$. 
\end{thm}

This result indicates that for small values of $\rho_r^k(\D)$, we can achieve a very good robust sample complexity for kernel classifiers. However, as the size of the perturbations approach this margin, this quantity goes to infinity. This phenomenon mirrors the linearly separable case, and suggests that a similar overall dynamic holds for kernel classification. We leave finding a full generalization (including our lower bound) for a direction in future work.






%\newcommand{\crop}[1]{\mathrm{crop}({#1})}
\newcommand{\object}[1]{\mathrm{object}({#1})}
\newcommand{\ba}{A_i}
\newcommand{\bb}{B_i}
\newcommand{\calA}{\mathcal{A}}
\newcommand{\calB}{\mathcal{B}}
\newcommand{\calX}{\mathcal{X}}
\newcommand{\masked}[1]{\mathrm{masked}({#1})}
\newcommand{\bx}{\mathbf{x}}
\newcommand{\SSL}{\textsc{SSL}}
\newcommand{\SSLbb}{\SSL^\mathrm{back}}
\newcommand{\SSLpj}{\SSL^\mathrm{proj}}
\newcommand{\CLF}{\textsc{CLF}}
\newcommand{\CLFbb}{\CLF^\mathrm{back}}
\newcommand{\CLFpj}{\CLF^\mathrm{proj}}
\newcommand{\SUP}{\textsc{SUP}}
\newcommand{\KNN}{\textsc{KNN}}
\newcommand{\KNNset}{\textsc{KNN}^\mathrm{set}}
\newcommand{\KNNprob}{\textsc{KNN}^\mathrm{prob}}
\newcommand{\KNNcl}{\textsc{KNN}^\mathrm{cl}}
\newcommand{\KNNconf}{\textsc{KNN}^\mathrm{conf}}
\newcommand{\RCDM}{\textsc{RCDM}}
\newcommand{\cl}{\mathrm{cl}}
\newcommand{\clpred}{\tilde{\mathrm{cl}}}
\newcommand{\Abox}{\overline{\calA}}
\newcommand{\Bbox}{\overline{\calB}}
\newcommand{\dejavu}{\emph{déjà vu }}
\newcommand{\Dejavu}{\emph{Déjà vu }}

\newcommand{\citations}{{\color{green}[CITE]}}

\definecolor{part_blue}{rgb}{0.2824, 0.4706, .8157}
\definecolor{part_red}{rgb}{0.8392, 0.3725, 0.3725}
\definecolor{part_orange}{rgb}{0.9333, 0.5216, 0.2902}

\DeclareRobustCommand{\mybox}[2][gray!20]{%
\begin{tcolorbox}[   %% Adjust the following parameters at will.
        % breakable,
        left=0pt,
        right=0pt,
        top=0pt,
        bottom=0pt,
        colback=#1,
        colframe=#1,
        width=\dimexpr\columnwidth\relax, 
        % width=\textwidth, 
        enlarge left by=0mm,
        boxsep=5pt,
        arc=0pt,outer arc=0pt,
        ]
        #2
\end{tcolorbox}
}
%\section{Introduction}
\label{sec:intro}
Self-supervised learning (SSL)~\citep{chen2020simclr, chen2020simsiam, zbontar2021barlow, vicreg, caron2020swav, MAE} aims to learn general representations of content-rich data without explicit labels by solving a \textit{pretext task}. In many recent works, such pretext tasks rely on joint-embedding architectures whereby randomized image augmentations are applied to create multiple views of a training sample, and the model is trained to produce similar representations for those views. When using cropping as random image augmentation, the model learns to associate objects or parts (including the background scenery) that co-occur in an image.
However, doing so also arguably exposes the training data to higher privacy risk as objects in training images can be explicitly memorized by the SSL model. For example, if the training data contains the photos of individuals, the SSL model may learn to associate the face of a person with their activity or physical location in the photo. This may allow an adversary to extract such information from the trained model for targeted individuals.

\begin{figure}[t]
    \centering
    \includegraphics[width=1.0\columnwidth]{figures/new_black_swan.pdf}
    \caption{\textbf{Left:} Reconstruction of an SSL training image from a crop containing only the background. The SSL model memorizes the association of this \emph{specific} patch of water (pink square) to this \emph{specific} foreground object (a black swan) in its embedding, which we decode to visualize the full training image. \textbf{Right:} The reconstruction technique fails on a public test image that the SSL model has not seen before.}
    \label{fig:black_swan}
\end{figure}

In this work, we aim to evaluate to what extent SSL models memorize the association of specific objects in training images or the association of objects and their specific backgrounds, and whether this memorization signal can be used to reconstruct the model's training samples. Our results demonstrate that SSL models memorize such associations beyond simple correlation. For instance, in Figure \ref{fig:black_swan} (\textbf{left}), we use the SSL representation of a \emph{training image crop containing only water} and this enables us to reconstruct the object in the foreground with remarkable specificity---in this case a black swan.
By contrast, in Figure \ref{fig:black_swan} (\textbf{right}), when using the \emph{crop from the background of a test set image} that the SSL model \emph{has not seen before}, its representation only contains enough information to infer, through correlation, that the foreground object was likely some kind of waterbird --- but not the specific one in the image.

Figure \ref{fig:black_swan} shows that SSL models suffer from the unintended memorization of images in their training data---a phenomenon we refer to as \emph{déjà vu memorization}
%\footnote{The French loanword \emph{déjà vu} means already-seen, which reflects the type of unintended memorization of objects that the SSL model saw during training.}.
\footnote{The French loanword \emph{déjà vu} means `already-seen', just as an image is seen and memorized in training.}
Beyond visualizing \emph{déjà vu} memorization through data reconstruction, we also design a series of experiments to quantify the degree of memorization for different SSL algorithms, model architectures, training set size, \emph{etc.} We observe that \emph{déjà vu} memorization is exacerbated by the atypically large number of training epochs often recommended in SSL training, as well as certain hyperparameters in the SSL training objective. Perhaps surprisingly, we show that \emph{déjà vu} memorization occurs even when the training set is large---as large as half of ImageNet~\citep{imagenet}---and can continually worsen even when standard techniques for evaluating learned representation quality (such as linear probing) do not suggest increased overfitting. Our work serves as the first systematic study of unintended memorization in SSL models and motivates future work on understanding and preventing this behavior. Specifically, we: 
\begin{itemize}
    \vspace{-0.5em}
    \item Elucidate how SSL representations memorize aspects of individual training images, what we call \emph{déjà vu} memorization;
    \item Design a novel training data reconstruction pipeline for non-generative vision models. This is in contrast to many prominent reconstruction algorithms like \citep{carlini2021extracting, google_diffusion}, which rely on the model itself to generate its own memorized samples and is not possible for SSL models or classifiers;
    \item Propose metrics to quantify the degree of \dejavu memorization committed by an SSL model. This allows us to observe how \dejavu changes with training epochs, dataset size, training criteria, model architecture and more. 
\end{itemize}

%\section{Preliminaries and Related Work}
\label{sec:related}

\textbf{Self-supervised learning} (SSL) is a machine learning paradigm that leverages unlabeled data to learn representations. Many SSL algorithms rely on \emph{joint-embedding} architectures (\emph{e.g.}, SimCLR~\citep{chen2020simclr}, Barlow Twins~\citep{zbontar2021barlow}, VICReg~\citep{vicreg} and Dino~\citep{Dino}), which are trained to associate different augmented views of a given image. For example, in SimCLR, given a set of images $\calA = \{A_1,\ldots,A_n\}$ and a randomized augmentation function $\mathrm{aug}$, the model is trained to maximize the cosine similarity of draws of $\SSL(\mathrm{aug}(A_i))$ with each other and minimize their similarity with $\SSL(\mathrm{aug}(A_j))$ for $i \neq j$. The augmentation function $\mathrm{aug}$ typically consists of operations such as cropping, horizontal flipping, and color transformations to create different views that preserve an image's semantic properties. 

\paragraph{SSL representations.} Once an SSL model is trained, its learned representation can be transferred to different downstream tasks. This is often done by extracting the representation of an image from the \emph{backbone model}\footnote{SSL methods often use a trick called \emph{guillotine regularization}~\citep{Guillotine}, which decomposes the model into two parts: a \emph{backbone model} and a \emph{projector} consisting of a few fully-connected layers. Such trick is needed to handle the misalignment between the pretext SSL task and the downstream task.} and either training a linear probe on top of this representation or finetuning the backbone model with a task-specific head~\citep{Guillotine}.
%Compared to representations learned by supervised learning, SSL representations are often more robust and transferable~\citep{hendrycks2019using, ericsson2021self}, leading to state-of-the-art result on many downstream tasks. To understand the effectiveness of SSL algorithms, several prior works investigated what kind of information the SSL model has learned~\citep{jing2021understanding, ericsson2021self, kalibhat2022towards, RCDM}. In particular, \citet{RCDM} trained a conditional generative model on SSL representations and showed that they encode richer visual details about the input image compared to supervised learning. 
%However, from a privacy perspective, this may be a cause for concern as the model also has more potential to overfit and memorize precise details about the training data compared to supervised learning. We show concretely that this privacy risk can indeed be realized by defining and measuring \emph{déjà vu} memorization.
It has been shown that SSL representations encode richer visual details about input images than supervised models do \cite{RCDM}. However, from a privacy perspective, this may be a cause for concern as the model also has more potential to overfit and memorize precise details about the training data compared to supervised learning. We show concretely that this privacy risk can indeed be realized by defining and measuring \emph{déjà vu} memorization.
\vspace{-0.5em} 
% \paragraph{Privacy risks in ML.} Overfitting in ML occurs when a model memorizes information specific to its training data rather than general population-level information. When the model is trained on privacy-sensitive data, overfitting is especially harmful as an adversary can infer private information about the training data when given access to the model~\citep{yeom2018privacy, feldman2020does}. The simplest and most well-studied form of privacy risk in ML is susceptibility to \emph{membership inference attacks}~\citep{shokri2017membership, salem2018ml, sablayrolles2019white}, where the adversary infers whether an individual is part of the training set or not. More sophisticated privacy attacks include \emph{attribute inference}~\citep{fredrikson2014privacy, mehnaz2022your, jayaraman2022attribute}, where specific attributes about an individual are inferred given others, and \emph{data reconstruction}~\citep{carlini2021extracting, balle2022reconstructing, guo2022bounding}, where entire training samples are recovered from the trained model. Our study of \emph{déjà vu} memorization is similar to both attribute inference and data reconstruction, leveraging SSL representations of the training image background to infer and reconstruct the foreground object.
% \vspace{-0.5em} 
% \paragraph{Training data extraction in NLP.} Our study of \dejavu memorization in SSL models is inspired by similar work in the natural language processing (NLP) domain. \citet{carlini2019secret} first showed that language models exhibit unintended memorization, where given a context string present in its training data, the model can generate the remaining text at test time. This unintended memorization has been further exploited in \citet{carlini2021extracting} to extract training data from GPT-2~\citep{radford2019language} and, more recently, extended to extract memorized images from Stable Diffusion \citep{google_diffusion}. The way by which these works exploit unintended memorization is similar to ours: given partial information about a training sample, the model is prompted to reveal the rest of the sample. In our case, however, since the SSL model is not generative, extraction is significantly harder and requires careful design.

\paragraph{Privacy risks in ML.} When a model is overfit on privacy-sensitive data, it memorizes specific information about its training examples, allowing an adversary with access to the model to learn private information~\citep{yeom2018privacy, feldman2020does}. Privacy attacks in ML range from the simplest and best-studied \emph{membership inference attacks}~\citep{shokri2017membership, salem2018ml, sablayrolles2019white} to \emph{attribute inference}~\citep{fredrikson2014privacy, mehnaz2022your, jayaraman2022attribute} and \emph{data reconstruction}~\citep{carlini2021extracting, balle2022reconstructing, guo2022bounding} attacks. In the former, the adversary only infers whether an individual participated in the training set. Our study of \emph{déjà vu} memorization is most similar to the latter: we leverage SSL representations of the training image background to infer and reconstruct the foreground object. Our approach reflects similar work in the NLP domain \citep{carlini2019secret, carlini2021extracting}: when prompted with a context string present in the training data, a large language model is shown to generate the remainder of string at test time, revealing sensitive text like home addresses. This method was recently extended to extract memorized images from Stable Diffusion \citep{google_diffusion}.  We exploit memorization in a similar manner: given partial information about a training sample, the model is prompted to reveal the rest of the sample. In our case, however, since the SSL model is not generative, extraction is significantly harder and requires careful design.

%\section{Defining \emph{Déjà Vu} Memorization}
\label{sec:definition}

\paragraph{What is \dejavu memorization?} At a high level, the objective of SSL is to learn general representations of objects that occur in nature. This is often accomplished by associating different parts of an image with one another in the learned embedding. Returning to our example in Figure \ref{fig:black_swan}, given an image whose background contains a patch of water, the model may learn that the foreground object is a water animal such as duck, pelican, otter, \emph{etc.}, by observing different images that contain water from the training set. We refer to this type of learning as \emph{correlation}: the association of objects that tend to co-occur in images from the training data distribution.

A natural question to ask is \emph{``Can the reconstruction of the black swan in Figure \ref{fig:black_swan} be reasoned as correlation?''} The intuitive answer may be no, since the reconstructed image is qualitatively very similar to the original image. However, this reasoning implicitly assumes that for a random image from the training data distribution containing a patch of water, the foreground object is unlikely to be a black swan. Mathematically, if we denote by $\mathcal{P}$ the training data distribution and $A$ the image, then
\begin{equation*}
\label{eq:p_corr}
p_\text{corr} := \mathbb{P}_{A \sim \mathcal{P}}(\mathrm{object}(A) = \texttt{black swan} ~|~ \mathrm{crop}(A) = \texttt{water})
\end{equation*}
is the probability of inferring that the foreground object is a black swan through \emph{correlation}. This probability may be naturally high due to biases in the distribution $\mathcal{P}$, \emph{e.g.}, if $\mathcal{P}$ contains no other water animal except for black swans. In fact, such correlations are often exploited to learn a model for image inpainting with great success~\citep{yu2018generative, ulyanov2018deep}.

Despite this, we argue that reconstruction of the black swan in Figure \ref{fig:black_swan} is \emph{not} due to correlation, but rather due to \emph{unintended memorization}: the association of objects unique to a single training image. As we will show in the following sections, the example in Figure \ref{fig:black_swan} is not a rare success case and can be replicated across many training samples. More importantly, failure to reconstruct the foreground object in Figure \ref{fig:black_swan} (\textbf{right}) on test images hints at inferring through correlation is unlikely to succeed---a fact that we verify quantitatively in Section \ref{sec:label inference accuracy}. Motivated by this discussion, we give a verbal definition of \dejavu memorization below, and design a testing methodology to quantify \dejavu memorization in Section \ref{sec:notation and setup}.
\mybox{\textbf{Definition:} A model exhibits \emph{déjà vu memorization} when it retains information so specific to an individual training image, that it enables recovery of aspects particular to that image given a part that does not contain them.
The recovered aspect must be beyond what can be inferred using only correlations in the data distribution.} 

% \textbf{Definition:} A model exhibits \emph{déjà vu memorization} when it retains information so specific to an individual training image, that it enables recovery of aspects particular to that image given a part that does not contain them.
% The recovered aspect must be beyond what can be inferred using only correlations in the data distribution.


 We intentionally kept the above definition broad enough to encompass different types of information that can be inferred about the training image, including but not restricted to object category, shape, color and position. For example, if one can infer that the foreground object is red given the background patch with accuracy significantly beyond correlation, we consider this an instance of \dejavu memorization as well. We mainly focus on object category to quantify \dejavu memorization in Section \ref{sec:quant} since the ground truth label can be easily obtained. We consider other types of information more qualitatively in the visual reconstruction experiments in Section \ref{sec:visualizing}.

\paragraph{Privacy implications of \dejavu memorization.} \Dejavu memorization can be a cause for concern when the training data contains privacy-sensitive information. As a motivating example, consider an SSL model trained on photos of individuals. If the model exhibits \dejavu memorization then, given the face of an individual, it may be possible to infer where the individual was or even visually reconstruct their location in the training image. Such information leakage raises privacy concerns, especially if there was no prior agreement that the trained model may reveal such information to third parties. This hypothetical scenario serves as a motivation that \dejavu memorization should be carefully examined to avoid unintended disclosure of private information in practical applications.

% \begin{figure*}[h]
%     \centering
%     \includegraphics[width = 0.85\textwidth]{figures/SSL_attack_cartoon.png}
%     \caption{We measure memorization by comparing the `target model' trained on the target image ($\SSL_A$ trained on $A_i$ in above example) with the `reference model' not trained on it ($\SSL_B$, above). \textbf{[Top Strip]} A cropping of the image disjoint from the labeled foreground object is embedded using the target model. This embedding is then labeled by a K-Nearest Neighbor (KNN) adversary built on a public set of labeled images, $X$, which it has also embedded using the target model. \textbf{[Bottom Strip]} To account for correlation, the same procedure is followed with the reference model. If the label is only extracted using the target model, it is counted as memorization. If it is extracted using either model, it is counted as correlation. We find that the KNN adversary's predictions using the target model (trained on attacked examples) are significantly more accurate than they are using the reference model, indicating routine memorization of training examples.}
%     \label{fig:ssl attack cartoon}
% \end{figure*}

\begin{figure}[t]
%%%
%SPIDER
%%%
     % \centering
     % \begin{subfigure}[b]{0.25\textwidth}
     %     \centering
     %     \includegraphics[width=\textwidth]{figures/data_split.png}
     %     % \caption{SimCLR correlated \textit{yellow garden spider} examples}
     %     \label{fig:data split}
     % \end{subfigure}
     % \hfill
     % \begin{subfigure}[b]{0.7\textwidth}
     %     \centering
     %     \includegraphics[width=\textwidth]{figures/pipeline_cartoon.png}
     %     \begin{minipage}{5cm}
     %        \vfill
     %    \end{minipage}
     %     % \caption{SimCLR memorized \textit{yellow garden spider} examples}
     %     \label{fig:pipeline cartoon}
     % \end{subfigure}
     \includegraphics[width=\textwidth]{figures/split_and_pipeline_cartoon.png}
\caption[Overview of testing methodology.]{
Overview of testing methodology. \textbf{Left:} Data is split into \emph{target set} $\calA$, \emph{reference set} $\calB$ and \emph{public set} $\calX$ that are pairwise disjoint. $\calA$ and $\calB$ are used to train two SSL models $\SSL_A$ and $\SSL_B$ in the same manner. $\calX$ is used for KNN decoding or for training an RCDM to reconstruct the input at test time. \textbf{Right:} Given a training image $A_i \in \calA$, we use $\SSL_A$ to embed $\crop{A_i}$ containing only the background, as well as the entire set $\calX$ and find the $k$-nearest neighbors of $\crop{A_i}$ in $\calX$ in the embedding space. These KNN samples can be used directly to infer the foreground object (\emph{i.e.}, class label) in $A_i$ using a KNN classifier, or their embeddings can be averaged as input to the trained RCDM to visually reconstruct the image $A_i$. For instance, the RCDM reconstruction results in Figure \ref{fig:black_swan} (left) when given $\SSL_A(\crop{A_i})$ and results in Figure \ref{fig:black_swan} (right) when given $\SSL_A(\crop{B_i})$ for an image $B_i \in \calB$.
%\textbf{Left:} illustration of the three datasets used in our tests. Two private data sets, $A$ and $B$, of equal size are used to train two SSL models, $\SSL_A$ and $\SSL_B$, respectively. A disjoint public set, $X$, is made available to the memorization test to help decode model embeddings. Memorization is only tested on examples $A_i \in A$ that are unique to set $A$. \textbf{Right:} illustration of inference pipeline used in tests. A periphery cropping that excludes the foreground object is taken from private image $A_i$. The KNN then finds the $k$ public set nearest neighbors of the periphery crop in the embedding space of $\SSL_A$. 
%The $\SSL_A$ representation of these $k$ neighbors and of the crop are used by the conditional generative model, RCDM, to reconstruct the foreground object. The labels of these $k$ neighbors are used to recover the foreground object label. (Not pictured) We repeat this process using reference model $\SSL_B$, not trained on image $A_i$, to determine whether the foreground object is still recoverable by learned correlations, e.g. if black swans were the only objects appearing near water in the data distribution. In this instance, the crop's public set neighbors in $\SSL_B$'s representation space include a variety of water animals like ducks, pelicans, and otters. Meanwhile, with $\SSL_A$, the neighbors are nearly all black swans in the same position as the swan of $A_i$.
}
\label{fig:split_and_pipeline_cartoon}
\end{figure}

\textbf{Distinguishing memorization from correlation.} When measuring \dejavu memorization, it is crucial to differentiate what the model associates through \emph{memorization} and what it associates through \emph{correlation}. Our testing methodology is based on the following intuitive definition.
\mybox{\textbf{Definition:} If an SSL model associates two parts in a training image, we say that it is due to \emph{correlation} if other SSL models trained on a similar dataset from $\mathcal{P}$ without this image would likely make the same association. Otherwise, we say that it is due to \emph{memorization}.}

Notably, such intuition forms the basis for differential privacy (DP; \cite{dwork2006calibrating, dwork2013algorithmic})---the most widely accepted notion of privacy in ML.

\subsection{Testing Methodology for Measuring \emph{Déjà Vu} Memorization}
\label{sec:notation and setup}

In this section, we use the above intuition to measure the extent of \dejavu memorization in SSL. Figure \ref{fig:split_and_pipeline_cartoon} gives an overview of our testing methodology.
\vspace{-0.75em}
\paragraph{Dataset splitting.} We focus on testing \dejavu memorization for SSL models trained on the ImageNet-1K dataset~\citep{imagenet}. Our test first splits the ImageNet training set into three independent and disjoint subsets $\calA$, $\calB$ and $\calX$. The dataset $\calA$ is called the \emph{target set} and $\calB$ is called the \emph{reference set}. The two datasets are used to train two separate SSL models, $\SSL_A$ and $\SSL_B$, called the \emph{target model} and the \emph{reference model}. Finally, the dataset set $\calX$ is used as an auxiliary public dataset to extract information from $\SSL_A$ and $\SSL_B$.
%\footnote{See Appendix \ref{sec:appx splits} for details on how the dataset splits are generated.}.
Our dataset splitting serves the purpose of distinguishing memorization from correlation in the following manner. Given a sample $A_i \in \calA$, if our test returns the same result on $\SSL_A$ and $\SSL_B$ then it is likely due to correlation because $A_i$ is not a training sample for $\SSL_B$. Otherwise, because $\calA$ and $\calB$ are drawn from the same underlying distribution, our test must have inferred some information unique to $A_i$ due to memorization. Thus, by comparing the difference in the test results for $\SSL_A$ and $\SSL_B$, we can measure the degree of \dejavu memorization\footnote{See Appendix \ref{sec:appx splits} for details on how the dataset splits are generated.}.
\vspace{-0.75em}
\paragraph{Extracting foreground and background crops.} Our testing methodology aims at measuring what can be inferred about the foreground object in an ImageNet sample given a background crop. This is made possible because ImageNet provides bounding box annotations for a subset of its training images---around 150K out of 1.3M samples. We split these annotated images equally between $\calA$ and $\calB$. Given an annotated image $A_i$, we treat everything inside the bounding box as the foreground object associated with the image label, denoted $\object{A_i}$. We take the largest possible crop that does not intersect with any bounding box as the background crop (or \emph{periphery crop}), denoted $\crop{A_i}$\footnote{We also present another heuristic in \cref{sec:appx corner crop} which takes a corner crop as the background crop, allowing our test to be run without bounding box annotations.}
%Since the labeled object tends to be at the image's center, the corner crop usually excludes it. }
%Because most images in ImageNet are object centric, an image's corner would not include the foreground object.}.
\vspace{-0.75em}
\paragraph{KNN-based test design.} Joint-embedding SSL approaches encourage the embeddings of random crops of a training image $A_i \in \calA$ to be similar. Intuitively, if the model exhibits \dejavu memorization, it is reasonable to expect that the embedding of $\crop{A_i}$ is similar to that of $\object{A_i}$ since both crops are from the same training image. In other words, $\SSL_A(\crop{A_i})$ encodes information about $\object{A_i}$ that cannot be inferred through correlation. However, decoding such information is challenging as these approaches do not learn a decoder associated with the encoder $\SSL_A$.

Here, we leverage the public set $\calX$ to decode the information contained in $\crop{A_i}$ about $\object{A_i}$. More specifically, we map images in $\calX$ to their embeddings using $\SSL_A$ and extract the $k$-nearest-neighbor (KNN) subset of $\SSL_A(\crop{A_i})$ in $\calX$. We can then decode the information contained in $\crop{A_i}$ in one of two ways:
\begin{itemize}
\item \emph{Label inference:} Since $\calX$ is a subset of ImageNet, each embedding in the KNN subset is associated with a class label. If $\crop{A_i}$ encodes information about the foreground object, its embedding will be close to samples in $\calX$ that have the same class label (\emph{i.e.}, foreground object category). We can then use a KNN classifier to infer the foreground object in $A_i$ given $\crop{A_i}$.
\item \emph{Visual reconstruction:} Following \citet{RCDM}, we train an RCDM---a conditional generative model---on $\calX$ to decode $\SSL_A$ embeddings into images. The RCDM reconstruction can recover qualitative aspects of an image remarkably well, such as recovering object color or spatial orientation using its SSL embedding. Given the KNN subset, we average their SSL embeddings and use the trained RCDM model to visually reconstruct $A_i$.
\end{itemize}
In Section \ref{sec:quant}, we focus on quantitatively measuring \dejavu memorization with label inference, and then use the RCDM reconstruction to visualize \dejavu memorization in Section \ref{sec:visualizing}.
%\section{Quantifying \emph{Déjà Vu} Memorization}
\label{sec:quant}

We apply our testing methodology to quantify a specific form of \dejavu memorization: inferring the foreground object (class label) given a crop of the background.

% \paragraph{Extracting model embeddings.} We test \dejavu memorization on two popular SSL algorithms, SimCLR~\citep{chen2020simclr} and VICReg~\citep{vicreg}.
% %\footnote{We present additional SSL models in \cref{sec:appx simclr results}} 
% As described in Section \ref{sec:related}, these algorithms produce two embeddings given an input image: a \emph{backbone} embedding and a \emph{projector} embedding that is derived by applying a small fully-connected network on top of the backbone embedding. Unless otherwise noted, all SSL embeddings refer to the projector embedding.
% To understand whether \dejavu memorization is particular to SSL, we also evaluate embeddings produced by a supervised model $\CLF_A$ trained on $\calA$. We apply the same set of image augmentations as those used in SSL and train $\CLF_A$ using the cross-entropy loss to predict ground truth labels. 
\vspace{-0.75em}
\paragraph{Extracting model embeddings.} We test \dejavu memorization on a variety of popular SSL algorithms, with a focus on VICReg~\citep{vicreg}. These algorithms produce two embeddings given an input image: a \emph{backbone} embedding and a \emph{projector} embedding that is derived by applying a small fully-connected network on top of the backbone embedding. Unless otherwise noted, all SSL embeddings refer to the projector embedding. 
To understand whether \dejavu memorization is particular to SSL, we also evaluate embeddings produced by a supervised model $\CLF_A$ trained on $\calA$. We apply the same set of image augmentations as those used in SSL and train $\CLF_A$ using the cross-entropy loss to predict ground truth labels. 
\vspace{-0.75em}
\paragraph{Identifying the most memorized samples.} Prior works have shown that certain training samples can be identified as more prone to memorization than others~\citep{feldman2020does, watson2021importance, ye2021enhanced}. Similarly, we provide a heuristic to identify the most memorized samples in our label inference test using confidence of the KNN prediction.
Given a periphery crop, $\crop{A_i}$, let $\KNN_A \big( \crop{A_i} \big) \subseteq \calX$ denote its $k$-nearest neighbors in the embedding space of $\SSL_A$. From this KNN subset we can obtain: \textbf{(1)} $\KNNprob_A \big( \crop{A_i} \big)$, the vector of class probabilities (normalized counts) induced by the KNN subset, and \textbf{(2)} $\KNNconf_A \big( \crop{A_i} \big)$, the negative entropy of the probability vector $\KNNprob_A \big( \crop{A_i} \big)$, as confidence of the KNN prediction. When entropy is low, the neighbors agree on the class of $A_i$ and hence confidence is high. 
% \begin{itemize}[noitemsep, leftmargin=*, topsep=0pt]
%     \item $\KNN_A \big( \crop{A_i} \big)$: The most prevalent class in the KNN subset as prediction for the class label $\cl(A_i)$. 
%     \item $\KNNprob_A \big( \crop{A_i} \big)$: The vector of class probabilities (normalized counts) induced by the KNN subset.
%     \item $\KNNconf_A \big( \crop{A_i} \big)$: Negative entropy of the probability vector $\KNNprob_A \big( \crop{A_i} \big)$ as confidence of the KNN prediction. When entropy is low, the neighbors agree on the class of $A_i$ and hence confidence is high. 
% \end{itemize}
We can sort the confidence score $\KNNconf_A \big( \crop{A_i} \big)$ across samples $A_i$ in decreasing order to identify the most confidently predicted samples, which likely correspond to the most memorized samples when $A_i \in \calA$.

\subsection{Population-level Memorization}
\label{sec:label inference accuracy}

%ORIGINAL FIGURE SETUP IN ARXIV: 
% \input{dejavu_training_epochs.tex}
% \input{dejavu_training_set_size.tex}
%PUT ORIGINAL FIGURES SIDE BY SIDE: 
% \input{dejavu_training_epochs_set_size.tex}
%PUT IN NEW FIGURES: 

\begin{wrapfigure}{r}{0.4\textwidth} 
    \centering
    \includegraphics[width=0.4\textwidth]{figures/dejavu_main.pdf}
    \caption{Accuracy of label inference using the target model (trained on $\calA$) vs. the reference model (trained on $\calB$) on the top $\%$ most confident examples $A_i \in \calA$ using only $\crop{A_i}$. For VICReg, there is a large accuracy gap between the two models, indicating a significant degree of \dejavu memorization.}
    \label{fig:dejavu main}
    \vspace{-2ex}
\end{wrapfigure}

Our first measure of \dejavu memorization is population-level label inference accuracy: \emph{What is the average label inference accuracy over a subset of SSL training images given their periphery crops?} 
To understand how much of this accuracy is due to $\SSL_A$'s \dejavu memorization, we compare with a correlation baseline using the reference model: $\KNN_B$'s label inference accuracy on images $A_i \in \calA$. 
In principle, this inference accuracy should be significantly above chance level ($1/1000$ for ImageNet) because the periphery crop may be highly indicative of the foreground object through correlation, \emph{e.g.}, if the periphery crop is a basketball player then the foreground object is likely a basketball.

Figure \ref{fig:dejavu main} compares the accuracy of $\KNN_A$ to that of $\KNN_B$ when inferring the labels of images in $A_i \in \calA$\footnote{The sets $\calA$ and $\calB$ are exchangeable, and in practice we repeat this test on images from $\calB$ using $\SSL_B$ as the target model and $\SSL_A$ as the reference model, and average the two sets of results.} using $\crop{A_i}$.
Results are shown for VICReg and the supervised model; trends for other models are shown in Appendix \ref{sec:appx simclr results}. For both VICReg and supervised models, inferring the class of $\crop{A_i}$ using $\KNN_B$ (dashed line) through correlation achieves a reasonable accuracy that is significantly above chance level. However, for VICReg, the inference accuracy using $\KNN_A$ (solid red line) is significantly higher, and the accuracy gap between $\KNN_A$ and $\KNN_B$ indicates the degree of \dejavu memorization. We highlight two observations: 
\begin{itemize}
    \item The accuracy gap of VICReg is significantly larger than that of the supervised model. This is especially notable when accounting for the fact that the supervised model is trained to associate randomly augmented crops of images with their ground truth labels. In contrast, VICReg has no label access during training but the embedding of a periphery crop can still encode the image label. 
    \item For VICReg, inference accuracy on the $1\%$ most confident examples is nearly $95\%$, which shows that our simple confidence heuristic can effectively identify the most memorized samples. This result suggests that an adversary can use this heuristic to identify vulnerable training samples to launch a more focused privacy attack.
\end{itemize}
\vspace{-.75em}
\paragraph{The \dejavu score. }
The curves of Figure \ref{fig:dejavu main} show memorization across confidence values for a single training scenario.  To study how memorization changes with different hyperparamters, we extract a single value from these curves: the \dejavu \emph{score} at confidence level $p$. In Figure \ref{fig:dejavu main}, this is the gap between the solid red (or gray) and dashed red (or gray) where confidence ($x$-axis) equal $p\%$. In other words, given the periphery crops of set $\calA$, $\KNN_A$ and $\KNN_B$ separately select and label their top $p\%$ most confident examples, and we report the difference in their accuracy. The \dejavu score captures both the degree of memorization by the accuracy gap and the \emph{ability to identify memorized examples} by the confidence level. If the score is 10\% for $p=33\%$, $\KNN_A$ has 10\% higher accuracy on its most confident third of $\calA$ than $\KNN_B$ does on its most confident third. In the following, we set $p = 20\%$, approximately the largest gap for VICReg (red lines) in Figure \ref{fig:dejavu main}. 
% Specifically, the \dejavu \emph{score} on the top $p\%$ most confident examples is,  
% \begin{equation}
%     \mathrm{DejaVu}(p) = \mathrm{Acc}_{\SSL_A}\big( \calA_{\SSL_A, p}  \big) - \mathrm{Acc}_{\SSL_B}\big( \calA_{\SSL_B, p}  \big) \ ,
%     \label{eqn:dejavu score}
% \end{equation}
% where $\calA_{\SSL_A, p}$
% Here we introduce a DejaVu memorization metric that quantify how much a target model is able to retrieve more class information from a crop than the reference model. We define it as:
% where $p$ is a function that take the $p$ purcent most confident samples.
%Figure \ref{fig:dejavu v. training epochs} shows how \dejavu memorization changes with the number of epochs used to train the embedding model (VICReg and supervised, respectively). The training set size is fixed to 300K samples, and label inference accuracy is computed on the top $20\%$ highest confidence examples. The number of epochs has a very strong influence on the degree of memorization for VICReg as the accuracy gap widens when number of epochs increases. We note that 1000 training epochs is used in several SSL works \citep{vicreg, simclr}. Remarkably, this trend in memorization is \emph{not} reflected in the standard metric for evaluating SSL representations: linear probe accuracy. The gray line in Figure \ref{fig:dejavu v. training epochs} shows the train-test accuracy gap of a linear classifier trained on top of the VICReg embeddings. Although there is a sizeable train-test gap, it does not grow significantly beyond 500 epochs. In contrast, \dejavu memorization (blue line) continues to worsen after 500 epochs. Thus, our test can be used as an alternative to linear probe accuracy to evaluate the memorization of SSL models.
% \vspace{-.75em}

% \paragraph{Comparison with the generalization gap} A network that perform very well on a training set while performing poorly on a test set (assuming the training set and test set sampled uniformly from the same distribution) is probably memorizing the training examples without being able to generalize on the test data. One could expect that measuring the difference in accuracy between the training and test set could give us insights on the degree of \dejavu memorization. However, we show in Figure  \ref{fig:dejavu v. training epochs} and \ref{fig:dejavu v. n} that this is not the case. In fact \dejavu memorization can significantly increase while the train-test gap decrease. In our experiments, we did not find a correlation between \dejavu and generalization.
\vspace{-0.75em}
\paragraph{Comparison with the linear probe train-test gap.} A standard method for measuring SSL performance is to train a linear classifier---what we call a `linear probe'---on its embeddings and compute its performance on a held out test set. From a learning theory standpoint, one might expect the linear probe's train-test accuracy gap to be indicative of memorization: the more a model overfits, the larger is the difference between train set and test set accuracy. However, as seen in Figure \ref{fig:dejavu epochs train set size}, the linear probe gap (dark blue) fails to reveal memorization captured by the \dejavu score (red) \footnote{See section \ref{sec:mitigation} for further discussion of the \dejavu score trends of Figure \ref{fig:dejavu epochs train set size}.}.

% \paragraph{Effect of training epochs.} 
% Figure \ref{fig:dejavu v. training epochs} shows how \dejavu memorization changes with training epochs for VICReg. The training set size is fixed to 300K samples. We observe that the number of epochs has a very strong influence on the degree of memorization for VICReg. From 250 to 1000 epochs, the \dejavu score (red curve) grows threefold: from under 10\% to over 30\%. Remarkably, this trend in memorization is \emph{not} reflected in the standard metric for evaluating SSL representations: linear probe accuracy. The dark blue curve shows the train-test linear probe accuracy gap. Although there is a sizeable train-test gap, it only changes by a few percent beyond 250 epochs. %Thus, our test can be used as an alternative to linear probe accuracy to evaluate the memorization of SSL models.
% \vspace{-.75em}
\begin{figure}[ht]
\label{fig:dejavu epochs and dataset}
\begin{minipage}[t]{0.49\textwidth}
\centering
     \begin{subfigure}[b]{0.48\textwidth}
         \centering
         \includegraphics[width=\textwidth]{figures/deja_vu_vs_epochs.png}
         \vspace{-1.5em}
         \caption{\dejavu vs. epochs}
         \label{fig:dejavu v. training epochs}
     \end{subfigure}
     \begin{subfigure}[b]{0.48\textwidth}
         \centering
         \includegraphics[width=\textwidth]{figures/deja_vu_vs_n.png}
         \vspace{-1.5em}
         \caption{\dejavu vs. train set size}
         \label{fig:dejavu v. n}
     \end{subfigure}~
     \vspace{-0.5em}
    \caption{
    Effect of training epochs and train set size with VICReg on \dejavu score (red) in comparison with linear probe accuracy train-test gap (dark blue). 
    \textbf{Left:} \dejavu score increases with training epochs, indicating growing memorization while the linear probe baseline decreases significantly.  
    \textbf{Right:} \dejavu score stays roughly constant with training set size suggesting that memorization may be problematic even for large datasets. %By comparison, the baseline \emph{declines} by half, spuriously suggesting less memorization. 
    %Both trends are not captured according to the linear probe train-test gap---a common method to evaluate generalization of SSL representations.}
    }
    \label{fig:dejavu epochs train set size}
\end{minipage}
\hfill
\begin{minipage}[t]{0.49\textwidth}
\centering
     \begin{subfigure}[b]{0.48\textwidth}
         \centering
         \includegraphics[width=\textwidth]{figures/vicreg_samples_epochs.pdf}
         \vspace{-1.5em}
         \caption{\dejavu vs. epochs}
         \label{fig:per sample v. training epochs}
     \end{subfigure}
     \begin{subfigure}[b]{0.48\textwidth}
         \centering
         \includegraphics[width=\textwidth]{figures/vicreg_samples_datasets.pdf}
         \vspace{-1.5em}
         \caption{\dejavu vs. train set size}
         \label{fig:per sample v. n}
     \end{subfigure}~
     \vspace{-0.5em}
    \caption{
    \definecolor{part_blue}{rgb}{0.2824, 0.4706, .8157}
	\definecolor{part_red}{rgb}{0.8392, 0.3725, 0.3725}
	\definecolor{part_orange}{rgb}{0.9333, 0.5216, 0.2902}
    Partition of samples $A_i \in \calA$ into the four categories: {\color{gray}unassociated} (not shown), {\color{part_orange}memorized}, {\color{part_red}misrepresented} and {\color{part_blue}correlated} for VICReg. The {\color{part_orange}memorized} samples---those whose labels are predicted by $\KNN_A$ but not by $\KNN_B$---occupy a significantly larger share of the training set than the {\color{part_red}misrepresented} samples---those predicted by $\KNN_B$ but not $\KNN_A$ by chance. %At 1000 epochs, $\approx 15\%$ of the training set is {\color{part_orange}memorized}. The trends across training epochs and training set sizes are consistent with those observed in Figure \ref{fig:dejavu epochs train set size}
    }
    \label{fig:partition attack main}
    \end{minipage}
\vspace{-1em} 
\end{figure}

\iffalse

\begin{minipage}[t]{0.49\textwidth}
\centering
     \begin{subfigure}[b]{0.48\textwidth}
         \centering
         \includegraphics[width=0.95\textwidth]{figures/deja_vu_vs_parameters.png}
         \vspace{-0.4em}
         \caption{\dejavu vs. capacity}
         \label{fig:dejavu v. capacity}
     \end{subfigure}
     \hfill
     \begin{subfigure}[b]{0.48\textwidth}
          \tiny
          \centering
          \setlength{\tabcolsep}{3pt}
          \begin{tabular}{|c|c|c|}
            \hline
            Criteria & DV & Acc P/B \\
            \hline
            Supervised & 8.9 & 55.3/61.1\\
            \hline
            Byol\citep{grill2020byol} & 8.0& 54.3/59.4\\
            \hline
            SimCLR\citep{chen2020simclr} & 10.0 & 44.2/54.1\\
            \hline
            Dino\citep{Dino} & 14.5 & 26.3/55.7 \\
            \hline
            Barlow T.\citep{zbontar2021barlow} & 30.5 & 33.7/54.4\\
            \hline
            VICReg\citep{vicreg} & \textbf{33.2} & 40.3/55.2\\
            \hline
          \end{tabular}
          \vspace{1.3em}
          % \caption{\dejavu (DV) vs. SSL Criterion}
          \caption{\dejavu (DV) vs. Criterion}
          \label{tab:dejavu vs. criterion}
    \end{subfigure}
    \vspace{-0.5em}
    \caption{
    Comparison of \dejavu score for different architectures and training criteria. \textbf{Left:} \dejavu score with VICReg for resnet (purple) and vision transformer (green) architectures versus number of model parameters. As expected, memorization grows with larger model capacity. This trend is more pronounced for convolutional (resnet) than transformer (ViT) architectures. \textbf{Right:} Comparison of \dejavu score and ImageNet validation accuracy (P: using projector embeddings, B: using backbone embeddings) for various SSL criteria. \textbf{Nearly all SSL models have more memorization than the supervised baseline.} 
    % Effect of training epochs and train set size on \dejavu score.
    % \textbf{Left:} \dejavu score increases with higher number of training epochs, indicating worsening memorization.
    % \textbf{Right:} \dejavu score stays roughly constant with training set size. Both trends are not captured according to the linear probe train-test gap---a common method to evaluate generalization of SSL representations.
    }
\end{minipage}
\vspace{-2em} 
\end{figure}

\begin{figure}[ht]
\begin{minipage}[t]{0.49\textwidth}
\centering
     \begin{subfigure}[b]{0.49\textwidth}
         \centering
         \includegraphics[width=\textwidth]{figures/epochs_lb_attk_epochs_acc_top1_legend.pdf}
         \caption{\dejavu vs. epochs}
         \label{fig:dejavu v. training epochs}
     \end{subfigure}
     \begin{subfigure}[b]{0.49\textwidth}
         \centering
         \includegraphics[width=\textwidth]{figures/epochs_lb_attk_datasets_acc_top1_legend.pdf}
         \caption{\dejavu vs. train set size}
         \label{fig:dejavu v. n}
     \end{subfigure}~
     \begin{subfigure}[b]{0.32\textwidth}
         \centering
         \includegraphics[width=0.8\textwidth]{figures/dejavu_vs_parameters.pdf}
         \caption{\dejavu vs. capacity}
         \label{fig:dejavu v. n}
     \end{subfigure}
    \caption{
    Effect of training epochs and train set size on \dejavu score.
    \textbf{Left:} \dejavu score increases with higher number of training epochs, indicating worsening memorization.
    \textbf{Right:} \dejavu score stays roughly constant with training set size. Both trends are not captured according to the linear probe train-test gap---a common method to evaluate generalization of SSL representations.}
    \end{minipage}
\vspace{-1em} 
\end{figure}

\begin{table}[ht]
  \footnotesize
  \centering
  \begin{tabular}{|c|c|}
    \hline
    Supervised & 8.9\\
    \hline
    SimCLR\citep{chen2020simclr} & 10.0\\
    \hline
    Byol\citep{grill2020byol} & 8.0\\
    \hline
    Dino\citep{Dino} & 14.5\\
    \hline
    Barlow T.\citep{zbontar2021barlow} & 30.5\\
    \hline
    VICReg\citep{vicreg} & \textbf{33.2}\\
    \hline
  \end{tabular}
  \caption{DejaVu Score 20\% Conf for various SSL methods.}
  \label{tab:two-row-table}
\end{table}
\vspace{-1em} 
\fi

\iffalse
\begin{figure}[ht]
\begin{minipage}[t]{.49\textwidth}
\centering
     \begin{subfigure}[b]{0.49\textwidth}
         \centering
         \includegraphics[width=\textwidth]{figures/epochs_lb_attk_epochs_acc_top1_legend.pdf}
         \caption{\dejavu vs. epochs}
         \label{fig:dejavu v. training epochs}
     \end{subfigure}
     \hfill
     \begin{subfigure}[b]{0.49\textwidth}
         \centering
         \includegraphics[width=\textwidth]{figures/epochs_lb_attk_datasets_acc_top1_legend.pdf}
         \caption{\dejavu vs. train set size}
         \label{fig:dejavu v. n}
     \end{subfigure}
\caption{
Effect of training epochs and train set size on \dejavu score.
\textbf{Left:} \dejavu score increases with higher number of training epochs, indicating worsening memorization.
\textbf{Right:} \dejavu score stays roughly constant with training set size. Both trends are not captured according to the linear probe train-test gap---a common method to evaluate generalization of SSL representations.}
\label{fig:dejavu epochs and dataset}
\end{minipage}
\hfill
\begin{minipage}[t]{.49\textwidth}
     \centering
     \begin{subfigure}[b]{0.49\textwidth}
         \centering
         \includegraphics[width=\textwidth]{figures/criteria_epochs.pdf}
         \caption{criteria comparison}
         \label{fig:dejavu v. criteria}
     \end{subfigure}
     \hfill
     \begin{subfigure}[b]{0.49\textwidth}
         \centering
         \includegraphics[width=\textwidth]{figures/architecture_epochs.pdf}
         \caption{architecture comparison}
         \label{fig:dejavu v. arch}
     \end{subfigure}
\caption{
Effect of SSL training criteria and model architectures on \dejavu score.
%the accuracy gap between target model (trained on $\calA$) and reference model (trained on $\calB$) making predictions on their 20\% most confident examples.
\textbf{Left:} \dejavu score for various training criteria.
%Barlow and VICReg have the heaviest degree of memorization, while SimCLR and BYOL have the least. 
%Note that we show detailed reconstructions of SimCLR's training data in Section \ref{sec:visualizing} despite its relatively low degree of \dejavu. 
%Regardless, Although SimCLR and BYOL have the least, we  visualize detailed reconstructions with SimCLR in section \ref{sec:mem v corr} 
All SSL models have significantly more \dejavu than the supervised baseline. \textbf{Right:} \dejavu score versus epochs for various training architectures. As expected, lower capacity architectures (Resnet18, Resnet34) reduce \dejavu but not completely. 
}
\label{fig:dejavu criteria and architecture}
\end{minipage}
\vspace{-1em} 
\end{figure}
\fi
% %\begin{figure}[ht]
%%%
%VICREG
%%%
     \centering
     \begin{subfigure}[b]{0.49\textwidth}
         \centering
         \includegraphics[width=\textwidth]{figures/sample_level_training_epochs.pdf}
         \caption{Categories of training samples vs. number of epochs}
         \label{fig:sample level epochs}
     \end{subfigure}
     \hfill
     \begin{subfigure}[b]{0.49\textwidth}
         \centering
         \includegraphics[width=\textwidth]{figures/sample_level_training_set_size.pdf}
         \caption{Categories of training samples vs. training set size}
         \label{fig:sample level training size}
     \end{subfigure}
\caption{
\definecolor{part_blue}{rgb}{0.2824, 0.4706, .8157}
\definecolor{part_red}{rgb}{0.8392, 0.3725, 0.3725}
\definecolor{part_orange}{rgb}{0.9333, 0.5216, 0.2902}
Partition of samples $A_i \in \calA$ into the four categories: {\color{gray}unassociated} (not shown), {\color{part_orange}memorized}, {\color{part_red}misrepresented} and {\color{part_blue}correlated}. The {\color{part_orange}memorized} samples---ones whose labels are predicted by $\KNN_A$ but not by $\KNN_B$---occupy a significantly larger share for VICReg compared to the supervised model, indicating that sample-level \dejavu memorization is more prevalent in VICReg. %The trends across number of training epochs and training set sizes are consistent with those observed in Figures \ref{fig:dejavu epochs and dataset} and \ref{fig:dejavu criteria and architecture}.
}
\label{fig:partition attack main appendix}
\end{figure}
% \paragraph{Effect of training set size.} 
% Figure \ref{fig:dejavu v. n} shows how \dejavu memorization responds to the model's training set size. The number of training epochs is fixed to 1000. Interestingly, training set size appears to have almost \emph{no} influence on the \dejavu score (red line), indicating that memorization is equally prevalent with a 100K dataset and a 500K dataset (which suggests that \dejavu memorization may be detectable for larger datasets). Meanwhile, the linear probe train-test accuracy gap \emph{declines} by half as the dataset size grows, failing to represent the memorization quantified by our test. 
% The trend is completely different according to linear probe accuracy (dark blue line), the train-test gap shrinks substantially when increasing the training set size from 100K to 500K. This highlights that the train-test gap is not able to capture \dejavu memorization. %Our evidence suggests that \dejavu memorization may be detectable even for large-scale training datasets. 
%\vspace{-.75em}

\vspace{-.75em} 
\subsection{Sample-level Memorization}
\label{sec:dissection}

% Section \ref{sec:label inference accuracy} shows the \emph{average} level of \dejavu memorization on a subset of the training set $\calA$. However, this average tell us only what the attacker success rate might be without explicitly describing how much of the datatset is \dejavu memorized.
The \dejavu score shows, \emph{on average}, how much better an adversary can select and classify images when using the target model trained on them. 
This average score does not tell us how many individual images have their label successfully recovered by $\KNN_A$ but not by $\KNN_B$. In other words, how many images are exposed by virtue of \emph{being in training set} $\calA$: a risk notion foundational to differential privacy. 
% However, from the perspective of an individual image $A_i \in \calA$, it is informative to know whether it was correctly classified 
To better quantify what fraction of the dataset is at risk, we perform a sample-level analysis by fixing a sample $A_i \in \calA$ and observing the label inference result of $\KNN_A$ vs. $\KNN_B$.
To this end, we partition samples $A_i \in \calA$ based on the result of label inference into four distinct categories: {\color{gray}\textbf{Unassociated}} - label inferred with neither KNN; {\color{part_orange}\textbf{Memorized}} - label inferred only with $\KNN_A$; {\color{part_red}\textbf{Misrepresented}} - label inferred only with $\KNN_B$; {\color{part_blue}\textbf{Correlated}} - label inferred with both KNNs. 
% \begin{multicols}{2}
% \begin{itemize}
%     \vspace{-.75em}
%     \setlength\itemsep{0.15em}
%     \item {\color{gray}Unassociated}: label inferred with neither KNN   
%     \item {\color{part_orange}Memorized}: label only inferred by $\KNN_A$
%     \item {\color{part_red}Misrepresented}: label only inferred with $\KNN_B$
%     \item {\color{part_blue}Correlated}: label inferred with both KNNs
%     \vspace{-.75em}
% \end{itemize}
% \end{multicols}
Intuitively, {\color{gray}unassociated} samples are ones where the embedding of $\crop{A_i}$ does not encode information about the label. {\color{part_blue}Correlated} samples are ones where the label can be inferred from $\crop{A_i}$ using correlation, \emph{e.g.}, inferring the foreground object is basketball given a crop showing a basketball player. Ideally, the {\color{part_red}misrepresented} set should be empty but contains a small portion of examples due to chance.
\emph{Déjà vu} memorization occurs for {\color{part_orange}memorized} samples where the embedding of $\SSL_B$ does not encode the label but the embedding of $\SSL_A$ does. To measure the pervasiveness of \dejavu memorization, we compare the size of the {\color{part_orange}memorized} and {\color{part_red}misrepresented} sets.
Figure \ref{fig:partition attack main} shows how the four categories of examples change with number of training epochs and training set size. The {\color{gray}unassociated} set is not shown since the total share adds up to one. The {\color{part_red}misrepresented} set remains under $5\%$ and roughly unchanged across all settings, consistent with our explanation that it is due to chance. In comparison, VICReg's {\color{part_orange}memorized} set surpasses $15\%$ at 1000 epochs. Considering that up to 5\% of these memorized examples could also be due to chance, we conclude that \textbf{at least 10\% of VICReg's training set is \dejavu memorized.} 
%is many times larger than its {\color{part_red}misrepresented} set, indicating substantial sample-level \dejavu memorization. 
%In fact, \textbf{it is 15\% of the training set that is \dejavu memorized with VICReg.}
%The trends across different number of training epochs and training set sizes match those observed in Section \ref{sec:label inference accuracy}. % On the other hand, the supervised model's {\color{part_orange}memorized} set is only marginally larger than its {\color{part_red}misrepresented} set.

% The trends across different number of training epochs and training set sizes match those observed in Section \ref{sec:label inference accuracy}: Increasing the number of epochs increases \dejavu memorization (Figure \ref{fig:per sample v. training epochs}), while increasing the training set size does not appear to reduce \dejavu memorization (Figure \ref{fig:per sample v. n}). 
%\section{Visualizing \emph{Déjà Vu} Memorization}
\label{sec:visualizing}
Beyond enabling label inference using a periphery crop, we show that \dejavu memorization allows the SSL model to encode other forms of information about a training image. Namely, we train an RCDM \citep{RCDM} on the public dataset $\calX$ and use it to visually reconstruct training images given their periphery crop.
We aim to answer the following two questions: \textbf{(1)} Can we visualize the distinction between correlation and \dejavu memorization? \textbf{(2)} What foreground object details can be extracted from the SSL model beyond class label? 
% \begin{enumerate}[noitemsep, leftmargin=*, topsep=0pt]
%     \item Can we visualize the distinction between correlation and \dejavu memorization? 
%     \item What foreground object details can be extracted from the SSL model beyond class label? 
% \end{enumerate}
\vspace{-0.5em}
\paragraph{Reconstruction pipeline.}
RCDM is a conditional generative model that is trained on the \emph{backbone embedding} of images $X_i \in \calX$ to generate an image that resembles $X_i$. All training images are first face-blurred for privacy purposes. \citet{RCDM} showed that the backbone embedding of SSL models contains more low-level information about the image, making them better suited for conditioning the RCDM.
At test time, following the pipeline in Figure \ref{fig:split_and_pipeline_cartoon}, we first use the projector embedding to find the KNN subset for the periphery crop, $\crop{A_i}$, and then average their backbone embeddings as input to the RCDM model. Ideally, when the public set contains enough representative images, the average representation of the KNN subset encodes objects present in $A_i$, and the RCDM model decodes this representation to visualize these objects.
% \begin{figure}[ht]
%%%
%VICREG
%%%
     \centering
     \begin{subfigure}[b]{0.49\textwidth}
         \centering
         \includegraphics[width=\textwidth]{figures/sample_level_training_epochs.pdf}
         \caption{Categories of training samples vs. number of epochs}
         \label{fig:sample level epochs}
     \end{subfigure}
     \hfill
     \begin{subfigure}[b]{0.49\textwidth}
         \centering
         \includegraphics[width=\textwidth]{figures/sample_level_training_set_size.pdf}
         \caption{Categories of training samples vs. training set size}
         \label{fig:sample level training size}
     \end{subfigure}
\caption{
\definecolor{part_blue}{rgb}{0.2824, 0.4706, .8157}
\definecolor{part_red}{rgb}{0.8392, 0.3725, 0.3725}
\definecolor{part_orange}{rgb}{0.9333, 0.5216, 0.2902}
Partition of samples $A_i \in \calA$ into the four categories: {\color{gray}unassociated} (not shown), {\color{part_orange}memorized}, {\color{part_red}misrepresented} and {\color{part_blue}correlated}. The {\color{part_orange}memorized} samples---ones whose labels are predicted by $\KNN_A$ but not by $\KNN_B$---occupy a significantly larger share for VICReg compared to the supervised model, indicating that sample-level \dejavu memorization is more prevalent in VICReg. %The trends across number of training epochs and training set sizes are consistent with those observed in Figures \ref{fig:dejavu epochs and dataset} and \ref{fig:dejavu criteria and architecture}.
}
\label{fig:partition attack main appendix}
\end{figure}
%\begin{figure*}[t!]
%%%
%DAM
%%%
     \centering
     \begin{subfigure}[b]{0.49\textwidth}
         \centering
         \includegraphics[width=\textwidth]{figures/dam_corr.png}
         \caption{A {\color{part_blue}correlated} dam example}
         \label{fig:dam correlated}
     \end{subfigure}
     \hfill
     \begin{subfigure}[b]{0.49\textwidth}
         \centering
         \includegraphics[width=\textwidth]{figures/dam_mem.png}
         \caption{A {\color{part_orange}memorized} dam example}
         \label{fig:dam memorized}
     \end{subfigure}
\caption{
{\color{part_blue}Correlated} and {\color{part_orange}Memorized} examples from the \emph{dam} class. Both $\SSL_A$ and $\SSL_B$ are SimCLR models.
\textbf{Left:} The periphery crop (pink square) contains a concrete structure that is often present in images of dams. Consequently, the trained RCDM can reconstruct the foreground object using representations from both $\SSL_A$ and $\SSL_B$ through this correlation.
\textbf{Right:} The periphery crop only contains a patch of water. The embedding produced by $\SSL_B$ only contains enough information to infer that the foreground object is related to water, as reflected by its KNN set and RCDM reconstruction. In contrast, the embedding produced by $\SSL_A$ memorizes the association of this patch of water with dam and the RCDM can visualize the embedding to produce images of dams.
}
\vspace{-1ex}
\label{fig:mem v corr dam}
\end{figure*}


\begin{figure*}[t!]
%%%
%DAM
%%%
     \centering
     \begin{subfigure}[b]{0.49\textwidth}
         \centering
         \includegraphics[width=\textwidth]{figures/dam_corr.png}
         \caption{A {\color{part_blue}correlated} dam example}
         \label{fig:dam correlated}
     \end{subfigure}
     \hfill
     \begin{subfigure}[b]{0.49\textwidth}
         \centering
         \includegraphics[width=\textwidth]{figures/dam_mem.png}
         \caption{A {\color{part_orange}memorized} dam example}
         \label{fig:dam memorized}
     \end{subfigure}
\caption[Correlated and Memorized examples from the \emph{dam} class.]{
Correlated and Memorized examples from the \emph{dam} class. Both $\SSL_A$ and $\SSL_B$ are SimCLR models.
\textbf{Left:} The periphery crop (pink square) contains a concrete structure that is often present in images of dams. Consequently, the trained RCDM can reconstruct the foreground object using representations from both $\SSL_A$ and $\SSL_B$ through this correlation.
\textbf{Right:} The periphery crop only contains a patch of water. The embedding produced by $\SSL_B$ only contains enough information to infer that the foreground object is related to water, as reflected by its KNN set and RCDM reconstruction. In contrast, the embedding produced by $\SSL_A$ memorizes the association of this patch of water with dam and the RCDM can visualize the embedding to produce images of dams.
}
\label{fig:mem v corr dam}
\end{figure*}


\begin{figure}[t!]
%%%
%BADGER
%%%
     \centering
     \begin{subfigure}[b]{0.49\textwidth}
         \centering
         \includegraphics[width=\textwidth]{figures/euro_badgers.png}
         \caption{{\color{part_orange}Memorized} European badgers}
         \label{fig:euro badgers}
     \end{subfigure}
     \hfill
     \begin{subfigure}[b]{0.49\textwidth}
         \centering
         \includegraphics[width=\textwidth]{figures/amer_badgers.png}
         \caption{{\color{part_orange}Memorized} American badgers}
         \label{fig:amer badgers}
     \end{subfigure}
\caption[Visualization of \dejavu memorization beyond class label.]{
Visualization of \dejavu memorization beyond class label. Both $\SSL_A$ and $\SSL_B$ are VICReg models. 
The four images shown belong to the memorized set of $\SSL_A$ from the \emph{badger} class. RCDM reconstruction using embeddings from $\SSL_A$ can reveal not only the correct class label, but also the specific badger species: \emph{European} (left) and \emph{American} (right). Such information does not appear to be memorized by the reference model $\SSL_B$.
} 
\label{fig:in class badger}
\end{figure}


% \subsection{Visualizing Correlation vs. Memorization}
\label{sec:mem v corr}
\vspace{-0.5em} 
\paragraph{Visualizing Correlation vs. Memorization.}
Figure \ref{fig:mem v corr dam} shows examples of dams from the {\color{part_blue}correlated} set (left) and the {\color{part_orange}memorized} set (right) as defined in Section \ref{sec:dissection}, along with the associated KNN set and RCDM reconstruction. Both $\SSL_A$ and $\SSL_B$ are SimCLR models. In Figure \ref{fig:dam correlated}, the periphery crop is represented by the pink square, which contains concrete structure attached to the dam's main structure. As a result, both $\SSL_A$ and $\SSL_B$ produce embeddings of $\crop{A_i}$ whose KNN set in $\calX$ consist of dams, \emph{i.e.}, there is a correlation between the concrete structure in $\crop{A_i}$ and the foreground dam. The RCDM reconstructions also consist of dams or structures that closely resemble dams. 
In Figure \ref{fig:dam memorized}, the periphery crop only contains a patch of water, which does not strongly correlate with dams in the ImageNet distribution. Evidently, the reference model $\SSL_B$ embeds $\crop{A_i}$ close to that of other objects commonly found in water, such as sea turtle and submarine. In contrast, the KNN set according to $\SSL_A$ all contain dams despite the vast number of alternative possibilities within the ImageNet classes, and the RCDM reconstruction outputs dams as well which highlight memorization in $\SSL_A$ between this specific patch of water and the dam. %\footnote{See Appendix \ref{sec:appx visualization} to see the same trend in the \emph{yellow garden spider} class.}


% \subsection{Visualizing Memorization Beyond Class Label}
% \label{sec:in class variation}
\vspace{-0.5em} 
\paragraph{Visualizing Memorization Beyond Class Label.}
We now use our reconstruction algorithm to show that \dejavu memorization can be exploited to reveal detailed information beyond class label. Figure \ref{fig:in class badger} shows four examples of badgers from the {\color{part_orange}memorized} set. In all four images, the periphery crop (pink square) does not contain any indication that the foreground object is a badger. Despite this, the KNN set and the RCDM reconstruction using $\SSL_A$ consistently produce images of badgers, while the same does not hold for $\SSL_B$.
More interestingly, reconstructions using $\SSL_A$ in Figure \ref{fig:euro badgers} all contain \emph{European} badgers, while reconstructions in Figure \ref{fig:amer badgers} all contain \emph{American} badgers, accurately reflecting the species of badger present in the respective training images. Since ImageNet-1K does \emph{not} differentiate between these two species of badgers, our reconstructions show that SSL models can memorize information that is highly specific to a training sample beyond its class label\footnote{See Appendix \ref{sec:appx visualization} for additional visualization experiments.}.%\footnote{See Appendix \ref{sec:appx visualization} for the same trend in the \emph{aircraft carrier} class.}.





%\vspace{-.5em} 
\section{Mitigation of \dejavu memorization}
\label{sec:mitigation}
% We do not have an understanding on why \dejavu occur so strongly in some SSL pretraining, however we present additional experiments that shed light on which parameters have the biggest impact on \dejavu memorization.
\begin{figure}[ht]
\label{fig:mitigations}
\begin{minipage}[t]{0.5\textwidth}
\centering
     \begin{subfigure}[b]{0.47\textwidth}
         \centering
         \includegraphics[width=\textwidth]{figures/dejavu_vicreg_param.png}
         \vspace{-1.5em}
         \caption{Loss hyper-parameter}
         \label{fig:dejavu v. invariance}
     \end{subfigure}
     \begin{subfigure}[b]{0.49\textwidth}
         \centering
         \includegraphics[width=\textwidth]{figures/deja_vu_vs_layer.png}
         \vspace{-1.5em}
         \caption{Guillotine regularization}
         \label{fig:dejavu v. guillotine}
     \end{subfigure}~
     \vspace{-0.5em}
    \caption[Effect of two kinds of hyper-parameters on VICReg memorization. ]{
    Effect of two kinds of hyper-parameters on VICReg memorization. \textbf{Left:} \dejavu score (red) versus the \emph{invariance} loss parameter, $\lambda$, used in the VICReg criterion (100k dataset). Larger $\lambda$ significantly reduces \dejavu, with minimal effect on linear probe validation performance (green). $\lambda = 25$ (near maximum \dejavu) is recommended in the original paper \textbf{Right:} \dejavu score versus projector layer---guillotine regularization \cite{Guillotine}---from projector to backbone. Removing the projector can significantly reduce \dejavu. Appendix \ref{sec:guillotine} shows that the backbone still can memorize, however; we demonstrate reconstructions using the SimCLR backbone.
    }
\end{minipage}
\hfill
\begin{minipage}[t]{0.48\textwidth}
\centering
     \begin{subfigure}[b]{0.46\textwidth}
         \centering
         \includegraphics[width=\textwidth]{figures/deja_vu_vs_parameters.png}
         \vspace{-1.3em}
         \caption{\dejavu vs. capacity}
         \label{fig:dejavu v. capacity}
     \end{subfigure}
     \hfill
     \begin{subfigure}[b]{0.52\textwidth}
          \tiny
          \centering
          \setlength{\tabcolsep}{3pt}
          \begin{tabular}{|c|c|c|}
            \hline
            Criteria & DV & Acc P/B \\
            \hline
            Supervised & 8.9 & 55.3/61.1\\
            \hline
            Byol\citep{grill2020byol} & 8.0& 54.3/59.4\\
            \hline
            SimCLR\citep{chen2020simclr} & 10.0 & 44.2/54.1\\
            \hline
            Dino\citep{Dino} & 14.5 & 26.3/55.7 \\
            \hline
            Barlow T.\citep{zbontar2021barlow} & 30.5 & 33.7/54.4\\
            \hline
            VICReg\citep{vicreg} & \textbf{33.2} & 40.3/55.2\\
            \hline
          \end{tabular}
          \vspace{1.3em}
          % \caption{\dejavu (DV) vs. SSL Criterion}
          \caption{\dejavu (DV) vs. Criterion}
          \label{tab:dejavu vs. criterion}
    \end{subfigure}
    \vspace{-1.4em}
    \caption[Effect of model architecture and criterion on \dejavu memorization.]{
    %Comparison of \dejavu score for different architectures and training criteria. 
    Effect of model architecture and criterion on \dejavu memorization. 
    \textbf{Left:} \dejavu score with VICReg for resnet (purple) and vision transformer (green) architectures versus number of model parameters. As expected, memorization grows with larger model capacity. This trend is more pronounced for convolutional (resnet) than transformer (ViT) architectures. \textbf{Right:} Comparison of \dejavu score 20\% conf. and ImageNet linear probe validation accuracy (P: using projector embeddings, B: using backbone embeddings) for various SSL criteria. %\textbf{Nearly all SSL models have more memorization than the supervised baseline.} 
    % Effect of training epochs and train set size on \dejavu score.
    % \textbf{Left:} \dejavu score increases with higher number of training epochs, indicating worsening memorization.
    % \textbf{Right:} \dejavu score stays roughly constant with training set size. Both trends are not captured according to the linear probe train-test gap---a common method to evaluate generalization of SSL representations.
    }
    \end{minipage}
\end{figure}
We cannot yet make claims on why \dejavu occurs so strongly for some SSL training settings and not for others. To gain some intuition for future work, we present additional observations that shed light on which parameters have the most salient impact on \dejavu memorization.
\vspace{-.75em}
\paragraph{Déjà vu memorization worsens by increasing number of training epochs.} 
Figure \ref{fig:dejavu v. training epochs} shows how \dejavu memorization changes with number of training epochs for VICReg. The training set size is fixed to 300K samples. From 250 to 1000 epochs, the \dejavu score (red curve) grows \emph{threefold}: from under 10\% to over 30\%. Remarkably, this trend in memorization is \emph{not} reflected by the linear probe gap (dark blue), which only changes by a few percent beyond 250 epochs. 

%\vspace{-.75em}
\paragraph{Training set size has minimal effect on \dejavu memorization.} Figure \ref{fig:dejavu v. n} shows how \dejavu memorization responds to the model's training set size. The number of training epochs is fixed to 1000. Interestingly, training set size appears to have almost \emph{no} influence on the \dejavu score (red line), indicating that memorization is equally prevalent with a 100K dataset and a 500K dataset. This result suggests that \dejavu memorization may be detectable even for large datasets. Meanwhile, the standard linear probe train-test accuracy gap \emph{declines} by more than half as the dataset size grows, failing to represent the memorization quantified by our test. 
% The trend is completely different according to linear probe accuracy (dark blue line), the train-test gap shrinks substantially when increasing the training set size from 100K to 500K. This highlights that the train-test gap is not able to capture \dejavu memorization. Our evidence suggests that \dejavu memorization may be detectable even for large-scale training datasets. 
\vspace{-0.5em}
\paragraph{Training loss hyper-parameter has a strong effect.} 
%We show in Figure \ref{fig:dejavu v. training epochs} that the number of training epochs is an important factor that can increase significantly \dejavu memorization. In contrast, the dataset size does not impact much \dejavu as shown in Figure \ref{fig:dejavu epochs train set size}. 
Loss hyper-parameters, like VICReg's invariance coefficient (Figure \ref{fig:dejavu v. invariance}) or SimCLR's temperature parameter (Appendix Figure \ref{fig:simclr temperature}) significantly impact \dejavu with minimal impact on the linear probe validation accuracy.

\vspace{-0.5em}
\paragraph{Some SSL criteria promote stronger \dejavu memorization.} Table \ref{tab:dejavu vs. criterion} demonstrates that the degree of memorization varies widely for different training criteria. VICReg and Barlow Twins have the highest \dejavu scores while SimCLR and Byol have the lowest.
%\footnote{We show detailed reconstructions of SimCLR's training data in Section \ref{sec:visualizing} despite its relatively low degree of \dejavu.}.
With the exception of Byol, all SSL models have more \dejavu memorization than the supervised model. Interestingly, different criteria can lead to similar linear probe validation accuracy and very different degrees of \dejavu as seen with SimCLR and Barlow Twins. Note that low degrees of \dejavu can still risk training image reconstruction, as exemplified by the SimCLR reconstructions in Figures \ref{fig:mem v corr dam} and \ref{fig:mem v corr spider}. 
%\vspace{-1em}
\vspace{-0.5em}
\paragraph{Larger models have increased \dejavu memorization.} Figure \ref{fig:dejavu v. capacity} validates the common intuition that lower capacity architectures (Resnet18/34) result in less memorization than their high capacity counterparts (Resnet50/101). 
% \begin{wrapfigure}{r}{0.25\textwidth} 
%     \centering
%     \includegraphics[width=0.25\textwidth]{figures/attk_layer_acc_top1_legend.pdf}
%     \caption{\dejavu memorization versus layer from backbone (0) to projector output (3).}
%     \label{fig:dejavu vs layer}
%     \vspace{-8ex}
% \end{wrapfigure}
We see the same trend for vision transformers as well. %This comes with a tradeoff, since reduced model capacity can result in a nontrivial degradation of representation quality\cite{vicreg, simclr}.  
\vspace{-0.5em}
\paragraph{Guillotine regularization can help reduce \dejavu memorization.} Previous experiments were done using the projector embedding. In Figure \ref{fig:dejavu v. guillotine}, we present how Guillotine regularization\citep{Guillotine} (removing final layers in a trained SSL model) impacts \dejavu with VICReg\footnote{Further experiments are available in Appendix \ref{sec:guillotine}.}. Using the backbone embedding instead of the projector embedding seems to be the most straightforward way to mitigate \dejavu memorization. However, as demonstrated in Appendix \ref{sec:appx backbone results}, backbone representation with low \dejavu score can still be leveraged to reconstruct some of the training images.

\section{Conclusion}
\label{sec:conclusion}

We defined and analyzed \dejavu memorization, a notion of unintended memorization of partial information in image data. As shown in Sections \ref{sec:quant} and \ref{sec:visualizing}, SSL models can largely exhibit \dejavu memorization on their training data, and this memorization signal can be extracted to infer or visualize image-specific information.
Since SSL models are becoming increasingly widespread as foundation models for image data, negative consequences of \dejavu memorization can have profound downstream impact and thus deserves further attention. 
Future work should focus on understanding how \dejavu emerges in the training of SSL models and why methods like Byol are much more robust to \dejavu than VICReg and Barlow Twins. In addition, trying to characterize which data points are the most at risk of \dejavu could be crucial to get a better understanding on this phenomenon. 
