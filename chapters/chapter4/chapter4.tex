\graphicspath{{./chapters/chapter3/}}
%\newtheorem{thm}{Theorem}
%\newtheorem{lem}[thm]{Lemma}
%\DeclareMathOperator*{\argmax}{arg\,max}
%\DeclareMathOperator*{\argmin}{arg\,min}

\newtheorem*{rep@theorem}{\rep@title}
\newcommand{\newreptheorem}[2]{%
\newenvironment{rep#1}[1]{%
 \def\rep@title{#2 \ref{##1}}%
 \begin{rep@theorem}}%
 {\end{rep@theorem}}}
\makeatother
%\newtheorem{thm}{Theorem}
\newreptheorem{thm}{Theorem}
%\newtheorem{lem}[thm]{Lemma}
\newreptheorem{lem}{Lemma}
\newtheorem{prop}[thm]{Proposition}
\newreptheorem{prop}{Proposition}
%\newtheorem{cor}[thm]{Corollary}
\newreptheorem{cor}{Corollary}
\newtheorem{claim}{Claim}

\newtheorem{innercustomgeneric}{\customgenericname}
\providecommand{\customgenericname}{}
\newcommand{\newcustomtheorem}[2]{%
  \newenvironment{#1}[1]
  {%
   \renewcommand\customgenericname{#2}%
   \renewcommand\theinnercustomgeneric{##1}%
   \innercustomgeneric
  }
  {\endinnercustomgeneric}
}

\newcustomtheorem{customthm}{Theorem}
\newcustomtheorem{customlemma}{Lemma}
\newcustomtheorem{customprop}{Proposition}

\newcommand{\defref}[1]{Definition~\ref{#1}}
\newcommand{\tabref}[1]{Table~\ref{#1}}
\newcommand{\figref}[1]{Fig.~\ref{#1}}
\newcommand{\eqnref}[1]{\text{Eq.}~(\ref{#1})}
\newcommand{\secref}[1]{\textnormal{Section}~\ref{#1}}
\newcommand{\appref}[1]{Appendix \ref{#1}}
\newcommand{\stepref}[1]{Step \ref{#1}}
%\newcommand{\appref}[1]{the Appendix} % for short version of the paper
\newcommand{\thmref}[1]{Theorem~\ref{#1}}
\newcommand{\corref}[1]{Corollary~\ref{#1}}
\newcommand{\propref}[1]{Proposition~\ref{#1}}
\newcommand{\lemref}[1]{Lemma~\ref{#1}}
\newcommand{\conref}[1]{Condition~\ref{#1}}
\newcommand{\assref}[1]{Assumption~\ref{#1}}
\newcommand{\algref}[1]{Algorithm~\ref{#1}}
\newcommand{\egref}[1]{Example~\ref{#1}}
\newcommand{\algoref}[1]{Algorithm~\ref{#1}}

\newcommand{\op}[1]{\operatorname{#1}}
\newcommand{\paren} [1] {\ensuremath{ \left( {#1} \right) }}
\newcommand{\parenb} [1] {\ensuremath{ \big( {#1} \big) }}
\newcommand{\bigparen} [1] {\ensuremath{ \Big( {#1} \Big) }}
\newcommand{\biggparen} [1] {\ensuremath{ \bigg( {#1} \bigg) }}
\newcommand{\Biggparen} [1] {\ensuremath{ \Bigg( {#1} \Bigg) }}
\newcommand{\bracket}[1]{\left[#1\right]}
\newcommand{\tuple}[1]{\ensuremath{\left\langle #1 \right\rangle}}
\newcommand{\set}[1]{\ensuremath{\left\{#1\right\}}}
\newcommand{\curlybracket}[1]{\ensuremath{\left\{#1\right\}}}
\newcommand{\norm}[2]{\ensuremath{\left\langle#1,\:#2\right\rangle_{\cH_{\rK}}}}
\newcommand{\normg}[2]{\ensuremath{\left\langle#1,\:#2\right\rangle}}
\newcommand{\condcurlybracket}[2]{\ensuremath{\left\{#1\left\lvert\:#2\right.\right\}}}
\newcommand{\inmod}[1]{\ensuremath{\left\lvert\left\lvert#1\right\rvert\right\rvert}}
\newcommand{\boldalpha}{\ensuremath{\boldsymbol{\alpha}}}

\def\ind{\mathbbm{1}}
\def\oc{Online\_Cluster}
\def\ocns{No\_Sub\_Cluster}
\def\mem{\mathcal M}
\def\E{\mathbb{E}}
\def\R{\mathbb{R}}
\def\OC{\text{OC}}
\def\cH{\mathcal H}
\def\reals{\mathbb{R}}
\def\cD{\mathcal D}
\def\Ev{\mathbb{E}}
\def\cO{\mathcal O}
\def\cX{\mathcal X}
\def\cU{\mathcal U}
\def\cM{\mathcal M}



\chapter{Robust Empirical Risk Minimization with Tolerance} 

\section{Introduction}

Adversarially robust classification is a staple of modern machine learning. In the robust setting, along with meeting standard accuracy guarantees, predictions made by a learner at test time must additionally be robust to adversarial perturbations to the input, typically defined by a fixed family $\mathcal{U}=\{U_x\}_{x \in X}$ of possible perturbations. Developing robust algorithms with provable guarantees has been an important research direction in recent years, both for parametric \cite{loh18, attias19, Srebro19, bartlett19, pathak20} and non-parametric \cite{WJC18, YRWC19, Bhattacharjee20, Bhattacharjee21} classifiers, but understanding the performance of even the most basic algorithms in the setting remains open.

In this work, we study one of the simplest, most fundamental algorithmic paradigms in learning, a classical method called \textit{empirical risk minimization} (ERM). In the robust setting, an algorithm is said to be an empirical risk minimizer (RERM) if it always outputs a hypothesis in the class with minimal \textit{robust} risk over its training data. In the standard setting, it is a classical result that any learnable class is learnable (near-optimally) by any ERM. Unfortunately, this is known to fail drastically in the robust setting---\citet{Srebro19} showed that there exist finite VC classes, $\mathcal{H}$, where no algorithm outputting hypotheses in $\mathcal{H}$ (called a \textit{proper} learner) can converge towards the optimal classifier, even with arbitrary amounts of training data. Conversely, such classes \textit{are} in fact robustly learnable, but require complicated improper learning rules and a potentially exponential number of samples.

The failure of Robust ERM for general classes raises an interesting question: \textit{are there natural sufficient conditions for the success of RERM?} One obvious answer to this question is the notion of robust VC dimension, a combinatorial parameter promising the success of RERM. However, bounding robust VC is typically difficult, and such results are only known for very specialized examples of classifiers and robustness regions (e.g.\ linear classifiers under fixed-radius balls \citep{Cullina18} or other simple margin structures \cite{pathak20}, or VC-classes under finite perturbation sets \citep{attias19}). To our knowledge there are no corresponding results for more general robustness regions and hypothesis classes beyond these special cases.

% Although there is some recent work indicating this may be possible (for example  gives sample complexity bounds for learning robust linear classifiers), these works are typically restricted to specific examples of classifiers and robustness regions, with the most common example being the case that all robustness regions are ball within some norm of a fixed radius. 
% However, to our knowledge, there are no corresponding results for the general case of arbitrary robustness region.

Given the current failure of combinatorial techniques in this setting, one might instead hope to show RERM works given sufficiently nice \textit{geometric} conditions on the hypothesis class. Sadly, this is not the case. We show that there exist robustness regions for which RERM (indeed any proper algorithm) fails even for settings as simple as (bounded) linear classifiers.
\begin{thm}[Failure of RERM for Linear Classifiers]\label{thm:intro1}
For any $W>0$ and $d>1$, let $\cH_W$ denote the set of linear classifiers with distance at most $W$ from the origin. Then there exists a set of robustness regions $U$ over $\R^d$ such that for any proper learning algorithm $L$ there exists a distribution $\cD$ for which the following hold:
\begin{itemize}
	\item \textbf{$\cD$ is realizable}: There exists $h^* \in \cH_W$ such that $\ell_U(h^*, \cD) = 0$.
	\item \textbf{$L$ has high error}: With probability at least $\frac{1}{7}$ over $S \sim \cD^m$, $\ell_U(L(S), \cD) > \frac{1}{8}$. 
\end{itemize}
\end{thm}

With this in mind, we turn our attention to a different approach: relaxing the notion of robustness itself. We'll consider a recent model of \citet{Urner22} called \textit{tolerant} robust learning. In the tolerant setting, the learner is only required to compete with the best loss over a relaxed family of perturbation sets $\mathcal{U}^\gamma$ for a (potentially arbitrary) tolerance parameter $\gamma >0$. \citet{Urner22} studied this setting in the special case of radius $r$ balls, where the learner competes with robust error against $r(1 + \gamma)$-balls. Under this framework, \citet{Urner22} give an algorithm with PAC-guarantees for VC classes using significantly fewer samples, but their techniques remain improper and only hold for the simplest robustness setting.

In this work, we show that a simple variant of RERM in the tolerant model indeed succeeds under natural geometric conditions on the hypothesis class. In particular, we study a notion of smoothness called \textit{regularity}, which roughly promises that every point in the instance space should be contained in some ball of the same label. This captures many well-studied settings, such as cases where the decision boundaries are compact, differential manifolds in $\mathbb{R}^d$.
\begin{thm}[Tolerant RERM for Regular Classes]\label{thm:upper_bound1}
Let $\cH$ be a regular hypothesis class with VC dimension $v$ over $\mathbb{R}^d$, and let $\mathcal{U}$ be any set of robustness regions. Then $TolRERM$ tolerantly PAC-learns $(\cH, \mathcal{U})$ with tolerant sample complexity 
\[
m(\epsilon, \delta, \gamma)  = O\left( \frac{vd\log \frac{d\text{Diam}(U)}{\epsilon\gamma\delta}}{\epsilon^2}\right),
\]
where $\text{Diam}(U)$ denotes the maximum $\ell_2$ diameter across robustness regions $U_x$. 
\end{thm}
Theorem \ref{thm:upper_bound1} matches the sample complexity given in \citet{Urner22} up to logarithmic factors and enjoys the additional benefits of applying to more general robustness regions along with its properness and general algorithmic simplicity. For completeness, we also analyze our algorithm's performance over non-regular classifiers in Appendix \ref{sec:proof_extension}, and show that it has a similar performance albeit at the cost of replacing the VC-dimension with $v_{\text{ball}}$, the robust VC dimension of $\cH$ over balls of a fixed radius. Thus, for non-regular hypothesis classes, our algorithm gives a reduction from arbitrary robustness regions to the case where they are all balls of a fixed radius.

Finally it's worth noting that while \citet{Urner22} only requires sampling access to the perturbation sets, stronger access such as an empirical risk minimizer is inevitable in the general setting where $\mathcal{U}$ is unknown. We show that there exists hypothesis classes where $\Omega((\frac{D}{\gamma})^d)$ queries to a sampling oracle are required for robust learning with tolerance if no other interaction with $U_x$ is permitted.

While Theorem \ref{thm:upper_bound1} gives a natural sufficient condition for the success of RERM in relaxed settings, many questions in this direction remain wide open. It would be interesting to identify a necessary condition for the success of RERM, both in the tolerant and original robust models. Furthermore, it should be noted that while we prove RERM fails to learn nice classes in the latter, the perturbation family we use to achieve this is highly combinatorial. As such, there is still hope that RERM may be sufficient in the traditional setting under \textit{joint} niceness conditions on $\mathcal{H}$ and $\mathcal{U}$, though the close interplay between the two families seems to make identifying such a condition difficult, if it is indeed possible at all.


\section{Related Work}

Much of the work on adversarial robustness \citep{Carlini17, Liu17, Papernot17, Papernot16, Szegedy14, Hein17, Katz17, Wu16,Steinhardt18, Sinha18} is done in the context of neural networks.

On the theoretical side, there has been a recent focus on developing algorithms with guarantees in convergence towards an optimal classifier. On the parametric side, several works \citep{loh18, attias19, Srebro19, bartlett19, pathak20, Cullina18} have focused on distribution agnostic bounds on the amount of data required to converge towards the optimal classifier in a given hypothesis class. For example, \citet{Srebro19} showed through an example that the VC dimension of robust learning may be much larger than standard or accurate learning indicating that the sample complexity bounds may be higher. There has also been some work considering the computation complexity required for robust learning such as \citet{Kane20}.


Aside from \citet{Urner22}, there are several works which also consider variations on robust learning with tolerance. \citet{YRWC19} and \citet{Bhattacharjee20} show that certain non-parametric algorithms exhibit a type of tolerant behavior when robustness regions are constrained to be balls of radius $r$. \citet{Omar22} considers robustness in the \textit{transductive learning setting}. Their work employs a similar idea to \citet{Urner22} in that they consider expanded perturbation sets when giving their formal guarantees. However, their expansions are not based on tolerance $\gamma > 0$.

\section{Preliminaries}
Let $\cH$ be a family of binary classifiers $\{h: \mathbb{R}^d \to \{\pm 1\}\}$, and $U = \{U_x \subseteq \R^d: x \in \reals^d\}$ any set of robustness regions. 
% We make no assumptions about $U$ except for each $U_x$ having $\ell_2$ diameter at most $D$. 
We define the robust loss function with respect to $U$ as follows.
\begin{defn}
Let $h \in \cH$ be a classifier and $(x,y) \in \reals^d \times \{\pm 1\}$ be a labeled point. Then the \textbf{robust loss} of $h$ over $(x, y)$, denoted $\ell_U(h, (x,y))$, is defined as 
\begin{align*}
  \ell_{U}(h, (x,y)) = \begin{cases} 1 & \exists x' \in U_x\text{ such that }h(x') \neq y \\0 & \text{otherwise.} \end{cases}.  
\end{align*}
%$$$$ 
That is, $h$ achieves a loss of $0$ only if it labels all points in $U_x$ as $y$. 
\end{defn}

For a distribution, $\cD$ over $\reals^d \times \{\pm 1\}$, we let $ \ell_U(h, \cD)$ denote the expected loss $h$ pays over a labeled point drawn from $\cD$. That is, $\ell_U(h, \cD) = \Ev_{(x,y) \sim \cD}[\ell_U(h, (x,y))]$. 

Similarly, for a set of $n$ labeled points, $S$, we let $\ell_U(h, S)$ denote the average robust loss $h$ pays over $S$. that is, $\ell_U(h, S) = \frac{1}{n} \sum_{i=1}^n \ell_U(h, (x_i, y_i))$. 

We will also use $||x - x'||$ to denote the $\ell_2$ distance between $x$ and $x'$, and $B(x, r)$ to denote the (closed) $\ell_2$ ball centered at $x$ with radius $r$.

\subsection{Robust PAC-learning}

We now review a natural generalization of PAC learning to the robust setting called robust PAC-learning \citep{Srebro19}. 

\begin{defn}\label{defn:rob_pac}
Let $\cH$ be a hypothesis class and $U$ be a set of robustness regions. A learner $L$ \textbf{robustly PAC-learns} $(\cH, U)$ if for every  $\epsilon, \delta > 0$, there exists $m(\epsilon, \delta)$ such that for all $n \geq m(\epsilon, \delta)$, for all data distributions, $\cD$, with probability $1-\delta$ over $S \sim \cD^n$, $$\ell_{U}(\hat{h}, \cD) \leq \min_{h \in \cH} \ell_{U}(h, \cD) + \epsilon,$$ where $\hat{h} = L(S)$ denotes the classifier in $\cH$ outputted by $L$ from training sample $S$. $m(\epsilon, \delta)$ is said to be the \textbf{sample complexity} of $L$ with respect to $(\cH, U)$. 
\end{defn}

Algorithms that are able to robustly PAC-learn a pair $(\cH, U)$ are the natural robust analogs of standard learning algorithms, and thus an important question is understanding how the sample complexities, $m(\epsilon, \delta)$, for doing so are bounded.

\section{Robust Empirical Risk Minimization on Linear Classifiers}

\looseness-1\citet{Srebro19} showed that there exist hypothesis classes $\cH$ with bounded VC dimension, and robustness regions $U$, such that proper robust PAC-learning is not possible, meaning no matter how much data one is allowed, there always exists a distribution where the learner will suffer high robust loss.

However, for many practical examples, this does not appear to be the case -- for example, \cite{Cullina18} showed that when $\cH$ is the set of all linear classifiers and $U$ is the set of robustness regions with $U_x = B(x, r)$, the sample complexity of robustly learning with RERM is at most $m(\epsilon, \delta) = \tilde{O}\left(\frac{d}{\epsilon^2}\right)$, matching the standard complexity for linear classification. 

Motivated by recent interest in more general robustness regions than balls of a fixed radius, we consider the case where $\cH$ is a natural hypothesis class, but $U$ is a potentially arbitrary robustness region. That is, we ask the following question: are there examples of natural hypothesis classes for which there exist robustness regions leading to arbitrary high sample complexities?

Unfortunately, the answer turns out to be yes. To show this, we begin by defining the natural hypothesis class of \textit{bounded} linear classifiers.
\begin{defn}\label{defn:bounded_linear}
A $W$-bounded linear classifier, $f: \R^d \to \R^d$, is a linear classifier $h$ whose decision boundary has distance at most $W$ from the origin. That is, there exist $w \in \R^d$ and , $b\in \R$ with $\frac{|b|}{||w||} \leq W$ such that $$h(x) = \begin{cases}1 & \langle w, x \rangle + b \geq 0 \\ -1 & otherwise \end{cases}.$$ We let $\cH_W$ denote the class of all $W$-bounded linear classifiers
\end{defn}
The boundedness condition, $W$, can be thought of as a regularization term which is common during any kind of practical optimization.

We now show that there exist robustness regions, $U$, for which $(\cH_W, U)$ is not robustly PAC-learnable, even in the realizable setting. For convenience, we restate Theorem \ref{thm:lower_bound1} from the introduction.

% \begin{theorem}\label{thm:lower_bound}
% Let $W > 0$ be arbitrary and let $m > 0$ be any integer. Then there exists a set of robustness regions $U$ over $\R^d$ such that for any learning algorithm $L$, there exists a distribution $\cD$ for which the following hold:
% \begin{itemize}
% 	\item There exists $h^* \in \cH_W$ such that $\ell_U(h^*, \cD) = 0$.
% 	\item With probability at least $\frac{1}{7}$ over $S \sim \cD^m$, $\ell_U(L(S), \cD) > \frac{1}{8}$. 
% \end{itemize}
% \end{theorem}
\begin{thm}\label{thm:lower_bound1}
For any $W>0$ and $d>1$, there exists a set of robustness regions $U$ over $\R^d$ such that for any learning algorithm $L$ there exists a distribution $\cD$ for which the following hold:
\begin{itemize}
	\item \textbf{$\cD$ is realizable}: There exists $h^* \in \cH_W$ such that $\ell_U(h^*, \cD) = 0$.
	\item \textbf{$L$ has high error}: With probability at least $\frac{1}{7}$ over $S \sim \cD^m$, $\ell_U(L(S), \cD) > \frac{1}{8}$. 
\end{itemize}
\end{thm}

% Theorem \ref{thm:lower_bound} immediately implies that there exist $U$ for which the sample complexity, $m(\epsilon, \delta)$ is arbitrarily high for any learner.
Theorem \ref{thm:lower_bound1} consequently shows that the observations made in \cite{Srebro19} hold even over practical hypothesis classes such as (bounded) linear classifiers.

To prove Theorem \ref{thm:lower_bound1}, we begin with the following critical lemma.
\begin{lem}\label{lem:finding_shatter}
For every $M \in \mathbb{N}$ there exists a family of $M$ subsets of $\R^d$
\[
Z^{(M)} \coloneqq \left\{Z^{(M)}_1, Z^{(M)}_2, \dots, Z^{(M)}_M\right\}
\]
satisfying the following conditions:
\begin{itemize}
	\item For every $h \in \cH_W$, there exists $z \in Z^{(M)}$ such that $h(z) = 1$.
	\item For every $1 \leq i \leq M$, there exists $h_i \in \cH_W$ such that $h_i(z) = -1$ for all $z \in \cup_{j \neq i} Z^{(M)}_j$.
	\item The sets $\{Z^{(M)}\}_{M \in \mathbb{N}}$ are mutually disjoint.
\end{itemize}
\end{lem}

\begin{proof}
Let $\{\beta_i\}_{i \in \mathbb{N}} > 0$ be a strictly decreasing sequence of sufficiently small real numbers (that we will specify later). For notational simplicity, fix an $M \in \mathbb{N}$ and write $\beta=\beta_M$ and $W' = (1 + \beta)W$. For any $r > 0$, let $S_r^{d-1}$ denote the $(d-1)$-sphere centered at the origin of radius $r$.

Observe that for any $x \in S_W^{d-1}$, there exists a unique classifier $h \in \cH_W$ whose decision boundary is tangent to $S_W^{d-1}$ at $x$ so that $h(x) = 1$. We denote this classifier as $h_x$. It follows that the set of all points on $S_{W'}^{d-1}$ that $h_x$ classifies as $1$ can be easily characterized in terms of $x$. In particular, by the definition of $h_x$, it follows from geometry that
\begin{equation}\label{eqn:lol_i_literally_used_law_of_cosines}
\curlybracket{z: h_x(z) = 1, z \in S_{W'}^{d-1}} = \curlybracket{z: ||z - (1+\beta)x|| \leq W\sqrt{2\beta(\beta + 1)}, z \in S_{W'}^{d-1}}.
\end{equation}

Let $r_\beta = 2W\sqrt{2\beta(\beta + 1)}$, and let $z_1, z_2, \dots, z_{M_{\beta}}$ denote a a greedy $r_\beta$ cover of $S_{W'}^{d-1}$, meaning that points are successively selected from $S_{W'}^{d-1}$ until no point with distance strictly greater than $r_\beta$ from all other points can be selected. Finally, define $Z_i=Z_i^{(M)}$ as the set of elements in $S_{W'}^{d-1}$ with nearest neighbor $z_i$ (ties broken arbitrarily).

We claim that this construction suffices for $M_\beta \geq M$. First, observe that $\lim_{\beta \to 0} r_\beta = 0$, which means that for sufficiently small $\beta$ that $M_\beta$ will be arbitrarily large (thus satisfying $M_\beta \geq M$). So select any $\beta$ for which this hold, and merge enough regions so that we are left with exactly $M$ regions (i.e. set $Z_{M} = \cup_{i = M}^{M_\beta} Z_i$). Note that we can always choose $0<\beta< \beta_{M-1}$ since the naturals can be embedded into any interval. We now verify the two stipulations of Lemma \ref{lem:finding_shatter}. 

The first stipulation clearly holds since $\{Z_i\}_{i=1}^M$ partition $S_{W'}^{d-1}$ and every halfspace $h \in \cH_W$ intersects the latter by construction.

For the second stipulation, observe that for any $i$, the ball centered at $z_i$ of radius $\frac{r_\beta}{2}$, $B\left(z_i, \frac{r_\beta}{2}\right)$, does not intersect $Z_j$ for any $i \neq j$. This is because such an intersection would imply by the triangle inequality that $||z_i - z_j|| \leq r_\beta$, which is a contradiction. This observation allows us to find a classifier, $h_i$, as desired --  we set $h_i$ to be the previously defined classifier, $h_{\frac{z_i}{1 + \beta}}$. Equation \ref{eqn:lol_i_literally_used_law_of_cosines} implies that the only points in $S_{W'}^{d-1}$ that it will classify as $1$ are precisely the points in $B\left(z_i, \frac{r_\beta}{2}\right) \cap S_{W'}^{d-1}$. Since this is a subset of $Z_i$, the second stipulation is met, as desired.

Finally, it is left to observe that over each choice of $M$ these $Z^{(M)}$ are mutually disjoint. This is true so long as the choices of $\beta$ themselves are disjoint, since $Z^{(M)}$ lies in the sphere of radius $W(1+\beta_M)$. As noted previously it is easy to see $\{\beta_M\}$ can be chosen in this manner in an inductive fashion.
\end{proof}

We are now sketching a proof for Theorem \ref{thm:lower_bound1}, with the full proof deferred Appendix \ref{sec:lower_bound_proof}.

\paragraph{Proof Sketch: (Theorem \ref{thm:lower_bound1})} 

Our goal is to show that for any $m \in \mathbb{N}$, any learner on $m$ samples must fail with constant probability. Fix any $m$. The main idea will be to construct a set of robustness regions, $U_{x_1}, U_{x_2}, \dots, U_{x_{3m}}$ such that any classifier in $\cH_W$ will lack robustness on at least $m$ of them.  T

Toward this end, set $M = \binom{3m}{m}$, and let $Z^{(M)}_1, Z^{(M)}_2, \dots, Z^{(M)}_M$ be subsets of $\R^d$ as described by Lemma \ref{lem:finding_shatter} (we will drop the superscript in what follows). Let $\mathcal{M}$ denote the set of all subsets of $\{1, \dots, 3m\}$ with exactly $m$ elements. Associate with each $Z_i$ a unique element of $\mathcal{M}$, thus allowing us to rename our subsets as $\{Z_T: T \in \mathcal{M}\}.$ We now define $$U_{x_i} = \cup_{T: i \in T} Z_T,$$ where $x_i$ is an arbitrary point inside $U_{x_i}$.

Lemma \ref{lem:finding_shatter} that if all $x_i$ are given a label of $-1$, then any $h \in \cH_W$ will label some (for some set $T$) some $z \in Z_T$ as $+1$, thus causing it to lack robustness on \textit{all} $i \in T$. Conversely, we see that for any $T$, there is a classifier $h_T \in \cH_W$ that is accurate and robust at all $x_i$ with $i \notin T$. 


With these observations, we are now prepared to show that for any learner $L$, there exists a distribution $D$ for which $L$ has large expected robust loss. To do this, we use a standard lower bound technique found in \cite{ml_book} that was adapted to the robust setting in \cite{Srebro19}. The idea will be to pick $D$ to be the uniform distribution over a random subset of $2m$ points in $\{x_1, \dots, x_{3m}\}$. We will then argue that because $L$ only has access to $m$ points from $D$, it won't be able to distinguish which subset $D$ corresponds to, and this will lead to a large expected loss. $\square$

As demonstrated in Lemma \ref{lem:finding_shatter}, the robustness regions $U$ used in our lower bound are combinatorial in nature and unlikely to represent any practical kinds of robustness regions. Nevertheless, our lower bound does show that naturality assumptions on the hypothesis class alone are \textit{not} sufficient for ensuring robust PAC-learnability.

A natural next step would be to fully characterizes pairs $(\cH, U)$ for which proper robust PAC-learnability is possible, but we leave this as a direction for future work. We instead turn towards relaxing the requirements of the robust PAC-learning model in order to find algorithms that are able to succeed in the case that $\cH$ is natural but $U$ is arbitrary. 

\section{Tolerant PAC learning}

Theorem \ref{thm:lower_bound1} implies that for complex robustness regions, robust PAC-learning (Definition \ref{defn:rob_pac}) is not possible, even when $\cH$ is a simple hypothesis class. Thus, robust learning will require other ideas.

One such idea is Tolerant PAC-learning, introduced in \citet{Urner22}. Here, the idea is to relax the goal of robust PAC-learning by introducing a tolerance parameter $\gamma$ representing the amount of ``slack" the learner gets with respect to the robustness regions $U$. We now expand their definition to arbitrary robustness regions by introducing \textit{perturbed regions}, $U^\gamma$, which are defined as follows.  

\begin{defn}
Let $U$ be a set of robustness regions and $\gamma > 0$ be a distance. For any point $x \in \reals^d$, define $U_x^\gamma$ as the set of all points with distance at most $\gamma$ from $U_x$. That is, $$U_x^\gamma = \{x': ||x' - U_x|| \leq \gamma\}.$$ Finally, we let $U^\gamma = \{U_x^{\gamma}: x \in \reals^d\}$ denote the set of \textbf{$\gamma$-perturbed regions} of $U$. 
\end{defn}

Tolerant PAC-learning is then defined as follows
\begin{defn}\label{defn:tol_pac}
Let $\cH$ be a hypothesis class and $U$ a set of robustness regions. A learner $L$ \textbf{tolerantly PAC-learns} $(\cH, U)$ if for every  $\epsilon, \delta, \gamma > 0$, there exists $m(\epsilon, \delta, \gamma)$ such that for all $n \geq m(\epsilon, \delta, \gamma)$, for all data distributions, $\cD$, with probability $1-\delta$ over $S \sim \cD^n$, $$\ell_U(\hat{h}, \cD) \leq \min_{h \in \cH} \ell_{U^\gamma}(h, \cD) + \epsilon,$$ where $\hat{h} = L(S)$ denotes the classifier outputted by $L$ from training sample $S$. As before, we let $m(\epsilon, \delta, \gamma)$ denote the \textbf{tolerant sample complexity} of $L$ with respect to $(\cH, U)$. 
\end{defn}

\subsection{Tolerant RERM oracles}

Because our robustness regions, $U_x$, are arbitrary subsets of $\R^d$, any learning algorithm will require some sort of access to $U$. We describe this access through an oracle for $U$. 

\citet{Urner22} employs a \textit{sampling oracle} for $U$ which allows the learner to sample points at uniform from the set $U_x$ for any point $x$. In their setting, $U_x$ is constrained to be a closed ball of known radius centered at $x$, and consequently the sampling oracle selects points from the uniform distribution over the ball. We say that a robust learner is in the \textit{sampling model} if its only way of interacting with the regions $U_x$ is through a sampling oracle.

In our setting, where $U_x$ can be an arbitrary regions, sampling oracles pose a significant challenge -- there exists choices of $U$ for which tolerant PAC learning requires an exponential number of queries to the sampling oracle. We state this as a proposition with the proof in Appendix \ref{app: lower bound}.
\begin{prop}\label{prop: lowerbound}
For any $D>10\gamma > 0$, there exists a hypothesis class $\mathcal{H}$ and a set of robustness regions, $U$ such that the following holds. There exist constants $\epsilon$ and $\delta$, along with a data distribution $\cD$, such that for any $n > 0$, any learner $L$ that achieves $$\ell_U(L(S), \cD) \leq \min_{h \in \cH} \ell_{U^\gamma}(h, \cD) + \epsilon$$ with probability at least $1-\delta$ over $S \sim \cD^n$ must make at least $\Omega\left(\left(\frac{D}{\gamma}\right)^d\right)$ sampling oracle calls.
\end{prop}

To circumvent this issue, we turn our attention to a different natural oracle first proposed in \citet{Srebro19} that is based on Robust Empirical Risk Minimization (RERM). An RERM oracle, $\cO_{U, \cH}(S)$, is a function that returns the classifier $h \in \cH$ with minimal robust empirical risk over $S$. That is, $$\cO_{U, \cH}(S) = \argmin_{h \in \cH} \ell_U(h, S).$$ In our work, we will assume access to a mild strengthening of this oracle that allows empirical risk minimization over any perturbed robustness region, $U^r$.

\begin{defn}\label{defn:oracle}
\textbf{A tolerant RERM-oracle} for robustness regions $U$ and hypothesis class $\cH$ is a function $\cO_{U, \cH}(S, r)$ that maps any set of labeled points $S$ and any distance $r > 0$ to the classifier with minimal empirical risk over $S$ with respect to $U^r$. That is, $$\cO_{U, \cH}(S, r) = \argmin_{h \in H} \ell_{U^r}(h, S).$$
\end{defn}

Observe that in the case that $U$ consists of balls of radius $r$, a tolerant oracle merely implies we can also minimize empirical risk for balls of larger radii. 

\section{Tolerant PAC learning for Regular Hypothesis Classes}

Before presenting our algorithm, we first present a key assumption on our hypothesis class, $\cH$, that we refer to as \textit{regularity.}

\subsection{Regular hypothesis classes}

\begin{defn}
We say that a hypothesis class, $\cH$ is \textbf{$\alpha$-regular} for $\alpha > 0$ if for all $h \in \cH$ and for all $x \in \reals^d$, there exists a closed ball $B$ of radius $\alpha$ containing $x$ such that $h(x') = h(x)$ for all $x' \in B$. We also say that $\cH$ is \textbf{regular} if it is $\alpha$-regular for some $\alpha > 0$. 
\end{defn}

One important example is hypothesis classes with relatively smooth manifolds as decision boundaries. In particular, the parameter $\alpha$ can be tied to the smoothness measure of a manifold known as its \textit{reach}.

\begin{defn}
Let $M$ be a closed manifold embedded in $\reals^d$. The \textbf{reach} of $M$ is the largest $\alpha > 0$ such that for all $x \in \reals^d$, if $||x - M|| \leq \alpha$, then $x$ has a unique nearest neighbor in $M$.
\end{defn}

This parameter directly translates to regularity.

\begin{prop}\label{prop:reach}
Let $h$ be a classifier with decision boundary $M$. Suppose that $M$ is a closed $(d-1)$-dimensional submanifold over $\R^d$ with reach $\alpha$. Then $h$ is $\alpha/2$-regular. 
\end{prop}

\begin{proof}
Let  $h \in \cH$ be a classifier with decision boundary $M$. Let $x$ be an arbitrary point with $h(x) = y$. We desire to exhibit a ball $B$ of radius $\alpha/2$ containing $x$ for which $h$ is uniformly $y$. 

Let $\rho: \reals^d \to \reals_{\geq 0}$ be the distance function $\rho(x) = ||x - M||$. It is well known that this function is everywhere continuous and has a continuous derivative over $\{x: 0 < \rho(x) < \alpha\}.$

If $\rho(x) > \alpha/2$, then we can simply take $B= B(x, \alpha/2)$ as all points here must be classified as $y$ by the definition of a decision boundary. Thus, assume $\rho(x) \leq \alpha/2$. 

Let $V$ be the gradient vector field of $\rho$ defined over $\{x: \rho(x) < \alpha\}$. Since all points in this region have a unique nearest neighbor in $M$, the gradient has magnitude $1$ for all such points, and the direction is precisely opposite the straight line path from the point's nearest neighbor in $M$. 

\looseness-1Since $V$ is continuous, (and Lipshitz over a bounded region), there exists a unique curve $\tau$ starting at $x$ of length $\frac{\alpha}{2}$ that is always tangent to $V$. It follows that the endpoint of this path, $x'$ must satisfy $\rho(x') = \frac{\alpha}{2} + \rho(x) > \frac{\alpha}{2}$ and $||x - x'|| \leq \frac{\alpha}{2}$. This means that $B = B(x', \frac{\alpha}{2})$ suffices, as desired. 
\end{proof}

\vspace{-5mm}
\subsection{Our Algorithm} 

\looseness-1 We now give a tolerant PAC learning algorithm called $TolRERM$ (Algorithm \ref{alg:estimate}) which assumes access to a tolerant RERM oracle (Definition \ref{defn:oracle}). $TolRERM$ is essentially robust empirically risk minimization with a slight modification: rather than using the original robustness regions, $U$, we use the perturbed regions, $U^r$ where $0 < r < \gamma$ is chosen at random. $TolRERM$'s performance is given by Theorem \ref{thm:upper_bound1}, which is restated here for convenience. 
\begin{thm}
\looseness-1Let $\cH$ be a regular hypothesis class with VC dimension $v$, and let $U$ be a set of robustness regions. Then $TolRERM$ tolerantly PAC-learns $(\cH, U)$ with tolerant sample complexity, $m(\epsilon, \delta, \gamma)  = O\left( \frac{vd\log \frac{dD}{\epsilon\gamma\delta}}{\epsilon^2}\right)$, where $D$ denotes the maximum $\ell_2$ diameter of any region, $U_x$. 
\end{thm}
\vspace{-2mm}
\begin{algorithm}
   \caption{$TolRERM(\cD, \epsilon, \delta, \gamma, n)$}
   \label{alg:estimate}

   Sample $r \sim [\frac{\epsilon\delta\gamma}{7}, \gamma]$ at uniform\;
   
   Sample $S \sim \cD^n$\;
    
   Output $\hat{h} = \cO_{U, \cH}(S, r)$\;

\end{algorithm}
\vspace{-2mm}
Since the set of bounded linear classifiers, $\cH_W$ (Definition \ref{defn:bounded_linear}) is clearly regular and has VC dimension $O(d)$, Theorem \ref{thm:upper_bound1} immediately implies the following corollary. 

\begin{cor}
For any set of robustness regions, $U$, $TolRERM$ tolerantly PAC-learns $(\cH_W, U)$ with tolerant sample complexity $m(\epsilon, \delta, \gamma) = O\left( \frac{d^2\log \frac{dD}{\epsilon\gamma\delta}}{\epsilon^2}\right)$, where $D$ denotes the maximum $\ell_2$ diameter of any robustness region, $U_x$.  
\end{cor}

$TolRERM$ matches the sample complexities for linear classifiers found in \cite{Srebro19} and \cite{Urner22}. However, it enjoys the advantage of being simpler (as it is essentially an empirical risk minimization algorithm) and a \textit{proper} learning algorithm (as it outputs a linear classifier). 






\paragraph{Beyond regular hypothesis classes:} It turns out that Algorithm \ref{alg:estimate} has bounded sample complexity for \textit{any} hypothesis class with finite robust VC-dimension for balls (see Appendix \ref{sec:proof_extension} for a full description).
% However, this comes at the expense of using the adversarial vc dimension, $v_{ball}$, which is the vc dimension of the loss function when the robustness regions are restricted to be balls of fixed radii centered at each point (see Appendix \ref{sec:proof_extension} for a full description). 
Thus, Algorithm \ref{alg:estimate} can alternatively be thought of as a reduction from the sample complexity for learning robust classifiers over arbitrary robustness regions to the sample complexity for balls of fixed radii. This is expressed in the following result (proved in Appendix \ref{sec:proof_extension}). 

\begin{thm}\label{thm:upper_bound_general}
Let $\cH$ be any hypothesis class with maximal adversarial VC dimension $v_{ball}$, and let $U$ be any set of robustness regions. Then $TolRERM$ tolerantly PAC-learns $(\cH, U)$ with tolerant sample complexity $m(\epsilon, \delta, \gamma)  = O\left( \frac{v_{ball}d\log \frac{dD}{\epsilon\gamma\delta}}{\epsilon^2}\right),$ where $D$ denotes the maximum $\ell_2$ diameter of any robustness region, $U_x$. 
\end{thm}

\subsection{Proof of Theorem \ref{thm:upper_bound1}}

We begin by showing that randomly choosing $r$ allows the optimal empirical loss $U^r$ to change relatively smoothly with respect to $r.$
\begin{lem}\label{lem:r_works}
For $r \in [0, \gamma]$, let $OPT_S^r = \min_{h \in H} \ell_{U^r}(h, S)$. Then with probability at least $1 - \frac{\delta}{2}$ over $r \sim [\frac{\epsilon\delta\gamma}{7}, \gamma]$, $OPT_S^r \leq OPT_S^{r -\frac{\epsilon\delta\gamma}{7}} + \frac{\epsilon}{3}.$
\end{lem}
\textit{Proof. }Let $\alpha = \frac{\epsilon\delta\gamma}{7}.$ Our goal is to show that $OPT_S^r - OPT_S^{r-\alpha}$ is likely to be small. Our strategy is to bound the expected value of $OPT_S^r - OPT_S^{r-\alpha}$ and then apply Markov's inequality. As a technical note, the function $r \mapsto OPT_S^r$ is monotonic and bounded, and consequently measurable, which ensures that our expectations are well defined. To this end, we have,
\begin{equation*}
\begin{split}
\Ev[OPT_S^r &- OPT_S^{r-\alpha}] = \Ev[OPT_S^r] - \Ev[OPT_S^{r-\alpha}] \\
&= \frac{1}{\gamma - \alpha}\left(\int_\alpha^\gamma OPT_S^rdr - \int_\alpha^\gamma OPT_S^{r-\alpha}dr \right) \\
&= \frac{1}{\gamma - \alpha}\left(\int_\alpha^\gamma OPT_S^rdr - \int_0^{\gamma-\alpha} OPT_S^rdr \right) \\
&= \frac{1}{\gamma - \alpha}\left(\int_{\gamma-\alpha}^\gamma OPT_S^rdr - \int_0^{\alpha} OPT_S^rdr \right) \\
&\leq \frac{\alpha}{\gamma - \alpha} = \frac{\delta\epsilon\gamma}{7\gamma - \delta\epsilon\gamma} \leq \frac{\delta\epsilon}{6},
\end{split}
\end{equation*}
since $\epsilon, \delta \leq 1$. Applying Markov's inequality, with probability at least $1 - \frac{\delta}{2}$, $OPT_S^r - OPT_S^{r-\alpha} \leq \frac{\epsilon}{3}$. $\square$.


Next, we construct a set of robustness regions $V^r$ that have similar robust loss to $U^r$ and are also finite.

\begin{lem}\label{lem:v_construct}
Suppose that $\cH$ is $\gamma$-regular. For all $r \in [\frac{\epsilon\delta\gamma}{7}, \gamma]$, there exists a set of robustness regions $V^r = \{V_x^r: x \in \reals^d\}$ satisfying the following two properties.  
\begin{enumerate}
	\item $|V_x^r| = O\left(\left(\frac{D}{\epsilon\delta\gamma}\right)^d\right)$, where $D$ denotes the maximum diameter of $U_x$. 
	\item Let $\alpha = \frac{\epsilon\delta\gamma}{7}$. For all labeled points $(x,y)$ and for all classifiers $h \in \cH$, $$\ell_{U^{r - \alpha}}(h, (x,y)) \leq \ell_{V^r}(h, (x,y)) \leq \ell_{U^r}(h, (x,y)).$$
\end{enumerate}
\end{lem}

\begin{proof}
For any $x \in \reals^d$, we will show how to construct $V_x$ so that it satisfies the two conditions above.

Observe that $U_x^r$ is closed and bounded as it is a union of closed balls of radius $r$. Since each $U_x$ has diameter at most $D$, this means that $U_x^r$ is compact. Thus, there exists a finite set of balls of radius $\alpha/2$ that cover $U_x^r$. Note that these balls are \textit{not} necessarily contained within $U_x^r$ -- only that $U_x^r$ is a subset of their union. Let $C_x$ denote the set of all centers of the smallest such cover. We claims that $V_x = C_x \cap U_x^r$ suffices.

First, $|C_x| \leq O\left((\frac{D}{\alpha})^d\right)$ because any ball of diameter $D$ can be covered by $O\left((\frac{D}{\alpha})^d\right)$ balls of radius $\alpha/2$, and $U_x^r$ is a subset of a ball of diameter $D+2r$. This implies that the first condition holds.

Second, pick any labeled point $(x,y)$ and any classifier $h \in \cH$. If $\ell_{V^r}(h, (x,y)) = 1$, then we immediately have $\ell_{U^r}(h, (x,y) = 1$ since $V^r \subseteq U^r$. This implies that $\ell_{V^r}(h, (x,y)) \leq \ell_{U^r}(h, (x,y)$ giving the second half of the second condition.   

If $\ell_{U^{r-\alpha}}(h, (x,y)) = 1$, then there exists $x' \in U_x^{r-\alpha}$ such that $h(x') \neq y$. It follows that since $h$ is $\gamma$-regular, $h$ must also be $\alpha$-regular (as $\alpha < \gamma$). This means that there exists a ball $B$ of radius $\alpha/2$ containing $x'$ such that $h$ does not output $y$ for any point in $B$. 

By the triangle inequality, $B \subseteq U_x^r$, and since $C_x$ covers $U_x^r$, it follows that there exists $x^* \in C_x \cap B$. By definition, this also means $x^* \in V_x^r$. However, by the definition of $B$, we must have $h(x^*) \neq y$, and this means that $\ell_{V_x^r}(h, (x,y)) = 1$. Since $(x,y)$ was arbitrary, this proves the second half of the second condition. 
\end{proof}

We are now prepared to prove Theorem \ref{thm:upper_bound1}.

\begin{proof}
(\textbf{Theorem \ref{thm:upper_bound1}}) Let $\alpha = \frac{\epsilon\delta\gamma}{7}$. For all $s > 0$, let $h^s \in \cH$ denote any fixed choice of classifier with minimal empirical loss with respect to $U^s$. That is, $h^s = \argmin_{h \in \cH} \ell_{U^s}(h, S).$ Then by Lemma \ref{lem:r_works}, with probability at least $1 - \frac{\delta}{2}$ over $r \sim [\alpha, \gamma]$, 
\begin{equation}\label{eqn:r_works}
\ell_{U^{r}}(h^{r}, S) \leq \ell_{U^{r-\alpha}}(h^{r-\alpha}, S) + \frac{\epsilon}{3}.
\end{equation}
Next, let $V^r$ be as defined in Lemma \ref{lem:v_construct} and suppose that $\gamma$ is small enough so that $\cH$ is $\gamma$-regular (this must occur since $\cH$ is regular by assumption). By the second condition in the lemma, it follows that for all $h \in H$:
\begin{equation}\label{eqn:v_bound_D}
\ell_{U^{r-\alpha}}(h, \cD) \leq \ell_{V^r}(h, \cD) \leq \ell_{U^r}(h, \cD),
\end{equation}
\begin{equation}\label{eqn:v_bound}
\ell_{U^{r-\alpha}}(h, S) \leq \ell_{V^r}(h, S) \leq \ell_{U^r}(h, S).
\end{equation}
 Next, since $|V_x| = O\left(\left(\frac{D}{\epsilon\delta\gamma}\right)^d\right)$, Proposition \ref{prop:finite-RVC} (proved in the Appendix \ref{app: robust vc bound}) implies that the Robust VC dimension of $\cH$ with respect to $V_x$ is at most $O\left(vd\log \frac{Dv}{\epsilon\delta\gamma} \right)$, where $v$ denotes the VC dimension of $\cH$. 
 
 
Because $S$ is independent from $r$, there exists an absolute constant $C$ such that if $n \geq C\frac{vd\log \frac{Dv}{\epsilon\delta\gamma} +\log\frac{1}{\delta}}{\epsilon^2}$, then classical connections with uniform convergence \cite{vapnik1974theory} imply that with probability at least $1 - \frac{\delta}{2}$ over $S \sim \cD^n$, for all $h \in \cH$, 
\begin{equation}\label{eqn:uniform}
|\ell_{V^r}(h, S) - \ell_{V^r}(h, \cD)| \leq \frac{\epsilon}{3}.
\end{equation}
Applying a union bound, we see that for $n = \Omega\left( \frac{vd\log \frac{dD}{\epsilon\gamma\delta}}{\epsilon^2}\right)$, with probability at least $1-\delta$ over $r \sim [\alpha,\gamma]$ and $S \sim \cD^n$, Equations \ref{eqn:r_works},
\ref{eqn:v_bound_D}, \ref{eqn:v_bound}, and \ref{eqn:uniform} simultaneously hold. Thus it suffices to show that under these assumptions, any $\hat{h} \in H$ minimizing the robust empirical risk of $S$ under $U^r$ satisfies $\ell_U(\hat{h}, \cD) \leq \min_{h \in \cH} \ell_{U^\gamma}(h^\gamma, \cD) + \epsilon,$ as this will imply that $m(\epsilon, \delta, \gamma) = O\left( \frac{vd\log \frac{dD}{\epsilon\gamma\delta}}{\epsilon^2}\right)$ as desired. 

To do so, we use a series of manipulations applying Equations \ref{eqn:r_works} and \ref{eqn:uniform}. For convenience, we let $h^* = \argmin_{h \in \cH} \ell_{U^\gamma}(h, \cD)$. 
% Also, note that our outputted classifier, $\hat{h}$, precisely equals $h^r$ as we are performing RERM over $U^r$.

Since $U \subset U^r$ and $\ell_{U^r}$ is bounded by $\ell_{V^r}$ (Equation \ref{eqn:v_bound_D}), we have that $$\ell_U(h^r, \cD) \leq \ell_{U^r}(h^r, \cD) \leq \ell_{V^r}(h^r, \cD),$$ 
and further by Equations \ref{eqn:uniform} and \ref{eqn:v_bound} that
$$\ell_{V^r}(h^r, \cD) \leq \ell_{V^r}(h^r, S) + \frac{\epsilon}{3} \leq \ell_{U^r}(h^r, S) + \frac{\epsilon}{3}.$$
Since $h^s$ is defined as the classifier of lowest empirical risk over $U^s$, it follows from Equation \ref{eqn:r_works} and this definition that 
$$\ell_{U^r}(h^r, S) \leq \ell_{U^{r-\alpha}}(h^{r-\alpha}, S) + \frac{\epsilon}{3} \leq \ell_{U^{r-\alpha}}(h^*, S) + \frac{\epsilon}{3}.$$
% where $h^*$ denotes the classifier with optimal true loss over $U^{\gamma}$. 
Applying the same trick we did earlier with Equations \ref{eqn:v_bound} (bounding $U^{r-\alpha}$ with $V^r$) and \ref{eqn:uniform} (uniform convergence of the loss over $V^r$), we have $$\ell_{U^{r-\alpha}}(h^*, S) \leq \ell_{V^r}(h^*, S) \leq \ell_{V^r}(h^*, \cD) + \frac{\epsilon}{3}.$$ Finally, applying Equation \ref{eqn:v_bound_D} to bound the loss over $V^r$ with the loss over $U^r$ and then noting that $r \leq \gamma$, we have that $$\ell_{V^r}(h^*, \cD) \leq \ell_{U^r}(h^*, \cD) \leq \ell_{U^\gamma}(h^*, \cD).$$ Combining all of our observations with the transitive property, it follows that $$\ell_U(h^r, \cD) \leq \ell_{U^\gamma}(h^*, \cD) + \epsilon.$$ Finally, since this holds for any choice of $h^r$ minimizing $\ell_{U^r}(h^r,S)$, it holds for the particular choice of the Tolerant RERM oracle which completes the result.
\end{proof}





%\newcommand{\crop}[1]{\mathrm{crop}({#1})}
\newcommand{\object}[1]{\mathrm{object}({#1})}
\newcommand{\ba}{A_i}
\newcommand{\bb}{B_i}
\newcommand{\calA}{\mathcal{A}}
\newcommand{\calB}{\mathcal{B}}
\newcommand{\calX}{\mathcal{X}}
\newcommand{\masked}[1]{\mathrm{masked}({#1})}
\newcommand{\bx}{\mathbf{x}}
\newcommand{\SSL}{\textsc{SSL}}
\newcommand{\SSLbb}{\SSL^\mathrm{back}}
\newcommand{\SSLpj}{\SSL^\mathrm{proj}}
\newcommand{\CLF}{\textsc{CLF}}
\newcommand{\CLFbb}{\CLF^\mathrm{back}}
\newcommand{\CLFpj}{\CLF^\mathrm{proj}}
\newcommand{\SUP}{\textsc{SUP}}
\newcommand{\KNN}{\textsc{KNN}}
\newcommand{\KNNset}{\textsc{KNN}^\mathrm{set}}
\newcommand{\KNNprob}{\textsc{KNN}^\mathrm{prob}}
\newcommand{\KNNcl}{\textsc{KNN}^\mathrm{cl}}
\newcommand{\KNNconf}{\textsc{KNN}^\mathrm{conf}}
\newcommand{\RCDM}{\textsc{RCDM}}
\newcommand{\cl}{\mathrm{cl}}
\newcommand{\clpred}{\tilde{\mathrm{cl}}}
\newcommand{\Abox}{\overline{\calA}}
\newcommand{\Bbox}{\overline{\calB}}
\newcommand{\dejavu}{\emph{déjà vu }}
\newcommand{\Dejavu}{\emph{Déjà vu }}

\newcommand{\citations}{{\color{green}[CITE]}}

\definecolor{part_blue}{rgb}{0.2824, 0.4706, .8157}
\definecolor{part_red}{rgb}{0.8392, 0.3725, 0.3725}
\definecolor{part_orange}{rgb}{0.9333, 0.5216, 0.2902}

\DeclareRobustCommand{\mybox}[2][gray!20]{%
\begin{tcolorbox}[   %% Adjust the following parameters at will.
        % breakable,
        left=0pt,
        right=0pt,
        top=0pt,
        bottom=0pt,
        colback=#1,
        colframe=#1,
        width=\dimexpr\columnwidth\relax, 
        % width=\textwidth, 
        enlarge left by=0mm,
        boxsep=5pt,
        arc=0pt,outer arc=0pt,
        ]
        #2
\end{tcolorbox}
}
%\section{Introduction}
\label{sec:intro}
Self-supervised learning (SSL)~\citep{chen2020simclr, chen2020simsiam, zbontar2021barlow, vicreg, caron2020swav, MAE} aims to learn general representations of content-rich data without explicit labels by solving a \textit{pretext task}. In many recent works, such pretext tasks rely on joint-embedding architectures whereby randomized image augmentations are applied to create multiple views of a training sample, and the model is trained to produce similar representations for those views. When using cropping as random image augmentation, the model learns to associate objects or parts (including the background scenery) that co-occur in an image.
However, doing so also arguably exposes the training data to higher privacy risk as objects in training images can be explicitly memorized by the SSL model. For example, if the training data contains the photos of individuals, the SSL model may learn to associate the face of a person with their activity or physical location in the photo. This may allow an adversary to extract such information from the trained model for targeted individuals.

\begin{figure}[t]
    \centering
    \includegraphics[width=1.0\columnwidth]{figures/new_black_swan.pdf}
    \caption{\textbf{Left:} Reconstruction of an SSL training image from a crop containing only the background. The SSL model memorizes the association of this \emph{specific} patch of water (pink square) to this \emph{specific} foreground object (a black swan) in its embedding, which we decode to visualize the full training image. \textbf{Right:} The reconstruction technique fails on a public test image that the SSL model has not seen before.}
    \label{fig:black_swan}
\end{figure}

In this work, we aim to evaluate to what extent SSL models memorize the association of specific objects in training images or the association of objects and their specific backgrounds, and whether this memorization signal can be used to reconstruct the model's training samples. Our results demonstrate that SSL models memorize such associations beyond simple correlation. For instance, in Figure \ref{fig:black_swan} (\textbf{left}), we use the SSL representation of a \emph{training image crop containing only water} and this enables us to reconstruct the object in the foreground with remarkable specificity---in this case a black swan.
By contrast, in Figure \ref{fig:black_swan} (\textbf{right}), when using the \emph{crop from the background of a test set image} that the SSL model \emph{has not seen before}, its representation only contains enough information to infer, through correlation, that the foreground object was likely some kind of waterbird --- but not the specific one in the image.

Figure \ref{fig:black_swan} shows that SSL models suffer from the unintended memorization of images in their training data---a phenomenon we refer to as \emph{déjà vu memorization}
%\footnote{The French loanword \emph{déjà vu} means already-seen, which reflects the type of unintended memorization of objects that the SSL model saw during training.}.
\footnote{The French loanword \emph{déjà vu} means `already-seen', just as an image is seen and memorized in training.}
Beyond visualizing \emph{déjà vu} memorization through data reconstruction, we also design a series of experiments to quantify the degree of memorization for different SSL algorithms, model architectures, training set size, \emph{etc.} We observe that \emph{déjà vu} memorization is exacerbated by the atypically large number of training epochs often recommended in SSL training, as well as certain hyperparameters in the SSL training objective. Perhaps surprisingly, we show that \emph{déjà vu} memorization occurs even when the training set is large---as large as half of ImageNet~\citep{imagenet}---and can continually worsen even when standard techniques for evaluating learned representation quality (such as linear probing) do not suggest increased overfitting. Our work serves as the first systematic study of unintended memorization in SSL models and motivates future work on understanding and preventing this behavior. Specifically, we: 
\begin{itemize}
    \vspace{-0.5em}
    \item Elucidate how SSL representations memorize aspects of individual training images, what we call \emph{déjà vu} memorization;
    \item Design a novel training data reconstruction pipeline for non-generative vision models. This is in contrast to many prominent reconstruction algorithms like \citep{carlini2021extracting, google_diffusion}, which rely on the model itself to generate its own memorized samples and is not possible for SSL models or classifiers;
    \item Propose metrics to quantify the degree of \dejavu memorization committed by an SSL model. This allows us to observe how \dejavu changes with training epochs, dataset size, training criteria, model architecture and more. 
\end{itemize}

%\section{Preliminaries and Related Work}
\label{sec:related}

\textbf{Self-supervised learning} (SSL) is a machine learning paradigm that leverages unlabeled data to learn representations. Many SSL algorithms rely on \emph{joint-embedding} architectures (\emph{e.g.}, SimCLR~\citep{chen2020simclr}, Barlow Twins~\citep{zbontar2021barlow}, VICReg~\citep{vicreg} and Dino~\citep{Dino}), which are trained to associate different augmented views of a given image. For example, in SimCLR, given a set of images $\calA = \{A_1,\ldots,A_n\}$ and a randomized augmentation function $\mathrm{aug}$, the model is trained to maximize the cosine similarity of draws of $\SSL(\mathrm{aug}(A_i))$ with each other and minimize their similarity with $\SSL(\mathrm{aug}(A_j))$ for $i \neq j$. The augmentation function $\mathrm{aug}$ typically consists of operations such as cropping, horizontal flipping, and color transformations to create different views that preserve an image's semantic properties. 

\paragraph{SSL representations.} Once an SSL model is trained, its learned representation can be transferred to different downstream tasks. This is often done by extracting the representation of an image from the \emph{backbone model}\footnote{SSL methods often use a trick called \emph{guillotine regularization}~\citep{Guillotine}, which decomposes the model into two parts: a \emph{backbone model} and a \emph{projector} consisting of a few fully-connected layers. Such trick is needed to handle the misalignment between the pretext SSL task and the downstream task.} and either training a linear probe on top of this representation or finetuning the backbone model with a task-specific head~\citep{Guillotine}.
%Compared to representations learned by supervised learning, SSL representations are often more robust and transferable~\citep{hendrycks2019using, ericsson2021self}, leading to state-of-the-art result on many downstream tasks. To understand the effectiveness of SSL algorithms, several prior works investigated what kind of information the SSL model has learned~\citep{jing2021understanding, ericsson2021self, kalibhat2022towards, RCDM}. In particular, \citet{RCDM} trained a conditional generative model on SSL representations and showed that they encode richer visual details about the input image compared to supervised learning. 
%However, from a privacy perspective, this may be a cause for concern as the model also has more potential to overfit and memorize precise details about the training data compared to supervised learning. We show concretely that this privacy risk can indeed be realized by defining and measuring \emph{déjà vu} memorization.
It has been shown that SSL representations encode richer visual details about input images than supervised models do \cite{RCDM}. However, from a privacy perspective, this may be a cause for concern as the model also has more potential to overfit and memorize precise details about the training data compared to supervised learning. We show concretely that this privacy risk can indeed be realized by defining and measuring \emph{déjà vu} memorization.
\vspace{-0.5em} 
% \paragraph{Privacy risks in ML.} Overfitting in ML occurs when a model memorizes information specific to its training data rather than general population-level information. When the model is trained on privacy-sensitive data, overfitting is especially harmful as an adversary can infer private information about the training data when given access to the model~\citep{yeom2018privacy, feldman2020does}. The simplest and most well-studied form of privacy risk in ML is susceptibility to \emph{membership inference attacks}~\citep{shokri2017membership, salem2018ml, sablayrolles2019white}, where the adversary infers whether an individual is part of the training set or not. More sophisticated privacy attacks include \emph{attribute inference}~\citep{fredrikson2014privacy, mehnaz2022your, jayaraman2022attribute}, where specific attributes about an individual are inferred given others, and \emph{data reconstruction}~\citep{carlini2021extracting, balle2022reconstructing, guo2022bounding}, where entire training samples are recovered from the trained model. Our study of \emph{déjà vu} memorization is similar to both attribute inference and data reconstruction, leveraging SSL representations of the training image background to infer and reconstruct the foreground object.
% \vspace{-0.5em} 
% \paragraph{Training data extraction in NLP.} Our study of \dejavu memorization in SSL models is inspired by similar work in the natural language processing (NLP) domain. \citet{carlini2019secret} first showed that language models exhibit unintended memorization, where given a context string present in its training data, the model can generate the remaining text at test time. This unintended memorization has been further exploited in \citet{carlini2021extracting} to extract training data from GPT-2~\citep{radford2019language} and, more recently, extended to extract memorized images from Stable Diffusion \citep{google_diffusion}. The way by which these works exploit unintended memorization is similar to ours: given partial information about a training sample, the model is prompted to reveal the rest of the sample. In our case, however, since the SSL model is not generative, extraction is significantly harder and requires careful design.

\paragraph{Privacy risks in ML.} When a model is overfit on privacy-sensitive data, it memorizes specific information about its training examples, allowing an adversary with access to the model to learn private information~\citep{yeom2018privacy, feldman2020does}. Privacy attacks in ML range from the simplest and best-studied \emph{membership inference attacks}~\citep{shokri2017membership, salem2018ml, sablayrolles2019white} to \emph{attribute inference}~\citep{fredrikson2014privacy, mehnaz2022your, jayaraman2022attribute} and \emph{data reconstruction}~\citep{carlini2021extracting, balle2022reconstructing, guo2022bounding} attacks. In the former, the adversary only infers whether an individual participated in the training set. Our study of \emph{déjà vu} memorization is most similar to the latter: we leverage SSL representations of the training image background to infer and reconstruct the foreground object. Our approach reflects similar work in the NLP domain \citep{carlini2019secret, carlini2021extracting}: when prompted with a context string present in the training data, a large language model is shown to generate the remainder of string at test time, revealing sensitive text like home addresses. This method was recently extended to extract memorized images from Stable Diffusion \citep{google_diffusion}.  We exploit memorization in a similar manner: given partial information about a training sample, the model is prompted to reveal the rest of the sample. In our case, however, since the SSL model is not generative, extraction is significantly harder and requires careful design.

%\section{Defining \emph{Déjà Vu} Memorization}
\label{sec:definition}

\paragraph{What is \dejavu memorization?} At a high level, the objective of SSL is to learn general representations of objects that occur in nature. This is often accomplished by associating different parts of an image with one another in the learned embedding. Returning to our example in Figure \ref{fig:black_swan}, given an image whose background contains a patch of water, the model may learn that the foreground object is a water animal such as duck, pelican, otter, \emph{etc.}, by observing different images that contain water from the training set. We refer to this type of learning as \emph{correlation}: the association of objects that tend to co-occur in images from the training data distribution.

A natural question to ask is \emph{``Can the reconstruction of the black swan in Figure \ref{fig:black_swan} be reasoned as correlation?''} The intuitive answer may be no, since the reconstructed image is qualitatively very similar to the original image. However, this reasoning implicitly assumes that for a random image from the training data distribution containing a patch of water, the foreground object is unlikely to be a black swan. Mathematically, if we denote by $\mathcal{P}$ the training data distribution and $A$ the image, then
\begin{equation*}
\label{eq:p_corr}
p_\text{corr} := \mathbb{P}_{A \sim \mathcal{P}}(\mathrm{object}(A) = \texttt{black swan} ~|~ \mathrm{crop}(A) = \texttt{water})
\end{equation*}
is the probability of inferring that the foreground object is a black swan through \emph{correlation}. This probability may be naturally high due to biases in the distribution $\mathcal{P}$, \emph{e.g.}, if $\mathcal{P}$ contains no other water animal except for black swans. In fact, such correlations are often exploited to learn a model for image inpainting with great success~\citep{yu2018generative, ulyanov2018deep}.

Despite this, we argue that reconstruction of the black swan in Figure \ref{fig:black_swan} is \emph{not} due to correlation, but rather due to \emph{unintended memorization}: the association of objects unique to a single training image. As we will show in the following sections, the example in Figure \ref{fig:black_swan} is not a rare success case and can be replicated across many training samples. More importantly, failure to reconstruct the foreground object in Figure \ref{fig:black_swan} (\textbf{right}) on test images hints at inferring through correlation is unlikely to succeed---a fact that we verify quantitatively in Section \ref{sec:label inference accuracy}. Motivated by this discussion, we give a verbal definition of \dejavu memorization below, and design a testing methodology to quantify \dejavu memorization in Section \ref{sec:notation and setup}.
\mybox{\textbf{Definition:} A model exhibits \emph{déjà vu memorization} when it retains information so specific to an individual training image, that it enables recovery of aspects particular to that image given a part that does not contain them.
The recovered aspect must be beyond what can be inferred using only correlations in the data distribution.} 

% \textbf{Definition:} A model exhibits \emph{déjà vu memorization} when it retains information so specific to an individual training image, that it enables recovery of aspects particular to that image given a part that does not contain them.
% The recovered aspect must be beyond what can be inferred using only correlations in the data distribution.


 We intentionally kept the above definition broad enough to encompass different types of information that can be inferred about the training image, including but not restricted to object category, shape, color and position. For example, if one can infer that the foreground object is red given the background patch with accuracy significantly beyond correlation, we consider this an instance of \dejavu memorization as well. We mainly focus on object category to quantify \dejavu memorization in Section \ref{sec:quant} since the ground truth label can be easily obtained. We consider other types of information more qualitatively in the visual reconstruction experiments in Section \ref{sec:visualizing}.

\paragraph{Privacy implications of \dejavu memorization.} \Dejavu memorization can be a cause for concern when the training data contains privacy-sensitive information. As a motivating example, consider an SSL model trained on photos of individuals. If the model exhibits \dejavu memorization then, given the face of an individual, it may be possible to infer where the individual was or even visually reconstruct their location in the training image. Such information leakage raises privacy concerns, especially if there was no prior agreement that the trained model may reveal such information to third parties. This hypothetical scenario serves as a motivation that \dejavu memorization should be carefully examined to avoid unintended disclosure of private information in practical applications.

% \begin{figure*}[h]
%     \centering
%     \includegraphics[width = 0.85\textwidth]{figures/SSL_attack_cartoon.png}
%     \caption{We measure memorization by comparing the `target model' trained on the target image ($\SSL_A$ trained on $A_i$ in above example) with the `reference model' not trained on it ($\SSL_B$, above). \textbf{[Top Strip]} A cropping of the image disjoint from the labeled foreground object is embedded using the target model. This embedding is then labeled by a K-Nearest Neighbor (KNN) adversary built on a public set of labeled images, $X$, which it has also embedded using the target model. \textbf{[Bottom Strip]} To account for correlation, the same procedure is followed with the reference model. If the label is only extracted using the target model, it is counted as memorization. If it is extracted using either model, it is counted as correlation. We find that the KNN adversary's predictions using the target model (trained on attacked examples) are significantly more accurate than they are using the reference model, indicating routine memorization of training examples.}
%     \label{fig:ssl attack cartoon}
% \end{figure*}

\begin{figure}[t]
%%%
%SPIDER
%%%
     % \centering
     % \begin{subfigure}[b]{0.25\textwidth}
     %     \centering
     %     \includegraphics[width=\textwidth]{figures/data_split.png}
     %     % \caption{SimCLR correlated \textit{yellow garden spider} examples}
     %     \label{fig:data split}
     % \end{subfigure}
     % \hfill
     % \begin{subfigure}[b]{0.7\textwidth}
     %     \centering
     %     \includegraphics[width=\textwidth]{figures/pipeline_cartoon.png}
     %     \begin{minipage}{5cm}
     %        \vfill
     %    \end{minipage}
     %     % \caption{SimCLR memorized \textit{yellow garden spider} examples}
     %     \label{fig:pipeline cartoon}
     % \end{subfigure}
     \includegraphics[width=\textwidth]{figures/split_and_pipeline_cartoon.png}
\caption[Overview of testing methodology.]{
Overview of testing methodology. \textbf{Left:} Data is split into \emph{target set} $\calA$, \emph{reference set} $\calB$ and \emph{public set} $\calX$ that are pairwise disjoint. $\calA$ and $\calB$ are used to train two SSL models $\SSL_A$ and $\SSL_B$ in the same manner. $\calX$ is used for KNN decoding or for training an RCDM to reconstruct the input at test time. \textbf{Right:} Given a training image $A_i \in \calA$, we use $\SSL_A$ to embed $\crop{A_i}$ containing only the background, as well as the entire set $\calX$ and find the $k$-nearest neighbors of $\crop{A_i}$ in $\calX$ in the embedding space. These KNN samples can be used directly to infer the foreground object (\emph{i.e.}, class label) in $A_i$ using a KNN classifier, or their embeddings can be averaged as input to the trained RCDM to visually reconstruct the image $A_i$. For instance, the RCDM reconstruction results in Figure \ref{fig:black_swan} (left) when given $\SSL_A(\crop{A_i})$ and results in Figure \ref{fig:black_swan} (right) when given $\SSL_A(\crop{B_i})$ for an image $B_i \in \calB$.
%\textbf{Left:} illustration of the three datasets used in our tests. Two private data sets, $A$ and $B$, of equal size are used to train two SSL models, $\SSL_A$ and $\SSL_B$, respectively. A disjoint public set, $X$, is made available to the memorization test to help decode model embeddings. Memorization is only tested on examples $A_i \in A$ that are unique to set $A$. \textbf{Right:} illustration of inference pipeline used in tests. A periphery cropping that excludes the foreground object is taken from private image $A_i$. The KNN then finds the $k$ public set nearest neighbors of the periphery crop in the embedding space of $\SSL_A$. 
%The $\SSL_A$ representation of these $k$ neighbors and of the crop are used by the conditional generative model, RCDM, to reconstruct the foreground object. The labels of these $k$ neighbors are used to recover the foreground object label. (Not pictured) We repeat this process using reference model $\SSL_B$, not trained on image $A_i$, to determine whether the foreground object is still recoverable by learned correlations, e.g. if black swans were the only objects appearing near water in the data distribution. In this instance, the crop's public set neighbors in $\SSL_B$'s representation space include a variety of water animals like ducks, pelicans, and otters. Meanwhile, with $\SSL_A$, the neighbors are nearly all black swans in the same position as the swan of $A_i$.
}
\label{fig:split_and_pipeline_cartoon}
\end{figure}

\textbf{Distinguishing memorization from correlation.} When measuring \dejavu memorization, it is crucial to differentiate what the model associates through \emph{memorization} and what it associates through \emph{correlation}. Our testing methodology is based on the following intuitive definition.
\mybox{\textbf{Definition:} If an SSL model associates two parts in a training image, we say that it is due to \emph{correlation} if other SSL models trained on a similar dataset from $\mathcal{P}$ without this image would likely make the same association. Otherwise, we say that it is due to \emph{memorization}.}

Notably, such intuition forms the basis for differential privacy (DP; \cite{dwork2006calibrating, dwork2013algorithmic})---the most widely accepted notion of privacy in ML.

\subsection{Testing Methodology for Measuring \emph{Déjà Vu} Memorization}
\label{sec:notation and setup}

In this section, we use the above intuition to measure the extent of \dejavu memorization in SSL. Figure \ref{fig:split_and_pipeline_cartoon} gives an overview of our testing methodology.
\vspace{-0.75em}
\paragraph{Dataset splitting.} We focus on testing \dejavu memorization for SSL models trained on the ImageNet-1K dataset~\citep{imagenet}. Our test first splits the ImageNet training set into three independent and disjoint subsets $\calA$, $\calB$ and $\calX$. The dataset $\calA$ is called the \emph{target set} and $\calB$ is called the \emph{reference set}. The two datasets are used to train two separate SSL models, $\SSL_A$ and $\SSL_B$, called the \emph{target model} and the \emph{reference model}. Finally, the dataset set $\calX$ is used as an auxiliary public dataset to extract information from $\SSL_A$ and $\SSL_B$.
%\footnote{See Appendix \ref{sec:appx splits} for details on how the dataset splits are generated.}.
Our dataset splitting serves the purpose of distinguishing memorization from correlation in the following manner. Given a sample $A_i \in \calA$, if our test returns the same result on $\SSL_A$ and $\SSL_B$ then it is likely due to correlation because $A_i$ is not a training sample for $\SSL_B$. Otherwise, because $\calA$ and $\calB$ are drawn from the same underlying distribution, our test must have inferred some information unique to $A_i$ due to memorization. Thus, by comparing the difference in the test results for $\SSL_A$ and $\SSL_B$, we can measure the degree of \dejavu memorization\footnote{See Appendix \ref{sec:appx splits} for details on how the dataset splits are generated.}.
\vspace{-0.75em}
\paragraph{Extracting foreground and background crops.} Our testing methodology aims at measuring what can be inferred about the foreground object in an ImageNet sample given a background crop. This is made possible because ImageNet provides bounding box annotations for a subset of its training images---around 150K out of 1.3M samples. We split these annotated images equally between $\calA$ and $\calB$. Given an annotated image $A_i$, we treat everything inside the bounding box as the foreground object associated with the image label, denoted $\object{A_i}$. We take the largest possible crop that does not intersect with any bounding box as the background crop (or \emph{periphery crop}), denoted $\crop{A_i}$\footnote{We also present another heuristic in \cref{sec:appx corner crop} which takes a corner crop as the background crop, allowing our test to be run without bounding box annotations.}
%Since the labeled object tends to be at the image's center, the corner crop usually excludes it. }
%Because most images in ImageNet are object centric, an image's corner would not include the foreground object.}.
\vspace{-0.75em}
\paragraph{KNN-based test design.} Joint-embedding SSL approaches encourage the embeddings of random crops of a training image $A_i \in \calA$ to be similar. Intuitively, if the model exhibits \dejavu memorization, it is reasonable to expect that the embedding of $\crop{A_i}$ is similar to that of $\object{A_i}$ since both crops are from the same training image. In other words, $\SSL_A(\crop{A_i})$ encodes information about $\object{A_i}$ that cannot be inferred through correlation. However, decoding such information is challenging as these approaches do not learn a decoder associated with the encoder $\SSL_A$.

Here, we leverage the public set $\calX$ to decode the information contained in $\crop{A_i}$ about $\object{A_i}$. More specifically, we map images in $\calX$ to their embeddings using $\SSL_A$ and extract the $k$-nearest-neighbor (KNN) subset of $\SSL_A(\crop{A_i})$ in $\calX$. We can then decode the information contained in $\crop{A_i}$ in one of two ways:
\begin{itemize}
\item \emph{Label inference:} Since $\calX$ is a subset of ImageNet, each embedding in the KNN subset is associated with a class label. If $\crop{A_i}$ encodes information about the foreground object, its embedding will be close to samples in $\calX$ that have the same class label (\emph{i.e.}, foreground object category). We can then use a KNN classifier to infer the foreground object in $A_i$ given $\crop{A_i}$.
\item \emph{Visual reconstruction:} Following \citet{RCDM}, we train an RCDM---a conditional generative model---on $\calX$ to decode $\SSL_A$ embeddings into images. The RCDM reconstruction can recover qualitative aspects of an image remarkably well, such as recovering object color or spatial orientation using its SSL embedding. Given the KNN subset, we average their SSL embeddings and use the trained RCDM model to visually reconstruct $A_i$.
\end{itemize}
In Section \ref{sec:quant}, we focus on quantitatively measuring \dejavu memorization with label inference, and then use the RCDM reconstruction to visualize \dejavu memorization in Section \ref{sec:visualizing}.
%\section{Quantifying \emph{Déjà Vu} Memorization}
\label{sec:quant}

We apply our testing methodology to quantify a specific form of \dejavu memorization: inferring the foreground object (class label) given a crop of the background.

% \paragraph{Extracting model embeddings.} We test \dejavu memorization on two popular SSL algorithms, SimCLR~\citep{chen2020simclr} and VICReg~\citep{vicreg}.
% %\footnote{We present additional SSL models in \cref{sec:appx simclr results}} 
% As described in Section \ref{sec:related}, these algorithms produce two embeddings given an input image: a \emph{backbone} embedding and a \emph{projector} embedding that is derived by applying a small fully-connected network on top of the backbone embedding. Unless otherwise noted, all SSL embeddings refer to the projector embedding.
% To understand whether \dejavu memorization is particular to SSL, we also evaluate embeddings produced by a supervised model $\CLF_A$ trained on $\calA$. We apply the same set of image augmentations as those used in SSL and train $\CLF_A$ using the cross-entropy loss to predict ground truth labels. 
\vspace{-0.75em}
\paragraph{Extracting model embeddings.} We test \dejavu memorization on a variety of popular SSL algorithms, with a focus on VICReg~\citep{vicreg}. These algorithms produce two embeddings given an input image: a \emph{backbone} embedding and a \emph{projector} embedding that is derived by applying a small fully-connected network on top of the backbone embedding. Unless otherwise noted, all SSL embeddings refer to the projector embedding. 
To understand whether \dejavu memorization is particular to SSL, we also evaluate embeddings produced by a supervised model $\CLF_A$ trained on $\calA$. We apply the same set of image augmentations as those used in SSL and train $\CLF_A$ using the cross-entropy loss to predict ground truth labels. 
\vspace{-0.75em}
\paragraph{Identifying the most memorized samples.} Prior works have shown that certain training samples can be identified as more prone to memorization than others~\citep{feldman2020does, watson2021importance, ye2021enhanced}. Similarly, we provide a heuristic to identify the most memorized samples in our label inference test using confidence of the KNN prediction.
Given a periphery crop, $\crop{A_i}$, let $\KNN_A \big( \crop{A_i} \big) \subseteq \calX$ denote its $k$-nearest neighbors in the embedding space of $\SSL_A$. From this KNN subset we can obtain: \textbf{(1)} $\KNNprob_A \big( \crop{A_i} \big)$, the vector of class probabilities (normalized counts) induced by the KNN subset, and \textbf{(2)} $\KNNconf_A \big( \crop{A_i} \big)$, the negative entropy of the probability vector $\KNNprob_A \big( \crop{A_i} \big)$, as confidence of the KNN prediction. When entropy is low, the neighbors agree on the class of $A_i$ and hence confidence is high. 
% \begin{itemize}[noitemsep, leftmargin=*, topsep=0pt]
%     \item $\KNN_A \big( \crop{A_i} \big)$: The most prevalent class in the KNN subset as prediction for the class label $\cl(A_i)$. 
%     \item $\KNNprob_A \big( \crop{A_i} \big)$: The vector of class probabilities (normalized counts) induced by the KNN subset.
%     \item $\KNNconf_A \big( \crop{A_i} \big)$: Negative entropy of the probability vector $\KNNprob_A \big( \crop{A_i} \big)$ as confidence of the KNN prediction. When entropy is low, the neighbors agree on the class of $A_i$ and hence confidence is high. 
% \end{itemize}
We can sort the confidence score $\KNNconf_A \big( \crop{A_i} \big)$ across samples $A_i$ in decreasing order to identify the most confidently predicted samples, which likely correspond to the most memorized samples when $A_i \in \calA$.

\subsection{Population-level Memorization}
\label{sec:label inference accuracy}

%ORIGINAL FIGURE SETUP IN ARXIV: 
% \input{dejavu_training_epochs.tex}
% \input{dejavu_training_set_size.tex}
%PUT ORIGINAL FIGURES SIDE BY SIDE: 
% \input{dejavu_training_epochs_set_size.tex}
%PUT IN NEW FIGURES: 

\begin{wrapfigure}{r}{0.4\textwidth} 
    \centering
    \includegraphics[width=0.4\textwidth]{figures/dejavu_main.pdf}
    \caption{Accuracy of label inference using the target model (trained on $\calA$) vs. the reference model (trained on $\calB$) on the top $\%$ most confident examples $A_i \in \calA$ using only $\crop{A_i}$. For VICReg, there is a large accuracy gap between the two models, indicating a significant degree of \dejavu memorization.}
    \label{fig:dejavu main}
    \vspace{-2ex}
\end{wrapfigure}

Our first measure of \dejavu memorization is population-level label inference accuracy: \emph{What is the average label inference accuracy over a subset of SSL training images given their periphery crops?} 
To understand how much of this accuracy is due to $\SSL_A$'s \dejavu memorization, we compare with a correlation baseline using the reference model: $\KNN_B$'s label inference accuracy on images $A_i \in \calA$. 
In principle, this inference accuracy should be significantly above chance level ($1/1000$ for ImageNet) because the periphery crop may be highly indicative of the foreground object through correlation, \emph{e.g.}, if the periphery crop is a basketball player then the foreground object is likely a basketball.

Figure \ref{fig:dejavu main} compares the accuracy of $\KNN_A$ to that of $\KNN_B$ when inferring the labels of images in $A_i \in \calA$\footnote{The sets $\calA$ and $\calB$ are exchangeable, and in practice we repeat this test on images from $\calB$ using $\SSL_B$ as the target model and $\SSL_A$ as the reference model, and average the two sets of results.} using $\crop{A_i}$.
Results are shown for VICReg and the supervised model; trends for other models are shown in Appendix \ref{sec:appx simclr results}. For both VICReg and supervised models, inferring the class of $\crop{A_i}$ using $\KNN_B$ (dashed line) through correlation achieves a reasonable accuracy that is significantly above chance level. However, for VICReg, the inference accuracy using $\KNN_A$ (solid red line) is significantly higher, and the accuracy gap between $\KNN_A$ and $\KNN_B$ indicates the degree of \dejavu memorization. We highlight two observations: 
\begin{itemize}
    \item The accuracy gap of VICReg is significantly larger than that of the supervised model. This is especially notable when accounting for the fact that the supervised model is trained to associate randomly augmented crops of images with their ground truth labels. In contrast, VICReg has no label access during training but the embedding of a periphery crop can still encode the image label. 
    \item For VICReg, inference accuracy on the $1\%$ most confident examples is nearly $95\%$, which shows that our simple confidence heuristic can effectively identify the most memorized samples. This result suggests that an adversary can use this heuristic to identify vulnerable training samples to launch a more focused privacy attack.
\end{itemize}
\vspace{-.75em}
\paragraph{The \dejavu score. }
The curves of Figure \ref{fig:dejavu main} show memorization across confidence values for a single training scenario.  To study how memorization changes with different hyperparamters, we extract a single value from these curves: the \dejavu \emph{score} at confidence level $p$. In Figure \ref{fig:dejavu main}, this is the gap between the solid red (or gray) and dashed red (or gray) where confidence ($x$-axis) equal $p\%$. In other words, given the periphery crops of set $\calA$, $\KNN_A$ and $\KNN_B$ separately select and label their top $p\%$ most confident examples, and we report the difference in their accuracy. The \dejavu score captures both the degree of memorization by the accuracy gap and the \emph{ability to identify memorized examples} by the confidence level. If the score is 10\% for $p=33\%$, $\KNN_A$ has 10\% higher accuracy on its most confident third of $\calA$ than $\KNN_B$ does on its most confident third. In the following, we set $p = 20\%$, approximately the largest gap for VICReg (red lines) in Figure \ref{fig:dejavu main}. 
% Specifically, the \dejavu \emph{score} on the top $p\%$ most confident examples is,  
% \begin{equation}
%     \mathrm{DejaVu}(p) = \mathrm{Acc}_{\SSL_A}\big( \calA_{\SSL_A, p}  \big) - \mathrm{Acc}_{\SSL_B}\big( \calA_{\SSL_B, p}  \big) \ ,
%     \label{eqn:dejavu score}
% \end{equation}
% where $\calA_{\SSL_A, p}$
% Here we introduce a DejaVu memorization metric that quantify how much a target model is able to retrieve more class information from a crop than the reference model. We define it as:
% where $p$ is a function that take the $p$ purcent most confident samples.
%Figure \ref{fig:dejavu v. training epochs} shows how \dejavu memorization changes with the number of epochs used to train the embedding model (VICReg and supervised, respectively). The training set size is fixed to 300K samples, and label inference accuracy is computed on the top $20\%$ highest confidence examples. The number of epochs has a very strong influence on the degree of memorization for VICReg as the accuracy gap widens when number of epochs increases. We note that 1000 training epochs is used in several SSL works \citep{vicreg, simclr}. Remarkably, this trend in memorization is \emph{not} reflected in the standard metric for evaluating SSL representations: linear probe accuracy. The gray line in Figure \ref{fig:dejavu v. training epochs} shows the train-test accuracy gap of a linear classifier trained on top of the VICReg embeddings. Although there is a sizeable train-test gap, it does not grow significantly beyond 500 epochs. In contrast, \dejavu memorization (blue line) continues to worsen after 500 epochs. Thus, our test can be used as an alternative to linear probe accuracy to evaluate the memorization of SSL models.
% \vspace{-.75em}

% \paragraph{Comparison with the generalization gap} A network that perform very well on a training set while performing poorly on a test set (assuming the training set and test set sampled uniformly from the same distribution) is probably memorizing the training examples without being able to generalize on the test data. One could expect that measuring the difference in accuracy between the training and test set could give us insights on the degree of \dejavu memorization. However, we show in Figure  \ref{fig:dejavu v. training epochs} and \ref{fig:dejavu v. n} that this is not the case. In fact \dejavu memorization can significantly increase while the train-test gap decrease. In our experiments, we did not find a correlation between \dejavu and generalization.
\vspace{-0.75em}
\paragraph{Comparison with the linear probe train-test gap.} A standard method for measuring SSL performance is to train a linear classifier---what we call a `linear probe'---on its embeddings and compute its performance on a held out test set. From a learning theory standpoint, one might expect the linear probe's train-test accuracy gap to be indicative of memorization: the more a model overfits, the larger is the difference between train set and test set accuracy. However, as seen in Figure \ref{fig:dejavu epochs train set size}, the linear probe gap (dark blue) fails to reveal memorization captured by the \dejavu score (red) \footnote{See section \ref{sec:mitigation} for further discussion of the \dejavu score trends of Figure \ref{fig:dejavu epochs train set size}.}.

% \paragraph{Effect of training epochs.} 
% Figure \ref{fig:dejavu v. training epochs} shows how \dejavu memorization changes with training epochs for VICReg. The training set size is fixed to 300K samples. We observe that the number of epochs has a very strong influence on the degree of memorization for VICReg. From 250 to 1000 epochs, the \dejavu score (red curve) grows threefold: from under 10\% to over 30\%. Remarkably, this trend in memorization is \emph{not} reflected in the standard metric for evaluating SSL representations: linear probe accuracy. The dark blue curve shows the train-test linear probe accuracy gap. Although there is a sizeable train-test gap, it only changes by a few percent beyond 250 epochs. %Thus, our test can be used as an alternative to linear probe accuracy to evaluate the memorization of SSL models.
% \vspace{-.75em}
\begin{figure}[ht]
\label{fig:dejavu epochs and dataset}
\begin{minipage}[t]{0.49\textwidth}
\centering
     \begin{subfigure}[b]{0.48\textwidth}
         \centering
         \includegraphics[width=\textwidth]{figures/deja_vu_vs_epochs.png}
         \vspace{-1.5em}
         \caption{\dejavu vs. epochs}
         \label{fig:dejavu v. training epochs}
     \end{subfigure}
     \begin{subfigure}[b]{0.48\textwidth}
         \centering
         \includegraphics[width=\textwidth]{figures/deja_vu_vs_n.png}
         \vspace{-1.5em}
         \caption{\dejavu vs. train set size}
         \label{fig:dejavu v. n}
     \end{subfigure}~
     \vspace{-0.5em}
    \caption{
    Effect of training epochs and train set size with VICReg on \dejavu score (red) in comparison with linear probe accuracy train-test gap (dark blue). 
    \textbf{Left:} \dejavu score increases with training epochs, indicating growing memorization while the linear probe baseline decreases significantly.  
    \textbf{Right:} \dejavu score stays roughly constant with training set size suggesting that memorization may be problematic even for large datasets. %By comparison, the baseline \emph{declines} by half, spuriously suggesting less memorization. 
    %Both trends are not captured according to the linear probe train-test gap---a common method to evaluate generalization of SSL representations.}
    }
    \label{fig:dejavu epochs train set size}
\end{minipage}
\hfill
\begin{minipage}[t]{0.49\textwidth}
\centering
     \begin{subfigure}[b]{0.48\textwidth}
         \centering
         \includegraphics[width=\textwidth]{figures/vicreg_samples_epochs.pdf}
         \vspace{-1.5em}
         \caption{\dejavu vs. epochs}
         \label{fig:per sample v. training epochs}
     \end{subfigure}
     \begin{subfigure}[b]{0.48\textwidth}
         \centering
         \includegraphics[width=\textwidth]{figures/vicreg_samples_datasets.pdf}
         \vspace{-1.5em}
         \caption{\dejavu vs. train set size}
         \label{fig:per sample v. n}
     \end{subfigure}~
     \vspace{-0.5em}
    \caption{
    \definecolor{part_blue}{rgb}{0.2824, 0.4706, .8157}
	\definecolor{part_red}{rgb}{0.8392, 0.3725, 0.3725}
	\definecolor{part_orange}{rgb}{0.9333, 0.5216, 0.2902}
    Partition of samples $A_i \in \calA$ into the four categories: {\color{gray}unassociated} (not shown), {\color{part_orange}memorized}, {\color{part_red}misrepresented} and {\color{part_blue}correlated} for VICReg. The {\color{part_orange}memorized} samples---those whose labels are predicted by $\KNN_A$ but not by $\KNN_B$---occupy a significantly larger share of the training set than the {\color{part_red}misrepresented} samples---those predicted by $\KNN_B$ but not $\KNN_A$ by chance. %At 1000 epochs, $\approx 15\%$ of the training set is {\color{part_orange}memorized}. The trends across training epochs and training set sizes are consistent with those observed in Figure \ref{fig:dejavu epochs train set size}
    }
    \label{fig:partition attack main}
    \end{minipage}
\vspace{-1em} 
\end{figure}

\iffalse

\begin{minipage}[t]{0.49\textwidth}
\centering
     \begin{subfigure}[b]{0.48\textwidth}
         \centering
         \includegraphics[width=0.95\textwidth]{figures/deja_vu_vs_parameters.png}
         \vspace{-0.4em}
         \caption{\dejavu vs. capacity}
         \label{fig:dejavu v. capacity}
     \end{subfigure}
     \hfill
     \begin{subfigure}[b]{0.48\textwidth}
          \tiny
          \centering
          \setlength{\tabcolsep}{3pt}
          \begin{tabular}{|c|c|c|}
            \hline
            Criteria & DV & Acc P/B \\
            \hline
            Supervised & 8.9 & 55.3/61.1\\
            \hline
            Byol\citep{grill2020byol} & 8.0& 54.3/59.4\\
            \hline
            SimCLR\citep{chen2020simclr} & 10.0 & 44.2/54.1\\
            \hline
            Dino\citep{Dino} & 14.5 & 26.3/55.7 \\
            \hline
            Barlow T.\citep{zbontar2021barlow} & 30.5 & 33.7/54.4\\
            \hline
            VICReg\citep{vicreg} & \textbf{33.2} & 40.3/55.2\\
            \hline
          \end{tabular}
          \vspace{1.3em}
          % \caption{\dejavu (DV) vs. SSL Criterion}
          \caption{\dejavu (DV) vs. Criterion}
          \label{tab:dejavu vs. criterion}
    \end{subfigure}
    \vspace{-0.5em}
    \caption{
    Comparison of \dejavu score for different architectures and training criteria. \textbf{Left:} \dejavu score with VICReg for resnet (purple) and vision transformer (green) architectures versus number of model parameters. As expected, memorization grows with larger model capacity. This trend is more pronounced for convolutional (resnet) than transformer (ViT) architectures. \textbf{Right:} Comparison of \dejavu score and ImageNet validation accuracy (P: using projector embeddings, B: using backbone embeddings) for various SSL criteria. \textbf{Nearly all SSL models have more memorization than the supervised baseline.} 
    % Effect of training epochs and train set size on \dejavu score.
    % \textbf{Left:} \dejavu score increases with higher number of training epochs, indicating worsening memorization.
    % \textbf{Right:} \dejavu score stays roughly constant with training set size. Both trends are not captured according to the linear probe train-test gap---a common method to evaluate generalization of SSL representations.
    }
\end{minipage}
\vspace{-2em} 
\end{figure}

\begin{figure}[ht]
\begin{minipage}[t]{0.49\textwidth}
\centering
     \begin{subfigure}[b]{0.49\textwidth}
         \centering
         \includegraphics[width=\textwidth]{figures/epochs_lb_attk_epochs_acc_top1_legend.pdf}
         \caption{\dejavu vs. epochs}
         \label{fig:dejavu v. training epochs}
     \end{subfigure}
     \begin{subfigure}[b]{0.49\textwidth}
         \centering
         \includegraphics[width=\textwidth]{figures/epochs_lb_attk_datasets_acc_top1_legend.pdf}
         \caption{\dejavu vs. train set size}
         \label{fig:dejavu v. n}
     \end{subfigure}~
     \begin{subfigure}[b]{0.32\textwidth}
         \centering
         \includegraphics[width=0.8\textwidth]{figures/dejavu_vs_parameters.pdf}
         \caption{\dejavu vs. capacity}
         \label{fig:dejavu v. n}
     \end{subfigure}
    \caption{
    Effect of training epochs and train set size on \dejavu score.
    \textbf{Left:} \dejavu score increases with higher number of training epochs, indicating worsening memorization.
    \textbf{Right:} \dejavu score stays roughly constant with training set size. Both trends are not captured according to the linear probe train-test gap---a common method to evaluate generalization of SSL representations.}
    \end{minipage}
\vspace{-1em} 
\end{figure}

\begin{table}[ht]
  \footnotesize
  \centering
  \begin{tabular}{|c|c|}
    \hline
    Supervised & 8.9\\
    \hline
    SimCLR\citep{chen2020simclr} & 10.0\\
    \hline
    Byol\citep{grill2020byol} & 8.0\\
    \hline
    Dino\citep{Dino} & 14.5\\
    \hline
    Barlow T.\citep{zbontar2021barlow} & 30.5\\
    \hline
    VICReg\citep{vicreg} & \textbf{33.2}\\
    \hline
  \end{tabular}
  \caption{DejaVu Score 20\% Conf for various SSL methods.}
  \label{tab:two-row-table}
\end{table}
\vspace{-1em} 
\fi

\iffalse
\begin{figure}[ht]
\begin{minipage}[t]{.49\textwidth}
\centering
     \begin{subfigure}[b]{0.49\textwidth}
         \centering
         \includegraphics[width=\textwidth]{figures/epochs_lb_attk_epochs_acc_top1_legend.pdf}
         \caption{\dejavu vs. epochs}
         \label{fig:dejavu v. training epochs}
     \end{subfigure}
     \hfill
     \begin{subfigure}[b]{0.49\textwidth}
         \centering
         \includegraphics[width=\textwidth]{figures/epochs_lb_attk_datasets_acc_top1_legend.pdf}
         \caption{\dejavu vs. train set size}
         \label{fig:dejavu v. n}
     \end{subfigure}
\caption{
Effect of training epochs and train set size on \dejavu score.
\textbf{Left:} \dejavu score increases with higher number of training epochs, indicating worsening memorization.
\textbf{Right:} \dejavu score stays roughly constant with training set size. Both trends are not captured according to the linear probe train-test gap---a common method to evaluate generalization of SSL representations.}
\label{fig:dejavu epochs and dataset}
\end{minipage}
\hfill
\begin{minipage}[t]{.49\textwidth}
     \centering
     \begin{subfigure}[b]{0.49\textwidth}
         \centering
         \includegraphics[width=\textwidth]{figures/criteria_epochs.pdf}
         \caption{criteria comparison}
         \label{fig:dejavu v. criteria}
     \end{subfigure}
     \hfill
     \begin{subfigure}[b]{0.49\textwidth}
         \centering
         \includegraphics[width=\textwidth]{figures/architecture_epochs.pdf}
         \caption{architecture comparison}
         \label{fig:dejavu v. arch}
     \end{subfigure}
\caption{
Effect of SSL training criteria and model architectures on \dejavu score.
%the accuracy gap between target model (trained on $\calA$) and reference model (trained on $\calB$) making predictions on their 20\% most confident examples.
\textbf{Left:} \dejavu score for various training criteria.
%Barlow and VICReg have the heaviest degree of memorization, while SimCLR and BYOL have the least. 
%Note that we show detailed reconstructions of SimCLR's training data in Section \ref{sec:visualizing} despite its relatively low degree of \dejavu. 
%Regardless, Although SimCLR and BYOL have the least, we  visualize detailed reconstructions with SimCLR in section \ref{sec:mem v corr} 
All SSL models have significantly more \dejavu than the supervised baseline. \textbf{Right:} \dejavu score versus epochs for various training architectures. As expected, lower capacity architectures (Resnet18, Resnet34) reduce \dejavu but not completely. 
}
\label{fig:dejavu criteria and architecture}
\end{minipage}
\vspace{-1em} 
\end{figure}
\fi
% %\begin{figure}[ht]
%%%
%VICREG
%%%
     \centering
     \begin{subfigure}[b]{0.49\textwidth}
         \centering
         \includegraphics[width=\textwidth]{figures/sample_level_training_epochs.pdf}
         \caption{Categories of training samples vs. number of epochs}
         \label{fig:sample level epochs}
     \end{subfigure}
     \hfill
     \begin{subfigure}[b]{0.49\textwidth}
         \centering
         \includegraphics[width=\textwidth]{figures/sample_level_training_set_size.pdf}
         \caption{Categories of training samples vs. training set size}
         \label{fig:sample level training size}
     \end{subfigure}
\caption{
\definecolor{part_blue}{rgb}{0.2824, 0.4706, .8157}
\definecolor{part_red}{rgb}{0.8392, 0.3725, 0.3725}
\definecolor{part_orange}{rgb}{0.9333, 0.5216, 0.2902}
Partition of samples $A_i \in \calA$ into the four categories: {\color{gray}unassociated} (not shown), {\color{part_orange}memorized}, {\color{part_red}misrepresented} and {\color{part_blue}correlated}. The {\color{part_orange}memorized} samples---ones whose labels are predicted by $\KNN_A$ but not by $\KNN_B$---occupy a significantly larger share for VICReg compared to the supervised model, indicating that sample-level \dejavu memorization is more prevalent in VICReg. %The trends across number of training epochs and training set sizes are consistent with those observed in Figures \ref{fig:dejavu epochs and dataset} and \ref{fig:dejavu criteria and architecture}.
}
\label{fig:partition attack main appendix}
\end{figure}
% \paragraph{Effect of training set size.} 
% Figure \ref{fig:dejavu v. n} shows how \dejavu memorization responds to the model's training set size. The number of training epochs is fixed to 1000. Interestingly, training set size appears to have almost \emph{no} influence on the \dejavu score (red line), indicating that memorization is equally prevalent with a 100K dataset and a 500K dataset (which suggests that \dejavu memorization may be detectable for larger datasets). Meanwhile, the linear probe train-test accuracy gap \emph{declines} by half as the dataset size grows, failing to represent the memorization quantified by our test. 
% The trend is completely different according to linear probe accuracy (dark blue line), the train-test gap shrinks substantially when increasing the training set size from 100K to 500K. This highlights that the train-test gap is not able to capture \dejavu memorization. %Our evidence suggests that \dejavu memorization may be detectable even for large-scale training datasets. 
%\vspace{-.75em}

\vspace{-.75em} 
\subsection{Sample-level Memorization}
\label{sec:dissection}

% Section \ref{sec:label inference accuracy} shows the \emph{average} level of \dejavu memorization on a subset of the training set $\calA$. However, this average tell us only what the attacker success rate might be without explicitly describing how much of the datatset is \dejavu memorized.
The \dejavu score shows, \emph{on average}, how much better an adversary can select and classify images when using the target model trained on them. 
This average score does not tell us how many individual images have their label successfully recovered by $\KNN_A$ but not by $\KNN_B$. In other words, how many images are exposed by virtue of \emph{being in training set} $\calA$: a risk notion foundational to differential privacy. 
% However, from the perspective of an individual image $A_i \in \calA$, it is informative to know whether it was correctly classified 
To better quantify what fraction of the dataset is at risk, we perform a sample-level analysis by fixing a sample $A_i \in \calA$ and observing the label inference result of $\KNN_A$ vs. $\KNN_B$.
To this end, we partition samples $A_i \in \calA$ based on the result of label inference into four distinct categories: {\color{gray}\textbf{Unassociated}} - label inferred with neither KNN; {\color{part_orange}\textbf{Memorized}} - label inferred only with $\KNN_A$; {\color{part_red}\textbf{Misrepresented}} - label inferred only with $\KNN_B$; {\color{part_blue}\textbf{Correlated}} - label inferred with both KNNs. 
% \begin{multicols}{2}
% \begin{itemize}
%     \vspace{-.75em}
%     \setlength\itemsep{0.15em}
%     \item {\color{gray}Unassociated}: label inferred with neither KNN   
%     \item {\color{part_orange}Memorized}: label only inferred by $\KNN_A$
%     \item {\color{part_red}Misrepresented}: label only inferred with $\KNN_B$
%     \item {\color{part_blue}Correlated}: label inferred with both KNNs
%     \vspace{-.75em}
% \end{itemize}
% \end{multicols}
Intuitively, {\color{gray}unassociated} samples are ones where the embedding of $\crop{A_i}$ does not encode information about the label. {\color{part_blue}Correlated} samples are ones where the label can be inferred from $\crop{A_i}$ using correlation, \emph{e.g.}, inferring the foreground object is basketball given a crop showing a basketball player. Ideally, the {\color{part_red}misrepresented} set should be empty but contains a small portion of examples due to chance.
\emph{Déjà vu} memorization occurs for {\color{part_orange}memorized} samples where the embedding of $\SSL_B$ does not encode the label but the embedding of $\SSL_A$ does. To measure the pervasiveness of \dejavu memorization, we compare the size of the {\color{part_orange}memorized} and {\color{part_red}misrepresented} sets.
Figure \ref{fig:partition attack main} shows how the four categories of examples change with number of training epochs and training set size. The {\color{gray}unassociated} set is not shown since the total share adds up to one. The {\color{part_red}misrepresented} set remains under $5\%$ and roughly unchanged across all settings, consistent with our explanation that it is due to chance. In comparison, VICReg's {\color{part_orange}memorized} set surpasses $15\%$ at 1000 epochs. Considering that up to 5\% of these memorized examples could also be due to chance, we conclude that \textbf{at least 10\% of VICReg's training set is \dejavu memorized.} 
%is many times larger than its {\color{part_red}misrepresented} set, indicating substantial sample-level \dejavu memorization. 
%In fact, \textbf{it is 15\% of the training set that is \dejavu memorized with VICReg.}
%The trends across different number of training epochs and training set sizes match those observed in Section \ref{sec:label inference accuracy}. % On the other hand, the supervised model's {\color{part_orange}memorized} set is only marginally larger than its {\color{part_red}misrepresented} set.

% The trends across different number of training epochs and training set sizes match those observed in Section \ref{sec:label inference accuracy}: Increasing the number of epochs increases \dejavu memorization (Figure \ref{fig:per sample v. training epochs}), while increasing the training set size does not appear to reduce \dejavu memorization (Figure \ref{fig:per sample v. n}). 
%\section{Visualizing \emph{Déjà Vu} Memorization}
\label{sec:visualizing}
Beyond enabling label inference using a periphery crop, we show that \dejavu memorization allows the SSL model to encode other forms of information about a training image. Namely, we train an RCDM \citep{RCDM} on the public dataset $\calX$ and use it to visually reconstruct training images given their periphery crop.
We aim to answer the following two questions: \textbf{(1)} Can we visualize the distinction between correlation and \dejavu memorization? \textbf{(2)} What foreground object details can be extracted from the SSL model beyond class label? 
% \begin{enumerate}[noitemsep, leftmargin=*, topsep=0pt]
%     \item Can we visualize the distinction between correlation and \dejavu memorization? 
%     \item What foreground object details can be extracted from the SSL model beyond class label? 
% \end{enumerate}
\vspace{-0.5em}
\paragraph{Reconstruction pipeline.}
RCDM is a conditional generative model that is trained on the \emph{backbone embedding} of images $X_i \in \calX$ to generate an image that resembles $X_i$. All training images are first face-blurred for privacy purposes. \citet{RCDM} showed that the backbone embedding of SSL models contains more low-level information about the image, making them better suited for conditioning the RCDM.
At test time, following the pipeline in Figure \ref{fig:split_and_pipeline_cartoon}, we first use the projector embedding to find the KNN subset for the periphery crop, $\crop{A_i}$, and then average their backbone embeddings as input to the RCDM model. Ideally, when the public set contains enough representative images, the average representation of the KNN subset encodes objects present in $A_i$, and the RCDM model decodes this representation to visualize these objects.
% \begin{figure}[ht]
%%%
%VICREG
%%%
     \centering
     \begin{subfigure}[b]{0.49\textwidth}
         \centering
         \includegraphics[width=\textwidth]{figures/sample_level_training_epochs.pdf}
         \caption{Categories of training samples vs. number of epochs}
         \label{fig:sample level epochs}
     \end{subfigure}
     \hfill
     \begin{subfigure}[b]{0.49\textwidth}
         \centering
         \includegraphics[width=\textwidth]{figures/sample_level_training_set_size.pdf}
         \caption{Categories of training samples vs. training set size}
         \label{fig:sample level training size}
     \end{subfigure}
\caption{
\definecolor{part_blue}{rgb}{0.2824, 0.4706, .8157}
\definecolor{part_red}{rgb}{0.8392, 0.3725, 0.3725}
\definecolor{part_orange}{rgb}{0.9333, 0.5216, 0.2902}
Partition of samples $A_i \in \calA$ into the four categories: {\color{gray}unassociated} (not shown), {\color{part_orange}memorized}, {\color{part_red}misrepresented} and {\color{part_blue}correlated}. The {\color{part_orange}memorized} samples---ones whose labels are predicted by $\KNN_A$ but not by $\KNN_B$---occupy a significantly larger share for VICReg compared to the supervised model, indicating that sample-level \dejavu memorization is more prevalent in VICReg. %The trends across number of training epochs and training set sizes are consistent with those observed in Figures \ref{fig:dejavu epochs and dataset} and \ref{fig:dejavu criteria and architecture}.
}
\label{fig:partition attack main appendix}
\end{figure}
%\begin{figure*}[t!]
%%%
%DAM
%%%
     \centering
     \begin{subfigure}[b]{0.49\textwidth}
         \centering
         \includegraphics[width=\textwidth]{figures/dam_corr.png}
         \caption{A {\color{part_blue}correlated} dam example}
         \label{fig:dam correlated}
     \end{subfigure}
     \hfill
     \begin{subfigure}[b]{0.49\textwidth}
         \centering
         \includegraphics[width=\textwidth]{figures/dam_mem.png}
         \caption{A {\color{part_orange}memorized} dam example}
         \label{fig:dam memorized}
     \end{subfigure}
\caption{
{\color{part_blue}Correlated} and {\color{part_orange}Memorized} examples from the \emph{dam} class. Both $\SSL_A$ and $\SSL_B$ are SimCLR models.
\textbf{Left:} The periphery crop (pink square) contains a concrete structure that is often present in images of dams. Consequently, the trained RCDM can reconstruct the foreground object using representations from both $\SSL_A$ and $\SSL_B$ through this correlation.
\textbf{Right:} The periphery crop only contains a patch of water. The embedding produced by $\SSL_B$ only contains enough information to infer that the foreground object is related to water, as reflected by its KNN set and RCDM reconstruction. In contrast, the embedding produced by $\SSL_A$ memorizes the association of this patch of water with dam and the RCDM can visualize the embedding to produce images of dams.
}
\vspace{-1ex}
\label{fig:mem v corr dam}
\end{figure*}


\begin{figure*}[t!]
%%%
%DAM
%%%
     \centering
     \begin{subfigure}[b]{0.49\textwidth}
         \centering
         \includegraphics[width=\textwidth]{figures/dam_corr.png}
         \caption{A {\color{part_blue}correlated} dam example}
         \label{fig:dam correlated}
     \end{subfigure}
     \hfill
     \begin{subfigure}[b]{0.49\textwidth}
         \centering
         \includegraphics[width=\textwidth]{figures/dam_mem.png}
         \caption{A {\color{part_orange}memorized} dam example}
         \label{fig:dam memorized}
     \end{subfigure}
\caption[Correlated and Memorized examples from the \emph{dam} class.]{
Correlated and Memorized examples from the \emph{dam} class. Both $\SSL_A$ and $\SSL_B$ are SimCLR models.
\textbf{Left:} The periphery crop (pink square) contains a concrete structure that is often present in images of dams. Consequently, the trained RCDM can reconstruct the foreground object using representations from both $\SSL_A$ and $\SSL_B$ through this correlation.
\textbf{Right:} The periphery crop only contains a patch of water. The embedding produced by $\SSL_B$ only contains enough information to infer that the foreground object is related to water, as reflected by its KNN set and RCDM reconstruction. In contrast, the embedding produced by $\SSL_A$ memorizes the association of this patch of water with dam and the RCDM can visualize the embedding to produce images of dams.
}
\label{fig:mem v corr dam}
\end{figure*}


\begin{figure}[t!]
%%%
%BADGER
%%%
     \centering
     \begin{subfigure}[b]{0.49\textwidth}
         \centering
         \includegraphics[width=\textwidth]{figures/euro_badgers.png}
         \caption{{\color{part_orange}Memorized} European badgers}
         \label{fig:euro badgers}
     \end{subfigure}
     \hfill
     \begin{subfigure}[b]{0.49\textwidth}
         \centering
         \includegraphics[width=\textwidth]{figures/amer_badgers.png}
         \caption{{\color{part_orange}Memorized} American badgers}
         \label{fig:amer badgers}
     \end{subfigure}
\caption[Visualization of \dejavu memorization beyond class label.]{
Visualization of \dejavu memorization beyond class label. Both $\SSL_A$ and $\SSL_B$ are VICReg models. 
The four images shown belong to the memorized set of $\SSL_A$ from the \emph{badger} class. RCDM reconstruction using embeddings from $\SSL_A$ can reveal not only the correct class label, but also the specific badger species: \emph{European} (left) and \emph{American} (right). Such information does not appear to be memorized by the reference model $\SSL_B$.
} 
\label{fig:in class badger}
\end{figure}


% \subsection{Visualizing Correlation vs. Memorization}
\label{sec:mem v corr}
\vspace{-0.5em} 
\paragraph{Visualizing Correlation vs. Memorization.}
Figure \ref{fig:mem v corr dam} shows examples of dams from the {\color{part_blue}correlated} set (left) and the {\color{part_orange}memorized} set (right) as defined in Section \ref{sec:dissection}, along with the associated KNN set and RCDM reconstruction. Both $\SSL_A$ and $\SSL_B$ are SimCLR models. In Figure \ref{fig:dam correlated}, the periphery crop is represented by the pink square, which contains concrete structure attached to the dam's main structure. As a result, both $\SSL_A$ and $\SSL_B$ produce embeddings of $\crop{A_i}$ whose KNN set in $\calX$ consist of dams, \emph{i.e.}, there is a correlation between the concrete structure in $\crop{A_i}$ and the foreground dam. The RCDM reconstructions also consist of dams or structures that closely resemble dams. 
In Figure \ref{fig:dam memorized}, the periphery crop only contains a patch of water, which does not strongly correlate with dams in the ImageNet distribution. Evidently, the reference model $\SSL_B$ embeds $\crop{A_i}$ close to that of other objects commonly found in water, such as sea turtle and submarine. In contrast, the KNN set according to $\SSL_A$ all contain dams despite the vast number of alternative possibilities within the ImageNet classes, and the RCDM reconstruction outputs dams as well which highlight memorization in $\SSL_A$ between this specific patch of water and the dam. %\footnote{See Appendix \ref{sec:appx visualization} to see the same trend in the \emph{yellow garden spider} class.}


% \subsection{Visualizing Memorization Beyond Class Label}
% \label{sec:in class variation}
\vspace{-0.5em} 
\paragraph{Visualizing Memorization Beyond Class Label.}
We now use our reconstruction algorithm to show that \dejavu memorization can be exploited to reveal detailed information beyond class label. Figure \ref{fig:in class badger} shows four examples of badgers from the {\color{part_orange}memorized} set. In all four images, the periphery crop (pink square) does not contain any indication that the foreground object is a badger. Despite this, the KNN set and the RCDM reconstruction using $\SSL_A$ consistently produce images of badgers, while the same does not hold for $\SSL_B$.
More interestingly, reconstructions using $\SSL_A$ in Figure \ref{fig:euro badgers} all contain \emph{European} badgers, while reconstructions in Figure \ref{fig:amer badgers} all contain \emph{American} badgers, accurately reflecting the species of badger present in the respective training images. Since ImageNet-1K does \emph{not} differentiate between these two species of badgers, our reconstructions show that SSL models can memorize information that is highly specific to a training sample beyond its class label\footnote{See Appendix \ref{sec:appx visualization} for additional visualization experiments.}.%\footnote{See Appendix \ref{sec:appx visualization} for the same trend in the \emph{aircraft carrier} class.}.





%\vspace{-.5em} 
\section{Mitigation of \dejavu memorization}
\label{sec:mitigation}
% We do not have an understanding on why \dejavu occur so strongly in some SSL pretraining, however we present additional experiments that shed light on which parameters have the biggest impact on \dejavu memorization.
\begin{figure}[ht]
\label{fig:mitigations}
\begin{minipage}[t]{0.5\textwidth}
\centering
     \begin{subfigure}[b]{0.47\textwidth}
         \centering
         \includegraphics[width=\textwidth]{figures/dejavu_vicreg_param.png}
         \vspace{-1.5em}
         \caption{Loss hyper-parameter}
         \label{fig:dejavu v. invariance}
     \end{subfigure}
     \begin{subfigure}[b]{0.49\textwidth}
         \centering
         \includegraphics[width=\textwidth]{figures/deja_vu_vs_layer.png}
         \vspace{-1.5em}
         \caption{Guillotine regularization}
         \label{fig:dejavu v. guillotine}
     \end{subfigure}~
     \vspace{-0.5em}
    \caption[Effect of two kinds of hyper-parameters on VICReg memorization. ]{
    Effect of two kinds of hyper-parameters on VICReg memorization. \textbf{Left:} \dejavu score (red) versus the \emph{invariance} loss parameter, $\lambda$, used in the VICReg criterion (100k dataset). Larger $\lambda$ significantly reduces \dejavu, with minimal effect on linear probe validation performance (green). $\lambda = 25$ (near maximum \dejavu) is recommended in the original paper \textbf{Right:} \dejavu score versus projector layer---guillotine regularization \cite{Guillotine}---from projector to backbone. Removing the projector can significantly reduce \dejavu. Appendix \ref{sec:guillotine} shows that the backbone still can memorize, however; we demonstrate reconstructions using the SimCLR backbone.
    }
\end{minipage}
\hfill
\begin{minipage}[t]{0.48\textwidth}
\centering
     \begin{subfigure}[b]{0.46\textwidth}
         \centering
         \includegraphics[width=\textwidth]{figures/deja_vu_vs_parameters.png}
         \vspace{-1.3em}
         \caption{\dejavu vs. capacity}
         \label{fig:dejavu v. capacity}
     \end{subfigure}
     \hfill
     \begin{subfigure}[b]{0.52\textwidth}
          \tiny
          \centering
          \setlength{\tabcolsep}{3pt}
          \begin{tabular}{|c|c|c|}
            \hline
            Criteria & DV & Acc P/B \\
            \hline
            Supervised & 8.9 & 55.3/61.1\\
            \hline
            Byol\citep{grill2020byol} & 8.0& 54.3/59.4\\
            \hline
            SimCLR\citep{chen2020simclr} & 10.0 & 44.2/54.1\\
            \hline
            Dino\citep{Dino} & 14.5 & 26.3/55.7 \\
            \hline
            Barlow T.\citep{zbontar2021barlow} & 30.5 & 33.7/54.4\\
            \hline
            VICReg\citep{vicreg} & \textbf{33.2} & 40.3/55.2\\
            \hline
          \end{tabular}
          \vspace{1.3em}
          % \caption{\dejavu (DV) vs. SSL Criterion}
          \caption{\dejavu (DV) vs. Criterion}
          \label{tab:dejavu vs. criterion}
    \end{subfigure}
    \vspace{-1.4em}
    \caption[Effect of model architecture and criterion on \dejavu memorization.]{
    %Comparison of \dejavu score for different architectures and training criteria. 
    Effect of model architecture and criterion on \dejavu memorization. 
    \textbf{Left:} \dejavu score with VICReg for resnet (purple) and vision transformer (green) architectures versus number of model parameters. As expected, memorization grows with larger model capacity. This trend is more pronounced for convolutional (resnet) than transformer (ViT) architectures. \textbf{Right:} Comparison of \dejavu score 20\% conf. and ImageNet linear probe validation accuracy (P: using projector embeddings, B: using backbone embeddings) for various SSL criteria. %\textbf{Nearly all SSL models have more memorization than the supervised baseline.} 
    % Effect of training epochs and train set size on \dejavu score.
    % \textbf{Left:} \dejavu score increases with higher number of training epochs, indicating worsening memorization.
    % \textbf{Right:} \dejavu score stays roughly constant with training set size. Both trends are not captured according to the linear probe train-test gap---a common method to evaluate generalization of SSL representations.
    }
    \end{minipage}
\end{figure}
We cannot yet make claims on why \dejavu occurs so strongly for some SSL training settings and not for others. To gain some intuition for future work, we present additional observations that shed light on which parameters have the most salient impact on \dejavu memorization.
\vspace{-.75em}
\paragraph{Déjà vu memorization worsens by increasing number of training epochs.} 
Figure \ref{fig:dejavu v. training epochs} shows how \dejavu memorization changes with number of training epochs for VICReg. The training set size is fixed to 300K samples. From 250 to 1000 epochs, the \dejavu score (red curve) grows \emph{threefold}: from under 10\% to over 30\%. Remarkably, this trend in memorization is \emph{not} reflected by the linear probe gap (dark blue), which only changes by a few percent beyond 250 epochs. 

%\vspace{-.75em}
\paragraph{Training set size has minimal effect on \dejavu memorization.} Figure \ref{fig:dejavu v. n} shows how \dejavu memorization responds to the model's training set size. The number of training epochs is fixed to 1000. Interestingly, training set size appears to have almost \emph{no} influence on the \dejavu score (red line), indicating that memorization is equally prevalent with a 100K dataset and a 500K dataset. This result suggests that \dejavu memorization may be detectable even for large datasets. Meanwhile, the standard linear probe train-test accuracy gap \emph{declines} by more than half as the dataset size grows, failing to represent the memorization quantified by our test. 
% The trend is completely different according to linear probe accuracy (dark blue line), the train-test gap shrinks substantially when increasing the training set size from 100K to 500K. This highlights that the train-test gap is not able to capture \dejavu memorization. Our evidence suggests that \dejavu memorization may be detectable even for large-scale training datasets. 
\vspace{-0.5em}
\paragraph{Training loss hyper-parameter has a strong effect.} 
%We show in Figure \ref{fig:dejavu v. training epochs} that the number of training epochs is an important factor that can increase significantly \dejavu memorization. In contrast, the dataset size does not impact much \dejavu as shown in Figure \ref{fig:dejavu epochs train set size}. 
Loss hyper-parameters, like VICReg's invariance coefficient (Figure \ref{fig:dejavu v. invariance}) or SimCLR's temperature parameter (Appendix Figure \ref{fig:simclr temperature}) significantly impact \dejavu with minimal impact on the linear probe validation accuracy.

\vspace{-0.5em}
\paragraph{Some SSL criteria promote stronger \dejavu memorization.} Table \ref{tab:dejavu vs. criterion} demonstrates that the degree of memorization varies widely for different training criteria. VICReg and Barlow Twins have the highest \dejavu scores while SimCLR and Byol have the lowest.
%\footnote{We show detailed reconstructions of SimCLR's training data in Section \ref{sec:visualizing} despite its relatively low degree of \dejavu.}.
With the exception of Byol, all SSL models have more \dejavu memorization than the supervised model. Interestingly, different criteria can lead to similar linear probe validation accuracy and very different degrees of \dejavu as seen with SimCLR and Barlow Twins. Note that low degrees of \dejavu can still risk training image reconstruction, as exemplified by the SimCLR reconstructions in Figures \ref{fig:mem v corr dam} and \ref{fig:mem v corr spider}. 
%\vspace{-1em}
\vspace{-0.5em}
\paragraph{Larger models have increased \dejavu memorization.} Figure \ref{fig:dejavu v. capacity} validates the common intuition that lower capacity architectures (Resnet18/34) result in less memorization than their high capacity counterparts (Resnet50/101). 
% \begin{wrapfigure}{r}{0.25\textwidth} 
%     \centering
%     \includegraphics[width=0.25\textwidth]{figures/attk_layer_acc_top1_legend.pdf}
%     \caption{\dejavu memorization versus layer from backbone (0) to projector output (3).}
%     \label{fig:dejavu vs layer}
%     \vspace{-8ex}
% \end{wrapfigure}
We see the same trend for vision transformers as well. %This comes with a tradeoff, since reduced model capacity can result in a nontrivial degradation of representation quality\cite{vicreg, simclr}.  
\vspace{-0.5em}
\paragraph{Guillotine regularization can help reduce \dejavu memorization.} Previous experiments were done using the projector embedding. In Figure \ref{fig:dejavu v. guillotine}, we present how Guillotine regularization\citep{Guillotine} (removing final layers in a trained SSL model) impacts \dejavu with VICReg\footnote{Further experiments are available in Appendix \ref{sec:guillotine}.}. Using the backbone embedding instead of the projector embedding seems to be the most straightforward way to mitigate \dejavu memorization. However, as demonstrated in Appendix \ref{sec:appx backbone results}, backbone representation with low \dejavu score can still be leveraged to reconstruct some of the training images.

\section{Conclusion}
\label{sec:conclusion}

We defined and analyzed \dejavu memorization, a notion of unintended memorization of partial information in image data. As shown in Sections \ref{sec:quant} and \ref{sec:visualizing}, SSL models can largely exhibit \dejavu memorization on their training data, and this memorization signal can be extracted to infer or visualize image-specific information.
Since SSL models are becoming increasingly widespread as foundation models for image data, negative consequences of \dejavu memorization can have profound downstream impact and thus deserves further attention. 
Future work should focus on understanding how \dejavu emerges in the training of SSL models and why methods like Byol are much more robust to \dejavu than VICReg and Barlow Twins. In addition, trying to characterize which data points are the most at risk of \dejavu could be crucial to get a better understanding on this phenomenon. 
