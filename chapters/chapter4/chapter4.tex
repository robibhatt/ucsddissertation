\graphicspath{{./chapters/chapter4/}}
%\newtheorem{thm}{Theorem}
%\newtheorem{lem}[thm]{Lemma}
%\DeclareMathOperator*{\argmax}{arg\,max}
%\DeclareMathOperator*{\argmin}{arg\,min}
\newtheorem{theorem}[thm]{Theorem}
\newtheorem{proposition}[thm]{Proposition}
\newtheorem{lemma}[thm]{Lemma}
\newtheorem{corollary}[thm]{Corollary}
%\theoremstyle{definition}
\newtheorem{definition}[thm]{Definition}
\newtheorem{assumption}[thm]{Assumption}
%\theoremstyle{remark}
\newtheorem{remark}[thm]{Remark}

\def\c{cr}
\def\hc{\hat{cr}}
\def\d{\Lambda}
\def\ind{\mathbbm{1}}
\def\oc{Online\_Cluster}
\def\mem{\mathcal M}
\def\E{\mathbb{E}}
\def\R{\mathbb{R}}
\def\O{\mathcal{O}}
\def\calP{\mathcal{P}}
\def\dc{Data\_Copy\_Detect}
\def\a{\kappa}
\def\hq{\widehat{q(B(x, r))}}
\def\alpaca{\epsilon}

\chapter{Data-Copying in Generative Models: A Formal Framework} 

\section{Introduction}

Deep generative models have shown impressive performance. However, given how large, diverse, and uncurated their training sets are, a big question is whether, how often, and how closely they are memorizing their training data. This question has been of considerable interest in generative modeling~\citep{lopez2016revisiting,XHYGSWK18} as well as supervised learning~\citep{BBFST21, Feldman20}. However, a clean and formal definition of memorization that captures the numerous complex aspects of the problem, particularly in the context of continuous data such as images, has largely been elusive.

For generative models,~\cite{MCD2020} proposed a formal definition of memorization called ``data-copying'', and showed that it was orthogonal to various prior notions of overfitting such as mode collapse~\citep{TT20}, mode dropping~\citep{YFWYC20}, and precision-recall~\citep{SBLBG18}. Specifically, their definition looks at three datasets -- a training set, a set of generated example, and an independent test set. Data-copying happens when the training points are considerably closer on average to the generated data points than to an independently drawn test sample. Otherwise, if the training points are further on average to the generated points than test, then there is underfitting. They proposed a three sample test to detect this kind of data-copying, and empirically showed that their test had good performance.

\begin{figure}[ht]
\centering
	\includegraphics[width=.45\textwidth]{page_2_figure_yo.png}
	\caption{In this figure, the blue points are sampled from the halfmoons dataset (with Gaussian noise). The red points are sampled from a generated distribution that is a mixture of (40 \%) blatant data copier (that outputs a random subset of the training set), and (60 \%) a noisy underfit version of halfmoons. Although the generated distribution is clearly doing some form of copying at points $x_1$ and $x_2$, detecting this is challenging because of the canceling effect of the underfit points.}
	
	\label{fig:page_2_figure}
\end{figure}

However, despite its practical success, this method may not capture even blatant cases of memorization. To see this, consider the example illustrated in Figure \ref{fig:page_2_figure}, in which a generated model for the halfmoons dataset outputs one of its training points with probability $0.4$, and otherwise outputs a random point from an underfit distribution. When the test of~\cite{MCD2020} is applied to this distribution, it is unable to detect any form of data copying; the generated samples drawn from the underfit distribution are sufficient to cancel out the effect of the memorized examples. Nevertheless, this generative model is clearly an egregious memorizer as shown in points $x_1$ and $x_2$ of Figure \ref{fig:page_2_figure}.

This example suggests a notion of \textit{point-wise} data copying, where a model $q$ can be thought of as copying a given training point $x$. Such a notion would be able to detect $q$'s behavior nearby $x_1$ and $x_2$ regardless of the confounding samples that appear at a global level. This stands in contrast to the more global distance based approach taken in Meehan et. al. which is unable to detect such instances. Motivated by this, we propose an alternative point-by-point approach to defining data-copying.

We say that a generative model $q$  data-copies an individual training point, $x$, if it has an unusually high concentration in a small area centered at $x$. Intuitively, this implies $q$ is highly likely to output examples that are very similar to $x$. In the example above, this definition would flag $q$ as copying $x_1$ and $x_2$. 

To parlay this definition into a global measure of data-copying, we define the overall \textit{data-copying rate} as the total fraction of examples from $q$ that are copied from some training example. In the example above, this rate is $40\%$, as this is the fraction of examples that are blatant copies of the training data.

\begin{figure}[ht]
    \begin{subfigure}{0.31\textwidth}\includegraphics[width=\linewidth]{default.png}
    \end{subfigure}\hspace*{\fill}
	\begin{subfigure}{0.31\textwidth}\includegraphics[width=\linewidth]{add_regions.png}
	\end{subfigure}\hspace*{\fill}
	\begin{subfigure}{0.31\textwidth}\includegraphics[width=\linewidth]{data_copy.png}
	\end{subfigure}
	\caption{In the three panels above, the blue points are a training sample from $p$, and the red points are generated examples from $q$. In the middle panel, we highlight in green regions that are defined to be \textit{data-copying regions}, as $q$ overrepresents them with comparison to $p$. In the third panel, we then color all points from $q$ that are considered to be copied green.}
	
	\label{fig:triptic}
\end{figure}

Next, we consider how to detect data-copying according to this definition. To this end, we provide an algorithm, \dc{}, that outputs an estimate for the overall data-copying rate. We then show that under a natural smoothness assumption on the data distribution, which we call \textit{regularity}, \dc{} is able to guarantee an accurate estimate of the total data-copying rate. We then give an upper bound on the amount of data needed for doing so. 

We complement our algorithm with a lower bound on the minimum amount of a data needed for data-copying detection. Our lower bound also implies that some sort of smoothness condition (such as regularity) is necessary for guaranteed data-copying detection; otherwise, the required amount of data can be driven arbitrarily high.

\subsection{Related Work}

Recently, understanding failure modes for generative models has been an important growing body of work e.g. \citep{SGZCRC16, RW18, SBLBG18}. However, much of this work has been focused on other forms of overfitting, such as mode dropping or mode collapse.

A more related notion of overfitting is \textit{memorization} \citep{lopez2016revisiting,XHYGSWK18, C18}, in which a model outputs exact copies of its training data. This has been studied in both supervised \citep{BBFST21, Feldman20} and unsupervised \citep{BGWC21, CHCW21} contexts. Memorization has also been considered in language generation models \cite{Carlini22}. 

The first work to explicitly consider the more general notion of \textit{data-copying} is \citep{MCD2020}, which gives a three sample test for data-copy detection. We include an empirical comparison between our methods in Section \ref{sec:experiments}, where we demonstrate that ours is able to capture certain forms of data-copying that theirs is not. 

Finally, we note that this work focuses on detecting natural forms of memorization or data-copying, that likely arises out of poor generalization, and is not concerned with detecting \textit{adversarial} memorization or prompting, such as in \cite{Carlini19}, that are designed to obtain sensitive information about the training set. This is reflected in our definition and detection algorithm which look at the specific generative model, and not the algorithm that trains it.  Perhaps the best approach to prevent adversarial  memorization is training the model with differential privacy~\cite{Dwork06}, which ensures that the model does not change much when one training sample changes. However such solutions come at an utility cost. 

\section{A Formal Definition of Data-Copying}

We begin with the following question: what does it mean for a generated distribution $q$ to copy a single training example $x$? Intuitively, this means that $q$ is guilty of overfitting $x$ in some way, and consequently produces examples that are very similar to it. 

However, determining what constitutes a `very similar'  generated example must be done contextually. Otherwise the original data distribution, $p$, may itself be considered a copier, as it will output points nearby $x$ with some frequency depending on its density at $x$. Thus, we posit that $q$ data copies training point $x$ if it has a significantly higher concentration nearby $x$ than $p$ does. We express this in the following definition. 

\begin{definition}\label{defn:data_copy}
Let $p$ be a data distribution, $S \sim p^n$ a training sample, and $q$ be a generated distribution trained on $S$. Let $x \in S$ be a training  point, and let $\lambda > 1$ and $0 < \gamma < 1$ be constants. A generated example $x' \sim q$ is said to be a \textbf{$(\lambda, \gamma)$-copy} of $x$ if there exists a ball $B$ centered at $x$ (i.e. $\{x': ||x' - x|| \leq r\}$) such that following hold:
\begin{itemize}
	\item $x' \in B$.
	\item $q(B) \geq \lambda p(B)$
	\item $p(B) \leq \gamma$
\end{itemize}
\end{definition}

Here $q(B)$ and $p(B)$ denote the probability mass assigned to $B$ by $p$ and $q$ respectively.

The parameters $\lambda$ and $\gamma$ are user chosen parameters that characterize data-copying. $\lambda$ represents the rate at which $q$ must overrepresent points close to $x$, with higher values of $\lambda$ corresponding to more egregious examples of data-copying. $\gamma$ represents the maximum size (by probability mass) of a region that is considered to be data-copying -- the ball $B$ represents all points that are ``copies" of $x$. Together, $\lambda$ and $\gamma$ serve as practitioner controlled knobs that characterize data-copying about $x$.

Our definition is illustrated in Figure \ref{fig:triptic} -- the training data is shown in blue, and generated samples are shown in red. For each training point, we highlight a region (in green) about that point in which the red density is much higher than the blue density, thus constituting data-copying. The intuition for this is that the red points within any ball can be thought of as ``copies" of the blue point centered in the ball.

Having defined data-copying with respect to a single training example, we can naturally extend this notion to the entire training dataset. We say that $x' \sim q$ is copied from training set $S$ if $x'$ is a $(\lambda,\gamma)$-copy of some training example $x \in S$. We then define the \textit{data-copy rate} of $q$ as the fraction of examples it generates that are copied from $S$. Formally, we have the following: 

\begin{definition}
Let $p, S, q, \lambda,$ and $\gamma$ be as defined in Definition \ref{defn:data_copy}. Then the \textbf{data-copy rate}, $\c\left(q, \lambda, \gamma\right)$ of $q$ (with respect to $p, S$) is the fraction of examples from $q$ that are $(\lambda, \gamma)$-copied. That is, $$\c\left(q, \lambda, \gamma\right) = \Pr_{x' \sim q}[q\text{ }(\lambda,\gamma)\text{-copies }x'].$$ In cases where $\lambda, \gamma$ are fixed, we use $\c_q = \c(q, \lambda, \gamma)$ to denote the data-copy rate.
\end{definition}

Despite its seeming global nature, $\c_q$ is simply an aggregation of the point by point data-copying done by $q$ over its entire training set. As we will later see, estimating $\c_q$ is often reduced to determining which subset of the training data $q$ copies. 

\subsection{Examples of data-copying}

We now give several examples illustrating our definitions. In all cases, we let $p$ be a data distribution, $S$, a training sample from $p$, and $q$, a generated distribution that is trained over $S$. 

\paragraph{The uniform distribution over $S$:} In this example, $q$ is an egregious data copier that memorizes its training set and randomly outputs a training point. This can be considered as the canonical \textit{worst} data copier. This is reflected in the value of $\c_q$ -- if $p$ is a continuous distribution with finite probability density, then for any $x \in S$, there exists a ball $B$ centered at $x$ for which $q(B) >> p(B)$. It follows that $q$ $(\lambda,\gamma)$- copies $x$ for all $x \in S$ which implies that $\c_q = 1$.

\paragraph{The perfect generative model: $q = p$:} In this case, $q(B) = p(B)$ for all balls, $B$, which implies that $q$ does not perform any data-copying (Definition \ref{defn:data_copy}). It follows that $\c_q = 0$, matching the intuition that $q$ does not data-copy at all.

\paragraph{Kernel Density Estimators:} Finally, we consider a more general situation, where $q$ is trained by a \textit{kernel density estimator} (KDE) over $S \sim p^n$. Recall that a kernel density estimator outputs a generated distribution, $q$, with pdf defined by $$q(x) = \frac{1}{n\sigma_n}\sum_{x_i \in S} K\left(\frac{x - x_i}{\sigma_n}\right).$$ Here, $K$ is a kernel similarity function, and $\sigma_n$ is the bandwidth parameter. It is known that for $\sigma_n = O(n^{-1/5})$, $q$ converges towards $p$ for sufficiently well behaved probability distributions. 

Despite this guarantee, KDEs intuitively appear to perform some form of data-copying -- after all they implicitly include each training point in memory as it forms a portion of their outputted pdf. However, recall that our main focus is in understanding \textit{overfitting} due to data-copying. That is, we view data-copying as a function of the outputted pdf, $q$, and not of the training algorithm used. 

To this end, for KDEs the question of data-copying reduces to the question of whether $q$ overrepresents areas around its training points. As one would expect, this occurs \textit{before} we reach the large sample limit. This is expressed in the following theorem.

\begin{theorem}\label{thm:KDE}
Let $1 < \lambda$ and $\gamma > 0$. Let $\sigma_n$ be a sequence of bandwidths and $K$ be any regular kernel function. For any $n > 0$ there exists a probability distribution $\pi$ with full support over $\R^d$ such that with probability at least $\frac{1}{3}$ over $S \sim \pi^n$, a KDE trained with bandwidth $\sigma_n$ and kernel function $K$ has data-copy rate $\c_q \geq \frac{1}{10}$.
\end{theorem}

This theorem completes the picture for KDEs with regards to data-copying -- when $n$ is too low, it is possible for the KDE to have a significant amount of data-copying, but as $n$ continues to grow, this is eventually smoothed out.

\paragraph{The Halfmoons dataset}

Returning to the example given in Figure \ref{fig:page_2_figure}, observe that our definition exactly captures the notion of data-copying that occurs at points $x_1$ and $x_2$. For even strict choices of $\lambda$ and $\gamma$, Definition \ref{defn:data_copy} indicates that the red distribution copies both $x_1$ and $x_2$. Furthermore, the data-copy rate, $\c_q$, is $40\%$ by construction, as this is the proportion of points that are outputted nearby $x_1$ and $x_2$.

\subsection{Limitations of our definition}\label{sec:limitations}

Definition \ref{defn:data_copy} implicitly assumes that the goal of the generator is to output a distribution $q$ that approaches $p$ in a mathematical sense; a perfect generator would output $q$ so that $q(M) = p(M)$ for all measurable sets. In particular, instances where $q$ outputs examples that are far away from the training data are considered completely irrelevant in our definition.

This restriction prevents our definition from capturing instances in which $q$ memorizes its training data and then applies some sort of transformation to it. For example, consider an image generator that applies a color filter to its training data. This would not be considered a data-copier as its output would be quite far from the training data in pixel space. Nevertheless, such a generated distribution can be very reasonably considered as an egregious data copier, and a cursory investigation between its training data and its outputs would reveal as much. 

The key difference in this example is that the generative algorithm is no longer trying to closely approximate $p$ with $q$ -- it is rather trying to do so in some kind of transformed space. Capturing such interactions is beyond the scope of our paper, and we firmly restrict ourselves to the case where a generator is evaluated based on how close $q$ is to $p$ with respect to their measures over the input space. 

\section{Detecting data-copying}

Having defined $\c_q$, we now turn our attention towards \textit{estimating it.} To formalize this problem, we will require a few definitions. We begin by defining a generative algorithm.

\begin{definition}
A \textbf{generative algorithm}, $A$, is a potentially randomized algorithm that outputs a distribution $q$ over $\R^d$ given an input of training points, $S \subset \R^d$. We denote this relationship by $q \sim A(S)$.
\end{definition}

This paradigm captures most typical generative algorithms including both non-parametric methods such as KDEs and parametric methods such as variational autoencoders.

As an important distinction, in this work we define data-copying as a property of the generated distribution, $q$, rather than the generative algorithm, $A$. This is reflected in our definition which is given solely with respect to $q, S,$ and $p$. For the purposes of this paper, $A$ can be considered an arbitrary process that takes $S$ and outputs a distribution $q$. We include it in our definitions to emphasize that while $S$ is an i.i.d sample from $p$, it is \textit{not} independent from $q$. 

Next, we define a \textit{data-copying detector} as an algorithm that estimates $\c_q$ based on access to the training sample, $S$, along with the ability to draw any number of samples from $q$. The latter assumption is quite typical as sampling from $q$ is a purely computational operation. We do not assume any access to $p$ beyond the training sample $S$. Formally, we have the following definition.

\begin{definition}\label{def:data_copy_detector}
A \textbf{data-copying detector} is an algorithm $D$ that takes as input a training sample, $S \sim p^n$, and access to a sampling oracle for $q \sim A(S)$ (where $A$ is an arbitrary generative algorithm). $D$ then outputs an estimate, $D(S, q) = \hc_q$, for the data-copy rate of $q$. 
\end{definition}

Naturally, we assume $D$ has access to $\lambda, \gamma >0$ (as these are practitioner chosen values), and by convention don't include $\lambda, \gamma$ as formal inputs into $D$. 

The goal of a data-copying detector is to provide accurate estimates for $\c_q$. However, the precise definition of $\c_q$ poses an issue: data-copy rates for varying values of $\lambda$ and $ \gamma$ can vastly differ. This is because $\lambda, \gamma$ act as thresholds with everything above the threshold being counted, and everything below it being discarded. Since $\lambda, \gamma$ cannot be perfectly accounted for, we will require some tolerance in dealing with them. This motivates the following.

\begin{definition}\label{defn:approx_data_copy_rate}
Let $0 < \alpaca$ be a tolerance parameter. Then the \textbf{approximate data-copy rates}, $\c_q^{-\alpaca}$ and $\c_q^\alpaca$, are defined as the values of $\c_q$ when the parameters $(\lambda, \gamma)$ are shifted by a factor of $(1+\alpaca)$ to respectively decrease and increase the copy rate. That is, $$\c_q^{-\alpaca} = \c\left(q, \lambda (1+\alpaca), \gamma (1+\alpaca)^{-1}\right),$$ $$\c_q^{\alpaca} = \c\left(q, \lambda (1+\alpaca)^{-1}, \gamma (1+\alpaca)\right).$$
\end{definition}

The shifts in $\lambda$ and $\gamma$ are chosen as above because increasing $\lambda$ and decreasing $\gamma$ both reduce $\c_q$ seeing as both result in more restrictive conditions for what qualifies as data-copying. Conversely, decreasing $\lambda$ and increasing $\gamma$ has the opposite effect. It follows that $$\c_q^{-\alpaca} \leq \c_q \leq \c_q^{\alpaca},$$ meaning that $\c_q^{-\alpaca}$ and $\c_q^{\alpaca}$ are lower and upper bounds on $\c_q$. 

In the context of data-copying detection, the goal is now to estimate $\c_q$ in comparison to $\c_q^{\pm \alpaca}$. We formalize this by defining \textit{sample complexity} of a data-copying detector as the amount of data needed for accurate estimation of $\c_q$. 

\begin{definition}\label{def:sample_complexity}
Let $D$ be a data-copying detector and $p$ be a data distribution. Let $\epsilon, \delta > 0$ be standard tolerance parameters. Then $D$ has \textbf{sample complexity}, $m_p(\epsilon, \delta)$, with respect to $p$ if for all $n \geq m_p(\epsilon, \delta)$, $\lambda >1$, $0 < \gamma < 1$, and generative algorithms $A$, with probability at least $1 - \delta$ over $S \sim p^n$ and $q \sim A(S)$, $$\c_q^{-\alpaca} - \epsilon \leq D(S, q) \leq \c_q^{\alpaca} + \epsilon.$$
\end{definition}

Here the parameter $\epsilon$ takes on a somewhat expanded as it is both used to additively bound our estimation of $\c_q$ and to multiplicatively bound $\lambda$ and $\gamma$.

Observe that there is no mention of the number of calls that $D$ makes to its sampling oracle for $q$. This is because samples from $q$ are viewed as \textit{purely computational}, as they don't require any natural data source. In most cases, $q$ is simply some type of generative model (such as a VAE or a GAN), and thus sampling from $q$ is a matter of running the corresponding neural network.

\section{Regular Distributions}\label{sec:regular_dist}

Our definition of data-copying (Definition \ref{defn:data_copy}) motivates a straightforward point by point method for data-copying detection, in which for every training point, $x_i$, we compute the largest ball $B_i$ centered at $x_i$ for which $q(B_i) \geq \lambda p(B_i)$ and $p(B_i) \leq \gamma$. Assuming we compute these balls accurately, we can then query samples from $q$ to estimate the total rate at which $q$ outputs within those balls, giving us our estimate of $\c_q$.

The key ingredient necessary for this idea to work is to be able to reliably estimate the masses, $q(B)$ and $p(B)$ for any ball in $\R^d$. The standard approach to doing this is through \textit{uniform convergence}, in which large samples of points are drawn from $p$ and $q$ (in $p$'s case we use $S$), and then the mass of a ball is estimated by counting the proportion of sampled points within it. For balls with a sufficient number of points (typically $O( d\log n)$), standard uniform convergence arguments show that these estimates are reliable.

However, this method has a major pitfall for our purpose -- in most cases the balls $B_i$ will be very small because data-copying intrinsically deals with points that are very close to a given training point. While one might hope that we can simply ignore all balls below a certain threshold, this does not work either, as the sheer number of balls being considered means that their union could be highly non-trivial. 

To circumvent this issue, we will introduce an interpolation technique that estimates the probability mass of a small ball by scaling down the mass of a sufficiently large ball with the same center. While obtaining a general guarantee is impossible -- there exist pathological  distributions that drastically change their behavior at small scales -- it turns out there is a relatively natural condition under which such interpolation will work. We refer to this condition as \textit{regularity,} which is defined as follows.

\begin{definition}\label{def:regular}
Let $k> 0$ be an integer. A probability distribution $p$ is \textbf{$k$-regular} the following holds. For all $\alpaca > 0$, there exists a constant $0 < p_\alpaca \leq 1$ such that for all $x$ in the support of $p$, if $0 < s < r$ satisfies that $p(B(x, r)) \leq p_\alpaca$, then $$\left(1+\frac{\alpaca}{3}\right)^{-1}\frac{r^k}{s^{k}} \leq \frac{p(B(x, r))}{p(B(x, s))} \leq \left(1+\frac{\alpaca}{3}\right)\frac{r^k}{s^{k}}.$$ Finally, a distribution is \textbf{regular} if it is $k$-regular for some integer $k > 0$. 
\end{definition}

Here we let $B(x, r) = \{x': ||x - x'|| \leq r\}$ denote the closed $\ell_2$ ball centered at $x$ with radius $r$. 

The main intuition for a $k$-regular distribution is that at a sufficiently small scale, its probability mass scales with distance according to a power law, determined by $k$. The parameter $k$ dictates how the probability density behaves with respect to the distance scale. In most common examples, $k$ will equal the \textit{intrinsic dimension}  of $p$.

As a technical note, we use an error factor of $\frac{\alpaca}{3}$ instead of $\alpaca$ for technical details that enable cleaner statements and proofs in our results (presented later). 

\subsection{Distributions with Manifold Support}

We now give an important class of $k$-regular distributions.

\begin{proposition}\label{prop:manifold_works}
Let $p$ be a probability distribution with support precisely equal to a compact $k$ dimensional sub-manifold (with or without boundary) of $\R^d$, $M$. Additionally, suppose that $p$ has a continuous density function over $M$. Then it follows that $p$ is $k$-regular.
\end{proposition}

Proposition \ref{prop:manifold_works} implies that most data distributions that adhere to some sort of manifold-hypothesis will also exhibit regularity, with the regularity constant, $k$, being the intrinsic dimension of the manifold.

\subsection{Estimation over regular distributions}

We now turn our attention towards designing estimation algorithms over regular distributions, with our main goal being to estimate the probability mass of arbitrarily small balls. We begin by first addressing a slight technical detail -- although the data distribution $p$ may be regular, this does not necessarily mean that the regularity constant, $k$, is known. Knowledge of $k$ is crucial because it determines how to properly interpolate probability masses from large radius balls to smaller ones. 

Luckily, estimating $k$ turns out to be an extremely well studied task, as for most probability distributions, $k$ is a measure of the \textit{intrinsic dimension}. Because there is a wide body of literature in this topic, we will assume from this point that $k$ has been correctly estimated from $S$ using any known algorithm for doing so (for example \cite{BJPR22}). Nevertheless, for completeness, we provide an algorithm with provable guarantees for estimating $k$ (along with a corresponding bound on the amount of needed data) in Appendix \ref{sec:estimating_alpha}.

We now return to the problem of $p(B(x, r))$ for a small value of $r$, and present an algorithm, $Est(x, r, S)$ (Algorithm \ref{alg:estimate}), that estimates $p(B(x, r))$ from an i.i.d sample $S \sim p^n$.

\begin{algorithm}
   \caption{$Est(x, r, S)$}
   \label{alg:estimate}

   \DontPrintSemicolon
   
	$n \leftarrow |S|$\;
	
   $b \leftarrow O\left(\frac{d \ln \frac{n}{\delta}}{\epsilon^2} \right)$\;
   
   $r_* = \min \{s > 0, |S \cap B(x, s)| = b\}$.\;
   
   \uIf{$r_* > r$}{
   Return $\frac{br^k}{nr_*^k}$\;
   }
   \uElse {
	Return $\frac{|T \cap B(x, r)|}{n}$\;
	}

\end{algorithm}

$Est$ uses two ideas: first, it leverages standard uniform convergence results to estimate the probability mass of all balls that contain a sufficient number of training examples. This is what leads to the specific value of $b$ that is chosen. Second, it estimates the mass of smaller balls by interpolating from its estimates from larger balls. The $k$-regularity assumption is crucial for this second step as it is the basis on which such interpolation is done. 

$Est$ has the following performance guarantee, which follows from standard uniform convergence bounds and the definition of $k$-regularity. 
\begin{proposition}\label{prop:est_works}
Let $p$ be a regular distribution, and let $\alpaca >0$ be arbitrary. Then if $n = O\left(\frac{d\ln\frac{d}{\delta \alpaca p_\alpaca}}{\alpaca^2 p_\alpaca}\right)$ with probability at least $1 - \delta$ over $S \sim p^n$, for all $x \in \R^d$ and $r > 0$, $$\left(1+\frac{\alpaca}{2}\right)^{-1}\leq \frac{Est(x, r, S)}{p(B(x, r))} \leq \left(1+\frac{\alpaca}{2}\right).$$
\end{proposition}

\section{A Data-copy detecting algorithm}

\begin{algorithm}    

\caption{$DataCopyDetect(S, q, m)$}
\label{alg:main}   

   \DontPrintSemicolon
   
   $m \leftarrow O\left(\frac{dn^2\ln \frac{nd}{\delta\epsilon}}{\epsilon^4}\right)$\;
   
   Sample $T \sim q^m$\;
   
   $\{x_1, x_2, \dots, x_n\} \leftarrow S$\;
   
   $\{z_1, z_2, \dots, z_m\} \leftarrow T$\;

	\For{$i = 1, \dots, n$}{
	
	Let $p_i(r)$ denote $Est(x_i, r, S)$\;
	
	Let $q_i(r)$ denote $\frac{|B(x_i, r) \cap T|}{m}$\;
	
	$radii \leftarrow \{||z - x_i||: z \in T\} \cup \{0\}$\;
	
	$radii \leftarrow \{r: p_i(r) \leq \gamma, r \in radii\}$\;

	$r_i^* \leftarrow \max \{r: q_i(r) \geq \lambda p_i(r), r \in radii\}$\;
		
	}
	Sample $U \sim q^{20/\epsilon^2}$\;
	$V \leftarrow U \cap \left(\bigcup_{i=1}^n B(x_i, r_i^*)\right)$\;
	Return $\frac{|V|}{|U|}$.\;
	
	

\end{algorithm}


We now now leverage our subroutine, $Est$, to construct a data-copying detector, $Data\_Copy\_Detect$ (Algorithm \ref{alg:main}), that has bounded sample complexity when $p$ is a regular distribution. Like all data-copying detectors (Definition \ref{def:data_copy_detector}), $Data\_Copy\_Detect$ takes as input the training sample $S$, along with the ability to sample from a generated distribution $q$ that is trained from $S$. It then performs the following steps:
\begin{enumerate}
	\item (line 1) Draw an i.i.d sample of $m = O\left(\frac{dn^2\ln \frac{nd}{\delta\epsilon}}{\epsilon^4}\right)$ points from $q$. 
	\item (lines 6 - 10) For each training point, $x_i$, determine the largest radius $r_i$ for which 
	\begin{equation*}
	\begin{split}
	&\frac{|B(x_i, r_i) \cap T|}{m} \geq \lambda Est(x_i, r_i ,S), \\ 
	&Est(x_i, r_i , S) \leq \gamma.
	\end{split}
	\end{equation*}
	\item (lines 12 - 13) Draw a fresh sample of points from $U \sim q^{O(1/\epsilon^2)}$, and use it to estimate the probability mass under $q$ of $\cup_{i=1}^n B(x_i, r_i)$.
\end{enumerate}

In the first step, we draw a \textit{large} sample from $q$. While this is considerably larger than the amount of training data we have, we note that samples from $q$ are considered free, and thus do not affect the sample complexity. The reason we need this many samples is simple -- unlike $p$, $q$ is not necessarily regular, and consequently we need enough points to properly estimate $q$ around every training point in $S$.

The core technical details of $\dc{}$ are contained within step 2, in which data-copying regions surrounding each training point, $x_i$, are found. We use $Est(x, r, S)$ and $\frac{|B(x, r) \cap T|}{m}$ as proxies for $p$ and $q$ in Definition \ref{defn:data_copy}, and then search for the maximal radius $r_i$ over which the desired criteria of data-copying are met for these proxies.  

The only difficulty in doing this is that this could potentially require checking an infinite number of radii, $r_i$. Fortunately, this turns out not to be needed because of the following observation -- we only need to check radii at which a new point from $T$ is included in the estimation $q_i(r)$. This is because these our estimation for $q_i(r)$ does not change between them meaning that our estimate of the ratio between $q$ and $p$ is maximal nearby these points. 

Once we have computed $r_i$, all that is left is to estimate the data-copy rate by sampling $q$ once more to find the total mass of data-copying region, $\cup_{i=1}^n B(x_i, r_i)$. 

\subsection{Performance of Algorithm \ref{alg:main}}

We now show that given enough data, $\dc{}$ provides a close approximation of $\c_q$. 

\begin{theorem}\label{thm:upper_bound5}
$\dc{}$ is a data-copying detector (Definition \ref{def:data_copy_detector}) with sample complexity at most $$m_p(\epsilon, \delta) = O\left(\frac{d\ln\frac{d}{\delta\alpaca p_\alpaca}}{\alpaca^2 p_\alpaca}\right),$$ for all regular distributions, $p$. 
\end{theorem}

Theorem \ref{alg:main} shows that our algorithm's sample complexity has standard relationships with the tolerance parameters, $\epsilon$ and $\delta$, along with the input space dimension $d$. However, it includes an additional factor of $\frac{1}{p_\epsilon}$, which is a distribution specific factor measuring the regularity of the probability distribution. Thus, our bound cannot be used to give a bound on the amount of data needed without having a bound on $p_\epsilon$. 

We consequently view our upper bound as more akin to a convergence result, as it implies that our algorithm is guaranteed to converge as the amount of data goes towards infinity.

\subsection{Applying Algorithm \ref{alg:main} to Halfmoons}\label{sec:experiments}

We now return to the example presented in Figure \ref{fig:halfmoons} and empirically investigate the following question: is our algorithm able to outperform the one given in \cite{MCD2020} over this example? 

To investigate this, we test both algorithms over a series of distributions by varying the parameter $\rho$, which is the proportion of points that are ``copied." Figure \ref{fig:halfmoons} demonstrates a case in which $\rho = 0.4$. Additionally, we include a parameter, $c$, for \cite{MCD2020}'s algorithm which represents the number of clusters the data is partitioned into (with $c$-means clustering) prior to running their test. Intuitively, a larger number of clusters means a better chance of detecting more localized data-copying.

The results are summarized in the following table where we indicate whether the algorithm determined a statistically significant amount of data-copying over the given generated distribution and corresponding training dataset. Full experimental details can be found in Sections \ref{sec:app_experiments} and \ref{sec:experiments_details} of the appendix.

\begin{table}[h]
\caption{Statistical Significance of data-copying Rates over Halfmoons} \label{results_main}
\begin{center}
\begin{tabular}{ |c||c|c|c|c|c| } 
 \hline
 \textbf{Algo} & $\mathbf{q = p}$ & $\mathbf{\rho = 0.1}$ & $\mathbf{0.2}$ & $\mathbf{0.3}$ & $\mathbf{0.4}$ \\ 
 \hline
 \hline
 \textbf{Ours} & \color{blue}no & \color{red}yes & \color{red}yes & \color{red}yes & \color{red}yes \\ 
 \hline
 $\mathbf{c=1}$ & \color{blue}no & \color{blue}no & \color{blue}no & \color{blue}no & \color{blue}no \\ 
 \hline
 $\mathbf{c=5}$ & \color{blue}no & \color{blue}no & \color{blue}no & \color{blue}no & \color{red}yes \\ 
 \hline
 $\mathbf{c=10}$ & \color{blue}no & \color{blue}no & \color{blue}no & \color{blue}no & \color{red}yes \\ 
 \hline
 $\mathbf{c=20}$ & \color{blue}no & \color{blue}no& \color{blue}no & \color{red}yes & \color{red}yes\\ 
 \hline
\end{tabular}
\end{center}
\end{table}

As the table indicates, our algorithm is able to detect statistically significant data-copying rates in all cases it exists. By contrast, \cite{MCD2020}'s test is only capable of doing so when there is a large data-copy rate and when the number of clusters, $c$, is quite large.

\section{Is smoothness necessary for data copying detection?}\label{sec:lower_bound}

Algorithm \ref{alg:main}'s performance guarantee requires that the input distribution, $p$, be regular (Definition \ref{def:regular}). This condition is essential for the algorithm to successfully estimate the probability mass of arbitrarily small balls. Additionally, the parameter, $p_\epsilon$, plays a key role as it serves as a measure of how ``smooth" $p$ is with larger values implying a higher degree of smoothness. 

This motivates a natural question -- can data copying detection be done over unsmooth data distributions? Unfortunately, the answer turns out to be no. In the following result, we show that if the parameter, $p_\epsilon$ is allowed to be arbitrarily small, then this implies that for any data-copy detector, there exists $p$ for which the sample complexity is arbitrarily large.

\begin{theorem}\label{thm:lower_bound5}
Let $B$ be a data-copying detector. Let $\epsilon = \delta = \frac{1}{3}$. Then, for all integers $\a > 0$, there exists a probability distribution $p$ such that $\frac{1}{9\a} \leq p_\alpaca \leq \frac{1}{\a}$, and $m_p(\epsilon, \delta) \geq \a$, implying that $$m_p(\epsilon, \delta) \geq \Omega\left(\frac{1}{p_\epsilon}\right).$$
\end{theorem}

Although Theorem \ref{thm:lower_bound5} is restricted to regular distributions, it nevertheless demonstrates that a bound on smoothness is essential for data copying detection. In particular, non-regular distributions (with no bound on smoothness) can be thought of as a degenerate case in which $p_\epsilon = 0$. 

Additionally, Theorem \ref{thm:lower_bound5} provides a lower bound that complements the Algorithm \ref{alg:main}'s performance guarantee (Theorem \ref{thm:upper_bound5}). Both bounds have the same dependence on $p_\alpaca$ implying that our algorithm is optimal at least in regards to $p_\alpaca$. However, our upper bound is significantly larger in its dependence on $d$, the ambient dimension, and $\alpaca$, the tolerance parameter itself. 

While closing this gap remains an interesting direction for future work, we note that the existence of a gap isn't too surprising for our algorithm, $\dc{}$. This is because $\dc{}$ essentially relies on manually finding the entire region in which data-copying occurs, and doing this requires precise estimates of $p$ at all points in the training sample.  

Conversely, detecting data-copying only requires an \textit{overall} estimate for the data-copying rate, and doesn't necessarily require finding all of the corresponding regions. It is plausible that more sophisticated techniques might able to estimate the data-copy rate \textit{without} directly finding these regions.

\section{Conclusion}

In conclusion, we provide a new modified definition of ``data-copying'' or generating memorized training samples for generative models that addresses some of the failure modes of previous definitions~\cite{MCD2020}. We provide an algorithm for detecting data-copying according to our definition, establish performance guarantees, and show that at least some smoothness conditions are needed on the data distribution for successful detection. 

With regards to future work, one important direction is in addressing the limitations discussed in section \ref{sec:limitations}. Our definition and algorithm are centered around the assumption that the goal of a generative model is to output $q$ that is close to $p$ in a mathematical sense. As a result, we are unable to handle cases where the generator tries to generate \textit{transformed} examples that lie outside the support of the training distribution. For example, a generator restricted to outputting black and white images (when trained on color images) would remain completely undetected by our algorithm regardless of the degree with which it copies its training data. To this end, we are very interested in finding generalizations of our framework that are able to capture such broader forms of data-copying. 








%\newcommand{\crop}[1]{\mathrm{crop}({#1})}
\newcommand{\object}[1]{\mathrm{object}({#1})}
\newcommand{\ba}{A_i}
\newcommand{\bb}{B_i}
\newcommand{\calA}{\mathcal{A}}
\newcommand{\calB}{\mathcal{B}}
\newcommand{\calX}{\mathcal{X}}
\newcommand{\masked}[1]{\mathrm{masked}({#1})}
\newcommand{\bx}{\mathbf{x}}
\newcommand{\SSL}{\textsc{SSL}}
\newcommand{\SSLbb}{\SSL^\mathrm{back}}
\newcommand{\SSLpj}{\SSL^\mathrm{proj}}
\newcommand{\CLF}{\textsc{CLF}}
\newcommand{\CLFbb}{\CLF^\mathrm{back}}
\newcommand{\CLFpj}{\CLF^\mathrm{proj}}
\newcommand{\SUP}{\textsc{SUP}}
\newcommand{\KNN}{\textsc{KNN}}
\newcommand{\KNNset}{\textsc{KNN}^\mathrm{set}}
\newcommand{\KNNprob}{\textsc{KNN}^\mathrm{prob}}
\newcommand{\KNNcl}{\textsc{KNN}^\mathrm{cl}}
\newcommand{\KNNconf}{\textsc{KNN}^\mathrm{conf}}
\newcommand{\RCDM}{\textsc{RCDM}}
\newcommand{\cl}{\mathrm{cl}}
\newcommand{\clpred}{\tilde{\mathrm{cl}}}
\newcommand{\Abox}{\overline{\calA}}
\newcommand{\Bbox}{\overline{\calB}}
\newcommand{\dejavu}{\emph{déjà vu }}
\newcommand{\Dejavu}{\emph{Déjà vu }}

\newcommand{\citations}{{\color{green}[CITE]}}

\definecolor{part_blue}{rgb}{0.2824, 0.4706, .8157}
\definecolor{part_red}{rgb}{0.8392, 0.3725, 0.3725}
\definecolor{part_orange}{rgb}{0.9333, 0.5216, 0.2902}

\DeclareRobustCommand{\mybox}[2][gray!20]{%
\begin{tcolorbox}[   %% Adjust the following parameters at will.
        % breakable,
        left=0pt,
        right=0pt,
        top=0pt,
        bottom=0pt,
        colback=#1,
        colframe=#1,
        width=\dimexpr\columnwidth\relax, 
        % width=\textwidth, 
        enlarge left by=0mm,
        boxsep=5pt,
        arc=0pt,outer arc=0pt,
        ]
        #2
\end{tcolorbox}
}
%\section{Introduction}
\label{sec:intro}
Self-supervised learning (SSL)~\citep{chen2020simclr, chen2020simsiam, zbontar2021barlow, vicreg, caron2020swav, MAE} aims to learn general representations of content-rich data without explicit labels by solving a \textit{pretext task}. In many recent works, such pretext tasks rely on joint-embedding architectures whereby randomized image augmentations are applied to create multiple views of a training sample, and the model is trained to produce similar representations for those views. When using cropping as random image augmentation, the model learns to associate objects or parts (including the background scenery) that co-occur in an image.
However, doing so also arguably exposes the training data to higher privacy risk as objects in training images can be explicitly memorized by the SSL model. For example, if the training data contains the photos of individuals, the SSL model may learn to associate the face of a person with their activity or physical location in the photo. This may allow an adversary to extract such information from the trained model for targeted individuals.

\begin{figure}[t]
    \centering
    \includegraphics[width=1.0\columnwidth]{figures/new_black_swan.pdf}
    \caption{\textbf{Left:} Reconstruction of an SSL training image from a crop containing only the background. The SSL model memorizes the association of this \emph{specific} patch of water (pink square) to this \emph{specific} foreground object (a black swan) in its embedding, which we decode to visualize the full training image. \textbf{Right:} The reconstruction technique fails on a public test image that the SSL model has not seen before.}
    \label{fig:black_swan}
\end{figure}

In this work, we aim to evaluate to what extent SSL models memorize the association of specific objects in training images or the association of objects and their specific backgrounds, and whether this memorization signal can be used to reconstruct the model's training samples. Our results demonstrate that SSL models memorize such associations beyond simple correlation. For instance, in Figure \ref{fig:black_swan} (\textbf{left}), we use the SSL representation of a \emph{training image crop containing only water} and this enables us to reconstruct the object in the foreground with remarkable specificity---in this case a black swan.
By contrast, in Figure \ref{fig:black_swan} (\textbf{right}), when using the \emph{crop from the background of a test set image} that the SSL model \emph{has not seen before}, its representation only contains enough information to infer, through correlation, that the foreground object was likely some kind of waterbird --- but not the specific one in the image.

Figure \ref{fig:black_swan} shows that SSL models suffer from the unintended memorization of images in their training data---a phenomenon we refer to as \emph{déjà vu memorization}
%\footnote{The French loanword \emph{déjà vu} means already-seen, which reflects the type of unintended memorization of objects that the SSL model saw during training.}.
\footnote{The French loanword \emph{déjà vu} means `already-seen', just as an image is seen and memorized in training.}
Beyond visualizing \emph{déjà vu} memorization through data reconstruction, we also design a series of experiments to quantify the degree of memorization for different SSL algorithms, model architectures, training set size, \emph{etc.} We observe that \emph{déjà vu} memorization is exacerbated by the atypically large number of training epochs often recommended in SSL training, as well as certain hyperparameters in the SSL training objective. Perhaps surprisingly, we show that \emph{déjà vu} memorization occurs even when the training set is large---as large as half of ImageNet~\citep{imagenet}---and can continually worsen even when standard techniques for evaluating learned representation quality (such as linear probing) do not suggest increased overfitting. Our work serves as the first systematic study of unintended memorization in SSL models and motivates future work on understanding and preventing this behavior. Specifically, we: 
\begin{itemize}
    \vspace{-0.5em}
    \item Elucidate how SSL representations memorize aspects of individual training images, what we call \emph{déjà vu} memorization;
    \item Design a novel training data reconstruction pipeline for non-generative vision models. This is in contrast to many prominent reconstruction algorithms like \citep{carlini2021extracting, google_diffusion}, which rely on the model itself to generate its own memorized samples and is not possible for SSL models or classifiers;
    \item Propose metrics to quantify the degree of \dejavu memorization committed by an SSL model. This allows us to observe how \dejavu changes with training epochs, dataset size, training criteria, model architecture and more. 
\end{itemize}

%\section{Preliminaries and Related Work}
\label{sec:related}

\textbf{Self-supervised learning} (SSL) is a machine learning paradigm that leverages unlabeled data to learn representations. Many SSL algorithms rely on \emph{joint-embedding} architectures (\emph{e.g.}, SimCLR~\citep{chen2020simclr}, Barlow Twins~\citep{zbontar2021barlow}, VICReg~\citep{vicreg} and Dino~\citep{Dino}), which are trained to associate different augmented views of a given image. For example, in SimCLR, given a set of images $\calA = \{A_1,\ldots,A_n\}$ and a randomized augmentation function $\mathrm{aug}$, the model is trained to maximize the cosine similarity of draws of $\SSL(\mathrm{aug}(A_i))$ with each other and minimize their similarity with $\SSL(\mathrm{aug}(A_j))$ for $i \neq j$. The augmentation function $\mathrm{aug}$ typically consists of operations such as cropping, horizontal flipping, and color transformations to create different views that preserve an image's semantic properties. 

\paragraph{SSL representations.} Once an SSL model is trained, its learned representation can be transferred to different downstream tasks. This is often done by extracting the representation of an image from the \emph{backbone model}\footnote{SSL methods often use a trick called \emph{guillotine regularization}~\citep{Guillotine}, which decomposes the model into two parts: a \emph{backbone model} and a \emph{projector} consisting of a few fully-connected layers. Such trick is needed to handle the misalignment between the pretext SSL task and the downstream task.} and either training a linear probe on top of this representation or finetuning the backbone model with a task-specific head~\citep{Guillotine}.
%Compared to representations learned by supervised learning, SSL representations are often more robust and transferable~\citep{hendrycks2019using, ericsson2021self}, leading to state-of-the-art result on many downstream tasks. To understand the effectiveness of SSL algorithms, several prior works investigated what kind of information the SSL model has learned~\citep{jing2021understanding, ericsson2021self, kalibhat2022towards, RCDM}. In particular, \citet{RCDM} trained a conditional generative model on SSL representations and showed that they encode richer visual details about the input image compared to supervised learning. 
%However, from a privacy perspective, this may be a cause for concern as the model also has more potential to overfit and memorize precise details about the training data compared to supervised learning. We show concretely that this privacy risk can indeed be realized by defining and measuring \emph{déjà vu} memorization.
It has been shown that SSL representations encode richer visual details about input images than supervised models do \cite{RCDM}. However, from a privacy perspective, this may be a cause for concern as the model also has more potential to overfit and memorize precise details about the training data compared to supervised learning. We show concretely that this privacy risk can indeed be realized by defining and measuring \emph{déjà vu} memorization.
\vspace{-0.5em} 
% \paragraph{Privacy risks in ML.} Overfitting in ML occurs when a model memorizes information specific to its training data rather than general population-level information. When the model is trained on privacy-sensitive data, overfitting is especially harmful as an adversary can infer private information about the training data when given access to the model~\citep{yeom2018privacy, feldman2020does}. The simplest and most well-studied form of privacy risk in ML is susceptibility to \emph{membership inference attacks}~\citep{shokri2017membership, salem2018ml, sablayrolles2019white}, where the adversary infers whether an individual is part of the training set or not. More sophisticated privacy attacks include \emph{attribute inference}~\citep{fredrikson2014privacy, mehnaz2022your, jayaraman2022attribute}, where specific attributes about an individual are inferred given others, and \emph{data reconstruction}~\citep{carlini2021extracting, balle2022reconstructing, guo2022bounding}, where entire training samples are recovered from the trained model. Our study of \emph{déjà vu} memorization is similar to both attribute inference and data reconstruction, leveraging SSL representations of the training image background to infer and reconstruct the foreground object.
% \vspace{-0.5em} 
% \paragraph{Training data extraction in NLP.} Our study of \dejavu memorization in SSL models is inspired by similar work in the natural language processing (NLP) domain. \citet{carlini2019secret} first showed that language models exhibit unintended memorization, where given a context string present in its training data, the model can generate the remaining text at test time. This unintended memorization has been further exploited in \citet{carlini2021extracting} to extract training data from GPT-2~\citep{radford2019language} and, more recently, extended to extract memorized images from Stable Diffusion \citep{google_diffusion}. The way by which these works exploit unintended memorization is similar to ours: given partial information about a training sample, the model is prompted to reveal the rest of the sample. In our case, however, since the SSL model is not generative, extraction is significantly harder and requires careful design.

\paragraph{Privacy risks in ML.} When a model is overfit on privacy-sensitive data, it memorizes specific information about its training examples, allowing an adversary with access to the model to learn private information~\citep{yeom2018privacy, feldman2020does}. Privacy attacks in ML range from the simplest and best-studied \emph{membership inference attacks}~\citep{shokri2017membership, salem2018ml, sablayrolles2019white} to \emph{attribute inference}~\citep{fredrikson2014privacy, mehnaz2022your, jayaraman2022attribute} and \emph{data reconstruction}~\citep{carlini2021extracting, balle2022reconstructing, guo2022bounding} attacks. In the former, the adversary only infers whether an individual participated in the training set. Our study of \emph{déjà vu} memorization is most similar to the latter: we leverage SSL representations of the training image background to infer and reconstruct the foreground object. Our approach reflects similar work in the NLP domain \citep{carlini2019secret, carlini2021extracting}: when prompted with a context string present in the training data, a large language model is shown to generate the remainder of string at test time, revealing sensitive text like home addresses. This method was recently extended to extract memorized images from Stable Diffusion \citep{google_diffusion}.  We exploit memorization in a similar manner: given partial information about a training sample, the model is prompted to reveal the rest of the sample. In our case, however, since the SSL model is not generative, extraction is significantly harder and requires careful design.

%\section{Defining \emph{Déjà Vu} Memorization}
\label{sec:definition}

\paragraph{What is \dejavu memorization?} At a high level, the objective of SSL is to learn general representations of objects that occur in nature. This is often accomplished by associating different parts of an image with one another in the learned embedding. Returning to our example in Figure \ref{fig:black_swan}, given an image whose background contains a patch of water, the model may learn that the foreground object is a water animal such as duck, pelican, otter, \emph{etc.}, by observing different images that contain water from the training set. We refer to this type of learning as \emph{correlation}: the association of objects that tend to co-occur in images from the training data distribution.

A natural question to ask is \emph{``Can the reconstruction of the black swan in Figure \ref{fig:black_swan} be reasoned as correlation?''} The intuitive answer may be no, since the reconstructed image is qualitatively very similar to the original image. However, this reasoning implicitly assumes that for a random image from the training data distribution containing a patch of water, the foreground object is unlikely to be a black swan. Mathematically, if we denote by $\mathcal{P}$ the training data distribution and $A$ the image, then
\begin{equation*}
\label{eq:p_corr}
p_\text{corr} := \mathbb{P}_{A \sim \mathcal{P}}(\mathrm{object}(A) = \texttt{black swan} ~|~ \mathrm{crop}(A) = \texttt{water})
\end{equation*}
is the probability of inferring that the foreground object is a black swan through \emph{correlation}. This probability may be naturally high due to biases in the distribution $\mathcal{P}$, \emph{e.g.}, if $\mathcal{P}$ contains no other water animal except for black swans. In fact, such correlations are often exploited to learn a model for image inpainting with great success~\citep{yu2018generative, ulyanov2018deep}.

Despite this, we argue that reconstruction of the black swan in Figure \ref{fig:black_swan} is \emph{not} due to correlation, but rather due to \emph{unintended memorization}: the association of objects unique to a single training image. As we will show in the following sections, the example in Figure \ref{fig:black_swan} is not a rare success case and can be replicated across many training samples. More importantly, failure to reconstruct the foreground object in Figure \ref{fig:black_swan} (\textbf{right}) on test images hints at inferring through correlation is unlikely to succeed---a fact that we verify quantitatively in Section \ref{sec:label inference accuracy}. Motivated by this discussion, we give a verbal definition of \dejavu memorization below, and design a testing methodology to quantify \dejavu memorization in Section \ref{sec:notation and setup}.
\mybox{\textbf{Definition:} A model exhibits \emph{déjà vu memorization} when it retains information so specific to an individual training image, that it enables recovery of aspects particular to that image given a part that does not contain them.
The recovered aspect must be beyond what can be inferred using only correlations in the data distribution.} 

% \textbf{Definition:} A model exhibits \emph{déjà vu memorization} when it retains information so specific to an individual training image, that it enables recovery of aspects particular to that image given a part that does not contain them.
% The recovered aspect must be beyond what can be inferred using only correlations in the data distribution.


 We intentionally kept the above definition broad enough to encompass different types of information that can be inferred about the training image, including but not restricted to object category, shape, color and position. For example, if one can infer that the foreground object is red given the background patch with accuracy significantly beyond correlation, we consider this an instance of \dejavu memorization as well. We mainly focus on object category to quantify \dejavu memorization in Section \ref{sec:quant} since the ground truth label can be easily obtained. We consider other types of information more qualitatively in the visual reconstruction experiments in Section \ref{sec:visualizing}.

\paragraph{Privacy implications of \dejavu memorization.} \Dejavu memorization can be a cause for concern when the training data contains privacy-sensitive information. As a motivating example, consider an SSL model trained on photos of individuals. If the model exhibits \dejavu memorization then, given the face of an individual, it may be possible to infer where the individual was or even visually reconstruct their location in the training image. Such information leakage raises privacy concerns, especially if there was no prior agreement that the trained model may reveal such information to third parties. This hypothetical scenario serves as a motivation that \dejavu memorization should be carefully examined to avoid unintended disclosure of private information in practical applications.

% \begin{figure*}[h]
%     \centering
%     \includegraphics[width = 0.85\textwidth]{figures/SSL_attack_cartoon.png}
%     \caption{We measure memorization by comparing the `target model' trained on the target image ($\SSL_A$ trained on $A_i$ in above example) with the `reference model' not trained on it ($\SSL_B$, above). \textbf{[Top Strip]} A cropping of the image disjoint from the labeled foreground object is embedded using the target model. This embedding is then labeled by a K-Nearest Neighbor (KNN) adversary built on a public set of labeled images, $X$, which it has also embedded using the target model. \textbf{[Bottom Strip]} To account for correlation, the same procedure is followed with the reference model. If the label is only extracted using the target model, it is counted as memorization. If it is extracted using either model, it is counted as correlation. We find that the KNN adversary's predictions using the target model (trained on attacked examples) are significantly more accurate than they are using the reference model, indicating routine memorization of training examples.}
%     \label{fig:ssl attack cartoon}
% \end{figure*}

\begin{figure}[t]
%%%
%SPIDER
%%%
     % \centering
     % \begin{subfigure}[b]{0.25\textwidth}
     %     \centering
     %     \includegraphics[width=\textwidth]{figures/data_split.png}
     %     % \caption{SimCLR correlated \textit{yellow garden spider} examples}
     %     \label{fig:data split}
     % \end{subfigure}
     % \hfill
     % \begin{subfigure}[b]{0.7\textwidth}
     %     \centering
     %     \includegraphics[width=\textwidth]{figures/pipeline_cartoon.png}
     %     \begin{minipage}{5cm}
     %        \vfill
     %    \end{minipage}
     %     % \caption{SimCLR memorized \textit{yellow garden spider} examples}
     %     \label{fig:pipeline cartoon}
     % \end{subfigure}
     \includegraphics[width=\textwidth]{figures/split_and_pipeline_cartoon.png}
\caption[Overview of testing methodology.]{
Overview of testing methodology. \textbf{Left:} Data is split into \emph{target set} $\calA$, \emph{reference set} $\calB$ and \emph{public set} $\calX$ that are pairwise disjoint. $\calA$ and $\calB$ are used to train two SSL models $\SSL_A$ and $\SSL_B$ in the same manner. $\calX$ is used for KNN decoding or for training an RCDM to reconstruct the input at test time. \textbf{Right:} Given a training image $A_i \in \calA$, we use $\SSL_A$ to embed $\crop{A_i}$ containing only the background, as well as the entire set $\calX$ and find the $k$-nearest neighbors of $\crop{A_i}$ in $\calX$ in the embedding space. These KNN samples can be used directly to infer the foreground object (\emph{i.e.}, class label) in $A_i$ using a KNN classifier, or their embeddings can be averaged as input to the trained RCDM to visually reconstruct the image $A_i$. For instance, the RCDM reconstruction results in Figure \ref{fig:black_swan} (left) when given $\SSL_A(\crop{A_i})$ and results in Figure \ref{fig:black_swan} (right) when given $\SSL_A(\crop{B_i})$ for an image $B_i \in \calB$.
%\textbf{Left:} illustration of the three datasets used in our tests. Two private data sets, $A$ and $B$, of equal size are used to train two SSL models, $\SSL_A$ and $\SSL_B$, respectively. A disjoint public set, $X$, is made available to the memorization test to help decode model embeddings. Memorization is only tested on examples $A_i \in A$ that are unique to set $A$. \textbf{Right:} illustration of inference pipeline used in tests. A periphery cropping that excludes the foreground object is taken from private image $A_i$. The KNN then finds the $k$ public set nearest neighbors of the periphery crop in the embedding space of $\SSL_A$. 
%The $\SSL_A$ representation of these $k$ neighbors and of the crop are used by the conditional generative model, RCDM, to reconstruct the foreground object. The labels of these $k$ neighbors are used to recover the foreground object label. (Not pictured) We repeat this process using reference model $\SSL_B$, not trained on image $A_i$, to determine whether the foreground object is still recoverable by learned correlations, e.g. if black swans were the only objects appearing near water in the data distribution. In this instance, the crop's public set neighbors in $\SSL_B$'s representation space include a variety of water animals like ducks, pelicans, and otters. Meanwhile, with $\SSL_A$, the neighbors are nearly all black swans in the same position as the swan of $A_i$.
}
\label{fig:split_and_pipeline_cartoon}
\end{figure}

\textbf{Distinguishing memorization from correlation.} When measuring \dejavu memorization, it is crucial to differentiate what the model associates through \emph{memorization} and what it associates through \emph{correlation}. Our testing methodology is based on the following intuitive definition.
\mybox{\textbf{Definition:} If an SSL model associates two parts in a training image, we say that it is due to \emph{correlation} if other SSL models trained on a similar dataset from $\mathcal{P}$ without this image would likely make the same association. Otherwise, we say that it is due to \emph{memorization}.}

Notably, such intuition forms the basis for differential privacy (DP; \cite{dwork2006calibrating, dwork2013algorithmic})---the most widely accepted notion of privacy in ML.

\subsection{Testing Methodology for Measuring \emph{Déjà Vu} Memorization}
\label{sec:notation and setup}

In this section, we use the above intuition to measure the extent of \dejavu memorization in SSL. Figure \ref{fig:split_and_pipeline_cartoon} gives an overview of our testing methodology.
\vspace{-0.75em}
\paragraph{Dataset splitting.} We focus on testing \dejavu memorization for SSL models trained on the ImageNet-1K dataset~\citep{imagenet}. Our test first splits the ImageNet training set into three independent and disjoint subsets $\calA$, $\calB$ and $\calX$. The dataset $\calA$ is called the \emph{target set} and $\calB$ is called the \emph{reference set}. The two datasets are used to train two separate SSL models, $\SSL_A$ and $\SSL_B$, called the \emph{target model} and the \emph{reference model}. Finally, the dataset set $\calX$ is used as an auxiliary public dataset to extract information from $\SSL_A$ and $\SSL_B$.
%\footnote{See Appendix \ref{sec:appx splits} for details on how the dataset splits are generated.}.
Our dataset splitting serves the purpose of distinguishing memorization from correlation in the following manner. Given a sample $A_i \in \calA$, if our test returns the same result on $\SSL_A$ and $\SSL_B$ then it is likely due to correlation because $A_i$ is not a training sample for $\SSL_B$. Otherwise, because $\calA$ and $\calB$ are drawn from the same underlying distribution, our test must have inferred some information unique to $A_i$ due to memorization. Thus, by comparing the difference in the test results for $\SSL_A$ and $\SSL_B$, we can measure the degree of \dejavu memorization\footnote{See Appendix \ref{sec:appx splits} for details on how the dataset splits are generated.}.
\vspace{-0.75em}
\paragraph{Extracting foreground and background crops.} Our testing methodology aims at measuring what can be inferred about the foreground object in an ImageNet sample given a background crop. This is made possible because ImageNet provides bounding box annotations for a subset of its training images---around 150K out of 1.3M samples. We split these annotated images equally between $\calA$ and $\calB$. Given an annotated image $A_i$, we treat everything inside the bounding box as the foreground object associated with the image label, denoted $\object{A_i}$. We take the largest possible crop that does not intersect with any bounding box as the background crop (or \emph{periphery crop}), denoted $\crop{A_i}$\footnote{We also present another heuristic in \cref{sec:appx corner crop} which takes a corner crop as the background crop, allowing our test to be run without bounding box annotations.}
%Since the labeled object tends to be at the image's center, the corner crop usually excludes it. }
%Because most images in ImageNet are object centric, an image's corner would not include the foreground object.}.
\vspace{-0.75em}
\paragraph{KNN-based test design.} Joint-embedding SSL approaches encourage the embeddings of random crops of a training image $A_i \in \calA$ to be similar. Intuitively, if the model exhibits \dejavu memorization, it is reasonable to expect that the embedding of $\crop{A_i}$ is similar to that of $\object{A_i}$ since both crops are from the same training image. In other words, $\SSL_A(\crop{A_i})$ encodes information about $\object{A_i}$ that cannot be inferred through correlation. However, decoding such information is challenging as these approaches do not learn a decoder associated with the encoder $\SSL_A$.

Here, we leverage the public set $\calX$ to decode the information contained in $\crop{A_i}$ about $\object{A_i}$. More specifically, we map images in $\calX$ to their embeddings using $\SSL_A$ and extract the $k$-nearest-neighbor (KNN) subset of $\SSL_A(\crop{A_i})$ in $\calX$. We can then decode the information contained in $\crop{A_i}$ in one of two ways:
\begin{itemize}
\item \emph{Label inference:} Since $\calX$ is a subset of ImageNet, each embedding in the KNN subset is associated with a class label. If $\crop{A_i}$ encodes information about the foreground object, its embedding will be close to samples in $\calX$ that have the same class label (\emph{i.e.}, foreground object category). We can then use a KNN classifier to infer the foreground object in $A_i$ given $\crop{A_i}$.
\item \emph{Visual reconstruction:} Following \citet{RCDM}, we train an RCDM---a conditional generative model---on $\calX$ to decode $\SSL_A$ embeddings into images. The RCDM reconstruction can recover qualitative aspects of an image remarkably well, such as recovering object color or spatial orientation using its SSL embedding. Given the KNN subset, we average their SSL embeddings and use the trained RCDM model to visually reconstruct $A_i$.
\end{itemize}
In Section \ref{sec:quant}, we focus on quantitatively measuring \dejavu memorization with label inference, and then use the RCDM reconstruction to visualize \dejavu memorization in Section \ref{sec:visualizing}.
%\section{Quantifying \emph{Déjà Vu} Memorization}
\label{sec:quant}

We apply our testing methodology to quantify a specific form of \dejavu memorization: inferring the foreground object (class label) given a crop of the background.

% \paragraph{Extracting model embeddings.} We test \dejavu memorization on two popular SSL algorithms, SimCLR~\citep{chen2020simclr} and VICReg~\citep{vicreg}.
% %\footnote{We present additional SSL models in \cref{sec:appx simclr results}} 
% As described in Section \ref{sec:related}, these algorithms produce two embeddings given an input image: a \emph{backbone} embedding and a \emph{projector} embedding that is derived by applying a small fully-connected network on top of the backbone embedding. Unless otherwise noted, all SSL embeddings refer to the projector embedding.
% To understand whether \dejavu memorization is particular to SSL, we also evaluate embeddings produced by a supervised model $\CLF_A$ trained on $\calA$. We apply the same set of image augmentations as those used in SSL and train $\CLF_A$ using the cross-entropy loss to predict ground truth labels. 
\vspace{-0.75em}
\paragraph{Extracting model embeddings.} We test \dejavu memorization on a variety of popular SSL algorithms, with a focus on VICReg~\citep{vicreg}. These algorithms produce two embeddings given an input image: a \emph{backbone} embedding and a \emph{projector} embedding that is derived by applying a small fully-connected network on top of the backbone embedding. Unless otherwise noted, all SSL embeddings refer to the projector embedding. 
To understand whether \dejavu memorization is particular to SSL, we also evaluate embeddings produced by a supervised model $\CLF_A$ trained on $\calA$. We apply the same set of image augmentations as those used in SSL and train $\CLF_A$ using the cross-entropy loss to predict ground truth labels. 
\vspace{-0.75em}
\paragraph{Identifying the most memorized samples.} Prior works have shown that certain training samples can be identified as more prone to memorization than others~\citep{feldman2020does, watson2021importance, ye2021enhanced}. Similarly, we provide a heuristic to identify the most memorized samples in our label inference test using confidence of the KNN prediction.
Given a periphery crop, $\crop{A_i}$, let $\KNN_A \big( \crop{A_i} \big) \subseteq \calX$ denote its $k$-nearest neighbors in the embedding space of $\SSL_A$. From this KNN subset we can obtain: \textbf{(1)} $\KNNprob_A \big( \crop{A_i} \big)$, the vector of class probabilities (normalized counts) induced by the KNN subset, and \textbf{(2)} $\KNNconf_A \big( \crop{A_i} \big)$, the negative entropy of the probability vector $\KNNprob_A \big( \crop{A_i} \big)$, as confidence of the KNN prediction. When entropy is low, the neighbors agree on the class of $A_i$ and hence confidence is high. 
% \begin{itemize}[noitemsep, leftmargin=*, topsep=0pt]
%     \item $\KNN_A \big( \crop{A_i} \big)$: The most prevalent class in the KNN subset as prediction for the class label $\cl(A_i)$. 
%     \item $\KNNprob_A \big( \crop{A_i} \big)$: The vector of class probabilities (normalized counts) induced by the KNN subset.
%     \item $\KNNconf_A \big( \crop{A_i} \big)$: Negative entropy of the probability vector $\KNNprob_A \big( \crop{A_i} \big)$ as confidence of the KNN prediction. When entropy is low, the neighbors agree on the class of $A_i$ and hence confidence is high. 
% \end{itemize}
We can sort the confidence score $\KNNconf_A \big( \crop{A_i} \big)$ across samples $A_i$ in decreasing order to identify the most confidently predicted samples, which likely correspond to the most memorized samples when $A_i \in \calA$.

\subsection{Population-level Memorization}
\label{sec:label inference accuracy}

%ORIGINAL FIGURE SETUP IN ARXIV: 
% \input{dejavu_training_epochs.tex}
% \input{dejavu_training_set_size.tex}
%PUT ORIGINAL FIGURES SIDE BY SIDE: 
% \input{dejavu_training_epochs_set_size.tex}
%PUT IN NEW FIGURES: 

\begin{wrapfigure}{r}{0.4\textwidth} 
    \centering
    \includegraphics[width=0.4\textwidth]{figures/dejavu_main.pdf}
    \caption{Accuracy of label inference using the target model (trained on $\calA$) vs. the reference model (trained on $\calB$) on the top $\%$ most confident examples $A_i \in \calA$ using only $\crop{A_i}$. For VICReg, there is a large accuracy gap between the two models, indicating a significant degree of \dejavu memorization.}
    \label{fig:dejavu main}
    \vspace{-2ex}
\end{wrapfigure}

Our first measure of \dejavu memorization is population-level label inference accuracy: \emph{What is the average label inference accuracy over a subset of SSL training images given their periphery crops?} 
To understand how much of this accuracy is due to $\SSL_A$'s \dejavu memorization, we compare with a correlation baseline using the reference model: $\KNN_B$'s label inference accuracy on images $A_i \in \calA$. 
In principle, this inference accuracy should be significantly above chance level ($1/1000$ for ImageNet) because the periphery crop may be highly indicative of the foreground object through correlation, \emph{e.g.}, if the periphery crop is a basketball player then the foreground object is likely a basketball.

Figure \ref{fig:dejavu main} compares the accuracy of $\KNN_A$ to that of $\KNN_B$ when inferring the labels of images in $A_i \in \calA$\footnote{The sets $\calA$ and $\calB$ are exchangeable, and in practice we repeat this test on images from $\calB$ using $\SSL_B$ as the target model and $\SSL_A$ as the reference model, and average the two sets of results.} using $\crop{A_i}$.
Results are shown for VICReg and the supervised model; trends for other models are shown in Appendix \ref{sec:appx simclr results}. For both VICReg and supervised models, inferring the class of $\crop{A_i}$ using $\KNN_B$ (dashed line) through correlation achieves a reasonable accuracy that is significantly above chance level. However, for VICReg, the inference accuracy using $\KNN_A$ (solid red line) is significantly higher, and the accuracy gap between $\KNN_A$ and $\KNN_B$ indicates the degree of \dejavu memorization. We highlight two observations: 
\begin{itemize}
    \item The accuracy gap of VICReg is significantly larger than that of the supervised model. This is especially notable when accounting for the fact that the supervised model is trained to associate randomly augmented crops of images with their ground truth labels. In contrast, VICReg has no label access during training but the embedding of a periphery crop can still encode the image label. 
    \item For VICReg, inference accuracy on the $1\%$ most confident examples is nearly $95\%$, which shows that our simple confidence heuristic can effectively identify the most memorized samples. This result suggests that an adversary can use this heuristic to identify vulnerable training samples to launch a more focused privacy attack.
\end{itemize}
\vspace{-.75em}
\paragraph{The \dejavu score. }
The curves of Figure \ref{fig:dejavu main} show memorization across confidence values for a single training scenario.  To study how memorization changes with different hyperparamters, we extract a single value from these curves: the \dejavu \emph{score} at confidence level $p$. In Figure \ref{fig:dejavu main}, this is the gap between the solid red (or gray) and dashed red (or gray) where confidence ($x$-axis) equal $p\%$. In other words, given the periphery crops of set $\calA$, $\KNN_A$ and $\KNN_B$ separately select and label their top $p\%$ most confident examples, and we report the difference in their accuracy. The \dejavu score captures both the degree of memorization by the accuracy gap and the \emph{ability to identify memorized examples} by the confidence level. If the score is 10\% for $p=33\%$, $\KNN_A$ has 10\% higher accuracy on its most confident third of $\calA$ than $\KNN_B$ does on its most confident third. In the following, we set $p = 20\%$, approximately the largest gap for VICReg (red lines) in Figure \ref{fig:dejavu main}. 
% Specifically, the \dejavu \emph{score} on the top $p\%$ most confident examples is,  
% \begin{equation}
%     \mathrm{DejaVu}(p) = \mathrm{Acc}_{\SSL_A}\big( \calA_{\SSL_A, p}  \big) - \mathrm{Acc}_{\SSL_B}\big( \calA_{\SSL_B, p}  \big) \ ,
%     \label{eqn:dejavu score}
% \end{equation}
% where $\calA_{\SSL_A, p}$
% Here we introduce a DejaVu memorization metric that quantify how much a target model is able to retrieve more class information from a crop than the reference model. We define it as:
% where $p$ is a function that take the $p$ purcent most confident samples.
%Figure \ref{fig:dejavu v. training epochs} shows how \dejavu memorization changes with the number of epochs used to train the embedding model (VICReg and supervised, respectively). The training set size is fixed to 300K samples, and label inference accuracy is computed on the top $20\%$ highest confidence examples. The number of epochs has a very strong influence on the degree of memorization for VICReg as the accuracy gap widens when number of epochs increases. We note that 1000 training epochs is used in several SSL works \citep{vicreg, simclr}. Remarkably, this trend in memorization is \emph{not} reflected in the standard metric for evaluating SSL representations: linear probe accuracy. The gray line in Figure \ref{fig:dejavu v. training epochs} shows the train-test accuracy gap of a linear classifier trained on top of the VICReg embeddings. Although there is a sizeable train-test gap, it does not grow significantly beyond 500 epochs. In contrast, \dejavu memorization (blue line) continues to worsen after 500 epochs. Thus, our test can be used as an alternative to linear probe accuracy to evaluate the memorization of SSL models.
% \vspace{-.75em}

% \paragraph{Comparison with the generalization gap} A network that perform very well on a training set while performing poorly on a test set (assuming the training set and test set sampled uniformly from the same distribution) is probably memorizing the training examples without being able to generalize on the test data. One could expect that measuring the difference in accuracy between the training and test set could give us insights on the degree of \dejavu memorization. However, we show in Figure  \ref{fig:dejavu v. training epochs} and \ref{fig:dejavu v. n} that this is not the case. In fact \dejavu memorization can significantly increase while the train-test gap decrease. In our experiments, we did not find a correlation between \dejavu and generalization.
\vspace{-0.75em}
\paragraph{Comparison with the linear probe train-test gap.} A standard method for measuring SSL performance is to train a linear classifier---what we call a `linear probe'---on its embeddings and compute its performance on a held out test set. From a learning theory standpoint, one might expect the linear probe's train-test accuracy gap to be indicative of memorization: the more a model overfits, the larger is the difference between train set and test set accuracy. However, as seen in Figure \ref{fig:dejavu epochs train set size}, the linear probe gap (dark blue) fails to reveal memorization captured by the \dejavu score (red) \footnote{See section \ref{sec:mitigation} for further discussion of the \dejavu score trends of Figure \ref{fig:dejavu epochs train set size}.}.

% \paragraph{Effect of training epochs.} 
% Figure \ref{fig:dejavu v. training epochs} shows how \dejavu memorization changes with training epochs for VICReg. The training set size is fixed to 300K samples. We observe that the number of epochs has a very strong influence on the degree of memorization for VICReg. From 250 to 1000 epochs, the \dejavu score (red curve) grows threefold: from under 10\% to over 30\%. Remarkably, this trend in memorization is \emph{not} reflected in the standard metric for evaluating SSL representations: linear probe accuracy. The dark blue curve shows the train-test linear probe accuracy gap. Although there is a sizeable train-test gap, it only changes by a few percent beyond 250 epochs. %Thus, our test can be used as an alternative to linear probe accuracy to evaluate the memorization of SSL models.
% \vspace{-.75em}
\begin{figure}[ht]
\label{fig:dejavu epochs and dataset}
\begin{minipage}[t]{0.49\textwidth}
\centering
     \begin{subfigure}[b]{0.48\textwidth}
         \centering
         \includegraphics[width=\textwidth]{figures/deja_vu_vs_epochs.png}
         \vspace{-1.5em}
         \caption{\dejavu vs. epochs}
         \label{fig:dejavu v. training epochs}
     \end{subfigure}
     \begin{subfigure}[b]{0.48\textwidth}
         \centering
         \includegraphics[width=\textwidth]{figures/deja_vu_vs_n.png}
         \vspace{-1.5em}
         \caption{\dejavu vs. train set size}
         \label{fig:dejavu v. n}
     \end{subfigure}~
     \vspace{-0.5em}
    \caption{
    Effect of training epochs and train set size with VICReg on \dejavu score (red) in comparison with linear probe accuracy train-test gap (dark blue). 
    \textbf{Left:} \dejavu score increases with training epochs, indicating growing memorization while the linear probe baseline decreases significantly.  
    \textbf{Right:} \dejavu score stays roughly constant with training set size suggesting that memorization may be problematic even for large datasets. %By comparison, the baseline \emph{declines} by half, spuriously suggesting less memorization. 
    %Both trends are not captured according to the linear probe train-test gap---a common method to evaluate generalization of SSL representations.}
    }
    \label{fig:dejavu epochs train set size}
\end{minipage}
\hfill
\begin{minipage}[t]{0.49\textwidth}
\centering
     \begin{subfigure}[b]{0.48\textwidth}
         \centering
         \includegraphics[width=\textwidth]{figures/vicreg_samples_epochs.pdf}
         \vspace{-1.5em}
         \caption{\dejavu vs. epochs}
         \label{fig:per sample v. training epochs}
     \end{subfigure}
     \begin{subfigure}[b]{0.48\textwidth}
         \centering
         \includegraphics[width=\textwidth]{figures/vicreg_samples_datasets.pdf}
         \vspace{-1.5em}
         \caption{\dejavu vs. train set size}
         \label{fig:per sample v. n}
     \end{subfigure}~
     \vspace{-0.5em}
    \caption{
    \definecolor{part_blue}{rgb}{0.2824, 0.4706, .8157}
	\definecolor{part_red}{rgb}{0.8392, 0.3725, 0.3725}
	\definecolor{part_orange}{rgb}{0.9333, 0.5216, 0.2902}
    Partition of samples $A_i \in \calA$ into the four categories: {\color{gray}unassociated} (not shown), {\color{part_orange}memorized}, {\color{part_red}misrepresented} and {\color{part_blue}correlated} for VICReg. The {\color{part_orange}memorized} samples---those whose labels are predicted by $\KNN_A$ but not by $\KNN_B$---occupy a significantly larger share of the training set than the {\color{part_red}misrepresented} samples---those predicted by $\KNN_B$ but not $\KNN_A$ by chance. %At 1000 epochs, $\approx 15\%$ of the training set is {\color{part_orange}memorized}. The trends across training epochs and training set sizes are consistent with those observed in Figure \ref{fig:dejavu epochs train set size}
    }
    \label{fig:partition attack main}
    \end{minipage}
\vspace{-1em} 
\end{figure}

\iffalse

\begin{minipage}[t]{0.49\textwidth}
\centering
     \begin{subfigure}[b]{0.48\textwidth}
         \centering
         \includegraphics[width=0.95\textwidth]{figures/deja_vu_vs_parameters.png}
         \vspace{-0.4em}
         \caption{\dejavu vs. capacity}
         \label{fig:dejavu v. capacity}
     \end{subfigure}
     \hfill
     \begin{subfigure}[b]{0.48\textwidth}
          \tiny
          \centering
          \setlength{\tabcolsep}{3pt}
          \begin{tabular}{|c|c|c|}
            \hline
            Criteria & DV & Acc P/B \\
            \hline
            Supervised & 8.9 & 55.3/61.1\\
            \hline
            Byol\citep{grill2020byol} & 8.0& 54.3/59.4\\
            \hline
            SimCLR\citep{chen2020simclr} & 10.0 & 44.2/54.1\\
            \hline
            Dino\citep{Dino} & 14.5 & 26.3/55.7 \\
            \hline
            Barlow T.\citep{zbontar2021barlow} & 30.5 & 33.7/54.4\\
            \hline
            VICReg\citep{vicreg} & \textbf{33.2} & 40.3/55.2\\
            \hline
          \end{tabular}
          \vspace{1.3em}
          % \caption{\dejavu (DV) vs. SSL Criterion}
          \caption{\dejavu (DV) vs. Criterion}
          \label{tab:dejavu vs. criterion}
    \end{subfigure}
    \vspace{-0.5em}
    \caption{
    Comparison of \dejavu score for different architectures and training criteria. \textbf{Left:} \dejavu score with VICReg for resnet (purple) and vision transformer (green) architectures versus number of model parameters. As expected, memorization grows with larger model capacity. This trend is more pronounced for convolutional (resnet) than transformer (ViT) architectures. \textbf{Right:} Comparison of \dejavu score and ImageNet validation accuracy (P: using projector embeddings, B: using backbone embeddings) for various SSL criteria. \textbf{Nearly all SSL models have more memorization than the supervised baseline.} 
    % Effect of training epochs and train set size on \dejavu score.
    % \textbf{Left:} \dejavu score increases with higher number of training epochs, indicating worsening memorization.
    % \textbf{Right:} \dejavu score stays roughly constant with training set size. Both trends are not captured according to the linear probe train-test gap---a common method to evaluate generalization of SSL representations.
    }
\end{minipage}
\vspace{-2em} 
\end{figure}

\begin{figure}[ht]
\begin{minipage}[t]{0.49\textwidth}
\centering
     \begin{subfigure}[b]{0.49\textwidth}
         \centering
         \includegraphics[width=\textwidth]{figures/epochs_lb_attk_epochs_acc_top1_legend.pdf}
         \caption{\dejavu vs. epochs}
         \label{fig:dejavu v. training epochs}
     \end{subfigure}
     \begin{subfigure}[b]{0.49\textwidth}
         \centering
         \includegraphics[width=\textwidth]{figures/epochs_lb_attk_datasets_acc_top1_legend.pdf}
         \caption{\dejavu vs. train set size}
         \label{fig:dejavu v. n}
     \end{subfigure}~
     \begin{subfigure}[b]{0.32\textwidth}
         \centering
         \includegraphics[width=0.8\textwidth]{figures/dejavu_vs_parameters.pdf}
         \caption{\dejavu vs. capacity}
         \label{fig:dejavu v. n}
     \end{subfigure}
    \caption{
    Effect of training epochs and train set size on \dejavu score.
    \textbf{Left:} \dejavu score increases with higher number of training epochs, indicating worsening memorization.
    \textbf{Right:} \dejavu score stays roughly constant with training set size. Both trends are not captured according to the linear probe train-test gap---a common method to evaluate generalization of SSL representations.}
    \end{minipage}
\vspace{-1em} 
\end{figure}

\begin{table}[ht]
  \footnotesize
  \centering
  \begin{tabular}{|c|c|}
    \hline
    Supervised & 8.9\\
    \hline
    SimCLR\citep{chen2020simclr} & 10.0\\
    \hline
    Byol\citep{grill2020byol} & 8.0\\
    \hline
    Dino\citep{Dino} & 14.5\\
    \hline
    Barlow T.\citep{zbontar2021barlow} & 30.5\\
    \hline
    VICReg\citep{vicreg} & \textbf{33.2}\\
    \hline
  \end{tabular}
  \caption{DejaVu Score 20\% Conf for various SSL methods.}
  \label{tab:two-row-table}
\end{table}
\vspace{-1em} 
\fi

\iffalse
\begin{figure}[ht]
\begin{minipage}[t]{.49\textwidth}
\centering
     \begin{subfigure}[b]{0.49\textwidth}
         \centering
         \includegraphics[width=\textwidth]{figures/epochs_lb_attk_epochs_acc_top1_legend.pdf}
         \caption{\dejavu vs. epochs}
         \label{fig:dejavu v. training epochs}
     \end{subfigure}
     \hfill
     \begin{subfigure}[b]{0.49\textwidth}
         \centering
         \includegraphics[width=\textwidth]{figures/epochs_lb_attk_datasets_acc_top1_legend.pdf}
         \caption{\dejavu vs. train set size}
         \label{fig:dejavu v. n}
     \end{subfigure}
\caption{
Effect of training epochs and train set size on \dejavu score.
\textbf{Left:} \dejavu score increases with higher number of training epochs, indicating worsening memorization.
\textbf{Right:} \dejavu score stays roughly constant with training set size. Both trends are not captured according to the linear probe train-test gap---a common method to evaluate generalization of SSL representations.}
\label{fig:dejavu epochs and dataset}
\end{minipage}
\hfill
\begin{minipage}[t]{.49\textwidth}
     \centering
     \begin{subfigure}[b]{0.49\textwidth}
         \centering
         \includegraphics[width=\textwidth]{figures/criteria_epochs.pdf}
         \caption{criteria comparison}
         \label{fig:dejavu v. criteria}
     \end{subfigure}
     \hfill
     \begin{subfigure}[b]{0.49\textwidth}
         \centering
         \includegraphics[width=\textwidth]{figures/architecture_epochs.pdf}
         \caption{architecture comparison}
         \label{fig:dejavu v. arch}
     \end{subfigure}
\caption{
Effect of SSL training criteria and model architectures on \dejavu score.
%the accuracy gap between target model (trained on $\calA$) and reference model (trained on $\calB$) making predictions on their 20\% most confident examples.
\textbf{Left:} \dejavu score for various training criteria.
%Barlow and VICReg have the heaviest degree of memorization, while SimCLR and BYOL have the least. 
%Note that we show detailed reconstructions of SimCLR's training data in Section \ref{sec:visualizing} despite its relatively low degree of \dejavu. 
%Regardless, Although SimCLR and BYOL have the least, we  visualize detailed reconstructions with SimCLR in section \ref{sec:mem v corr} 
All SSL models have significantly more \dejavu than the supervised baseline. \textbf{Right:} \dejavu score versus epochs for various training architectures. As expected, lower capacity architectures (Resnet18, Resnet34) reduce \dejavu but not completely. 
}
\label{fig:dejavu criteria and architecture}
\end{minipage}
\vspace{-1em} 
\end{figure}
\fi
% %\begin{figure}[ht]
%%%
%VICREG
%%%
     \centering
     \begin{subfigure}[b]{0.49\textwidth}
         \centering
         \includegraphics[width=\textwidth]{figures/sample_level_training_epochs.pdf}
         \caption{Categories of training samples vs. number of epochs}
         \label{fig:sample level epochs}
     \end{subfigure}
     \hfill
     \begin{subfigure}[b]{0.49\textwidth}
         \centering
         \includegraphics[width=\textwidth]{figures/sample_level_training_set_size.pdf}
         \caption{Categories of training samples vs. training set size}
         \label{fig:sample level training size}
     \end{subfigure}
\caption{
\definecolor{part_blue}{rgb}{0.2824, 0.4706, .8157}
\definecolor{part_red}{rgb}{0.8392, 0.3725, 0.3725}
\definecolor{part_orange}{rgb}{0.9333, 0.5216, 0.2902}
Partition of samples $A_i \in \calA$ into the four categories: {\color{gray}unassociated} (not shown), {\color{part_orange}memorized}, {\color{part_red}misrepresented} and {\color{part_blue}correlated}. The {\color{part_orange}memorized} samples---ones whose labels are predicted by $\KNN_A$ but not by $\KNN_B$---occupy a significantly larger share for VICReg compared to the supervised model, indicating that sample-level \dejavu memorization is more prevalent in VICReg. %The trends across number of training epochs and training set sizes are consistent with those observed in Figures \ref{fig:dejavu epochs and dataset} and \ref{fig:dejavu criteria and architecture}.
}
\label{fig:partition attack main appendix}
\end{figure}
% \paragraph{Effect of training set size.} 
% Figure \ref{fig:dejavu v. n} shows how \dejavu memorization responds to the model's training set size. The number of training epochs is fixed to 1000. Interestingly, training set size appears to have almost \emph{no} influence on the \dejavu score (red line), indicating that memorization is equally prevalent with a 100K dataset and a 500K dataset (which suggests that \dejavu memorization may be detectable for larger datasets). Meanwhile, the linear probe train-test accuracy gap \emph{declines} by half as the dataset size grows, failing to represent the memorization quantified by our test. 
% The trend is completely different according to linear probe accuracy (dark blue line), the train-test gap shrinks substantially when increasing the training set size from 100K to 500K. This highlights that the train-test gap is not able to capture \dejavu memorization. %Our evidence suggests that \dejavu memorization may be detectable even for large-scale training datasets. 
%\vspace{-.75em}

\vspace{-.75em} 
\subsection{Sample-level Memorization}
\label{sec:dissection}

% Section \ref{sec:label inference accuracy} shows the \emph{average} level of \dejavu memorization on a subset of the training set $\calA$. However, this average tell us only what the attacker success rate might be without explicitly describing how much of the datatset is \dejavu memorized.
The \dejavu score shows, \emph{on average}, how much better an adversary can select and classify images when using the target model trained on them. 
This average score does not tell us how many individual images have their label successfully recovered by $\KNN_A$ but not by $\KNN_B$. In other words, how many images are exposed by virtue of \emph{being in training set} $\calA$: a risk notion foundational to differential privacy. 
% However, from the perspective of an individual image $A_i \in \calA$, it is informative to know whether it was correctly classified 
To better quantify what fraction of the dataset is at risk, we perform a sample-level analysis by fixing a sample $A_i \in \calA$ and observing the label inference result of $\KNN_A$ vs. $\KNN_B$.
To this end, we partition samples $A_i \in \calA$ based on the result of label inference into four distinct categories: {\color{gray}\textbf{Unassociated}} - label inferred with neither KNN; {\color{part_orange}\textbf{Memorized}} - label inferred only with $\KNN_A$; {\color{part_red}\textbf{Misrepresented}} - label inferred only with $\KNN_B$; {\color{part_blue}\textbf{Correlated}} - label inferred with both KNNs. 
% \begin{multicols}{2}
% \begin{itemize}
%     \vspace{-.75em}
%     \setlength\itemsep{0.15em}
%     \item {\color{gray}Unassociated}: label inferred with neither KNN   
%     \item {\color{part_orange}Memorized}: label only inferred by $\KNN_A$
%     \item {\color{part_red}Misrepresented}: label only inferred with $\KNN_B$
%     \item {\color{part_blue}Correlated}: label inferred with both KNNs
%     \vspace{-.75em}
% \end{itemize}
% \end{multicols}
Intuitively, {\color{gray}unassociated} samples are ones where the embedding of $\crop{A_i}$ does not encode information about the label. {\color{part_blue}Correlated} samples are ones where the label can be inferred from $\crop{A_i}$ using correlation, \emph{e.g.}, inferring the foreground object is basketball given a crop showing a basketball player. Ideally, the {\color{part_red}misrepresented} set should be empty but contains a small portion of examples due to chance.
\emph{Déjà vu} memorization occurs for {\color{part_orange}memorized} samples where the embedding of $\SSL_B$ does not encode the label but the embedding of $\SSL_A$ does. To measure the pervasiveness of \dejavu memorization, we compare the size of the {\color{part_orange}memorized} and {\color{part_red}misrepresented} sets.
Figure \ref{fig:partition attack main} shows how the four categories of examples change with number of training epochs and training set size. The {\color{gray}unassociated} set is not shown since the total share adds up to one. The {\color{part_red}misrepresented} set remains under $5\%$ and roughly unchanged across all settings, consistent with our explanation that it is due to chance. In comparison, VICReg's {\color{part_orange}memorized} set surpasses $15\%$ at 1000 epochs. Considering that up to 5\% of these memorized examples could also be due to chance, we conclude that \textbf{at least 10\% of VICReg's training set is \dejavu memorized.} 
%is many times larger than its {\color{part_red}misrepresented} set, indicating substantial sample-level \dejavu memorization. 
%In fact, \textbf{it is 15\% of the training set that is \dejavu memorized with VICReg.}
%The trends across different number of training epochs and training set sizes match those observed in Section \ref{sec:label inference accuracy}. % On the other hand, the supervised model's {\color{part_orange}memorized} set is only marginally larger than its {\color{part_red}misrepresented} set.

% The trends across different number of training epochs and training set sizes match those observed in Section \ref{sec:label inference accuracy}: Increasing the number of epochs increases \dejavu memorization (Figure \ref{fig:per sample v. training epochs}), while increasing the training set size does not appear to reduce \dejavu memorization (Figure \ref{fig:per sample v. n}). 
%\section{Visualizing \emph{Déjà Vu} Memorization}
\label{sec:visualizing}
Beyond enabling label inference using a periphery crop, we show that \dejavu memorization allows the SSL model to encode other forms of information about a training image. Namely, we train an RCDM \citep{RCDM} on the public dataset $\calX$ and use it to visually reconstruct training images given their periphery crop.
We aim to answer the following two questions: \textbf{(1)} Can we visualize the distinction between correlation and \dejavu memorization? \textbf{(2)} What foreground object details can be extracted from the SSL model beyond class label? 
% \begin{enumerate}[noitemsep, leftmargin=*, topsep=0pt]
%     \item Can we visualize the distinction between correlation and \dejavu memorization? 
%     \item What foreground object details can be extracted from the SSL model beyond class label? 
% \end{enumerate}
\vspace{-0.5em}
\paragraph{Reconstruction pipeline.}
RCDM is a conditional generative model that is trained on the \emph{backbone embedding} of images $X_i \in \calX$ to generate an image that resembles $X_i$. All training images are first face-blurred for privacy purposes. \citet{RCDM} showed that the backbone embedding of SSL models contains more low-level information about the image, making them better suited for conditioning the RCDM.
At test time, following the pipeline in Figure \ref{fig:split_and_pipeline_cartoon}, we first use the projector embedding to find the KNN subset for the periphery crop, $\crop{A_i}$, and then average their backbone embeddings as input to the RCDM model. Ideally, when the public set contains enough representative images, the average representation of the KNN subset encodes objects present in $A_i$, and the RCDM model decodes this representation to visualize these objects.
% \begin{figure}[ht]
%%%
%VICREG
%%%
     \centering
     \begin{subfigure}[b]{0.49\textwidth}
         \centering
         \includegraphics[width=\textwidth]{figures/sample_level_training_epochs.pdf}
         \caption{Categories of training samples vs. number of epochs}
         \label{fig:sample level epochs}
     \end{subfigure}
     \hfill
     \begin{subfigure}[b]{0.49\textwidth}
         \centering
         \includegraphics[width=\textwidth]{figures/sample_level_training_set_size.pdf}
         \caption{Categories of training samples vs. training set size}
         \label{fig:sample level training size}
     \end{subfigure}
\caption{
\definecolor{part_blue}{rgb}{0.2824, 0.4706, .8157}
\definecolor{part_red}{rgb}{0.8392, 0.3725, 0.3725}
\definecolor{part_orange}{rgb}{0.9333, 0.5216, 0.2902}
Partition of samples $A_i \in \calA$ into the four categories: {\color{gray}unassociated} (not shown), {\color{part_orange}memorized}, {\color{part_red}misrepresented} and {\color{part_blue}correlated}. The {\color{part_orange}memorized} samples---ones whose labels are predicted by $\KNN_A$ but not by $\KNN_B$---occupy a significantly larger share for VICReg compared to the supervised model, indicating that sample-level \dejavu memorization is more prevalent in VICReg. %The trends across number of training epochs and training set sizes are consistent with those observed in Figures \ref{fig:dejavu epochs and dataset} and \ref{fig:dejavu criteria and architecture}.
}
\label{fig:partition attack main appendix}
\end{figure}
%\begin{figure*}[t!]
%%%
%DAM
%%%
     \centering
     \begin{subfigure}[b]{0.49\textwidth}
         \centering
         \includegraphics[width=\textwidth]{figures/dam_corr.png}
         \caption{A {\color{part_blue}correlated} dam example}
         \label{fig:dam correlated}
     \end{subfigure}
     \hfill
     \begin{subfigure}[b]{0.49\textwidth}
         \centering
         \includegraphics[width=\textwidth]{figures/dam_mem.png}
         \caption{A {\color{part_orange}memorized} dam example}
         \label{fig:dam memorized}
     \end{subfigure}
\caption{
{\color{part_blue}Correlated} and {\color{part_orange}Memorized} examples from the \emph{dam} class. Both $\SSL_A$ and $\SSL_B$ are SimCLR models.
\textbf{Left:} The periphery crop (pink square) contains a concrete structure that is often present in images of dams. Consequently, the trained RCDM can reconstruct the foreground object using representations from both $\SSL_A$ and $\SSL_B$ through this correlation.
\textbf{Right:} The periphery crop only contains a patch of water. The embedding produced by $\SSL_B$ only contains enough information to infer that the foreground object is related to water, as reflected by its KNN set and RCDM reconstruction. In contrast, the embedding produced by $\SSL_A$ memorizes the association of this patch of water with dam and the RCDM can visualize the embedding to produce images of dams.
}
\vspace{-1ex}
\label{fig:mem v corr dam}
\end{figure*}


\begin{figure*}[t!]
%%%
%DAM
%%%
     \centering
     \begin{subfigure}[b]{0.49\textwidth}
         \centering
         \includegraphics[width=\textwidth]{figures/dam_corr.png}
         \caption{A {\color{part_blue}correlated} dam example}
         \label{fig:dam correlated}
     \end{subfigure}
     \hfill
     \begin{subfigure}[b]{0.49\textwidth}
         \centering
         \includegraphics[width=\textwidth]{figures/dam_mem.png}
         \caption{A {\color{part_orange}memorized} dam example}
         \label{fig:dam memorized}
     \end{subfigure}
\caption[Correlated and Memorized examples from the \emph{dam} class.]{
Correlated and Memorized examples from the \emph{dam} class. Both $\SSL_A$ and $\SSL_B$ are SimCLR models.
\textbf{Left:} The periphery crop (pink square) contains a concrete structure that is often present in images of dams. Consequently, the trained RCDM can reconstruct the foreground object using representations from both $\SSL_A$ and $\SSL_B$ through this correlation.
\textbf{Right:} The periphery crop only contains a patch of water. The embedding produced by $\SSL_B$ only contains enough information to infer that the foreground object is related to water, as reflected by its KNN set and RCDM reconstruction. In contrast, the embedding produced by $\SSL_A$ memorizes the association of this patch of water with dam and the RCDM can visualize the embedding to produce images of dams.
}
\label{fig:mem v corr dam}
\end{figure*}


\begin{figure}[t!]
%%%
%BADGER
%%%
     \centering
     \begin{subfigure}[b]{0.49\textwidth}
         \centering
         \includegraphics[width=\textwidth]{figures/euro_badgers.png}
         \caption{{\color{part_orange}Memorized} European badgers}
         \label{fig:euro badgers}
     \end{subfigure}
     \hfill
     \begin{subfigure}[b]{0.49\textwidth}
         \centering
         \includegraphics[width=\textwidth]{figures/amer_badgers.png}
         \caption{{\color{part_orange}Memorized} American badgers}
         \label{fig:amer badgers}
     \end{subfigure}
\caption[Visualization of \dejavu memorization beyond class label.]{
Visualization of \dejavu memorization beyond class label. Both $\SSL_A$ and $\SSL_B$ are VICReg models. 
The four images shown belong to the memorized set of $\SSL_A$ from the \emph{badger} class. RCDM reconstruction using embeddings from $\SSL_A$ can reveal not only the correct class label, but also the specific badger species: \emph{European} (left) and \emph{American} (right). Such information does not appear to be memorized by the reference model $\SSL_B$.
} 
\label{fig:in class badger}
\end{figure}


% \subsection{Visualizing Correlation vs. Memorization}
\label{sec:mem v corr}
\vspace{-0.5em} 
\paragraph{Visualizing Correlation vs. Memorization.}
Figure \ref{fig:mem v corr dam} shows examples of dams from the {\color{part_blue}correlated} set (left) and the {\color{part_orange}memorized} set (right) as defined in Section \ref{sec:dissection}, along with the associated KNN set and RCDM reconstruction. Both $\SSL_A$ and $\SSL_B$ are SimCLR models. In Figure \ref{fig:dam correlated}, the periphery crop is represented by the pink square, which contains concrete structure attached to the dam's main structure. As a result, both $\SSL_A$ and $\SSL_B$ produce embeddings of $\crop{A_i}$ whose KNN set in $\calX$ consist of dams, \emph{i.e.}, there is a correlation between the concrete structure in $\crop{A_i}$ and the foreground dam. The RCDM reconstructions also consist of dams or structures that closely resemble dams. 
In Figure \ref{fig:dam memorized}, the periphery crop only contains a patch of water, which does not strongly correlate with dams in the ImageNet distribution. Evidently, the reference model $\SSL_B$ embeds $\crop{A_i}$ close to that of other objects commonly found in water, such as sea turtle and submarine. In contrast, the KNN set according to $\SSL_A$ all contain dams despite the vast number of alternative possibilities within the ImageNet classes, and the RCDM reconstruction outputs dams as well which highlight memorization in $\SSL_A$ between this specific patch of water and the dam. %\footnote{See Appendix \ref{sec:appx visualization} to see the same trend in the \emph{yellow garden spider} class.}


% \subsection{Visualizing Memorization Beyond Class Label}
% \label{sec:in class variation}
\vspace{-0.5em} 
\paragraph{Visualizing Memorization Beyond Class Label.}
We now use our reconstruction algorithm to show that \dejavu memorization can be exploited to reveal detailed information beyond class label. Figure \ref{fig:in class badger} shows four examples of badgers from the {\color{part_orange}memorized} set. In all four images, the periphery crop (pink square) does not contain any indication that the foreground object is a badger. Despite this, the KNN set and the RCDM reconstruction using $\SSL_A$ consistently produce images of badgers, while the same does not hold for $\SSL_B$.
More interestingly, reconstructions using $\SSL_A$ in Figure \ref{fig:euro badgers} all contain \emph{European} badgers, while reconstructions in Figure \ref{fig:amer badgers} all contain \emph{American} badgers, accurately reflecting the species of badger present in the respective training images. Since ImageNet-1K does \emph{not} differentiate between these two species of badgers, our reconstructions show that SSL models can memorize information that is highly specific to a training sample beyond its class label\footnote{See Appendix \ref{sec:appx visualization} for additional visualization experiments.}.%\footnote{See Appendix \ref{sec:appx visualization} for the same trend in the \emph{aircraft carrier} class.}.





%\vspace{-.5em} 
\section{Mitigation of \dejavu memorization}
\label{sec:mitigation}
% We do not have an understanding on why \dejavu occur so strongly in some SSL pretraining, however we present additional experiments that shed light on which parameters have the biggest impact on \dejavu memorization.
\begin{figure}[ht]
\label{fig:mitigations}
\begin{minipage}[t]{0.5\textwidth}
\centering
     \begin{subfigure}[b]{0.47\textwidth}
         \centering
         \includegraphics[width=\textwidth]{figures/dejavu_vicreg_param.png}
         \vspace{-1.5em}
         \caption{Loss hyper-parameter}
         \label{fig:dejavu v. invariance}
     \end{subfigure}
     \begin{subfigure}[b]{0.49\textwidth}
         \centering
         \includegraphics[width=\textwidth]{figures/deja_vu_vs_layer.png}
         \vspace{-1.5em}
         \caption{Guillotine regularization}
         \label{fig:dejavu v. guillotine}
     \end{subfigure}~
     \vspace{-0.5em}
    \caption[Effect of two kinds of hyper-parameters on VICReg memorization. ]{
    Effect of two kinds of hyper-parameters on VICReg memorization. \textbf{Left:} \dejavu score (red) versus the \emph{invariance} loss parameter, $\lambda$, used in the VICReg criterion (100k dataset). Larger $\lambda$ significantly reduces \dejavu, with minimal effect on linear probe validation performance (green). $\lambda = 25$ (near maximum \dejavu) is recommended in the original paper \textbf{Right:} \dejavu score versus projector layer---guillotine regularization \cite{Guillotine}---from projector to backbone. Removing the projector can significantly reduce \dejavu. Appendix \ref{sec:guillotine} shows that the backbone still can memorize, however; we demonstrate reconstructions using the SimCLR backbone.
    }
\end{minipage}
\hfill
\begin{minipage}[t]{0.48\textwidth}
\centering
     \begin{subfigure}[b]{0.46\textwidth}
         \centering
         \includegraphics[width=\textwidth]{figures/deja_vu_vs_parameters.png}
         \vspace{-1.3em}
         \caption{\dejavu vs. capacity}
         \label{fig:dejavu v. capacity}
     \end{subfigure}
     \hfill
     \begin{subfigure}[b]{0.52\textwidth}
          \tiny
          \centering
          \setlength{\tabcolsep}{3pt}
          \begin{tabular}{|c|c|c|}
            \hline
            Criteria & DV & Acc P/B \\
            \hline
            Supervised & 8.9 & 55.3/61.1\\
            \hline
            Byol\citep{grill2020byol} & 8.0& 54.3/59.4\\
            \hline
            SimCLR\citep{chen2020simclr} & 10.0 & 44.2/54.1\\
            \hline
            Dino\citep{Dino} & 14.5 & 26.3/55.7 \\
            \hline
            Barlow T.\citep{zbontar2021barlow} & 30.5 & 33.7/54.4\\
            \hline
            VICReg\citep{vicreg} & \textbf{33.2} & 40.3/55.2\\
            \hline
          \end{tabular}
          \vspace{1.3em}
          % \caption{\dejavu (DV) vs. SSL Criterion}
          \caption{\dejavu (DV) vs. Criterion}
          \label{tab:dejavu vs. criterion}
    \end{subfigure}
    \vspace{-1.4em}
    \caption[Effect of model architecture and criterion on \dejavu memorization.]{
    %Comparison of \dejavu score for different architectures and training criteria. 
    Effect of model architecture and criterion on \dejavu memorization. 
    \textbf{Left:} \dejavu score with VICReg for resnet (purple) and vision transformer (green) architectures versus number of model parameters. As expected, memorization grows with larger model capacity. This trend is more pronounced for convolutional (resnet) than transformer (ViT) architectures. \textbf{Right:} Comparison of \dejavu score 20\% conf. and ImageNet linear probe validation accuracy (P: using projector embeddings, B: using backbone embeddings) for various SSL criteria. %\textbf{Nearly all SSL models have more memorization than the supervised baseline.} 
    % Effect of training epochs and train set size on \dejavu score.
    % \textbf{Left:} \dejavu score increases with higher number of training epochs, indicating worsening memorization.
    % \textbf{Right:} \dejavu score stays roughly constant with training set size. Both trends are not captured according to the linear probe train-test gap---a common method to evaluate generalization of SSL representations.
    }
    \end{minipage}
\end{figure}
We cannot yet make claims on why \dejavu occurs so strongly for some SSL training settings and not for others. To gain some intuition for future work, we present additional observations that shed light on which parameters have the most salient impact on \dejavu memorization.
\vspace{-.75em}
\paragraph{Déjà vu memorization worsens by increasing number of training epochs.} 
Figure \ref{fig:dejavu v. training epochs} shows how \dejavu memorization changes with number of training epochs for VICReg. The training set size is fixed to 300K samples. From 250 to 1000 epochs, the \dejavu score (red curve) grows \emph{threefold}: from under 10\% to over 30\%. Remarkably, this trend in memorization is \emph{not} reflected by the linear probe gap (dark blue), which only changes by a few percent beyond 250 epochs. 

%\vspace{-.75em}
\paragraph{Training set size has minimal effect on \dejavu memorization.} Figure \ref{fig:dejavu v. n} shows how \dejavu memorization responds to the model's training set size. The number of training epochs is fixed to 1000. Interestingly, training set size appears to have almost \emph{no} influence on the \dejavu score (red line), indicating that memorization is equally prevalent with a 100K dataset and a 500K dataset. This result suggests that \dejavu memorization may be detectable even for large datasets. Meanwhile, the standard linear probe train-test accuracy gap \emph{declines} by more than half as the dataset size grows, failing to represent the memorization quantified by our test. 
% The trend is completely different according to linear probe accuracy (dark blue line), the train-test gap shrinks substantially when increasing the training set size from 100K to 500K. This highlights that the train-test gap is not able to capture \dejavu memorization. Our evidence suggests that \dejavu memorization may be detectable even for large-scale training datasets. 
\vspace{-0.5em}
\paragraph{Training loss hyper-parameter has a strong effect.} 
%We show in Figure \ref{fig:dejavu v. training epochs} that the number of training epochs is an important factor that can increase significantly \dejavu memorization. In contrast, the dataset size does not impact much \dejavu as shown in Figure \ref{fig:dejavu epochs train set size}. 
Loss hyper-parameters, like VICReg's invariance coefficient (Figure \ref{fig:dejavu v. invariance}) or SimCLR's temperature parameter (Appendix Figure \ref{fig:simclr temperature}) significantly impact \dejavu with minimal impact on the linear probe validation accuracy.

\vspace{-0.5em}
\paragraph{Some SSL criteria promote stronger \dejavu memorization.} Table \ref{tab:dejavu vs. criterion} demonstrates that the degree of memorization varies widely for different training criteria. VICReg and Barlow Twins have the highest \dejavu scores while SimCLR and Byol have the lowest.
%\footnote{We show detailed reconstructions of SimCLR's training data in Section \ref{sec:visualizing} despite its relatively low degree of \dejavu.}.
With the exception of Byol, all SSL models have more \dejavu memorization than the supervised model. Interestingly, different criteria can lead to similar linear probe validation accuracy and very different degrees of \dejavu as seen with SimCLR and Barlow Twins. Note that low degrees of \dejavu can still risk training image reconstruction, as exemplified by the SimCLR reconstructions in Figures \ref{fig:mem v corr dam} and \ref{fig:mem v corr spider}. 
%\vspace{-1em}
\vspace{-0.5em}
\paragraph{Larger models have increased \dejavu memorization.} Figure \ref{fig:dejavu v. capacity} validates the common intuition that lower capacity architectures (Resnet18/34) result in less memorization than their high capacity counterparts (Resnet50/101). 
% \begin{wrapfigure}{r}{0.25\textwidth} 
%     \centering
%     \includegraphics[width=0.25\textwidth]{figures/attk_layer_acc_top1_legend.pdf}
%     \caption{\dejavu memorization versus layer from backbone (0) to projector output (3).}
%     \label{fig:dejavu vs layer}
%     \vspace{-8ex}
% \end{wrapfigure}
We see the same trend for vision transformers as well. %This comes with a tradeoff, since reduced model capacity can result in a nontrivial degradation of representation quality\cite{vicreg, simclr}.  
\vspace{-0.5em}
\paragraph{Guillotine regularization can help reduce \dejavu memorization.} Previous experiments were done using the projector embedding. In Figure \ref{fig:dejavu v. guillotine}, we present how Guillotine regularization\citep{Guillotine} (removing final layers in a trained SSL model) impacts \dejavu with VICReg\footnote{Further experiments are available in Appendix \ref{sec:guillotine}.}. Using the backbone embedding instead of the projector embedding seems to be the most straightforward way to mitigate \dejavu memorization. However, as demonstrated in Appendix \ref{sec:appx backbone results}, backbone representation with low \dejavu score can still be leveraged to reconstruct some of the training images.

\section{Conclusion}
\label{sec:conclusion}

We defined and analyzed \dejavu memorization, a notion of unintended memorization of partial information in image data. As shown in Sections \ref{sec:quant} and \ref{sec:visualizing}, SSL models can largely exhibit \dejavu memorization on their training data, and this memorization signal can be extracted to infer or visualize image-specific information.
Since SSL models are becoming increasingly widespread as foundation models for image data, negative consequences of \dejavu memorization can have profound downstream impact and thus deserves further attention. 
Future work should focus on understanding how \dejavu emerges in the training of SSL models and why methods like Byol are much more robust to \dejavu than VICReg and Barlow Twins. In addition, trying to characterize which data points are the most at risk of \dejavu could be crucial to get a better understanding on this phenomenon. 
