\graphicspath{{./chapters/chapter1}}
\chapter{When are Non-Parametric Methods Robust?} 

\newtheorem{thm}{Theorem}
\newtheorem{lem}[thm]{Lemma}
\newtheorem{cor}[thm]{Corollary}

\newtheorem{defn}[thm]{Definition}
\newtheorem{ex}{Example}

\def\D{{\mathcal D}}
\def\X{\mathcal X}
\def\R{\mathbb R}
\def\Y{\{\pm 1\}}
\def\w{\hat{w}}
\def\P{\mathbb{P}}
\def\I{\hat{I}}
\def\b{g^*}
\def\r{\rho}
\def\rcons{r-consistent}
\def\rconsy{r-consistency}
\def\ap{AdvPrun}
\def\ga{RobustNonPar}

\section{Introduction}

Recent work has shown that many classifiers tend to be highly non-robust and that small strategic modifications to regular test inputs can cause them to misclassify~\cite{Szegedy14, Goodfellow14, MeekLowd05}. Motivated by the use of machine learning in safety-critical applications, this phenomenon has recently received considerable interest; however, what exactly causes this phenomenon -- known in the literature as {\em{adversarial examples}} -- still remains a mystery.

Prior work has looked at three plausible reasons why adversarial examples might exist. The first, of course, is the possibility that in real data distributions, different classes are very close together in space -- which does not seem plausible in practice. Another possibility is that classification algorithms may require more data to be robust than to be merely accurate; some prior work~\cite{Madry18, WJC18, Srebro19} suggests that this might be true for certain classifiers or algorithms. Finally, others~\cite{Bubeck19, Vinod19, WJC18} have suggested that better training algorithms may give rise to more robust classifiers -- and that in some cases, finding robust classifiers may even be computationally challenging.

In this work, we consider this problem in the context of general non-parametric classifiers. Contrary to parametrics, non-parametric methods are a form of local classifiers, and include a large number of pattern recognition methods such as nearest neighbors, decision trees, random forests and kernel classifiers. There is a richly developed statistical theory of non-parametric methods~\cite{devroye96}, which focuses on accuracy, and provides very general conditions under which these methods converge to the Bayes optimal with growing number of samples. We, in contrast, analyze robustness properties of these methods, and ask instead when they converge to the classifier with the highest astuteness at a desired radius $r$. Recall that the astuteness of a classifier at radius $r$ is the fraction of points from the distribution on which it is accurate and has the same prediction up to a distance $r$~\cite{WJC18, Madry18}.

 We begin by looking at the very simple case when data from different classes is well-separated -- by at least a distance $2r$. Although achieving astuteness in this case may appear trivial, we show that even in this highly favorable case, not all non-parametric methods provide robust classifiers -- and this even holds for methods that converge to the Bayes optimal in the large sample limit.  

This raises the natural question -- when do non-parametric methods produce astute classifiers? We next provide conditions under which a non-parametric method converges to the most astute classifier in the large sample limit under well-separated data. Our conditions are analogous to the classical conditions for convergence to the Bayes optimal~\cite{devroye96, Stone77}, but a little stronger. We show that nearest neighbors and kernel classifiers whose kernel functions decay fast enough, satisfy these conditions, and hence converge to astute classifiers in the large sample limit. In constrast, histogram classifiers, which do converge to the Bayes optimal in the large sample limit, may not converge to the most astute classifier. This indicates that there may be some non-parametric methods, such as nearest neighbors and kernel classifiers, that are more naturally robust when trained on well-separated data, and some that are not.

What happens when different classes in the data are not as well-separated? For this case, \cite{YRWC19} proposes a method called Adversarial Pruning that preprocesses the training data by retaining the maximal set of points such that different classes are distance $\geq 2r$ apart, and then trains a non-parametric method on the pruned data. We next prove that if a non-parametric method has certain properties, then the classifier produced by Adversarial Pruning followed by the method does converges to the most astute classifier in the large sample limit. We show that again nearest neighbors and kernel classifiers whose kernel functions decay faster than inverse polynomials satisfy these properties. Our results thus complement and build upon the empirical results of~\cite{YRWC19} by providing a performance guarantee. 

What can we conclude about the cause for adversarial examples? Our results seem to indicate that at least for non-parametrics, it is mostly the training algorithms that are responsible. With a few exceptions, decades of prior work in machine learning and pattern recognition has largely focussed on designing training methods that provide increasingly accurate classifiers -- perhaps to the detriment of other aspects such as robustness. In this context, our results serve to (a) provide a set of guidelines that can be used for designing non-parametric methods that are robust and accurate on well-separated data and (b) demonstrate that when data is not well-separated, preprocessing through adversarial pruning~\cite{YRWC19} may be used to ensure convergence to optimally astute solutions in the large sample limit. 

\subsection{Related Work}

There is a large body of work on adversarial attacks~\cite{Carlini17, Liu17, Papernot17, Papernot16,Szegedy14} and defenses~\cite{Hein17,Katz17,Schmidt18,Wu16,Steinhardt18, Sinha18} in the parametric setting, specifically focusing on neural networks. On the other hand, adversarial examples for nonparametric classifiers have mostly been studied in a much more ad-hoc manner, and to our knowledge, there has been no theoretical investigation into general properties of algorithms that promote robustness in non-parametric classifiers.

For nearest neighbors, there has been some prior work on adversarial attacks~\cite{Amsaleg17, Sitawarin19, WJC18, YRWC19} as well as defenses. Wang et. al. \cite{WJC18} proposes a defense for 1-NN by pruning the input sample. However, their defense learns a classifier whose robustness regions converge towards those of the Bayes optimal classifier, which itself may potentially have poor robustness properties. Yang et. al. \cite{YRWC19} accounts for this problem by proposing the notion of the $r$-optimal classifier, and propose an algorithm called Adversarial Pruning which can be interpreted as a finite sample approximation to the $r$-optimal. However, they do not provide formal performance guarantees for Adversarial Pruning, which we do. 

For Kernel methods, Hein and Andriushchenko \cite{Hein17} study lower bounds on the norm of the adversarial manipulation that is required for changing a classifiers output. They specifically study bounds for Kernel Classifiers, and propose an empirically based regularization idea that improves robustness. In this work, we improve the robustness properties of kernel classification through adversarial pruning, and show formal guarantees regarding convergence towards the $r$-optimal classifier. 

For decision trees and random forests, attacks and defenses have been provided by \cite{Hein19, Kantchelian15, Hsiehicml19}. Again, most of the work here is empirical in nature, and convergence guarantees are not provided. 

Pruning has a long history of being applied for improving nearest neighbors \cite{Gates72, Gottlieb14, Hart68, KontorovichSW17, KontorovichW15, Hanneke19}, but this has been entirely done in the context of generalization, without accounting for robustness. In their work, Yang et. al. empirically show that adversarial pruning can improve robustness for nearest neighbor classifiers. However, they do not provide any formal guarantees for their algorithms. In this work, we prove formal guarantees for \textit{adversarial pruning} in the large sample limit, both for nearest neighbors as well as for more general \textit{weight functions.} 

There is a long history of literature for understanding the consistency of Kernel classifiers \cite{Steinwart05, Stone77}, but this has only been done for accuracy and generalization. In this work, we find different conditions are needed to ensure that a Kernel classifier converges in robustness in addition to accuracy.

\section{Preliminaries}

\subsection{Setting}
We consider binary classification where instances are drawn from a totally bounded metric space $\X$ that is equipped with distance metric denoted by $d$, and the label space is $\Y = \{ -1, +1 \}$. The classical goal of classification is to build a highly \textit{accurate} classifier, which we define as follows.

\begin{defn}
(Accuracy) Let $\D$ be a distribution over $\X \times \Y$, and let $f \in \Y^\X$ be a classifier. Then the \textbf{accuracy} of $f$ over $\D$, denoted $A(f, \D)$, is the fraction of examples $(x,y) \sim \D$ for which $f(x) = y$. Thus $$A(f, \D) = P_{(x,y) \sim \D}[f(x) = y].$$
\end{defn}

In this work, we consider \textit{robustness} in addition to accuracy. Let $B(x,r)$ denoted the closed ball of radius $r$ centered at $x$. 

\begin{defn}
(Robustness) A classifier $f \in \Y^\X$ is said to be \textbf{robust} at $x$ with radius $r$ if $f(x) = f(x')$ for all $x' \in B(x,r)$.
\end{defn}

Our goal is to find non-parametric algorithms that output classifiers that are robust, in addition to being accurate. To account for both criteria, we combine them into a notion of \textit{astuteness}~\cite{WJC18, Madry18}. 

\begin{defn}
(Astuteness) A classifier $f \in \Y^\X$ is said to be \textbf{astute} at $(x,y)$ with radius $r$ if $f$ is robust at $x$ with radius $r$ and $f(x) = y$. The \textbf{astuteness} of $f$ over $\D$, denoted $A_r(f, \D)$, is the fraction of examples $(x,y) \sim \D$ for which $f$ is astute at $(x,y)$ with radius $r$. Thus $$A_r(f, \D) = P_{(x, y) \sim \D}[f(x') = y, \forall x' \in B(x,r)].$$
\end{defn}

It is worth noting that $A_0(f, \D) = A(f, \D)$, since astuteness with radius $0$ is simply the accuracy. For this reason, we will use $A_0(f, \D)$ to denote accuracy from this point forwards.

\subsection{Notions of Consistency}

Traditionally, a classification algorithm is said to be consistent if as the sample size grows to infinity, the accuracy of the classifier it learns converges towards the best possible accuracy on the underlying data distribution. We next introduce and formalize an alternative form of consistency, called $r$-consistency, that applies to robust classifiers.

We begin with a formal definition of the Bayes Optimal Classifier -- the most accurate classifier on a distribution -- and consistency. 

\begin{defn}
(Bayes Optimal Classifier) The \textbf{Bayes Optimal Classifier} on a distribution $\D$, denoted by $\b$, is defined as follows. Let $\eta(x) = p_\D(+1|x)$. Then
 \[ \b(x) = \begin{cases} 
      +1 & \eta(x) \geq 0.5 \\
      -1 & \eta(x) < 0.5 \\
   \end{cases}
\]
It can be shown that $\b$ achieves the highest accuracy over $\D$ over all classifiers.
\end{defn}

\begin{defn}
(Consistency) Let $M$ be a classification algorithm  over $\X \times \Y$. $M$ is said to be \textbf{consistent} if for any $\D$ over $\X \times \Y$, and any $\epsilon, \delta$ over $(0,1)$, there exists $N$ such that for $n \geq N$, with probability $1-\delta$ over $S \sim \D^n$, we have: $$A(M(S), \D) \geq A(\b, \D) - \epsilon,$$ where $\b$ is the Bayes optimal classifier for $\D$. 
\end{defn}

How can we incorporate robustness in addition to accuracy in this notion? A plausible way, as used in~\cite{WJC18}, is that the classifier should converge towards being astute where the Bayes Optimal classifier is astute. However, the Bayes Optimal classifier is not necessarily the most astute classifier and may even have poor astuteness. To see this, consider the following example. 

\paragraph{Example 1}
Consider $\D$ over $\X = [0,1]$ such that $\D_\X$ is the uniform distribution and $$p(y=1|x) = \frac{1}{2} + \sin \frac{4 \pi x}{r}.$$ For any point $x$, there exists $x_1, x_2 \in ([x-r, x+r] \cap [0,1])$ such that $p(y=1|x_1) > \frac{1}{2}$ and $p(y=1|x_2) < \frac{1}{2}$. $A_r(\b, r) = 0$. However, the classifier that always predicts $f(x) = +1$ does better. It is robust everywhere, and since $P_{(x,y) \sim \D}[y = +1] = \frac{1}{2}$, it follows that $A_r(f, \D) = \frac{1}{2}$. \\ \\

This motivates the notion of the $r$-optimal classifier, introduced by~\cite{YRWC19}, which is the classifier with maximum astuteness. 

\begin{defn}
($r$-optimal classifier) The \textbf{$r$-optimal classifier} of a distribution $G$ denoted by $\b_r$ is the classifier with maximum astuteness. Thus $$\b_r = \argmax_{f \in \Y^\X} A_r(f, \D).$$ We let $A_r^*(\D)$ denote $A_r(\b_r, \D)$. 
\end{defn}

Observe that $\b_r$ is not necessarily unique. To account for this, we use $A_r^*(\D)$ in our definition for \rconsy. 

\begin{defn} \label{defn_archons}
(\rcons) Let $M$ be a classification algorithm over $\X \times \Y$. $M$ is said to be \textbf{\rcons} if for any $\D$,  any $\epsilon, \delta \in (0,1)$, and $0 < \gamma < r$, there exists $N$ such that for $n \geq N$, with probability $1-\delta$ over $S \sim \D^n$, $$A_{r-\gamma}(M(S), \D) \geq A_r^*(\D) - \epsilon.$$ if the above conditions hold for a specific distribution $\D$, we say that $M$ is \rcons\emph{ }with respect to $\D$. 
\end{defn}

Observe that in addition to the usual $\epsilon$ and $\delta$, there is an extra parameter $\gamma$ which measures the gap in the robustness radius. We may need this parameter as when classes are exactly $2r$ apart, we may not be able to find the exact robust boundary with only finite samples. 

Our analysis will be centered around understanding what kinds of algorithms $M$ provide highly astute classifiers for a given radius $r$. We begin by first considering the special case of \textit{$r$-separated} distributions. 

\begin{defn}
($r$-separated distributions) A distribution $\D$ is said to be \textbf{$r$-separated} if there exist subsets $T^+, T^- \subset \X$ such that 
\begin{enumerate}
	\item $\P_{(x,y) \sim \D}[x \in T^y] = 1$. 
	\item $\forall x_1 \in T^+, \forall x_2 \in T^-$, $d(x_1, x_2) > 2r$.
\end{enumerate}
\end{defn}

Observe that if $\D$ is $r$-separated, $A_r(\b_r, \D) = 1$.

\subsection{Non-parametric Classifiers}\label{classifiers}

Many non-parametric algorithms classify points by averaging labels over a local neighborhood from their training data. A very general form of this idea is encapsulated in \textit{weight functions} -- which is the general form we will use.

\begin{defn} \label{def:weight}
\cite{devroye96} A \textbf{weight function} $W$ is a non-parametric classifier with the following properties.
\begin{enumerate}
	\item Given input $S = \{(x_1, y_1), (x_2, y_2,), \dots, (x_n, y_n)\} \sim \D^n$, $W$ constructs functions $w_1^S, \dots, w_n^S: \X \to [0, 1]$ such that for all $x \in \X$, $\sum_1^n w_i^S(x) = 1$. The functions $w_i^S$ are allowed to depend on $x_1, x_2, \dots x_n$ but must be independent of $y_1, y_2, \dots, y_n$. 
	\item $W$ has output $W_S$ defined as \[ W_S(x) = \begin{cases} 
      +1 & \sum_1^n w_i^S(x)y_i > 0 \\
      -1 & \sum_1^n w_i^S(x)y_i \leq 0 \\
   \end{cases}
\]
As a result, $w_i^S(x)$ can be thought of as the weight that $(x_i, y_i)$ has in classifying $x$.
\end{enumerate}
\end{defn}

Weight functions encompass a fairly extensive set of common non-parametric classifiers, which is the motivation for considering them. We now define several common non-parametric algorithms that can be construed as weight functions. 

\begin{defn}
A \textbf{histogram classifier}, $H$, is a non-parametric classification algorithm over $\R^d \times \Y$ that works as follows. For a distribution $\D$ over $\R \times \Y$, $H$ takes $S = \{(x_i, y_i): 1 \leq i \leq n\} \sim \D^n$ as input. Let $k_i$ be a sequence with $\lim_{i \to \infty} k_i = \infty$ and $\lim_{i \to \infty} \frac{k_i}{i} = 0$. $H$ constructs a set of hypercubes $C = \{c_1, c_2, \dots, c_m\}$ as follows:
\begin{enumerate}
	\item Initially $C = \{c\}$, where $S \subset c$.
	\item For $c \in C$, if $c$ contains more than $k_n$ points of $S$, then partition $c$ into $2^d$ equally sized hypercubes, and insert them into $C$.
	\item Repeat step $2$ until all cubes in $C$ have at most $k_n$ points. 
\end{enumerate}
For $x \in \R$ let $c(x)$ denote the unique cell in $C$ containing $x$. If $c(x)$ doesn't exist, then $H_S(x) = -1$ by default. Otherwise, \[ H_S(x) = \begin{cases} 
      +1 & \sum_{x_i \in c(x)} y_i > 0 \\
      -1 & \sum_{x_i \in c(x)}y_i \leq 0 \\
   \end{cases}.
\]
\end{defn}

Histogram classifiers are weight functions in which all $x_i$ contained within the same cell as $x$ are given the same weight $w_i^S(x)$ in predicting $x$, while all other $x_i$ are given weight $0$. 

\begin{defn}
A \textbf{kernel classifier} is a weight function $W$ over $\X \times \Y$ constructed from function $K: \R^+ \cup \{0\} \to \R^+$ and some sequence $\{h_n\} \subset \R^+$ in the following manner. Given $S = \{(x_i, y_i)\} \sim \D^n$, we have $$w_i^S(x) = \frac{K(\frac{d(x, x_i)}{h_n})}{\sum_{j = 1}^n K(\frac{d(x, x_j)}{h_n})}.$$ Then, as above, $W$ has output \[ W_S(x) = \begin{cases} 
      +1 & \sum_1^n w_i^S(x)y_i > 0 \\
      -1 & \sum_1^n w_i^S(x)y_i \leq 0 \\
   \end{cases}
\]
\end{defn}

Finally, we note that $k_n$-nearest neighbors is also a weight function; $w_i^S(x) = \frac{1}{k_n}$ if $x_i$ is one of the $k_n$ closest neighbors of $x$ and $0$ otherwise.

\section{Warm Up: $r$-separated distributions}

We begin by considering the case when the data distribution is $r$-separated; the more general case is considered in Section~\ref{sec:general}. While classifying $r$-separated distributions robustly may appear almost trivial, learning an arbitrary classifier does not necessarily produce an astute result. To see this, consider the following example of a histogram classifier -- which is known to be consistent. 


We let $H$ denote the histogram classifier over $\R$.

\paragraph{Example 2}
Consider the data distribution $\D = \D^+ \cup \D^-$ where $D^+$ is the uniform distribution over $[0, \frac{1}{4})$ and $D^-$ is the uniform distribution over $(\frac{1}{2}, 1]$, $p(+1|x) = 1$ for $x \in \D^+$, and $p(-1|x) = 1$ for $x \in \D^-$. 

We make the following observations (refer to Figure \ref{fig:histogram}).
\begin{enumerate}
	\item $\D$ is $0.1$-separated, since the supports of $\D^+$ and $\D^-$ have distance $0.25 > 0.2$. 
	\item If $n$ is sufficiently large, $H$ will construct the cell $[0.25, 0.5)$, which will not be split because it will never contain any points. 
	\item $H_S(x) = -1$ for $x \in [0.25, 0.5)$.
	\item $H_S$ is not astute at $(x,1)$ for $x \in (0.15, 0.25)$. Thus $A_{0.1}(H_S, \D) = 0.8$.
\end{enumerate}

\begin{figure}
\centering
\definecolor{dtsfsf}{rgb}{0.8274509803921568,0.1843137254901961,0.1843137254901961}
\definecolor{sexdts}{rgb}{0.1803921568627451,0.49019607843137253,0.19607843137254902}
\begin{tikzpicture}[line cap=round,line join=round,>=triangle 45,x=1cm,y=1cm]
\clip(-4.68,0) rectangle (2.46,3);
\draw [line width=2pt] (-3.8971542001619626,1.7601524817792173)-- (-3.90319974718047,1.0800284421971598);
\draw [line width=2pt] (-2.7721542001619626,1.7501524817792171)-- (-2.7781997471804694,1.0700284421971598);
\draw [line width=2pt] (-1.6471542001619623,1.7401524817792173)-- (-1.6531997471804694,1.0600284421971597);
\draw [line width=2pt] (0.6028457998380374,1.720152481779217)-- (0.5968002528195302,1.0400284421971597);
\draw [line width=2pt] (-2.775,1.43)-- (-1.65,1.42);
\draw [line width=2pt,color=sexdts] (-1.65,1.42)-- (0.6,1.4);
\draw [line width=2pt,color=sexdts] (-3.9,1.44)-- (-3.2002330680045032,1.4337798494933733);
\draw [line width=2pt,color=dtsfsf] (-3.2002330680045032,1.4337798494933733)-- (-2.775,1.43);
\draw (-3.54,2.48) node[anchor=north west] {$\mathcal{D}^+$};
\draw (-0.76,2.46) node[anchor=north west] {$\mathcal{D}^-$};
\draw (-4,1.1) node[anchor=north west] {0};
\draw (-3.14,1.08) node[anchor=north west] {0.25};
\draw (-1.86,1.1) node[anchor=north west] {0.5};
\draw (0.5,1.08) node[anchor=north west] {1};
\end{tikzpicture}
\caption{$H_S$ is astute in the green region, but not robust in the red region.} \label{fig:histogram}
\end{figure}

Example 2 shows that histogram classifiers do not always learn astute classifiers even when run on $r$-separated distributions. This motivates the question: which non-parametric classifiers do?

We answer this question in the following theorem, which gives sufficient conditions for a weight function (definition \ref{def:weight}) to be $r$-consistent over an $r$-separated distribution.

\begin{thm}\label{thm_stone_cons}
Let $\D$ be a distribution over $\X \times \Y$, and let $W$ be a weight function. Let $X$ be a random variable with distribution $\D_\X$, and $S = \{(x_1, y_1), (x_2, y_2), \dots, (x_n, y_n)\} \sim \D^n$. Suppose that for any $0 < a < b,$ $$\lim_{n \to \infty} \mathbb{E}_{X, S} \big [ \sup_{x' \in B(X, a)} \sum_1^n w_i^S(x')I_{||x_i - x'|| > b} \big] = 0.$$  Then if $\D$ is $r$-separated, $W$ is \rcons\emph{ } with respect to $\D$.  
\end{thm}

First, we compare Theorem \ref{thm_stone_cons} to Stone's theorem \cite{Stone77}, which gives sufficient conditions for a weight function to be consistent (i.e. converge in accuracy towards the Bayes optimal). For convenience, we include a statement of Stone's theorem. 
\begin{thm}\label{thm_stone}
\cite{Stone77} Let $W$ be weight function over $\X \times \Y$. Suppose the following conditions hold for any distribution $\D$ over $\X \times \Y$.  Let $X$ be a random variable with distribution $\D_\X$, and $S = \{(x_1, y_1), (x_2, y_2), \dots, (x_n, y_n)\} \sim \D^n$. All expectations are taken over $X$ and $S$. 
\begin{enumerate}
	\item There is a constant $c$ such that, for every nonnegative measurable function $f$ satisfying $\mathbb{E} [f(X)] < \infty$, $$\mathbb{E} [\sum_1^n w_i^S(X)f(x_i)] \leq c \mathbb{E} [f(x)].$$
	\item For all $a > 0$, $$\lim_{n \to \infty} \mathbb{E}[\sum_1^n w_i^S(x)I_{||x_i - X|| > a}] = 0,$$ where $I_{||x_i - X|| > a}$ is an indicator variable. 
	\item $$\lim_{n \to \infty} \mathbb{E}[\max_{1 \leq i \leq n} w_i^S(X)] = 0.$$
\end{enumerate}
Then $W$ is consistent. 
\end{thm}
There are two main differences between Theorem \ref{thm_stone_cons} and Stone's theorem.
 \begin{enumerate}
 	\item Conditions 1. and 3. of Stone's theorem are no longer necessary. This is because $r$-separated distributions are well-separated and thus have simpler conditions for consistency. In fact, a slight modification of the arguments of~\cite{Stone77} shows that for $r$-separated distributions, condition 2. alone is sufficient for consistency.
 	\item Condition 2. is strengthened. Instead of requiring the weight of $x_i$'s outside of a given radius to go to $0$ for $X \sim \D$, we require the same to \textit{uniformly} hold over a ball centered at $X$. 
\end{enumerate}  
 


Theorem \ref{thm_stone_cons} provides a general condition that allows us to verify the $r$-consistency of non-parametric methods. We now show below that two common non-parametric algorithms -- $k_n$-nearest neighbors and kernel classifiers with rapidly decaying kernel functions -- satisfy the conditions of Theorem~\ref{thm_stone_cons}.
 
\begin{cor}\label{nn_sep_thm}
Let $\D$ be any $r$-separated distribution. Let $k_n$ be any sequence such that $\lim_{n \to \infty} \frac{k_n}{n} = 0$, and let $M$ be the $k_n$-nearest neighbors classifier on a sample $S \sim \D^n$. Then $M$ is \rcons\emph{ }with respect to $\D$. 
\end{cor}

\paragraph{\textbf{Remarks:}}
\begin{enumerate}
	\item Because the data distribution is $r$-separated, $k_n = 1$ will be $r$-consistent. Also observe that for $r$-separated distributions, $k_n = 1$ will converge towards the Bayes Optimal classifier.
	\item In general, $M$ converges towards the Bayes Optimal classifier provided that $k_n \to \infty$ in addition to $k_n /n \to 0$. This condition is not necessary for \rconsy -- because the distribution is $r$-separated. 
\end{enumerate}



We next show that kernel classifiers are also $r$-consistent on $r$-separated data distributions, provided the kernel function decreases rapidly enough. 

\begin{cor}\label{thm_kernel}
Let $W$ be a kernel classifier over $\X \times \Y$ constructed from $K$ and $h_n$. Suppose the following properties hold for $K$ and $h_n$.
\begin{enumerate}
	\item For any $c > 1$, $\lim_{x \to \infty} \frac{K(cx)}{K(x)} = 0.$
	\item $\lim_{n \to \infty} h_n = 0.$
\end{enumerate}
If $\D$ is an $r$-separated distribution over $\X \times \Y$, then $W$ is \rcons\emph{ }with respect to $\D$. 
\end{cor}

Observe that Condition 1. is satisfied for any $K(x)$ that decreases more rapidly than an inverse polynomial -- and is hence satisfied by most popular kernels like the Gaussian kernel. Is the condition on $K$ in Corollary~\ref{thm_kernel} necessary? The following example illustrates that a kernel classifier with any arbitrary $K$ is not necessarily $r$-consistent. This indicates that some sort of condition needs to be imposed on $K$ to ensure $r$-consistency; finding a tight necessary condition however is left for future work. 

 \paragraph{Example 3} Let $\X = [-1, 1]$ and let $\D$ be a distribution with $p_\D(-1, -1) = 0.1$ and $p_\D(1, 1) = 0.9$. Clearly, $\D$ is $0.3$-separated. Let $K(x) = e^{-\min(|x|, 0.2)^2}$. Let $h_n$ be any sequence with $\lim_{n \to \infty} h_n = 0$ and $\lim_{n \to \infty} nh_n = \infty$. Let $W$ be the weight classifier with input $S = \{(x_1, y_1), (x_2, y_2), \dots, (x_n, y_n)\}$ such that $$w_i^S(x) = \frac{K(\frac{|x- x_i|}{h_n})}{\sum_{j=1}^n K(\frac{|x-x_j|}{h_n})}.$$ $W$ can be shown to satisfy all the conditions of Theorem \ref{thm_stone} (the proof is analogous to the case for a Gaussian Classifier), and is therefore consistent. However, $W$ does not learn a robust classifier on $\D$ for $r = 0.3$. 

Consider $x = -0.7$. For any $\{(x_1, y_1), (x_2, y_2), \dots, (x_n, y_n)\} \sim \D^n$, all $x_i$ will either be $-1$ or $1$. Therefore, since $K(|x - (-1)|) = K(|x - 1|)$, it follows that $w_i^S(x) = \frac{1}{n}$ for all $1 \leq i \leq n$. Since $x_i = 1$ with probability $0.9$, it follows that with high probability $x$ will be classified as $1$ which means that $f$, the output of $W$, is not robust at $x = -1$. Thus $f$ has astuteness at most $0.9$ which means that $W$ is \textit{not} \rcons\ for $r=0.3$. 

\section{General Distributions}\label{sec:general}


We next consider more general data distributions, where data from different classes may be close together in space, and may even overlap. Observe that unlike the $r$-separated case, here there may be no classifier with astuteness one. Thus, a natural question is: what does the optimally astute classifier look like, and how can we build non-parametric classifiers to this limit?

\subsection{The $r$-Optimal Classifier and Adversarial Pruning}

\cite{YRWC19} propose a large-sample limit -- called the $r$-optimal -- and show that it is analogous to the Bayes Optimal classifier for robustness. More specifically, given a data distribution $D$, to find the $r$-optimal classifier, we solve the following optimization problem.  

\begin{equation}\label{optim_prob}
\begin{split}
\max_{S_{+1}, S_{-1}} &\int_{x \in S_{+1}} p(y=+1|x)d\mu_{\D}(x) + \\
&\int_{x \in S_{-1}} p(y=-1|x)d\mu_{\D}(x) \\
&\text{ subject to } d(S_{+1}, S_{-1}) > 2r 
\end{split}
\end{equation}

Then, the $r$-optimal classifier is defined as follows. 

\begin{defn}
\cite{YRWC19} Fix $r, \D$. Let $S_{+1}^*$ and $S_{-1}^*$ be any optimizers of (\ref{optim_prob}). Then the $r$-optimal classifier, $\b_r$ is any classifier such that $\b_r(x) = j$ whenever $d(S_j^*, x) \leq r$. 
\end{defn}

\cite{YRWC19} show that the $r$-optimal classifier achieves the optimal astuteness -- out of all classifiers on the data distribution $\D$; hence, it is a robustness analogue to the Bayes Optimal Classifier. Therefore, for general distributions, the goal in robust classification is to find non-parametric algorithms that output classifiers that converge towards $\b_r$. 

To find robust classifiers, \cite{YRWC19} propose Adversarial Pruning -- a defense method that preprocesses the training data by making it better separated. More specifically, Adversarial Pruning takes as input a training dataset $S$ and a radius $r$, and finds the largest subset of the training set where differently labeled points are at least distance $2r$ apart. 


\begin{defn}
A set $S_r \subset \X \times \Y$ is said to be \textbf{$r$-separated} if for all $(x_1, y_1), (x_2, y_2) \in S_r$, if $y_1 \neq y_2$, then $d(x_1, x_2) > 2r$. To \textbf{adversarially prune} a set $S$ is to return its largest $r$-separated subset. We let $\ap(S, r)$ denote the result of adversarially pruning $S$.  
\end{defn}

Once an $r$-separated subset $S_r$ of the training set is found, a standard non-parametric method is trained on $S_r$.  While~\cite{YRWC19} show good empirical performance of such algorithms, no formal guarantees are provided. We next formally characterize when adversarial pruning followed by a non-parametric method results in a classifier that is provably $r$-consistent.

Specifically, we consider analyzing the general algorithm provided in Algorithm \ref{alg:gen}.

%\begin{algorithm}[tb]
%   \caption{\ga}
%   \label{alg:gen}
%\begin{algorithmic}
%   \STATE {\bfseries Input:} $S \sim \D^n$, weight function $W$, 
%   robustness radius $r$
%   \STATE $S_r \leftarrow \ap(S, r)$
%   \STATE{\bfseries Output:} $W_{S_r}$
%\end{algorithmic}
%\end{algorithm}

\begin{algorithm}[H]
    \SetAlgoLined
    {\bfseries Input:} $S \sim \D^n$, weight function $W$, robustness radius $r$\;
    
    $S_r \leftarrow \ap(S, r)$\;
    
    {\bfseries Output:} $W_{S_r}$\;
    

\caption{\ga}\label{alg:gen}
\end{algorithm}

\subsection{Convergence Guarantees}

We begin with some notation. For any weight function $W$ and radius $r > 0$, we let $\ga(W,r)$ represent the weight function that outputs weights for $S \sim \D^n$ according to $\ga(S, W, r)$. In particular, this can be used to convert any weight function algorithm into a new weight function which takes robustness into account. A natural question is, for which weight functions $W$ is $\ga(W,r)$ \rcons? Our next theorem provides sufficient conditions for this.

\begin{thm}\label{thm_weight_general}
Let $W$ be a weight function over $\X \times \Y$, and let $\D$ be a distribution over $\X \times \Y$. Fix $r >0$. Let $S_r = \ap(S, r)$.  For convenience, relabel $x_i, y_i$ so that $S_r = \{(x_1, y_1), (x_2, y_2), \dots, (x_m, y_m)\}$. Suppose that for any $0 < a < b,$ 
\begin{equation*}\label{condition}
\lim_{n \to \infty} \mathbb{E}_{S \sim \D^n}\big [ \frac{1}{m} \sum_{i = 1}^m \sup_{x \in B(x_i, a)} \sum_{j = 1}^m w_j^{S_r}(x)I_{||x_j - x|| > b} \big] = 0. 
\end{equation*}
Then $\ga(W,r)$ is \rcons\emph{ }with respect to $\D$. 
\end{thm}

\paragraph{\textbf{Remark}:}
There are two important differences between the conditions in Theorem \ref{thm_weight_general} and Theorem~\ref{thm_stone_cons}.
\begin{enumerate}
	\item We replace $S$ with $S_r$.
	\item The expectation over $X \sim \D_\X$ is replaced with an average over $\{x_1, x_2, \dots, x_m\}$. The intuition here is that we are replacing $\D$ with a uniform distribution over $S_r$. While $\D$ may not be $r$-separated, the uniform distribution over $S_r$ is, and represents the region of points where our classifier is astute. 
\end{enumerate}

A natural question is what satisfies the conditions in Theorem~\ref{thm_weight_general}. We next show that $k_n$-nearest neighbors and kernel classifiers with rapidly decaying kernel functions continue to satisfy the conditions in Theorem \ref{thm_weight_general}; this means that these classifiers, when combined with Adversarial Pruning, will converge to $r$-optimal classifiers in the large sample limit.

\begin{cor}\label{thm_NN_gen}
Let $k_n$ be a sequence with $\lim_{n \to \infty} \frac{k_n}{n} = 0$, and let $M$ denote the $k_n$-nearest neighbor algorithm. Then for any $r > 0$, $\ga(M, r)$ is \rcons.
\end{cor}
\paragraph{\textbf{Remark}:} Corollary \ref{thm_NN_gen} gives a formal guarantee in the large sample limit for the modified nearest-neighbor algorithm proposed by \cite{YRWC19}.

\begin{cor}\label{thm_kern_gen}
Let $W$ be a kernel classifier over $\X \times \Y$ constructed from $K$ and $h_n$. Suppose the following properties hold for $K$ and $h_n$.
\begin{enumerate}
	\item For any $c > 1$, $\lim_{x \to \infty} \frac{K(cx)}{K(x)} = 0.$
	\item $\lim_{n \to \infty} h_n = 0.$
\end{enumerate}
Then for any $r > 0$, $\ga(W, r)$ is \rcons.
\end{cor}

Observe again that Condition 1. is satisfied by any $K$ that decreases more rapidly than an inverse polynomial kernel; it is thus satisfied by most popular kernels, such as the Gaussian kernel. 

\section{Validation}
%\begin{figure}[ht]
%\vskip 0.2in
%\begin{center}
%\subfloat[][Noiseless Histogram]{\includegraphics[width=.29\textwidth]{hist_noiseless_final}}\quad
%   \subfloat[][Noisy Histogram]{\includegraphics[width=.29\textwidth]{hist_noisy_final}} \quad
%   \subfloat[][Histogram trained on 500 samples]{\includegraphics[width=.29\textwidth]{visual500}}\\
%   \subfloat[][Noiseless 1-NN]{\includegraphics[width=.29\textwidth]{nn_noiseless_final}} \quad
%   \subfloat[][Noisy 1-NN]{\includegraphics[width=.29\textwidth]{nn_noisy_final}}\quad
%   \subfloat[][Histogram trained on 3000 samples]{\includegraphics[width=.29\textwidth]{visual3000}}
%\end{center}
%\caption{Empirical accuracy/astuteness of different classifiers as a function of training sample size. Accuracy is shown in green, astuteness in purple. Left : Noiseless Setting. Right: Noisy Setting. Top Row: Histogram Classifier, Bottom Row: 1-Nearest Neighbor}
%\label{fig:1}
%\vskip -0.2in
%\end{figure}

\begin{figure}
\begin{subfigure}{0.31\textwidth}
\includegraphics[width=\linewidth]{hist_noiseless_final}
%\caption{First subfigure} \label{fig:a}
\end{subfigure}\hspace*{\fill}
\begin{subfigure}{0.31\textwidth}
\includegraphics[width=\linewidth]{hist_noisy_final}
%\caption{Second subfigure} \label{fig:b}
\end{subfigure}\hspace*{\fill}
\begin{subfigure}{0.31\textwidth}
\includegraphics[width=\linewidth]{visual500}
%\caption{Fifth subfigure} \label{fig:e}
\end{subfigure}

\medskip
\begin{subfigure}{0.31\textwidth}
\includegraphics[width=\linewidth]{nn_noiseless_final}
%\caption{Third subfigure} \label{fig:c}
\end{subfigure}\hspace*{\fill}
\begin{subfigure}{0.31\textwidth}
\includegraphics[width=\linewidth]{nn_noisy_final}
%\caption{Fourth subfigure} \label{fig:d}
\end{subfigure}\hspace*{\fill}
\begin{subfigure}{0.31\textwidth}
 \includegraphics[width=\linewidth]{visual3000}
%\caption{Sixth subfigure} \label{fig:f}

\end{subfigure}

\caption{Empirical accuracy/astuteness of different classifiers as a function of training sample size. Accuracy is shown in green, astuteness in purple. Left : Noiseless Setting. Right: Noisy Setting. Top Row: Histogram Classifier, Bottom Row: 1-Nearest Neighbor} \label{fig:1}
\end{figure}

Our theoretical results are, by nature, large sample; we next validate how well they apply to the finite sample case by trying them out on a simple example. In particular, we ask the following question:

\begin{quote}
How does the robustness of non-parametric classifiers change with increasing sample size?
\end{quote}

This question is considered in the context of two simple non-parametric classifiers -- one nearest neighbor (which is guaranteed to be $r$-consistent) and histograms (which is not). To be able to measure performance with increasing data size, we look at a simple synthetic dataset -- the Half Moons. 

\subsection{Experimental Setup}

\paragraph{Classifiers and Dataset.} We consider two different classification algorithms -- one nearest neighbor (NN) and a Histogram Classifier (HC).  We use the Halfmoon dataset with two settings of the gaussian noise parameter $\sigma$, $\sigma = 0$ (Noiseless) and $\sigma =0.08$ (Noisy). For the Noiseless setting, observe that the data is already $0.1$-separated; for the Noisy setting, we use Adversarial Pruning (Algorithm~\ref{alg:gen}) with parameter $r = 0.1$ for both classification methods.

\paragraph{Performance Measure.} We evaluate robustness with respect to the $\ell_{\infty}$ metric, that is commonly used in the adversarial examples literature. Specifically, for each classifier, we calculate the {\em{empirical astuteness}}, which is the fraction of test examples on which it is astute.

Observe that computing the empirical astuteness of a classifier around an input $x$ amounts to finding the adversarial example that is {\em{closest to}} $x$ according to the $\ell_{\infty}$ norm. For the $1$-nearest neighbor, we do this using the optimal attack algorithm proposed by Yang et. al.~\cite{YRWC19}. For the histogram classifier, we use the optimal attack framework proposed by~\cite{YRWC19}, and show that the structure of the classifier can be exploited to solve the convex program efficiently. Details are in Appendix C.

We use an attack radius of $r = 0.1$ for the Noiseless setting, and $r = 0.09$ for the Noisy setting. For all classification algorithms, we plot the empirical astuteness as a function of the training set size. As a baseline, we also plot their standard accuracy on the test set. 

\subsection{Results}

The results are presented in Figure~\ref{fig:1}; the left two panels are for the Noiseless setting while the two center ones are for the Noisy setting.  

The results show that as predicted by our theory, for the Noiseless setting, the empirical astuteness of nearest neighbors converges to $1$ as the training set grows. For Histogram Classifiers, the astuteness converges to $0.5$ -- indicating that the classifier may grow less and less astute with higher sample size even for well-separated data. This is plausibly because the cell size induced by the histogram grows smaller with growing training data; thus, the classifier that outputs the default label $-1$ in empty cells is incorrect on adversarial examples that are close to a point with $+1$ label, but belongs to a different, empty cell. The rightmost panels in Figure~\ref{fig:1} provide a visual illustration of this process. 

For the Noisy setting, the empirical astuteness of adversarial pruning followed by nearest neighbors converges to $0.8$. For histograms with adversarial pruning, the astuteness converges to $0.7$, which is higher than the noiseless case but still clearly sub-optimal.

\subsection{Discussion}

Our results show that even though our theory is asymptotic, our predictions continue to be relevant in finite sample regimes. In particular, on well-separated data, nearest neighbors that we theoretically predict to be intrinsically robust is robust; histogram classifiers, which do not satisfy the conditions in Theorem~\ref{thm_stone_cons} are not. Our predictions continue to hold for data that is not well-separated. Nearest neighbors coupled with Adversarial
Pruning continues to be robust with growing sample size, while histograms continue to be non-robust. Thus our theory is confirmed by practice.

\section{Conclusion}

In conclusion, we rigorously analyze when non-parametric methods provide classifiers that are robust in the large sample limit. We provide a general condition that characterizes when non-parametric methods are robust on well-separated data, and show that Adversarial Pruning of~\cite{YRWC19} works on data that is not well-separated. 

Our results serve to provide a set of guidelines that can be used for designing non-parametric methods that are robust and accurate on well-separated data; additionally, we demonstrate that when data is not well-separated, preprocessing by adversarial pruning~\cite{YRWC19} does lead to optimally astute solutions in the large sample limit. 




%\newcommand{\crop}[1]{\mathrm{crop}({#1})}
\newcommand{\object}[1]{\mathrm{object}({#1})}
\newcommand{\ba}{A_i}
\newcommand{\bb}{B_i}
\newcommand{\calA}{\mathcal{A}}
\newcommand{\calB}{\mathcal{B}}
\newcommand{\calX}{\mathcal{X}}
\newcommand{\masked}[1]{\mathrm{masked}({#1})}
\newcommand{\bx}{\mathbf{x}}
\newcommand{\SSL}{\textsc{SSL}}
\newcommand{\SSLbb}{\SSL^\mathrm{back}}
\newcommand{\SSLpj}{\SSL^\mathrm{proj}}
\newcommand{\CLF}{\textsc{CLF}}
\newcommand{\CLFbb}{\CLF^\mathrm{back}}
\newcommand{\CLFpj}{\CLF^\mathrm{proj}}
\newcommand{\SUP}{\textsc{SUP}}
\newcommand{\KNN}{\textsc{KNN}}
\newcommand{\KNNset}{\textsc{KNN}^\mathrm{set}}
\newcommand{\KNNprob}{\textsc{KNN}^\mathrm{prob}}
\newcommand{\KNNcl}{\textsc{KNN}^\mathrm{cl}}
\newcommand{\KNNconf}{\textsc{KNN}^\mathrm{conf}}
\newcommand{\RCDM}{\textsc{RCDM}}
\newcommand{\cl}{\mathrm{cl}}
\newcommand{\clpred}{\tilde{\mathrm{cl}}}
\newcommand{\Abox}{\overline{\calA}}
\newcommand{\Bbox}{\overline{\calB}}
\newcommand{\dejavu}{\emph{déjà vu }}
\newcommand{\Dejavu}{\emph{Déjà vu }}

\newcommand{\citations}{{\color{green}[CITE]}}

\definecolor{part_blue}{rgb}{0.2824, 0.4706, .8157}
\definecolor{part_red}{rgb}{0.8392, 0.3725, 0.3725}
\definecolor{part_orange}{rgb}{0.9333, 0.5216, 0.2902}

\DeclareRobustCommand{\mybox}[2][gray!20]{%
\begin{tcolorbox}[   %% Adjust the following parameters at will.
        % breakable,
        left=0pt,
        right=0pt,
        top=0pt,
        bottom=0pt,
        colback=#1,
        colframe=#1,
        width=\dimexpr\columnwidth\relax, 
        % width=\textwidth, 
        enlarge left by=0mm,
        boxsep=5pt,
        arc=0pt,outer arc=0pt,
        ]
        #2
\end{tcolorbox}
}
%\section{Introduction}
\label{sec:intro}
Self-supervised learning (SSL)~\citep{chen2020simclr, chen2020simsiam, zbontar2021barlow, vicreg, caron2020swav, MAE} aims to learn general representations of content-rich data without explicit labels by solving a \textit{pretext task}. In many recent works, such pretext tasks rely on joint-embedding architectures whereby randomized image augmentations are applied to create multiple views of a training sample, and the model is trained to produce similar representations for those views. When using cropping as random image augmentation, the model learns to associate objects or parts (including the background scenery) that co-occur in an image.
However, doing so also arguably exposes the training data to higher privacy risk as objects in training images can be explicitly memorized by the SSL model. For example, if the training data contains the photos of individuals, the SSL model may learn to associate the face of a person with their activity or physical location in the photo. This may allow an adversary to extract such information from the trained model for targeted individuals.

\begin{figure}[t]
    \centering
    \includegraphics[width=1.0\columnwidth]{figures/new_black_swan.pdf}
    \caption{\textbf{Left:} Reconstruction of an SSL training image from a crop containing only the background. The SSL model memorizes the association of this \emph{specific} patch of water (pink square) to this \emph{specific} foreground object (a black swan) in its embedding, which we decode to visualize the full training image. \textbf{Right:} The reconstruction technique fails on a public test image that the SSL model has not seen before.}
    \label{fig:black_swan}
\end{figure}

In this work, we aim to evaluate to what extent SSL models memorize the association of specific objects in training images or the association of objects and their specific backgrounds, and whether this memorization signal can be used to reconstruct the model's training samples. Our results demonstrate that SSL models memorize such associations beyond simple correlation. For instance, in Figure \ref{fig:black_swan} (\textbf{left}), we use the SSL representation of a \emph{training image crop containing only water} and this enables us to reconstruct the object in the foreground with remarkable specificity---in this case a black swan.
By contrast, in Figure \ref{fig:black_swan} (\textbf{right}), when using the \emph{crop from the background of a test set image} that the SSL model \emph{has not seen before}, its representation only contains enough information to infer, through correlation, that the foreground object was likely some kind of waterbird --- but not the specific one in the image.

Figure \ref{fig:black_swan} shows that SSL models suffer from the unintended memorization of images in their training data---a phenomenon we refer to as \emph{déjà vu memorization}
%\footnote{The French loanword \emph{déjà vu} means already-seen, which reflects the type of unintended memorization of objects that the SSL model saw during training.}.
\footnote{The French loanword \emph{déjà vu} means `already-seen', just as an image is seen and memorized in training.}
Beyond visualizing \emph{déjà vu} memorization through data reconstruction, we also design a series of experiments to quantify the degree of memorization for different SSL algorithms, model architectures, training set size, \emph{etc.} We observe that \emph{déjà vu} memorization is exacerbated by the atypically large number of training epochs often recommended in SSL training, as well as certain hyperparameters in the SSL training objective. Perhaps surprisingly, we show that \emph{déjà vu} memorization occurs even when the training set is large---as large as half of ImageNet~\citep{imagenet}---and can continually worsen even when standard techniques for evaluating learned representation quality (such as linear probing) do not suggest increased overfitting. Our work serves as the first systematic study of unintended memorization in SSL models and motivates future work on understanding and preventing this behavior. Specifically, we: 
\begin{itemize}
    \vspace{-0.5em}
    \item Elucidate how SSL representations memorize aspects of individual training images, what we call \emph{déjà vu} memorization;
    \item Design a novel training data reconstruction pipeline for non-generative vision models. This is in contrast to many prominent reconstruction algorithms like \citep{carlini2021extracting, google_diffusion}, which rely on the model itself to generate its own memorized samples and is not possible for SSL models or classifiers;
    \item Propose metrics to quantify the degree of \dejavu memorization committed by an SSL model. This allows us to observe how \dejavu changes with training epochs, dataset size, training criteria, model architecture and more. 
\end{itemize}

%\section{Preliminaries and Related Work}
\label{sec:related}

\textbf{Self-supervised learning} (SSL) is a machine learning paradigm that leverages unlabeled data to learn representations. Many SSL algorithms rely on \emph{joint-embedding} architectures (\emph{e.g.}, SimCLR~\citep{chen2020simclr}, Barlow Twins~\citep{zbontar2021barlow}, VICReg~\citep{vicreg} and Dino~\citep{Dino}), which are trained to associate different augmented views of a given image. For example, in SimCLR, given a set of images $\calA = \{A_1,\ldots,A_n\}$ and a randomized augmentation function $\mathrm{aug}$, the model is trained to maximize the cosine similarity of draws of $\SSL(\mathrm{aug}(A_i))$ with each other and minimize their similarity with $\SSL(\mathrm{aug}(A_j))$ for $i \neq j$. The augmentation function $\mathrm{aug}$ typically consists of operations such as cropping, horizontal flipping, and color transformations to create different views that preserve an image's semantic properties. 

\paragraph{SSL representations.} Once an SSL model is trained, its learned representation can be transferred to different downstream tasks. This is often done by extracting the representation of an image from the \emph{backbone model}\footnote{SSL methods often use a trick called \emph{guillotine regularization}~\citep{Guillotine}, which decomposes the model into two parts: a \emph{backbone model} and a \emph{projector} consisting of a few fully-connected layers. Such trick is needed to handle the misalignment between the pretext SSL task and the downstream task.} and either training a linear probe on top of this representation or finetuning the backbone model with a task-specific head~\citep{Guillotine}.
%Compared to representations learned by supervised learning, SSL representations are often more robust and transferable~\citep{hendrycks2019using, ericsson2021self}, leading to state-of-the-art result on many downstream tasks. To understand the effectiveness of SSL algorithms, several prior works investigated what kind of information the SSL model has learned~\citep{jing2021understanding, ericsson2021self, kalibhat2022towards, RCDM}. In particular, \citet{RCDM} trained a conditional generative model on SSL representations and showed that they encode richer visual details about the input image compared to supervised learning. 
%However, from a privacy perspective, this may be a cause for concern as the model also has more potential to overfit and memorize precise details about the training data compared to supervised learning. We show concretely that this privacy risk can indeed be realized by defining and measuring \emph{déjà vu} memorization.
It has been shown that SSL representations encode richer visual details about input images than supervised models do \cite{RCDM}. However, from a privacy perspective, this may be a cause for concern as the model also has more potential to overfit and memorize precise details about the training data compared to supervised learning. We show concretely that this privacy risk can indeed be realized by defining and measuring \emph{déjà vu} memorization.
\vspace{-0.5em} 
% \paragraph{Privacy risks in ML.} Overfitting in ML occurs when a model memorizes information specific to its training data rather than general population-level information. When the model is trained on privacy-sensitive data, overfitting is especially harmful as an adversary can infer private information about the training data when given access to the model~\citep{yeom2018privacy, feldman2020does}. The simplest and most well-studied form of privacy risk in ML is susceptibility to \emph{membership inference attacks}~\citep{shokri2017membership, salem2018ml, sablayrolles2019white}, where the adversary infers whether an individual is part of the training set or not. More sophisticated privacy attacks include \emph{attribute inference}~\citep{fredrikson2014privacy, mehnaz2022your, jayaraman2022attribute}, where specific attributes about an individual are inferred given others, and \emph{data reconstruction}~\citep{carlini2021extracting, balle2022reconstructing, guo2022bounding}, where entire training samples are recovered from the trained model. Our study of \emph{déjà vu} memorization is similar to both attribute inference and data reconstruction, leveraging SSL representations of the training image background to infer and reconstruct the foreground object.
% \vspace{-0.5em} 
% \paragraph{Training data extraction in NLP.} Our study of \dejavu memorization in SSL models is inspired by similar work in the natural language processing (NLP) domain. \citet{carlini2019secret} first showed that language models exhibit unintended memorization, where given a context string present in its training data, the model can generate the remaining text at test time. This unintended memorization has been further exploited in \citet{carlini2021extracting} to extract training data from GPT-2~\citep{radford2019language} and, more recently, extended to extract memorized images from Stable Diffusion \citep{google_diffusion}. The way by which these works exploit unintended memorization is similar to ours: given partial information about a training sample, the model is prompted to reveal the rest of the sample. In our case, however, since the SSL model is not generative, extraction is significantly harder and requires careful design.

\paragraph{Privacy risks in ML.} When a model is overfit on privacy-sensitive data, it memorizes specific information about its training examples, allowing an adversary with access to the model to learn private information~\citep{yeom2018privacy, feldman2020does}. Privacy attacks in ML range from the simplest and best-studied \emph{membership inference attacks}~\citep{shokri2017membership, salem2018ml, sablayrolles2019white} to \emph{attribute inference}~\citep{fredrikson2014privacy, mehnaz2022your, jayaraman2022attribute} and \emph{data reconstruction}~\citep{carlini2021extracting, balle2022reconstructing, guo2022bounding} attacks. In the former, the adversary only infers whether an individual participated in the training set. Our study of \emph{déjà vu} memorization is most similar to the latter: we leverage SSL representations of the training image background to infer and reconstruct the foreground object. Our approach reflects similar work in the NLP domain \citep{carlini2019secret, carlini2021extracting}: when prompted with a context string present in the training data, a large language model is shown to generate the remainder of string at test time, revealing sensitive text like home addresses. This method was recently extended to extract memorized images from Stable Diffusion \citep{google_diffusion}.  We exploit memorization in a similar manner: given partial information about a training sample, the model is prompted to reveal the rest of the sample. In our case, however, since the SSL model is not generative, extraction is significantly harder and requires careful design.

%\section{Defining \emph{Déjà Vu} Memorization}
\label{sec:definition}

\paragraph{What is \dejavu memorization?} At a high level, the objective of SSL is to learn general representations of objects that occur in nature. This is often accomplished by associating different parts of an image with one another in the learned embedding. Returning to our example in Figure \ref{fig:black_swan}, given an image whose background contains a patch of water, the model may learn that the foreground object is a water animal such as duck, pelican, otter, \emph{etc.}, by observing different images that contain water from the training set. We refer to this type of learning as \emph{correlation}: the association of objects that tend to co-occur in images from the training data distribution.

A natural question to ask is \emph{``Can the reconstruction of the black swan in Figure \ref{fig:black_swan} be reasoned as correlation?''} The intuitive answer may be no, since the reconstructed image is qualitatively very similar to the original image. However, this reasoning implicitly assumes that for a random image from the training data distribution containing a patch of water, the foreground object is unlikely to be a black swan. Mathematically, if we denote by $\mathcal{P}$ the training data distribution and $A$ the image, then
\begin{equation*}
\label{eq:p_corr}
p_\text{corr} := \mathbb{P}_{A \sim \mathcal{P}}(\mathrm{object}(A) = \texttt{black swan} ~|~ \mathrm{crop}(A) = \texttt{water})
\end{equation*}
is the probability of inferring that the foreground object is a black swan through \emph{correlation}. This probability may be naturally high due to biases in the distribution $\mathcal{P}$, \emph{e.g.}, if $\mathcal{P}$ contains no other water animal except for black swans. In fact, such correlations are often exploited to learn a model for image inpainting with great success~\citep{yu2018generative, ulyanov2018deep}.

Despite this, we argue that reconstruction of the black swan in Figure \ref{fig:black_swan} is \emph{not} due to correlation, but rather due to \emph{unintended memorization}: the association of objects unique to a single training image. As we will show in the following sections, the example in Figure \ref{fig:black_swan} is not a rare success case and can be replicated across many training samples. More importantly, failure to reconstruct the foreground object in Figure \ref{fig:black_swan} (\textbf{right}) on test images hints at inferring through correlation is unlikely to succeed---a fact that we verify quantitatively in Section \ref{sec:label inference accuracy}. Motivated by this discussion, we give a verbal definition of \dejavu memorization below, and design a testing methodology to quantify \dejavu memorization in Section \ref{sec:notation and setup}.
\mybox{\textbf{Definition:} A model exhibits \emph{déjà vu memorization} when it retains information so specific to an individual training image, that it enables recovery of aspects particular to that image given a part that does not contain them.
The recovered aspect must be beyond what can be inferred using only correlations in the data distribution.} 

% \textbf{Definition:} A model exhibits \emph{déjà vu memorization} when it retains information so specific to an individual training image, that it enables recovery of aspects particular to that image given a part that does not contain them.
% The recovered aspect must be beyond what can be inferred using only correlations in the data distribution.


 We intentionally kept the above definition broad enough to encompass different types of information that can be inferred about the training image, including but not restricted to object category, shape, color and position. For example, if one can infer that the foreground object is red given the background patch with accuracy significantly beyond correlation, we consider this an instance of \dejavu memorization as well. We mainly focus on object category to quantify \dejavu memorization in Section \ref{sec:quant} since the ground truth label can be easily obtained. We consider other types of information more qualitatively in the visual reconstruction experiments in Section \ref{sec:visualizing}.

\paragraph{Privacy implications of \dejavu memorization.} \Dejavu memorization can be a cause for concern when the training data contains privacy-sensitive information. As a motivating example, consider an SSL model trained on photos of individuals. If the model exhibits \dejavu memorization then, given the face of an individual, it may be possible to infer where the individual was or even visually reconstruct their location in the training image. Such information leakage raises privacy concerns, especially if there was no prior agreement that the trained model may reveal such information to third parties. This hypothetical scenario serves as a motivation that \dejavu memorization should be carefully examined to avoid unintended disclosure of private information in practical applications.

% \begin{figure*}[h]
%     \centering
%     \includegraphics[width = 0.85\textwidth]{figures/SSL_attack_cartoon.png}
%     \caption{We measure memorization by comparing the `target model' trained on the target image ($\SSL_A$ trained on $A_i$ in above example) with the `reference model' not trained on it ($\SSL_B$, above). \textbf{[Top Strip]} A cropping of the image disjoint from the labeled foreground object is embedded using the target model. This embedding is then labeled by a K-Nearest Neighbor (KNN) adversary built on a public set of labeled images, $X$, which it has also embedded using the target model. \textbf{[Bottom Strip]} To account for correlation, the same procedure is followed with the reference model. If the label is only extracted using the target model, it is counted as memorization. If it is extracted using either model, it is counted as correlation. We find that the KNN adversary's predictions using the target model (trained on attacked examples) are significantly more accurate than they are using the reference model, indicating routine memorization of training examples.}
%     \label{fig:ssl attack cartoon}
% \end{figure*}

\begin{figure}[t]
%%%
%SPIDER
%%%
     % \centering
     % \begin{subfigure}[b]{0.25\textwidth}
     %     \centering
     %     \includegraphics[width=\textwidth]{figures/data_split.png}
     %     % \caption{SimCLR correlated \textit{yellow garden spider} examples}
     %     \label{fig:data split}
     % \end{subfigure}
     % \hfill
     % \begin{subfigure}[b]{0.7\textwidth}
     %     \centering
     %     \includegraphics[width=\textwidth]{figures/pipeline_cartoon.png}
     %     \begin{minipage}{5cm}
     %        \vfill
     %    \end{minipage}
     %     % \caption{SimCLR memorized \textit{yellow garden spider} examples}
     %     \label{fig:pipeline cartoon}
     % \end{subfigure}
     \includegraphics[width=\textwidth]{figures/split_and_pipeline_cartoon.png}
\caption[Overview of testing methodology.]{
Overview of testing methodology. \textbf{Left:} Data is split into \emph{target set} $\calA$, \emph{reference set} $\calB$ and \emph{public set} $\calX$ that are pairwise disjoint. $\calA$ and $\calB$ are used to train two SSL models $\SSL_A$ and $\SSL_B$ in the same manner. $\calX$ is used for KNN decoding or for training an RCDM to reconstruct the input at test time. \textbf{Right:} Given a training image $A_i \in \calA$, we use $\SSL_A$ to embed $\crop{A_i}$ containing only the background, as well as the entire set $\calX$ and find the $k$-nearest neighbors of $\crop{A_i}$ in $\calX$ in the embedding space. These KNN samples can be used directly to infer the foreground object (\emph{i.e.}, class label) in $A_i$ using a KNN classifier, or their embeddings can be averaged as input to the trained RCDM to visually reconstruct the image $A_i$. For instance, the RCDM reconstruction results in Figure \ref{fig:black_swan} (left) when given $\SSL_A(\crop{A_i})$ and results in Figure \ref{fig:black_swan} (right) when given $\SSL_A(\crop{B_i})$ for an image $B_i \in \calB$.
%\textbf{Left:} illustration of the three datasets used in our tests. Two private data sets, $A$ and $B$, of equal size are used to train two SSL models, $\SSL_A$ and $\SSL_B$, respectively. A disjoint public set, $X$, is made available to the memorization test to help decode model embeddings. Memorization is only tested on examples $A_i \in A$ that are unique to set $A$. \textbf{Right:} illustration of inference pipeline used in tests. A periphery cropping that excludes the foreground object is taken from private image $A_i$. The KNN then finds the $k$ public set nearest neighbors of the periphery crop in the embedding space of $\SSL_A$. 
%The $\SSL_A$ representation of these $k$ neighbors and of the crop are used by the conditional generative model, RCDM, to reconstruct the foreground object. The labels of these $k$ neighbors are used to recover the foreground object label. (Not pictured) We repeat this process using reference model $\SSL_B$, not trained on image $A_i$, to determine whether the foreground object is still recoverable by learned correlations, e.g. if black swans were the only objects appearing near water in the data distribution. In this instance, the crop's public set neighbors in $\SSL_B$'s representation space include a variety of water animals like ducks, pelicans, and otters. Meanwhile, with $\SSL_A$, the neighbors are nearly all black swans in the same position as the swan of $A_i$.
}
\label{fig:split_and_pipeline_cartoon}
\end{figure}

\textbf{Distinguishing memorization from correlation.} When measuring \dejavu memorization, it is crucial to differentiate what the model associates through \emph{memorization} and what it associates through \emph{correlation}. Our testing methodology is based on the following intuitive definition.
\mybox{\textbf{Definition:} If an SSL model associates two parts in a training image, we say that it is due to \emph{correlation} if other SSL models trained on a similar dataset from $\mathcal{P}$ without this image would likely make the same association. Otherwise, we say that it is due to \emph{memorization}.}

Notably, such intuition forms the basis for differential privacy (DP; \cite{dwork2006calibrating, dwork2013algorithmic})---the most widely accepted notion of privacy in ML.

\subsection{Testing Methodology for Measuring \emph{Déjà Vu} Memorization}
\label{sec:notation and setup}

In this section, we use the above intuition to measure the extent of \dejavu memorization in SSL. Figure \ref{fig:split_and_pipeline_cartoon} gives an overview of our testing methodology.
\vspace{-0.75em}
\paragraph{Dataset splitting.} We focus on testing \dejavu memorization for SSL models trained on the ImageNet-1K dataset~\citep{imagenet}. Our test first splits the ImageNet training set into three independent and disjoint subsets $\calA$, $\calB$ and $\calX$. The dataset $\calA$ is called the \emph{target set} and $\calB$ is called the \emph{reference set}. The two datasets are used to train two separate SSL models, $\SSL_A$ and $\SSL_B$, called the \emph{target model} and the \emph{reference model}. Finally, the dataset set $\calX$ is used as an auxiliary public dataset to extract information from $\SSL_A$ and $\SSL_B$.
%\footnote{See Appendix \ref{sec:appx splits} for details on how the dataset splits are generated.}.
Our dataset splitting serves the purpose of distinguishing memorization from correlation in the following manner. Given a sample $A_i \in \calA$, if our test returns the same result on $\SSL_A$ and $\SSL_B$ then it is likely due to correlation because $A_i$ is not a training sample for $\SSL_B$. Otherwise, because $\calA$ and $\calB$ are drawn from the same underlying distribution, our test must have inferred some information unique to $A_i$ due to memorization. Thus, by comparing the difference in the test results for $\SSL_A$ and $\SSL_B$, we can measure the degree of \dejavu memorization\footnote{See Appendix \ref{sec:appx splits} for details on how the dataset splits are generated.}.
\vspace{-0.75em}
\paragraph{Extracting foreground and background crops.} Our testing methodology aims at measuring what can be inferred about the foreground object in an ImageNet sample given a background crop. This is made possible because ImageNet provides bounding box annotations for a subset of its training images---around 150K out of 1.3M samples. We split these annotated images equally between $\calA$ and $\calB$. Given an annotated image $A_i$, we treat everything inside the bounding box as the foreground object associated with the image label, denoted $\object{A_i}$. We take the largest possible crop that does not intersect with any bounding box as the background crop (or \emph{periphery crop}), denoted $\crop{A_i}$\footnote{We also present another heuristic in \cref{sec:appx corner crop} which takes a corner crop as the background crop, allowing our test to be run without bounding box annotations.}
%Since the labeled object tends to be at the image's center, the corner crop usually excludes it. }
%Because most images in ImageNet are object centric, an image's corner would not include the foreground object.}.
\vspace{-0.75em}
\paragraph{KNN-based test design.} Joint-embedding SSL approaches encourage the embeddings of random crops of a training image $A_i \in \calA$ to be similar. Intuitively, if the model exhibits \dejavu memorization, it is reasonable to expect that the embedding of $\crop{A_i}$ is similar to that of $\object{A_i}$ since both crops are from the same training image. In other words, $\SSL_A(\crop{A_i})$ encodes information about $\object{A_i}$ that cannot be inferred through correlation. However, decoding such information is challenging as these approaches do not learn a decoder associated with the encoder $\SSL_A$.

Here, we leverage the public set $\calX$ to decode the information contained in $\crop{A_i}$ about $\object{A_i}$. More specifically, we map images in $\calX$ to their embeddings using $\SSL_A$ and extract the $k$-nearest-neighbor (KNN) subset of $\SSL_A(\crop{A_i})$ in $\calX$. We can then decode the information contained in $\crop{A_i}$ in one of two ways:
\begin{itemize}
\item \emph{Label inference:} Since $\calX$ is a subset of ImageNet, each embedding in the KNN subset is associated with a class label. If $\crop{A_i}$ encodes information about the foreground object, its embedding will be close to samples in $\calX$ that have the same class label (\emph{i.e.}, foreground object category). We can then use a KNN classifier to infer the foreground object in $A_i$ given $\crop{A_i}$.
\item \emph{Visual reconstruction:} Following \citet{RCDM}, we train an RCDM---a conditional generative model---on $\calX$ to decode $\SSL_A$ embeddings into images. The RCDM reconstruction can recover qualitative aspects of an image remarkably well, such as recovering object color or spatial orientation using its SSL embedding. Given the KNN subset, we average their SSL embeddings and use the trained RCDM model to visually reconstruct $A_i$.
\end{itemize}
In Section \ref{sec:quant}, we focus on quantitatively measuring \dejavu memorization with label inference, and then use the RCDM reconstruction to visualize \dejavu memorization in Section \ref{sec:visualizing}.
%\section{Quantifying \emph{Déjà Vu} Memorization}
\label{sec:quant}

We apply our testing methodology to quantify a specific form of \dejavu memorization: inferring the foreground object (class label) given a crop of the background.

% \paragraph{Extracting model embeddings.} We test \dejavu memorization on two popular SSL algorithms, SimCLR~\citep{chen2020simclr} and VICReg~\citep{vicreg}.
% %\footnote{We present additional SSL models in \cref{sec:appx simclr results}} 
% As described in Section \ref{sec:related}, these algorithms produce two embeddings given an input image: a \emph{backbone} embedding and a \emph{projector} embedding that is derived by applying a small fully-connected network on top of the backbone embedding. Unless otherwise noted, all SSL embeddings refer to the projector embedding.
% To understand whether \dejavu memorization is particular to SSL, we also evaluate embeddings produced by a supervised model $\CLF_A$ trained on $\calA$. We apply the same set of image augmentations as those used in SSL and train $\CLF_A$ using the cross-entropy loss to predict ground truth labels. 
\vspace{-0.75em}
\paragraph{Extracting model embeddings.} We test \dejavu memorization on a variety of popular SSL algorithms, with a focus on VICReg~\citep{vicreg}. These algorithms produce two embeddings given an input image: a \emph{backbone} embedding and a \emph{projector} embedding that is derived by applying a small fully-connected network on top of the backbone embedding. Unless otherwise noted, all SSL embeddings refer to the projector embedding. 
To understand whether \dejavu memorization is particular to SSL, we also evaluate embeddings produced by a supervised model $\CLF_A$ trained on $\calA$. We apply the same set of image augmentations as those used in SSL and train $\CLF_A$ using the cross-entropy loss to predict ground truth labels. 
\vspace{-0.75em}
\paragraph{Identifying the most memorized samples.} Prior works have shown that certain training samples can be identified as more prone to memorization than others~\citep{feldman2020does, watson2021importance, ye2021enhanced}. Similarly, we provide a heuristic to identify the most memorized samples in our label inference test using confidence of the KNN prediction.
Given a periphery crop, $\crop{A_i}$, let $\KNN_A \big( \crop{A_i} \big) \subseteq \calX$ denote its $k$-nearest neighbors in the embedding space of $\SSL_A$. From this KNN subset we can obtain: \textbf{(1)} $\KNNprob_A \big( \crop{A_i} \big)$, the vector of class probabilities (normalized counts) induced by the KNN subset, and \textbf{(2)} $\KNNconf_A \big( \crop{A_i} \big)$, the negative entropy of the probability vector $\KNNprob_A \big( \crop{A_i} \big)$, as confidence of the KNN prediction. When entropy is low, the neighbors agree on the class of $A_i$ and hence confidence is high. 
% \begin{itemize}[noitemsep, leftmargin=*, topsep=0pt]
%     \item $\KNN_A \big( \crop{A_i} \big)$: The most prevalent class in the KNN subset as prediction for the class label $\cl(A_i)$. 
%     \item $\KNNprob_A \big( \crop{A_i} \big)$: The vector of class probabilities (normalized counts) induced by the KNN subset.
%     \item $\KNNconf_A \big( \crop{A_i} \big)$: Negative entropy of the probability vector $\KNNprob_A \big( \crop{A_i} \big)$ as confidence of the KNN prediction. When entropy is low, the neighbors agree on the class of $A_i$ and hence confidence is high. 
% \end{itemize}
We can sort the confidence score $\KNNconf_A \big( \crop{A_i} \big)$ across samples $A_i$ in decreasing order to identify the most confidently predicted samples, which likely correspond to the most memorized samples when $A_i \in \calA$.

\subsection{Population-level Memorization}
\label{sec:label inference accuracy}

%ORIGINAL FIGURE SETUP IN ARXIV: 
% \input{dejavu_training_epochs.tex}
% \input{dejavu_training_set_size.tex}
%PUT ORIGINAL FIGURES SIDE BY SIDE: 
% \input{dejavu_training_epochs_set_size.tex}
%PUT IN NEW FIGURES: 

\begin{wrapfigure}{r}{0.4\textwidth} 
    \centering
    \includegraphics[width=0.4\textwidth]{figures/dejavu_main.pdf}
    \caption{Accuracy of label inference using the target model (trained on $\calA$) vs. the reference model (trained on $\calB$) on the top $\%$ most confident examples $A_i \in \calA$ using only $\crop{A_i}$. For VICReg, there is a large accuracy gap between the two models, indicating a significant degree of \dejavu memorization.}
    \label{fig:dejavu main}
    \vspace{-2ex}
\end{wrapfigure}

Our first measure of \dejavu memorization is population-level label inference accuracy: \emph{What is the average label inference accuracy over a subset of SSL training images given their periphery crops?} 
To understand how much of this accuracy is due to $\SSL_A$'s \dejavu memorization, we compare with a correlation baseline using the reference model: $\KNN_B$'s label inference accuracy on images $A_i \in \calA$. 
In principle, this inference accuracy should be significantly above chance level ($1/1000$ for ImageNet) because the periphery crop may be highly indicative of the foreground object through correlation, \emph{e.g.}, if the periphery crop is a basketball player then the foreground object is likely a basketball.

Figure \ref{fig:dejavu main} compares the accuracy of $\KNN_A$ to that of $\KNN_B$ when inferring the labels of images in $A_i \in \calA$\footnote{The sets $\calA$ and $\calB$ are exchangeable, and in practice we repeat this test on images from $\calB$ using $\SSL_B$ as the target model and $\SSL_A$ as the reference model, and average the two sets of results.} using $\crop{A_i}$.
Results are shown for VICReg and the supervised model; trends for other models are shown in Appendix \ref{sec:appx simclr results}. For both VICReg and supervised models, inferring the class of $\crop{A_i}$ using $\KNN_B$ (dashed line) through correlation achieves a reasonable accuracy that is significantly above chance level. However, for VICReg, the inference accuracy using $\KNN_A$ (solid red line) is significantly higher, and the accuracy gap between $\KNN_A$ and $\KNN_B$ indicates the degree of \dejavu memorization. We highlight two observations: 
\begin{itemize}
    \item The accuracy gap of VICReg is significantly larger than that of the supervised model. This is especially notable when accounting for the fact that the supervised model is trained to associate randomly augmented crops of images with their ground truth labels. In contrast, VICReg has no label access during training but the embedding of a periphery crop can still encode the image label. 
    \item For VICReg, inference accuracy on the $1\%$ most confident examples is nearly $95\%$, which shows that our simple confidence heuristic can effectively identify the most memorized samples. This result suggests that an adversary can use this heuristic to identify vulnerable training samples to launch a more focused privacy attack.
\end{itemize}
\vspace{-.75em}
\paragraph{The \dejavu score. }
The curves of Figure \ref{fig:dejavu main} show memorization across confidence values for a single training scenario.  To study how memorization changes with different hyperparamters, we extract a single value from these curves: the \dejavu \emph{score} at confidence level $p$. In Figure \ref{fig:dejavu main}, this is the gap between the solid red (or gray) and dashed red (or gray) where confidence ($x$-axis) equal $p\%$. In other words, given the periphery crops of set $\calA$, $\KNN_A$ and $\KNN_B$ separately select and label their top $p\%$ most confident examples, and we report the difference in their accuracy. The \dejavu score captures both the degree of memorization by the accuracy gap and the \emph{ability to identify memorized examples} by the confidence level. If the score is 10\% for $p=33\%$, $\KNN_A$ has 10\% higher accuracy on its most confident third of $\calA$ than $\KNN_B$ does on its most confident third. In the following, we set $p = 20\%$, approximately the largest gap for VICReg (red lines) in Figure \ref{fig:dejavu main}. 
% Specifically, the \dejavu \emph{score} on the top $p\%$ most confident examples is,  
% \begin{equation}
%     \mathrm{DejaVu}(p) = \mathrm{Acc}_{\SSL_A}\big( \calA_{\SSL_A, p}  \big) - \mathrm{Acc}_{\SSL_B}\big( \calA_{\SSL_B, p}  \big) \ ,
%     \label{eqn:dejavu score}
% \end{equation}
% where $\calA_{\SSL_A, p}$
% Here we introduce a DejaVu memorization metric that quantify how much a target model is able to retrieve more class information from a crop than the reference model. We define it as:
% where $p$ is a function that take the $p$ purcent most confident samples.
%Figure \ref{fig:dejavu v. training epochs} shows how \dejavu memorization changes with the number of epochs used to train the embedding model (VICReg and supervised, respectively). The training set size is fixed to 300K samples, and label inference accuracy is computed on the top $20\%$ highest confidence examples. The number of epochs has a very strong influence on the degree of memorization for VICReg as the accuracy gap widens when number of epochs increases. We note that 1000 training epochs is used in several SSL works \citep{vicreg, simclr}. Remarkably, this trend in memorization is \emph{not} reflected in the standard metric for evaluating SSL representations: linear probe accuracy. The gray line in Figure \ref{fig:dejavu v. training epochs} shows the train-test accuracy gap of a linear classifier trained on top of the VICReg embeddings. Although there is a sizeable train-test gap, it does not grow significantly beyond 500 epochs. In contrast, \dejavu memorization (blue line) continues to worsen after 500 epochs. Thus, our test can be used as an alternative to linear probe accuracy to evaluate the memorization of SSL models.
% \vspace{-.75em}

% \paragraph{Comparison with the generalization gap} A network that perform very well on a training set while performing poorly on a test set (assuming the training set and test set sampled uniformly from the same distribution) is probably memorizing the training examples without being able to generalize on the test data. One could expect that measuring the difference in accuracy between the training and test set could give us insights on the degree of \dejavu memorization. However, we show in Figure  \ref{fig:dejavu v. training epochs} and \ref{fig:dejavu v. n} that this is not the case. In fact \dejavu memorization can significantly increase while the train-test gap decrease. In our experiments, we did not find a correlation between \dejavu and generalization.
\vspace{-0.75em}
\paragraph{Comparison with the linear probe train-test gap.} A standard method for measuring SSL performance is to train a linear classifier---what we call a `linear probe'---on its embeddings and compute its performance on a held out test set. From a learning theory standpoint, one might expect the linear probe's train-test accuracy gap to be indicative of memorization: the more a model overfits, the larger is the difference between train set and test set accuracy. However, as seen in Figure \ref{fig:dejavu epochs train set size}, the linear probe gap (dark blue) fails to reveal memorization captured by the \dejavu score (red) \footnote{See section \ref{sec:mitigation} for further discussion of the \dejavu score trends of Figure \ref{fig:dejavu epochs train set size}.}.

% \paragraph{Effect of training epochs.} 
% Figure \ref{fig:dejavu v. training epochs} shows how \dejavu memorization changes with training epochs for VICReg. The training set size is fixed to 300K samples. We observe that the number of epochs has a very strong influence on the degree of memorization for VICReg. From 250 to 1000 epochs, the \dejavu score (red curve) grows threefold: from under 10\% to over 30\%. Remarkably, this trend in memorization is \emph{not} reflected in the standard metric for evaluating SSL representations: linear probe accuracy. The dark blue curve shows the train-test linear probe accuracy gap. Although there is a sizeable train-test gap, it only changes by a few percent beyond 250 epochs. %Thus, our test can be used as an alternative to linear probe accuracy to evaluate the memorization of SSL models.
% \vspace{-.75em}
\begin{figure}[ht]
\label{fig:dejavu epochs and dataset}
\begin{minipage}[t]{0.49\textwidth}
\centering
     \begin{subfigure}[b]{0.48\textwidth}
         \centering
         \includegraphics[width=\textwidth]{figures/deja_vu_vs_epochs.png}
         \vspace{-1.5em}
         \caption{\dejavu vs. epochs}
         \label{fig:dejavu v. training epochs}
     \end{subfigure}
     \begin{subfigure}[b]{0.48\textwidth}
         \centering
         \includegraphics[width=\textwidth]{figures/deja_vu_vs_n.png}
         \vspace{-1.5em}
         \caption{\dejavu vs. train set size}
         \label{fig:dejavu v. n}
     \end{subfigure}~
     \vspace{-0.5em}
    \caption{
    Effect of training epochs and train set size with VICReg on \dejavu score (red) in comparison with linear probe accuracy train-test gap (dark blue). 
    \textbf{Left:} \dejavu score increases with training epochs, indicating growing memorization while the linear probe baseline decreases significantly.  
    \textbf{Right:} \dejavu score stays roughly constant with training set size suggesting that memorization may be problematic even for large datasets. %By comparison, the baseline \emph{declines} by half, spuriously suggesting less memorization. 
    %Both trends are not captured according to the linear probe train-test gap---a common method to evaluate generalization of SSL representations.}
    }
    \label{fig:dejavu epochs train set size}
\end{minipage}
\hfill
\begin{minipage}[t]{0.49\textwidth}
\centering
     \begin{subfigure}[b]{0.48\textwidth}
         \centering
         \includegraphics[width=\textwidth]{figures/vicreg_samples_epochs.pdf}
         \vspace{-1.5em}
         \caption{\dejavu vs. epochs}
         \label{fig:per sample v. training epochs}
     \end{subfigure}
     \begin{subfigure}[b]{0.48\textwidth}
         \centering
         \includegraphics[width=\textwidth]{figures/vicreg_samples_datasets.pdf}
         \vspace{-1.5em}
         \caption{\dejavu vs. train set size}
         \label{fig:per sample v. n}
     \end{subfigure}~
     \vspace{-0.5em}
    \caption{
    \definecolor{part_blue}{rgb}{0.2824, 0.4706, .8157}
	\definecolor{part_red}{rgb}{0.8392, 0.3725, 0.3725}
	\definecolor{part_orange}{rgb}{0.9333, 0.5216, 0.2902}
    Partition of samples $A_i \in \calA$ into the four categories: {\color{gray}unassociated} (not shown), {\color{part_orange}memorized}, {\color{part_red}misrepresented} and {\color{part_blue}correlated} for VICReg. The {\color{part_orange}memorized} samples---those whose labels are predicted by $\KNN_A$ but not by $\KNN_B$---occupy a significantly larger share of the training set than the {\color{part_red}misrepresented} samples---those predicted by $\KNN_B$ but not $\KNN_A$ by chance. %At 1000 epochs, $\approx 15\%$ of the training set is {\color{part_orange}memorized}. The trends across training epochs and training set sizes are consistent with those observed in Figure \ref{fig:dejavu epochs train set size}
    }
    \label{fig:partition attack main}
    \end{minipage}
\vspace{-1em} 
\end{figure}

\iffalse

\begin{minipage}[t]{0.49\textwidth}
\centering
     \begin{subfigure}[b]{0.48\textwidth}
         \centering
         \includegraphics[width=0.95\textwidth]{figures/deja_vu_vs_parameters.png}
         \vspace{-0.4em}
         \caption{\dejavu vs. capacity}
         \label{fig:dejavu v. capacity}
     \end{subfigure}
     \hfill
     \begin{subfigure}[b]{0.48\textwidth}
          \tiny
          \centering
          \setlength{\tabcolsep}{3pt}
          \begin{tabular}{|c|c|c|}
            \hline
            Criteria & DV & Acc P/B \\
            \hline
            Supervised & 8.9 & 55.3/61.1\\
            \hline
            Byol\citep{grill2020byol} & 8.0& 54.3/59.4\\
            \hline
            SimCLR\citep{chen2020simclr} & 10.0 & 44.2/54.1\\
            \hline
            Dino\citep{Dino} & 14.5 & 26.3/55.7 \\
            \hline
            Barlow T.\citep{zbontar2021barlow} & 30.5 & 33.7/54.4\\
            \hline
            VICReg\citep{vicreg} & \textbf{33.2} & 40.3/55.2\\
            \hline
          \end{tabular}
          \vspace{1.3em}
          % \caption{\dejavu (DV) vs. SSL Criterion}
          \caption{\dejavu (DV) vs. Criterion}
          \label{tab:dejavu vs. criterion}
    \end{subfigure}
    \vspace{-0.5em}
    \caption{
    Comparison of \dejavu score for different architectures and training criteria. \textbf{Left:} \dejavu score with VICReg for resnet (purple) and vision transformer (green) architectures versus number of model parameters. As expected, memorization grows with larger model capacity. This trend is more pronounced for convolutional (resnet) than transformer (ViT) architectures. \textbf{Right:} Comparison of \dejavu score and ImageNet validation accuracy (P: using projector embeddings, B: using backbone embeddings) for various SSL criteria. \textbf{Nearly all SSL models have more memorization than the supervised baseline.} 
    % Effect of training epochs and train set size on \dejavu score.
    % \textbf{Left:} \dejavu score increases with higher number of training epochs, indicating worsening memorization.
    % \textbf{Right:} \dejavu score stays roughly constant with training set size. Both trends are not captured according to the linear probe train-test gap---a common method to evaluate generalization of SSL representations.
    }
\end{minipage}
\vspace{-2em} 
\end{figure}

\begin{figure}[ht]
\begin{minipage}[t]{0.49\textwidth}
\centering
     \begin{subfigure}[b]{0.49\textwidth}
         \centering
         \includegraphics[width=\textwidth]{figures/epochs_lb_attk_epochs_acc_top1_legend.pdf}
         \caption{\dejavu vs. epochs}
         \label{fig:dejavu v. training epochs}
     \end{subfigure}
     \begin{subfigure}[b]{0.49\textwidth}
         \centering
         \includegraphics[width=\textwidth]{figures/epochs_lb_attk_datasets_acc_top1_legend.pdf}
         \caption{\dejavu vs. train set size}
         \label{fig:dejavu v. n}
     \end{subfigure}~
     \begin{subfigure}[b]{0.32\textwidth}
         \centering
         \includegraphics[width=0.8\textwidth]{figures/dejavu_vs_parameters.pdf}
         \caption{\dejavu vs. capacity}
         \label{fig:dejavu v. n}
     \end{subfigure}
    \caption{
    Effect of training epochs and train set size on \dejavu score.
    \textbf{Left:} \dejavu score increases with higher number of training epochs, indicating worsening memorization.
    \textbf{Right:} \dejavu score stays roughly constant with training set size. Both trends are not captured according to the linear probe train-test gap---a common method to evaluate generalization of SSL representations.}
    \end{minipage}
\vspace{-1em} 
\end{figure}

\begin{table}[ht]
  \footnotesize
  \centering
  \begin{tabular}{|c|c|}
    \hline
    Supervised & 8.9\\
    \hline
    SimCLR\citep{chen2020simclr} & 10.0\\
    \hline
    Byol\citep{grill2020byol} & 8.0\\
    \hline
    Dino\citep{Dino} & 14.5\\
    \hline
    Barlow T.\citep{zbontar2021barlow} & 30.5\\
    \hline
    VICReg\citep{vicreg} & \textbf{33.2}\\
    \hline
  \end{tabular}
  \caption{DejaVu Score 20\% Conf for various SSL methods.}
  \label{tab:two-row-table}
\end{table}
\vspace{-1em} 
\fi

\iffalse
\begin{figure}[ht]
\begin{minipage}[t]{.49\textwidth}
\centering
     \begin{subfigure}[b]{0.49\textwidth}
         \centering
         \includegraphics[width=\textwidth]{figures/epochs_lb_attk_epochs_acc_top1_legend.pdf}
         \caption{\dejavu vs. epochs}
         \label{fig:dejavu v. training epochs}
     \end{subfigure}
     \hfill
     \begin{subfigure}[b]{0.49\textwidth}
         \centering
         \includegraphics[width=\textwidth]{figures/epochs_lb_attk_datasets_acc_top1_legend.pdf}
         \caption{\dejavu vs. train set size}
         \label{fig:dejavu v. n}
     \end{subfigure}
\caption{
Effect of training epochs and train set size on \dejavu score.
\textbf{Left:} \dejavu score increases with higher number of training epochs, indicating worsening memorization.
\textbf{Right:} \dejavu score stays roughly constant with training set size. Both trends are not captured according to the linear probe train-test gap---a common method to evaluate generalization of SSL representations.}
\label{fig:dejavu epochs and dataset}
\end{minipage}
\hfill
\begin{minipage}[t]{.49\textwidth}
     \centering
     \begin{subfigure}[b]{0.49\textwidth}
         \centering
         \includegraphics[width=\textwidth]{figures/criteria_epochs.pdf}
         \caption{criteria comparison}
         \label{fig:dejavu v. criteria}
     \end{subfigure}
     \hfill
     \begin{subfigure}[b]{0.49\textwidth}
         \centering
         \includegraphics[width=\textwidth]{figures/architecture_epochs.pdf}
         \caption{architecture comparison}
         \label{fig:dejavu v. arch}
     \end{subfigure}
\caption{
Effect of SSL training criteria and model architectures on \dejavu score.
%the accuracy gap between target model (trained on $\calA$) and reference model (trained on $\calB$) making predictions on their 20\% most confident examples.
\textbf{Left:} \dejavu score for various training criteria.
%Barlow and VICReg have the heaviest degree of memorization, while SimCLR and BYOL have the least. 
%Note that we show detailed reconstructions of SimCLR's training data in Section \ref{sec:visualizing} despite its relatively low degree of \dejavu. 
%Regardless, Although SimCLR and BYOL have the least, we  visualize detailed reconstructions with SimCLR in section \ref{sec:mem v corr} 
All SSL models have significantly more \dejavu than the supervised baseline. \textbf{Right:} \dejavu score versus epochs for various training architectures. As expected, lower capacity architectures (Resnet18, Resnet34) reduce \dejavu but not completely. 
}
\label{fig:dejavu criteria and architecture}
\end{minipage}
\vspace{-1em} 
\end{figure}
\fi
% %\begin{figure}[ht]
%%%
%VICREG
%%%
     \centering
     \begin{subfigure}[b]{0.49\textwidth}
         \centering
         \includegraphics[width=\textwidth]{figures/sample_level_training_epochs.pdf}
         \caption{Categories of training samples vs. number of epochs}
         \label{fig:sample level epochs}
     \end{subfigure}
     \hfill
     \begin{subfigure}[b]{0.49\textwidth}
         \centering
         \includegraphics[width=\textwidth]{figures/sample_level_training_set_size.pdf}
         \caption{Categories of training samples vs. training set size}
         \label{fig:sample level training size}
     \end{subfigure}
\caption{
\definecolor{part_blue}{rgb}{0.2824, 0.4706, .8157}
\definecolor{part_red}{rgb}{0.8392, 0.3725, 0.3725}
\definecolor{part_orange}{rgb}{0.9333, 0.5216, 0.2902}
Partition of samples $A_i \in \calA$ into the four categories: {\color{gray}unassociated} (not shown), {\color{part_orange}memorized}, {\color{part_red}misrepresented} and {\color{part_blue}correlated}. The {\color{part_orange}memorized} samples---ones whose labels are predicted by $\KNN_A$ but not by $\KNN_B$---occupy a significantly larger share for VICReg compared to the supervised model, indicating that sample-level \dejavu memorization is more prevalent in VICReg. %The trends across number of training epochs and training set sizes are consistent with those observed in Figures \ref{fig:dejavu epochs and dataset} and \ref{fig:dejavu criteria and architecture}.
}
\label{fig:partition attack main appendix}
\end{figure}
% \paragraph{Effect of training set size.} 
% Figure \ref{fig:dejavu v. n} shows how \dejavu memorization responds to the model's training set size. The number of training epochs is fixed to 1000. Interestingly, training set size appears to have almost \emph{no} influence on the \dejavu score (red line), indicating that memorization is equally prevalent with a 100K dataset and a 500K dataset (which suggests that \dejavu memorization may be detectable for larger datasets). Meanwhile, the linear probe train-test accuracy gap \emph{declines} by half as the dataset size grows, failing to represent the memorization quantified by our test. 
% The trend is completely different according to linear probe accuracy (dark blue line), the train-test gap shrinks substantially when increasing the training set size from 100K to 500K. This highlights that the train-test gap is not able to capture \dejavu memorization. %Our evidence suggests that \dejavu memorization may be detectable even for large-scale training datasets. 
%\vspace{-.75em}

\vspace{-.75em} 
\subsection{Sample-level Memorization}
\label{sec:dissection}

% Section \ref{sec:label inference accuracy} shows the \emph{average} level of \dejavu memorization on a subset of the training set $\calA$. However, this average tell us only what the attacker success rate might be without explicitly describing how much of the datatset is \dejavu memorized.
The \dejavu score shows, \emph{on average}, how much better an adversary can select and classify images when using the target model trained on them. 
This average score does not tell us how many individual images have their label successfully recovered by $\KNN_A$ but not by $\KNN_B$. In other words, how many images are exposed by virtue of \emph{being in training set} $\calA$: a risk notion foundational to differential privacy. 
% However, from the perspective of an individual image $A_i \in \calA$, it is informative to know whether it was correctly classified 
To better quantify what fraction of the dataset is at risk, we perform a sample-level analysis by fixing a sample $A_i \in \calA$ and observing the label inference result of $\KNN_A$ vs. $\KNN_B$.
To this end, we partition samples $A_i \in \calA$ based on the result of label inference into four distinct categories: {\color{gray}\textbf{Unassociated}} - label inferred with neither KNN; {\color{part_orange}\textbf{Memorized}} - label inferred only with $\KNN_A$; {\color{part_red}\textbf{Misrepresented}} - label inferred only with $\KNN_B$; {\color{part_blue}\textbf{Correlated}} - label inferred with both KNNs. 
% \begin{multicols}{2}
% \begin{itemize}
%     \vspace{-.75em}
%     \setlength\itemsep{0.15em}
%     \item {\color{gray}Unassociated}: label inferred with neither KNN   
%     \item {\color{part_orange}Memorized}: label only inferred by $\KNN_A$
%     \item {\color{part_red}Misrepresented}: label only inferred with $\KNN_B$
%     \item {\color{part_blue}Correlated}: label inferred with both KNNs
%     \vspace{-.75em}
% \end{itemize}
% \end{multicols}
Intuitively, {\color{gray}unassociated} samples are ones where the embedding of $\crop{A_i}$ does not encode information about the label. {\color{part_blue}Correlated} samples are ones where the label can be inferred from $\crop{A_i}$ using correlation, \emph{e.g.}, inferring the foreground object is basketball given a crop showing a basketball player. Ideally, the {\color{part_red}misrepresented} set should be empty but contains a small portion of examples due to chance.
\emph{Déjà vu} memorization occurs for {\color{part_orange}memorized} samples where the embedding of $\SSL_B$ does not encode the label but the embedding of $\SSL_A$ does. To measure the pervasiveness of \dejavu memorization, we compare the size of the {\color{part_orange}memorized} and {\color{part_red}misrepresented} sets.
Figure \ref{fig:partition attack main} shows how the four categories of examples change with number of training epochs and training set size. The {\color{gray}unassociated} set is not shown since the total share adds up to one. The {\color{part_red}misrepresented} set remains under $5\%$ and roughly unchanged across all settings, consistent with our explanation that it is due to chance. In comparison, VICReg's {\color{part_orange}memorized} set surpasses $15\%$ at 1000 epochs. Considering that up to 5\% of these memorized examples could also be due to chance, we conclude that \textbf{at least 10\% of VICReg's training set is \dejavu memorized.} 
%is many times larger than its {\color{part_red}misrepresented} set, indicating substantial sample-level \dejavu memorization. 
%In fact, \textbf{it is 15\% of the training set that is \dejavu memorized with VICReg.}
%The trends across different number of training epochs and training set sizes match those observed in Section \ref{sec:label inference accuracy}. % On the other hand, the supervised model's {\color{part_orange}memorized} set is only marginally larger than its {\color{part_red}misrepresented} set.

% The trends across different number of training epochs and training set sizes match those observed in Section \ref{sec:label inference accuracy}: Increasing the number of epochs increases \dejavu memorization (Figure \ref{fig:per sample v. training epochs}), while increasing the training set size does not appear to reduce \dejavu memorization (Figure \ref{fig:per sample v. n}). 
%\section{Visualizing \emph{Déjà Vu} Memorization}
\label{sec:visualizing}
Beyond enabling label inference using a periphery crop, we show that \dejavu memorization allows the SSL model to encode other forms of information about a training image. Namely, we train an RCDM \citep{RCDM} on the public dataset $\calX$ and use it to visually reconstruct training images given their periphery crop.
We aim to answer the following two questions: \textbf{(1)} Can we visualize the distinction between correlation and \dejavu memorization? \textbf{(2)} What foreground object details can be extracted from the SSL model beyond class label? 
% \begin{enumerate}[noitemsep, leftmargin=*, topsep=0pt]
%     \item Can we visualize the distinction between correlation and \dejavu memorization? 
%     \item What foreground object details can be extracted from the SSL model beyond class label? 
% \end{enumerate}
\vspace{-0.5em}
\paragraph{Reconstruction pipeline.}
RCDM is a conditional generative model that is trained on the \emph{backbone embedding} of images $X_i \in \calX$ to generate an image that resembles $X_i$. All training images are first face-blurred for privacy purposes. \citet{RCDM} showed that the backbone embedding of SSL models contains more low-level information about the image, making them better suited for conditioning the RCDM.
At test time, following the pipeline in Figure \ref{fig:split_and_pipeline_cartoon}, we first use the projector embedding to find the KNN subset for the periphery crop, $\crop{A_i}$, and then average their backbone embeddings as input to the RCDM model. Ideally, when the public set contains enough representative images, the average representation of the KNN subset encodes objects present in $A_i$, and the RCDM model decodes this representation to visualize these objects.
% \begin{figure}[ht]
%%%
%VICREG
%%%
     \centering
     \begin{subfigure}[b]{0.49\textwidth}
         \centering
         \includegraphics[width=\textwidth]{figures/sample_level_training_epochs.pdf}
         \caption{Categories of training samples vs. number of epochs}
         \label{fig:sample level epochs}
     \end{subfigure}
     \hfill
     \begin{subfigure}[b]{0.49\textwidth}
         \centering
         \includegraphics[width=\textwidth]{figures/sample_level_training_set_size.pdf}
         \caption{Categories of training samples vs. training set size}
         \label{fig:sample level training size}
     \end{subfigure}
\caption{
\definecolor{part_blue}{rgb}{0.2824, 0.4706, .8157}
\definecolor{part_red}{rgb}{0.8392, 0.3725, 0.3725}
\definecolor{part_orange}{rgb}{0.9333, 0.5216, 0.2902}
Partition of samples $A_i \in \calA$ into the four categories: {\color{gray}unassociated} (not shown), {\color{part_orange}memorized}, {\color{part_red}misrepresented} and {\color{part_blue}correlated}. The {\color{part_orange}memorized} samples---ones whose labels are predicted by $\KNN_A$ but not by $\KNN_B$---occupy a significantly larger share for VICReg compared to the supervised model, indicating that sample-level \dejavu memorization is more prevalent in VICReg. %The trends across number of training epochs and training set sizes are consistent with those observed in Figures \ref{fig:dejavu epochs and dataset} and \ref{fig:dejavu criteria and architecture}.
}
\label{fig:partition attack main appendix}
\end{figure}
%\begin{figure*}[t!]
%%%
%DAM
%%%
     \centering
     \begin{subfigure}[b]{0.49\textwidth}
         \centering
         \includegraphics[width=\textwidth]{figures/dam_corr.png}
         \caption{A {\color{part_blue}correlated} dam example}
         \label{fig:dam correlated}
     \end{subfigure}
     \hfill
     \begin{subfigure}[b]{0.49\textwidth}
         \centering
         \includegraphics[width=\textwidth]{figures/dam_mem.png}
         \caption{A {\color{part_orange}memorized} dam example}
         \label{fig:dam memorized}
     \end{subfigure}
\caption{
{\color{part_blue}Correlated} and {\color{part_orange}Memorized} examples from the \emph{dam} class. Both $\SSL_A$ and $\SSL_B$ are SimCLR models.
\textbf{Left:} The periphery crop (pink square) contains a concrete structure that is often present in images of dams. Consequently, the trained RCDM can reconstruct the foreground object using representations from both $\SSL_A$ and $\SSL_B$ through this correlation.
\textbf{Right:} The periphery crop only contains a patch of water. The embedding produced by $\SSL_B$ only contains enough information to infer that the foreground object is related to water, as reflected by its KNN set and RCDM reconstruction. In contrast, the embedding produced by $\SSL_A$ memorizes the association of this patch of water with dam and the RCDM can visualize the embedding to produce images of dams.
}
\vspace{-1ex}
\label{fig:mem v corr dam}
\end{figure*}


\begin{figure*}[t!]
%%%
%DAM
%%%
     \centering
     \begin{subfigure}[b]{0.49\textwidth}
         \centering
         \includegraphics[width=\textwidth]{figures/dam_corr.png}
         \caption{A {\color{part_blue}correlated} dam example}
         \label{fig:dam correlated}
     \end{subfigure}
     \hfill
     \begin{subfigure}[b]{0.49\textwidth}
         \centering
         \includegraphics[width=\textwidth]{figures/dam_mem.png}
         \caption{A {\color{part_orange}memorized} dam example}
         \label{fig:dam memorized}
     \end{subfigure}
\caption[Correlated and Memorized examples from the \emph{dam} class.]{
Correlated and Memorized examples from the \emph{dam} class. Both $\SSL_A$ and $\SSL_B$ are SimCLR models.
\textbf{Left:} The periphery crop (pink square) contains a concrete structure that is often present in images of dams. Consequently, the trained RCDM can reconstruct the foreground object using representations from both $\SSL_A$ and $\SSL_B$ through this correlation.
\textbf{Right:} The periphery crop only contains a patch of water. The embedding produced by $\SSL_B$ only contains enough information to infer that the foreground object is related to water, as reflected by its KNN set and RCDM reconstruction. In contrast, the embedding produced by $\SSL_A$ memorizes the association of this patch of water with dam and the RCDM can visualize the embedding to produce images of dams.
}
\label{fig:mem v corr dam}
\end{figure*}


\begin{figure}[t!]
%%%
%BADGER
%%%
     \centering
     \begin{subfigure}[b]{0.49\textwidth}
         \centering
         \includegraphics[width=\textwidth]{figures/euro_badgers.png}
         \caption{{\color{part_orange}Memorized} European badgers}
         \label{fig:euro badgers}
     \end{subfigure}
     \hfill
     \begin{subfigure}[b]{0.49\textwidth}
         \centering
         \includegraphics[width=\textwidth]{figures/amer_badgers.png}
         \caption{{\color{part_orange}Memorized} American badgers}
         \label{fig:amer badgers}
     \end{subfigure}
\caption[Visualization of \dejavu memorization beyond class label.]{
Visualization of \dejavu memorization beyond class label. Both $\SSL_A$ and $\SSL_B$ are VICReg models. 
The four images shown belong to the memorized set of $\SSL_A$ from the \emph{badger} class. RCDM reconstruction using embeddings from $\SSL_A$ can reveal not only the correct class label, but also the specific badger species: \emph{European} (left) and \emph{American} (right). Such information does not appear to be memorized by the reference model $\SSL_B$.
} 
\label{fig:in class badger}
\end{figure}


% \subsection{Visualizing Correlation vs. Memorization}
\label{sec:mem v corr}
\vspace{-0.5em} 
\paragraph{Visualizing Correlation vs. Memorization.}
Figure \ref{fig:mem v corr dam} shows examples of dams from the {\color{part_blue}correlated} set (left) and the {\color{part_orange}memorized} set (right) as defined in Section \ref{sec:dissection}, along with the associated KNN set and RCDM reconstruction. Both $\SSL_A$ and $\SSL_B$ are SimCLR models. In Figure \ref{fig:dam correlated}, the periphery crop is represented by the pink square, which contains concrete structure attached to the dam's main structure. As a result, both $\SSL_A$ and $\SSL_B$ produce embeddings of $\crop{A_i}$ whose KNN set in $\calX$ consist of dams, \emph{i.e.}, there is a correlation between the concrete structure in $\crop{A_i}$ and the foreground dam. The RCDM reconstructions also consist of dams or structures that closely resemble dams. 
In Figure \ref{fig:dam memorized}, the periphery crop only contains a patch of water, which does not strongly correlate with dams in the ImageNet distribution. Evidently, the reference model $\SSL_B$ embeds $\crop{A_i}$ close to that of other objects commonly found in water, such as sea turtle and submarine. In contrast, the KNN set according to $\SSL_A$ all contain dams despite the vast number of alternative possibilities within the ImageNet classes, and the RCDM reconstruction outputs dams as well which highlight memorization in $\SSL_A$ between this specific patch of water and the dam. %\footnote{See Appendix \ref{sec:appx visualization} to see the same trend in the \emph{yellow garden spider} class.}


% \subsection{Visualizing Memorization Beyond Class Label}
% \label{sec:in class variation}
\vspace{-0.5em} 
\paragraph{Visualizing Memorization Beyond Class Label.}
We now use our reconstruction algorithm to show that \dejavu memorization can be exploited to reveal detailed information beyond class label. Figure \ref{fig:in class badger} shows four examples of badgers from the {\color{part_orange}memorized} set. In all four images, the periphery crop (pink square) does not contain any indication that the foreground object is a badger. Despite this, the KNN set and the RCDM reconstruction using $\SSL_A$ consistently produce images of badgers, while the same does not hold for $\SSL_B$.
More interestingly, reconstructions using $\SSL_A$ in Figure \ref{fig:euro badgers} all contain \emph{European} badgers, while reconstructions in Figure \ref{fig:amer badgers} all contain \emph{American} badgers, accurately reflecting the species of badger present in the respective training images. Since ImageNet-1K does \emph{not} differentiate between these two species of badgers, our reconstructions show that SSL models can memorize information that is highly specific to a training sample beyond its class label\footnote{See Appendix \ref{sec:appx visualization} for additional visualization experiments.}.%\footnote{See Appendix \ref{sec:appx visualization} for the same trend in the \emph{aircraft carrier} class.}.





%\vspace{-.5em} 
\section{Mitigation of \dejavu memorization}
\label{sec:mitigation}
% We do not have an understanding on why \dejavu occur so strongly in some SSL pretraining, however we present additional experiments that shed light on which parameters have the biggest impact on \dejavu memorization.
\begin{figure}[ht]
\label{fig:mitigations}
\begin{minipage}[t]{0.5\textwidth}
\centering
     \begin{subfigure}[b]{0.47\textwidth}
         \centering
         \includegraphics[width=\textwidth]{figures/dejavu_vicreg_param.png}
         \vspace{-1.5em}
         \caption{Loss hyper-parameter}
         \label{fig:dejavu v. invariance}
     \end{subfigure}
     \begin{subfigure}[b]{0.49\textwidth}
         \centering
         \includegraphics[width=\textwidth]{figures/deja_vu_vs_layer.png}
         \vspace{-1.5em}
         \caption{Guillotine regularization}
         \label{fig:dejavu v. guillotine}
     \end{subfigure}~
     \vspace{-0.5em}
    \caption[Effect of two kinds of hyper-parameters on VICReg memorization. ]{
    Effect of two kinds of hyper-parameters on VICReg memorization. \textbf{Left:} \dejavu score (red) versus the \emph{invariance} loss parameter, $\lambda$, used in the VICReg criterion (100k dataset). Larger $\lambda$ significantly reduces \dejavu, with minimal effect on linear probe validation performance (green). $\lambda = 25$ (near maximum \dejavu) is recommended in the original paper \textbf{Right:} \dejavu score versus projector layer---guillotine regularization \cite{Guillotine}---from projector to backbone. Removing the projector can significantly reduce \dejavu. Appendix \ref{sec:guillotine} shows that the backbone still can memorize, however; we demonstrate reconstructions using the SimCLR backbone.
    }
\end{minipage}
\hfill
\begin{minipage}[t]{0.48\textwidth}
\centering
     \begin{subfigure}[b]{0.46\textwidth}
         \centering
         \includegraphics[width=\textwidth]{figures/deja_vu_vs_parameters.png}
         \vspace{-1.3em}
         \caption{\dejavu vs. capacity}
         \label{fig:dejavu v. capacity}
     \end{subfigure}
     \hfill
     \begin{subfigure}[b]{0.52\textwidth}
          \tiny
          \centering
          \setlength{\tabcolsep}{3pt}
          \begin{tabular}{|c|c|c|}
            \hline
            Criteria & DV & Acc P/B \\
            \hline
            Supervised & 8.9 & 55.3/61.1\\
            \hline
            Byol\citep{grill2020byol} & 8.0& 54.3/59.4\\
            \hline
            SimCLR\citep{chen2020simclr} & 10.0 & 44.2/54.1\\
            \hline
            Dino\citep{Dino} & 14.5 & 26.3/55.7 \\
            \hline
            Barlow T.\citep{zbontar2021barlow} & 30.5 & 33.7/54.4\\
            \hline
            VICReg\citep{vicreg} & \textbf{33.2} & 40.3/55.2\\
            \hline
          \end{tabular}
          \vspace{1.3em}
          % \caption{\dejavu (DV) vs. SSL Criterion}
          \caption{\dejavu (DV) vs. Criterion}
          \label{tab:dejavu vs. criterion}
    \end{subfigure}
    \vspace{-1.4em}
    \caption[Effect of model architecture and criterion on \dejavu memorization.]{
    %Comparison of \dejavu score for different architectures and training criteria. 
    Effect of model architecture and criterion on \dejavu memorization. 
    \textbf{Left:} \dejavu score with VICReg for resnet (purple) and vision transformer (green) architectures versus number of model parameters. As expected, memorization grows with larger model capacity. This trend is more pronounced for convolutional (resnet) than transformer (ViT) architectures. \textbf{Right:} Comparison of \dejavu score 20\% conf. and ImageNet linear probe validation accuracy (P: using projector embeddings, B: using backbone embeddings) for various SSL criteria. %\textbf{Nearly all SSL models have more memorization than the supervised baseline.} 
    % Effect of training epochs and train set size on \dejavu score.
    % \textbf{Left:} \dejavu score increases with higher number of training epochs, indicating worsening memorization.
    % \textbf{Right:} \dejavu score stays roughly constant with training set size. Both trends are not captured according to the linear probe train-test gap---a common method to evaluate generalization of SSL representations.
    }
    \end{minipage}
\end{figure}
We cannot yet make claims on why \dejavu occurs so strongly for some SSL training settings and not for others. To gain some intuition for future work, we present additional observations that shed light on which parameters have the most salient impact on \dejavu memorization.
\vspace{-.75em}
\paragraph{Déjà vu memorization worsens by increasing number of training epochs.} 
Figure \ref{fig:dejavu v. training epochs} shows how \dejavu memorization changes with number of training epochs for VICReg. The training set size is fixed to 300K samples. From 250 to 1000 epochs, the \dejavu score (red curve) grows \emph{threefold}: from under 10\% to over 30\%. Remarkably, this trend in memorization is \emph{not} reflected by the linear probe gap (dark blue), which only changes by a few percent beyond 250 epochs. 

%\vspace{-.75em}
\paragraph{Training set size has minimal effect on \dejavu memorization.} Figure \ref{fig:dejavu v. n} shows how \dejavu memorization responds to the model's training set size. The number of training epochs is fixed to 1000. Interestingly, training set size appears to have almost \emph{no} influence on the \dejavu score (red line), indicating that memorization is equally prevalent with a 100K dataset and a 500K dataset. This result suggests that \dejavu memorization may be detectable even for large datasets. Meanwhile, the standard linear probe train-test accuracy gap \emph{declines} by more than half as the dataset size grows, failing to represent the memorization quantified by our test. 
% The trend is completely different according to linear probe accuracy (dark blue line), the train-test gap shrinks substantially when increasing the training set size from 100K to 500K. This highlights that the train-test gap is not able to capture \dejavu memorization. Our evidence suggests that \dejavu memorization may be detectable even for large-scale training datasets. 
\vspace{-0.5em}
\paragraph{Training loss hyper-parameter has a strong effect.} 
%We show in Figure \ref{fig:dejavu v. training epochs} that the number of training epochs is an important factor that can increase significantly \dejavu memorization. In contrast, the dataset size does not impact much \dejavu as shown in Figure \ref{fig:dejavu epochs train set size}. 
Loss hyper-parameters, like VICReg's invariance coefficient (Figure \ref{fig:dejavu v. invariance}) or SimCLR's temperature parameter (Appendix Figure \ref{fig:simclr temperature}) significantly impact \dejavu with minimal impact on the linear probe validation accuracy.

\vspace{-0.5em}
\paragraph{Some SSL criteria promote stronger \dejavu memorization.} Table \ref{tab:dejavu vs. criterion} demonstrates that the degree of memorization varies widely for different training criteria. VICReg and Barlow Twins have the highest \dejavu scores while SimCLR and Byol have the lowest.
%\footnote{We show detailed reconstructions of SimCLR's training data in Section \ref{sec:visualizing} despite its relatively low degree of \dejavu.}.
With the exception of Byol, all SSL models have more \dejavu memorization than the supervised model. Interestingly, different criteria can lead to similar linear probe validation accuracy and very different degrees of \dejavu as seen with SimCLR and Barlow Twins. Note that low degrees of \dejavu can still risk training image reconstruction, as exemplified by the SimCLR reconstructions in Figures \ref{fig:mem v corr dam} and \ref{fig:mem v corr spider}. 
%\vspace{-1em}
\vspace{-0.5em}
\paragraph{Larger models have increased \dejavu memorization.} Figure \ref{fig:dejavu v. capacity} validates the common intuition that lower capacity architectures (Resnet18/34) result in less memorization than their high capacity counterparts (Resnet50/101). 
% \begin{wrapfigure}{r}{0.25\textwidth} 
%     \centering
%     \includegraphics[width=0.25\textwidth]{figures/attk_layer_acc_top1_legend.pdf}
%     \caption{\dejavu memorization versus layer from backbone (0) to projector output (3).}
%     \label{fig:dejavu vs layer}
%     \vspace{-8ex}
% \end{wrapfigure}
We see the same trend for vision transformers as well. %This comes with a tradeoff, since reduced model capacity can result in a nontrivial degradation of representation quality\cite{vicreg, simclr}.  
\vspace{-0.5em}
\paragraph{Guillotine regularization can help reduce \dejavu memorization.} Previous experiments were done using the projector embedding. In Figure \ref{fig:dejavu v. guillotine}, we present how Guillotine regularization\citep{Guillotine} (removing final layers in a trained SSL model) impacts \dejavu with VICReg\footnote{Further experiments are available in Appendix \ref{sec:guillotine}.}. Using the backbone embedding instead of the projector embedding seems to be the most straightforward way to mitigate \dejavu memorization. However, as demonstrated in Appendix \ref{sec:appx backbone results}, backbone representation with low \dejavu score can still be leveraged to reconstruct some of the training images.

\section{Conclusion}
\label{sec:conclusion}

We defined and analyzed \dejavu memorization, a notion of unintended memorization of partial information in image data. As shown in Sections \ref{sec:quant} and \ref{sec:visualizing}, SSL models can largely exhibit \dejavu memorization on their training data, and this memorization signal can be extracted to infer or visualize image-specific information.
Since SSL models are becoming increasingly widespread as foundation models for image data, negative consequences of \dejavu memorization can have profound downstream impact and thus deserves further attention. 
Future work should focus on understanding how \dejavu emerges in the training of SSL models and why methods like Byol are much more robust to \dejavu than VICReg and Barlow Twins. In addition, trying to characterize which data points are the most at risk of \dejavu could be crucial to get a better understanding on this phenomenon. 
